Property ownership is distinct from physical \term{possession}. Someone other
than the owner of land may be standing on it, occupying space and preventing
the owner from using the land; someone other than the owner of personal property
may be holding it, preventing the owner from accessing and using it. This other
person may possess the property with permission from the owner, against the
owner's will, or without the owner's knowledge.

Physical possession may seem irrelevant for property law---after all, isn't the
whole point of the rule of law that legal rights, not physical might, are
determinative? And yet possession alone can, in some situations, give rise to
legal rights over things, rights that can properly be deemed ``property
rights.'' With respect to land, \having{intro-adverse-possession}{as we have
seen, }{as we will see later, }{}physical possession in the right conditions can
turn into actual ownership by the doctrine of adverse possession.
\unskip\footnote{Adverse possession of personal property is also possible,
though somewhat more complicated\having{okeeffe-v-snyder}{, as we saw in
\emph{O'Keeffe v. Snyder}.}{, as we will see in \emph{O'Keeffe v. Snyder}.}{.
\defcase{okeeffe-v-snyder}{O'Keeffe v. Snyder, 416 A.2d 862 (N.J.
1980)}\sentence{see okeeffe-v-snyder}.}}
And the story
for personal property is even more interesting, because of the number of ways in
which movable items can come into someone else's possession. They can be lost,
found, borrowed, stored, stolen, mixed up with other things, and more.

This chapter will consider three ways in which possession can give rise to
property rights in personalty: finding lost items, improvements, and bailment
arrangements such as lending. In each of these situations, identify the
circumstances that give the physical possessor rights, what rights the possessor
has, and against whom those rights apply. What legal relationship does the
possessor have with respect to the true owner, and what rights does the
possessor have against third parties?

