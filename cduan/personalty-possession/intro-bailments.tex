A \term{bailment} is an arrangement where the owner of personal property
entrusts the property to another. The owner is called the \term{bailor}, while
the recipient is called the \term{bailee}. Common bailees include delivery
services, dry cleaners, and friends who borrow others' casebooks. These
arrangements split full ownership from physical possession, and raise several
issues regarding the parties' respective rights.

\defcase{rhodes-v-pioneer}{
Rhodes v. Pioneer Parking Lot, Inc., 501 S.W.2d 569 (Tenn. 1973)
}

A conventional statement of the rule for formation of a bailment is as follows:
\begin{quote}
A bailment is the delivery of personalty to another for a particular purpose or
on mere deposit, on a contract express or implied, that after the purpose has
been fulfilled, it shall be redelivered to the person who delivered it, or
otherwise dealt with according to his direction or kept until he reclaims it.
\end{quote}
\sentence{rhodes-v-pioneer at 570}. In the following case, which of the elements
of the above definition are at issue? What are the duties between the bailor and
bailee? And what rights and duties do each of them have with respect to third
parties?

\expectnext{allen-v-hyatt}
