\chapter{Theory of Property}

\import intro-theory


\section{What Is Property?}

\import intro-what-is-property

\import ownership/hohfeld-fundamental

\begin{questions}
\import hohfeld-addl-qs
\end{questions}

\import ownership/merrill-smith-economics

\begin{questions}

\editrepofile{base}{ownership}{hohfeld-merrill-smith-qs}
    \replacestart{``A right against whom?''}{%
        \item This time, it's your turn to explain the reading. It should be
        easier, because you know what Merrill and Smith were responding to.
        Here's the key question: what is the defining right of property,
        according to Merrill and Smith? (\emph{Hint}: it's in the paragraph that
        says ``this feature is key.'') Once you've found that, then follow the
        argument for why that feature is key, and
        why Merrill and Smith think Hohfeld was wrong in rejecting property as
        having any, let alone this particular, defining right..
    }

    \replaceend{\item Hohfeld observes that, when}{}

\endedit

\end{questions}



\section{Why Have Property Law?}

\import intro-property-rationales


\section{Subject Matter}

\import subject-matter/intro-subject-matter.tex

\import intangible-property/us-v-turoff

\begin{questions}
\import intangible-property/us-v-turoff-qs
\end{questions}

