\expected{intro-property-rationales}

To begin our journey into property law, the most basic starting point is, what
is \term{property}? This might seem like a strange question to begin with---most
toddlers have a sense for what property is, as they go around calling everything
``mine''---but it's worth beginning at the beginning.

Consider a distant planet with no humans or living beings on it. The planet has
rocks, soil, water, perhaps even vegetation. On Earth, all of these things could
be someone's property, say a farmer's. But on this distant planet, are they
property? In other words, can there be property with no people?

If you at least hesitated in answering that question, you've reached a key
insight about property law. \emph{Property is not things, it requires people.}
In fact, it requires more than one person. The sole survivor of a shipwreck on a
deserted island has no need for property law; property ownership matters only if
there's someone to have a dispute with. So the concept of ``property'' involves
not just things, but also multiple people. Property is about the legal
relationships among them.

The following readings are from the classic literature on the nature of
property, and explore the nature of this relationship among people and things
that we call ``property.'' Beware: they are difficult reading. A walk-through is
given in the notes that follow, but try your best to work through them and get
as much out of them as you can. A big part of the skill of being a lawyer is
reading difficult texts, and the only way to develop that skill is to do it over
and over again.

