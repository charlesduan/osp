\expected{hohfeld-fundamental}

We have considered that, roughly speaking, property involves rights that a
person has against third parties, perhaps with respect to some thing. (For
purposes of this book, we'll take a middle ground between Hohfeld and
Merrill/Smith, with property being a person's rights against others with respect
to a thing, which may be tangible or intangible.)

Why are rights over things a good idea? What values do property rights serve?
Who reaps the benefits of property rights, who is disadvantaged, and is society
overall better off having property law? Scholars and philosophers have for
centuries debated the justifications for having property.

This section provides a brief overview of some of those justifications. It is
worth taking some time to understand them, because these theories will help to
shape how you think about issues in property law, and they will give you a
vocabulary for arguing for or against rules and outcomes. Consider which of
these theories resonates with you, which you would take issue with, and which
strike you as new and unexpected.



\defjrnart{epstein-possession-and-title}{
Richard A. Epstein, Possession as the Root of Title, 13 Georgia Law Review
1221 (1979)
}
\defjrnart{rose-possession-origin-property}{
Carol M. Rose, Possession as the Origin of Property, 52 Chicago Law Review 73
(1985)
}


\paragraph{First possession} Property law often favors first comers: The first
person to come upon an open plot of land, a wild fox, or a seashell washed
ashore can take possession, and the law will protect that first possession as a
property right. Note that ``possession'' in property law means physical
occupation of a resource (standing on the land, picking up the shell), which is
different from the legal concept of property ownership.

On the one hand, first possession may seem like just a description of how
property law works, rather than a justification for why there are property
rights in the first place. Indeed, for one scholar, the best reason why the law
should protect first possession is that the law has always protected first
possession. \sentence{see epstein-possession-and-title at 1241}. There may be
better reasons, though. For one thing, giving property ownership to the first
possessor is a simple, easily administered rule that gives a single owner
decisionmaking power over a resource, rather than relying on potentially complex
and changing rules and customs. \sentence{see epstein-possession-and-title at
1235}. Also, protecting first possession encourages first possessors to
communicate their rights early, letting others know who owns what. \sentence{see
rose-possession-origin-property at 82}.


\paragraph{Natural rights} Perhaps the most famous defense of property rights
comes from the 17th century philosopher John Locke's \emph{Two Treatises of
Government}. According to Locke, one has a right to one's body and one's labor;
as a result, when one mixes labor with an unowned resource, then that resource
becomes the property of the person. One who picks an apple from a tree, for
example, has a natural right of ownership over the apple by virtue of having put
in the labor to pick it.

\defbook{nozick-anarchy-state-utopia}{
Robert Nozick, Anarchy, State and Utopia (1974)
}

There are plenty of difficulties with the Lockean natural rights justification.
What
is the scope of the mixing between labor and resources, for example? If one
pours a can of tomato juice into the ocean, does that person now own the ocean?
\sentence{see nozick-anarchy-state-utopia at 175}. And how does Locke's land of
plentiful, unowned
apple trees, which anyone can harvest, comport with the world of today?
Nevertheless, there is a strong intuitive appeal to the idea that a person's
labor should be protected from others freeloading off of it, and property law
seems like a way of satisfying that intuition.

\paragraph{Incentives to improve} A partygoer who rents a hotel room might very
well trash the place---it's someone else's mess to clean up, after all. A few
years later, that same person buys a home. As a homeowner, that person trims the
hedges, mows the lawn, renovates the kitchen, and repaints the walls. This is a
key virtue of property ownership: It gives the owner a stake in the value of the
resource, and thus gives the owner reason to maintain and improve it.

This justification of property is an economic one. By giving individuals
property
ownership rights, those individuals have incentives to make their property worth
as much as possible. This can be simply a matter of increasing the property's
resale value in the case of home ownership, but it can also be commercial
production. Land can be used to build factories, office buildings, and research
laboratories that create value for society. Without stable expectations of
ownership in that land, it might be argued, no one would invest in building
those factories, offices, or laboratories.

How far can you push this ``stable expectations of ownership'' argument? Could
SpaceX argue that without a stable expectation of ownership of space travel, it
would not invest in developing new spacecraft technology, and therefore it
should have an exclusive property right to space?

\paragraph{Efficient allocation} Closely related to the incentive justification
is
another economic justification for property rights: They ensure that resources
make their way to whoever can make the best use of them. This justification is
sometimes called the ``Coase Theorem'' after the law and economics scholar
Ronald Coase, and it works in the following way.

Say that there there are a couple of fine flutes, and a few people who are
especially good flutists. How will will the flutes get to the right people? One
answer is to have the government hold a big talent contest and award the flutes
to the best player. That would be expensive and
time-consuming. And who should judge the ``best flute player''?

Here's another way: The government gives property rights in the flutes to
anyone arbitrarily (say, based on first possession). Having a property right
allows the flute owners to sell the flutes to the highest bidder. The best
flutists,
being able to profit the most off of having the best flutes (they can put on
concerts, for example), will be willing to pay the most for them. So assuming
that there are no transaction costs (a big assumption), property ownership and
pure economics allocate resources efficiently.

Do you think that efficient allocation works in practice? Is
there some reason that society might not always want property to go to whoever
is the most willing to pay for it?


\defwebsite{capps-cities-ease-homelessness}{
author=Kriston Capps,
title=Can Cities Ease Homelessness with Storage Units?,
date=Aug 25 2014,
journal=Bloomberg,
url=https://www.bloomberg.com/news/articles/2014-08-25/can-cities-ease-homelessness-with-storage-units,
}

\defjrnart{radin-property-personhood}{
Margaret Jane Radin, Property and Personhood, 34 Stanford Law Review 957 (1982) 
}

\paragraph{Personhood interests}
\Inline{radin-property-personhood} observes:
\begin{quotation}
Most people possess certain objects they feel are almost part of themselves.
These objects are closely bound up with personhood because they are part of the
way we constitute ourselves as continuing personal entities in the world. They
may be as different as people are different, but some common examples might be a
wedding ring, a portrait, an heirloom, or a house.\ldots

Once we admit that a person can be bound up with an external
``thing'' in some constitutive sense, we can argue that by virtue of this
connection the person should be accorded broad liberty with respect
to control over that ``thing.''
\end{quotation}
\sentence{radin-property-personhood at 959-960}. Consider also this story about
cities providing free storage services to those experiencing homelessness:
\begin{quote}
For the homeless, simply being able to store belongings can be transformative.
Storage bins or storage units allow them to safeguard important documents,
especially identification and other paperwork that can be hard or expensive to
replace, as well as sentimental items and keepsakes, which can't be replaced at
all. At the First United Church facility, users tend to check in sleeping
equipment during the morning---things like blankets, sleeping bags, and     
pillows---and check them out again at night. This frees people to pursue    
medical check-ups, job interviews, and housing appointments during the day: 
normal activities that are off limits for anyone who has to protect his or her
things around the clock.                                      
\end{quote}
\sentence{capps-cities-ease-homelessness}. How does this idea of property as
personality compare to the economic justifications for property above?

\defjrnart{rose-keystone}{
Carol M. Rose, Property as the Keystone Right?, 71 Notre Dame Law Review 329
(1996)
}

\paragraph{Political self-governance}
Can property ownership advance democracy? At first, the two may seem unrelated:
Property is about economic wealth and status, while governance is about civic
duties and liberties. But perhaps there is a connection. In a 1996 article,
Professor Carol Rose identifies and critiques seven possible reasons why
property might be the keystone right safeguarding all other political rights and
liberties. Among these:
\begin{quotation}
The Power-Spreading Argument\ldots. Wealth is an alternative source of power
to politics, and as long as many people can own property and attempt to earn
money, power---including political power---will necessarily remain more or less
diffused. Money talks, and in a free market economy, the freedom that everyone
has to own property or enter the market, in any way that she chooses, means that
many people can talk, and they can and will resist the political temptations to
suppress other rights.

The Independence Argument\ldots. All people should have a voice in the
political order, but to acquire that voice they need a secure baseline of
property---and if necessary, this baseline must be secured by redistribution.

The Distraction Argument\ldots. The pursuit of property can open up competing
attractions to passion-driven political feuds, and thus safeguard all the other
rights. Why muck about in politics to try to destroy the rights of others, when
money-making and business are so vastly more exciting?

The Luxury-Good Argument\ldots. Most liberties are luxury goods---they follow
after wealth is secured. On this argument, property and the resultant prosperity
may not be \emph{sufficient} for the enjoyment of liberties, but they are
certainly \emph{necessary}; without property and prosperity, other rights are in
danger.
\end{quotation}
\sentence{rose-keystone}. Do you agree with these? Do you think that, on these
arguments or others, property rights could \emph{detract} from self-governance?


\paragraph{Human flourishing}
The modern ``progressive property'' academic school posits that property serves
underlying human and social values, including ``life and human flourishing, the
protection of physical security, the ability to acquire knowledge and make
choices, and the freedom to live one's life on one's own terms. They also
include wealth, happiness, and other aspects of individual and social
well-being.''

\defjrnart{alexander-progressive-property}{
Gregory S. Alexander et al., A Statement of Progressive Property, 94 Cornell Law
Review 743 (2009)
}

By incorporating these values into the fabric of property, the proponents of the
progressive property school push back on the economic welfare-maximization views
of property, including the incentives and efficiency rationales described above,
deeming those views too focused on individual autonomy. Instead, they place
``community life'' and human relationships at the center of the objectives that
property law should serve. \sentence{alexander-progressive-property}.

\defjrnart{rosser-ambition}{
Ezra Rosser, The Ambition and Transformative Potential of Progressive
Property, 101 California Law Review 107 (2013)
}

\defjrnart{weiss-housing-justice}{
Brandon M. Weiss, Progressive Property Theory and Housing Justice Campaigns, 10
University of California Irvine Law Review 21 (2019)
}


Although progressive property has largely remained a theory within academic
circles, at least one scholar has argued that it has application to contemporary
real-world problems such as residential zoning law and rent control.
\sentence{see weiss-housing-justice}. Another critiques the theory, arguing that
it should be extended to address historical inequities and wrongdoing in the
acquisition of property. \sentence{see rosser-ambition}.
