Welcome to property law! Of the standard first-year law school curriculum,
property is arguably the subject with which you are already the most familiar,
if only because property is everywhere. The textbook you're reading: if it's on
paper, someone owns it (perhaps you); if it's electronic, then someone owns the
computer it's being shown on; plus, isn't there something called ``copyright''
in the words? The residence you live in, the buildings where you study,
the food you eat, the clothes you wear, the devices you use for work and
entertainment---all property. This is the law of the stuff that is yours, and
stuff that is everyone else's.

You enter the subject of property law filled with plenty of intuitions about how
the law should work. Indeed, much of this course will be about wrapping a legal
vocabulary around those intuitions. Are you annoyed that the landlord won't fix
the leak in your apartment? Take a look at the warranty of habitability. Arguing
with your sibling over who gets to keep the baseball cards? Perhaps you have a
question of first possession or finders' rights. Neighbors' loud parties keeping
you up at night? Check out the doctrines on nuisance and zoning law. In large
part, property law is about concepts that are already all too familiar.

But also in large part, property law is about pushing these familiar concepts
into unfamiliar territory. For example, it may seem obvious beyond question that
the law should protect property ownership.
Even toddlers develop a concept of property, as they go around
labeling toys as ``mine.'' But let's ask the seemingly obvious question: Why
does the law protect property ownership? Should we have property law in the
first place? If so, what exactly are those laws and institutions that we
call ``property''? And how far can we stretch this idea of ``property'' beyond
the conventional subject matter of land and objects?

This chapter introduces these high-level topics about the nature of property
law. It is at the beginning of the book, to help set up a mental framework of
tools and questions that you can use as you learn about the rules, doctrines,
and cases in this book. And consider revisiting this chapter every once in a
while, as you learn. If a doctrine strikes you as surprising or unexpected,
the ideas in this chapter may give voice to those thoughts.
