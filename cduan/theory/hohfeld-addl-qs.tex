\expected{hohfeld-fundamental}

\item \textbf{Wait\ldots what?} Reading legal texts is hard, academic ones even
more so. Again, though, a key skill of a lawyer is gleaning information even
from dense, complicated texts. The more you practice, the faster and better you
will get at it.

Given that this is the beginning of the book, though, let's walk through it
together. The excerpt starts with a complaint about the
misuse of the terms \emph{in personam} and\emph{in rem}. So we're jumping into
the middle of a debate, and don't even know what the debate is about! We could
go to a dictionary to look up the terms, but let's see if the article itself
defines them.

Sure enough, the next two paragraphs do that. An \emph{in personam} right seems
to be some sort of right ``availing against a single person,'' while an \emph{in
rem} right is ``against persons constituting a very large and indefinite class
of people.'' What could these be? The third paragraph gives two examples: a debt
between two people is a right \emph{in personam}, while one's own money is a
right \emph{in rem}. How can a debt between two people arise? Usually by
contract. Aha---an \emph{in personam} right must be some sort of contract right,
which is typically ``against a single person,'' while an \emph{in rem} right is
something like ``one's own money.''

One's own money---that's what you'd typically call property. So Hohfeld seems to
be talking about contract or tort rights on the one hand, which he seems to be
calling \emph{in personam}, and property rights on the other hand, which he is
calling \emph{in rem}.

\item \label{hohfeld-right}
But why is property a right against a ``very large and indefinite
class of people''? Look at the next paragraph: ``If A owns and occupies
Whiteacre, not only B but also a great many other persons\ldots are under a
duty, e.g., not to enter A's land.'' In other words, A has a legal right against
B (to kick B out of Whiteacre), and has the same right against C, D, E, and
anyone else. This may seem logical, but appreciate the conclusion: Hohfeld has
just shown that property rights are basically super-contracts. Indeed, he
emphasizes this later: ``A's Blackacre right against B [based on property] is,
intrinsically considered, of the same general character as A's Whiteacre right
against B [based on contract].''

Now observe the logical magic in the paragraph labeled ``(b).''
So far, an \emph{in rem} right is any right against a large
and indefinite class of people. This obviously includes right to a physical
object like a ``horse, watch, book, etc.,'' but what else? How about a general
right to be free from torts? Hohfeld specifically notes ``right of privacy''
torts---because these torts fit the definition of a right against an indefinite
class of people, but involve no physical thing. If, as we deduced earlier, that
\emph{in rem} rights are property rights, then Hohfeld has effectively proved
that a property right can exist with no underlying physical thing.

So far, this may seem like an exercise in logical deduction. But now is a
good time to look up \emph{in rem}. (This is generally a good
approach: Try to discern an important term in a text based on what the author
says it means, and then look it up elsewhere to see if the author is pulling a
fast one on you.) It's Latin for ``against a thing,'' precisely the opposite of
what Hohfeld says. And generally, that's probably what you thought property
was in the first place---rights that one has over land or objects.

Now we have enough information to piece everything together. The misuse of terms
that Hohfeld is complaining about is the treatment of property as rights over
things. Hohfeld shows us that even the most quintessential property, like A
owning Blackacre, is not that different from contracts. And there are other
rights, like freedom from torts or privacy violations, that are
indistinguishable from property in things insofar as all of them are against a
large and indeterminate class of people. Thus, Hohfeld is essentially arguing,
\emph{any right against a large and indeterminate class} should be considered
property.

\item \textbf{Property is anything?}
This is a tremendously broad conception of property, potentially encompassing
not just the traditional subject matter of land and physical objects, but also
pure legal relationships, statutory and constitutional rights, entitlements to
judicial remedies, rights in information, and more. If this is correct, then it
makes learning about property law really useful! But are you skeptical? Is there
something special about ``property'' that makes it distinct from other legal
rights? That's what Merrill and Smith will argue next.

\expectnext{merrill-smith-economics}

\item \textbf{Property versus possession.}
Hohfeld further notes a key distinction between physical and legal
relationships. A person's ability to ``physically control and use such thing''
and ``physically exclude others'' from it ``could as well exist quite apart
from, or occasionally in spite of, the law of organized society.'' In other
words, whatever this ``property'' thing is, it's a relationship created by law
that is distinct from physical control over property. We will call physical
control \term{possession}, and carefully distinguish possession from property
rights---something that you may not typically do in ordinary conversation.


\item \textbf{Hohfeld's legacy.} Generally, Hohfeld is cited for two major
concepts. The first is the understanding of property as a collection of rights
against others, typically an indeterminate class of third parties. Today, we
often refer to this concept as the \term{bundle of rights}, or sometimes a
metaphorical ``bundle of sticks.''

What's in that bundle of rights? We will find out through the case law and
readings, of course. But for now, think about what rights you would expect to
have over your own property against others. What do you expect other people not
to do with respect to your house, your books, or your car? What do you expect
that you yourself are allowed to do with them? Do you expect the list of rights
to depend on whether the underlying thing being protected is land, objects,
privacy rights, or something else?

Hohfeld's second concept, often called ``Hohfeldian analysis,'' is a collection
of terms that can be used to describe legal relationships between people. The
full terminology is complex, but there is one important insight from it that
shows up in the reading above. Read note~\ref{hohfeld-right} again---Hohfeld
says that B and others ``are under a \term{duty}, e.g., not to enter A's land,''
to explain why A has a \term{right} \emph{in rem}. Duties and rights are ``jural
correlatives,'' as Hohfeld calls them: If A has a right with respect to B, then
B owes a legal duty to A\@. The idea that, for every right, there is a equal and
opposite duty, will be incredibly useful throughout property law (and all of law
generally).
