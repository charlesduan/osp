\item \textbf{Wait\ldots what?} Reading legal texts is hard, academic ones even
more so. Again, though, a key skill of a lawyer is gleaning information even
from dense, complicated texts. The more you practice, the faster and better you
will get at it.

Given that this is the beginning of the book, though, let's walk through these
articles together. The excerpt from Hohfeld starts with a complaint about the
misuse of the terms \emph{in personam} and\emph{in rem}. So we're jumping into
the middle of a debate, and don't even know what the debate is about! We could
go to a dictionary to look up the terms, but let's see if the article itself
defines them.

Sure enough, the next two paragraphs do that. An \emph{in personam} right seems
to be some sort of right ``availing against a single person,'' while an \emph{in
rem} right is ``against persons constituting a very large and indefinite class
of people.'' What could these be? The third paragraph gives two examples: a debt
between two people is a right \emph{in personam}, while one's own money is a
right \emph{in rem}. How can a debt between two people arise? By tort or
contract? Aha---an \emph{in personam} right must be some sort of tort or
contract right, which is typically ``against a single person,'' while an
\emph{in rem} right is property ownership!

But why is property ownership a right against a ``very large and indefinite
class of people''? Look at the next paragraph: ``If A owns and occupies
Whiteacre, not only B but also a great many other persons\ldots are under a
duty, e.g., not to enter A's land.'' In other words, A has a legal right against
B (to kick B out of Whiteacre), and has the same right against C, D, E, and
anyone else. This may seem logical, but appreciate the conclusion: Hohfeld has
just shown that property rights are basically super-contracts. Indeed, he
emphasizes this later: ``A's Blackacre right against B [based on property] is,
intrinsically considered, of the same general character as A's Whiteacre right
against B [based on contract].''

Now observe the logical magic Hohfeld performs in the paragraph labeled ``(b).''
So far, his definition of an \emph{in rem} right is any right against a large
and indefinite class of people. This obviously includes right to a physical
object like a ``horse, watch, book, etc.,'' but what else? How about a general
right to be free from torts? Hohfeld specifically notes ``right of privacy''
torts---because these torts fit the definition of a right against an indefinite
class of people, but involve no physical thing. If, as we deduced earlier, by
\emph{in rem} right Hohfeld is referring to property rights, then Hohfeld has
effectively proved that a property right can exist with no underlying physical
thing.

\item \textbf{Property Is Anything?}
This is a tremendously broad conception of property, potentially encompassing
not just the traditional subject matter of land and physical objects, but also
pure legal relationships, statutory and constitutional rights, entitlements to
judicial remedies, rights in information, and more. If this is correct, then it
makes learning about property law really useful! But are you skeptical? Is there
something special about ``property'' that makes it distinct from other legal
rights? That's what Merrill and Smith are arguing. If you didn't follow their
argument the first time around, take a second look and see if you can grasp
their disagreement.

\item \textbf{Property Versus Possession}
Hohfeld further notes a key distinction between physical and legal
relationships. A person's ability to ``physically control and use such thing''
and ``physically exclude others'' from it ``could as well exist quite apart
from, or occasionally in spite of, the law of organized society.'' In other
words, whatever this ``property'' thing is, it's a relationship created by law
that is distinct from physical control over property. We will call physical
control \term{possession}, and carefully distinguish possession from property
rights---something that you may not typically do in ordinary conversation.


\item What's in that bundle of rights anyway? The case law and other law will
establish the rights for any particular type of property, of course, but think
about what rights you would expect to have over your own property against
others. Do you expect the list of rights to be consistent across many different
kinds of property? Does your answer inform whether Hohfeld or Merrill and Smith
have the better theory of what property is?

