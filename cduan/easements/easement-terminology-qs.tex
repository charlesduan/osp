\item Characterize the following easements using the terms above. Is it
affirmative or negative? Appurtenant or in gross? An easement of access, a
profit, or something else? What is the dominant estate and the servient estate?
Who has the benefit, and who has the burden?

\begin{enumerate}
\item Anna owns Blackacre. Brad has the right to walk across Blackacre to reach
his own property, Whiteacre.

\item Anna owns Blackacre, which has several blackberry bushes. Connie, a chef,
has the right to come onto Blackacre and harvest blackberries.

\item Anna owns Blackacre, which is at the bottom of a hill. Durant owns
Whiteacre, at the top of the hill, with a house on the hill. There is a
retaining wall on Blackacre, preventing the soil from the hill from falling down
(and bringing Durant's house down with it). A court prohibits Anna from tearing
down the retaining wall. (This is called an \term{easement of support}.)
\end{enumerate}
To be sure, there are a lot of terms here, and it is not necessary to fully
memorize them all. But the exercise of categorizing easements and the relevant
parties will help you keep track of the relevant rights and duties, especially
when ownership starts changing hands.

\item The terminology suggests a sharp distinction between affirmative
easements, that give the benefit holder a right to do something, and negative
easements, which impose a duty on the burdened estate not to do something. Is
the distinction really so sharp? In the examples above, doesn't Anna have a duty
not to build a fence to keep Brad out? Doesn't Durant have a right to keep the
retaining wall in place?\having{hohfeld-fundamental}{ (This is the power of the
Hohfeldian theory that rights and duties always come in pairs.)}{}{}

\item \emph{United States v.~Turoff}, \having{us-v-turoff}{presented earlier in
this book}{presented later in this book}{701 F. Supp. 981 (E.D.N.Y. 1988)},
considers the nature of government-issued taxicab medallions that authorize the
bearer to operate a taxi service:
\begin{quote}
The government also contends that the medallion is, in essence, the equivalent
of an easement to use the city streets. At the risk of dwelling too long on the
esoterica of property, the medallions could not properly be equated with
easements. An easement is generally appurtenant, which is to say that it is a
right which the owner of one parcel of land (the dominant tenement) may
exercise in or over the land of another (the servient tenement) for the benefit
of the former. An easement in gross is a right created in a person to use the
land of another, which the owner of that easement may enjoy even though he does
not own or possess a dominant estate. Although the concept of an easement in
gross has been recognized, such an easement is rare.
\end{quote}
You should now be equipped to follow and evaluate the logic of this argument.
