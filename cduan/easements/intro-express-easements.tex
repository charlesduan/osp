Because easements are interests in land, express easements are subject to the
Statute of Frauds. Failures to comply with the statute may still be enforced in
cases of reasonable detrimental reliance. \textit{See, e.g.},
\textsc{Restatement (Third) of Property (Servitudes)} \S~2.9. 

\paragraph{Third parties}
Easements are often created as part of the transfer of land (e.g., selling a
property, but retaining the right to use its parking lot). Traditionally,
grantors could reserve an easement in the conveyed land for themselves, but
could not create an easement for the benefit of a third party. This rule led to
extra transactions. Where the traditional rule applied, if A wanted to convey to
B while creating an easement for C, A could convey to C who would then convey to
B, while reserving an easement. 

The modern trend discards this restriction. \textit{See, e.g.}, \emph{Minton v.
Long}, 19 S.W.3d 231, 238 (Tenn. Ct. App. 1999);
\emph{Willard v. First Church of Christ, Scientist}, 7 Cal. 3d 473, 476-77
(1972) (describing the traditional reservation rule as ``clearly an inapposite
feudal shackle today'').
The modern \textsc{Restatement} likewise dispenses
with the traditional approach, allowing the direct creation of easements on
behalf of third parties. \textsc{Restatement} \S~2.6. Some jurisdictions
nonetheless retain the bar, citing reliance interests and the prospect that such
easements create instability in title records. \emph{Estate of Thomson v. Wade},
509 N.E.2d 309, 310 (N.Y. 1987).

There is an argument that the extra
transactions required by the traditional rule promote better title indexing. The
\textsc{Restatement} observes:
\begin{quote}
To avoid the prohibition, two conveyances must be used: the first conveys the
easement to the intended beneficiary; the second conveys the servient estate to
the intended transferee. The only virtue of the rule is that it tends to ensure
that a recorded easement will be properly indexed in the land-records system,
but there are so many exceptions to the rule, where it is still in force, that
it does not fill that function very well.
\end{quote}
\textsc{Restatement (Third) of Property (Servitudes)} \S~2.6 cmt. a (2000).
\having{hartig-v-stratman}{In light of \textit{Hartig v. Stratman}, from our
chapter on Recording Acts, are}{In light of \textit{Hartig v. Stratman}, from
our chapter on Recording Acts, are}{Are}
you persuaded that the benefits of a separate transaction for recording purposes
outweigh the costs?

