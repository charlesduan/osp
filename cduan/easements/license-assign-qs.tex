\defcase{blackburn}{Blackburn v. Lefebvre, 976 So. 2d 482 (Ala. Civ. App. 2007)}
\defcase{shearer}{Shearer v. Hodnette, 674 So. 2d 548 (Ala. Civ. App. 1995)}
\defbook{miller-starr}{
author=Harry D. Miller et al.,
title=Miller \& Starr California Real Estate,
edition=4,
year=2023,
}
\defbook{bruce}{
Jon W. Bruce et al., The Law of Easements and Licenses in Land (2024)
}
\defbook{rest-1-prop}{
title=Restatement (First) of Property,
year=2024,
}
\defcase{richardson-v-franc}{
Richardson v. Franc, 182 Cal. Rptr. 3d 853 (Cal. App. 2015)
}
\defcasedoc{franc-brief}{
name=Opening Brief of Appellants Greg and Terrie Franc,
date=july 25 2013,
citation=richardson-v-franc,
docket=No. A137815,
}


\item \textbf{Assigning licenses.} \emph{Richardson} affirms a trial court's
determination of an ``irrevocable parol license for both respondents, \emph{and
respondents' successors-in-interest}.'' Is the benefit of a license
assignable to others along with the underlying land, like an easement
appurtenant? The authorities pretty uniformly say no: ``an irrevocable license
is a nontransferable personal interest.'' \sentence{bruce at S 11:9 (citing:
blackburn at 493-494; shearer at 551); rest-1-prop at S 517, comment a
(``Privileges of use constituting licenses often arise out of relations that are
highly personal. Where they do so arise they are commonly intended to constitute
privileges personal to the licensee alone.'')}. Even the very section of the
treatise that \emph{Richardson} itself cites states, ``The privilege conferred
by a license is personal to the licensee and cannot be inherited, conveyed, or
assigned'' under California law. \sentence{miller-starr at S 15:2}.

Is \emph{Richardson} inconsistent with its own authorities? It is possible that
the court thought this license unusual and thus assignable, but the opinion
offers no reasoning along these lines. A better explanation is that the Francs'
attorney simply didn't make the point. In relevant part, the Francs'
appellate brief levies three arguments against the scope of the irrevocable
license: (1) that the license should have a time limit, (2) that the license
``took fee title'' away from the Francs, and (3) that it was unclear how much
new planting was allowed under the license. \sentence{franc-brief at 37-40}. The
brief did not specifically argue that licenses were personal and not
transferable, and so the appellate court may not have had occasion to question
that aspect of the judgment.
