\expected{steinberg-v-columbia}
\expected{hinman-pacific-air}

\item Have you seen one of these \emph{New Yorker}--style map drawings before?
(Perhaps you've drawn one?) Did you stop to think whether they were copyright
infringing?

\item As this case shows, the ``substantial similarity'' standard for copyright
infringement---that is, the boundary of copyright protection---comes with an
important caveat. The ``idea--expression dichotomy'' doctrine specifies that
ideas are not subject to copyright protection and thus may be freely copied. The
dividing line between ideas and expression is not (and probably cannot be)
precisely defined, but you probably have a basic intuition for it. One can copy
the basic framework of \emph{Romeo and Juliet} into a new work, say \emph{West
Side Story}, without taking any of the words of Shakespeare's expression.

The court here seems convinced that expression, and not just ideas, was copied
in this case. Do you agree? Is there any specific element that is actually the
same between the two pictures (other than the fonts, which the court notes are
not copyrightable), that seems significant enough to call ``expression''?

\item Should overall style be within the boundaries of a copyright? For example,
if an artist is known for paintings that use bright primary colors and grid-like
black lines (that is, Piet Mondrian), is it copyright infringement to make
another painting with those same colors and grid lines, but arranged
differently?

Generally, courts have been reluctant to treat artistic style alone as
copyrightable expression. The question has received increasing prominence
recently, though, due to the ability of generative artificial intelligence to
produce artwork in the style of known artists.

\item The shape of useful articles, like cars, is typically not copyright
protectable. The reason is somewhat complex, but generally it is because if the
shape of the car makes the car perform better, then the car shape should be the
subject of a patent, not a copyright. Unrelatedly, characters in stories are
copyrightable. That is why it is an infringement to make a sequel, among other
things.


\defcase{dc-comics-v-towle}{
parties=DC Comics v. Towle,
cite=802 F.3d 1012,
court=9th Cir.,
year=2015,
}

What about the Batmobile? \sentence{see dc-comics-v-towle}.

