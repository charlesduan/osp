\having{walters-v-tucker}{\emph{Walters} exemplifies}{\emph{Walters v.~Tucker},
later in this book, exemplifies}{One way of describing property is}
\term{peripheral claiming}, in which the owned property is defined based on the
positions of its edges (the ``periphery'' of the property). The other option for
describing property boundaries is \term{central claiming}, in which the owner
places a stake in the ground and asserts ownership in anything sufficiently
close to the stake. (You might have seen this idea in movies about the old
American West.) For real property, peripheral claiming is used almost
exclusively today, as it has the tremendous advantage of precision.

With intellectual property, though, it is often not so simple to identify the
boundaries. Patents rely on peripheral claiming: The text of a patent
document
contains paragraphs called ``patent claims'' that lay out, in detailed legal and
technical terminology, the boundaries of what the patent holder considers the
``invention,'' such that products or services falling within those boundaries
are infringing. Here's an example of one such patent claim, from U.S. Patent
No.~6,004,596:
\begin{quotation}
\noindent 1. A sealed crustless sandwich, comprising:

a first bread layer having a first perimeter surface coplanar to a contact
surface;

at least one filling of an edible food juxtaposed to said contact surface;

a second bread layer juxtaposed to said at least one filling opposite of
said first bread layer, wherein said second bread layer includes a second
perimeter surface similar to said first perimeter surface;

a crimped edge directly between said first perimeter surface and said
second perimeter surface for sealing said at least one filling between said
first bread layer and said second bread layer;

wherein a crust portion of said first bread layer and said second bread
layer has been removed.
\end{quotation}
You can think of the patent claim as a checklist. A sandwich meeting all the
listed requirements (first bread layer, filling of an edible food, etc.) would
``fall within the scope of the patent claim,'' and thus infringe the patent
holder's rights.

Despite the seeming complexity of the words in patent claims, they are a far cry
from the precision of real estate boundaries. In the sandwich patent claim, for
example, what's ``bread''? Does a cracker count? There's a ``first bread layer''
and a ``second bread layer,'' but what about three-layer club sandwiches? Does a
hot dog bun count as one bread layer or two? The process of a court resolving
these ambiguities and determining what exactly a patent claim covers is called
\emph{claim construction}, and it is one of the most difficult and uncertain
parts of patent litigation.

\defcase{nautilus-v-biosig}{Nautilus, Inc. v. Biosig Instruments, Inc., 572 U.S.
898 (2014)}

There is a limit on how much ambiguity a patent claim can have. According to 35
U.S.C. \S~112(b), a patent claim must ``particularly point[] out and distinctly
claim[] the subject matter which the inventor or a joint inventor regards as
the invention.'' A patent claim that fails this requirement is considered
``indefinite'' and invalid. In \inline{nautilus-v-biosig}\optclause, the Supreme
Court interpreted this provision generously, holding that a patent claim need
only provide ``reasonable certainty'' and not absolute precision.

What are the arguments for and against greater precision in patent claims?
Think from the perspective of a manufacturer trying to avoid infringing a
patent---does it seem unfair that the manufacturer can't know without expensive
litigation? Could ambiguity in patent claims be exploited in problematic ways?
On the other hand, do you see any difficulties in requiring inventors to find
precise words to describe their inventions?

Peripheral claiming is difficult for patents, but virtually impossible for
copyrights and trademarks. Could you describe, in precise words, the extent of
things that are too much like \emph{Harry Potter} or the Nike swoosh? As a
result, central claiming is necessary for these. A copyright, for example,
creates a right to exclude other works that are ``substantially similar'' to the
copyright-protected one---a form of central claiming. A trademark similarly
blocks ``confusingly similar'' marks and uses. That means, however, that the
task of determining infringement of these centrally-claimed forms of
intellectual property is a fact-intensive, difficult question for courts, often
involving fuzzy multi-factor tests.

