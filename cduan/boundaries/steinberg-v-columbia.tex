\reading{Steinberg v. Columbia Pictures Industries, Inc.}

\readingcite{663 F.Supp. 706 (S.D.N.Y. 1987)}

\opinion \textsc{Stanton}, District Judge.

In these actions for copyright infringement, plaintiff Saul Steinberg is suing
the producers, promoters, distributors and advertisers of the movie ``Moscow on
the Hudson'' (``Moscow''). Steinberg is an artist whose fame derives in part
from cartoons and illustrations he has drawn for \textit{The New Yorker}
magazine.\ldots
%Defendant Columbia Pictures Industries, Inc. (Columbia) is in the
%business of producing, promoting and distributing motion pictures, including
%``Moscow.'' Defendant RCA Corporation (RCA) was involved with Columbia in
%promoting and distributing the home video version of ``Moscow,'' and defendant
%Diener Hauser Bates Co. (DHB) acted as an advertising agent for ``Moscow.'' The
%other defendants were added to the complaint pursuant to a memorandum decision
%of this court dated November 17, 1986. These defendants fall into two
%categories: (1) affiliates of Columbia and RCA that were involved in the
%distribution of ``Moscow'' here and/or abroad, and (2) owners of major
%newspapers that published the allegedly infringing advertisement.
%
%The defendants in the second-captioned action either are joint ventures
%affiliated with Columbia or are newspapers that published the allegedly
%infringing advertisement for ``Moscow.'' This action was consolidated with the
%first by stipulation dated April 3, 1987.
%
Plaintiff alleges that defendants' promotional poster for ``Moscow'' infringes
his copyright on an illustration that he drew  for \textit{The New Yorker} and
that appeared on the cover of the March 29, 1976 issue of the magazine, in
violation of 17 U.S.C. \S\S~101-810. Defendants deny this allegation and assert
the affirmative defenses of fair use as a parody, estoppel and laches.

Defendants have moved, and plaintiff has cross-moved, for summary judgment. For
the reasons set forth below, this court rejects defendants' asserted defenses
and grants summary judgment on the issue of copying to plaintiff.



%\readinghead{I}
%
%To grant summary judgment, Fed.R. Civ.P. 56 requires a court to find that
%``there is no genuine issue as to any material fact and that the moving party is
%entitled to a judgment as a matter of law.'' In reaching its decision, the court
%must ``assess whether there are any factual issues to be tried, while resolving
%ambiguities and drawing reasonable inferences against the moving party.''
%\textit{Knight v. U.S. Fire Ins. Co.}, 804 F.2d 9, 11 (2d Cir.1986),
%\textit{citing} \textit{Anderson v. Liberty Lobby}, 477 U.S. 242, 106 S.Ct.
%2505, 2509-11, 91 L.Ed.2d 202 (1986).
%
%Summary judgment is often disfavored in copyright cases, for courts are
%generally reluctant to make subjective comparisons and determinations.
%\textit{Hoehling v. Universal City Studios, Inc.}, 618 F.2d 972, 977 (2d
%Cir.1980), \textit{citing} \textit{Arnstein v. Porter}, 154 F.2d 464, 474 (2d
%Cir.1946). Recently, however, this circuit has ``recognized that a court may
%determine non-infringement as a matter of law on a motion for summary
%judgment.'' \textit{Warner Brothers v. American Broadcasting Cos.}, 720 F.2d
%231, 240 (2d Cir.1983), \textit{quoting} \textit{Durham Industries, Inc. v. Tomy
%Corp.}, 630 F.2d 905, 918 (2d Cir.1980). \textit{See also} \textit{Hoehling},
%618 F.2d at 977; \textit{Walker v. Time-Life Films, Inc.}, 615 F.Supp. 430, 434
%(S.D.N.Y. 1985), \textit{aff'd}, 784 F.2d 44 (2d Cir.1986), \textit{cert.
%denied}, \_\_\_ U.S. \_\_\_, 106 S.Ct. 2278, 90 L.Ed.2d 721 (1986). ``When the
%evidence is so overwhelming that a court would be justified in ordering a
%directed verdict at trial, it is proper to grant summary judgment.''
%\textit{Silverman v. CBS Inc.}, 632 F.Supp. 1344, 1352 (S.D.N.Y.1986) (awarding
%summary judgment to defendant on counterclaim of copyright infringement).
%
%The voluminous submissions that accompanied these cross-motions leave no factual
%issues concerning which further evidence is likely to be presented at a trial.
%Moreover, the factual determinations necessary to this decision do not involve
%conflicts in testimony that would depend for their resolution on an assessment
%of witness credibility. In addition, this case is different from most copyright
%infringement actions, in which it is preferable to leave the determination of
%the issue to a jury: each party has implied that its case is complete by moving
%for summary judgment, and as neither side has requested a jury, the court would
%be the trier of fact at trial. Finally, the interests of judicial economy are
%also served by deciding the case at its present stage. Summary judgment is
%therefore appropriate.
%
%

\readinghead{II}

The essential facts are not disputed by the parties despite their disagreements
on nonessential matters. On March 29, 1976, \textit{The New Yorker} published as
a cover illustration the work at issue in this suit, widely known as a parochial
New Yorker's view of the world. The magazine registered this illustration with
the United States Copyright Office and subsequently assigned the copyright to
Steinberg. Approximately three months later, plaintiff and \textit{The New
Yorker} entered into an agreement to print and sell a certain number of posters
of the cover illustration.

It is undisputed that unauthorized duplications of the poster were made and
distributed by unknown persons, although the parties disagree on the extent to
which plaintiff attempted to prevent the distribution of those counterfeits.
Plaintiff has also conceded that numerous posters have been created and
published depicting other localities in the same manner that he depicted New
York in his illustration. These facts, however, are irrelevant to the merits of
this case, which concerns only the relationship  between plaintiff's and
defendants' illustrations.

\begin{figure}
\begin{center}
\usegraphic[width=0.4\textwidth, height=0.5\textheight]{steinberg}
\usegraphic[width=0.4\textwidth, height=0.5\textheight]{moscow}
\end{center}
\caption{Steinberg's copyrighted magazine cover, and the accused movie poster.}
\end{figure}

Defendants' illustration was created to advertise the movie ``Moscow on the
Hudson,'' which recounts the adventures of a Muscovite who defects in New York.
In designing this illustration, Columbia's executive art director, Kevin Nolan,
has admitted that he specifically referred to Steinberg's poster, and indeed,
that he purchased it and hung it, among others, in his office. Furthermore,
Nolan explicitly directed the outside artist whom he retained to execute his
design, Craig Nelson, to use Steinberg's poster to achieve a more recognizably
New York look. Indeed, Nelson acknowledged having used the facade of one
particular edifice, at Nolan's suggestion that it would render his drawing more
``New York-ish.'' Curtis Affidavit ¶ 28(c). While the two buildings are not
identical, they are so similar that it is impossible, especially in view of the
artist's testimony, not to find that defendants' impermissibly copied
plaintiff's.\readingfootnote{1}{Nolan claimed also to have been inspired by some
of the posters that were inspired by Steinberg's; such secondary inspiration,
however, is irrelevant to whether or not the ``Moscow'' poster infringes
plaintiff's copyright by having impermissibly copied it.}

To decide the issue of infringement, it is necessary to consider the posters
themselves. Steinberg's illustration presents a bird's eye view across a portion
of the western edge of Manhattan, past the Hudson River and a telescoped version
of the rest of the United States and the Pacific Ocean, to a red strip of
horizon, beneath which are three flat land masses labeled China, Japan and
Russia. The name of the magazine, in \textit{The New Yorker}'s usual typeface,
occupies the top fifth of the poster, beneath a thin band of blue wash
representing a stylized sky.

The parts of the poster beyond New York are minimalized, to symbolize a New
Yorker's myopic view of the centrality of his city to the world. The entire
United States west of the Hudson River, for example, is reduced to a brown strip
labeled ``Jersey,'' together with a light green trapezoid with a few rudimentary
rock outcroppings and the names of only seven cities and two states scattered
across it. The few blocks of Manhattan, by contrast, are depicted and colored in
detail. The four square blocks of the city, which occupy the whole lower half of
the poster, include numerous buildings, pedestrians and cars, as well as parking
lots and lamp posts, with water towers atop a few of the buildings. The
whimsical, sketchy style and spiky lettering are recognizable as Steinberg's.

The ``Moscow'' illustration depicts the three main characters of the film on the
lower third of their poster, superimposed on a bird's eye view of New York City,
and continues eastward across Manhattan and the Atlantic Ocean, past a
rudimentary evocation of Europe, to a clump of recognizably Russian-styled
buildings on the horizon, labeled ``Moscow.'' The movie credits appear over the
lower portion of the characters. The central part of the poster depicts
approximately four New York city blocks, with fairly detailed buildings,
pedestrians and vehicles, a parking lot, and some water towers and lamp posts.
Columbia's artist added a few New York landmarks at apparently random places in
his illustration, apparently to render the locale more easily recognizable.
Beyond the blue strip labeled ``Atlantic Ocean,'' Europe is represented by
London, Paris and Rome, each anchored by a single landmark (although the
landmark used for Rome is the Leaning Tower of Pisa).

The horizon behind Moscow is delineated by a red crayoned strip, above which are
the title of the movie and a brief textual introduction to the plot. The poster
is crowned by a thin strip of blue wash, apparently a stylization of the sky.
This poster is executed in a blend of styles: the three characters, whose
likenesses were copied from a photograph, have realistic faces and somewhat
sketchy clothing, and the city blocks are drawn in a fairly detailed but sketchy
style. The lettering on the drawing is spiky, in block-printed handwritten
capital letters substantially identical to plaintiff's, while the printed texts
at the top and bottom of the poster are in the  typeface commonly associated
with \textit{The New Yorker} magazine.\readingfootnote{2}{The typeface is not a
subject of copyright, but the similarity reinforces the impression that
defendants copied plaintiff's illustration.}



\readinghead{III}

To succeed in a copyright infringement action, a plaintiff must prove ownership
of the copyright and copying by the defendant. \textit{Reyher v. Children's
Television Workshop}, 533 F.2d 87, 90 (2d Cir.1976); \textit{Durham Industries},
630 F.2d at 911; \textit{Novelty Textile Mills, Inc. v. Joan Fabrics Corp.}, 558
F.2d 1090, 1092 (2d Cir.1977). There is no substantial dispute concerning
plaintiff's ownership of a valid copyright in his illustration. Therefore, in
order to prevail on liability, plaintiff need establish only the second element
of the cause of action.

``Because of the inherent difficulty in obtaining direct evidence of copying, it
is usually proved by circumstantial evidence of access to the copyrighted work
and substantial similarities as to protectible material in the two works.''
\textit{Reyher}, 533 F.2d at 90, \textit{citing} \textit{Arnstein v. Porter},
154 F.2d 464, 468 (2d Cir.1946). \textit{See also} \textit{Novelty Textile
Mills}, 558 F.2d at 1092. ``Of course, if there are no similarities, no amount
of evidence of access will suffice to prove copying.'' \textit{Arnstein v.
Porter}, 154 F.2d at 468. \textit{See also} \textit{Novelty Textile Mills}, 558
F.2d at 1092 n. 2.

Defendants' access to plaintiff's illustration is established beyond
peradventure. Therefore, the sole issue remaining with respect to liability is
whether there is such substantial similarity between the copyrighted and accused
works as to establish a violation of plaintiff's copyright. The central issue of
``substantial similarity,'' which can be considered a close question of fact,
may also validly be decided as a question of law. \textit{Berkic v. Crichton},
761 F.2d 1289, 1292 (9th Cir.1985), \textit{citing} \textit{Sid \& Marty Krofft
Television Productions, Inc. v. McDonald's Corp.}, 562 F.2d 1157 (9th Cir.1977).

``Substantial similarity'' is an elusive concept. This circuit has recently
recognized that

\begin{quote} [t]he ``substantial similarity'' that supports an inference of
copying sufficient to establish infringement of a copyright is not a concept
familiar to the public at large. It is a term to be used in a courtroom to
strike a delicate balance between the protection to which authors are entitled
under an act of Congress and the freedom that exists for all others to create
their works outside the area protected by infringement. \end{quote}
\textit{Warner Bros.}, 720 F.2d at 245.

The definition of ``substantial similarity'' in this circuit is ``whether an
average lay observer would recognize the alleged copy as having been
appropriated from the copyrighted work.'' \textit{Ideal Toy Corp. v. Fab-Lu
Ltd.}, 360 F.2d 1021, 1022 (2d Cir.1966); \textit{Silverman v. CBS, Inc.}, 632
F.Supp. at 1351-52. A plaintiff need no longer meet the severe ``ordinary
observer'' test established by Judge Learned Hand in \textit{Peter Pan Fabrics,
Inc. v. Martin Weiner Corp.}, 274 F.2d 487 (2d Cir.1960). \textit{Uneeda Doll
Co., Inc. v. Regent Baby Products Corp.}, 355 F.Supp. 438, 450 (E.D.N.Y.1972).
Under Judge Hand's formulation, there would be substantial similarity only where
``the ordinary observer, unless he set out to detect the disparities, would be
disposed to overlook them, and regard their aesthetic appeal as the same.'' 274
F.2d at 489.

Moreover, it is now recognized that ``[t]he copying need not be of every detail
so long as the copy is substantially similar to the copyrighted work.''
\textit{Comptone Co. v. Rayex Corp.}, 251 F.2d 487, 488 (2d Cir. 1958).
\textit{See also} \textit{Durham Industries}, 630 F.2d at 911-12;
\textit{Novelty Textile Mills}, 558 F.2d at 1092-93.

In determining whether there is substantial similarity between two works, it is
crucial to distinguish between an idea and its expression. It is an axiom of
copyright law, established in the case law and since codified at 17 U.S.C.
\S~102(b), that only the  particular expression of an idea is protectible, while
the idea itself is not. \textit{See, e.g.}, \textit{Durham Industries}, 630 F.2d
at 912; \textit{Reyher}, 533 F.2d at 90, \textit{citing} \textit{Mazer v.
Stein}, 347 U.S. 201, 217, 74 S.Ct. 460, 470, 98 L.Ed. 630 (1954); \textit{Baker
v. Selden}, 101 U.S. (11 Otto) 99, 25 L.Ed. 841 (1879). \textit{See also}
\textit{Warner Bros.}, 720 F.2d at 239.

``The idea/expression distinction, although an imprecise tool, has not been
abandoned because we have as yet discovered no better way to reconcile the two
competing societal interests that provide the rationale for the granting of and
restrictions on copyright protection,'' namely, both rewarding individual
ingenuity, and nevertheless allowing progress and improvements based on the same
subject matter by others than the original author. \textit{Durham Industries},
630 F.2d at 912, \textit{quoting} \textit{Reyher}, 533 F.2d at 90.

There is no dispute that defendants cannot be held liable for using the
\textit{idea} of a map of the world from an egocentrically myopic perspective.
No rigid principle has been developed, however, to ascertain when one has gone
beyond the idea to the expression, and ``[d]ecisions must therefore inevitably
be ad hoc.'' \textit{Peter Pan Fabrics, Inc. v. Martin Weiner Corp.}, 274 F.2d
487, 489 (2d Cir.1960) (L. Hand, J.). As Judge Frankel once observed, ``Good
eyes and common sense may be as useful as deep study of reported and unreported
cases, which themselves are tied to highly particularized facts.''
\textit{Couleur International Ltd. v. Opulent Fabrics, Inc.}, 330 F.Supp. 152,
153 (S.D.N.Y.1971).

Even at first glance, one can see the striking stylistic relationship between
the posters, and since style is one ingredient of ``expression,'' this
relationship is significant. Defendants' illustration was executed in the
sketchy, whimsical style that has become one of Steinberg's hallmarks. Both
illustrations represent a bird's eye view across the edge of Manhattan and a
river bordering New York City to the world beyond. Both depict approximately
four city blocks in detail and become increasingly minimalist as the design
recedes into the background. Both use the device of a narrow band of blue wash
across the top of the poster to represent the sky, and both delineate the
horizon with a band of primary red.\readingfootnote{3}{Defendants claim that
since this use of thin bands of primary colors is a traditional Japanese
technique, their adoption of it cannot infringe Steinberg's copyright. This
argument ignores the principle that while ``[o]thers are free to copy the
original ... [t]hey are not free to copy the copy.'' \textit{Bleistein v.
Donaldson Lithographing Co.}, 188 U.S. 239, 250, 23 S.Ct. 298, 300, 47 L.Ed. 460
(1903) (Holmes, J.). \textit{Cf.} \textit{Dave Grossman Designs, Inc. v.
Bortin}, 347 F.Supp. 1150, 1156-57 (N.D.Ill.1972) (an artist may use the same
subject and style as another ``so long as the second artist does not
\textit{substantially copy} [the first artist's] specific expression of his
idea.'')}

The strongest similarity is evident in the rendering of the New York City
blocks. Both artists chose a vantage point that looks directly down a wide
two-way cross street that intersects two avenues before reaching a river.
Despite defendants' protestations, this is not an inevitable way of depicting
blocks in a city with a grid-like street system, particularly since most New
York City cross streets are one-way. Since even a photograph may be copyrighted
because ``no photograph, however simple, can be unaffected by the personal
influence of the author,'' \textit{Time Inc. v. Bernard Geis Assoc.}, 293
F.Supp. 130, 141 (S.D.N.Y. 1968), \textit{quoting} \textit{Bleistein,
supra}\textit{}, one can hardly gainsay the right of an artist to protect his
choice of perspective and layout in a drawing, especially in conjunction with
the overall concept and individual details. Indeed, the fact that defendants
changed the names of the streets while retaining the same graphic depiction
weakens their case: had they intended their illustration realistically to depict
the streets labeled on the poster, their four city blocks would not so closely
resemble plaintiff's four city blocks. Moreover, their argument that they
intended the jumble of streets and landmarks and buildings to symbolize their
Muscovite protagonist's confusion in a new city does not detract from the strong
similarity between their poster and Steinberg's.

 While not all of the details are identical, many of them could be mistaken for
one another; for example, the depiction of the water towers, and the cars, and
the red sign above a parking lot, and even many of the individual buildings. The
shapes, windows, and configurations of various edifices are substantially
similar. The ornaments, facades and details of Steinberg's buildings appear in
defendants', although occasionally at other locations. In this context, it is
significant that Steinberg did not depict any buildings actually erected in New
York; rather, he was inspired by the general appearance of the structures on the
West Side of Manhattan to create his own New York-ish structures. Thus, the
similarity between the buildings depicted in the ``Moscow'' and Steinberg
posters cannot be explained by an assertion that the artists happened to choose
the same buildings to draw. The close similarity can be explained only by the
defendants' artist having copied the plaintiff's work. Similarly, the locations
and size, the errors and anomalies of Steinberg's shadows and streetlight, are
meticulously imitated.

In addition, the Columbia artist's use of the childlike, spiky block print that
has become one of Steinberg's hallmarks to letter the names of the streets in
the ``Moscow'' poster can be explained only as copying. There is no inherent
justification for using this style of lettering to label New York City streets
as it is associated with New York only through Steinberg's poster.

While defendants' poster shows the city of Moscow on the horizon in far greater
detail than anything is depicted in the background of plaintiff's illustration,
this fact alone cannot alter the conclusion. ``Substantial similarity'' does not
require identity, and ``duplication or near identity is not necessary to
establish infringement.'' \textit{Krofft}, 562 F.2d at 1167. Neither the
depiction of Moscow, nor the eastward perspective, nor the presence of randomly
scattered New York City landmarks in defendants' poster suffices to eliminate
the substantial similarity between the posters. As Judge Learned Hand wrote,
``no plagiarist can excuse the wrong by showing how much of his work he did not
pirate.'' \textit{Sheldon v. Metro-Goldwyn Pictures Corp.}, 81 F.2d 49, 56 (2d
Cir.), \textit{cert. denied}, 298 U.S. 669, 56 S.Ct. 835, 80 L.Ed. 1392 (1936).

Defendants argue that their poster could not infringe plaintiff's copyright
because only a small proportion of its design could possibly be considered
similar. This argument is both factually and legally without merit. ``[A]
copyright infringement may occur by reason of a substantial similarity that
involves only a small portion of each work.'' \textit{Burroughs v.
Metro-Goldwyn-Mayer, Inc.}, 683 F.2d 610, 624 n. 14 (2d Cir.1982). Moreover,
this case involves the entire protected work and an iconographically, as well as
proportionately, significant portion of the allegedly infringing work.
\textit{Cf.} \textit{Mattel, Inc. v. Azrak-Hamway Intern., Inc.}, 724 F.2d 357,
360 (2d Cir.1983); \textit{Elsmere Music, Inc. v. National Broadcasting Co.},
482 F.Supp. 741, 744 (S.D.N.Y.), \textit{aff'd}, 623 F.2d 252 (2d Cir. 1980)
(taking small part of protected work can violate copyright).

The process by which defendants' poster was created also undermines this
argument. The ``map,'' that is, the portion about which plaintiff is
complaining, was designed separately from the rest of the poster. The likenesses
of the three main characters, which were copied from a photograph, and the
blocks of text were superimposed on the completed map. Nelson Deposition at
21-22; Nolan Deposition at 28.

I also reject defendants' argument that any similarities between the works are
unprotectible \textit{scenes a faire}, or ``incidents, characters or settings
which, as a practical matter, are indispensable or standard in the treatment of
a given topic.'' \textit{Walker}, 615 F.Supp. at 436. \textit{See also}
\textit{Reyher}, 533 F.2d at 92. It is undeniable that a drawing of New York
City blocks could be expected to include buildings, pedestrians, vehicles,
lampposts and water towers. Plaintiff, however, does not complain of defendants'
mere use of these elements in their poster; rather, his complaint is that
defendants  copied his \textit{expression} of those elements of a street scene.

While evidence of independent creation by the defendants would rebut plaintiff's
prima facie case, ``the absence of any countervailing evidence of creation
independent of the copyrighted source may well render clearly erroneous a
finding that there was not copying.'' \textit{Roth Greeting Cards v. United Card
Co.}, 429 F.2d 1106, 1110 (9th Cir.1970). \textit{See also} \textit{Novelty
Textile Mills}, 558 F.2d at 1092 n. 2.

Moreover, it is generally recognized that ``... since a very high degree of
similarity is required in order to dispense with proof of access, it must
logically follow that where proof of access is offered, the required degree of
similarity may be somewhat less than would be necessary in the absence of such
proof.'' 2 Nimmer \S~143.4 at 634, \textit{quoted in} \textit{Krofft}, 562 F.2d
at 1172. As defendants have conceded access to plaintiff's copyrighted
illustration, a somewhat lesser degree of similarity suffices to establish a
copyright infringement than might otherwise be required. Here, however, the
demonstrable similarities are such that proof of access, although in fact
conceded, is almost unnecessary.



%\readinghead{IV}
%
%I find meritless defendants' assertion that, to the extent that the ``Moscow''
%poster evokes Steinberg's, that evocation is justified under the parody branch
%of the ``fair use'' doctrine, codified at 17 U.S.C. \S~107. As this circuit has
%held, the copyrighted work must be ``at least in part an object of the parody,''
%\textit{MCA, Inc. v. Wilson}, 677 F.2d 180, 185 (2d Cir.1981). The record does
%not support a claim that defendants intended to satirize plaintiff's
%illustration; indeed, the deposition testimony of Columbia's executive art
%director tends to contradict such a claim. Moreover, an assertion that
%defendants consciously parodied the \textit{idea} of a parochial view of the
%world is immaterial: ideas are not protected by copyright, and the infringement
%alleged is of Steinberg's particular expression of that idea. Defendants'
%variation on the visual joke of plaintiff's illustration does not, without an
%element of humor aimed at some aspect of the illustration itself, render it a
%parody and therefore a fair use of plaintiff's work.
%
%In codifying the case law on determining whether one work constitutes a fair use
%of another, Congress instructed the courts to consider certain factors, the
%first of which is whether the intended use of the allegedly infringing work is
%``of a commercial nature or is for nonprofit educational purposes.'' 17 U.S.C.
%\S~107(1). As the Second Circuit said in a different artistic context, ``We are
%not prepared to hold that a commercial [artist] can plagiarize a ... copyrighted
%[work], substitute [certain elements] of his own, [produce] it for commercial
%gain, and then escape liability by calling the end result a parody or satire on
%the mores of society.'' \textit{MCA, Inc.}, 677 F.2d at 185.
%
%In analyzing the commercial or noncommercial nature of the ``Moscow'' poster, it
%is useful to distinguish between two conceptually different situations:
%advertising material that promotes a parody of a copyrighted work, and
%advertising material that itself infringes a copyright. In the first case, the
%fact that the advertisement uses elements of the copyrighted work does not
%necessarily mean that it infringes the copyright, if the product that it
%advertises constitutes a fair use of the copyrighted work. \textit{See, e.g.},
%\textit{Warner Bros.}, 720 F.2d at 242-44 (promotional broadcasts for television
%series legally parodying the Superman comic strip character did not infringe
%copyright in Superman character).
%
%In the second case, the work being advertised bears no relationship to the
%copyrighted work, but the advertisement itself infringes the copyright. In such
%a case, the owners of the copyright can prevent the advertisement from being
%used. As the Second Circuit has said, ``[n]o matter how well known a copyrighted
%phrase becomes, its author is entitled to guard against its appropriation to
%promote the sale of commercial products.'' \textit{Warner Bros.}, 720 F.2d at
%242. \textit{See, e.g.}, \textit{D.C. Comics, Inc. v. Crazy Eddie, Inc.}, 205
%U.S.P.Q. 1177 (S.D.N.Y.1979) (discount electronics chain not permitted to
%advertise its stores using parody of well-known lines associated with
%copyrighted Superman character).
%
%This situation fits the second case. Neither the ``Moscow'' movie nor the poster
%was designed to be a parody of the Steinberg illustration. The poster merely
%borrowed numerous elements from Steinberg to create an appealing advertisement
%to promote an unrelated commercial product, the movie. No parody of the
%illustration is involved, and defendants are not entitled to the protection of
%the parody branch of the fair use doctrine.
%
%The other factors mandated by 17 U.S.C. \S~107 do nothing to mitigate this
%determination. The copyrighted work at issue is an artistic creation, 17 U.S.C.
%\S~107(2), a very substantial portion of which was appropriated in the
%defendants' work, 17 U.S.C. \S~107(3). As for the value of the copyrighted work,
%17 U.S.C. \S~107(4), plaintiff submitted testimony to the court to show that his
%reputation was injured by having the public believe that he voluntarily lent his
%work to a profit-making enterprise.
%
%
%
%\readinghead{V}
%
%In their motion, defendants raised the affirmative defenses of estoppel and
%laches. Although Fed.R.Civ.P. 8(c) generally requires affirmative defenses to be
%pleaded, courts have been more lenient in the context of motions for summary
%judgment. ``[A]bsent prejudice to the plaintiff, a defendant may raise an
%affirmative defense in a motion for summary judgment for the first time.''
%\textit{Rivera v. Anaya}, 726 F.2d 564, 566 (9th Cir.1984). \textit{See} 2A, 6
%J. Moore, \textit{Moore's Federal Practice} ¶¶ 8.28, 56.02[2], 56.17[4] (2d ed.
%1986). It is therefore appropriate for this court to consider these defenses on
%the merits.
%
%Defendants base their assertions of these equitable defenses on the following
%factual claims: (1) plaintiff's alleged ``deliberate inaction'' for eight years
%in the face of numerous counterfeits of his poster and adaptations of his idea
%to various other localities; (2) plaintiff's alleged failure to act in response
%to the newspaper advertisements that appeared to promote ``Moscow''; and (3)
%defendants' assertion that Steinberg waited six months before even complaining
%to Columbia about their alleged infringement of his copyright on the poster,
%which defendants claim in their brief was a tactic on plaintiff's part to
%maximize the damages he hoped to receive.
%
%The record, however, does not support defendants' claims. First, Steinberg
%specifically requested that \textit{The New Yorker} magazine attempt to identify
%the sources of the counterfeit posters and prevent their continued distribution.
%As for the so-called adaptations of Steinberg's idea, there is no evidence that
%they infringed his copyright or that anyone ever believed that they did. As
%plaintiff freely and necessarily admits, the law does not protect an idea, but
%only the specific expression of that idea. The examples that defendants use to
%support their defense can at most be considered derivative of Steinberg's idea;
%none is a close copy of the poster itself, as defendants' is. Finally,
%defendants' last two assertions are rebutted by evidence that \textit{The New
%Yorker} protested to \textit{The New York Times} on plaintiff's behalf and at
%his request when ``Moscow'' opened, and that Columbia learned of this protest
%only a few weeks later.
%
%Moreover, even were defendants' factual assertions borne out by the record,
%their equitable defenses would have to be rejected because they have failed to
%establish the elements of either estoppel or laches.
%
%``A party seeking to invoke the doctrine of estoppel must plead and prove each
%of the essential elements: (1) a representation of fact ...; (2) rightful
%reliance thereon; and (3) injury or damage ... resulting from denial by the
%party making the representation.'' \textit{Galvez v. Local 804 Welfare Trust
%Fund}, 543 F.Supp. 316, 317 (E.D.N. Y.1982), \textit{citing} \textit{Haeberle v.
%Board of Trustees}, 624 F.2d 1132 (2d Cir.1980).
%
%Defendants have not established even the first of these elements. They argue
%that plaintiff's alleged silence  during the course of their advertisement
%campaign constitutes a sufficient representation of his acquiescence to meet the
%first requirement of the doctrine. As noted above, however, plaintiff did not
%remain silent, and the record shows that defendants, despite their awareness of
%his objections, continued to promote the film with the same advertisements and
%subsequently released a videocassette version of ``Moscow'' using the same
%promotional design. \textit{See} \textit{Lottie Joplin Thomas Trust v. Crown
%Publishers}, 592 F.2d 651, 655 (2d Cir.1978) (defense of estoppel falls where
%defendants fail to produce any evidence of detrimental reliance on plaintiff's
%alleged representations). Defendants overlook, moreover, that silence or
%inaction, in the absence of any duty or relationship between the parties, cannot
%give rise to an estoppel. \textit{Whiting Corp. v. Home Ins. Co.}, 516 F.Supp.
%643, 656 (S.D.N.Y.1981). \textit{Cf.} \textit{Columbia Broadcasting System, Inc.
%v. Stokely-Van Camp, Inc.}, 522 F.2d 369, 378 (2d Cir.1975). No such duty
%existed here.
%
%Defendants have likewise failed to establish the defense of laches. The party
%asserting laches must show that the opposing party ``did not assert her or their
%rights diligently, and that such asserted lack of diligence ... resulted in
%prejudice to them.'' \textit{Lottie Joplin}, 592 F.2d at 655, \textit{citing,
%inter alia}, \textit{Costello v. United States}, 365 U.S. 265, 282, 81 S.Ct.
%534, 543, 5 L.Ed.2d 551 (1961). In \textit{Lottie Joplin}, the Second Circuit
%held that a gap of approximately half a year between the publication of the
%allegedly infringing work and the institution of the lawsuit did not constitute
%a delay sufficient to establish a claim of laches. In this case, defendants were
%informed within weeks of plaintiff's disapproval of their poster; moreover, they
%have presented no evidence that, even if they had acknowledged any awareness of
%plaintiff's reaction, they would in any way have modified their subsequent
%actions. Consequently, they have failed to prove prejudice to themselves.
%
%
%
%\readinghead{VI}
%
%For the reasons set out above, summary judgment is granted to plaintiffs as to
%copying.
%
%A pretrial conference will be held on September 11, 1987, at 2 o'clock P.M., in
%Courtroom 35, to determine the proper measure and allocation of damages, other
%appropriate matters, and the parties' proposed schedule of further proceedings.
%The parties are to confer in advance of this conference, with the goal of
%reaching agreement on these matters, if possible.
%
