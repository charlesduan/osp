Whew! It's easy to get bogged down in the intricacies of the system of estates,
which is full of odd terminology, tricky mechanics, and a generally archaic
objective of family property management. It is necessary that you become
familiar with the terms and the mechanics. In part this is because it's on the
bar exam, and in part it is because the procedural rule-based thinking required
to work out estates problems is generally useful in the practice of law.

But there is something deeper at work here. Strip away the particulars and the
medieval phrasing from life estates and future interests, and a general
framework appears. For a single piece of land or other property, there can be a
someone with a current possessory interest, and others with future interests in
the property that are \emph{also property rights}---their interests can be
transferred to others, and the future interest holders have rights over the
property against the world. A key event occurs, such as the death of a
relevant person for a life estate. Upon that event occurring, a cascade of legal
consequences follow, which can change the nature of those property interests
automatically, perhaps even without the intervention of a court.

This pattern---a key event causes automatic conversion of property
rights---is what is meant by ``operation of law.'' It is the basic pattern for
the more exotic possessory estates and future interests that were not presented
in this chapter, and it is the pattern for far more of property law too. A
tenant who leases an apartment has a possessory estate, with the landlord
holding a future interest; the key events might include expiration of the
lease or nonpayment of rent. We will study other property arrangements that fit
this pattern, including mortgages, security interests, and joint tenancies. 

All this is to say that this material is no doubt some of the most difficult to
conceptualize. But if you can grasp the concepts here, that work will pay off as
you start to see similar patterns arise everywhere else.

