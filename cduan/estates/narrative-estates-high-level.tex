Whew! It's easy to get bogged down in the intricacies of the system of estates,
which is full of odd terminology, tricky mechanics, and a generally archaic
objective of family property management. It is necessary that you become
familiar with the terms and the mechanics. In part this is because it's on the
bar exam, and in part it is because the procedural rule-based thinking required
to work out estates problems is generally useful in the practice of law.

But there is something deeper at work here. Strip away the particulars and the
medieval phrasing from life estates and future interests, and a general
framework appears. For a single piece of land (or other item), there can be
someone with a current possessory interest, and others with future interests in
the land that are \emph{also property rights}---their interests can be
transferred to others, and the future interest holders have rights over the
land against the world. A key event occurs, such as the death of a
relevant person for a life estate. Upon that event occurring, a cascade of legal
consequences follows, which can change the nature of those property interests
automatically, perhaps even without the intervention of a court.

This pattern---a key event causes automatic conversion of property
rights---is what is meant by \term{operation of law}. It is the basic pattern
for
the more exotic possessory estates and future interests that were not presented
in this chapter, and it is the pattern for far more of property law too. A
tenant who leases an apartment has a possessory estate, with the landlord
holding a future interest; the key events might include expiration of the
lease or nonpayment of rent. We will study other property arrangements that fit
this pattern, including mortgages, security interests, and joint tenancies. 

There is a second observation worth mentioning, because it shows why property
law is fundamentally different from contracts. In contract law, parties are
generally free to agree to any arrangement of rights they see fit (within
reason, of course). This offers the parties tremendous flexibility, but the cost
is complexity if a dispute arises. After many events under the contract have
occurred, a court can only determine the parties' rights and responsibilities by
revisiting the original language of the contract, along with the history of all
relevant events. For a contract lasting a few years or decades, this is not a
problem. But over lifetimes and centuries---the scale of property law---that
information can easily be lost.

By contrast, look at how one determines legal rights in the system of estates.
When an event occurs, the only information needed to compute the next row in the
interest table is the previous row in the table. In fact, the original
conveyance language and all the prior rows can be forgotten or discarded. As
long as everyone knows what interests they held immediately before the event,
their ownership rights are fully determined, and the history of how they got to
those rights is irrelevant. Hopefully you can see the benefit of this odd
system---it is more resilient to information loss over long spans of time. More
importantly, though, it shows that you are learning an entirely new way of
managing rights and relationships over time, based on labels rather than textual
interpretation.

All this is to say that this material is no doubt some of the most difficult to
conceptualize. But if you can grasp the concepts here, that work will pay off as
you start to see similar patterns arise everywhere else.

