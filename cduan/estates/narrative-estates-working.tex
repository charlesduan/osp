To grasp how they system of estates works, \textbf{you must work out problems on
your own}. It is not enough to read the text passively here. (That's why you
have so few pages to read for this chapter.) You should do all the problems
given above, and also try to construct some hypothetical scenarios of your own.

The reason you need to work out problems on your own is that you need to develop
a method for notating the property interests involved in any given problem. I
will show you one way of keeping track of the interests; you are free to come up
with your own if it works better for you. Whatever you choose, though, must be
precise enough to track each interest by name, holder, and relationship with
other interests.

Consider, for example, a situation where O conveys Blackacre ``to A for life,
then to B.'' You might say that A holds a life estate and B has a remainder, but
that description has plenty of ambiguity. Whose life does A's life estate depend
upon? What does B's remainder follow? And when B's future interest converts to a
possessory estate, which one does it become? In this simple example the answers
might be obvious, but throw in a more complicated conveyance and several
property transfers, and the exact nature of the interests can easily become
lost.

A more complete description of the interests is:
\begin{itemize}
\item A has a life estate in Blackacre for the life of A.
\item B has a remainder, following the life estate in Blackacre for
the life of A, which will become fee simple absolute in Blackacre.
\end{itemize}
This is complete and unambiguous, but also pretty wordy for purposes of
notetaking and working out problems. We can omit ``Blackacre'' given that only
one plot of land is involved (but don't omit it if there are more!). We'll give
the interests identifiers (interest \#1, interest \#2) to make them easier to
talk about. Let's also introduce some abbreviations.
\begin{center}
\begin{tabular}{ll}
FSA & Fee simple absolute \\
LE(P) & Life estate for the life of person P \\
Rem(\#I, E) & Remainder following interest \#I, \\
 & which will become estate E \\
Rev(\#I, E) & Reversion following interest \#I, \\
 & which will become estate E \\
\end{tabular}
\end{center}

Now we can fully characterize the interests in Blackacre as follows:
\begin{itemize}
\item A has interest \#1: LE(A)
\item B has interest \#2: Rem(\#1, FSA)
\end{itemize}
Even better, let's put it in tabular form:
\begin{center}
\begin{tabular}{l|ll}
Event & A & B \\
\hline
Grant from O & \#1: LE(A) & \#2: Rem(\#1, FSA) \\
\end{tabular}
\end{center}

At this point, the table and abbreviations may seem unnecessarily cryptic.
Again, feel free to choose abbreviations that work best for you. But the value
of this structure comes when the interests start moving around. Say that A
decides to move to Hawaii, and gifts Blackacre to her sister C. What does C
have? All we need to do is to copy A's property interest over to a new column
for C:
\begin{center}
\begin{tabular}{l|lll}
Event & A & B & C \\
\hline
Grant from O & \#1: LE(A) & \#2: Rem(\#1, FSA) & \\
A to C & & \#2: Rem(\#1, FSA) & \#1: LE(A) \\
\end{tabular}
\end{center}
The notation makes clear that C's interest is based on A's life, not C's; that
is, C has a life estate pur autre vie for the life of A. If C dies, leaving all
her property to D:
\begin{center}
\begin{tabular}{l|llll}
Event & A & B & C & D \\
\hline
Grant from O & \#1: LE(A) & \#2: Rem(\#1, FSA) & &\\
A to C & & \#2: Rem(\#1, FSA) & \#1: LE(A) & \\
C dies & & \#2: Rem(\#1, FSA) & & \#1: LE(A) \\
\end{tabular}
\end{center}
The table thus makes clear that C's death only causes a transfer of the life
estate, but does not change the interest or any other interests.

What happens when A dies? We just follow two rules:
\begin{itemize}
\item When person P dies, LE(P) terminates.
\item When interest \#I terminates, then Rem(\#I, E) or Rev(\#I, E) turns into
E.
\end{itemize}
Applied to the table, that means that when A dies, the life estate disappears,
and the remainder converts as follows:
\begin{center}
\begin{tabular}{l|llll}
Event & A & B & C & D \\
\hline
Grant from O & \#1: LE(A) & \#2: Rem(\#1, FSA) & &\\
A to C & & \#2: Rem(\#1, FSA) & \#1: LE(A) & \\
C dies & & \#2: Rem(\#1, FSA) & & \#1: LE(A) \\
A dies & & \#2: FSA & & [terminated] \\
\end{tabular}
\end{center}

Let's now work out a more complex problem, the third question in note
\ref{estates-q1}. (Try to work it out yourself first.) Initially, we need to
translate the conveyance ``O to A for life, then to B for life'' into our
notation. A has a life estate for the life of A, which we will call \#1: LE(A).
B's interest follows A's, so it is a remainder that will become a life estate
for the life of B, which we notate \#2: Rem(\#1, LE(B)). This gives the
following table:
\begin{center}
\begin{tabular}{l|ll}
Event & A & B \\
\hline
Grant from O & \#1: LE(A) & \#2: Rem(\#1, LE(B)) \\
\end{tabular}
\end{center}
Are we done? A good check, which works for any row of these tables, is to start
with the possessory estate and follow the chain of interests. The last one must
be, or be convertible to, fee simple absolute---otherwise the property might
have no one entitled to possess it at some point. Here, we start with A's
interest \#1, which is followed by B's interest \#2 that can convert into a life
estate, and then what? Since the grant from O specifies nothing else, we infer a
reversion following the last interest in the chain:
\begin{center}
\begin{tabular}{l|lll}
Event & A & B & O \\
\hline
Grant from O & \#1: LE(A) & \#2: Rem(\#1, LE(B)) & \#3: Rev(\#2, FSA) \\
\end{tabular}
\end{center}
Now there is a complete chain of future interests, ending with someone receiving
fee simple absolute ownership.

Next in the problem, O dies. None of the life estates are based on O's life, so
no interests need to be converted. But O does own something, because O has an
entry in the table. That entry needs to go to someone else now, since O, being
dead, can't own property. We'll call this recipient of O's property ``O's
heir'':
\begin{center}
\begin{tabular}{l|llll}
Event & A & B & O & O's heir \\
\hline
Grant & \#1: LE(A) & \#2: Rem(\#1, LE(B)) & \#3: Rev(\#2, FSA) & \\
O dies & \#1: LE(A) & \#2: Rem(\#1, LE(B)) & & \#3: Rev(\#2, FSA) \\
\end{tabular}
\end{center}
(Challenge question: What happens if A is O's heir?)

Next, the problem says that A dies. There is a life estate for the life of A, so
we apply the conversion rules to interests \#1 and \#2:
\begin{center}
\begin{tabular}{l|llll}
Event & A & B & O & O's heir \\
\hline
Grant & \#1: LE(A) & \#2: Rem(\#1, LE(B)) & \#3: Rev(\#2, FSA) & \\
O dies & \#1: LE(A) & \#2: Rem(\#1, LE(B)) & & \#3: Rev(\#2, FSA) \\
A dies & [term.] & \#2: LE(B) & & \#3: Rev(\#2, FSA) \\
\end{tabular}
\end{center}
Finally, B dies. Again, since a life estate depends on B's life, we apply the
conversion rules, now to interests \#2 and \#3:
\begin{center}
\begin{tabular}{l|llll}
Event & A & B & O & O's heir \\
\hline
Grant & \#1: LE(A) & \#2: Rem(\#1, LE(B)) & \#3: Rev(\#2, FSA) & \\
O dies & \#1: LE(A) & \#2: Rem(\#1, LE(B)) & & \#3: Rev(\#2, FSA) \\
A dies & [term.] & \#2: LE(B) & & \#3: Rev(\#2, FSA) \\
B dies & & [term.] & & FSA \\
\end{tabular}
\end{center}
The result, as the table makes clear, is that O's heir takes Blackacre in fee
simple absolute.

