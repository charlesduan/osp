\expected{campbell-v-acuff-rose}

\item The Supreme Court's original opinion contained an appendix reprinting the
lyrics of both the original Roy Orbison song and 2 Live Crew's version. They are
omitted from the text to save space, but are worth a look if you're interested.

Was it copyright infringement for the Court to reprint the lyrics? Try applying
the fair use doctrine.


\item As this case makes clear, copyright protection can exclude not just exact
copies of a work, but also ``derivative works'' like translations, movie
adaptations, summaries, or sequels. That copyright protection extends beyond
exact copying ought to be intuitive: Others should not be able to get around a
copyright just by changing a few words or paint strokes. But it presents a
tremendous boundary problem for copyright. How does one know where
infringement ends and permissible uses begin? What's the difference between
plagiarism and research?

The answer to the copyright boundary question is a complex mix of statutory and
case law, which is beyond the scope of a survey text on property. But what does
the vagueness of copyright boundaries tell you about copyright as a species of
property? Are property boundaries similarly vague for other types of property?
Should uncertainty about how far any given copyright reaches affect the right to
exclude?


\defcase{sony-v-universal}{
parties={Sony Corp. of America v. Universal City Studios, Inc.},
cite=464 U.S. 417,
year=1984,
}

\defcase{authors-guild-v-hathitrust}{
parties={Authors Guild, Inc. v. HathiTrust},
cite=755 F.3d 87,
court=2d Cir.,
year=2014,
}

\defcase{google-v-oracle}{
parties={Google LLC v. Oracle America, Inc.},
cite=141 S. Ct. 1183,
year=2021,
}


\defcase{av-v-iparadigms}{
parties={A.V. ex rel. Vanderhye v. iParadigms, LLC},
cite=562 F.3d 630,
court=4th Cir.,
year=2009,
}

\item The fair use doctrine has been used in a wide range of seemingly unrelated
situations. Consider the following activities that courts have considered fair
use:
\begin{itemize}
\item Recording a television show to videocassette, in order to watch it later.
\sentence{sony-v-universal}.

\item Libraries scanning and digitizing books for full-text searching and
accessibility for print-disabled patrons. \sentence{authors-guild-v-hathitrust}.

\item Replicating key parts of a copyright-protected package of computer
software, in order to make it easier for third-party programmers to switch from
one software package to the other one. \sentence{google-v-oracle}.

\item Collecting student essays to build a plagiarism detection system.
\sentence{av-v-iparadigms}.
\end{itemize}


\item \having{ebay-v-mercexchange-qs}{The notes to \emph{eBay} mentioned the
possibility of a reasonable royalty payment as an alternative to injunctive
relief.}{\emph{eBay v.~MercExchange}, discussed later, notes that in patent
law, a patent holder can receive payment of a ``reasonable royalty'' as an
alternative to injunctive relief.}{In patent law, if a patent holder is denied
injunctive relief, it is still possible for a court to award a ``reasonable
royalty'' payment.} If a copyright holder cannot exclude another's use due to
the fair use doctrine, should the copyright holder receive a reasonable royalty
or other compensation for the use?




\defcase{golan-v-holder}{
parties=Golan v. Holder,
cite=565 U.S. 302,
year=2012,
}

\item Copyright law's right to exclude proscribes speech. Why doesn't copyright
run afoul of the First Amendment? In \inline{golan-v-holder}, the Supreme Court
observed that the fair use doctrine helps to resolve the tension between
copyright law and freedom of speech. \sentence{see golan-v-holder}.
\having{marsh-v-alabama}{Similarly, in \emph{Marsh}, we saw how the right to
exclude from real property can conflict with the First Amendment.}{\emph{Marsh
v.~Alabama}, presented later in this textbook, offers a similar conflict between
the First Amendment and real property.}{The First Amendment can also conflict
with other property rights to exclude. Can you think of how that might happen?}
What other constitutional rights might come into conflict with intellectual
property rights, or property rights generally?

\item In several of the footnotes, Justice Souter carefully distinguishes
parody, which (roughly) mocks the original work, from satire, which (again
roughly) uses a spin on the original work to make other commentary. What do you
think of this distinction? Should satire be fair use? Should it be within the
scope of a copyright holder's right to exclude?


\defcase{dr-seuss-v-comicmix}{
parties={Dr. Seuss Enterprises, LP v. ComicMix LLC},
cite=983 F.3d 443,
court=9th Cir.,
year=2020,
}

Consider, in particular, the mashup book \emph{Oh, the Places You'll Boldly
Go!\@}, which cast the classic Dr.~Seuss picture book \emph{Oh, the Places
You'll Go!\@} in combination with elements from the television show \emph{Star
Trek}. The Ninth Circuit appellate court held the mashup not to be fair use.
\sentence{see dr-seuss-v-comicmix}. Do you agree? What justifications or
theories of property ownership explain the estate of Dr.~Seuss wielding veto
power over mashups? Are there countervailing policy concerns?

