\item While \emph{Shack} presents a limitation on the right to exclude by
judicial analysis, legislatures can also limit the right to exclude by statute.
The Civil Rights Act of 1964, for example, provides:
\begin{quote}
All persons shall be entitled to the full and equal enjoyment of the goods,
services, facilities, privileges, advantages, and accommodations of any place
of public accommodation, as defined in this section, without discrimination or
segregation on the ground of race, color, religion, or national origin.
\end{quote}
The term ``place of public accommodation'' includes hotels, restaurants,
theaters, and entertainment venues, among other places, but excludes any
``private club or other establishment not in fact open to the public.'' 42
U.S.C. \S~2000a.

Similarly, the Americans with Disabilities Act of 1990 prohibits discrimination
``on the basis of disability in the full and equal enjoyment of the goods,
services, facilities, privileges, advantages, or accommodations of any place of
public accommodation.'' In addition to this restriction on the right to exclude,
the statute requires covered property owners to take affirmative steps to make
their facilities accessible. 42 U.S.C. \S\S~12182--12183.

One way of thinking of these statutes is that they are interventions to the
property right to exclude, serving different normative values like equality and
nondiscrimination that otherwise conflict with property ownership. Is that
conflict necessary? Could you imagine a concept of ``property'' that
incorporates these values?

