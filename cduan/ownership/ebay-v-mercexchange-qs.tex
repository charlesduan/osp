\expected{ebay-v-mercexchange}

\having{oil-states-v-greenes-energy}{%
\item The author of \emph{eBay}, Justice Thomas, was also the author of
\emph{Oil States}. Justice Thomas has written a substantial portion of the
Supreme Court's recent patent jurisprudence.%
}{}{}

\expected{jacque-v-steenberg}
\expected{marsh-v-alabama}
\expected{state-v-shack}
\item As the opinion observes, the four-factor test of \emph{eBay} is not
specific to patents, but is the general test for whether a court should issue
injunctive relief. How would it apply in \emph{Jacque}, \emph{Marsh}, and
\emph{Shack}? Would it be consistent with the results in those cases?

\item Why do you think the Court of Appeals adopted a rule rule that injunctions
in patent cases are largely automatic? In answering this question, consider the
justifications for having property rights (first possession, natural rights,
economic incentives, etc.). How do these justifications inform the differing
views between the Court of Appeals and the Supreme Court?

\item Justice Kennedy's concurrence responds to Chief Justice Roberts, arguing
that the availability of injunctions should depend on how the patent holder
chooses to exploit the patent, and possibly also on what kind of technology the
patent covers. Should a property owner's right to exclude depend on the nature
of the property or what the property owner does with it? Can you think of a
situation where a landowner should have less of a right to prosecute trespass
because of what the landowner is doing?

\item By some counts, there are between 250,000 and 350,000 patents on a
smartphone. What are the bargaining positions of each of the holders of those
thousands of patents, versus the companies who make smartphones? What happens if
every one of those patent holders can obtain an automatic injunction? Does
\emph{eBay} help here? For ideas, look up the concepts of ``patent holdup'' and
``royalty stacking,'' which are described among other places in the Federal
Trade Commission report cited in the Supreme Court's opinion.

Is this a problem specific to patents? Can you think of a way in which someone
trespasses on thousands of other people's land at the same time, or
misappropriates thousands of personal possessions at once? How should injunctive
relief work in these situations?

\item When a court declines to grant an injunction against patent infringement
under \emph{eBay}, it typically will instead award a ``reasonable royalty,''
namely a payment from the infringer to the patent holder that generally comes to
a percentage of the infringer's profits or revenues. How do these alternative
remedies affect the patent holder's negotiating position before litigation
begins? Consider that the reasonable royalty will only be awarded after both of
the parties have expended large sums of money for litigation.

If one can trespass upon another's property at the cost of a reasonable royalty,
is it still ``property''?

