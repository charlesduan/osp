\reading{eBay Inc. v. MercExchange, LLC} \readingcite{547 U.S. 388 (2006)}


\opinion \textsc{Justice Thomas} delivered the opinion of the Court.

Ordinarily, a federal court considering whether to award permanent injunctive
relief to a prevailing plaintiff applies the four-factor test historically
employed by courts of equity. Petitioners eBay Inc. and Half.com, Inc., argue
that this traditional test applies to disputes arising under the Patent Act. We
agree and, accordingly, vacate the judgment of the Court of Appeals.



\readinghead{I}

Petitioner eBay operates a popular Internet Web site that allows private sellers
to list goods they wish to sell, either through an auction or at a fixed price.
Petitioner Half.com, now a wholly owned subsidiary of eBay, operates a similar
Web site. Respondent MercExchange, L. L. C., holds a number of patents,
including a business method patent for an electronic market designed to
facilitate the sale of goods between private individuals by establishing a
central authority to promote trust among participants. See U. S. Patent No.
5,845,265. MercExchange sought to license its patent to eBay and Half.com, as it
had previously done with other companies, but the parties failed to reach an
agreement. MercExchange subsequently filed a patent infringement suit against
eBay and Half.com in the United States District Court for the Eastern District
of Virginia. A jury found that MercExchange's patent was valid, that eBay and
Half.com had infringed that patent, and that an award of damages was
appropriate.

Following the jury verdict, the District Court denied MercExchange's motion for
permanent injunctive relief. The Court of Appeals
for the Federal Circuit reversed, applying its general rule that courts will
issue permanent injunctions against patent infringement absent exceptional
circumstances. We granted certiorari to determine
the appropriateness of this general rule.


\readinghead{II}

According to well-established principles of equity, a plaintiff seeking a
permanent injunction must satisfy a four-factor test before a court may grant
such relief. A plaintiff must demonstrate: (1) that it has suffered an
irreparable injury; (2) that remedies available at law, such as monetary
damages, are inadequate to compensate for that injury; (3) that, considering the
balance of hardships between the plaintiff and defendant, a remedy in equity is
warranted; and (4) that the public interest would not be disserved by a
permanent injunction. The decision to grant or
deny permanent injunctive relief is an act of equitable discretion by the
district court, reviewable on appeal for abuse of discretion.

These familiar principles apply with equal force to disputes arising under the
Patent Act. As this Court has long recognized, a major departure from the long
tradition of equity practice should not be lightly implied.
Nothing in the Patent Act indicates that
Congress intended such a departure. To the contrary, the Patent Act expressly
provides that injunctions ``may'' issue ``in accordance with the principles of
equity.'' 35 U. S. C. \S~283.

To be sure, the Patent Act also declares that ``patents shall have the
attributes of personal property,'' \S~261, including ``the right to
exclude others from making, using, offering for sale, or selling the
invention,'' \S~154(a)(1). According to the Court of Appeals, this
statutory right to exclude alone justifies its general rule in favor of
permanent injunctive relief. But the creation of a right is
distinct from the provision of remedies for violations of that right. Indeed,
the Patent Act itself indicates that patents shall have the attributes of
personal property ``[s]ubject to the provisions of this title,'' 35 U. S. C.
\S~261, including, presumably, the provision that injunctive relief
``may'' issue only ``in accordance with the principles of equity,''
\S~283.

This approach is consistent with our treatment of injunctions under the
Copyright Act. Like a patent owner, a copyright holder possesses ``the right to
exclude others from using his property.'' \textit{Fox Film Corp.} v.
\textit{Doyal}, 286 U. S. 123, 127 (1932).
Like the Patent Act, the Copyright Act provides that
courts ``may'' grant injunctive relief ``on such terms as it may deem reasonable
to prevent or restrain infringement of a copyright.'' 17 U. S. C. \S~502(a). And
as in our decision today, this Court has consistently rejected
invitations to replace traditional equitable considerations with a rule that an
injunction automatically follows a determination that a copyright has been
infringed. See, \textit{e. g.}, \textit{New York Times Co.} v. \textit{Tasini},
533 U. S. 483, 505 (2001) (citing \textit{Campbell} v. \textit{Acuff-Rose Music,
Inc.}, 510 U. S. 569, 578, n. 10 (1994)); \textit{Dun} v. \textit{Lumbermen's
Credit Assn.}, 209 U. S. 20, 23-24 (1908).

Neither the District Court nor the Court of Appeals below fairly applied these
traditional equitable principles in deciding respondent's motion for a permanent
injunction. Although the District Court recited the traditional four-factor
test, it appeared to adopt certain expansive principles
suggesting that injunctive relief could not issue in a broad swath of cases.
Most notably, it concluded that a ``plaintiff's willingness to license its
patents'' and ``its lack of commercial activity in practicing the patents''
would be sufficient to establish that the patent holder would not suffer
irreparable harm if an injunction did not issue. But
traditional equitable principles do not permit such broad classifications. For
example, some patent holders, such as university researchers or self-made
inventors, might reasonably prefer to license their patents, rather than
undertake efforts to secure the financing necessary to bring their works to
market themselves. Such patent holders may be able to satisfy the traditional
four-factor test, and we see no basis for categorically denying them the
opportunity to do so.\ldots

In reversing the District Court, the Court of Appeals departed in the opposite
direction from the four-factor test. The court articulated a ``general rule,''
unique to patent disputes, ``that a permanent injunction will issue once
infringement and validity have been adjudged.'' The court
further indicated that injunctions should be denied only in the ``unusual''
case, under ``exceptional circumstances'' and ``in rare instances\ldots to
protect the public interest.'' Just as the District
Court erred in its categorical denial of injunctive relief, the Court of Appeals
erred in its categorical grant of such relief.

\ldots We take no position on whether
permanent injunctive relief should or should not issue in this particular case,
or indeed in any number of other disputes arising under the Patent Act. We hold
only that the decision whether to grant or deny injunctive relief rests within
the equitable discretion of the district courts, and that such discretion must
be exercised consistent with traditional principles of equity, in patent
disputes no less than in other cases governed by such standards.

Accordingly, we vacate the judgment of the Court of Appeals and remand the case
for further proceedings consistent with this opinion.

\textit{It is so ordered.}

\opinion \textsc{Chief Justice Roberts}, with whom \textsc{Justice Scalia} and
\textsc{Justice Ginsburg} join, concurring.

\ldots.

From at least the early 19th century, courts have granted injunctive relief upon
a finding of infringement in the vast majority of patent cases. This ``long
tradition of equity practice'' is not surprising, given the difficulty of
protecting a right to \textit{exclude} through monetary remedies that allow an
infringer to \textit{use} an invention against the patentee's wishes---a
difficulty that often implicates the first two factors of the traditional
four-factor test. This historical practice, as the Court holds, does not
\textit{entitle} a patentee to a permanent injunction or justify a
\textit{general rule} that such injunctions should issue.
At the same time, there is a
difference between exercising equitable discretion pursuant to the established
four-factor test and writing on an entirely clean slate. Discretion is not
whim, and limiting discretion according to legal standards helps promote the
basic principle of justice that like cases should be decided alike.\ldots



\opinion \textsc{Justice Kennedy}, with whom \textsc{Justice Stevens},
\textsc{Justice Souter}, and \textsc{Justice Breyer} join, concurring.

\ldots. To the extent
earlier cases establish a pattern of granting an injunction against patent
infringers almost as a matter of course, this pattern simply illustrates the
result of the four-factor test in the contexts then prevalent. The lesson of the
historical practice, therefore, is most helpful and instructive when the
circumstances of a case bear substantial parallels to litigation the courts have
confronted before.

In cases now arising trial courts should bear in mind that in many instances the
nature of the patent being enforced and the economic function of the patent
holder present considerations quite unlike earlier cases. An industry has
developed in which firms use patents not as a basis for producing and selling
goods but, instead, primarily for obtaining licensing fees. See FTC, \emph{To
Promote Innovation: The Proper Balance of Competition and Patent Law and
Policy}, ch. 3, pp. 38-39 (Oct. 2003).
For these firms, an injunction, and the potentially serious
sanctions arising from its violation, can be employed as a bargaining tool to
charge exorbitant fees to companies that seek to buy licenses to practice the
patent. When the patented invention is but a small component
of the product the companies seek to produce and the threat of an injunction is
employed simply for undue leverage in negotiations, legal damages may well be
sufficient to compensate for the infringement and an injunction may not serve
the public interest. In addition injunctive relief may have different
consequences for the burgeoning number of patents over business methods, which
were not of much economic and legal significance in earlier times. The potential
vagueness and suspect validity of some of these patents may affect the calculus
under the four-factor test.

The equitable discretion over injunctions, granted by the Patent Act, is well
suited to allow courts to adapt to the rapid technological and legal
developments in the patent system. For these reasons it should be recognized
that district courts must determine whether past practice fits the circumstances
of the cases before them. With these observations, I join the opinion of the
Court.

