\reading{Campbell v. Acuff-Rose Music, Inc.} \readingcite{510 U.S. 569 (1994)}

\opinion \textsc{Justice Souter} delivered the opinion of the Court.

We are called upon to decide whether 2 Live Crew's commercial parody of Roy
Orbison's song, ``Oh, Pretty Woman,'' may be a fair use within the meaning of
the Copyright Act of 1976, 17 U. S. C. \S~107\ldots.

\readinghead{I}

In 1964, Roy Orbison and William Dees wrote a rock ballad called ``Oh, Pretty
Woman'' and assigned their rights in it to respondent Acuff-Rose Music, Inc. See
Appendix A, \textit{infra}, at 594. Acuff-Rose registered the song for copyright
protection.

Petitioners Luther R. Campbell, Christopher Wongwon, Mark Ross, and David Hobbs
are collectively known as 2 Live Crew, a popular rap music
group. In 1989, Campbell wrote a song
entitled ``Pretty Woman,'' which he later described in an affidavit as intended,
``through comical lyrics, to satirize the original work\ldots.''
[Negotiations between 2 Live Crew and Acuff-Rose failed.]

Almost a year later, after nearly a quarter of a million copies of the recording
had been sold, Acuff-Rose sued 2 Live Crew and its record company, Luke
Skyywalker Records, for copyright infringement.\ldots

\readinghead{II}

It is uncontested here that 2 Live Crew's song would be an infringement of
Acuff-Rose's rights in ``Oh, Pretty Woman,'' under the Copyright Act of 1976,
but for a finding of fair use through parody. From the infancy of copyright
protection, some opportunity for fair use of copyrighted materials has been
thought necessary to fulfill copyright's very purpose, ``[t]o promote the
Progress of Science and useful Arts\ldots.'' U. S. Const., Art. I, \S~8, cl.
8.\readingfootnote{6}{The exclusion of facts and ideas from copyright protection
serves
that goal as well. See \textit{Feist Publications, Inc.} v. \textit{Rural
Telephone Service Co.}, 499 U. S. 340, 359 (1991)\ldots.}
For as Justice Story explained, ``[i]n truth, in literature, in science
and in art, there are, and can be, few, if any, things, which in an abstract
sense, are strictly new and original throughout. Every book in literature,
science and art, borrows, and must necessarily borrow, and use much which was
well known and used before.'' \textit{Emerson} v. \textit{Davies}, 8 F. Cas.
615, 619 (No. 4,436) (CCD Mass. 1845).\ldots

In \textit{Folsom} v. \textit{Marsh}, 9 F. Cas. 342 (No. 4,901) (CCD Mass.
1841), Justice Story distilled the essence of law and methodology from the
earlier cases: ``look to the nature and objects of the selections made, the
quantity and value of the materials used, and the degree in which the use may
prejudice the sale, or diminish the profits, or supersede the objects, of the
original work.'' Thus expressed, fair use remained
exclusively judge-made doctrine until the passage of the 1976 Copyright Act, in
which Justice Story's summary is discernible:
\begin{quotation}
\noindent \S~107. Limitations on exclusive rights: Fair use

Notwithstanding the provisions of sections 106 and 106A, the
fair use of a copyrighted work, including such use by reproduction in copies or
phonorecords or by any other means specified by that section, for purposes such
as criticism, comment, news reporting, teaching (including multiple copies for
classroom use), scholarship, or research, is not an infringement of copyright.
In determining whether the use made of a work in any particular case is a fair
use the factors to be considered shall include---
\begin{statute}
\item (1) the purpose and character of the use, including whether such use is of
a commercial nature or is for non-profit educational purposes;
\item (2) the nature of the copyrighted work;
\item (3) the amount and substantiality of the portion used in
relation to the copyrighted work as a whole; and
\item (4) the effect of the use upon the potential market for or value of the
copyrighted
work.
\end{statute}
The fact that a work is unpublished shall not
itself bar a finding of fair use if such finding is made upon consideration of
all the above factors.
\end{quotation}
17 U. S. C. \S~107.

Congress meant \S~107 ``to restate the present judicial doctrine of fair use,
not to change, narrow, or enlarge it in any way'' and intended that courts
continue the common-law tradition of fair use adjudication. The fair use
doctrine thus permits and requires courts to avoid rigid application of the
copyright statute when, on occasion, it would stifle the very creativity which
that law is designed to foster.

The task is not to be simplified with bright-line rules, for the statute, like
the doctrine it recognizes, calls for case-by-case
analysis.\ldots\unskip\readingfootnote{11}{Because the fair use enquiry often
requires close questions of judgment as to the extent of permissible borrowing
in cases involving parodies (or other critical works), courts may also wish to
bear in mind that the goals of the copyright law, to stimulate the creation and
publication of edifying matter, are not always best served by automatically
granting injunctive relief when parodists are found to have gone beyond the
bounds of fair use.}

\readinghead{A}

The first factor in a fair use enquiry is ``the purpose and character of the
use, including whether such use is of a commercial nature or is for nonprofit
educational purposes.'' The central purpose of
this investigation is to see whether the new work
merely supersedes the objects of the original creation,
or instead adds something new, with a further
purpose or different character, altering the first with new expression, meaning,
or message; it asks, in other words, whether and to what extent the new work is
``transformative.'' Leval, \emph{Toward a Fair Use Standard}, 103 Harv. L. Rev.
1105, 1111 (1990). Such works lie
at the heart of the fair use doctrine's guarantee of breathing space within the
confines of copyright, and the more transformative the new work, the less
will be the significance of other factors, like commercialism, that may weigh
against a finding of fair use.\ldots

The germ of parody lies in the definition of the Greek \textit{parodeia}, quoted
in Judge Nelson's Court of Appeals dissent, as ``a song sung alongside
another.''
Modern dictionaries accordingly describe a parody as a ``literary or
artistic work that imitates the characteristic style of an author or a work for
comic effect or ridicule,'' or as a ``composition in prose or verse in which the
characteristic
turns of thought and phrase in an author or class of authors are imitated in
such a way as to make them appear ridiculous.'' For the purposes of copyright
law, the nub of the
definitions, and the heart of any parodist's claim to quote from existing
material, is the use of some elements of a prior author's composition to create
a new one that, at least in part, comments on that author's works.
If, on the contrary,
the commentary has no critical bearing on the substance or style of the original
composition, which the alleged infringer merely uses to get attention or to
avoid the drudgery in working up something fresh, the claim to fairness in
borrowing from another's work diminishes accordingly (if it does not vanish),
and other factors, like the extent of its commerciality, loom
larger. Parody needs
to mimic an original to make its point, and so has some claim to use the
creation of its victim's (or collective victims') imagination, whereas satire
can stand on its own two feet and so requires justification for the very act of
borrowing.\readingfootnote{16}{Satire has been defined as a work ``in which
prevalent follies or vices are assailed with ridicule,''
or are ``attacked through irony, derision, or wit.''}

The fact that parody can claim legitimacy for some appropriation does not, of
course, tell either parodist or judge much about where to draw the line. Like a
book review quoting the copyrighted material criticized, parody may or may not
be fair use, and petitioners' suggestion that any parodic use is presumptively
fair has no more justification in law or fact than the equally hopeful claim
that any use for news reporting should be presumed fair,
The Act has no hint of an evidentiary preference for
parodists over their victims, and no workable presumption for parody could take
account of the fact that parody often shades into satire when society is
lampooned through its creative artifacts, or that a work may contain both
parodic and nonparodic elements. Accordingly, parody, like any other use, has to
work its way through the relevant factors, and be judged case by case, in light
of the ends of the copyright law.

\ldots. The threshold question when fair use is
raised in defense of parody is whether a parodic character may reasonably be
perceived. Whether, going beyond that, parody
is in good taste or bad does not and should not matter to fair use. As Justice
Holmes explained, ``[i]t would be a dangerous undertaking for persons trained
only to the law to constitute themselves final judges of the worth of [a work],
outside of the narrowest and most obvious limits. At the one extreme some works
of genius would be sure to miss appreciation. Their very novelty would make them
repulsive until the public had learned the new language in which their author
spoke.'' \textit{Bleistein} v. \textit{Donaldson Lithographing Co.}, 188 U. S.
239, 251 (1903)\ldots.

While we might not assign a high rank to the parodic element here, we think it
fair to say that 2 Live Crew's song reasonably could be perceived as commenting
on the original or criticizing it, to some degree. 2 Live Crew juxtaposes the
romantic musings of a man whose fantasy comes true, with degrading taunts, a
bawdy demand for sex, and a sigh of relief from paternal responsibility. The
later words can be taken as a comment on the naivete of the original of an
earlier day, as a rejection of its sentiment that ignores the ugliness of street
life and the debasement that it signifies. It is this joinder of reference and
ridicule that marks off the author's choice of parody from the other types of
comment and criticism that traditionally have had a claim to fair use protection
as transformative works.

The Court of Appeals, however, immediately cut short the enquiry into 2 Live
Crew's fair use claim by confining its treatment of the first factor essentially
to one relevant fact, the commercial nature of the use.\ldots
[But] the language of the statute makes clear that the commercial or nonprofit
educational purpose of a work is only one element of the first factor enquiry
into its purpose and character.\ldots
The mere fact that a use is educational and not for profit does
not insulate it from a finding of infringement, any more than the commercial
character of a use bars a finding of fairness. If, indeed, commerciality carried
presumptive force against a finding of fairness, the presumption would swallow
nearly all of the illustrative uses listed in the preamble paragraph of \S~107,
including news reporting, comment, criticism, teaching, scholarship, and
research, since these activities ``are generally conducted for profit in this
country.''
Congress could not have intended such a rule\ldots.

\readinghead{B}

The second statutory factor, ``the nature of the copyrighted work,''\ldots
calls for
recognition that some works are closer to the core of intended copyright
protection than others, with the consequence that fair use is more difficult to
establish when the former works are copied. We agree with both the
District Court and the Court of Appeals that the Orbison original's creative
expression for public dissemination falls within the core of the copyright's
protective purposes. This fact,
however, is not much help in this case, or ever likely to help much in
separating the fair use sheep from the infringing goats in a parody case, since
parodies almost invariably copy publicly known, expressive works.


\readinghead{C}

The third factor asks whether ``the amount and substantiality of the portion
used in relation to the copyrighted work as a whole.''\ldots
Here, attention turns to the persuasiveness of a parodist's
justification for the particular copying done, and the enquiry will harken back
to the first of the statutory factors, for, as in prior cases, we recognize that
the extent of permissible copying varies with the purpose and character of the
use.\ldots

Parody presents a
difficult case. Parody's humor, or in any event its comment, necessarily springs
from recognizable allusion to its object through distorted imitation. Its art
lies in the tension between a known original and its parodic twin. When parody
takes aim at a particular original work, the parody must be able to ``conjure
up'' at least enough of that original to make the object of its critical wit
recognizable. What makes for this
recognition is quotation of the original's most distinctive or memorable
features, which the parodist can be sure the audience will know. Once enough has
been taken to assure identification, how much more is reasonable will depend,
say, on the extent to which the song's overriding purpose and character is to
parody the original or, in contrast, the likelihood that the parody may serve as
a market substitute for the original. But using some characteristic features
cannot be avoided.

\ldots.

Suffice it to say here that, as to the lyrics, we think the Court of Appeals
correctly suggested that no more was taken than necessary,
but just for that reason, we fail to see how the copying can be excessive
in relation to its parodic purpose, even if the portion taken is the original's
heart. As to the music, we express no opinion whether repetition of the
[original song's] bass
riff is excessive copying, and we remand to permit evaluation of the amount
taken, in light of the song's parodic purpose and character, its transformative
elements, and considerations of the potential for market substitution sketched
more fully below.



\readinghead{D}

The fourth fair use factor is the effect of the use upon the potential market
for or value of the copyrighted work. It requires courts to
consider not only the extent of market harm caused by the particular actions of
the alleged infringer, but also whether unrestricted and widespread conduct of
the sort engaged in by the defendant would result in a substantially
adverse impact on the potential market for the original.
The enquiry must take account not only of harm to the original
but also of harm to the market for derivative works.

\ldots When a commercial use amounts to
mere duplication of the entirety of an original, it clearly supersedes the
objects of the original and
serves as a market replacement for it, making it likely that cognizable market
harm to the original will occur. But when, on the
contrary, the second use is transformative, market substitution is at least less
certain, and market harm may not be so readily inferred. Indeed, as to parody
pure and simple, it is more likely that the new work will not affect the market
for the original in a way cognizable under this factor, that is, by acting as a
substitute for it.
This is so because the parody and the original usually
serve different market functions.

We do not, of course, suggest that a parody may not harm the market at all, but
when a lethal parody, like a scathing theater review, kills demand for the
original, it does not produce a harm cognizable under the Copyright Act.\ldots
This distinction between potentially remediable displacement and unremediable
disparagement is reflected in the rule that there is no protectible derivative
market for criticism. The market for potential derivative uses includes only
those that creators of original works would in general develop or license others
to develop. Yet the unlikelihood that creators of imaginative works will license
critical reviews or lampoons of their own productions removes such uses from the
very notion of a potential licensing market.\ldots\unskip\readingfootnote{23}{We
express no opinion as to the derivative markets for works using elements of an
original as vehicles for satire or amusement, making no comment on the original
or criticism of it.}

[Here,] 2 Live Crew's song
comprises not only parody but also rap music, and the derivative market for rap
music is a proper focus of enquiry.
Evidence of substantial harm to it would weigh against a
finding of fair use, because
the licensing of derivatives is an important economic incentive to the creation
of originals.
Of course, the only harm to derivatives that need concern us,
as discussed above, is the harm of market substitution. The fact that a parody
may impair the market for derivative uses by the very effectiveness of its
critical commentary is no more relevant under copyright than the like threat to
the original market.

[The Court remanded the case to determine whether 2 Live Crew's song harmed
``the market for a non-parody rap version'' of the original song, and for
further determinations on other issues identified above. Justice Kennedy's
concurrence is omitted.]

