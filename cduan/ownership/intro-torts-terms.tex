The original \emph{Open Source Property} module on property torts provides a
wonderfully detailed history of the causes of action based on property. In
modern practice, the key terminology for you to know is as follows.

Terminology-wise, property is divided into \term{real property}, which refers to
rights in land and things like houses attached to it, and \term{personal
property}, which refers to rights in all other physical objects. (Property in
intangibles has no conventional general name.) The terms \term{realty} and
\term{personalty} (note the missing ``i'' in each) are synonymous; they are also
sometimes called ``immovable'' and ``movable'' property as well.
\unskip\footnote{Classic law school question: what is a mobile home?}
Personal property is also called ``chattels,'' though that term has a
problematic history.

For real property, the primary tort is \term{trespass}. The traditional remedy
for trespass is money damages for injuries caused by the trespass. If a
landowner hopes to have an intruder removed from the land, the cause of action
was traditionally called \term{ejectment}. \unskip\footnote{For historical
reasons, ejectment was a cause of action that only a lease tenant could bring,
not the actual landowner. As a result, landowners seeking ejectment would (and
were allowed to) invent a fictional lessee to be the ``plaintiff'' in the
ejectment case. If you see a case with a caption like \emph{Martin v.~Hunter's
Lessee}, 14 U.S. 304 (1816), that's what the ``lessee'' was for.}

For personal property, the tort of \term{conversion} refers to the wrongdoer
taking possession of another's property. The term is commonly used today, and
probably originates based on the theory that the wrongdoer has ``converted'' the
property to another use. A less common synonym for conversion is \term{trover}.
If the wrongdoer damages the property without appropriating it, then the action
is for \term{trespass to chattels}. Traditionally, these were both actions for
damages, since the property might have been used up, say by being eaten.
Recovery of the taken object itself was by the action of \term{replevin}.

Today, courts have more freedom to award legal and equitable remedies regardless
of the phrasing of the complaint---this was the major Civil Procedure reform of
1938. The terminology distinctions are thus generally not controlling.
Nevertheless, they are useful terms to know because they will show up in cases
and other sources.

Two other terms are important. If there is a dispute over who owns something and
a court is called in to decide, that is an action for \term{quiet title}.
Finally, \term{infringement} is a general-purpose term for any violation of a
property right, but it is specifically used to refer to intellectual property
violations.

The following table summarizes these torts.

\expectnext{narrative-torts-table}
