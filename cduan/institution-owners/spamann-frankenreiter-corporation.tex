\readingnote{This content has been made available under a Creative Commons
Attribution-Noncommercial-Sharealike 3.0 Unported license,
\url{https://creativecommons.org/licenses/by-nc-sa/3.0/}. Citations have been
removed, and other alterations are indicated.}

\reading{The Corporation}

\readingcite{In \textsc{Holger Spamann \& Jens Frankenreiter},
\textsc{Corporations} (3d ed. 2023),
\url{https://opencasebook.org/casebooks/261-corporations/}}

Formally speaking, a corporation is nothing but an \emph{abstraction} to which
we assign rights and duties. It exists independently of humans in the sense that
it has indefinite life, and its assets and obligations are legally separate from
those of any humans involved in its founding or administration.\ldots

Of course, being an abstraction rather than a real person, the corporation
cannot exercise its rights, discharge its duties, or consume its profits by
itself. Human beings must act on its behalf and ultimately consume its profits,
if any.\ldots The basic \emph{default governance} is simple: (common)
shareholders elect the board of directors, which formally manages the
corporation, mostly by appointing the chief executive officer and other top
management, who in turn act on behalf of the corporation in day-to-day matters.
As to consuming the profits, the board may decide to distribute available funds
to shareholders---or not.\ldots

To make this more concrete, think of your local pizza store. Perhaps it is
called ``Olivia's Pizza,'' and Olivia indeed runs the place. You might think
that Olivia is the ``owner'' of the store. In all likelihood, however, the
formal ``owner'' of the pizza place---or rather the contracting party on the
relevant contracts---is actually a corporation. The corporation might be called
``Olivia's Pizza Place Inc.,'' or ``XYZ Corp.'' for that matter. XYZ Corp. might
be (a) the lessee under any lease contract for the store building or other
leased items, (b) the employer of any employees, (c) the owner of any real
estate or chattel such as the pizza oven or the store sign, and (d) the
contracting party with the payment system operator (so your payment for the
pizza might show up under ``XYZ Corp.'' on your credit card statement).\ldots

One benefit of incorporating can be convenience in contracting in certain
transactions. If Olivia ever wanted to sell the pizza place after incorporating,
she would just sell the corporation---a single asset (or to be more precise, all
her shares in the corporation, still just one collection of a uniform asset). By
contrast, as a single owner, she would have to transfer all the assets
individually.

Another convenience is that incorporating changes the default rule from
unlimited liability to \term{limited liability}. The default rule for
corporations is that shareholders, directors, and corporate officers are not
liable for corporate debts (but they do stand to lose any assets they invested
in the corporation as shareholders: hence the expression ``limited liability''
rather than ``no liability''). By contrast, the default rule for single owners
is the same as that for any other individual debt: full liability except for
protection under the bankruptcy code.\ldots

Another benefit is \term{entity shielding}. Entity shielding refers to a
liability barrier in the opposite direction: Olivia's personal creditors cannot
demand payment or seize any assets from XYZ Corp. The personal creditors can
only seize Olivia's shares in XYZ Corp. Entity shielding is extremely useful
because it allows those interacting with XYZ Corp. to focus their attention on
the pizza store's assets and financial prospects, and not worry about Olivia's
other businesses. Imagine for example that Olivia also runs a construction
business in a different city. Without entity shielding, creditors from the
construction business might seize assets of the pizza store, and vice versa. As
a consequence, the two businesses' financial health could not be assessed
independently of each other. By contrast, with entity shielding, a bank making a
loan to develop the pizza store need only assess the financial prospects of the
pizza store, i.e., XYZ Corp. And if the construction business does fail, XYZ
Corp. can nevertheless continue business as usual.
