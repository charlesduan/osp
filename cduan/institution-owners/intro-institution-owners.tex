This book began with a premise that \useterm{property} is a relationship between
people and (tangible or intangible) things. Most often, the relationship is
direct---a person is the owner of a thing, enjoying property rights against the
world. But that is not the only arrangement. People can also have indirect
relationships with property, through a legally created entity. The people own
and control an entity, and the entity owns and controls some property. But there
is not necessarily transitivity: The people do not necessarily own or control
the property.

In this chapter, we will explore three of these institutional arrangements of
property ownership:
\begin{itemize}
\item First is the \term{trust}, an arrangement in which one person (or
property-holding entity) holds property for the benefit of another.
\item Second, there are a variety of business associations that have the power
to hold property as an independent legal entity, such as partnerships, limited
liability companies, and professional associations. A full treatment of these is
a matter for a business associations course; here our focus will be on one type
of association, the \term{corporation}.
\item Third, marriage is an institution that can hold property. As we will see,
property ownership within a marriage can have its own special rules, and in some
respects can resemble corporate ownership.
\end{itemize}
As you read, pay close attention to the structure of relationships. Be diligent
about tracking legal entities and their relationships with people and property,
and be careful not to mix them up. You will be rewarded with a new set of
powerful tools for arranging ownership and affairs.

