The institution of marriage is another legal vehicle for property ownership. A
married couple can hold property in two special forms that unmarried couples
cannot: in \term{tenancy by the entirety}, and as \term{community property}.
Each of these special forms offers distinct benefits, rules, and consequences.

As with other forms of concurrent ownership, you should be able to identify the
formation requirements, the powers and duties of the co-tenants, and the rules
for dissolution of these marital property structures. Indeed, typically marital
property is included in the same chapter as tenancy in common and joint tenancy.
But there are also similarities to corporate ownership and trusts---see if you
can find them.

One additional thought to consider: these forms of property are limited to
married people. Is it fair that the special features and benefits of tenancy by
the entirety and community property are limited in this way? And can you
creatively devise structures that enable unmarried people to own property with
the same, or at least similar, features and benefits?
