\reading{Ghen v. Rich}

\readingcite{8 F. 159 (D. Mass. 1881)}

\captionedgraphic{allocation-img002}{Source: ``Fast to a whale, shooting the
bomb lance.'' New Bedford Free Public Library. \emph{Digital Commonwealth},
\protect\url{http://ark.digitalcommonwealth.org/ark:/50959/sb398z14j}.}

\opinion \textsc{Nelson}, D.J.

This is an action to recover the value of a fin-back whale. Ghen, the
plaintiff,\edfootnote{In the original case, the action is called a ``libel,''
the plaintiff Ghen the ``libellant,'' and the defendant Rich the ``respondent.''
For simplicity, these terms have been modernized and the parties' names used,
without indication.}
lives in Provincetown and Rich, the defendant, in Wellfleet.

In the early spring months the easterly part of Massachusetts Bay is frequented
by the species of whale known as the fin-back whale. Fishermen from Provincetown
pursue them in open boats from the shore, and shoot them with bomb-lances fired
from guns made expressly for the purpose. When killed they sink at once to the
bottom, but in the course of from one to three days they rise and float on the
surface.\ldots
%Some of them are picked up by vessels and towed into Provincetown. Some
%float ashore at high water and are left stranded on the beach as the tide
%recedes. Others float out to sea and are never recovered.
The person who happens
to find them on the beach usually sends word to Provincetown, and the owner
comes to the spot and removes the blubber.\ldots
%The finder usually receives a small salvage for his services.
%Try-works are established in Provincetown for trying
%out the oil. The business is of considerable extent, but, since it requires
%skill and experience, as well as some outlay of capital, and is attended with
%great exposure and hardship, few persons engage in it. The average yield of oil
%is about 20 barrels to a whale.
[The whale] swims with great swiftness, and for that
reason cannot be taken by the harpoon and line. Each boat's crew engaged in the
business has its peculiar mark or device on its lances, and in this way it is
known by whom a whale is killed.

The usage on Cape Cod, for many years, has been that the person who kills a
whale in the manner and under the circumstances described, owns it, and this
right has never been disputed until this case. Ghen has been engaged in
this business for ten years past. On the morning of April 9, 1880\ldots
%, in
%Massachusetts Bay, near the end of Cape Cod,
he shot and instantly killed with a bomb-lance the whale in question.\ldots
% It sunk immediately, and on the morning of the
%12th was found stranded on the beach in Brewster, within the ebb and flow of the
%tide, by one Ellis, 17 miles from the spot where it was killed.
[Ellis found it.] Instead of
sending word to Provincetown, as is customary, Ellis advertised the whale for
sale at auction, and sold it to Rich\ldots.
%, who shipped off the blubber and
%tried out the oil. Ghen heard of the finding of the whale on the
%morning of the 15th, and immediately sent one of his boat's crew to the place
%and claimed it.
Neither Rich nor Ellis knew the whale had been killed
by Ghen, but they knew or might have known, if they had wished, that it
had been shot and killed with a bomb-lance, by some person engaged in this
species of business.

Ghen claims title to the whale under this usage. Rich insists
that this usage is invalid.

%It was decided by Judge Sprague, in \emph{Taber v.
%Jenny}, 1 Sprague, 315, that when a whale has been killed, and is anchored and
%left with marks of appropriation, it is the property of the captors; and if it
%is afterwards found, still anchored, by another ship, there is no usage or
%principle of law by which the property of the original captors is diverted, even
%though the whale may have dragged from its anchorage. The learned judge says:
%\begin{quote}
%When the whale had been killed and taken possession of by the boat of the
%Hillman, (the first taker,) it became the property of the owners of that ship,
%and all was done which was then practicable in order to secure it. They left it
%anchored, with unequivocal marks of appropriation.
%\end{quote}
%
%In \emph{Bartlett v. Budd}, 1 Low. 223, [the plaintiff killed a whale,
%anchored it, marked it with a ``waif'' (a flag identifying the possessing ship),
%and left it for the night.]
%the facts were these: The first officer
%of the plaintiff's ship killed a whale in the Okhotsk sea, anchored it,
%attached
%a waif\edfootnote{``The waif is a
%pennoned pole, two or three of which are carried by every boat; and which, when
%additional game is at hand, are inserted upright into the floating body of a
%dead whale, both to mark its place on the sea, and also as token of prior
%possession, should the boats of any other ship draw near.'' \textsc{Herman
%Melville, Moby-Dick} 368 (1922) [1892].} to the body, and then left it and went
%ashore at some distance for the night.
%The next morning the boats of the
%defendant's ship found the whale adrift, the anchor not holding, the cable
%coiled round the body, and no waif or irons attached to it. Judge Lowell held
%that, as the plaintiffs had killed and taken actual possession of the whale, the
%ownership vested in them. In his opinion the learned judge says:
%\begin{quote}
%A whale, being \textit{ferae naturae}, does not become property until a firm
%possession has been established by the taker. But when such possession has
%become firm and complete, the right of property is clear, and has all the
%characteristics of property.
%\end{quote}
%[The defendants argued]
%%He doubted whether a usage set up but not proved by the defendants,
%that a
%whale found adrift in the ocean is the property of the finder, unless the first
%taker should appear and claim it before it is cut in.
%[In response, the judge]
%%, would be valid, and
%remarked that ``there would be great difficulty in upholding a custom that
%should
%take the property of A. and give it to B., under so very short and uncertain a
%substitute for the statute of limitations, and one so open to fraud and
%deceit.''
%Both the cases cited were decided without reference to usage, upon the ground
%that the property had been acquired by the first taker by actual possession and
%appropriation.
%
%In \emph{Swift v. Gifford}, 2 Low, 110, Judge Lowell decided that a custom among
%whalemen in the Arctic seas, that the iron holds the whale, was reasonable and
%valid. In that case a boat's crew from the defendant's ship pursued and struck
%a whale in the Arctic Ocean, and the harpoon and the line attached to it
%remained in the whale, but did not remain fast to the boat. A boat's crew from
%the plaintiff's ship continued the pursuit and captured the whale, and the
%master of the defendant's ship claimed it on the spot. It was held by the
%learned judge that the whale belonged to the defendants. It was said by Judge
%Sprague, in \emph{Bourne v. Ashley}, an unprinted case referred to by Judge
%Lowell in \emph{Swift v. Gifford}, that the usage for the first iron, whether
%attached to the
%boat or not, to hold the whale was fully established; and he added that,
%although local usages of a particular port ought not to be allowed to set aside
%the general maritime law, this objection did not apply to a custom which
%embraced an entire business, and had been concurred in for a long time by every
%one engaged in the trade.
%
%In \emph{Swift v. Gifford}, Judge Lowell also said:
%\begin{quote}
%The rule of law invoked in this case is one of very limited application. The
%whale fishery is the only branch of industry of any importance in which it is
%likely to be much used, and if a usage is found to prevail generally in that
%business, it will not be open to the objection that it is likely to disturb the
%general understanding of mankind by the interposition of an arbitrary
%exception.
%\end{quote}

[The court reviewed several cases on conflicts over whale ownership, in
which the party who first harpooned the whale was awarded ownership.]

I see no reason why the usage proved in this case is not as reasonable as that
sustained in the cases cited. Its application must necessarily be extremely
limited, and can affect but a few persons. It has been recognized and acquiesced
in for many years. It requires in the first taker the only act of appropriation
that is possible in the nature of the case. Unless it is sustained, this branch
of industry must necessarily cease, for no person would engage in it if the
fruits of his labor could be appropriated by any chance finder.\ldots
%It gives reasonable salvage for securing or reporting the property.
That the rule works
well in practice is shown by the extent of the industry which has grown up under
it, and the general acquiescence of a whole community interested to dispute it.
It is by no means clear that without regard to usage the common law would not
reach the same result. That seems to be the effect of the decisions in
[some of the cases reviewed].
%\emph{Taber v. Jenny} and \emph{Bartlett v. Budd}.
If the fisherman does all that is possible to do to
make the animal his own, that would seem to be sufficient. Such a rule might
well be applied in the interest of trade, there being no usage or custom to the
contrary. Holmes, Com. Law, 217. But be that as it may, I hold the usage to be
valid, and that the property in the whale was in Ghen.\ldots


