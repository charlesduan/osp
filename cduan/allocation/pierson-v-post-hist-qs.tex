\expected{pierson-v-post}

\defjrnart{berger}{Bethany Berger, It's Not About the Fox: The Untold
Story of \emph{Pierson v. Post}, 55 Duke Law Journal 1089 (2006)}

\defjrnart{mcdowell}{Andrea McDowell, Legal Fictions in \emph{Pierson v. Post},
105 Michigan Law Review 735 (2007)}

\defjrnart{fernandez}{Angela Fernandez, The Lost Record of \emph{Pierson v.
Post}, the Famous Fox Case, 27 Law and History Review 149 (2009)}

\defnewsart{hedges}{
author=H.P. Hedges,
title=Pierson vs. Post,
journal=The Sag-Harbor Express,
date=oct 24 1895,
page=1,
}

\defjrnart{singer}{
Joseph William Singer, Starting Property, 46 Saint Louis University Law Journal
565 (2002)
}

\item \textbf{What the Fox?} \emph{Pierson} is perhaps the most celebrated case
in American property law, and scholars have delighted in debating its history
and theory. Among other things, research has shown that the case's presentation
of the facts is incomplete and likely misleading.

Lodowick Post and Jesse Pierson were young men of well-to-do families. The
Piersons were local farmers and longtime residents of the Southampton area
where the case took place, and the Posts were more recent newcomers who had made
their fortune in business (Lodowick's father Nathan Post may have been a whaling
ship captain). \sentence{mcdowell at 744-745}. Fox hunting was typically seen by
upper-class New Yorkers as a leisure sport, the agricultural New Englanders
would have treated foxes as dangerous vermin who stole farmers' chickens.
\sentence{see berger at 1131-1133; fernandez at 166; mcdowell at 764}.
Southampton, at the eastern end of Long Island and closer to Connecticut than
New York City, was ``thus on a boundary, of sorts,'' between these cultural and
professional difference.

The case report describes the land on which the fox was found as ``wild and
uninhabited, unpossessed and waste.'' But it may have been a community-owned
pasture in which the Piersons had an ownership share. \sentence{see berger at
1120-1121}. In any event, the fox was found not far from Pierson's house,
suggesting that Pierson may have killed the fox to protect his chickens.
\sentence{see fernandez at 167-168}. Pierson was not himself hunting, but saw
the fox run down a well and clubbed it to death. \sentence{see fernandez at
166}. Post arrived and demanded the body of the fox, to which Pierson apparently
responded, ``it may be you was going to kill him, but you did not kill him. I
was going to kill him and did kill him.'' \sentence{fernandez at 166 (quoting:
hedges)}.

Despite serious flaws in litigation procedure, the New York Supreme Court
plainly saw the case as an opportunity to develop the law, and waited for over
two years to issue an opinion. \sentence{see fernandez at 172-175}. As one
scholar notes, the opinions' citation to Roman and civil law rather than
Blackstone and common law suggests the court's desire to reject English law
after the American Revolution. \sentence{see berger at 1135-1136}. Furthermore,
the judges' backgrounds may have played a role in their conceptions of property:
The Livingston family was one of the largest landowners in New York state,
renting their property to thousands of tenant farmers; Tompkins' parents were
tenant farmers who rented land. \sentence{see berger at 1138-1139}.

What of this informs your understanding of the outcome of the case? The New
England--New York cultural divide? The different interests of farmers and
leisure hunters with respect to foxes? The title ownership of the land? English
versus American norms? Something else? And is this case really about the initial
allocation of property, as most textbooks suggest? \sentence{see berger at 1142;
singer at 569-570 (``\emph{Pierson} is not about the means of initial
acquisition but rather concerns social relationships.'')}.
