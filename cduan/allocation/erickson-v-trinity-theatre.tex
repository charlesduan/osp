\reading{Erickson v. Trinity Theatre, Inc.}
\readingcite{13 F.3d 1061 (7th Cir. 1994)}

\opinion \textsc{Ripple}, Circuit Judge.

The plaintiff Karen Erickson \ldots was one of the founders of a theatre company
in Evanston, Illinois, that ultimately became known as Trinity Theatre [the
defendant]. Between 1981 and January 1991, Ms. Erickson served Trinity in
various capacities: as playwright, artistic director, actress, play director,
business manager, and member of the board of directors. This suit revolves
around Ms. Erickson's role as playwright.

[Erickson wrote three plays, in collaboration with the Trinity actors. Here is
the court's description of one such collaboration over a play \emph{Much Ado
About Shakespeare}:]

\emph{Much Ado} is a
compilation of scenes and sonnets from William Shakespeare and other writers of
his time. Ms. Erickson revised this work from an earlier script entitled
\textit{Sounds and Sweet Aires.} Michael Osborne, a Trinity actor, testified
that Ms. Erickson compiled \textit{Much Ado} in 1988 and that many decisions
about what was to be included were made during rehearsals. Osborne identified
two portions of the copyrighted script that resulted from his suggestions: a
passage to \textit{Macbeth} and the  introduction to the play. The editing of
the text, Osborne continued, was accomplished largely by consensus; however,
when a consensus could not be had, Ms. Erickson made the final decisions.
Osborne further testified that he understood at the time that the play was being
created for Trinity and not for Ms. Erickson. Ms. Erickson does not dispute the
process described by Osborne, but characterizes it differently. She perceived
the process only as actors making suggestions for her script.

[After Erickson left the company, Trinity continued performing her plays.
Erickson sued for copyright infringement. Trinity's defense was that, by virtue
of its actors' collaboration with Erickson in writing the plays, they were joint
authors and so Trinity was a joint owner of the copyrights.]

We now turn to the issue of whether any of the material in question is a ``joint
work.'' In a joint work, the joint authors hold undivided interests in a work,
despite any differences in each author's contribution. Each
author as co-owner has the right to use or to license the use of the work,
subject to an accounting to the other co-owners for any profits.
Thus, even a person whose contribution is
relatively minor, if accorded joint authorship status, enjoys a significant
benefit.\ldots

[The court first held that, in order for a copyrighted work to be a joint work,
each author must ``intend that their respective contributions be merged into a
unitary whole.'']

Even if two or more persons collaborate with the intent to create a unitary
work, the product will be considered a ``joint work'' only if the collaborators
can be considered ``authors.'' Courts have applied two tests to evaluate the
contributions of authors claiming joint authorship status: Professor Nimmer's de
minimis test and Professor Goldstein's copyrightable subject matter
(``copyrightability'') test. The de minimis and copyrightability tests differ in
one fundamental respect. The de minimis test requires that only the combined
product of joint efforts must be copyrightable. By contrast, Professor
Goldstein's copyrightability test requires that each author's contribution be
copyrightable. We evaluate each of these tests in turn.\ldots

[Nimmer's] position has not found support in the courts. The lack
of support in all likelihood stems from one of several weaknesses in Professor
Nimmer's approach. First, Professor Nimmer's test is not consistent with one of
the Act's premises: ideas and concepts standing alone should not receive
protection. Because the creative process necessarily involves the development of
existing concepts into new forms, any restriction on the free exchange of ideas
stifles creativity to some extent. Restrictions on an author's use of existing
ideas in a work, such as the threat that accepting suggestions from another
party might jeopardize the author's sole entitlement to a copyright, would
hinder creativity. Second, contribution of an idea is an exceedingly ambiguous
concept. Professor Nimmer provides little guidance to courts or parties
regarding when a contribution rises to the level of joint authorship except to
state that the contribution must be ``more than a word or a line.''\ldots

[Goldstein's test]
has been adopted, in some form, by a majority of courts that have considered
the issue. According to Professor
Goldstein, ``[a] collaborative contribution will not produce a joint work, and a
contributor will not obtain a co-ownership interest, unless the contribution
represents original expression that could stand on its own as the subject matter
of copyright.''\ldots
We agree that the language of the [Copyright] Act supports the adoption of a
copyrightability requirement. Section 101 of the Act defines a ``joint work'' as
a ``work prepared by two or more \textit{authors}'' (emphasis added). To qualify
as an author, one must supply more than mere direction or ideas. An author is
the party who actually creates the work, that is, the person who translates an
idea into a fixed, tangible expression entitled to copyright protection.\ldots

The copyrightable subject matter test does not suffer from the same infirmities
as Professor Nimmer's de minimis test. The copyrightability test advances
creativity in science and art by allowing for the unhindered exchange of ideas,
and protects authorship rights in a consistent and predictable manner. It
excludes contributions such as ideas which are not protected under the Copyright
Act. This test also enables parties to
predict whether their contributions to a work will entitle them to copyright
protection as a joint author. Compared to the uncertain exercise of divining
whether a contribution is more than de minimis, reliance on the copyrightability
of an author's proposed contribution yields relatively certain answers.
The copyrightability standard allows contributors to avoid
post-contribution disputes concerning authorship, and to protect themselves by
contract if it appears that they would not enjoy the benefits accorded to
authors of joint works under the Act.

[Applying the test laid out above, the court found that there might be a dispute
over intent to create a joint work. It did not need to resolve that dispute,
because:]

In order for the plays to be joint works under the Act, Trinity also must show
that actors' contributions to Ms. Erickson's work could have been independently
copyrighted. Trinity cannot establish this requirement for any of the above
works. The actors, on the whole, could not identify specific contributions that
they had made to Ms. Erickson's works. Even when Michael Osborne was able to do
so, the contributions that he identified were not independently copyrightable.
Ideas, refinements, and suggestions, standing alone, are not the subjects of
copyrights. Consequently, Trinity cannot establish the two necessary elements of
the copyrightability test and its claims must fail.

