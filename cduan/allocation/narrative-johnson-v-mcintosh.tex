\defcase{johnson}{
Johnson v. McIntosh, 21 U.S. 240 (1823)
}
\defjrnart{collins}{
Michael G. Collins, M`Culloch and the Turned Comma, 12 Green Bag 2d 265 (2009)
}

A classic property law case,
not included in this book, is \inline{johnson}, in which the two
parties held competing titles to the same land, one obtained from the U.S.
government and the other from the American Indian nations.
\sentence{johnson}.\footnote{You will often see the case styled as
``\emph{Johnson v.~M'Intosh}.'' This is incorrect: The apostrophe should face
the other direction (\emph{M`Intosh}), such that it looks like a small
superscript letter c. \emph{See} \sentence{collins}.}
Chief Justice John Marshall, writing for the Supreme Court, held in favor of the
government-obtained title on the grounds that, by conquest of the American
Indians, the United States held superior title to the land.

A system of land ownership founded on violent conquest seems arbitrary and
unjustifiable today. Indeed, Chief Justice Marshall seems almost embarrassed to
confirm the ``extravagant\ldots pretension'' that European discovery and
conquest is not only a legitimate source of land titles in the United States,
but the \textit{only} legitimate source of such titles.

Is the United States' dispossession of Native Americans really a
``historical'' injustice? Professor Joseph Singer has long faulted the American
legal system for its continued mistreatment of Native Americans:
\begin{quotation}
[T]itle to land in the United States rests on the forced taking of land from
first possessors---the very opposite of respect for first possession. Conquest
is a mode of original acquisition that we cannot sweep under the rug by
pretending that it accords with any recognizable principle of justice. And
conquest, unfortunately, is where American history starts---as does the title to
almost every parcel of land in the United States. This is a highly inconvenient
(not to say stunningly demoralizing) fact, not least of all to the Indian
nations that continue to inhabit the North American continent\ldots .

Many of us protect ourselves from having to think too deeply about conquest by
distancing ourselves from it.\ldots If we can relegate conquest to the distant
past, we can concentrate instead on the fact that the United States was founded
on respect for property rights. We do not acquire property by conquest today.

This comforting story is misleading at best and false at worst. We cannot
comfort ourselves with the idea that conquest became a thing of the past with
the American Revolution, independence from Great Britain, and the adoption of
the U.S. Constitution.
\end{quotation}
Joseph William Singer, \textit{Original Acquisition of Property: From Conquest
\& Possession to Democracy \& Equal Opportunity}, 86 \textsc{Ind. L.J.} 763,
766-67 (2011) (reproduced with permission of the author). As Professor Singer
explains, \textit{id.} at 767--68, most of the federal government's
dispossession
of Native American land occurred during the 19th century. During the early 20th
century---while the Supreme Court was gaining a reputation for striking down
state economic legislation in the name of protecting freedom of contract and
private property (the so-called ``\textit{Lochner} era''\footnote{\emph{Lochner
v. New York}, 198 U.S. 45 (1905).})---the United States forcibly took
two-thirds of the remaining lands of the Indian nations. The Supreme Court held
in 1955 that Alaska natives possessed merely a license to live on the
land---revocable permission from whites to occupy Alaskan territory. As recently
as 2009, the Supreme Court held that the Navajo Nation had no right to sue the
federal government for damages where the Secretary of the Interior was alleged
to have colluded with a mining company to undercompensate the tribe for mining
rights on lands held under ``joint title'' between the Navajo and the United
States (by law, the Secretary must approve any leases of tribal land for mining
purposes). \textit{United States v. Navajo Nation}, 556 U.S. 287 (2009). As
Professor Singer reminds us, the conquest is not over.

\defmagart{laduke-ricekeepers}{
author=Winona LaDuke,
title=Ricekeepers: A Struggle to Protect Biodiversity and a Native American Way
of Life,
journal=Orion Magazine,
date=july-aug 2007,
url=https://orionmagazine.org/article/ricekeepers/,
}

What about patenting of indigenous technologies? \sentence{see
laduke-ricekeepers}.

