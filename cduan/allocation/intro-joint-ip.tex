\defcase{berge-v-board}{
parties=United States ex rel. Berge v. Board of Trustees of University of
Alabama,
cite=104 F.3d 1453,
court=4th Cir.,
year=1997,
}
\defcongdoc{hrep94-1476}{
serial=H.R. Rep. No. 94-1476,
year=1976,
}

Just as with land, there can be multiple owners of intellectual property
rights. The copyright laws contemplate a \emph{joint work} that is ``prepared by
two or more authors with the intention that their contributions be merged into
inseparable or interdependent parts of a unitary whole.'' 17 U.S.C. \S~101.
``The authors of a joint work are coowners of copyright in the work,''
\S~201(a), and they each have ``an undivided, independent right to use the work,
subject only to a duty of accounting for profits to other co-owners.''
\sentence{berge-v-board at 1461 (quoting: hrep94-1476 at 121)}. Similarly, a
patented invention may be ``made by two or more persons jointly,'' 35 U.S.C.
\S~116(a), and ``each of the joint owners of a patent'' may exploit the
invention ``without the consent of and without accounting to the others,'' in
the absence of an agreement otherwise, \S~262. \footnote{Notice the use of
``joint owners'' here. This does \emph{not} refer to joint tenancy in the
formal sense; the terminology is unfortunately domain-specific.}

So who counts as a joint author or inventor? Given the power of any joint
intellectual property owner to use and license the work, this question can have
tremendous ramifications. The answers are not always expected and often
debatable. As you read these materials, think about the last big group project
you did---who did what work and who received what credit---and see if the legal
rules line up with your intuitions.



