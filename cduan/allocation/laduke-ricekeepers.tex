\reading[LaDuke, \emph{Ricekeepers}]{Winona LaDuke, \emph{Ricekeepers: A
Struggle to Protect Biodiversity and a Native American Way of Life}}
\readingcite{\textsc{Orion Mag.}, July--Aug. 2007}

\textsc{As fall temperatures change} on the White Earth Reservation
and the mist lifts off the lakes, the Ojibwe take to the waters. Two
people to a canoe, one poles through the thick
rice beds, pushing the canoe forward, while the other, sitting toward
the front of the boat, uses two long sticks to gently bend the rice and
knock the seeds into the canoe. The sounds of \emph{manoominike}, the
wild rice harvest, are the gliding of the boat through the water and
across shafts of rice, the soft \emph{swish} of the rice bending, the
raining of the rice into the canoe. They are soothing sounds, reminding
my people of the continuity between the generations. We have been
harvesting rice here for centuries.

Each year, my family and I join hundreds of other harvesters who return
daily with hundreds of pounds of rice from the region's lakes and
rivers. We call it the Wild Rice Moon, Manoominike Giizis. On White
Earth, Leech Lake, Nett Lake, and other Ojibwe reservations in the Great
Lakes region, it is a time when people harvest a food to feed their
bellies and to sell for \emph{zhooniyaash}, or cash, to meet basic
expenses. But it is also a time to feed the soul.


\textsc{Fifteen hundred miles away}, in Woodland, California, a company
called Nor-Cal has received a patent on \emph{wild} rice. Conceptually,
it seems almost impossible---patenting something called wild rice. The
Ojibwe now find themselves at the center of an international battle over
who owns lifeforms, foods, and medicines that have throughout history
been the collective property of indigenous peoples.

An estimated 90 percent of the world's biodiversity lies within the
territories of indigenous peoples, whether the Amazon, the Indian
subcontinent, or the North Woods. A new form of colonialism, known as
biocolonialism, is reaching deep into the heart of these communities. As
Stephanie Howard wrote for the Indigenous People's Council on
Biocolonialism, ``The flow of genes is primarily from indigenous
communities and rural communities in `developing countries' to the
Northern-based genetics industry. Ninety-seven percent of all patents
are held by industrialized countries.''

In 1994, for example, two researchers at the University of Colorado were
able to secure a patent on quinoa, much to the surprise of native
farmers in the Andean region of Bolivia and
Ecuador
who had been cultivating and stewarding the grain for thousands of
years. The patent gave the university exclusive control over a
traditional Bolivian sterile male variety called Apelawa, and also
extended to hybrids developed from the breeding of forty-three
additional traditional varieties. In 1998, the Bolivian National Quinoa
Producers Association, with support from other groups internationally,
was able to convince the researchers to drop the patent. But similar
patents were issued on the neem tree, ayahuasca (a medicinal plant of
the Amazon), and many other medicinal plants. Some of these were also
eventually revoked. In September 1997, RiceTec, a Texas-based company,
even won a controversial patent on the famed basmati rice. When the
Indian government filed a complaint with the U.S. Patent and Trademark
Office, RiceTec was forced to give up fifteen of twenty patent claims.

It was within this climate that University of Minnesota plant geneticist
Ron Phillips, along with a few colleagues, mapped the wild rice genome
in 2000. According to Phillips, this work is considered ``important as a
foundation for genetic and crop improvement studies.'' The Ojibwe
believe that these studies, bearing names such as ``Molecular
Cytogenetics in Plant Improvement,'' could have far-reaching
implications. The wild rice gene map is now filed with GenBank, a
database operated by the National Institutes of Health, and its
availability essentially sets the stage for genetic modification.

Traditional breeding techniques attempt to enhance certain traits of the
wild rice and to repress others, but with genetic engineering, it
becomes possible to insert DNA from other plants into the wild rice. The
Ojibwe are alarmed by this possibility, viewing it as an attack on the
essential nature of the rice itself.

\textsc{Thousands of years ago}, according to our oral histories, the
Anishinaabeg---called the Ojibwe or Chippewa by the federal
government---followed a shell in the sky from the great waters of the East to
the
place where the food grows on the water. That food was wild rice, the
only grain indigenous to North America, and it has been a central food
in ceremony and sustenance for our people ever since. ``The[y] gain
their livelihood by fishing, hunting, gathering berries and wild rice
and making maple sugar, which constitutes their chief means of
support,'' Indian agents would write, noting that the Ojibwe also relied
on wild rice as a source of trade with the white settlers, and later as
a source of credit and cash.

The rice was so significant to the Ojibwe that the lands with the best
wild rice stands---including Big Rice Lake, Rice Lake Refuge, Lake
Winnibigoshish, Nett Lake, and other mother lodes of the great grain---were
reserved. Beyond the reservation borders, land was transferred to
the U.S. government, but the rice was not. In an 1837 treaty, the Ojibwe
ceded nearly 14 million acres of Wisconsin and Minnesota but retained
``the privilege of hunting, fishing, and gathering the wild rice upon
the lands, the rivers and the lakes included in the territory ceded.''
Federal and Supreme Court cases, including the 1999 Mille Lacs Supreme
Court case, have upheld the rights of the Ojibwe to traditional land-use
outside the reservations.

It was this close bond between a people and a food that University of
Minnesota professor Albert Jenks encountered when he came to White Earth
and other reservations to study wild rice in the late 1800s. He noted
with disdain the Ojibwe harvesting practices. ``Wild rice, which had led
to their advance thus far, held them back from further progress,'' he
determined. His perception of the Ojibwe wild rice harvest as a bastion
of primitiveness would become the prevailing opinion at the University
of Minnesota throughout the twentieth century---indeed, a sort of
battle cry for industrializing agriculture.

In the 1950s, University of Minnesota researchers decided it was time to
liberate the rice from the indigenous people. So they set out to
domesticate wild rice. A university scientist named Ervin Oelke began
the process, using germ plasm collected from twenty-four natural stands
within the 1837 treaty area. Over the years, the Minnesota Agricultural
Extension office was able to ``create'' several strains of ``wild''
rice: Johnson in 1968, M1 in 1970, M2 in 1972, M3 in 1974, Netum in
1978, Voyager in 1983, Meter in 1985, Franklin in 1992, and Purple
Petrowski in 2000.

In effect, what the Creator gave to the Anishinaabeg has become a
profit-making enterprise for others. These domesticated varieties are
engineered to ripen at the same time and, with a harder hull, can be
harvested mechanically. They are cultivated in paddies, flooded fields
that are drained to allow access with a combine. By 1968, Minnesota's
paddy wild rice production already represented some 20 percent of the
state's yield. This increase in production, along with growing national
demand for wild rice and subsequent interest from corporations such as
Uncle Ben's, Green Giant, and General Foods, permanently altered the
market for traditionally harvested wild rice. Lake rice could no longer
compete with the mass-manufactured paddy crop. The wholesale wild rice
price dropped from \$4.44 per pound in 1967 to \$2.68 a pound in 1976,
destabilizing the wild rice economy of the Ojibwe.

Then, in 1977, the Minnesota state legislature designated wild rice the
official state grain---a tragic turn of events for the lake harvest.
With an outpouring from the state coffers, the University of Minnesota
began to aggressively market a domesticated version of wild rice. By the
early 1980s paddy-grown wild rice had outstripped the indigenous
varieties in production.

Ironically, greed knows no state boundaries. Minnesota lost control over
production of its official state grain to California, which by 1983
produced over 8.3 million pounds, compared to Minnesota's 5 million
pounds. By 1986, more than 95 percent of the wild rice harvested was
paddy grown, the vast majority produced in California. As this glut of
wild rice hit the market, prices plummeted. Many Ojibwe lost their
source of livelihood. But to add insult to injury, many of the paddy
rice companies were selling their product as if it were \emph{wild} wild
rice, in some cases even using Ojibwe images in their advertising.

The Ojibwe fought back. In 1988, \emph{Wabizii v. Busch Agricultural
Resources}, a lawsuit on the issue of false and misleading advertising,
was filed. Busch Agricultural Resources (a division of the beer
conglomerate) was marketing a product called Onamia Wild Rice, which
plaintiffs Mike Swan and Frank Bibeau charged was in fact a
California-grown paddy product disguised as Minnesota lake rice. ``They
had two Indians on a canoe who appeared to be picking wild rice. They
were taking a California-grown product, trucking it to Minnesota, where
it was packaged and designated as a Minnesota product,'' Bibeau, a White
Earth tribal member, recalls. The case was settled out of court, and
eventually the state passed a law forcing paddy wild-rice producers to
label their product as such, with the words ``paddy rice'' no less than
half the size of the words ``wild rice.'' Still, the Minnesota labeling
law does not apply to California-grown wild rice, so three-quarters of
the nation's domesticated crop can be described as ``wild'' without
qualification.

\textsc{Wild rice, or} \emph{Zizania palustris}, is actually a grass,
sharing only some genetic traits with other rice crops internationally.
The differences in wild rice beds are well known to local harvesters.
Some plants grow tall and live in deep water; others have adapted to
shallow water. Some strains have fat grains; others have long grains.
They range in color from purple to light brown to greenish. That
biodiversity is the staff of life, and it is essential to the security
of the rice. That same biodiversity served as the genetic basis for the
domesticated varieties, an agricultural monocrop.

The Anishinaabeg believe there is a real possibility that wild rice
stands could be contaminated by the domesticated varieties. There are
around six thousand bodies of water with significant wild rice beds in
Minnesota, containing around sixty thousand acres of rice. And there are
around twenty thousand acres of cultivated wild rice paddies in close
proximity to most of those native beds. Ron Phillips claims there is
little chance of cross-pollination as long as approximately 660 feet
separate the two kinds of wild rice. However, in the summer of 2002,
university researchers noted the possibility of between 1 and 5 percent
of the pollen from test plots drifting up to two miles.

Then there is, in Donald Rumsfeld's vernacular, the unknown unknown of
the \emph{zhiishiibig}, the ducks. Ducks and other waterfowl do not
differentiate between paddy rice plots and natural stands of wild rice;
they move freely between them, carrying rice from one to the other.
Phillips himself acknowledges a problem: ``It depends on what you are
willing to accept as a threshold of risk. You can't guarantee . . . that
a bird won't pick up a weed and take it twenty miles away,'' he said.

As for Nor-Cal Wild Rice, U.S. patent number 5955648 secures its rights
to a traditional breeding process which uses something known as
``cytoplasmic genetic male sterility'' to produce hybrid varieties. John
Pershell of the Water Quality Research Department of the Minnesota
Chippewa Tribe read all thirty pages of the patent. ``Nowhere did it
mention anything about the wild rice being wild or coming from
somewhere,'' he said. The rice has basically been co-opted. But what's
worse, those confusing words, ``cytoplasmic genetic male sterility,''
are essentially a fancy way of saying that these varieties cannot
reproduce. They are sterile. The Ojibwe are concerned that, like the
notorious ``terminator'' seeds, Nor-Cal's strain of wild rice could
negatively affect the vitality of wild lake rice.

These fears were validated by two major contamination incidents in
August 2006. In the first, genetically engineered bentgrass escaped its
testing ground in Oregon. Three years earlier, farmers had joined with
environmentalists and the Center for Food Safety in pursuing a lawsuit
against the USDA, which had, in their assessment, failed to properly
regulate varieties of creeping bentgrass and Kentucky bluegrass that had
been genetically engineered to resist the weedkiller Roundup. In
February 2007, U.S. District Judge Henry H. Kennedy Jr. ruled in favor
of the plaintiffs, citing evidence that field tests had the potential to
be harmful to other crops, and instructing the USDA to cease approval
for field tests of genetically engineered crops until it can give more
scrutiny to applications.

Last August as well, news was released that a German company was
responsible for the contamination of a vast portion of the U.S.
long-grain white rice crop by a genetically engineered variety never
intended for human consumption. When the news spread, European and Asian
markets began strictly limiting their importation of all U.S. long-grain
white rice. Japan banned the white rice crop outright. The European
Union demanded that expensive genetic tests be conducted to guarantee no
presence of genetically engineered organisms. Rice futures prices
tumbled \$150 million in a single day and rice exports are estimated to
decline by as much as 16 percent in 2007. Following this fiasco, farmers
from Arkansas, Missouri, Mississippi, Louisiana, Texas, and California
filed a lawsuit against Bayer CropScience, charging the corporation with
tainting the domestic crop and damaging the U.S. export market. Industry
responded by filing a petition to deregulate its untested, genetically
engineered product. Meanwhile, scientists are still trying to figure out
how an experimental crop that was discontinued years ago, and was
apparently grown at distances beyond what the USDA considered adequate
to prevent contamination, managed to become commingled with long-grain
white rice harvested from many different locations.

Though no one has yet attempted to grow genetically engineered wild rice
in the out of doors, a similar contamination scenario would be
devastating. Tainted lake rice would be virtually unable to compete in
international markets, and over half of all wild rice is sold
internationally.


\textsc{For the past nine years}, the Anishinaabeg community has
repeatedly requested that the University of Minnesota stop its genetic
work on wild rice. ``We object to the exploitation of our wild rice for
pecuniary gain,'' wrote Minnesota Chippewa Tribal President Norman
Deschampe in a 1998 letter to the University of Minnesota. He continued:
``We are of the opinion that the wild rice rights assured by treaty
accrue not only to individual grains of rice, but to the very essence of
the resource. We were not promised just any wild rice; that promise
could be kept by delivering sacks of grain to our members each year. We
were promised the rice that grew in the waters of our people, and all
the value that rice holds.''

In September 2003, a coalition of Ojibwe tribal governments and members
demanded the following concessions from the university: a moratorium on
genomic research and genetic research of wild rice at the university, to
be effective December 31, 2004; protection of Anishinaabeg intellectual
property rights to wild rice, including a ban on selling these rights; a
cultural consultation program to be set in place by the university to
examine the ethics of research on cross-cultural issues; and mutually
agreed upon beneficial research to be done on behalf of Anishinaabeg
people, equal to that done on behalf of the cultivated wild-rice
industry. A satisfactory response is still pending. More recently, the
White Earth and Fond du Lac bands of Ojibwe have adopted ordinances
banning the genetic modification of wild rice, following the lead of
several California counties and a host of international ordinances on
GMOs.

In the spring of 2006, a letter signed by over seventy Minnesota state
legislators promoting the protection of wild rice was secured by state
representative Frank Moe, whose constituents include two Ojibwe bands,
but only after a pitched battle with the biotech industry and the
University of Minnesota. At the legislative hearings, representatives
for both the industry and the university testified against protecting
wild rice from genetic engineering, pushing instead for an open-door
policy for the future. Biotech giant Monsanto, unsurprisingly, argued
that such protection would send a ``chilling message'' to the biotech
industry, and perhaps diminish its investment in the state.

Despite this opposition, a protection bill for wild rice was signed into
law in early May 2007. The legislation requires that any entity wanting
to grow genetically engineered wild rice in Minnesota must file an
environmental impact statement with the state. It also requires that
state entities notify the tribes of any permits granted to grow
genetically engineered wild rice in other states, and that they engage
in studies to better understand the threat that genetic engineering
poses to wild rice.

The controversy over wild rice is similar to a recent dispute over taro,
a sacred food of Native Hawaiians. Since January 2006, Hawaiians had
been pressuring the University of Hawai'i to give up patents it held on
three varieties of taro, arguing that taro, the ``elder brother'' of
Native Hawaiians, should not be subject to transgenic experimentation.
Gary Ostrander, vice-chancellor for research and graduate education at
the University of Hawai'i at Manoa, describes how the three
disease-resistant taro strains were created after a leaf blight wiped
out 90 percent of Samoan taro in the 1990s. University scientists had
used traditional breeding techniques to cross Palauan and Hawaiian taro,
and the university had obtained plant patents on the resulting strains
in 2002. However, after negotiations with Native Hawaiian taro farmers
and legal counsel, the university filed ``terminal disclaimers'' with
the U.S. Patent Office, dissolving its proprietary interests. And in
June 2006, the university literally tore up its patents. ``It is as if
the patents were never filed,'' said Ostrander in an article in the
Honolulu \emph{Star-Bulletin}, adding that he had come to appreciate the
Native Hawaiians' point of view on the issue.

Earlier that spring, the pueblos of New Mexico had joined with Hispanic
communities in a historic declaration of seed sovereignty, reaffirming
seed-saving traditions and rejecting patents and genetically engineered
seed. The declaration states that the traditional farmers of
Indo-Hispano and Native-American ancestry in northern New Mexico
``consider genetic modification and the potential contamination of our
landraces by GE technology a continuation of genocide upon indigenous
people and as malicious and sacrilegious acts toward our ancestry,
culture, and future generations.'' In October 2006, the declaration was
passed by the National Congress of American Indians, an organization
comprised of the elected tribal leadership of federally recognized
tribes. And in early 2007, the New Mexico state legislature passed a
memorial ``recognizing the significance of indigenous agricultural
practices and native seeds to New Mexico's cultural heritage and food
security.'' Even though several clauses concerning the threat of genetic
engineering were deleted due to pressure from Monsanto and the State
Department of Agriculture, the final version resolved that the House of
Representatives ``supports efforts to prevent genetic contamination of
native seeds.''

Rowen White, a Mohawk seed saver and farmer, explains what's at stake:
``A cultural community that persists in its farming tradition does not
simply conserve indigenous seed stock because of economic
justifications. The seeds themselves become symbols, reflections of the
people's own spiritual and aesthetic identity, and of the land that
shaped them.''

``We stand to lose everything,'' says White Earth tribal member Joe
LaGarde, who has harvested wild rice since he was a small child. ``If we
lose our rice, we won't exist as a people for long.'' This is why tribal
entities in the North Country are determined to differentiate the wild
rice that is harvested from lakes and rivers from the corporatized
version, and are seeking national and international markets for their
rarefied product. Through work that is somewhat like the fair-trade
struggle of coffee farmers, the Ojibwe are beginning to regain an
economic foothold with the wild rice economy. The key is to keep the
rice and protect it; to remain connected to a traditional way of life
and the land.

It was in that spirit that I took my fifteen-year-old son out ricing on
the Ottertail River last year, far from the din of television, Game
Boys, NASCAR, and big cities. I let him pole for the first time. He's
quite a bit larger than I, and in the past I would do the poling out of
fear that he would dump the boat---and his mother---into the lake.
But over time he's become more steady, and I've become more docile. We
watched the \emph{wabiziwag}, the trumpeter swans, lift off the river
and listened to the sound of rice falling in the canoe.

There is something irreplaceable about following the canoe path of your
ancestors through the rice beds. It's sort of a miracle in this
millennium that this age-old tradition continues. But it does. And it
will.

\emph{Apane}. Always.

