
\defcase{eli-lilly-v-aradigm}{
parties=Eli Lilly and Co. v. Aradigm Corp.,
cite=376 F.3d 1352,
court=Fed. Cir.,
year=2004,
}


\defcase{burroughs-wellcome-v-barr}{
parties={Burroughs Wellcome Co. v. Barr Laboratories, Inc.},
cite=40 F.3d 1223,
court=Fed. Cir.,
year=1994,
}

\item Patent law similarly limits joint inventorship.
A joint inventor is one who ``contributes to the
conception of the claimed invention''---roughly speaking, someone who helps to
form the structural or functional idea of the invention.
\sentence{e.g., eli-lilly-v-aradigm at 1359}. One who merely explains the state
of the art is not a joint inventor; nor is a person who only contributes to
``reduction to practice,'' roughly taking the idea of the invention (from one
who conceived it) and making an operational product out of it.

In \inline{burroughs-wellcome-v-barr}, scientists at pharmaceutical company
Burroughs Wellcome identified a handful of chemicals they thought might be
useful for treating the HIV virus, and asked scientists at the National
Institutes of Health to test them. When the NIH scientists found one of the
candidate chemicals (AZT) to be effective, the company applied for a patent on
it. \sentence{burroughs-wellcome-v-barr at 1226}. Held: the NIH scientists were
not joint inventors, because they merely tested a chemical that Burroughs
Wellcome had conceived---even though the company scientists had no idea which
chemical would work and NIH's scientists did the experimental work.

Do the results in these cases seem fair to you? Would Erickson have written her
plays without the Trinity actors' help, or would Burroughs Wellcome have
discovered AZT as an HIV treatment without NIH\@? Who deserves rights in the
resulting commercial value?


\item In many communities, there are strong norms of who is considered an
``author.'' These norms often do not align with intellectual property law
definitions. In scientific research, for example, principal investigators of
laboratories are often named as ``authors'' on research papers even if they do
not write a single sentence; the scientists who do data analysis often are named
as well even though, \having{feist-v-rural}{as we saw in \emph{Feist}}{as the
Supreme Court held in \emph{Feist v.~Rural}, presented later in the book}{as
the Supreme Court held in \emph{Feist v.~Rural}, 499 U.S. 340 (1991)}, factual
data is not copyrightable matter.

Should community norms and collaborators' intentions matter more than formal
rules laid out by courts unfamiliar with those norms and intentions? On this
question, compare with judicial treatment of parties' intentions for concurrent
ownership in land.


\defcase{naruto-v-slater}{
parties=Naruto v. Slater,
cite=888 F.3d 418,
court=9th Cir.,
year=2018,
}

\defjrnart{bridy}{
author=Annemarie Bridy,
title=Coding Creativity: Copyright and the Artificially Intelligent Author,
date=2012,
cite=5 Stanford Technology Law Review 1,
}

\defgovdoc{uspto-inventorship-guidance}{
title=Inventorship Guidance for AI-Assisted Inventions,
agency=United States Patent and Trademark Office,
date=feb 13 2024,
cite=89 Fed. Reg. 10043,
}

\defcase{thaler}{
parties=Thaler v. Vidal,
cite=43 F.4th 1207,
court=Fed. Cir.,
year=2022,
}


\item What about non-human creators? Can they hold title to intellectual
property? This question has become especially prominent recently in view of
creative and inventive works generated by artificial intelligence. In the famous
``monkey selfie'' case, the Ninth Circuit relied on a textual interpretation of
the Copyright Act to conclude that only humans can maintain rights under
copyright law. \sentence{see naruto-v-slater at 426}. Similarly, courts have
held that only humans can qualify as inventors of patents. \sentence{see thaler
at 1212}.
Nevertheless, there has
been increasing pressure to clarify rights of ownership in intellectual
property from AI-generated works. \note{see, e.g., bridy;
uspto-inventorship-guidance}

What do you think should happen? What would need to change in the law, in order
to grant ownership of intellectual property (or property generally) to non-human
creators? Alternatively, who should own the rights to an AI-generated invention
or creative work? The person who typed the prompt? The computer scientists? The
creators of the training data? No one?


