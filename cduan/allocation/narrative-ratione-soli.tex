\expected{pierson-v-post}

\having{keeble-v-hickeringill}{\expected{pierson-v-post}Lord Holt, who decided
\textit{Keeble}, is also a key---if perhaps slightly confusing---expositor of
the related and peculiarly English doctrine of \textit{ratione soli} (Latin for
``by reason of the soil''), also referred to in \textit{Pierson}.
}{}{}\textit{Ratione soli} is the principle that the right to take
possession of wild animals belongs to the owner of the land where the animal may
be found; thus title to any animals captured or killed on owned land
automatically vests in the landowner. The English rule is in stark opposition to
the civil (i.e., Roman) law rule, reflected in the Institutes of
Justinian,\footnote{\textsc{J. Inst.} 2.1.12. The \textit{Institutes} are a
portion of the massive codification of Roman law under Byzantine (Roman) Emperor
Justinian I: the \textit{Corpus Iuris Civilis}. The \textit{Corpus}, in turn, is
an important predecessor of most modern civil law systems, which prevail in
Continental European nations and many of their former colonies. Unlike
common-law systems, which prevail in England and most of its former colonies
(including the United States, with the exception of Louisiana), legal authority
in civil law systems derives not from caselaw, but from comprehensive statutory
codes. A primary distinction between common law and civil law systems is the
sharply diminished role of precedent in civil law adjudication. (Recall note
\ref{popov-precedent} on page \pageref{popov-precedent}, \textit{supra}.)}
which is that the captor of a wild animal acquires property rights in the animal
wherever captured, though he may be liable in trespass to the owner of the real
property on which the animal was pursued or taken. This distinction affects not
only the right to possession of the animal itself, but also the measure of
damages, because the damages from the trespass may be less than the value of the
animal.

A strong principle of \textit{ratione soli} was consolidated in mid-19th century
England as part of the class wars between the landed gentry---who passionately
defended game hunting as an exclusive sport for the aristocracy---and the
upwardly-mobile merchant classes and more desperate farmers and poachers---who
saw game as a token of luxury and a means of sustenance, respectively.
\textit{See generally} Chester Kirby, \emph{The English Game Law System}, 38
\textsc{Am. Hist. Rev.} 240 (1933). The aristocrats won a decisive victory in a
suit by a game merchant against certain servants of the Marquis of Exeter, who
had forcibly seized several dozen rabbits purchased by the merchant for resale,
on grounds that they had been poached from the Marquis's lands. Blades v. Higgs,
(1865) 11 Eng. Rep. 1474, 11 H.L.Cas. 621. The Law Lords ruled that wild animals
are the property of the owner of the land on which they are taken, and that the
Marquis's servants were therefore within their rights in repossessing the
rabbits.

\textit{Ratione soli} was initially rejected by the newly independent American
states, in favor of a rule of ``free taking.'' This made some sense in the
America of John Locke's imagination: a vast, naturally bountiful, largely
undeveloped, and sparsely populated continent. Moreover, ``[i]n the New World,
game was no sporting matter, but rather a source of food and clothing.'' Thomas
A. Lund, \textit{Early American Wildlife Law}, 51 \textsc{N.Y.U. L. Rev.} 703
(1976). Thus, for the first century of the new Republic's life, landowners for
the most part enjoyed no special privileges to wild animals on their otherwise
idle land; hunters were presumed to be free to enter or cross unenclosed and
undeveloped land in pursuit of game, even where that land was privately owned.
Landowners could defeat this presumption by posting notices of their intent to
exclude hunters at the boundaries of their property, but in practice posting was
uncommon and generally ineffective for large holdings in the wilds of the
frontier. \textit{Id.} at 712-14.

Over time, even the vast American continent saw its natural resources threatened
with depletion by overexploitation, and its lands subject to increased
development that conflicted with the free taking regime. Nevertheless, while a
small number of American cases adopted \textit{ratione soli} (\textit{see,
e.g.}, \emph{Rexroth v. Coon}, 23 A. 37 (R.I. 1885) (bees); \emph{Schulte v.
Warren}, 75 N.E. 783 (Ill. 1905) (fish)), the rule never took hold here as it
did in England. Today, wild animals are subject to a variety of state and
federal regulations that fairly comprehensively govern whether, when, and under
what circumstances they may be hunted or captured, on the theory that wildlife
is a common resource to be managed by the government for the benefit of the
people. \textit{See generally} Michael C. Blumm \& Lucus Ritchie, \textit{The
Pioneer Spirit and the Public Trust: The American Rule of Capture and State
Ownership of Wildlife}, 35 \textsc{Environ. L.} 673 (2005). But a majority of
states still allow licensed hunters to take or pursue game on unenclosed private
land unless the landowner has posted against hunting or trespassing. Mark R.
Sigman, Note, {Hunting and Posting on Private Land in America}, 54 \textsc{Duke
L. J.} 549, 558-68 (2004).

\expected{jacque-v-steenberg}

One possible virtue of the doctrine of \textit{ratione soli} is the same as the
virtue of the punitive damages award in \textit{Jacque v. Steenberg Homes}: it
may marginally discourage trespasses on land by those who would trespass for the
purpose of capturing wild animals. But at what cost? And do we really need
\textit{ratione soli} when, as \textit{Jacque} makes clear, punitive damages are
already available against trespassers? Or when\having{keeble-v-hickeringill}{,
as \textit{Keeble} makes clear,}{}{}
there are other legal remedies available against those who interfere with
landowners' efforts to exploit wild animals on their land? Is there any other
principled justification for either \textit{ratione soli} or free taking, or are
the rules merely sops to particular political interests? In light of all this
history, what do you think \textit{ought} to be the legal rights of landowners
with respect to wild animals that happen to be on their land? Why? Is there any
reason landowners should have a superior claim to anyone else?

