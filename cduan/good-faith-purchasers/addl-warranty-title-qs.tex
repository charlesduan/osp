\item Warranties of \emph{title} are distinct from warranties on the safety or
quality of the underlying resource. One can convey perfectly good title to a
house with a crumbling foundation. A way of thinking of the difference is that
warranties of title guarantee the right to exclude; the latter would be
warranties on the right to possess or use the property.

\defcase{engelhart-v-kramer}{
Engelhart v. Kramer, 570 N.W.2d 550 (S.D. 1997)
}


Traditionally, the rule was that property was sold as-is and it was the buyer's
responsibility to inspect before buying---hence the phrase \emph{caveat emptor},
or ``buyer beware.'' Today, many states impose mandatory disclosures on real
property sellers, or require new homebuilders to give a non-waivable ``warranty
of habitability.'' For more information on this, see the original Land
Transactions module of \emph{Open Source Property}, which discusses
\inline{engelhart-v-kramer}.

Which do you think is the better rule? To what extent can private insurance
solve the problems of disclosure?

\defcase{brush-grocery-v-sure-fine}{
Brush Grocery Kart, Inc. v. Sure Fine Market, Inc., 47 P.3d 680 (Colo. 2002)
}

\item Typically, there is a period of time between execution of a contract for
property sale and the ``closing'' in which the deed is actually conveyed. What
happens if the property is damaged during that interim ``executory period''?
Absent agreements otherwise, the doctrine of equitable conversion holds that the
contract renders the buyer the equitable owner of the land, such that the buyer
must pay for and take the property. This rule is very much in question across
the states. \sentence{see brush-grocery-v-sure-fine}.

Who do you think should bear liability during the executory period? Should it
depend on whether the buyer has possession during that period, as some courts
(including \inline{brush-grocery-v-sure-fine}) have held? What role does private
insurance play here as well?

