\expected{ucc-warranty-title}

\item What was given above was a statutory text.\footnote{More precisely, it is
a model statute that has been adopted in just about every state.} You need to
read these
\emph{extremely carefully}, because they are not like cases. A judicial opinion
contains both rules and explanatory reasoning to help readers understand the
rule. Plus, judges often like to repeat themselves, so legal rules may be
stated multiple times throughout an opinion. So even if you read quickly,
there's a good chance you'll pick up on the key doctrines at some point in an
opinion. Statutes, on the other hand, are typically all rule and no explanation.
Every word is significant.

When reading a statute, you have two main objectives. First, you need to
identify the elements; second, you need to determine the statute's mechanics.

\item \textbf{Elements}. These are key terms that require interpretation and
match fact patterns. They are like the elements of common-law rules, but there
is even less room for imprecision with statutory elements.

In some cases, the definition of a key term is given elsewhere in the statute,
perhaps in a ``definitions'' section. In that case, you'll need to repeat the
process because the definition text will have more elements to be interpreted.
In other cases, the statute will leave terms undefined. In that case, it is the
job of a court (and, by extension, the lawyers predicting what a court will do)
to ``construe'' the terms' meanings through the logic of the statutory text,
case law, and other arguments.

A helpful hint: elements are typically nouns or noun phrases, though not always.
Sometimes the noun phrase can be very long (say, if a statute recites a list of
things). In that case, it can be helpful to invent an abbreviation or two as a
placeholder for the element---just remember that the abbreviation is for your
personal use only, and you should always use the original statutory text in
writing.

\expected{intro-adverse-possession}

\item \textbf{Mechanics}. These are how the elements work together to produce
legal results, like liability or rights. Given a fact pattern, a court will
reach yes-no answers for each statutory element---either an element is present
in the facts, or
it isn't. Obviously, the yes-no answers will be the subject of intense debate
between the parties and their lawyers. But once a court arrives at an answer for
each element, the mechanics of the statute tell you the consequences. The
mechanics are the flowchart or decision process that turns elements into
results.

Common-law rules (think adverse possession, or tort negligence) typically have
simple mechanics: ``If A, B, and C, then D.'' The consequence is D, which
happens only if A, B, and C are true for a given set of facts. But statutes (and
some judicial rules) can be more intricate, with exceptions or multi-part
decision pathways.

Again, a helpful hint is that the words that join the noun phrases in a statute
will tend to define the mechanics. Prepositional phrases show how
elements are connected. Verbs will separate the requirements of the legal test
(the ``if'' part) from the consequences (the ``then'' part). Be aware, though,
that sometimes the mechanics of a statute are not explicit in its words---they
may be implied through the sentence structure or organization.

\item Let's try this out on the second sentence of \S~2-403(1). (The first
sentence is hopefully pretty obvious in meaning---if a person buys goods from
another, the buyer gets all the property rights of the seller unless the seller
gives them only a ``limited interest'' like a bailment.)

\expectnext{kotis-v-nowlin}

First, pull out the noun phrases:
\begin{enumerate}
\item person with voidable title
\item power to transfer a good title
\item voidable title (sometimes there's an element inside an element)
\item good title
\item good faith purchaser for value
\end{enumerate}
What do these phrases mean? We don't know yet. The case that follows will give
us some definitions, but let's see what we can discern from the statute alone.
The term ``good title'' probably means full ownership, just by its plain
meaning. A ``person with voidable title'' must be someone with some level of
ownership less than that. The term ``power to transfer a good title'' shows up
in the third sentence (``such power''), so that sentence will hopefully give us
some insight, both into what the power is and the nature of the person holding
such power, namely the ``person with voidable title.'' And the words of ``good
faith purchaser for value'' tell us a lot---the person must purchase the good,
pay value for it, and have ``good faith'' (whatever that means).

\begin{figure}
\begin{flowchart}
\blocks
node (vt) {Seller is ``person\\with voidable title''?}
node[below=of vt] (gfp) {Buyer is ``good faith\\purchaser for value''?}
node[right=of gfp] (gt) {Buyer gets\\``good title''}
node[left=of gfp] (def) {Apply general rule\\(sentence 1)}
;
\conn["Yes"] vt to gfp;
\hvconn["No" near start] vt to def;
\conn["Yes"] gfp to gt;
\conn["No" above] gfp to def;
\end{flowchart}
\caption{Flowchart of UCC \S~2-403(1), sentence 2.}
\label{f:ucc-2-403-flowchart}
\end{figure}

The next step (which, you'll notice, can be done even if you don't yet know what
the elements mean) is to work out the mechanics. The verb ``has'' connects the
person with voidable title to the power; the preposition ``to'' connects the
good title with the good-faith purchaser for value. In other words: \textbf{if}
A is a ``person with voidable title'' \textbf{and} B is a ``good faith purchaser
for value'', \textbf{then} A has the ``power to transfer a good title'' to B, so
B ends up with good title. Also, the first sentence of \S~2-403(1) is a general
rule, while the second sentence is more specific. This suggests that the second
sentence is an exception to the first. So if B is not a good faith
purchaser, then the first sentence's general rule applies (so B would get A's
voidable title, or whatever else A had). This is shown in
Figure~\ref{f:ucc-2-403-flowchart}.

\item You may have found this analysis overkill for a reasonably simple
sentence. But you will run into statutes with more elements and more intricate
mechanics. Rigorously follow this procedure, and you will have a strong grasp
for how any statute works and how to use it.

Try it out now. What's going on in the third sentence of \S~2-403(1)? Who is the
``purchaser''? (Hint: it's not the ``good faith purchaser'' of the second
sentence.) What is required to have ``such power,'' which we identified above as
``the power to transfer a good title''?
