\editrepofile{base}{recording-acts}{intro-recording-acts}

\replaceend{ Recording systems try to prevent some}{}

\endedit

Despite that other common law maxim \emph{caveat emptor}---let the buyer
beware---the law of property transfers offers buyers multiple protections
against fraud and mistakes by sellers. This chapter considers several such
protections.

First, there are rules protecting the so-called \term[good faith purchaser]{good
faith purchaser for
value} of property, who buys from someone with less than perfect title to that
property. For this doctrine, pay close attention to what makes a buyer a good
faith purchaser, and also to the circumstances that allow a good faith purchaser
to take title. It is hornbook law that ``a thief takes no title and can give
none,'' so can a good faith purchaser receive title from a thief? A forger? A
fraudster?

Second, real property title deeds can include \term[warranty of
title]{warranties of title} that
provide buyers with protection against defects in ownership. Here, consider
carefully what defects are covered, and what recourse the buyer will have in
case a problem arises.

Third, we will look at \term{recordation} of property ownership of real estate.
Having a public record of who owns what can be immensely helpful to buyers, but
it can also raise difficult questions if records conflict with each other.


