\reading{Iron Bar Holdings, LLC v. Cape}
\readingcite{131 F.4th 1153 (10th Cir. 2025)}

\opinion \textsc{Tymkovich}, Circuit Judge.

The American West contains millions of acres platted into alternating squares of public and private land in a manner resembling a checkerboard. The question presented is whether a private landowner can prevent a person from stepping across adjoining corners of federal public land—a technique called ``corner-crossing.''

Appellant Iron Bar Holdings, LLC, owns a checkerboarded ranch in south-central Wyoming. Enmeshed within its holdings are federal and state public plats. The only way to access the federal or state land, other than aircraft, is by corner-crossing.

\captionedgraphic{iron-bar-fig-1}{iron-bar-1}{The court's diagram of
corner-crossing.}

Anyone familiar with the game checkers can visualize this corner-crossing problem: to move diagonally across the board, a piece must momentarily occupy the space on and above the opponent's squares. If the opposing player could foreclose that move, the opponent would be unable to travel the board.

Iron Bar seeks to prevent elk hunters, like Appellees, from corner-crossing under the theory that diagonal moves on the checkerboarded land are a trespass. The district court granted Appellees access. While the dispute may seem trivial, at its core, it implicates centuries of property law and the settlement of the American West.

This case turns on the interplay of state and federal law enacted against the backdrop of private settlement of public lands and the property disputes that inevitably followed among rival interests. Over a century ago, the Supreme Court held that private landowners cannot erect barriers which bar complete access to public lands based on the 1885 Unlawful Inclosures Act. And the Tenth Circuit has interpreted the UIA to allow corner-crossing if access to public lands is otherwise restricted. Those cases control and require us to affirm the district court.



\readinghead{ I. Background}

Passing through Carbon County, Wyoming —a vast expanse of the ``Great American Desert'' in south-central Wyoming— few might peg it as a venue for conflict between the rights of private property and public access. Yet this county has been the epicenter of a 150-year conflict touching on the core of property law and, simultaneously, defining the American West. That conflict arose, in part, because of the overlapping history of laws and pioneer practices that governed the opening of the American frontier.



\readinghead{\textit{A. The Expansion of the American West}}

Present-day Wyoming has a complicated territorial history reflective of the broader American West. Prior to European and American settlers, Indian tribes—including the Shoshone, Crow, Arapaho, Comanche, Cheyenne, Ute, and Lakota— lived in the area for millennia, and continue to do so to this day.\readingfootnote{1}{\textit{Indigenous People in Wyoming and the West}, WYOHISTORY.ORG, \url{https://www.wyohistory.org/encyclopedia/topics/indigenous-people-wyoming-and-west} (last visited Dec. 12, 2024).}

For several centuries, claims to the area became trading stakes among European powers.\readingfootnote{2}{\textit{See} Alain A. Levasseur \& Jackie McCreary, \textit{Avant Propos (Introduction)}, 63 LA. L. REV. 933, 933 (2003).} First claimed by French explorers in the 1600s, the territory was secretly acquired by Spain in the 1700s before it was transferred back to France in 1800. Three years later, the United States purchased most of present-day Wyoming (everything east of the Continental Divide) from France as part of the ``Louisiana Purchase'' for the modern-day equivalent of \$.70 per acre.\readingfootnote{3}{\textit{See id.} at 933-34.}

No longer a remote pawn in European politics, American explorers came soon after. John Colter—a member of the famed Louis and Clark Expedition—completed the first American expedition across what would become Wyoming in 1807.\readingfootnote{4}{\textit{John Colter, the Phantom Explorer—1807-1808}, NAT'L PARK SERV. (Mar. 5, 2004), \url{https://web.archive.org/web/XXXXXXXXXXXXXX/} \url{http://www.cr.nps.gov/history/online\_books/grte1/chap3.htm.}} Prior to that, the land had been the domain of French-Canadian trappers, with toponyms like ``Teton'' and ``La Ramie'' reflecting their influence.\readingfootnote{5}{\textit{See} Jim Hardee, \textit{The Fur Trade in Wyoming}, WYOHISTORY.ORG (Nov. 8, 2014), \url{https://www.wyohistory.org/encyclopedia/fur-trade-wyoming.}} In 1812, Robert Stuart charted South Pass, a wagon-friendly route across the Continental Divide that the Oregon Trail would later follow.\readingfootnote{6}{\textit{See generally} ROBERT STUART, THE DISCOVERY OF THE OREGON TRAIL: ROBERT STUART'S NARRATIVES OF HIS OVERLAND TRIP EASTWARD FROM ASTORIA IN 1812-13 (Philip Ashton Rollins ed., New York: Charles Scribner's Sons 1935).}

But even after the Louisiana Purchase, and until the formal creation of the Wyoming Territory in 1868, Wyoming's land was at various times claimed by ``Great Britain, France, Spain, Mexico, and the one-time Republic of Texas.''\readingfootnote{7}{Dick Blust, Jr., \textit{The Tri-Territory Site: Outpost of Invisible Empires}, WYOHISTORY.ORG (Dec. 22, 2022), \url{https://www.wyohistory.org/encyclopedia/tri-territory-site-outpost-invisible-empires.}} After the United States formally secured possession of the Pacific Northwest in 1848, it was organized into the Oregon Territory, and then annexed into the Washington Territory when Oregon became a state in 1853.\readingfootnote{8}{\textit{Id.}}

 The southwest corner of present-day Wyoming became part of the United States with the 1848 Mexican Cession and treaty of Guadalupe Hidalgo.\readingfootnote{9}{\textit{See} \textit{Buford v. Houtz}, 133 U.S. 320, 331, 10 S.Ct. 305, 33 L.Ed. 618 (1890).} The land changed hands between different named territories, until the Wyoming Territory was established in 1868, finally forming the state's modern geographic boundaries. The Wyoming Territory was admitted as the 44th state on July 10, 1890.\readingfootnote{10}{Phil Roberts, \textit{Wyoming Becomes a State: The Constitutional Convention and Statehood Debates of 1889 and 1890 and Their Aftermath}, WYOHISTORY.ORG (NOV. 8, 2014), \url{https://www.wyohistory.org/encyclopedia/wyoming-statehood.}}



\readinghead{\textit{B. Genesis of the Checkerboard}}

To help catalyze settlement, Congress began exploring ways to connect the American continent via a transcontinental railroad in the 1850s and 1860s. Its motivations were complex and driven by the belief that it was America's Manifest Destiny to expand westward.\readingfootnote{11}{\textit{See generally} STEPHEN E. AMBROSE, NOTHING LIKE IT IN THE WORLD: THE MEN WHO BUILT THE TRANSCONTINENTAL RAILROAD 1863-1869 (Simon \& Schuster Paperbacks 2005) (2000).}

The leading business figures of the era, however, considered a transcontinental railroad economically unfeasible. When private capital failed to materialize, Congress considered public funding. But public funding was a political and legal hot potato, with Justice Story (among others) arguing that funding ``internal improvements'' such as railroads was not an enumerated constitutional power of Congress, and so was constitutionally impermissible. \textit{Leo Sheep Co. v. United States}, 440 U.S. 668, 672, 99 S.Ct. 1403, 59 L.Ed.2d 677 (1979) (describing history of Wyoming settlement).

Congress sought to resolve the issue by implementing a checkerboard land-grant scheme. \textit{Id.} First, Congress created a 10-mile corridor extending in both directions from the railroad's proposed route (this would later be extended to 20 miles). Surveyors then platted this corridor into 6-mile by 6-mile squares called ``townships.'' Each township was then subdivided into 36 one-square mile (640 acre) ``sections,'' which were numbered 1-36 (beginning in the northeastern-most section and snaking west).

% Figure not included here

Aplt. Br. at 6.\readingfootnote{12}{This scheme was not unique to transcontinental railroad land grants, but it goes back to at least the Land Ordinance of 1785, which itself ``represented an amalgam of the colonial experience and ideals.'' JONATHAN HUGHES, THE GREAT LAND ORDINANCES: COLONIAL AMERICA'S THUMBPRINT ON HISTORY, THE ECONOMY OF THE OLD NORTHWEST (David C. Klingaman \& Richard K. Vedder eds., Ohio Univ. Press 1987).} Unlike many areas populated earlier in our nation's history, these regions were often surveyed, subdivided, and sold before they were fully settled.

Congress, through legislation, granted the odd-numbered squares to railroad companies corresponding to each mile of track they laid. While the odd-numbered squares went to the railroads, Congress  retained the even-numbered squares for the federal government.

\begin{quote}
[T]his was the beginning of a practice to be followed in most future instances of granting land for the construction of specific internal improvements: donating alternate sections or one half of the land within a strip along the line of the project and reserving the other half for sale.... In later donations the price of the reserved sections was doubled so that it could be argued, as the Congressional Globe shows ad infinitum, that by giving half the land away and thereby making possible construction of the road, canal, or railroad, the government would recover from the reserved sections as much as it would have received from the whole.
\end{quote}
\textit{Leo Sheep}, 440 U.S. at 672-73, 99 S.Ct. 1403 (citation omitted).

The economic incentives worked. Construction of the Transcontinental Railroad began in July 1865, triggering a race between the Union Pacific Railroad and Central Pacific Railroad to lay the most track and thereby acquire the most land. \textit{Id.}

If it considered access issues within the checkerboard it had created, Congress seemingly assumed ``the ordinary pressures of commercial and social intercourse would work itself into a pattern of access roads.'' \textit{Id.} at 686, 99 S.Ct. 1403. That was the case in some places; arid south-central Wyoming, however, was not one of them.



\readinghead{\textit{C. Early Checkerboard Land Disputes}}

Ranchers were among the first permanent eastern settlers to arrive in meaningful numbers to the western territories. \textit{See id.} at 683, 99 S.Ct. 1403. As the United States acquired territories—including the Wyoming Territory—land not yet placed into private ownership was publicly owned and freely available for livestock grazing. Until the introduction of barbed wire in the 1870s, it was more practical to fence grazing livestock out of developed land rather than to fence them in. From this custom birthed the West's open range system, which ``dominated stock-raising in Wyoming.''\readingfootnote{13}{``Open-range ranching depended on access to large, unfenced areas of free public grazing land. Cattle were allowed to graze year-round with little herd maintenance other than spring and fall roundups.'' David Johnson, \textit{The Frank Nevin Homestead, Carbon County, Wyoming}, WYOHISTORY.ORG (July 21, 2024), \url{https://www.wyohistory.org/encyclopedia/frank-nevin-homestead-carbon-county-wyoming.}}

Commercial ranchers were not the only ones competing for land. To encourage westward expansion, Congress passed a series of homesteading acts. The Preemption Act of 1841 allowed any American citizen or freed slave to claim up to 160 acres of federal land if they continued to reside on or improve the land for at least five years. Later, the Homestead Act of 1862 streamlined the process of claiming federal lands. ``Enticed by the federal government's offer of cheap and plentiful land, thousands began pouring into Wyoming, settling the land and tilling the soil.''\readingfootnote{14}{NEIL GORSUCH, A REPUBLIC, IF YOU CAN KEEP IT 238 (2019).}

This set the stage for conflict between landowners and incoming pioneers over meager pasturage and limited water. \textit{See} \textit{Pub. Lands Council v. Babbitt}, 529 U.S. 728, 731, 120 S.Ct. 1815, 146 L.Ed.2d 753 (2000). The competition ``was intensified by the arrival of sheep in the 1870s.'' \textit{Id.} at 732, 120 S.Ct. 1815. These conflicts culminated in the legendary ``range wars,'' which pitted cattlemen against sheep herders, ranchers against homesteaders, and  ``those who fenced [against] those who cut fences to protect an open range.'' \textit{Id.}

To counter such challenges, and to maintain profitability and their dominion over the ranges, cattlemen began using homestead and pre-emption laws to gain control of water sources. \textit{Leo Sheep}, 440 U.S. at 683-84, 99 S.Ct. 1403. ``With monopoly control of such sources, the cattlemen found that ownership over a relatively small area might yield effective control of thousands of acres of grassland.'' \textit{Id.} at 683, 99 S.Ct. 1403. ``Another exclusionary technique was the illegal fencing of public lands which was often the product of the checkerboard pattern of railroad grants.'' \textit{Id.} These ``cattle barons [did not] own the range, but they often acted like they did. Many were rich eastern heirs or sons of British aristocrats more in love with the myth of the cowboy than the reality of the cow.''\readingfootnote{15}{GORSUCH, \textit{supra} note 14, at 238.}

Congress set out to remedy the ``matter of great public complaint that cattle companies have unlawfully appropriated and even inclosed by fences great areas of the public domain to the deprivation of the rights of citizens to enter such lands under the laws of the United States, to the obstruction of public travel.'' 15 CONG. REC. H., 4,773 (daily ed. June 3, 1884). To ``prevent the absorption and ownership of vast tracts of our public domain'' by the cattle barons, \textit{id.} at H. 4,782, Congress passed the Unlawful Inclosures Act of 1885. 43 U.S.C. \S~1061 et seq. In essence, the UIA was designed to harmonize public access to the public domain with adjacent private landholdings.

Settlement never came to much of arid Wyoming in the way Congress envisioned when it created the checkerboard, and the United States eventually deeded its unclaimed lands to the American public, rather than selling it. 43 U.S.C. \S~1701(a)(1).

We recount all this history to explain how and why land ownership in Wyoming came to resemble a patchwork quilt, where private and public lands are stitched together. Much like the land, the modern legal doctrine has also been stitched together from multiple doctrines. Today, the federal public lands are managed and administered by agencies such as the Bureau of Land Management. And some of the privately held squares once granted to the Union Pacific Railroad have been passed to a new owner: Iron Bar Holdings, LLC.



\readinghead{\textit{D. Iron Bar's Ranch}}

Towering above the vast openness of this landscape, Carbon County's Elk Mountain ``stands as a beacon above the surrounding terrain.'' Aple. Br. at 7 (quoting MARK E. MILLER, BIG NOSE GEORGE: HIS TROUBLESOME TRAIL 40 (High Plains Press 2021)). And overlaying Elk Mountain is the checkerboard ownership of land.

Iron Bar's ranch spans 50 square miles. Interspersed within its holdings are 27 federal and state public parcels totaling 11,000 acres. Except for the points where the public corners touch, most of these parcels are completely enclosed by Iron Bar's private land. The public lands are valuable because Elk Mountain is appropriately named: its lands are full of its namesake game, and hunting by the public is authorized by the BLM, making it a desirable destination for elk hunting.



\readinghead{\textit{E. The 2020 Hunt}}

In the fall of 2020, Bradley Cape, Zachary Smith, and Phillip Yeomans traveled from Missouri to Elk Mountain to hunt elk. Upon arriving in Wyoming, the Hunters drove down the public Rattlesnake  Pass Road to BLM Section 14, where they parked and set up their camp. Using onX Hunt—a GPS mapping tool that helps users find property lines and determine land ownership—they navigated to the corners of public land overlaying Elk Mountain. These corners are physically denoted by a steel United States Geological Survey marker cap driven into the ground. Once at the cap, the Hunters ``corner-crossed'' and stepped directly from the corner of one public parcel to the corner of the other. The Hunters never made contact with the surface of Iron Bar's land, but they did momentarily occupy its airspace.

Over the next several days, they hunted on public parcels—Sections 24, 26, 30, and 36—denoted on the following map [shown in Figure~\ref{f:iron-bar-3}].\readingfootnote{16}{This map was used by the district court. \textit{Iron Bar Holdings, LLC v. Cape}, 674 F. Supp. 3d 1059, 1064 (D. Wyo. 2023). Iron Bar owns Sections 13 and 23, and all sections labeled IBH. \textit{Id.} The Hunters' tent is indicated in Section 14. The dropped pins represent corner-crossing points.}

\captionedgraphic{iron-bar-fig-3}{iron-bar-3}{Map from the court's opinion.}

Iron Bar is not friendly to corner-crossers. In seeking to prohibit corner-crossing, Iron Bar had erected signposts over the United States Geological Survey marker  denoting the corner of Sections 13, 14, 23, and 24 [shown in Figure~\ref{f:iron-bar-4}].

\captionedgraphic{iron-bar-fig-4}{iron-bar-4}{No trespassing signs picture from
the court's opinion.}


\textit{Iron Bar}, 674 F. Supp. 3d at 1065. There were no other posts, fencing, or buildings within a quarter mile of the corner. \textit{Id.}

The Hunters could not fit between the signposts and under the chain to corner-cross, but they were undeterred by this odd barricade: ``one by one, each grabbed one of the steel posts and swung around it, planting their feet only'' on Sections 14 and 24, but passing through the airspace above Iron Bar's Sections 23 and 13. \textit{Id.} (cleaned up). There is no showing that the Hunters did any damage to Iron Bar's property. \textit{Id.}

But the Hunters would soon discover that this was not Iron Bar's only corner-crossing deterrence strategy.

\begin{quote}
Prior to 2020, [Iron Bar] instituted an ongoing practice of having its employees confront or interact with a ``suspected trespasser'' found on or near [its] property, even if the person was found while on public land. The suspected trespasser is instructed to leave, but if they resist, [Iron Bar] will contact local law enforcement, including the Wyoming Game \& Fish Department, to seek a criminal trespass citation or other prosecution. And if in [Iron Bar's] view law enforcement takes insufficient action on the matter, [it] will continue to contact law enforcement to push the matter and will  also contact the local prosecutor's office to request criminal prosecution.
\end{quote}
\textit{Id.} at 1066.

Iron Bar learned of, but did not approve of the Hunters' presence. Consistent with its policy, Iron Bar's property manager found the Hunters on Elk Mountain public land and requested that they leave the area. The Hunters refused, so the manager contacted law enforcement. The responding sheriff, however, did not issue a warning or citation after the Hunters explained that they had merely corner-crossed. The Hunters completed their hunting trip and returned home without further incident.



\readinghead{\textit{F. The 2021 Hunt}}

The Hunters\readingfootnote{17}{The Hunters were joined by John Slowensky. This opinion refers to all four individual defendants as the Hunters or Defendants.} returned to the area in 2021. This time, they brought a steel A-frame ladder to avoid even touching Iron Bar's signposts [as shown in Figure~\ref{f:iron-bar-4}].

\captionedgraphic{iron-bar-fig-4}{iron-bar-4}{Photograph of the ladder the
defendants used, as shown in the court's opinion.}

\textit{Id.} at 1066-67.

Iron Bar's staff proved much more inhospitable this time around. The property manager and another employee confronted the Hunters multiple times. They also interfered with the Hunters' activities by driving motorized vehicles across public parcels to scare away game. As in 2020, there is no evidence the Hunters made physical contact with or damaged Iron  Bar's property. \textit{Id.} at 1069. When the Hunters refused to leave, the manager contacted the Wyoming Game and Fish Department and the local sheriff's office. Both refused to take action. \textit{Id.} at 1068-69.

The manager resorted to contacting the local prosecuting attorney's office, who agreed to prosecute the Hunters for criminal trespass. That office instructed the sheriff's office to write the Hunters citations for criminal trespassing, and directed the Wyoming Game and Fish Department to instruct the Hunters to leave and not reenter the public lands at issue. The Hunters went all the way to a jury trial on the Wyoming criminal trespass charges, but were ultimately acquitted.

That same day, Iron Bar served the Hunters with a lawsuit for \textit{civil} trespassing, alleging \$9 million in damages owing to alleged diminution of its property value. Following discovery, the parties cross-moved for summary judgment. The district court denied Iron Bar's motion and granted the Hunters' motion as to ``all claims of trespassing involving [Appellant]s' corner-crossing.'' \textit{Id.} at 1080. In reaching this outcome, the district court held ``corner-crossing on foot in the checkerboard pattern of land ownership without physically contacting private land and without causing damage to private property does not constitute an unlawful trespass.'' \textit{Id.} at 1076-77. Iron Bar appeals that holding.



\readinghead{II. Analysis}



\readinghead{\textit{A. Standard of Review}}

We review the grant of summary judgment de novo. \textit{Adler v. Wal-Mart Stores, Inc.}, 144 F.3d 664, 670 (10th Cir. 1998). If there is no genuine issue of material fact, we determine whether the district court correctly applied the substantive law. \textit{Id.} If there is a genuine issue of material facts, we reverse or remand for further proceedings. \textit{Id.}



\readinghead{\textit{B. Airspace Ownership}}

Iron Bar asserts a property right to the airspace above its land, and the corresponding right to exclude corner-crossers from that airspace.

In considering Iron Bar's argument, we begin with the premise that the ``protection of private property is indispensable to the promotion of individual freedom.'' \textit{Cedar Point Nursery v. Hassid}, 594 U.S. 139, 147, 141 S.Ct. 2063, 210 L.Ed.2d 369 (2021) (``Property must be secured, or liberty cannot exist.'' (quoting JOHN ADAMS, \textit{Discourses on Davila, in} 6 THE WORKS OF JOHN ADAMS 223, 280 (Charles Francis Adams ed., 1851))); \textit{see also} \textit{Murr v. Wisconsin}, 582 U.S. 383, 394, 137 S.Ct. 1933, 198 L.Ed.2d 497 (2017) (``Property rights are necessary to preserve freedom, for property ownership empowers persons to shape and to plan their own destiny in a world where governments are always eager to do so for them.'').

Despite the importance of property rights to our system of ordered liberty, they ``are not created by the Constitution.'' \textit{Bd. of Regents v. Roth}, 408 U.S. 564, 577, 92 S.Ct. 2701, 33 L.Ed.2d 548 (1972). ``Rather they are created and their dimensions are defined by existing rules or understandings that stem from an independent source such as state law—rules or understandings that secure certain benefits and that support claims of entitlement to those benefits.'' \textit{Id.} ``We thus must turn to state law'' and ``existing rules or understandings'' to assess ``the scope of property rights in land ownership.'' \textit{Jordan-Arapahoe, LLP v. Bd. of Cnty. Comm'rs}, 633 F.3d 1022, 1025-26 (10th Cir. 2011).



\readinghead{\textit{1. The Right to Exclude}}

The right to exclude has long been a core property right. And indeed, subject to  aircraft flight, Iron Bar does own the airspace above its land. \textit{See} WYO. STAT. \S~10-4-302 (1977) (``The ownership of the space above the lands and waters of this state is declared to be vested in the several owners of the surface beneath subject to the right of flight''). To better understand this case, we must briefly address whether the right to exclude includes airspace rights.

The right to exclude from airspace is a modern embodiment of the ``\textit{ad coelum}'' doctrine—the right to own your airspace. Early versions of this doctrine are credited to the Italian legal scholar, Accursius, who in the 13th century proclaimed ``\textit{cujus est solum, ejus est usque ad coelum et ad inferos}''—``[w]hoever owns the soil owns everything up to the sky and down to the depths.''\readingfootnote{18}{John G. Sprankling, \textit{Owning the Center of the Earth}, 55 UCLA L. REV. 979, 992 (2008).} Accursius's son, Franciscus, may have brought the principle to England when he traveled there with King Edward I upon his return from the Crusades.\readingfootnote{19}{Andrea B. Carroll, \textit{Examining a Comparative Law Myth: Two Hundred Years of Riparian Misconception}, 80 TUL. L. REV. 901, 916 (2006).} Nearly three centuries later, Sir Edward Coke cemented the principle in English law when he wrote: ``And lastly, the earth hath in law a great extend upwards, not only of water as hath been said, but of aire, and all other things even up to heaven, for \textit{cujus est solum est usque ad coelum}, as it is holden.''\readingfootnote{20}{\textit{Bury v. Pope}, 1 Cro. Eliz. 118, 78 Eng. Rep. 375 (Ex. 1587) (ascribing the maxim to the time of King Edward I).} Thus came the notion that a landowner's right extended down to the center of the earth and upward toward infinity. And with that right came the right to exclude and the law of trespass.

With English colonization, the doctrine came to America. \textit{See} \textit{Pa. Coal Co. v. Mahon}, 260 U.S. 393, 419, 43 S.Ct. 158, 67 L.Ed. 322 (1922) (``The estate of an owner in land is grandiloquently described as extending \textit{ab orco usque ad coelum.}''). From that point, the common law uniformly affirmed the principle in holding that ``a trespass may be committed on, beneath, or above the surface of the earth.'' RESTATEMENT (SECOND) OF TORTS \S~159 (1965). For example: ``A extends his arm over the boundary fence between A's land and B's land. A is a trespasser.'' \textit{Id.} at cmt. f, illus. 3.

Practical concerns over \textit{ad coelum} bubbled to the fore when air travel forced a reconsideration of the doctrine. \textit{United States v. Causby}, 328 U.S. 256, 260-61, 66 S.Ct. 1062, 90 L.Ed. 1206 (1946) (``It is ancient doctrine that at common law ownership of the land extended to the periphery of the universe.... But that doctrine has no place in the modern world,'' and ``[c]ommon sense revolts at the idea.''). In addressing the issue, the Supreme Court acknowledged and determined:

\begin{quote}
if the landowner is to have full enjoyment of the land, he must have exclusive control of the immediate reaches of the enveloping atmosphere. Otherwise buildings could not be erected, trees could not be planted, and even fences could not be run. The principle is recognized when the law gives a remedy in case overhanging structures are erected on adjoining land. The landowner owns at least as much of the space above the ground as [he] can occupy or use in connection with the land. The fact that he does not occupy it in a physical sense—by the erection of buildings and the like—is not material.
\end{quote}
 \textit{Id.} at 264 (internal citation and footnote omitted).\readingfootnote{21}{We note that the current draft of the Restatement of Property attempts to reconcile the practical aspects of \textit{ad coelum} against its historical brightline antecedent, but provides two somewhat contradictory approaches. \textit{See} RESTATEMENT (FOURTH) OF PROP., \S~1.2A: TRESPASS ABOVE THE SURFACE OF THE LAND (AM. L. INST., Tentative Draft, 2023). On one hand, it explains that ``a single flight of a kite over the zone of actual possession would lead to trespass liability \textit{only if it interfered substantially with the possessor's use and enjoyment of the subjacent space.}'' \textit{Id.} at 4 (emphasis added). On the other hand, it finds that ``[i]nvasions of airspace at low levels are trespasses ... just as they are at the surface, even if the low-level aerial intrusion does not interfere with the possessor's uses. \textit{The intrusion into space actually possessed is enough.}'' \textit{Id.} at 7-8 (emphasis added).} At the same time, the Supreme Court recognized that ``immediate and direct'' intrusions ``subtract from the owner's full enjoyment of the property and ... limit his exploitation of it.'' \textit{Id.} at 265, 66 S.Ct. 1062. Examples of such diminishment include impeding a landowner's ability to carry out vertical construction, \textit{see id.}, military aircraft ``grazing the treetops and terrorizing the poultry,'' \textit{Cedar Point Nursery}, 594 U.S. at 150, 141 S.Ct. 2063 (citing \textit{Causby}, 328 U.S. at 259, 66 S.Ct. 1062), or coastal defense cannons passing overhead, \textit{Portsmouth Harbor Land \& Hotel Co. v. United States}, 260 U.S. 327, 330, 43 S.Ct. 135, 67 L.Ed. 287 (1922).\readingfootnote{22}{There are few if any cases applying this principle absent some physical injury to the trespass. Iron Bar points to only one case where no injury occurred: \textit{Koenig v. Aldrich}, 394 Wis.2d 187, 949 N.W.2d 882 (Wis. Ct. App. Aug. 4, 2020). But that case is not only factually distinguishable, it is a state court decision outside the Tenth Circuit.} Moreover, it held that ``invasions'' of low-level airspace ``are in the same category as invasions of the surface.'' \textit{Cedar Point Nursery}, 594 U.S. at 150, 141 S.Ct. 2063 (citing \textit{Causby}, 328 U.S. at 265, 66 S.Ct. 1062). If airspace invasions are the same as surface invasions, then the right to exclude —``one of the most treasured rights of property''—must apply to it as well. \textit{Id.} at 149, 141 S.Ct. 2063 (quotation marks omitted).

While sometimes inconvenient for trespassers, that broad conception of the right to exclude makes sense. Without it, property owners could not control the airspace over their property—meaning neighbor's trees could overgrow, eaves could overhang, drones could hover, and the government could build an overpass or fire artillery barrages over your house—all without injunctive recourse or compensation.



\readinghead{\textit{2. Wyoming Property Law}}

Bearing these traditional rules in mind, we next consider whether Wyoming would treat the Hunters' intrusion as an actionable civil trespass under state law. Wyoming courts have not spoken directly on this issue, so we must make an \textit{Erie}-guess\readingfootnote{23}{The \textit{Erie} doctrine mandates that federal courts apply state substantive law when resolving disputes not directly implicating a federal question. \textit{Erie R.R. Co. v. Thompkins}, 304 U.S. 64, 58 S.Ct. 817, 82 L.Ed. 1188 (1938).} as to how the Wyoming Supreme Court would rule. \textit{Pehle v. Farm Bureau Life Ins.}, 397 F.3d 897, 901 (10th Cir. 2005). We conclude that Wyoming would deem the Hunters' corner-crossing an actionable civil trespass.

First, Wyoming adopted the common law. WYO. STAT. \S~8-1-101 (1977) (``The common law of England'' is ``considered as of full force''). As noted above, common law has deemed an incursion into the immediately enveloping atmosphere, however fleeting, a trespass. No Wyoming authority suggests its conception of civil trespass has  changed.\readingfootnote{24}{As the district court recognized, the Wyoming legislature passed a statute which, by its plain text, decriminalizes the Hunters' conduct during this lawsuit's pendency. \textit{See} \textit{Iron Bar}, 674 F. Supp. 3d at 1076 (citing WYO. STAT. \S~23-3-305(b) (2023) (titled, in part, ``entering or traveling through private property without permission'')). But decriminalization does not change the nature of a civil trespass and does not moot Iron Bar's civil injunctive claim.}

Second, Wyoming's ``Ownership of Space'' statute provides that ``[t]he ownership of the space above the lands and waters of this state is declared to be vested in the several owners of the surface beneath subject to the right of [aircraft] flight....'' WYO. STAT. \S~10-4-302. Beyond this statute, Wyoming imposes no upward limits on a surface owner's ownership of superjacent airspace. In short, the Hunters' stepping through Iron Bar's airspace would be a civil trespass.

Although we conclude that Wyoming would deem the Hunters' corner-crossing a civil trespass, that does not end the inquiry. We must consider whether federal law ratifies corner-crossing where state law would otherwise prohibit it.



\readinghead{\textit{C. Federal Law and the Checkerboard}}

We conclude the district court did not err in dismissing Iron Bar's claims despite Wyoming civil trespass law. The UIA and case law interpreting it have overridden the state's civil trespass regime in this context.



\readinghead{\textit{1. The Unlawful Inclosures Act (1885)}}

As discussed above, the UIA was passed to harmonize the rights of private landowners and those accessing public lands. The UIA, 43 U.S.C. \S\S~1061-1066, declares ``[a]ll inclosures of any public lands ... to be unlawful.'' \textit{Id.} The Act, accordingly, prohibits:

\begin{quote}
the maintenance, erection, construction, or control of any such inclosure ...; and the assertion of a right to the exclusive use and occupancy of any part of the public lands of the United States in any State or any of the Territories of the United States, without claim, color of title, or asserted right as above specified as to inclosure, is likewise declared unlawful, and prohibited.
\end{quote}
\textit{Id.} The UIA also restricts obstruction of settlement on or transit over public lands:

\begin{quote}
No person, by force, threats, intimidation, or by any fencing or inclosing, or any other unlawful means, shall prevent or obstruct, or shall combine and confederate with others to prevent or obstruct, any person from peaceably entering upon or establishing a settlement or residence on any tract of public land subject to settlement or entry under the public land laws of the United States, \textit{or shall prevent or obstruct free passage or transit over or through the public lands}....
\end{quote}
\textit{Id.} \S~1063 (emphasis added).

In short, the two sections together provide that any inclosure of public land is prohibited, and no one may completely prevent or obstruct another from peacefully entering or freely passing over or through public lands. \textit{Id.} \S\S~1063, 1066.

We ask then, what is an \textit{inclosure}? The textual inquiry begins with dictionary definitions, so we start there. \textit{See} \textit{Wis. Cent. Ltd. v. United States}, 585 U.S. 274, 277, 138 S.Ct. 2067, 201 L.Ed.2d 490 (2018) (``[O]ur job is to interpret the words consistent with their ordinary meaning ... at the time Congress enacted the statute.'' (citation and internal quotations omitted)). Black's Law Dictionary contemporaneously defined ``inclosure'' as ``the act of freeing land from rights of common, commonable  rights, and generally all rights which obstruct cultivation and the productive employment of labor on the soil.''\readingfootnote{25}{This definition also accords with the traditional Lockean conception of ``common'' property. JOHN LOCKE, \textit{Of Property, in} SECOND TREATISE ON GOVERNMENT 34 CH. V (London 1690) (``It is true, in land that is common in England or any other country, where there are plenty of people under government who have money and commerce, no one can enclose or appropriate any part without the consent of his fellow-commoners; because that is left common by compact—\textit{i.e.}, by the law of the land, which is not to be violated.'').} \textit{Inclosure}, BLACK'S LAW DICTIONARY (1891).\readingfootnote{26}{Variations of the term were also similarly defined. \textit{See e.g., Inclosed Lands}, BLACK'S LAW DICTIONARY (1891) (``Lands which are actually inclosed and surrounded with fences.'').}

Iron Bar argues the district court erred in finding for the Hunters because the term ``inclosure'' is limited to fenced-in tracts of land. While a fence may be the most common way of creating an inclosure, the term's definition does not limit the meaning in such a manner. Importantly, the UIA's text makes plain in two ways that inclosure does not refer solely to physical fencing. First, \S~1063 explicitly prohibits obstructing ``transit over public lands ... by force, threats, intimidation, or by any fencing \textit{or inclosing.}'' 43 U.S.C. \S~1063 (emphasis added). If ``fencing'' was coextensive with ``inclosing'' the statute would not include both ``fencing \textit{or} inclosing.'' \textit{See id.}; \textit{OXY USA, Inc. v. Babbitt}, 268 F.3d 1001, 1006 (10th Cir. 2001). Second, \S~1061 makes clear that the statute applies to ``all inclosures of any public land,'' not just those done through fencing. 43 U.S.C. \S~1061. As Wyoming territorial justices observed long ago, ``[t]he fence is made for beasts; the law is made for man.'' \textit{United States v. Douglas-Willan Sartoris Co.}, 3 Wyo. 287, 22 P. 92, 97 (Wyo. 1889) (``[A] legal obstacle ... presents an impassable barrier.''). So a purely legal barrier erected by ``no trespassing'' signs—like a virtual wall—could be considered an inclosure under the UIA.



\readinghead{\textit{2. The Taylor Grazing Act (1934)}}

The UIA was passed as a measure of protection for the open range, allowing cowboys and herds to graze and roam freely across public lands. But by 1934, Congress saw a need to impose structure on the land that was being overgrazed and eroded. It therefore passed the Taylor Grazing Act ``to preserve the land and its resources from destruction or unnecessary injury, [and] to provide for the orderly use, improvement, and development of the range.'' 43 U.S.C. \S~315a. The Act authorizes the Secretary of the Interior to ``establish grazing districts ... of vacant, unappropriated, and unreserved lands from any part of the public domain of the United States.'' \textit{Id.} \S~315. Public lands are divided into grazing districts that can be opened, closed, and rotated through based on a permit and application system. In other words, the Taylor Grazing Act closed the open range.



\readinghead{\textit{D. Federal Courts and the Checkerboard}}

The Supreme Court and circuit courts, including the Tenth Circuit, have affirmed the UIA's prohibition on inclosures includes non-physical barriers. Later cases also demonstrate that the UIA was not abrogated or repealed by the Taylor Grazing Act. A brief survey of the cases, culminating in our 1988 decision in \textit{Bergen}, largely controls the outcome here.



\readinghead{\textit{1.} Buford v. Houtz, \textit{133 U.S. 320, 10 S.Ct. 305, 33 L.Ed. 618 (1890)}}

The Supreme Court addressed its first significant case on a checkerboard dispute  in 1890. \textit{Buford v. Houtz}, 133 U.S. 320, 10 S.Ct. 305, 33 L.Ed. 618 (1890). In a quarrel emblematic of the range wars of the era, cattlemen-plaintiffs pursued a trespass action against sheep herder-defendants to exclude the sheep from grazing on the public lands enmeshed within plaintiffs' holdings. The Court held for the defendants because

\begin{quote}
there is an implied license, growing out of the custom of nearly a hundred years, that the public lands of the United States, especially those in which the native grasses are adapted to the growth and fattening of domestic animals, shall be free to the people who seek to use them, where they are left open and uninclosed, and no act of government forbids this use.
\end{quote}
\textit{Id.} at 326.

\textit{Buford} did not discuss the UIA, which had been enacted five years earlier. Rather, \textit{Buford} is best understood as abrogating the common law proposition ``that every man is bound to keep his beasts within his own close'' in favor of the prevailing open range custom. \textit{Id.} at 331, 10 S.Ct. 305; \textit{see} \textit{Leo Sheep}, 440 U.S. at 688 n.24, 99 S.Ct. 1403 (finding \textit{Buford} to have affirmed the right of sheep herder-defendants to move their herds across private parcels ``to graze their sheep on even-numbered public lots'' without committing actionable trespass). But \textit{Buford} did affirm that appropriating public lands is presumptively unlawful. \textit{See} \textit{Buford}, 133 U.S. at 332, 10 S.Ct. 305 (finding ``no equity in the relief sought by [plaintiffs], which undertakes to deprive [defendants] of this recognized right to permit their cattle to run at large over the lands of the United States and feed upon the grasses found in them, while, under pretense of owning a small proportion of the land which is the subject of controversy, [plaintiffs] themselves obtain the monopoly of this valuable privilege.''). As the Court would later clarify, its decision to permit access ``also was influenced by the sheep ranchers' lack of any alternative.'' \textit{Leo Sheep}, 440 U.S. at 688 n.24, 99 S.Ct. 1403 (citing \textit{Buford}, 133 U.S. at 332, 10 S.Ct. 305).



\readinghead{\textit{2.} Camfield v. United States, \textit{167 U.S. 518, 17 S.Ct. 864, 42 L.Ed. 260 (1897)}}

Some years later, the Supreme Court interpreted the UIA for the first time. \textit{Camfield v. United States}, 167 U.S. 518, 17 S.Ct. 864, 42 L.Ed. 260 (1897). The government accused a rancher-defendant of building a fence that ``inclosed and appropriated to the[ir] exclusive use and benefit'' about ``20,000 acres of public lands'' in Colorado. \textit{Id.} at 519, 17 S.Ct. 864. This fence was ``entirely on odd-numbered sections... so as to completely inclose all of the government lands'' and effectively ``exclude the United States and all other persons except the defendants'' from it. \textit{Id.} The Court was tasked with considering whether: (1) the UIA was constitutional, and (2) the UIA allowed the government to order defendants to abate the fence. The Court answered ``yes'' to both.

The Court began by examining the UIA and explaining its genesis:

\begin{quote}
[T]he evil of permitting persons who owned or controlled the alternate sections to inclose the entire tract, and thus to exclude or frighten off intending settlers, finally became so great that [C]ongress passed the act of February 25, 1885, forbidding all inclosures of public lands....
\end{quote}
\textit{Id.} at 524-25. The Court held that, in passing the UIA, ``[C]ongress exercised its constitutional right of protecting the public lands from nuisances erected upon adjoining property.'' \textit{Id.} at 528, 17 S.Ct. 864. Put another way, the UIA constitutionally proscribes \textit{nuisances} effecting public land inclosures.  The Court found that Camfield's fence was ``clearly a nuisance'' considering ``the obvious purposes of this structure, and the necessities of preventing the inclosure of public lands.'' \textit{Id.} at 525, 17 S.Ct. 864. Faced with such a nuisance, it was ``within the constitutional power of [C]ongress to order its abatement, notwithstanding such action may involve an entry upon the lands of a private individual.'' \textit{Id.} In so doing, the Court also relied on its finding that inclosure is nothing more than an \textit{appropriation}\readingfootnote{27}{\textit{See id.} at 528 (``[W]hen, under the guise of inclosing his own land, he builds a fence which... can only have been intended to inclose the lands of the government, he is plainly within the statute, and is guilty of an \textit{unwarrantable appropriation} of that which belongs to the public at large.'') (emphasis added); \textit{id.} at 519, 17 S.Ct. 864 (``[W]hereby about 20,000 acres of public lands were inclosed and \textit{appropriated} to the exclusive use and benefit of the defendants.'') (emphasis added); \textit{id.} at 527, 17 S.Ct. 864 (``[W]e know of no reason why the policy, so long tolerated, of permitting the public lands to be pastured, may not be still pursued ... [by] other means adopted by which the fencing in and the \textit{exclusive appropriation} of such land shall be avoided.'') (emphasis added); \textit{id.} (``[I]t would be but a step further to claim that the defendants, by long acquiescence of the government in their \textit{appropriation} of public lands, had acquired a title to them as against every one except the government, and perhaps even against the government itself.'') (emphasis added).} and \textit{monopolization}\readingfootnote{28}{\textit{See id.} at 519 (``[T]he defendants ... with intent to encroach and intrude upon the lands of the United States in an illegal manner, and to \textit{monopolize the use} of the same for their own special benefit, did ... construct and maintain a fence which inclosed and included about 20,000 acres of the public domain[.]'') (emphasis added); \textit{id.} at 520-21, 17 S.Ct. 864 (Defendants ``denied that they had any intention of \textit{monopolizing} the even-numbered sections inclosed by said fence, or to exclude the public therefrom.'') (emphasis added); \textit{id.} at 524, 17 S.Ct. 864 (The government ``would be recreant to its duties as trustee for the people of the United States to permit any individual or private corporation to \textit{monopolize} them for private gain.'') (emphasis added).} of public lands.

\textit{Camfield} also confirmed that an inclosure in the context of the UIA is a term of art broader than fencing. In describing the landowner's conduct, the Court explained ``[t]he inconvenience, or even damage, to the individual proprietor does not authorize an act which is in its nature a purpresture of government lands.'' \textit{Id.} ``Purpresture'' is defined as the ``wrongful appropriation of land subject to the rights of others: an encroachment upon or enclosure of real property (as highways, sidewalks, or harbors) subject to common or public rights.'' \textit{Purpresture}, MERRIAM-WEBSTER, \url{https://www.merriam-webster.com/legal/purpresture} (last visited Dec. 12, 2024). Notably, this definition mirrors Black's Law Dictionary's definition of inclosure. \textit{Inclosure}, BLACK'S LAW DICTIONARY (1891).



\readinghead{\textit{3.} Mackay v. Uinta Dev. Co., \textit{219 F. 116 (8th Cir. 1914)}}

Approximately two decades later, the Eighth Circuit considered the UIA in resolving a similar land dispute. \textit{Mackay v. Uinta Dev. Co.}, 219 F. 116 (8th Cir. 1914). Mackay was a sheep rancher who wanted to traverse the Wyoming checkerboard. Uinta Development Company owned the private parcels and ``admitted [Mackay's] right as to the public domain, but warned him not to go over any of its lands on penalty of prosecution for trespass.'' \textit{Id.} at 118. When Uinta's threats proved ineffective, Uinta pressed trespass charges. \textit{Id.}\readingfootnote{29}{The case briefly detoured from the merits while a procedural and jurisdictional appeal was taken to the Supreme Court, and ultimately affirmed. \textit{See} \textit{Mackay v. Uinta Dev. Co.}, 229 U.S. 173, 33 S.Ct. 638, 57 L.Ed. 1138 (1913).}

 Relying on the UIA, the Eighth Circuit held for Mackay, explaining:

\begin{quote}
[i]f the position of the company were sustained, a barrier embracing many thousand acres of public lands would be raised, unsurmountable except upon terms prescribed by it. Not even a solitary horseman could pick his way across without trespassing. In such a situation the law fixes the relative rights and responsibilities of the parties. It does not leave them to the determination of either party. As long as the present policy of the government continues, all persons as its licensees have an equal right of use of the public domain, which cannot be denied by interlocking lands held in private ownership.
\end{quote}
\textit{Id.} At bottom, it held that ``Mackay was entitled to a reasonable way of passage over the uninclosed tract of land without being guilty of trespass.'' \textit{Id.} at 120. That court also affirmed that physical fences are not the only barriers prohibited by the UIA because the ``statute has been construed to prohibit every method that works a practical denial of access to and passage over the public lands.'' \textit{Id.} at 119.

Spotting the issue we face today, the Eighth Circuit rejected Uinta's argument that ``notwithstanding the statute, [the private owner] might accomplish the result prohibited by erecting fences on their own land not physically touching the public domain, and that any obstruction was an allowable incident of the exercise of a private right.'' \textit{Id.} Instead, the circuit court held that while ``[i]t is difficult to say that a man may not inclose his own land, regardless of the effect upon others[,]'' \textit{Camfield} ``has been recognized as sustaining the doctrine that wholesome legislation may be constitutionally enacted, though it lessens in a moderate degree what are frequently regarded as absolute rights of private property.'' \textit{Id.} (citation omitted).

The district court in the case before us primarily relied on \textit{Mackay} to craft its holding. We agree that \textit{Mackay} speaks to our issue, but \textit{Mackay} is merely persuasive, neither binding on us nor dispositive here. While \textit{Mackay} was brought in Wyoming district court, that was before the formation of our Circuit, and ``we have never held that the decisions of our predecessor circuit are controlling in this court.'' \textit{Est. of McMorris v. Comm'r}, 243 F.3d 1254, 1258 (10th Cir. 2001). Ultimately the decisions of one circuit are not binding on other circuits.



\readinghead{\textit{4.} McKelvey v. United States, \textit{260 U.S. 353, 43 S.Ct. 132, 67 L.Ed. 301 (1922)}}

Eight years after \textit{Mackay}, the Supreme Court again confronted the UIA. \textit{McKelvey v. United States}, 260 U.S. 353, 43 S.Ct. 132, 67 L.Ed. 301 (1922). \textit{McKelvey} has a fact pattern pulled straight from a spaghetti western. A group of cowboys confronted a rival group of shepherds moving sheep across federal public land in Idaho. The cowboys claimed the land was ``used as a cattle range and demanded that the sheep be not driven along that trail, but taken to the trail on the other side of the river, 4 or 5 miles away.'' \textit{Id.} at 354, 43 S.Ct. 132. When the shepherds refused, the cowboys ``began shooting'' and seriously injured one of the men. \textit{Id.} at 355, 43 S.Ct. 132 (noting the cowboys ``threatened to finish him, and did other things calculated to put all three [shepherds] in terror'').

In an opinion written by Justice Van Devanter,\readingfootnote{30}{Justice Van Devanter was no stranger to controversies of checkerboarded properties. As a practitioner in Wyoming, he represented the infamous cattle baron ``invaders'' in the Johnson County War. Phil Roberts, \textit{Lawyers and the Law in Early Wyoming}, WYOHISTORY.ORG (Sept. 15, 2024), \textit{\url{https://www.wyohistory.org/encyclopedia/lawyers-and-law-early-wyoming.}} And before his appointment to the Supreme Court, as judge on the Eighth Circuit, he dissented from a checkerboard decision against private landowners. \textit{Homer v. United States}, 185 F. 741, 747-48 (8th Cir. 1911) (Van Devanter, J., dissenting).} the Court affirmed the cowboys'  convictions for violating \S~1063 of the UIA. The Court explained the land is ``\,`free' passage or transit that is to be unobstructed.'' \textit{Id.} at 357, 43 S.Ct. 132. It continued that ``[p]assage or transit is free in the sense intended when it is open to all. When some withhold it from others, whether permanently or temporarily, it is not free.'' \textit{Id.} So the Court's position on the UIA remained unchanged.



\readinghead{\textit{5.} Leo Sheep Co. v. United States, \textit{440 U.S. 668, 99 S.Ct. 1403, 59 L.Ed.2d 677 (1979)}}

\textit{Leo Sheep} is the Supreme Court's most recent case on the UIA. \textit{Leo Sheep Co. v. United States}, 440 U.S. 668, 99 S.Ct. 1403, 59 L.Ed.2d 677 (1979). Like our dispute, \textit{Leo Sheep} occurred in Carbon County, Wyoming. \textit{Id.} at 677-78, 99 S.Ct. 1403. The disputed land is by the Seminoe Reservoir, ``an area that is used by the public for fishing and hunting.'' \textit{Id.} at 678, 99 S.Ct. 1403. ``Because of the checkerboard configuration, it is physically impossible to enter the Seminoe Reservoir sector from [the southeast] without some minimum physical intrusion upon private land.'' \textit{Id.} Apparently, this land was not homogeneously owned by Leo Sheep, but by a menagerie of different parties, who for a while, permitted the public to pass. \textit{Id.} Eventually the landowners grew weary of this arrangement and began ``denying access over their lands to the reservoir area or requiring the payment of access fees.'' \textit{Id.}

In response, the government decided to ``clear[] a dirt road ... to the reservoir across both public domain lands and fee lands of the Leo Sheep Co.'' without compensating the landowners. \textit{Id.} Leo Sheep moved to quiet title against the government. But the government argued that ``settled rules of property law''—including the easement by necessity doctrine\readingfootnote{31}{The Court described the easement by necessity doctrine as follows: ``Where a private landowner conveys to another individual a portion of his lands in a certain area and retains the rest, it is presumed at common law that the grantor has reserved an easement to pass over the granted property if such passage is necessary to reach the retained property.'' \textit{Leo Sheep}, 440 U.S. at 679, 99 S.Ct. 1403.} and the UIA—established an implicit easement to build a road. \textit{Id.} at 679, 99 S.Ct. 1403. The district court disagreed with the government and ruled in Leo Sheep's favor. The Tenth Circuit reversed by holding that Congress ``implicitly reserved an easement to pass over the odd-numbered sections in order to reach the even-numbered sections that were held by the [g]overnment.'' \textit{Id.} at 678, 99 S.Ct. 1403 (citing \textit{Leo Sheep Co.}, 570 F.2d at 881).

In the Supreme Court's framing, the issue was mundane: ``Whether the government has an implied easement to build a road across land that was originally granted to the Union Pacific Railroad under the Union Pacific Act of 1862.'' \textit{Id.} at 669, 99 S.Ct. 1403.

The Court held that the government does not have ``an implied easement to build a road across land'' in the checkerboard. \textit{Id.} The Court held that the easement by necessity doctrine is generally not available to the sovereign because the ``[g]overnment has the power of eminent domain.'' \textit{Id.} at 680, 99 S.Ct. 1403. As a result, the Court found the ``pertinent inquiry'' was whether Congress reserved the  right asserted by the government to build a road in the Union Pacific Act of 1862. \textit{Id.} at 680-81, 99 S.Ct. 1403. With this holding, \textit{Camfield} and the UIA were left intact because both were not ``of any significance in this controversy'' given that Leo Sheep's ``unwillingness to entertain a public road without compensation can[not] be a violation of'' the UIA. \textit{Id.} at 683, 685, 99 S.Ct. 1403.



\readinghead{\textit{6.} U.S. ex rel. Bergen v. Lawrence, \textit{848 F.2d 1502 (10th Cir. 1988)}}

Finally, we come to this Circuit's decision in \textit{Bergen. U.S. ex rel. Bergen v. Lawrence}, 848 F.2d 1502 (10th Cir. 1988), \textit{cert denied} 488 U.S. 980, 109 S.Ct. 528, 102 L.Ed.2d 560 (1988). At issue was a cattle rancher's ``antelope-proof'' fence that enclosed 15 sections of public land in south-central Wyoming. Lawrence had constructed the 28-mile fence ``entirely on private lands, except where it crosses the common corners of state and federal sections.'' \textit{Id.} at 1504. The government challenged the fence as an attempt to inclose federal public lands. The district court ordered removal or modification of the fence to BLM standards, which would ``allow free and unrestricted access by pronghorn antelope'' to their winter forage. \textit{Id.} We affirmed. Four major findings from \textit{Bergen} inform our decision today.

\textit{First}, the UIA remains good law. We rejected the claim that the UIA is inapplicable outside the open range context. We observed that ``the UIA remains federal law, and was amended in 1984 when Congress modified a procedural provision.'' \textit{Id.} at 1506. Accordingly, we ``refuse[d] to repeal the UIA by implication'' and therefore, ``[gave] effect to its provisions.'' \textit{Id.} (citation omitted).

We similarly affirmed the district court's finding that the UIA was not repealed by the Taylor Grazing Act, even though it replaced the open range system with one of federal grazing districts. Instead, we held that the provisions of the UIA and Taylor Grazing Act should be read together ``to give effect to each ... while preserving their sense and purpose.'' \textit{Id.} at 1510.

\textit{Second}, we affirmed that \textit{Leo Sheep} did not control outside the context of an implied easement. \textit{Id.} at 1506. The \textit{Bergen} plaintiff argued that \textit{Leo Sheep} mandated the ``UIA's purpose was to prevent the continuation of `range wars,' and that it should not be extended beyond this purpose.'' \textit{Id.} We affirmed the district court's analysis:

\begin{quote}
[t]hat is not what the Court meant. The UIA indeed was a response to the range wars, but nothing in the act or its history limits its application in such a manner. If the UIA was only meant for such a limited purpose, the Court would have said so in \textit{Camfield}, and Congress should have repealed it in 1934 when the Taylor Grazing Act was passed to end public land disputes.
\end{quote}
\textit{Id.} (quoting \textit{U.S. ex rel. Bergen v. Lawrence}, 620 F. Supp. 1414, 1419 (D. Wyo. 1985) (``Certainly Congress would not have amended the UIA in 1984 if the act was now useless.'')).

Moreover, we held that \textit{Leo Sheep} was limited in its application to the government's assertion of an implied easement, which is a permanent, physical intrusion on private property. Outside of that context, \textit{Camfield} controls. \textit{See id.} at 1506-07. And under \textit{Camfield}, one ``cannot maintain a fence which encloses public lands'' and prevent lawful use of the land. \textit{See id.} at 1511-12 (citations omitted). Rejecting the claim that the district court's reasoning imposed an ``easement'' upon private land by impeding an owner's ability to exclude, we held that no easement was needed to  remove a nuisance that was unlawfully inclosing federal lands. \textit{Id.} at 1505.

\textit{Third}, we rejected the argument that declaring the fence unlawful was an unconstitutional taking, based again on the fact that the inclosing fence was a nuisance. There is no unconstitutional taking where the government abates a nuisance. ``The government's power to act in this regard was settled by \textit{Camfield.}'' \textit{Id.} at 1507. Further, ``we [could] find nothing of Lawrence's that has been `taken.'\,'' \textit{Id.} Rather, ``[a]ll that Lawrence ha[d] lost [was] the right to exclude others ... from the public domain—\textit{a right he never had.}'' \textit{Id.} at 1508 (emphasis added).

\textit{Finally}, like the Eighth Circuit, we reiterated that \S~1063 prevents ``the obstruction of free passage or transit for \textit{any and all lawful purposes over public lands.}'' \textit{Id.} at 1509 (emphasis added). As such, a checkerboard landowner cannot maintain a barrier ``which encloses public lands and prevents [lawful access].'' \textit{Id.} at 1511 (``It is not the fence itself, but its effect which constitutes the UIA violation.''). Put simply, we found any inclosure that effectively prevents access to public land for \textit{lawful} use is an \textit{unlawful} inclosure that is a proscribed violation of federal law.



\readinghead{* * * * *}

We extract from these cases that the UIA proscribes the ``exclusive use and occupancy of any part of the public lands'' and further prohibits conduct that ``prevent[s] or obstruct[s] free passage or transit over or through the public lands.'' \textit{See} \S\S~1061, 1063. The controlling principle is that checkerboard landowners cannot maintain a barrier that has the effect of fully enclosing public lands and preventing complete access for a lawful purpose. \textit{Bergen}, 848 F.2d at 1511-12. When a landowner denies checkerboard access, he imposes a proscribable nuisance under federal law, ``notwithstanding such action may involve an entry upon the lands of a private individual.'' \textit{Camfield}, 167 U.S. at 525, 17 S.Ct. 864. At the same time, the government— and by extension its licensees (\textit{i.e.}, the public)—do not have an ``implied easement to build a road'' across private holdings to reach public lands. \textit{Leo Sheep}, 440 U.S. at 669, 99 S.Ct. 1403. \textit{Leo Sheep} did not speak to, and is not controlling for, the type of limited airspace intrusion ratified by the district court.



\readinghead{\textit{E. Application}}

As our review of these cases demonstrate, courts have not been entirely consistent in their review of checkerboard cases. Courts have analyzed similar fact patterns under both a nuisance law approach, such as in \textit{Camfield}, and a no-implied-easement approach, such as in \textit{Leo Sheep.} However incongruous those cases are in theory, their application to Iron Bar's claim is made straightforward by \textit{Bergen}: a barrier to access, even a civil trespass action, becomes an abatable federal nuisance in the checkerboard when its effect is to inclose public lands by completely preventing access for a lawful purpose.



\readinghead{\textit{1.} Camfield \textit{controls, not} Leo Sheep}

Iron Bar urges us to broadly apply \textit{Leo Sheep} and reject the nuisance-oriented approach set forth in \textit{Camfield} and adopted by \textit{Bergen.} It contends that \textit{Camfield} and \textit{Bergen} are just ``fence cases'' and should not be extended to private party trespasses. But we are bound by \textit{Bergen} absent en banc review by this court.

The core principle of the UIA, as reiterated in \textit{Bergen}, is that a landowner cannot maintain a barrier ``which encloses public lands and prevents'' access for a ``lawful purpose.'' \textit{Bergen}, 848 F.2d at 1511-12. The barrier itself is not a UIA  violation—but it becomes one when its effect is to inclose. That was simply not at issue in \textit{Leo Sheep. See} 440 U.S. at 685, 686 n.22, 687 n.24, 99 S.Ct. 1403.\readingfootnote{32}{Following \textit{Camfield}, the circuits split as to the relevance of a landowner's intent. We, in line with the Eighth Circuit, hold intent is an irrelevant consideration so long as a barrier's effect is to inclose. \textit{Bergen}, 848 F.2d at 1511 (``It is not the fence itself, but its effect which constitutes the UIA violation.''); \textit{Homer v. United States}, 185 F. 741, 746 (8th Cir. 1911) (``[\textit{Camfield}] necessarily decided that building a fence on one's own land without an intention of inclosing [public] lands of the United States was no defense, if in fact the lands mentioned were actually inclosed.''). Since the effect of granting Iron Bar an injunction would be to inclose the public lands, our inquiry ends there. \textit{But see} \textit{Golconda Cattle Co. v. United States}, 214 F. 903, 908-09 (9th Cir. 1914) (holding a fence lawful where it both ``admitted of reasonable access by the public to the public domain'' and was ``not constructed with any intent to inclose any government land, or to exclude the public from entering upon the public domain'').}

In \textit{Bergen}, we found \textit{Camfield} was dispositive while \textit{Leo Sheep} was ``inapplicable'' to the case because the UIA did not create easements or servitudes. 848 F.2d at 1505-07. In other words, \textit{Bergen} concluded the easement question was ``simply not at issue'' because ``the district court did not grant ... any easement across [the] private lands....'' 848 F.2d at 1505; \textit{see also id.} at 1506, 1511 (``The UIA declares enclosures of federal lands to be unlawful and orders that such enclosures be removed.... [i]t is not the fence itself, but its effect which constitutes the UIA violation.'').

\textit{Bergen}'s logic can perhaps be best explained the following way. \textit{Leo Sheep}'s holding is narrow; the government does not have an ``\textit{implied} easement'' to ``construct a road for public access'' in the checkerboard. 440 U.S. at 679, 99 S.Ct. 1403 (emphasis added). That makes sense since the effect of the road would have been a permanent, physical appropriation of Leo Sheep's property with no corresponding benefit. In those cases, the ``traditional rule'' generally governs: If ``the government appropriate[s] a right to invade, compensation [is] due.''\readingfootnote{33}{\textit{See} discussion \textit{infra} Section II.E.4.} \textit{Cedar Point Nursery}, 594 U.S. at 156, 141 S.Ct. 2063; \textit{see also} \textit{Leo Sheep}, 440 U.S. at 678-88, 99 S.Ct. 1403 (``[W]e are unwilling to upset settled expectation to accommodate some ill-defined power to construct public thoroughfares without compensation.''). Even the nuisance cases would have required those crossing the checkerboard to pay for damages to private property. But that the government lacks ``an implied easement to build a road across'' the transcontinental railroad grants does not foreclose all physical invasions. 440 U.S. at 669, 99 S.Ct. 1403.

Even so, Iron Bar points to language in \textit{Leo Sheep} suggesting that \textit{no} access right survives its holding. \textit{See, e.g.}, Aplt. Br. at 6, 16, 35 (quoting 440 U.S. at 682, 99 S.Ct. 1403 (``[W]e are unwilling to imply rights-of-way, with the substantial impact that such implication would have on property rights granted over 100 years ago, in the absence of a stronger case for their implication than the [g]overnment makes here.'')). Yet the Court carefully explained that ``\textit{[t]hese} rights-of-way are referred to as `easements by necessity.'\,'' \textit{Leo Sheep}, 440 U.S. at 679, 99 S.Ct. 1403 (emphasis added). The UIA, in contrast, contemplates a limited physical intrusion necessary to abate a nuisance— inclosure of the public lands. \textit{Camfield}, 167 U.S. at 525, 17 S.Ct. 864. And the Court has repeatedly held that ``many government-authorized physical invasions... are consistent with longstanding background restrictions on property  rights''—including ``requiring him to abate a nuisance.'' \textit{Cedar Point Nursery}, 594 U.S. at 160, 141 S.Ct. 2063. Iron Bar's argument ignores that the reciprocal of preventing the right to exclude is to permit access. If a checkerboard landowner cannot impede access to public lands, then there is impliedly an access right.

Ultimately, we find that corner-crossing does not rise to the level of ``an implied easement to build a road across land that was originally granted to the Union Pacific Railroad.'' \textit{Leo Sheep}, 440 U.S. at 669, 99 S.Ct. 1403. While we recognize the doctrinal inconsistencies at play and that the access right here functionally operates like a limited easement, \textit{Bergen} forecloses that approach.

Further, \textit{Leo Sheep}'s fact pattern took the landowner's conduct outside the realm of nuisance law. There, the government bulldozed

\begin{quote}
a dirt road extending from a local county road to the reservoir across both public domain lands and fee lands of the Leo Sheep Co. It also erected signs inviting the public to use the road as a route to the reservoir.
\end{quote}
\textit{Id.} at 678. The Court was plainly rejecting the government's overreach. Moreover, it carefully explained that the UIA concerned ``the type of incursions on private property necessary to reach public land.'' \textit{Id.} at 685-86, 99 S.Ct. 1403. Nothing in the UIA case law suggests the government has the power to build a public road without compensation. If that was what the Hunters were asking for—rather than a momentary corner-cross—\textit{Leo Sheep} may well foreclose their case. \textit{Id.} at 685-86, 99 S.Ct. 1403 (``We cannot see how the Leo Sheep Co.'s unwillingness to entertain a public road without compensation can be a violation of the [UIA].'').

Other factors also distinguish the circumstances in \textit{Leo Sheep} from those presented here and in \textit{Bergen.} The concern there was reaching ``the Seminoe Reservoir, an area that is used by the public for fishing and hunting.'' \textit{Id.} at 677, 99 S.Ct. 1403. But ``because of the checkerboard configuration, it is physically impossible to enter the Seminoe Reservoir sector \textit{from this direction} [the southeast] without some minimum physical intrusion upon private land.'' \textit{Id.} at 678, 99 S.Ct. 1403. The public could, however, access the reservoir from \textit{another} direction—just not the southeast. The Court flagged this difference as distinguishing \textit{Buford}, where it allowed access based on ``the sheep ranchers' lack of any alternative.'' \textit{Id.} at 688, 99 S.Ct. 1403 n.24 (discussing \textit{Buford}, 133 U.S. at 332, 66 S.Ct. 1062). This case is more like the latter. No foot access to the landlocked federal sections of Elk Mountain is available absent corner-crossing.

Undeterred, Iron Bar points to dicta in \textit{Leo Sheep} in which the Court observed that—under some circumstances—private landowners \textit{might} be able to individually fence their plots meaning ``access to [public] lots is obstructed.'' \textit{Id.} at 685, 99 S.Ct. 1403. But as we noted in \textit{Bergen}, ``a separate enclosure for each of the square-mile sections of land owned by [Iron Bar] is `scarcely a practical question.'\,'' \textit{Bergen}, 848 F.2d at 1507 n.7 (quoting \textit{Camfield}, 167 U.S. at 528, 17 S.Ct. 864). Theoretically, one could individually fence parcels so long as such a fence does not impose an absolute barrier to access. As we explained in \textit{Bergen}, ``not every fence is a violation of the UIA,'' since a fence can encircle public land while still permitting adequate access via a gate, stile, or antelope-friendly design. \textit{Id.} at 1511.

Having determined that \textit{Camfield} and \textit{Bergen} control, we next address Iron Bar's other arguments against their application.



\readinghead{ \textit{2. Section 1063}}

First, like the landowner in \textit{Bergen}, Iron Bar argues the final clause of \S~1063 of the UIA affords it a statutory defense. That clause provides:

\begin{quote}
[t]his section shall not be held to affect the right or title of persons, who have gone upon, improved, or occupied \textit{said lands under the land laws of the United States}, claiming title thereto, in good faith.
\end{quote}
43 U.S.C. \S~1063 (emphasis added). Iron Bar interprets this clause as stating that because it has ``occupied'' its parcels, the UIA does not ``affect [its] right or title.'' \textit{See id.} But that reading takes this clause out of context.

The subject clause references ``said lands'' which persons ``have gone upon, improved, or occupied ... under the land laws of the United States.'' \textit{Id.} This refers to federal public lands subject to federal homesteading. One of the driving forces behind the UIA was Congress's incentivizing westward expansion after the Civil War. \textit{See} 15 CONG. REC. H., 4,769 (daily ed. June 3, 1884) (``It is intended that the public domain shall be reserved for homestead settlement.... It is to bring the empty lands and the empty hands together.''). The text and historical context make clear Congress was concerned the UIA might be interpreted as applying to lands that were still \textit{being} homesteaded. This language was introduced to clarify that it did not conflict with the homesteading acts. Iron Bar is no homestead.

Nor would it be relevant if Iron Bar held grazing leases on the public lands, as Amici Wyoming Stock Growers Association and Wyoming Wool Growers Association contend. As we explained in \textit{Bergen}, \S~1063 relies ``on color of \textit{fee} title.'' 848 F.2d at 1510. A grazing lease does not convey fee title. 43 U.S.C. \S~315(b) (``[T]he issuance of a [grazing] permit ... shall not create any right, title, interest, or estate in or to the lands.''); \textit{see also} \textit{Smith v. Third Nat'l Exch. Bank}, 244 U.S. 184, 37 S.Ct. 516, 61 L.Ed. 1071 (1917) (relying on color of fee title to invoke \S~1063's statutory defense); \textit{Cameron v. United States}, 148 U.S. 301, 13 S.Ct. 595, 37 L.Ed. 459 (1893) (same).

While the Taylor Grazing Act, and the pendant demise of the open range, may mean ranchers can no longer drive their cattle across the checkerboard under the guise of the UIA, these changing historical circumstances do not otherwise limit the UIA's applicability to other ``lawful purposes'' of the federal public lands. \textit{Bergen}, 848 F.2d at 1510 (citing 43 U.S.C. \S~315(e)).



\readinghead{\textit{3. Wyoming Law Is Preempted}}

Iron Bar also argues the UIA cannot and does not preempt its right to exclude under Wyoming state law. That argument is directly foreclosed by the case law. Beginning with \textit{Camfield}, the Court explained that the UIA supplants conflicting state law since a ``different rule would place the public domain of the United States completely at the mercy of state legislation.'' 167 U.S. at 525-26, 17 S.Ct. 864. As the Court said in \textit{McKelvey}: ``It also is settled that the states may prescribe police regulations applicable to public land areas, \textit{so long as the regulations are not arbitrary or inconsistent with applicable congressional enactments.}'' 260 U.S. at 359, 43 S.Ct. 132 (emphasis added).\readingfootnote{34}{This principle mirrors the doctrine under the Constitution's Property Clause. In \textit{Kleppe v. New Mexico}, 426 U.S. 529, 96 S.Ct. 2285, 49 L.Ed.2d 34 (1976), the Court rejected a challenge to the Wild Free-Roaming Horses and Burros Act, 16 U.S.C. \S\S~1331-1340, under the Constitution's Property Clause, which ``gives Congress the power over the public lands to control their occupancy and use, to protect them from trespass and injury, and to prescribe the conditions upon which others may obtain rights in them.'' \textit{Kleppe}, 426 U.S. at 540, 96 S.Ct. 2285 (citation and internal quotation marks omitted); U.S. CONST. art. IV, \S~3, cl. 2 (``Congress shall have Power to dispose of and make all needful Rules and Regulations respecting the Territory or other Property belonging to the United States[.]''). \textit{Camfield}, it explained, ``holds that the Property Clause is broad enough to permit federal regulation of fences built on private land adjoining public land when the regulation is for the protection of the federal property.'' \textit{Kleppe}, 426 U.S. at 538, 96 S.Ct. 2285. Thus, where ``state laws conflict with ... legislation passed pursuant to the Property Clause, the law is clear: the state laws must recede.'' \textit{Id.} at 543, 96 S.Ct. 2285 (citing \textit{McKelvey}, 260 U.S. at 359, 43 S.Ct. 132).}



\readinghead{ \textit{4. Takings Clause}}

Finally, Iron Bar contends that the UIA cannot limit its right to exclude since that would result in a diminishment of its property rights without just compensation and constitute an unconstitutional taking.

There is some force to this argument, but as we noted above, \textit{Bergen} assessed a similar argument and rejected it. We held that the abatement of a nuisance ``did not impose a servitude,'' and did not ``take[]'' any property right possessed by the private owner. \textit{Bergen}, 848 F.2d at 1507. As in that case, all that Iron Bar ``has lost is the right to exclude others ... from the public domain—a right [it] never had.'' \textit{Id.} at 1508.

That is not to say more recent Supreme Court precedent may cast doubt on \textit{Bergen}'s logic. \textit{Cedar Point Nursery}, for example, considered a California regulation granting union organizers a right to access private agricultural property. 594 U.S. 139, 141 S.Ct. 2063, 210 L.Ed.2d 369. The Court held this was a taking. The right to exclude, it explained, is ``one of the most treasured rights of property ownership.'' \textit{Id.} at 149, 141 S.Ct. 2063 (citation and internal quotations omitted). Given its importance, ``the Court has long treated government-authorized physical invasions as takings requiring just compensation.'' \textit{Id.} at 150, 141 S.Ct. 2063; \textit{see also} \textit{Kaiser Aetna v. United States}, 444 U.S. 164, 180, 100 S.Ct. 383, 62 L.Ed.2d 332 (1979) (``[E]ven if the [g]overnment physically invades only an easement in property, it must nonetheless pay just compensation.''). These ``physical invasions constitute takings even if they are intermittent.'' \textit{Cedar Point Nursery}, 594 U.S. at 153, 141 S.Ct. 2063. And thus compensation is due because ``the government appropriated a right to invade.'' \textit{Id.} at 156, 141 S.Ct. 2063.

But the UIA both explicitly and implicitly authorizes some right to invade. And the reciprocal loss of the right to exclude others from the public domain is losing the right to exclude others from one's \textit{private} domain when that involves only limited invasion over one's private domain. True, \textit{Bergen} said a landowner ``retains the right to exclude ... if he can accomplish that exclusion without at the same time effecting an enclosure of the public lands.'' \textit{Bergen}, 848 F.2d at 1507. But the catch-22 with Congress's checkerboard scheme is that a landowner cannot accomplish total exclusion without violating the UIA. So—by permitting limited trespass —the UIA diminishes a property right a landowner would otherwise have. And ``[w]hen the government physically acquires private property for a public use, the Takings Clause imposes a clear and categorical obligation to provide the owner with just compensation.'' \textit{Cedar Point Nursery}, 594 U.S. at 147, 141 S.Ct. 2063.

Yet even \textit{Cedar Point Nursery} acknowledges that

\begin{quote}
 many government-authorized physical invasions will not amount to takings because they are consistent with longstanding background restrictions on property rights. As we explained in \textit{Lucas v. South Carolina Coastal Council}, the government does not take a property interest when it merely asserts a ``pre-existing limitation upon the land owner's title.'' 505 U.S. at 1028-29, 112 S.Ct. 2886. For example, the government owes a landowner no compensation for requiring him to abate a nuisance on his property, \textit{because he never had a right to engage in the nuisance in the first place. See id.} at 1029-30, 112 S.Ct. 2886.
\end{quote}
\textit{Id.} at 160 (emphasis added).

\textit{Bergen} unquestionably sanctioned physical invasions without compensation, ostensibly diminishing ``one of the most essential sticks in the bundle of rights.'' \textit{Id.} at 150, 141 S.Ct. 2063. But it does so under the nuisance-abatement theory created by \textit{Camfield} and the UIA, and so may fall into a ``background restriction[]'' carved out by \textit{Cedar Point Nursery. See} \textit{Jordan-Arapahoe}, 633 F.3d at 1025-26 (``[E]xisting rules or understandings'' are used to define ``the scope of property rights in land ownership.''). Even if this were a taking, it occurred when the UIA was passed or when \textit{Camfield} was decided. Iron Bar theoretically acquired its private land subject to those preexisting restrictions. That is why the right to inclose federal land was a right that Iron Bar ``never had.'' \textit{Bergen}, 848 F.2d at 1508.\readingfootnote{35}{We do not here address whether or how our takings analysis might change if the access rights claimed over Iron Bar's property were significantly expanded to include new and far greater public usage.}

We appreciate this may be an unsatisfying result for property owners within the checkerboard. It leaves open questions for landowners and the public alike, including who might be liable during a corner-crossing incident, and what duty of care each party owes the other. Iron Bar may be correct that the government could solve these open questions by exercising its core institutional competency to condemn access easements to landlocked checkerboard lands. The Supreme Court can also reconsider the scope of \textit{Leo Sheep} as it applies to this case.



\readinghead{III. Conclusion}

The western checkerboard and UIA reflect a storied period of our history. Whatever the UIA's merits today, it—and the case law interpreting it—remain good federal law. \textit{Bergen}, 848 F.2d at 1506. Applying that law here, Iron Bar cannot implement a program which has the effect of ``deny[ing] access to [federal] public lands for lawful purposes[.]'' \textit{Id.} at 1509 (internal quotation omitted). So the district court was correct to hold that the Hunters could corner-cross as long as they did not physically touch Iron Bar's land.

We affirm.

