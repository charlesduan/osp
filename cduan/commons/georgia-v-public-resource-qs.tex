\item \textbf{Where do laws come from?} You probably take for granted that you
have access to cases, statutes, and regulations through electronic databases.
But how do those texts of the law get from lawmakers to you?

For legislation, the Government Publishing Office publishes statutes, in the
order
of enactment, in a book series called the \emph{Statutes at Large}. Some
statutes never make it past this point, but the text of some important statutes
are rearranged by topic and reprinted every six years in another government
publication, the \emph{U.S. Code}. In other words, the \emph{Statutes at Large}
are the actual enactments, while the \emph{Code} is technically just a helpful
guide. \unskip\footnote{This is complicated by the fact that Congress has
enacted some of the titles of the \emph{U.S. Code} into positive law.
Even in those cases, though, the text of the session law is still definitive;
the difference is that the session law is written in the form of instructions
for amending the \emph{U.S. Code}.} The
\emph{U.S. Code Annotated} and \emph{U.S. Code Service} are independent,
privately published books by West and LexisNexis, respectively. They generally
use the same volume and section numbers as the \emph{U.S. Code}, but contain
additional annotations and other materials.

Regulations and other federal agency decisions are published in order of
issuance in the \emph{Federal Register}, another government publication. Again
as a matter of convenience, some regulations are reorganized by topic and
published annually in the \emph{Code of Federal Regulations}. While most
agencies publish through the \emph{Federal Register}, a few maintain their own
publications (the \emph{Federal Communications Commission Record}, for example),
and the some like the Department of Justice have no official form of
promulgation at all.

Judicial decisions are the most complicated. When a federal judge today issues
an opinion, it is posted on PACER, the federal electronic docket service (or, in
the case of the Supreme Court, just on its website).
\unskip\footnote{In the old days, judges didn't write opinions, instead
delivering them orally from the bench. There was no official record of
decisions. Instead, attorneys would sit in the courtroom and take notes on
decisions the judges rendered. Often the notes would only circulate among
colleagues, but some especially entrepreneurial lawyers would publish
their notes. These are called ``nominative reports'' because they
are typically referenced by the compiler's name. For most of English history,
this is how judicial decisions were reported.\having{armory-v-delamirie}{ This
explains why \mref{armory-v-delamirie} is so short---it's not an opinion,
but just John Strange's notes.}{}{}}
A ``reporter'' collects
these opinions and publishes them in a volume. For the Supreme Court, there is
an official reporter who publishes the \emph{U.S. Reports}. There are also two
private reporters of Supreme Court decisions: the \emph{Supreme Court Reporter}
by West and the \emph{Lawyers' Edition} by LexisNexis. There is no official
reporter for other decisions; the \emph{Federal Reporter} and \emph{Federal
Supplement} for appellate and district court decisions are privately published
by West.

This of course just covers print publications; electronic media are yet another
story. And we haven't even touched on state law yet. What a mess! Does this
strike you as a tragedy of the commons, borne of a lack of property rights in
legal texts? Or corporate meddling with a resource that belongs to the public?

\item According to Chief Justice Roberts, ``it needs no argument to show\ldots
that all should have free access'' to the law. Let's make the argument anyway.
Here are a few possibilities:
\begin{itemize}
\item Due process: government cannot hold people liable for violations of law
that they cannot read.
\item Equal protection: those of less financial means still deserve
``first-class'' access to the law.
\item Free speech: political discourse and self-governance require that citizens
be able to know the laws.
\item We the people: since sovereign power arises from the people, the people
own the laws that they write.
\end{itemize}
Can you think of other rationales? How do these apply with respect to the
nonbinding annotations at issue in the case? For that matter, do any of these
rationales seem applicable to other forms of property?

\defcase{astm-v-pubres-2023}{
American Society for Testing and Materials v. Public.Resource.Org, Inc., 84
F.4th 1262 (2023)
}

\item What if a private entity writes a model code that is subsequently
incorporated into the law? For example, the International Code Council is a
private organization that writes a variety of building and housing safety codes.
As written, ICC's codes are just recommendations; they are also properly the
subject of copyrights owned by ICC. But almost all U.S. jurisdictions mandate
compliance with ICC codes. Is there a public right to access or redistribute a
mandatory, private building code without permission? \sentence{see
astm-v-pubres-2023}.


