\defjrnart{reid-revolution}{
{Charles J. Reid, Jr.}, The Seventeenth-Century Revolution in the English Land
Law, 43 Cleveland State Law Review 221 (1995)
}
\defjrnart{boyle-second-enclosure}{
James Boyle, The Second Enclosure Movement, 66 Law and Contemporary Problems 33
(2003)
}

Over half a millennium, a revolution in land occurred across England. Fields
that had by custom and tradition been held as a \term{commons}, available to all
commoners for grazing, farming, and other uses, were enclosed and converted to
private ownership, primarily by lords and nobles. Enclosure of the commons
sometimes happened by erection of physical barriers like fences, sometimes by
agreement with the local farmers and commoners, and sometimes by acts of
Parliament. The enclosure movement was controversial, arguably increasing
productivity of land but also sparking riots among those losing rights. But the
result was that, by the end of the 19th century, the commons had largely turned
into private land. \sentence{see generally reid-revolution at 252-261;
boyle-second-enclosure at 34-36}.

Property law is roughly concerned with private rights in resources. A commons
\unskip\footnote{Despite its appearance, ``commons'' is typically treated as a
singular noun.} seems the opposite of that---a resource for which no one holds a
right to exclude. As such, the notion of a commons seems to fly in the face of
theories justifying property ownership as an institution. Indeed, the
\term{tragedy of the commons} concept, \having{tragedy-commons-qs}{discussed
above, \mref{tragedy-commons-qs}}{discussed later,
\mref{tragedy-commons-qs}}{namely that commons resources will
tend to be overexploited because no one has an incentive to limit their use},
would suggest that a commons is an inferior manner of resource management
compared to private ownership.

Yet commons are, for lack of a better term, common. A commons can arise because
a resource is too hard to appropriate to one owner (like the air), because
society agrees to treat a resource as non-appropriable (like a public park), or
because legal institutions insufficiently govern the resource (like the deep sea
or space, though some treaties govern those). Especially in the world of
intellectual property, tremendous swaths of information are in the commons,
owned by no one. Among other things, that is why you are free to learn all the
property law doctrines and ideas you could possibly desire.

Theories of the commons---why they arise, how they operate, who manages them,
why they survive despite the tragedy of the commons---are extensive and complex.
The focus of this chapter is narrower. It considers how law, and property law in
particular, interacts with a commons. How do laws divide the realm of private
ownership from the commons? What happens when a property owner interferes with
access to a commons? And to what extent can interests in a commons be understood
as property rights?

