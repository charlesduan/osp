\reading{State of Georgia v. Public.Resource.Org, Inc.}
\readingcite{590 U.S. 255 (2020)}

\opinion Chief Justice \textsc{Roberts} delivered the opinion of the Court.

\ldots.

%The Copyright Act grants potent, decades-long monopoly protection for ``original
%works of authorship.'' 17 U.S.C. \S~102(a). The question in this case is whether
%that protection extends to the annotations  contained in Georgia's official
%annotated code.
%
%We hold that it does not. Over a century ago, we recognized a limitation on
%copyright protection for certain government work product, rooted in the
%Copyright Act's ``authorship'' requirement. Under what has been dubbed the
%government edicts doctrine, officials empowered to speak with the force of law
%cannot be the authors of---and therefore cannot copyright ---the works they create
%in the course of their official duties.
%
%We have previously applied that doctrine to hold that non-binding, explanatory
%legal materials are not copyrightable when created \textit{by judges} who
%possess the authority to make and interpret the law. See \textit{Banks v.
%Manchester}, 128 U.S. 244, 9 S.Ct. 36, 32 L.Ed. 425 (1888). We now recognize
%that the same logic applies to non-binding, explanatory legal materials created
%\textit{by a legislative body} vested with the authority to make law. Because
%Georgia's annotations are authored by an arm of the legislature in the course of
%its legislative duties, the government edicts doctrine puts them outside the
%reach of copyright protection.



%\readinghead{I}



%\readinghead{A}

The State of Georgia has one official code---the ``Official Code of Georgia
Annotated,'' or OCGA. The first page of each volume of the OCGA boasts the
State's official seal and announces to readers that it is ``Published Under
Authority of the State.''

The OCGA includes the text of every Georgia statute currently in force, as well
as various non-binding supplementary materials. At issue in this case is a set
of annotations that appear beneath each statutory provision. The annotations
generally include summaries of judicial decisions applying a given provision,
summaries of any pertinent opinions of the state attorney general, and a list of
related law review articles and similar reference materials. In addition, the
annotations often include editor's notes that provide information about the
origins of the statutory text, such as whether it derives from a particular
judicial decision or resembles an older provision that has been construed by
Georgia courts.

%The OCGA is assembled by a state entity called the Code Revision Commission. In
%1977, the Georgia Legislature established the Commission to recodify Georgia
%law for the first time in decades. The Commission was (and remains) tasked with
%consolidating disparate bills into a single Code for reenactment by the
%legislature and contracting with a third party to produce the annotations. A
%majority of the Commission's 15 members must be members of the Georgia Senate
%or House of Representatives. The Commission receives funding through
%appropriations ``provided for the legislative branch of state government.''
%OCGA \S~28-9-2(c) (2018). And it is staffed by the Office of Legislative
%Counsel, which is obligated by statute to provide services ``for the
%legislative branch of government.'' \S\S~28-4-3(c)(4), 28-9-4. Under the
%Georgia Constitution, the Commission's role in compiling the statutory text and
%accompanying annotations falls ``within the sphere of legislative authority.''
%\textit{Harrison Co. v. Code Revision Comm'n}, 244 Ga. 325, 330, 260 S.E.2d 30,
%34 (1979).
%
%Each year, the Commission submits its proposed statutory text and accompanying
%annotations to the legislature for approval. The legislature then votes to do
%three things: (1) ``enact[]'' the ``statutory portion of the codification of
%Georgia laws''; (2) ``merge[]'' the statutory portion ``with [the]
%annotations''; and (3) ``publish[]'' the  final merged product ``by authority
%of the state'' as ``the `Official Code of Georgia Annotated.'\,'' OCGA \S~1-1-1
%(2019); see \textit{Code Revision Comm'n v. Public.Resource.Org, Inc.}, 906
%F.3d 1229, 1245, 1255 (CA11 2018); Tr. of Oral Arg. 8.

The annotations in the current OCGA were prepared in the first instance by
Matthew Bender \& Co., Inc., a division of the LexisNexis Group, pursuant to a
work-for-hire agreement with the [Code Revision Commission, an office created by
and part of the Georgia Legislature]. The agreement between Lexis and the
Commission states that any copyright in the OCGA vests exclusively in ``the
State of Georgia, acting through the Commission.'' Lexis and its army
of researchers perform the lion's share of the work in drafting the annotations,
but the Commission supervises that work and specifies what the annotations must
include in exacting detail.
Under the agreement, Lexis enjoys the exclusive
right to publish, distribute, and sell the OCGA. In exchange, Lexis has agreed
to limit the price it may charge for the OCGA and to make an unannotated version
of the statutory text available to the public online for free. A hard copy of
the complete OCGA currently retails for \$412.00.



\readinghead{B}

Public.Resource.Org (PRO) is a nonprofit organization that aims to facilitate
public access to government records and legal materials. Without permission, PRO
posted a digital version of the OCGA on various websites, where it could be
downloaded by the public without charge. PRO also distributed copies of the OCGA
to various organizations and Georgia officials.

[Georgia and the Code Revision Commission sued PRO for copyright infringement,
and the Supreme Court granted certiorari.]

%In response, the Commission sent PRO several cease-and-desist letters asserting
%that PRO's actions constituted unlawful copyright infringement. When PRO
%refused to halt its distribution activities, the Commission sued PRO on behalf
%of the Georgia Legislature and the State of Georgia for copyright infringement.
%The Commission limited its assertion of copyright to the annotations described
%above; it did not claim copyright in the statutory text or numbering. PRO
%counterclaimed, seeking a declaratory judgment that the entire OCGA, including
%the annotations, fell in the public domain.
%
%The District Court sided with the Commission. The Court acknowledged that the
%annotations in the OCGA presented ``an unusual case because most official codes
%are not annotated and most annotated codes are not official.'' \textit{Code
%Revision Comm'n v. Public.Resource.Org, Inc.}, 244 F.Supp.3d 1350, 1356 (ND Ga.
%2017). But, ultimately, the Court concluded that the annotations were eligible
%for copyright protection because they were ``not enacted into law'' and lacked
%``the force of law.'' \textit{Ibid.} In light of that conclusion, the Court
%granted partial summary judgment to the Commission and entered a permanent
%injunction requiring PRO to cease its distribution activities and to remove the
%digital copies of the OCGA from the internet.
%
%The Eleventh Circuit reversed. 906 F.3d 1229. The Court began by reviewing the
%three 19th-century cases in which we articulated the government edicts
%doctrine. See \textit{Wheaton v. Peters}, 8 Pet. 591, 8 L.Ed. 1055 (1834);
%\textit{Banks v. Manchester}, 128 U.S. 244, 9 S.Ct. 36, 32 L.Ed. 425 (1888);
%\textit{Callaghan v. Myers}, 128 U.S. 617, 9 S.Ct. 177, 32 L.Ed. 547 (1888).
%The Court understood those cases to establish a ``rule'' based on an
%interpretation of the statutory term ``author'' that ``works created by courts
%in the performance of their official duties did not belong to the judges'' but
%instead fell ``in the public domain.'' 906 F.3d at 1239. In the Court's view,
%that rule ``derive[s] from first principles about the  nature of law in our
%democracy.'' \textit{Ibid.} In a democracy, the Court reasoned, ``the People''
%are ``the constructive authors'' of the law, and judges and legislators are
%merely ``draftsmen ... exercising delegated authority.'' \textit{Ibid.} The
%Court therefore deemed the ``ultimate inquiry'' to be whether a work is
%``attributable to the constructive authorship of the People.'' \textit{Id.}, at
%1242. The Court identified three factors to guide that inquiry: ``the identity
%of the public official who created the work; the nature of the work; and the
%process by which the work was produced.'' \textit{Id.}, at 1254. The Court
%found that each of those factors cut in favor of treating the OCGA annotations
%as government edicts authored by the People. It therefore rejected the
%Commission's assertion of copyright, vacated the
%injunction against PRO, and directed that judgment be entered for PRO.
%
%We granted certiorari. 588 U.S. \_\_\_, 139 S.Ct. 2746, 204 L.Ed.2d 1130
%(2019).



\readinghead{II}

We hold that the annotations in Georgia's Official Code are ineligible for
copyright protection\ldots.
%, though for reasons distinct from those relied on by the Court of Appeals.
A careful examination of our government edicts precedents
reveals a straightforward rule based on the identity of the author. Under the
government edicts doctrine, judges---and, we now confirm, legislators---may not
be considered the ``authors'' of the works they produce in the course of their
official duties as judges and legislators. That rule applies regardless of
whether a given material carries the force of law. And it applies to the
annotations here because they are authored by an arm of the legislature in the
course of its official duties.



\readinghead{A}

We begin with precedent. The government edicts doctrine traces back to a trio of
cases decided in the 19th century. In this Court's first copyright case,
\textit{Wheaton v. Peters}, the Court's third
Reporter of Decisions, Wheaton, sued the fourth, Peters, unsuccessfully
asserting a copyright interest in the Justices' opinions.
In Wheaton's view, the opinions ``must have belonged to some one''
because ``they were new, original,'' and much more ``elaborate'' than law or
custom required. Wheaton argued that the Justices were the
authors and had assigned their ownership interests to him through a tacit
``gift.'' The Court unanimously rejected that argument,
concluding that ``no reporter has or can have any copyright in the written
opinions delivered by this court'' and that ``the judges thereof cannot confer
on any reporter any such right.''

That conclusion apparently seemed too obvious to adorn with further explanation,
but the Court provided one a half century later in \textit{Banks v. Manchester}.
That case concerned whether
Wheaton's state-court counterpart, the official reporter of the Ohio Supreme
Court, held a copyright in the judges' opinions and several non-binding
explanatory materials prepared by the judges.
The Court concluded that he did not, explaining that ``the judge who, in his
judicial capacity, prepares the opinion or decision, the statement of the case
and the syllabus or head note'' cannot ``be regarded as their author or their
proprietor, in the sense of [the Copyright Act].''
Pursuant to ``a judicial \textit{consensus}'' dating back to
\textit{Wheaton}, judges could not assert copyright in ``whatever work they
perform in their capacity as judges.''
Rather, ``[t]he  whole work done by the judges
constitutes the authentic exposition and interpretation of the law, which,
binding every citizen, is free for publication to all.''

In a companion case decided later that Term, \textit{Callaghan v. Myers},
the Court identified an important
limiting principle. As in \textit{Wheaton} and \textit{Banks}, the Court
rejected the claim that an official reporter held a copyright interest in the
judges' opinions. But, resolving an issue not addressed in \textit{Wheaton} and
\textit{Banks}, the Court upheld the reporter's copyright interest in several
explanatory materials that the reporter had created himself: headnotes, syllabi,
tables of contents, and the like.
Although these works mirrored the judge-made materials rejected in
\textit{Banks}, they came from an author who had no authority to speak with the
force of law. Because the reporter was not a judge, he was free to ``obtain[] a
copyright'' for the materials that were ``the result of his [own] intellectual
labor.''

These cases establish a straightforward rule: Because judges are vested with the
authority to make and interpret the law, they cannot be the ``author'' of the
works they prepare ``in the discharge of their judicial duties.''
This rule applies both to binding
works (such as opinions) and to non-binding works (such as headnotes and
syllabi). It does not apply, however, to works created by
government officials (or private parties) who lack the authority to make or
interpret the law, such as court reporters.

The animating principle behind this rule is that no one can own the law. ``Every
citizen is presumed to know the law,'' and ``it needs no argument to show\ldots
that all should have free access'' to its contents.
Our cases give effect to that principle in the copyright context through
construction of the statutory term ``author.'' Rather than attempting to
catalog the materials that constitute ``the law,'' the doctrine bars the
officials responsible for creating the law from being considered the
``author[s]'' of ``\textit{whatever work} they perform in their capacity'' as
lawmakers. Because these officials are
generally empowered to make and interpret law, their ``whole work'' is deemed
part of the ``authentic exposition and interpretation of the law'' and must be
``free for publication to all.''

If judges, acting as judges, cannot be ``authors'' because of their authority to
make and interpret the law, it follows that legislators, acting as legislators,
cannot be either. Courts have thus long understood the government edicts
doctrine to apply to legislative materials.

Moreover, just as the doctrine applies to ``whatever work [judges] perform in
their capacity as judges,'' it
applies to whatever work legislators perform in their capacity as legislators.
That of course includes final legislation, but it also includes explanatory and
procedural materials legislators create in the discharge of their legislative
duties. In the same way that judges cannot be the authors of their headnotes and
syllabi, legislators cannot be the authors of (for example) their floor
statements, committee reports, and proposed bills. These materials are part of
the ``whole work done by [legislators],'' so they must be ``free for publication
to all.''

Under our precedents, therefore, copyright does not vest in works that are (1)
created by judges and legislators (2) in the course of their judicial and
legislative duties.



\readinghead{B}



\readinghead{1}

[The Court concluded that the Code Revision Commission was tantamount to the
Georgia legislature, and therefore was a ``legislator'' for purposes of the
above rule.]

%Applying that framework, Georgia's annotations are not copyrightable. The first
%step is to examine whether their purported author qualifies as a legislator.
%
%As we have explained, the annotations were prepared in the first instance by a
%private company (Lexis) pursuant to a work-for-hire agreement with Georgia's
%Code Revision Commission. The Copyright Act therefore deems the Commission the
%sole ``author'' of the work. 17 U.S.C. \S~201(b). Although Lexis expends
%considerable effort preparing the annotations, for purposes of copyright that
%labor redounds to the Commission as the statutory author. Georgia agrees that
%the author is the Commission. Brief for Petitioners 25.
%
%The Commission is not identical to the Georgia Legislature, but functions as an
%arm of it for the purpose of producing the annotations. The Commission is
%created by the legislature, for the legislature, and consists largely of
%legislators. The Commission receives funding and staff designated by law for the
%legislative branch. Significantly, the annotations the Commission creates are
%approved by the legislature before being ``merged'' with the statutory text and
%published in the official code alongside that text at the legislature's
%direction. OCGA \S~1-1-1; see 906 F.3d at 1245, 1255; Tr. of Oral Arg. 8.
%
%If there were any doubt about the link between the Commission and the
%legislature, the Georgia Supreme Court has dispelled it by holding that, under
%the Georgia Constitution, ``the work of the Commission; \textit{i.e.}, selecting
%a publisher and contracting for and supervising the codification of the laws
%enacted by the General Assembly, including court interpretations thereof,
%\textit{is within the sphere of legislative authority.}'' \textit{Harrison Co.},
%244 Ga. at 330, 260 S.E.2d at 34 (emphasis added). That holding is not limited
%to the Commission's role in codifying the statutory text. The Commission's
%``legislative authority'' specifically includes its ``codification of ... court
%interpretations'' of the State's laws. \textit{Ibid.} Thus, as a matter of state
%law, the Commission wields the legislature's authority when it works with Lexis
%to produce the annotations. All of this shows that the Commission serves as an
%extension of the Georgia Legislature in preparing and publishing the
%annotations. And it helps explain why the Commission brought this suit asserting
%copyright in the annotations ``on behalf of and for the benefit of the Georgia
%Legislature and the State of Georgia.  App. 20.\readingfootnote{3}{Justice
%THOMAS does not dispute that the Commission is an extension of the legislature;
%he instead faults us for highlighting the multiple features of the Commission
%that make clear that this is so. See \textit{post}, at 1521-1522 (dissenting
%opinion).}



\readinghead{2}

The second step is to determine whether the Commission creates the annotations
in the ``discharge'' of its legislative ``duties.''
It does. Although the annotations are not enacted into law
through bicameralism and presentment, the Commission's preparation of the
annotations is under Georgia law an act of ``legislative authority,''
and the annotations
provide commentary and resources that the legislature has deemed relevant to
understanding its laws. Georgia and Justice \textsc{Ginsburg} emphasize that the
annotations do not purport to provide authoritative explanations of the law and
largely summarize other materials, such as judicial decisions and law review
articles. But that does
not take them outside the exercise of legislative duty by the Commission and
legislature. Just as we have held that the ``statement of the case and the
syllabus or head note'' prepared by judges fall within the ``work they perform
in their capacity as judges,'' so
too annotations published by legislators alongside the statutory text fall
within the work legislators perform in their capacity as legislators.

In light of the Commission's role as an adjunct to the legislature and the fact
that the Commission authors the annotations in the course of its legislative
responsibilities, the annotations in Georgia's Official Code fall within the
government edicts doctrine and are not copyrightable.



\readinghead{III}

[The Court rejected Georgia's statutory interpretation arguments that the
Copyright Act had abrogated the edicts of government doctrine.]

%Georgia resists this conclusion on several grounds. At the outset, Georgia
%advances two arguments for why, in its view, excluding the OCGA annotations from
%copyright protection conflicts with the text of the Copyright Act. Both are
%unavailing.
%
%First, Georgia notes that \S~101 of the Act specifically lists ``annotations''
%among the kinds of works eligible for copyright protection. But that provision
%refers only to ``annotations ... which ... represent an original work of
%\textit{authorship.}'' 17 U.S.C. \S~101 (emphasis added). The whole point of the
%government edicts doctrine is that judges and legislators cannot serve as
%authors when they produce works in their official capacity. While the reference
%to ``annotations'' in \S~101 may help explain why supplemental, explanatory
%materials are copyrightable when prepared by a private party, or a non-lawmaking
%official like the reporter in \textit{Callaghan}, it does not speak to whether
%those same materials are copyrightable when prepared by a judge or a legislator.
%In the same way that judicial materials are ineligible for protection even
%though they plainly qualify as ``[l]iterary works ... expressed in words,''
%\textit{ibid.}, legislative materials are ineligible for protection even if they
%happen to fit the description of otherwise copyrightable ``annotations.''
%
%Second, Georgia draws a negative inference from the fact that the Act excludes
%from copyright protection ``work[s] prepared by an officer or employee of the
%United States Government as part of that person's official duties'' and does not
%establish a similar rule for the States. \S~101; see also \S~105. But the bar on
%copyright protection for federal works sweeps much more broadly than the
%government edicts  doctrine does. That bar applies to works created by all
%federal ``officer[s] or employee[s],'' without regard for the nature of their
%position or scope of their authority. Whatever policy reasons might justify the
%Federal Government's decision to forfeit copyright protection for its own
%proprietary works, that federal rule does not suggest an intent to displace the
%much narrower government edicts doctrine with respect to the States. That
%doctrine does not apply to non-lawmaking officials, leaving States free to
%assert copyright in the vast majority of expressive works they produce, such as
%those created by their universities, libraries, tourism offices, and so on.
%
%More generally, Georgia suggests that we should resist applying our government
%edicts precedents to the OCGA annotations because our 19th-century forebears
%interpreted the statutory term author by reference to ``public policy''---an
%approach that Georgia believes is incongruous with the ``modern era'' of
%statutory interpretation. Brief for Petitioners 21 (internal quotation marks
%omitted). But we are particularly reluctant to disrupt precedents interpreting
%language that Congress has since reenacted. As we explained last Term in
%\textit{Helsinn Healthcare S. A. v. Teva Pharmaceuticals USA, Inc.}, 586 U.S.
%\_\_\_, 139 S.Ct. 628, 202 L.Ed.2d 551 (2019), when Congress ``adopt[s] the
%language used in [an] earlier act,'' we presume that Congress ``adopted also the
%construction given by this Court to such language, and made it a part of the
%enactment.'' \textit{Id.}, at \_\_\_, 139 S.Ct., at 634 (quoting \textit{Shapiro
%v. United States}, 335 U.S. 1, 16, 68 S.Ct. 1375, 92 L.Ed. 1787 (1948)). A
%century of cases have rooted the government edicts doctrine in the word
%``author,'' and Congress has repeatedly reused that term without abrogating the
%doctrine. The term now carries this settled meaning, and ``critics of our ruling
%can take their objections across the street, [where] Congress can correct any
%mistake it sees.'' \textit{Kimble v. Marvel Entertainment, LLC}, 576 U.S. 446,
%456, 135 S.Ct. 2401, 192 L.Ed.2d 463 (2015).\readingfootnote{4}{Justice THOMAS
%disputes the applicability of the \textit{Helsinn Healthcare} presumption
%because States have asserted copyright in statutory annotations over the years
%notwithstanding our government edicts precedents. \textit{Post}, at 1518-1520.
%In Justice THOMAS's view, those assertions prove that our precedents could not
%have provided clear enough guidance for Congress to incorporate. But that
%inference from state behavior proves too much. The same study cited by Justice
%THOMAS to support a practice of claiming copyright in non-binding
%\textit{annotations} also reports that ``many states claim copyright interest in
%their \textit{primary} law materials,'' including statutes and regulations.
%Dmitrieva, State Ownership of Copyrights in Primary Law Materials, 23 Hastings
%Com. \& Entertainment L. J. 81, 109 (2000) (emphasis added). Justice THOMAS
%concedes that such assertions are plainly foreclosed by our government edicts
%precedents. \textit{Post}, at 1515. That interested parties have pursued
%ambitious readings of our precedents does not mean those precedents are
%incapable of providing meaningful guidance to us or to Congress.}
%
%Moving on from the text, Georgia invokes what it views as the official position
%of the Copyright Office, as reflected in the Compendium of U.S. Copyright Office
%Practices (Compendium). But, as Georgia concedes, the Compendium is a
%non-binding administrative manual that at most merits deference under
%\textit{Skidmore v. Swift \& Co.}, 323 U.S. 134, 65 S.Ct. 161, 89 L.Ed. 124
%(1944). That means we must follow it only to the extent it has the ``power to
%persuade.'' \textit{Id.}, at 140, 65 S.Ct. 161. Because our precedents answer
%the question before us, we find any competing guidance in the Compendium
%unpersuasive.
%
%In any event, the Compendium is largely consistent with our decision. Drawing on
%\textit{Banks}, it states that, ``[a]s a matter of longstanding public policy,
%the U.S. Copyright Office will not register a government edict that has been
%issued by any state, local, or territorial government, including legislative
%enactments, judicial decisions, administrative rulings, public ordinances,
%\textit{or similar types of official legal materials.}'' Compendium
%\S~313.6(C)(2) (rev. 3d ed. 2017) (emphasis added). And, under \textit{Banks},
%what counts as a ``similar'' material depends on what kind of officer created
%the material (\textit{i.e.}, a judge) and whether the officer created it in the
%course of official (\textit{i.e.}, judicial) duties. See Compendium
%\S~313.6(C)(2) (quoting \textit{Banks}, 128 U.S. at 253, 9 S.Ct. 36, for the
%proposition that copyright cannot vest ``in the products of the labor done by
%judicial officers in the discharge of their judicial duties'').
%
%The Compendium goes on to observe that ``the Office may register annotations
%that summarize or comment upon legal materials ... unless the annotations
%themselves have the force of law.'' Compendium \S~313.6(C)(2). But that broad
%statement--- true of annotations created by officials such as court reporters that
%lack the authority to make or interpret the law---does not engage with the
%critical issue of annotations created \textit{by judges or legislators} in their
%official capacities. Because the Compendium does not address that question and
%otherwise echoes our government edicts precedents, it is of little relevance
%here.

Georgia also appeals to the overall purpose of the Copyright Act to promote the
creation and dissemination of creative works. Georgia submits that, without
copyright protection, Georgia and many other States will be unable to induce
private parties like Lexis to assist in preparing affordable annotated codes for
widespread distribution. That appeal to copyright policy, however, is addressed
to the wrong forum. As Georgia acknowledges, ``[I]t is generally for Congress,
not the courts, to decide how best to pursue the Copyright Clause's
objectives.'' And that principle requires adherence to precedent when, as
here, we have construed the statutory text and ``tossed [the ball] into
Congress's court, for acceptance or not as that branch elects.''

Turning to our government edicts precedents, Georgia insists that they can and
should be read to focus exclusively on whether a particular work has ``the force
of law.''\ldots But that framing has multiple flaws.

Most obviously, it cannot be squared with the reasoning or results of our
cases---especially \textit{Banks}. \textit{Banks}, following \textit{Wheaton}
and the
``judicial consensus'' it inspired, denied copyright protection to judicial
opinions without excepting concurrences and dissents that carry no legal force.
As every judge learns the hard
way, ``comments in [a] dissenting opinion'' about legal principles and
precedents ``are just that: comments in a dissenting opinion.''
Yet such comments are covered by the government edicts doctrine
because they come from an official with authority to make and interpret the
law.\ldots

%Indeed, \textit{Banks} went even further and withheld copyright protection from
%headnotes and syllabi produced by judges. 128 U.S. at 253, 9 S.Ct. 36. Surely
%these supplementary materials do not have the force of law, yet they are covered
%by the doctrine. The simplest explanation is the one  \textit{Banks} provided:
%These non-binding works are not copyrightable because of who creates them---judges
%acting in their judicial capacity. See \textit{ibid.}

The same goes for non-binding legislative materials produced by legislative
bodies acting in a legislative capacity. There is a broad array of such works
ranging from floor statements to proposed bills to committee reports. Under the
logic of Georgia's ``force of law'' test, States would own such materials and
could charge the public for access to them.\ldots

%Furthermore, despite Georgia's and Justice THOMAS's purported concern for the
%text of the Copyright Act, their conception of the government edicts doctrine
%has \textit{less} of a textual footing than the traditional formulation. The
%textual basis for the doctrine is the Act's ``authorship'' requirement, which
%unsurprisingly focuses on--- the author. Justice THOMAS urges us to dig deeper to
%``the root'' of our government edicts precedents. \textit{Post}, at 1515. But,
%in our view, the text \textit{is} the root. The Court long ago interpreted the
%word ``author'' to exclude officials empowered to speak with the force of law,
%and Congress has carried that meaning forward in multiple iterations of the
%Copyright Act. This textual foundation explains why the doctrine distinguishes
%between some authors (who are empowered to speak with the force of law) and
%others (who are not). Compare \textit{Callaghan}, 128 U.S. at 647, 9 S.Ct. 177,
%with \textit{Banks}, 128 U.S. at 253, 9 S.Ct. 36. But the Act's reference to
%``authorship'' provides no basis for Georgia's rule distinguishing between
%different categories of content with different
%effects.\readingfootnote{5}{Instead of accepting our predecessors' textual
%reasoning at face value, Justice THOMAS conjures a trinity of alternative
%``origin[s] and justification[s]'' for the government edicts doctrine that the
%Court \textit{might} have had in mind. See \textit{post}, at 1515-1517. Without
%committing to one or all of these possibilities, Justice THOMAS suggests that
%each would yield a rule that requires federal courts to pick out the subset of
%judicial and legislative materials that independently carry the force of law.
%But a Court motivated by Justice THOMAS's three-fold concerns might just as
%easily have read them as supporting a rule that prevents the officials
%responsible for creating binding materials from qualifying as an ``author.''
%Regardless, it is more ``\,`[]consistent with the judicial role'' to apply the
%reasoning and results the Court voted on and committed to writing than to
%speculate about what practical considerations our predecessors ``may have had
%... in mind,'' what history ``may [have] suggest[ed],'' or what constitutional
%concerns ``may have animated'' our government edicts precedents. \textit{Ibid.}}

Georgia minimizes the OCGA annotations as non-binding and non-authoritative, but
that description undersells their practical significance. Imagine a Georgia
citizen interested in learning his legal rights and duties. If he reads the
economy-class version of the Georgia Code available online, he will see laws
requiring political candidates to pay hefty qualification fees (with no
indigency exception), criminalizing broad categories of consensual sexual
conduct, and exempting certain key evidence in criminal trials from standard
evidentiary limitations---with no hint that important aspects of those laws have
been held unconstitutional by the Georgia Supreme Court.
Meanwhile, first-class
readers with access to the annotations will be assured that these laws are, in
crucial respects, unenforceable relics that the legislature has not bothered to
narrow or repeal.

If everything short of statutes and opinions were copyrightable, then States
would  be free to offer a whole range of premium legal works for those who can
afford the extra benefit. A State could monetize its entire suite of legislative
history. With today's digital tools, States might even launch a subscription or
pay-per-law service.

There is no need to assume inventive or nefarious behavior for these concerns to
become a reality. Unlike other forms of intellectual property, copyright
protection is both instant and automatic. It vests as soon as a work is captured
in a tangible form, triggering a panoply of exclusive rights that can last over
a century. If Georgia were correct, then unless a
State took the affirmative step of transferring its copyrights to the public
domain, all of its judges' and legislators' non-binding legal works would be
copyrighted. And citizens, attorneys, nonprofits, and private research companies
would have to cease all copying, distribution, and display of those works or
risk severe and potentially criminal penalties. Some affected
parties might be willing to roll the dice with a potential fair use defense. But
that defense, designed to accommodate First Amendment concerns, is notoriously
fact sensitive and often cannot be resolved without a trial.
The less bold among us would have to think
twice before using official legal works that illuminate the law we are all
presumed to know and understand.

Thankfully, there is a clear path forward that avoids these concerns---the one
we
are already on. Instead of examining whether given material carries ``the force
of law,'' we ask only whether the author of the work is a judge or a legislator.
If so, then whatever work that judge or legislator produces in the course of his
judicial or legislative duties is not copyrightable. That is the framework our
precedents long ago established, and we adhere to those precedents today.



\readinghead{* * *}

For the foregoing reasons, we affirm the judgment of the Eleventh Circuit.

\textit{It is so ordered.}

[Dissents by Justices Thomas and Ginsburg are omitted.]

%\opinion Justice \textsc{Thomas}, with whom Justice \textsc{Alito} joins, and
%with whom Justice \textsc{Breyer} joins as to all but Part II-A and footnote 6,
%dissenting.
%
%According to the majority, this Court's 19th-century ``government edicts''
%precedents clearly stand for the proposition that ``judges and legislators
%cannot serve as authors [for copyright purposes] when they produce works in
%their official capacity.'' \textit{Ante}, at 1509. And, after straining to
%conclude that the Georgia Code Revision Commission (Commission) is an arm of the
%Georgia Legislature, \textit{ante}, at 1508-1509, the majority concludes that
%Georgia cannot hold a copyright in the annotations that are included as part of
%the Official Code of Georgia Annotated (OCGA). This ruling will likely come as a
%shock to the 25 other jurisdictions---22 States, 2 Territories, and the District
%of Columbia---that rely on arrangements similar to Georgia's to produce annotated
%codes. See Brief for State of Arkansas et al. as \textit{Amici Curiae} 15, and
%App. to \textit{id.}, at 1. Perhaps these jurisdictions all overlooked this
%Court's purportedly clear guidance. Or perhaps the widespread use of these
%arrangements indicates that today's decision extends the government edicts
%doctrine to a new context, rather than simply ``confirm[ing]'' what the
%precedents have always held. See \textit{ante}, at 1505-1506. Because I believe
%we should ``leave to Congress the task of deciding whether the Copyright Act
%needs an upgrade,'' \textit{American Broadcasting Cos. v. Aereo, Inc.}, 573 U.S.
%431, 463, 134 S.Ct.  2498, 189 L.Ed.2d 476 (2014) (Scalia, J., dissenting), I
%respectfully dissent.
%
%
%
%\readinghead{I}
%
%Like the majority, I begin with the three 19th-century precedents that the
%parties agree provide the foundation for the government edicts doctrine.
%
%In \textit{Wheaton v. Peters}, 8 Pet. 591, 8 L.Ed. 1055 (1834), the Court first
%regarded it as self-evident that judicial opinions cannot be copyrighted either
%by the judges who signed them or by a reporter under whose auspices they are
%published. Congress provided that, in return for a salary of \$1,000, the
%Reporter of Decisions for this Court would prepare reports consisting of
%judicial opinions and additional materials summarizing the cases. \textit{Id.},
%at 614, 617 (argument). Wheaton, one of this Court's earliest Reporters, argued
%that he owned a copyright for the entirety of his reports. He contended that he
%had ``acquired the right to the opinions by judges' gift'' once they became a
%part of his volume. \textit{Id.}, at 614 (same). The Court ultimately remanded
%on the question whether Wheaton had complied with the Copyright Act's procedural
%requirements. \textit{Id.}, at 667-668. In doing so, it observed in dicta that
%``the court [was] unanimously of [the] opinion, that no reporter has or can have
%any copyright in the written opinions delivered by this court; and that the
%judges thereof cannot confer on any reporter any such right.'' \textit{Id.}, at
%668.
%
%Fifty-four years later, the Court returned to the same subject, suggesting a
%doctrinal basis for the rule that judicial opinions and certain closely related
%materials cannot be copyrighted. In \textit{Banks v. Manchester}, 128 U.S. 244,
%9 S.Ct. 36, 32 L.Ed. 425 (1888), the state-authorized publisher of the Ohio
%Supreme Court's decisions, Banks \& Brothers, sued a competing publisher for
%copyright infringement. The competing publisher reproduced portions from Banks'
%reports, including Ohio Supreme Court decisions, statements of the cases, and
%syllabi, all of which were originally prepared by the opinion's authoring judge.
%This Court held that these materials were not the proper subject of copyright.
%In reaching that conclusion, the Court grounded its analysis in its
%interpretation of the word ``author'' in the Copyright Act. It anchored this
%interpretation in the ``public policy'' that ``the judge who, in his judicial
%capacity, prepares the opinion or decision [and other materials]'' is not
%``regarded as their author or their proprietor, in the sense of [the Copyright
%Act], so as to be able to confer any title by assignment.'' \textit{Banks}, 128
%U.S. at 253, 9 S.Ct. 36. The Court supported this conclusion by stating that
%``there has always been a judicial consensus ... that no copyright could[,]
%under the statutes passed by Congress, be secured in the products of the labor
%done by judicial officers in the discharge of their judicial duties.''
%\textit{Ibid.} (emphasis deleted). And the Court observed that this rule
%reflected the view that the ``authentic exposition and interpretation of the law
%... is free for publication to all,'' which in turn prevents a judge from
%qualifying as an author. \textit{Ibid.}
%
%Importantly, the Court also briefly discussed whether the State of Ohio could
%directly hold the copyright. In answering this question, the Court did not
%suggest that States were categorically prohibited from holding copyrights as
%authors or assignees. Instead, the Court simply noted that the State fell
%outside the scope of the Act because it was not a ``resident'' or ``citizen of
%the United States,'' as then required by statute, and because it did not meet
%other statutory criteria. \textit{Ibid.} The Court felt it necessary to observe,
%however, that ``[w]hether the State could take out a copyright for itself, or
%could enjoy  the benefit of one taken out by an individual for it, as the
%assignee of a citizen of the United States or a resident therein, who should be
%the author of a book, is a question not involved in the present case, and we
%refrain from considering it.'' \textit{Ibid.}
%
%Finally, in \textit{Callaghan v. Myers}, 128 U.S. 617, 9 S.Ct. 177, 32 L.Ed. 547
%(1888), the Court addressed the limits of the government edicts doctrine. In
%that case, the Court settled another dispute between a publisher of court
%decisions and an alleged infringer. The plaintiff purchased the proprietary
%rights to the reports prepared by the Illinois Supreme Court's reporter of
%decisions, Freeman, including the copyright to the reports. Unlike in
%\textit{Banks}, these reports also contained material authored by Freeman.
%\textit{Callaghan}, 128 U.S. at 645, 9 S.Ct. 177. The alleged infringers copied
%the judicial decisions and Freeman's materials. In finding for the plaintiff,
%this Court reiterated that ``there can be no copyright in the opinions of the
%judges, or in the work done by them in their official capacity as judges.''
%\textit{Id.}, at 647, 9 S.Ct. 177 (citing \textit{Banks}, 128 U.S. 244, 9 S.Ct.
%36, 32 L.Ed. 425). But the Court concluded that ``no [similar] ground of public
%policy'' justified denying a state official a copyright ``cover[ing] the matter
%which is the result of his intellectual labor.'' \textit{Callaghan}, 128 U.S. at
%647, 9 S.Ct. 177.
%
%
%
%\readinghead{II}
%
%These precedents establish that judicial opinions cannot be copyrighted. But
%they do not exclude from copyright protection notes that are prepared by an
%official court reporter and published together with the reported opinions. There
%is no apparent reason why the same logic would not apply to statutes and
%regulations. Thus, it must follow from our precedents that statutes and
%regulations cannot be copyrighted, but accompanying notes lacking legal force
%can be. See \textit{Howell v. Miller}, 91 F. 129 (CA6 1898) (Harlan, J.)
%(explaining that, under \textit{Banks} and \textit{Callaghan}, annotations to
%Michigan statutes could be copyrighted).
%
%
%
%\readinghead{A}
%
%It is fair to say that the Court's 19th-century decisions do not provide any
%extended explanation of the basis for the government edicts doctrine. The
%majority is nonetheless content to accept these precedents reflexively, without
%examining the origin or validity of the rule they announced. For the majority,
%it is enough that the precedents established a rule that ``seemed too obvious to
%adorn with further explanation.'' \textit{Ante}, at 1506. But the contours of
%the rule were far from clear, and to understand the scope of the doctrine, we
%must explore its underlying rationale.
%
%In my view, the majority's uncritical extrapolation of precedent is inconsistent
%with the judicial role. An unwillingness to examine the root of a precedent has
%led to the sprouting of many noxious weeds that distort the meaning of the
%Constitution and statutes alike. Although we have not been asked to revisit
%these precedents, it behooves us to explore the origin of and justification for
%them, especially when we are asked to apply their rule for the first time in
%over 130 years.
%
%The Court's precedents suggest three possible grounds supporting their
%conclusion. In \textit{Banks}, the Court referred to the meaning of the term
%``author'' in copyright law. While the Court did not develop this argument, it
%is conceivable that the contemporaneous public meaning of the term ``author''
%was narrower in the copyright context than in ordinary speech. At the time this
%Court decided \textit{Banks}, the Copyright Act provided protection for books,
%maps, prints, engravings, musical and dramatic compositions, photographs, and
%works of art.\readingfootnote{6}{See 1 Stat. 124; 2 Stat. 171; ch. 16, 4 Stat.
%436; 11 Stat. 138-139; 13 Stat. 540; 16 Stat. 212.} Judicial opinions differ
%markedly from these works. Books, for instance, express the thoughts of their
%authors. They typically have no power beyond the ability of their words to
%influence readers, and they usually are published at private expense. Judicial
%opinions, on the other hand, do not simply express the thoughts of the judges
%who write or endorse them. Instead, they elaborate and apply rules of law that,
%in turn, represent the implementation of the will of the people. Unlike other
%copyrightable works of authorship, judicial opinions have binding legal effect,
%and they are produced and issued at public expense. Moreover, copyright law
%understands an author to be one whose work will be encouraged by the grant of an
%exclusive right. See \textit{Kirtsaeng v. John Wiley \& Sons, Inc.}, 579 U.S.
%\_\_\_, \_\_\_, 136 S.Ct. 1979, 1986, 195 L.Ed.2d 368 (2016). But judges, when
%acting in an official capacity, do not fit that description. The Court in
%\textit{Banks} may have had these differences in mind when it concluded that a
%judge fell outside the scope of the term ``author.'' 128 U.S. at 253, 9 S.Ct.
%36.
%
%History may also suggest a narrower meaning of ``author'' in the copyright
%context. In England, at least as far back as 1666, courts and commentators
%agreed ``that the property of all law books is in the king, because he pays the
%judges who pronounce the law.'' G. Curtis, Law of Copyright 130 (1847); see also
%\textit{Banks \& Bros. v. West Publishing Co.}, 27 F. 50, 57 (CC Minn. 1886)
%(citing English cases and treatises and concluding that ``English courts
%generally sustain the crown's proprietary rights in judicial opinions'').
%Blackstone described this as a ``prerogative copyrigh[t],'' explaining that
%``[t]he king, as the executive magistrate, has the right of promulging to the
%people all acts of state and government. This gives him the exclusive privilege
%of printing, at his own press, or that of his grantees, all acts of parliament,
%proclamations, and orders of council.'' 2 W. Blackstone, Commentaries on the
%Laws of England 410 (1766) (emphasis deleted); see also \textit{Wheaton}, 8 Pet.
%at 659-660. This history helps to explain the dearth of cases permitting
%individuals to obtain copyrights in judicial opinions. But under the
%Constitution, sovereignty lies with the people, not a king. See The Federalist
%No. 22, p. 152 (C. Rossiter ed. 1961); \textit{id.}, No. 39, at 241. The English
%historical practice, when superimposed on the Constitution's recognition that
%sovereignty resides in the people, helps to explain the Court's conclusion that
%the ``authentic exposition and interpretation of the law ... is free for
%publication to all.'' \textit{Banks}, 128 U.S. at 253, 9 S.Ct. 36.
%
%Finally, concerns of fair notice, often recognized by this Court's precedents as
%an important component of due process, also may have animated the reasoning of
%these 19th-century cases. As one court put it, ``[t]he decisions and opinions of
%the justices are the authorized expositions and interpretations of the laws,
%which are binding upon all the citizens.... Every citizen is presumed to know
%the law thus declared, and it needs no argument to show that justice requires
%that all should have free access to the opinions.'' \textit{Nash v. Lathrop},
%142 Mass. 29, 35, 6 N.E. 559, 560 (1886) (cited in \textit{Banks}, 128 U.S. at
%253-254, 9 S.Ct. 36); see also \textit{American Soc. for Testing and Materials
%v. Public.Resource.Org. Inc.}, 896 F.3d 437, 458-459 (CADC 2018) (Katsas, J.,
%concurring).
%
%
%
%\readinghead{ B}
%
%Allowing annotations to be copyrighted does not run afoul of any of these
%possible justifications for the government edicts doctrine. First, unlike
%judicial opinions and statutes, these annotations do not even purport to embody
%the will of the people because they are not law. The General Assembly of Georgia
%has made abundantly clear through a variety of provisions that the annotations
%do not create any binding obligations. OCGA \S~1-1-7 states that ``[a]ll
%historical citations, title and chapter analyses, and notes set out in this Code
%are given for the purpose of convenient reference and do not constitute part of
%the law.'' Section 1-1-1 further provides that ``[t]he statutory portion of the
%codification of Georgia laws ... is enacted and shall have the effect of
%statutes enacted by the General Assembly of Georgia. The statutory portion of
%such codification shall be merged with annotations ... and other materials ...
%and shall be published by authority of the state.'' Thus, although the materials
%``merge'' prior to publication in the ``official'' code, the very provision
%calling for that merger makes clear that the annotations serve as commentary,
%not law.
%
%As additional evidence that the annotations do not represent the will of the
%people, the General Assembly does not enact statutory annotations under its
%legislative power. See Ga. Const., Art. III, \S~1, ¶ 1 (vesting the legislative
%power in the General Assembly). To enact state law, Georgia employs a process of
%bicameralism and presentment similar to that embodied in the United States
%Constitution. See Ga. Const., Art. III, \S~5; Art. V, \S~2, ¶ 4. The annotations
%do not go through this process, a fact that even the majority must acknowledge.
%\textit{Ante}, at 1508-1509; Ga. S. 52, Reg. Sess., \S~54(b) (2019-2020)
%(``Annotations... except as otherwise provided in the Code ... are not enacted
%as statutes by the provisions of this Act'').
%
%Second, unlike judges and legislators, the creators of annotations are
%incentivized by the copyright laws to produce a desirable product that will
%eventually earn them a profit. And though the Commission may require Lexis to
%follow strict guidelines, the independent synthesis, analysis, and creative
%drafting behind the annotations makes them analogous to other copyrightable
%materials. See Brief for Matthew Bender \& Co., Inc., as \textit{Amicus Curiae}
%4-7.
%
%Lastly, the annotations do not impede fair notice of the laws. As just stated,
%the annotations do not carry the binding force of law. They simply summarize
%independent sources of legal information and consolidate them in one place.
%Thus, OCGA annotations serve a similar function to other copyrighted research
%tools provided by private parties such as the American Law Reports and Westlaw,
%which also contain information of great ``practical significance.''
%\textit{Ante}, at 1512. Compare, \textit{e.g.}, OCGA \S~34-9-260 (annotation for
%\textit{Cho Carwash Property, L.L.C. v. Everett}, 326 Ga.App. 6, 755 S.E.2d 823
%(2014)) with Ga. Code Ann. \S~34-9-260 (Westlaw's annotation for the same).
%
%The majority resists this conclusion, suggesting that without access to the
%annotations, readers of Georgia law will be unable to fully understand the true
%meaning of Georgia's statutory provisions, such as provisions that have been
%undermined or nullified by court decisions. \textit{Ante}, at 1512-1513. That is
%simply incorrect. As the majority tacitly concedes, a person seeking information
%about changes in Georgia statutory law can find that information by consulting
%the original source for the change in the law's status---the court decisions
%themselves. See \textit{ante}, at 1512-1513. The inability to access the OCGA
%merely deprives a researcher of one specific tool, not to the underlying factual
%or legal information summarized in that tool. See also \textit{post}, at 1524
%(GINSBURG, J., dissenting).\readingfootnote{7}{The majority contends that,
%rather than seeking to understand the origins of our precedents, we should
%simply accept the text of the opinions that the Justices ``voted on and
%committed to writing.'' \textit{Ante}, at 1512, n. 4. But that begs the
%question: What does the text of the relevant opinions tell us? The answer is not
%much. It is precisely this lack of explication that makes it necessary to
%explore the ``judicial \textit{consensus}'' and public policy referred to in
%\textit{Banks v. Manchester}, 128 U.S. 244, 253, 9 S.Ct. 36, 32 L.Ed. 425
%(1888). Instead, the majority attempts to dissect the language of our prior
%opinions in the same way it would interpret a statute, an approach we have
%repeatedly cautioned against. See \textit{St. Mary's Honor Center v. Hicks}, 509
%U.S. 502, 515, 113 S.Ct. 2742, 125 L.Ed.2d 407 (1993); \textit{Reiter v.
%Sonotone Corp.}, 442 U.S. 330, 341, 99 S.Ct. 2326, 60 L.Ed.2d 931 (1979). The
%proper approach is to ``read general language in judicial opinions ... as
%referring in context to circumstances similar to the circumstances then before
%the Court and not referring to quite different circumstances that the Court was
%not then considering.'' \textit{Illinois v. Lidster}, 540 U.S. 419, 424, 124
%S.Ct. 885, 157 L.Ed.2d 843 (2004); see also \textit{Cohens v. Virginia}, 6
%Wheat. 264, 399, 5 L.Ed. 257 (1821) (Marshall, C. J., for the Court)
%(``[G]eneral expressions, in every opinion, are to be taken in connection with
%the case in which those expressions are used. If they go beyond the case, they
%may be respected, but ought not to control the judgment in a subsequent suit
%when the very point is presented for decision'').}
%
%
%
%\readinghead{C}
%
%The text of the Copyright Act supports my reading of the
%precedents.\readingfootnote{8}{As the majority explains, \textit{ante}, at 1508,
%the annotations were created as part of a work-for-hire agreement between the
%Commission and Lexis. See 17 U.S.C. \S~201(b). Because no party disputes the
%validity of the contract, I express no opinion regarding whether the contract
%established an employer/employee relationship or whether the Commission may be
%considered a ``person'' under \S~201(b).} Specifically, there are four
%indications in the text of the Copyright Act that the OCGA annotations are
%copyrightable. As an initial matter, the Act does not define the word
%``author,'' 17 U.S.C. \S~101, or make any reference to the government edicts
%doctrine. Accordingly, the term ``author'' itself does not shed any light on
%whether the doctrine covers statutory annotations. Second, while the Act
%excludes from copyright protection ``work[s] prepared by an officer or employee
%of the United States Government as part of that person's official duties,''
%\S~101; see also \S~105, the Act contains no similar prohibition against works
%of state governments or works prepared at their behest. ``Congress' use of
%explicit language in one provision cautions against inferring the same
%limitation'' elsewhere in the statute. \textit{State Farm Fire \& Casualty Co.
%v. United States ex rel. Rigsby}, 580 U.S. \_\_\_, \_\_\_, 137 S.Ct. 436, 442,
%196 L.Ed.2d 340 (2016) (internal quotation marks omitted); \textit{Pacific
%Operators Offshore, LLP v. Valladolid}, 565 U.S. 207, 216, 132 S.Ct. 680, 181
%L.Ed.2d 675 (2012). Third, the Act specifically notes that annotations are
%copyrightable derivative works. \S~101. Here, again, the Act does not expressly
%exclude from copyright protection annotations created either by the State or at
%the State's request. Fourth, the Act provides that an author may hold a
%copyright in ``material contributed'' in a derivative work, ``as distinguished
%from the preexisting material employed in the work.'' \S~103(b); see also
%\textit{Feist Publications, Inc. v. Rural Telephone Service Co.}, 499 U.S. 340,
%359, 111 S.Ct. 1282, 113 L.Ed.2d 358 (1991). These aspects of the statutory
%text, taken together, further support the conclusion that the OCGA annotations
%are copyrightable.
%
%For all these reasons, I would conclude that, as with the privately created
%annotations in \textit{Callaghan}, Georgia's statutory annotations at issue in
%this case are copyrightable.
%
%
%
%\readinghead{ III}
%
%The majority reads this Court's precedents differently. In its view, the Court
%in \textit{Banks} held that judges are not ``authors'' within the scope of the
%Copyright Act for ``whatever work they perform in their capacity as judges,''
%128 U.S. at 253, 9 S.Ct. 36, so the same must be true for legislators, see
%\textit{ante}, at 1507-1508. Accordingly, works created by legislators in their
%legislative capacity are not ``original works of authorship,'' \S~102, and
%therefore cannot be copyrighted. This argument is flawed in multiple respects.
%
%
%
%\readinghead{A}
%
%Most notably, the majority's textual analysis hinges on accepting that its
%construction of ``authorship,'' \textit{i.e.}, all works produced in a judge's
%or legislator's official capacity, was so well established by our 19th-century
%precedents that Congress incorporated it into the multiple revisions of the
%Copyright Act. See \textit{ante}, at 1509 1510. Such confidence is questionable,
%to say the least.
%
%The majority's understanding of the government edicts doctrine seems to have
%been lost on dozens of States and Territories, as well as the lower courts in
%this case. As already stated, the 25 jurisdictions with official annotated codes
%apparently did not view this Court's precedents as establishing the ``official
%duties'' definition of authorship. See Brief for State of Arkansas et al. as
%\textit{Amici Curiae.}\readingfootnote{9}{According to one study published in
%2000, approximately half of States owned copyright in official state statutory
%compilations, court reports, or administrative regulations. Dmitrieva, State
%Ownership of Copyrights in Primary Law Materials, 23 Hastings Com. \&
%Entertainment L. J. 81, 83, 97-105 (2000). The majority attempts to undermine
%this study by emphasizing that some of these States owned copyright in primary
%law materials. \textit{Ante}, at 1511-1512, n. 3. This misunderstands the point.
%I do not claim that this evidence demonstrates that the States necessarily
%interpreted the government edicts doctrine correctly. I merely point out that
%these divergent practices seriously undercut the majority's claim that its
%interpretation of ``authorship'' was well settled and universally understood. On
%this score, the majority has no answer but to insinuate that the lawmakers of
%over half the Nation's jurisdictions disregarded federal law and the
%Constitution to pursue their own agendas in the face of supposedly clear
%precedent.} And if ``our precedents answer the question'' so clearly,
%\textit{ante}, at 1510-1511, one wonders why the Eleventh Circuit reached its
%conclusion in such a roundabout fashion. Rather than following the majority's
%``straightforward'' path, \textit{ante}, at 1505-1506, the Eleventh Circuit
%looked to the ``zone of indeterminacy at the frontier between edicts that carry
%the force of law and those that do not'' to determine whether the annotations
%were ``sufficiently law-like'' to be ``constructively authored by the People.''
%\textit{Code Revision Comm'n v. Public.Resource.Org, Inc.}, 906 F.3d 1229, 1233,
%1242, 1243 (2018). The District Court likewise does not appear to have viewed
%the question as well settled. In a cursory analysis, it determined that the
%annotations were copyrightable based on \textit{Callaghan. Code Revision Comm'n
%v. Public.Resource.Org, Inc.}, 244 F.Supp.3d 1350, 1356 (ND Ga. 2017). It is
%risible to presume that Congress had knowledge of and incorporated a ``settled''
%meaning that eluded a multitude of States and Territories, as well as at least
%four Article III judges. \textit{Ante}, at 1510. Cf. \textit{Rimini Street, Inc.
%v. Oracle USA, Inc.}, 586 U.S. \_\_\_, \_\_\_-\_\_\_, 139 S.Ct. 873, 880-881,
%203 L.Ed.2d 180 (2019).
%
%This presumption of congressional knowledge also provides the basis for the
%majority's conclusion that the annotations are not ``original works of
%authorship.'' See \textit{ante}, at 1509-1510 (discussing \S~101). Stripped of
%the fiction that this Court's  19th-century precedents clearly demonstrated that
%``authorship'' encompassed all works performed as part of a legislator's duties,
%the majority's textual argument fails.
%
%The majority does not confront this criticism head on. Instead, it simply
%repeats, without any further elaboration, its unsupported conclusion that
%``[t]he Court long ago interpreted the word `author' to exclude officials
%empowered to speak with the force of law, and Congress has carried that meaning
%forward in multiple iterations of the Copyright Act.'' \textit{Ante}, at 1512.
%This wave of the ``magic wand of \textit{ipse dixit}'' does nothing to
%strengthen the majority's argument, and in fact only serves to underscore its
%weakness. \textit{United States v. Yermian}, 468 U.S. 63, 77, 104 S.Ct. 2936, 82
%L.Ed.2d 53 (1984) (Rehnquist, J., dissenting).\readingfootnote{10}{The
%majority's approach is also hard to reconcile with the recognition in
%\textit{Wheaton v. Peters}, 8 Pet. 591, 8 L.Ed. 1055 (1834), that annotations
%prepared by the Reporter of Decisions could be copyrighted. Wheaton was paid a
%salary of \$1,000, and it is difficult to say whether this salary funded his
%work on the opinions or his work on the annotations. See \textit{id.}, at 614,
%617 (argument).}
%
%
%
%\readinghead{B}
%
%In addition to its textual deficiencies, the majority's understanding of this
%Court's precedents fails to account for the critical differences between the
%role that judicial opinions play in expounding upon the law compared to that of
%statutes. The majority finds it meaningful, for instance, that \textit{Banks}
%prohibited dissents and concurrences from being copyrighted, even though they
%carry no legal force. \textit{Ante}, at 1511-1512. At an elementary level, it is
%true that the judgment is the only part of a judicial decision that has legal
%effect. But it blinks reality to ignore that every word of a judicial
%opinion---whether it is a majority, a concurrence, or a dissent---expounds upon the
%law in ways that do not map neatly on to the legislative function. Setting aside
%summary decisions, the reader of a judicial opinion will always gain critical
%insight into the reasoning underlying a judicial holding by reading all opinions
%in their entirety. Understanding the reasoning that animates the rule in turn
%provides pivotal insight into how the law will likely be applied in future
%judicial opinions.\readingfootnote{11}{For instance, this Court has not
%overruled \textit{Lemon v. Kurtzman}, 403 U.S. 602 (1971), which pronounced a
%test for evaluating Establishment Clause claims. But a reader would do well to
%carefully scrutinize the various opinions in \textit{American Legion v. American
%Humanist Assn.}, 588 U.S. \_\_\_, 139 S.Ct. 2067, 204 L.Ed.2d 452 (2019), to
%understand the markedly different way that this precedent functions in our
%current jurisprudence compared to when it was first decided. Moreover, sometimes
%a separate writing takes on canonical status, like Justice Jackson's concurrence
%regarding the executive power in \textit{Youngstown Sheet \& Tube Co. v.
%Sawyer}, 343 U.S. 579, 634-638, 72 S.Ct. 863, 96 L.Ed. 1153 (1952) (opinion
%concurring in judgment and opinion of the Court); see also \textit{Katz v.
%United States}, 389 U.S. 347, 360-361, 88 S.Ct. 507, 19 L.Ed.2d 576 (1967)
%(Harlan, J., concurring) (reasonable expectation of privacy Fourth Amendment
%test). Still other times, the reasoning in an opinion for less than a majority
%of the Court provides the explicit basis for a later majority's holding. See,
%\textit{e.g.}, \textit{McKinney v. Arizona}, 589 U.S. \_\_\_, \_\_\_, 140 S.Ct.
%702, 707-708, 206 L.Ed.2d 69 (2020) (discussing \textit{Ring v. Arizona}, 536
%U.S. 584, 612, 122 S.Ct. 2428, 153 L.Ed.2d 556 (2002) (Scalia J., concurring));
%\textit{Estelle v. Gamble}, 429 U.S. 97, 102, 97 S.Ct. 285, 50 L.Ed.2d 251
%(1976) (incorporating into the majority the Eighth Amendment ``evolving
%standards of decency'' test first announced in \textit{Trop v. Dulles}, 356 U.S.
%86, 101, 78 S.Ct. 590, 2 L.Ed.2d 630 (1958) (plurality opinion)). Even
%``\,`comments in [a] dissenting opinion,'\,'' \textit{ante}, at 1511, sometimes
%reemerge as the foundational reasoning in a majority opinion. See,
%\textit{e.g.}, \textit{Franchise Tax Bd. of Cal. v. Hyatt}, 587 U.S. \_\_\_, 139
%S.Ct. 1485, 203 L.Ed.2d 768 (2019) (discussing \textit{Nevada v. Hall}, 441 U.S.
%410, 433-439, 99 S.Ct. 1182, 59 L.Ed.2d 416 (1979) (Rehnquist, J., dissenting));
%\textit{Lawrence v. Texas}, 539 U.S. 558, 578, 123 S.Ct. 2472, 156 L.Ed.2d 508
%(2003) (``Justice STEVENS' [dissenting] analysis, in our view, should have been
%controlling in \textit{Bowers [v. Hardwick}, 478 U.S. 186, 106 S.Ct. 2841, 92
%L.Ed.2d 140 (1986),] and should control here''). These examples, and myriad
%more, demonstrate that the majority treats the role of separate judicial
%opinions in an overly simplistic fashion.} Thus, deprived of access to judicial
%opinions, individuals cannot access the primary, and therefore best, source of
%information for the meaning of the law.\readingfootnote{12}{\textit{Banks} also
%stated that judicially prepared syllabi and headnotes cannot be copyrighted. 128
%U.S. at 253, 9 S.Ct. 36. The majority cites these materials as further evidence
%of its broad rule, because the majority finds it beyond cavil that ``these
%supplementary materials do not have the force of law.'' \textit{Ante}, at 1511.
%The majority feels it appropriate to assume ---without any historical inquiry---that
%the words ``syllabus'' and ``headnote'' carried the same meaning, or served the
%same function, in 1888 as they do now. Without briefing on this issue, I am not
%willing to make that leap. See \textit{Hixson v. Burson}, 54 Ohio St. 470, 485,
%43 N.E. 1000, 1003 (1896) (``reluctantly overrul[ing] the second syllabus'' of a
%previous decision); \textit{Holliday v. Brown}, 34 Neb. 232, 234, 51 N.W. 839,
%840 (1892) (``It is an unwritten rule of this court that members thereof are
%bound only by the points as stated in the syllabus of each case''); see also
%\textit{Frazier v. State}, 15 Ga.App. 365, 365-367, 83 S.E. 273, 273-274 (1914)
%(clarifying the meaning of a court-written headnote and emphasizing that to
%understand an opinion's meaning, the headnote and opinion must be read
%together); \textit{United States v. Detroit Timber \& Lumber Co.}, 200 U.S. 321,
%337, 26 S.Ct. 282, 50 L.Ed. 499 (1906) (acknowledging that some state statutes
%rendered headnotes the work of the court carrying legal force).} And as true as
%that is today, access to these opinions was even more essential in the 19th
%century before the proliferation of federal and state regulatory law
%fundamentally altered the role that common-law judging played in expounding upon
%the law. See also \textit{post}, at 1523 (GINSBURG, J., dissenting).
%
%These differences provide crucial context for \textit{Banks'} reasoning.
%Specifically, to ensure that judicial ``exposition and interpretation of the
%law'' remains ``free for publication to all,'' the word ``author'' must be read
%to encompass all judicial duties. \textit{Banks}, 128 U.S. at 253, 9 S.Ct. 36.
%But these differences also demonstrate that the same rule does not \textit{a
%fortiori} apply to all legislative duties.\readingfootnote{13}{Although
%legislative history is not at issue in this case, the majority also contends
%that its rule is necessary to fend off the possibility that ``[a] State could
%monetize its entire suite of legislative history.'' \textit{Ante}, at 1513.
%Putting aside the jurisprudential debate over the use of such materials in
%interpreting federal statutes, many States can, and have, specifically
%authorized courts to consider legislative history when construing statutes. See,
%\textit{e.g.}, Colo. Rev. Stat. \S~2-4-203(1)(c) (2019); Iowa Code \S~4.6(3)
%(2019); Minn. Stat. \S~645.16(7) (2018); N. M. Stat. Ann. \S~12-2A-20(C)(2)
%(2019); N. D. Cent. Code Ann. \S~1-02-39(3) (2019); Ohio Rev. Code Ann.
%\S~1.49(C) (Lexis 2019); 1 Pa. Cons. Stat. \S~1921(c)(7) (2016). Given the
%direct role that legislative history plays in the construction of statutes in
%these States, it is hardly clear that such States could subject their
%legislative histories to copyright.}
%
%
%
%\readinghead{C}
%
%In addition to being flawed as a textual and precedential matter, the majority's
%rule will prove difficult to administer. According to one group of
%\textit{amici}, nearly all jurisdictions with annotated codes use private
%contractors that ``almost invariably prepare [annotations] under the supervision
%of legislative-branch or judicial-branch officials, including state legislators
%or state-court judges.'' Brief for State of Arkansas et al. as \textit{Amici
%Curiae} 16-17. Under the majority's view, any one of these commissions or
%counsels could potentially be reclassified as an ``adjunct to the legislature.''
%\textit{Ante}, at 1509. But the majority's  test for ascertaining the true
%nature of these commissions raises far more questions than it answers.
%
%The majority lists a number of factors--- including the Commission's membership
%and funding, how the annotations become part of the OCGA, and descriptions of
%the Commission from court cases---to support its conclusion that the Commission is
%really part of the legislature. See \textit{ante}, at 1508-1509. But it does not
%specify whether these factors are exhaustive or illustrative and, if the latter,
%what other factors may be important. The majority also does not specify whether
%some factors weigh more heavily than others when deciding whether to deem an
%oversight body a legislative adjunct.
%
%And even when the majority does list concrete factors, pivotal guidance remains
%lacking. For example, the majority finds it meaningful that 9 out of the
%Commission's 15 members are legislators. \textit{Ante}, at 1508; see OCGA
%\S~28-9-2 (noting that the other members of the Commission include the State's
%Lieutenant Governor, a judge, a district attorney, and three other state bar
%members). But how many legislative members are needed for a commission to become
%a legislative adjunct? The majority provides no answers to any of these
%questions.
%
%
%
%\readinghead{* * *}
%
%The majority's rule will leave in the lurch the many States, private parties,
%and legal researchers who relied on the previously bright-line rule. Perhaps, to
%the detriment of all, many States will stop producing annotated codes
%altogether. Were that to occur, the majority's fear of an ``economy-class''
%version of the law will truly become a reality. See \textit{ante}, at 1512-1513.
%As Georgia explains, its contract enables the OCGA to be sold at a fraction of
%the cost of competing annotated codes. For example, Georgia asserts that Lexis
%sold the OCGA for \$404 in 2016, while West Publishing's competing annotated
%code sold for \$2,570. Should state annotated codes disappear, those without the
%means to pay the competitor's significantly higher price tag will have a
%valuable research tool taken away from them. Meanwhile, this Court, which is
%privileged to have access to numerous research resources, will scarcely notice.
%These negative practical ramifications are unfortunate enough when they reflect
%the deliberative legislative choices that we as judges are bound to respect.
%They are all the more regrettable when they are the result of our own meddling.
%Fortunately, as the majority and I agree, ``\,`critics of [today's] ruling can
%take their objections across the street, [where] Congress can correct any
%mistake it sees.'\,'' \textit{Ante}, at 1510 (quoting \textit{Kimble v. Marvel
%Entertainment, LLC}, 576 U.S. 446, 456, 135 S.Ct. 2401, 192 L.Ed.2d 463 (2015)).
%
%We have ``stressed ... that it is generally for Congress, not the courts, to
%decide how best to pursue the Copyright Clause's objectives,'' \textit{Eldred v.
%Ashcroft}, 537 U.S. 186, 212, 123 S.Ct. 769, 154 L.Ed.2d 683 (2003), because
%``it is Congress that has been assigned the task of defining the scope of the
%limited monopoly that should be granted to authors,'' \textit{Sony Corp. of
%America v. Universal City Studios, Inc.}, 464 U.S. 417, 429, 104 S.Ct. 774, 78
%L.Ed.2d 574 (1984). Because the majority has strayed from its proper role, I
%respectfully dissent.
%
%\vskip\baselineskip
%
%\textbf{Justice GINSBURG, with whom Justice BREYER joins, dissenting.}
%
%Beyond doubt, state laws are not copyrightable. Nor are other materials created
%by state legislators in the course of performing their lawmaking
%responsibilities, \textit{e.g.}, legislative committee reports, floor
%statements, unenacted bills. \textit{Ante}, at 1517-1518. Not all that
%legislators do, however,  is ineligible for copyright protection; the government
%edicts doctrine shields only ``works that are (1) created by judges and
%legislators (2) \textit{in the course of their judicial and legislative
%duties.}'' \textit{Ante}, at 1508 (emphasis added). The core question this case
%presents, as I see it: Are the annotations in the Official Code of Georgia
%Annotated (OCGA) done in a legislative capacity? The answer, I am persuaded,
%should be no.
%
%To explain why, I proceed from common ground. All agree that headnotes and
%syllabi for judicial opinions---both a kind of annotation---are copyrightable when
%created by a reporter of decisions, \textit{Callaghan v. Myers}, 128 U.S. 617,
%645-650, 9 S.Ct. 177, 32 L.Ed. 547 (1888), but are not copyrightable when
%created by judges, \textit{Banks v. Manchester}, 128 U.S. 244, 253, 9 S.Ct. 36,
%32 L.Ed. 425 (1888). That is so because ``[t]he whole work done by ... judges,''
%\textit{ibid.}, including dissenting and concurring opinions, ranks as work
%performed in their judicial capacity. Judges do not outsource their writings to
%``arm[s]'' or ``adjunct[s],'' cf. \textit{ante}, at 1508, 1509, to be composed
%in their stead. Accordingly, the judicial opinion-drafting process in its
%entirety---including the drafting of headnotes and syllabi, in jurisdictions where
%that is done by judges---falls outside the reach of copyright protection.
%
%One might ask: If a judge's annotations are not copyrightable, why are those
%created by legislators? The answer lies in the difference between the role of a
%judge and the role of a legislator. ``[T]o the judiciary'' we assign ``the duty
%of interpreting and applying'' the law, \textit{Massachusetts v. Mellon}, 262
%U.S. 447, 488, 43 S.Ct. 597, 67 L.Ed. 1078 (1923), and sometimes making the
%applicable law, see Friendly, In Praise of \textit{Erie}---and of the New Federal
%Common Law, 39 N. Y. U. L. Rev. 383 (1964). See also \textit{Marbury v.
%Madison}, 1 Cranch 137, 177, 2 L.Ed. 60 (1803) (``It is emphatically the
%province and duty of the judicial department to say what the law is.''). In
%contrast, the role of the legislature encompasses the process of ``making
%laws''---not construing statutes after their enactment. \textit{Mellon}, 262 U.S.
%at 488, 43 S.Ct. 597; see \textit{Patchak v. Zinke}, 583 U.S. \_\_\_, \_\_\_ 138
%S.Ct. 897, 905, 200 L.Ed.2d 92 (2018) (plurality opinion) (``[T]he legislative
%power is the power to make law.''). The OCGA annotations, in my appraisal, do
%not rank as part of the Georgia Legislature's \textit{lawmaking process} for
%three reasons.
%
%First, the annotations are not created contemporaneously with the statutes to
%which they pertain; instead, the annotations comment on statutes already
%enacted. See, \textit{e.g.}, App. 268-269 (text of enacted laws are transmitted
%to the publisher for the addition of commentary); \textit{id.}, at 403-404
%(publisher adds new case notes on a rolling basis as courts construe existing
%statutes).\readingfootnote{14}{For example, OCGA \S~11-2A-213 was enacted, in
%its current form, in 1993. See 1993 Ga. Laws p. 633. The case notes contained in
%the OCGA summarize judicial decisions construing the statute years later. See
%\S~11-2A-213 (2002) (citing \textit{Griffith v. Medical Rental Supply of Albany,
%Ga., Inc.}, 244 Ga.App. 120, 534 S.E.2d 859 (2000); \textit{Bailey v. Tucker
%Equip. Sales, Inc.}, 236 Ga.App. 289, 510 S.E.2d 904 (1999)).} In short,
%annotating begins only after lawmaking ends. This sets the OCGA annotations
%apart from uncopyrightable legislative materials like committee reports,
%generated before a law's enactment, and tied tightly to the task of
%law-formulation.
%
%Second, the OCGA annotations are descriptive rather than prescriptive. Instead
%of stating the legislature's perception of what a law conveys, the annotations
%summarize writings in which others express \textit{their} views on a given
%statute. For example,  the OCGA contains ``case annotations'' for ``[a]ll
%decisions of the Supreme Court of Georgia and the Court of Appeals of Georgia
%and all decisions of the federal courts in cases which arose in Georgia
%construing any portion of the general statutory law of the state.''
%\textit{Id.}, at 403. Per the Code Revision Commission's instructions, each
%annotation should ``accurately reflect the facts, holding, and statutory
%construction'' adopted by the court. \textit{Id.}, at 404. The annotations are
%neutrally cast; they do not opine on whether the summarized case was correctly
%decided. See, \textit{e.g.}, OCGA \S~17-7-50 (2013) (case annotation summarizing
%facts and holdings of nine cases construing right to grand jury hearing). This
%characteristic of the annotations distinguishes them from preenactment
%legislative materials that touch or concern the correct interpretation of the
%legislature's work.
%
%Third, and of prime importance, the OCGA annotations are ``given for the purpose
%of convenient reference'' by the public, \S~1-1-7 (2019); they aim to inform the
%citizenry at large, they do not address, particularly, those seated in
%legislative chambers.\readingfootnote{15}{Suppose a committee of Georgia's
%legislature, to inform the public, instructs a staffer to write a guide titled
%``The Workways of the Georgia Legislature.'' The final text describing how the
%legislature operates is circulated to members of the legislature and approved by
%a majority. Contrary to the Court's decision, I take it that such a work, which
%entails no lawmaking, would be copyrightable.} Annotations are thus unlike, for
%example, surveys, work commissioned by a legislature to aid in determining
%whether existing law should be amended.
%
%The requirement that the statutory portions of the OCGA ``shall be merged with
%annotations,'' \S~1-1-1, does not render the annotations anything other than
%explanatory, referential, or commentarial material. See \textit{Harrison Co. v.
%Code Revision Comm'n}, 244 Ga. 325, 331, 260 S.E.2d 30, 35 (1979) (observation
%by the Supreme Court of Georgia that ``inclusion of annotations in [the]
%`official Code'\,'' does not ``give the annotations any official
%weight'').\readingfootnote{16}{That the Georgia Supreme Court described the
%Commission's work as ``within the sphere of legislative authority'' for state
%separation-of-powers purposes, \textit{Harrison Co. v. Code Revision Comm'n},
%244 Ga. 325, 330, 260 S.E.2d 30, 34 (1979), does not resolve the federal
%Copyright Act question before us. Cf. \textit{Yates v. United States}, 574 U.S.
%528, 537, 135 S.Ct. 1074, 191 L.Ed.2d 64 (2015) (plurality opinion) (``In law as
%in life, ... the same words, placed in different contexts, sometimes mean
%different things.''); Cook, ``Substance'' and ``Procedure'' in the Conflict of
%Laws, 42 Yale L. J. 333, 337 (1933) (``The tendency to assume that a word which
%appears in two or more legal rules, and so in connection with more than one
%purpose, has and should have precisely the same scope in all of them, runs all
%through legal discussions. It has all the tenacity of original sin and must
%constantly be guarded against.'').} Annotations aid the legal researcher, and
%that aid is enhanced when annotations are printed beneath or alongside the
%relevant statutory text. But the placement of annotations in the OCGA does not
%alter their auxiliary, nonlegislative character.
%
%
%
%\readinghead{* * *}
%
%Because summarizing judicial decisions and commentary bearing on enacted
%statutes, in contrast to, for example, drafting a committee report to accompany
%proposed legislation, is not done in a legislator's law-shaping capacity, I
%would hold the OCGA annotations copyrightable and therefore reverse the judgment
%of the Court of Appeals for the Eleventh Circuit.
%
