\defcase{alice-v-cls-bank}{
Alice Corp. Pty. Ltd. v. CLS Bank International, 573 U.S. 208 (2014);
inline=Alice,
}

\defcase{mayo-v-prometheus}{
p=Mayo Collaborative Services,
d={Prometheus Laboratories, Inc.},
cite=566 U.S. 66,
inline=Mayo,
year=2012
}


\item The doctrine of ``patent eligibility'' (also called ``subject matter
eligibility''), at issue in \emph{Myriad}, has been applied to exclude patenting
of computer-implemented financial processes, \clause{see alice-v-cls-bank}, and
diagnostic methods of adjusting drug dosages, \clause{see mayo-v-prometheus}.
The effect of these decisions is to place these ideas and others into an
``information commons'' that is not subject to patent protection, and thus
available to anyone to use. Such exclusions of subject matter from patentability
have been controversial, with Congress regularly considering legislation to
overturn these decisions.

\item The Court's rationale for excluding certain ideas from patent eligibility
is fundamentally an economic one: the ``basic tools of scientific and
technological work'' must remain open to everyone so that ``future innovation
premised upon'' those basic tools can occur. This is sometimes called a concern
for \term{downstream innovation}---that exclusive property rights can prevent
non-owners from being socially productive. Do you agree? Consider the following
possible responses:
\begin{itemize}
\item Without the economic incentive of patent protection, no one would
put in the effort to make discoveries of natural phenomena like the BRCA genes.
\item Patent protection enables firms to invest time and money to turn
scientific discoveries into commercial products---running clinical trials,
getting regulatory approval, marketing, and so on. (This is called
\term{commercialization} theory.)
\item Even if downstream innovation is a concern, it would be better to have a
single firm coordinating all the downstream researchers to make sure that
everyone is working as productively as possible without duplication; a single
patent-holding firm can do this. (This is a response to the tragedy of the
commons.)
\end{itemize}

\defbook{contreras-genome-defense}{
Jorge Contreras, The Genome Defense (2021)
}

\item On the flip side, the downstream innovation rationale is not the only one
that justifies the patent ineligibility of human genes. Perhaps the most
obvious explanation is that natural phenomena simply exist in nature; they
preexist any human activity and thus belong to no one. This is a traditional
explanation, which applies to a wide variety of natural products and phenomena.
But human genes add another dimension---they are a part of people, and the idea
that some company can get a patent on something \emph{inside your body} is
intuitively strange at best. Indeed, it was this latter concern that motivated
the American Civil Liberties Union to litigate the \emph{Myriad} case in the
first place. \sentence{see contreras-genome-defense at 12-13}.

Notice that these arguments are property-based arguments---human genes ought not
be patentable, either because they belong to the common heritage of all
humanity, or because genes belong to individuals as part of their bodies. Even
though a commons operates like nobody owns it, it is often justified on the
theory that everyone owns it. How well do these arguments translate to physical
resource commons, like water and public lands?


