\expected{matthews-v-bay-head}

\item As discussed with respect to the public trust doctrine, the law of trusts
and easements can be invoked to protect public rights to use privately owned
land. \emph{Iron Bar}, on the other hand, rejects an easement-based
characterization and instead draws from the doctrine of nuisance. Does one
approach strike you as better than the others? Indeed, do any of these doctrines
fit these situations precisely? What is the value of using existing property
concepts to describe the public right, rather than just coming up with a new
concept?

\item The court concludes that corner-crossing is a trespass under Wyoming law,
despite the public right to access public lands. Is that necessarily correct?
\having{state-v-shack}{Recall \mref{state-v-shack}, which held}{We will later
read \mref{state-v-shack}, which holds}{In \emph{State v.~Shack}, 58 N.J. 297
(1971), the Supreme Court of New Jersey held} that entering a farmer's land was
not a trespass when done to provide essential services to migrant workers living
there. Could you plausibly define ``trespass'' so as not to exclude
corner-crossing as well?

\item Is there a limit to the public's right of corner-crossing? What if
thousands of people a day start climbing ladders over Iron Bar's ranch? What if
the volume of crossers interferes with Iron Bar's operations---perhaps
constituting a nuisance? Does Iron Bar have any remedies, and against whom?

