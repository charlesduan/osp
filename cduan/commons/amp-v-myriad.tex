\reading{Ass'n for Molecular Pathology v. Myriad Genetics, Inc.}
\readingcite{569 U.S. 576 (2013)}

\opinion Justice \textsc{Thomas} delivered the opinion of the Court.

Respondent Myriad Genetics, Inc. (Myriad), discovered the precise location and
sequence of two human genes, mutations of which can substantially increase the
risks of breast and ovarian cancer. Myriad obtained a number of patents based
upon its discovery. This case involves claims from three of them and requires us
to resolve whether a naturally occurring segment of deoxyribonucleic acid (DNA)
is patent eligible under 35 U.S.C. \S~101 by virtue of its isolation from the
rest of the human genome.\ldots


\readinghead{I}



%\readinghead{A}
%
%Genes form the basis for hereditary traits in living organisms. The human genome
%consists of approximately 22,000 genes packed into 23 pairs of chromosomes. Each
%gene is encoded as DNA, which takes the shape of the familiar ``double helix''
%that Doctors James Watson and Francis Crick first described in 1953. Each
%``cross-bar'' in the DNA helix consists of two chemically joined nucleotides.
%The possible nucleotides are adenine (A), thymine (T), cytosine (C), and guanine
%(G), each of which binds naturally with another nucleotide: A pairs with T; C
%pairs with G. The nucleotide cross-bars are chemically connected to a
%sugar-phosphate backbone that forms the outside framework of the DNA helix.
%Sequences of DNA nucleotides contain the information necessary to create strings
%of amino acids, which in turn are used in the body to build proteins. Only some
%DNA nucleotides, however, code for amino acids; these nucleotides are known as
%``exons.'' Nucleotides that do not code for amino acids, in contrast, are known
%as ``introns.''
%
%Creation of proteins from DNA involves two principal steps, known as
%transcription and translation. In transcription, the bonds between DNA
%nucleotides separate, and the DNA helix unwinds into two single strands. A
%single strand is used as a template to create a complementary ribonucleic acid
%(RNA) strand. The nucleotides on the DNA strand pair naturally with their
%counterparts, with the exception that RNA uses the nucleotide base uracil (U)
%instead of thymine (T). Transcription results in a single strand RNA molecule,
%known as pre-RNA, whose nucleotides form an inverse image of the DNA strand from
%which it was created. Pre-RNA still contains nucleotides corresponding to both
%the exons and introns in the DNA molecule. The pre-RNA is then naturally
%``spliced'' by the physical removal of the introns. The resulting product is a
%strand of RNA that contains nucleotides corresponding only to the exons from the
%original DNA strand. The exons-only strand is known as messenger RNA (mRNA),
%which creates amino acids through translation. In translation, cellular
%structures known as ribosomes read each set of three nucleotides, known as
%codons, in the mRNA. Each codon either tells the ribosomes which of the 20
%possible amino acids to synthesize or provides a stop signal that ends amino
%acid production.
%
% DNA's informational sequences and the processes that create mRNA, amino acids,
%and proteins occur naturally within cells. Scientists can, however, extract DNA
%from cells using well known laboratory methods. These methods allow scientists
%to isolate specific segments of DNA---for instance, a particular gene or part of
%a gene---which can then be further studied, manipulated, or used. It is also
%possible to create DNA synthetically through processes similarly well known in
%the field of genetics.\ldots
%One such method begins with an mRNA molecule and uses the
%natural bonding properties of nucleotides to create a new, synthetic DNA
%molecule. The result is the inverse of the mRNA's inverse image of the original
%DNA, with one important distinction: Because the natural creation of mRNA
%involves splicing that removes introns, the synthetic DNA created from mRNA also
%contains only the exon sequences. This synthetic DNA created in the laboratory
%from mRNA is known as complementary DNA (cDNA).

%Changes in the genetic sequence are called mutations. Mutations can be as small
%as the alteration of a single nucleotide — a change affecting only one letter in
%the genetic code. Such small-scale changes can produce an entirely different
%amino acid or can end protein production altogether. Large changes, involving
%the deletion, rearrangement, or duplication of hundreds or even millions of
%nucleotides, can result in the elimination, misplacement, or duplication of
%entire genes. Some mutations are harmless, but others can cause disease or
%increase the risk of disease. As a result, the study of genetics can lead to
%valuable medical breakthroughs.



%\readinghead{B}

%This case involves patents filed by Myriad after it made one such medical
%breakthrough.
Myriad discovered the precise location and sequence of what are
now known as the BRCA1 and BRCA2 genes.
\unskip\edfootnote{On the assumption that most people know what genes are these
days, the Court's extensive discussion of the science of genetics has been
omitted. If you would like a summary or refresher, here is one. Inside all human
cells (as well as the cells of any living thing) are molecules called DNA\@. A
DNA molecule is made up of small chemicals called nucleotides, which are
sequentially strung together. There are four such nucleotides, abbreviated A, T,
G, and C\@. A ``gene'' is the informational sequence of these nucleotides, which
determines how the cell will construct protein molecules.}
Mutations in these genes can
dramatically increase an individual's risk of developing breast and ovarian
cancer. The average American woman has a 12- to 13-percent risk of developing
breast cancer, but for women with certain genetic mutations, the risk can range
between 50 and 80 percent for breast cancer and between 20 and 50 percent for
ovarian cancer. Before Myriad's discovery of the BRCA1 and BRCA2 genes,
scientists knew that heredity played a role in establishing a woman's risk of
developing breast and ovarian cancer, but they did not know which genes were
associated with those cancers.

Myriad identified the exact location of the BRCA1 and BRCA2 genes on chromosomes
17 and 13.\ldots
Knowledge of the location of the BRCA1 and BRCA2 genes allowed
Myriad to determine their typical nucleotide
sequence.\readingfootnote{1}{Technically, there is no ``typical'' gene because
nucleotide sequences vary between individuals, sometimes dramatically.
Geneticists refer to the most common variations of genes as ``wild types.''}
That information, in turn, enabled Myriad to develop medical tests that are
useful for detecting mutations in a patient's BRCA1 and BRCA2 genes and thereby
assessing  whether the patient has an increased risk of cancer.

Once it found the location and sequence of the BRCA1 and BRCA2 genes, Myriad
sought and obtained a number of patents. Nine composition claims from three of
those patents are at issue in this case.
Claims\edfootnote{Recall that ``claims'' in patents are the legally operative
language that defines the scope of what infringes, akin to a property boundary.}
1, 2, 5, and 6 from [Myriad's] patent are
representative. The first claim asserts a patent on ``[a]n isolated DNA coding
for a BRCA1 polypeptide,'' which has ``the amino acid sequence set forth in SEQ
ID NO:2.'' SEQ ID NO:2 sets forth a list of 1,863 amino acids that the
typical BRCA1 gene encodes. Put differently, claim
1 asserts a patent claim on the DNA code that tells a cell to produce the string
of BRCA1 amino acids listed in SEQ ID NO:2.

[The other patent claims describe different parts of the BRCA1 and BRCA2 gene
sequences.]



%\readinghead{C}

Myriad's patents would, if valid, give it the exclusive right to isolate an
individual's BRCA1 and BRCA2 genes (or any strand of 15 or more nucleotides
within the genes)\ldots.
But isolation is necessary to conduct genetic testing, and Myriad was not the
only entity to offer BRCA testing after it discovered the genes.
[So did one of the defendants, Dr. Harry Ostrer, thereby prompting this case.]


\readinghead{II}



\readinghead{A}

Section 101 of the Patent Act provides:
\begin{quote}
Whoever invents or discovers any new and useful\ldots composition
of matter, or any new and useful improvement thereof, may obtain a patent
therefor, subject to the conditions and requirements of this title.
\end{quote}
We have ``long held that this provision contains an
important implicit exception[:] Laws of nature, natural phenomena, and abstract
ideas are not patentable.''
Rather, ``\,`they are the
basic tools of scientific and technological work'\,'' that lie beyond the domain
of patent protection. As the Court
has explained, without this exception, there would be considerable danger that
the grant of patents would ``tie up'' the use of such tools and thereby
``inhibit future innovation premised upon them.''
This would be at odds with the very point of patents, which
exist to promote creation. \textit{Diamond v. Chakrabarty}, 447 U.S. 303, 309
(1980) (Products of nature are not created, and ``\,`manifestations\ldots of
nature [are] free to all men and reserved exclusively to none'\,'').

The rule against patents on naturally occurring things is not without limits,
however, for ``all inventions at some level embody, use, reflect, rest upon, or
apply laws of nature, natural phenomena, or abstract ideas,'' and ``too broad an
interpretation of this exclusionary principle could eviscerate patent law.''
As we have recognized before, patent
protection strikes a delicate balance between creating ``incentives that lead to
creation, invention, and discovery'' and ``imped[ing] the flow of information
that might permit, indeed spur, invention.''
We must apply this well-established standard to determine whether
Myriad's patents claim any ``new and useful\ldots composition of matter,''
or instead claim naturally occurring phenomena.



\readinghead{B}

It is undisputed that Myriad did not create or alter any of the genetic
information encoded in the BRCA1 and BRCA2 genes. The location and order of the
nucleotides existed in nature before Myriad found them. Nor did Myriad create or
alter the genetic structure of DNA. Instead, Myriad's principal contribution was
uncovering the precise location and genetic sequence of the BRCA1 and BRCA2
genes within chromosomes 17 and 13. The question is whether this renders the
genes patentable.

Myriad recognizes that our decision in \textit{Chakrabarty} is central to this
inquiry. In \textit{Chakrabarty}, scientists
added four plasmids to a bacterium, which enabled it to break down various
components of crude oil. The Court
held that the modified bacterium was patentable. It explained  that the patent
claim was ``not to a hitherto unknown natural phenomenon, but to a nonnaturally
occurring manufacture or composition of matter---a product of human ingenuity
`having a distinctive name, character [and] use.'\,''
The
\textit{Chakrabarty} bacterium was new ``with markedly different characteristics
from any found in nature,'' due to the
additional plasmids and resultant ``capacity for degrading oil.''
In this case, by contrast, Myriad did not create
anything. To be sure, it found an important and useful gene, but separating that
gene from its surrounding genetic material is not an act of invention.

Groundbreaking, innovative, or even brilliant discovery does not by itself
satisfy the \S~101 inquiry. In \textit{Funk Brothers Seed Co. v. Kalo Inoculant
Co.}, this Court considered a
composition patent that claimed a mixture of naturally occurring strains of
bacteria that helped leguminous plants take nitrogen from the air and fix it in
the soil. The ability of the bacteria to
fix nitrogen was well known, and farmers commonly ``inoculated'' their crops
with them to improve soil nitrogen levels. But farmers could not use the same
inoculant for all crops, both because plants use different bacteria and because
certain bacteria inhibit each other.
Upon learning that several nitrogen-fixing bacteria did not inhibit each other,
however, the patent applicant combined them into a single inoculant and obtained
a patent. The Court held that the
composition was not patent eligible because the patent holder did not alter the
bacteria in any way.
His
patent claim thus fell squarely within the law of nature exception. So do
Myriad's. Myriad found the location of the BRCA1 and BRCA2 genes, but that
discovery, by itself, does not render the BRCA genes ``new\ldots composition[s]
of
matter,'' \S~101, that are patent eligible.

Indeed, Myriad's patent descriptions highlight the problem with its claims. For
example, a section of [Myriad's] patent's Detailed Description of the Invention
indicates that Myriad found the location of a gene associated with increased
risk of breast cancer and identified mutations of that gene that increase the
risk.
In subsequent language Myriad explains that the location of the
gene was unknown until Myriad found it among the approximately eight million
nucleotide pairs contained in a subpart of chromosome  17.\ldots
Many of Myriad's patent
descriptions simply detail the ``iterative process'' of discovery by which
Myriad narrowed the possible locations for the gene sequences that it
sought.
Myriad seeks to import these extensive research
efforts into the \S~101 patent-eligibility inquiry.
But extensive effort alone is insufficient to satisfy the demands of \S~101.

Nor are Myriad's claims saved by the fact that isolating DNA from the human
genome severs chemical bonds and thereby creates a nonnaturally occurring
molecule. Myriad's claims are simply not expressed in terms of chemical
composition, nor do they rely in any way on the chemical changes that result
from the isolation of a particular section of DNA. Instead, the claims
understandably focus on the genetic information encoded in the BRCA1 and BRCA2
genes. If the patents depended upon the creation of a unique molecule, then a
would-be infringer could arguably avoid at least Myriad's patent claims on
entire genes (such as claims 1 and 2 of [Myriad's] patent) by isolating a DNA
sequence that included both the BRCA1 or BRCA2 gene and one additional
nucleotide pair. Such a molecule would not be chemically identical to the
molecule ``invented'' by Myriad. But Myriad obviously would resist that outcome
because its claim is concerned primarily with the information contained in the
genetic \textit{sequence}, not with the specific chemical composition of a
particular molecule.

Finally, Myriad argues that the PTO's past practice of awarding gene patents is
entitled to deference\ldots.
We disagree.\ldots
Congress has not endorsed the views of the PTO in subsequent legislation.\ldots
Further undercutting the PTO's practice, the United States argued in the Federal
Circuit and in this Court that isolated DNA was \textit{not} patent eligible
under \S~101, and that
the PTO's practice was not ``a sufficient reason to hold that isolated DNA is
patent-eligible.'' These
concessions weigh against deferring to the PTO's
determination.\readingfootnote{7}{Myriad also argues that we should uphold its
patents so as not to disturb the reliance interests of patent holders like
itself. Concerns about reliance interests arising
from PTO determinations, insofar as they are relevant, are better directed to
Congress.}

\ldots.

\readinghead{III}

It is important to note what is \textit{not} implicated by this decision. First,
there are no method claims before this Court. Had Myriad created an innovative
method of manipulating genes while searching for the BRCA1 and BRCA2 genes, it
could possibly have sought a method patent. But the processes used by Myriad to
isolate DNA were well understood by geneticists at the time of Myriad's patents
``were well understood, widely used, and fairly uniform insofar  as any
scientist engaged in the search for a gene would likely have utilized a similar
approach,'' and are not at issue in this case.

Similarly, this case does not involve patents on new \textit{applications} of
knowledge about the BRCA1 and BRCA2 genes. Judge Bryson aptly noted that, ``[a]s
the first party with knowledge of the [BRCA1 and BRCA2] sequences, Myriad was in
an excellent position to claim applications of that knowledge. Many of its
unchallenged claims are limited to such applications.''

Nor do we consider the patentability of DNA in which the order of the naturally
occurring nucleotides has been altered. Scientific alteration of the genetic
code presents a different inquiry, and we express no opinion about the
application of \S~101 to such endeavors. We merely hold that genes and the
information they encode are not patent eligible under \S~101 simply because they
have been isolated from the surrounding genetic material.


[Justice Scalia's concurrence is omitted.]
