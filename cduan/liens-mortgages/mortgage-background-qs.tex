\expected{intro-mortgage-background}
\expected{intro-estates}

\item As an initial matter, pay attention to the property interests involve.
First, there is the promissory note itself. Initially, the loan originator (the
bank) has the right to receive monthly payments based on the promissory note.
But the note is typically alienable---the originator can sell it to another
bank, or a loan servicer, or a financial institution. In that sense, the note
itself is a kind of property right.

\item Second, there are the property interests relating to the house or land
that secures the mortgage. These interests are comparable to those in the system
of estates. The mortgagor has some sort of possessory estate, insofar as the
mortgagor gets to live on the mortgaged land. The mortgagee has a kind of future
interest.

With a life estate, the event of the life tenant's death causes the property
interest of the life tenant to disappear, and the future interest of the
remainder or reversion holder to convert into a possessory estate. All this
occurs by operation of law. For a mortgage, what event causes a change to the
property interests of the mortgagor and mortgagee by operation of law? What
happens then, and what are the resulting property interests?

\item 


