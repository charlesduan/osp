\defjrnart{levitin-securitization-foreclosure}{
Adam J. Levitin, The Paper Chase: Securitization, Foreclosure, and the
Uncertainty of Mortgage Title, 63 Duke Law Journal 637 (2013)
}

\Inline{levitin-securitization-foreclosure} provides this summary of
``securitization'' of mortgages:
\begin{quotation}
Securitization is a relatively recent development in residential mortgage
lending.\ldots Today nearly two-thirds of mortgage dollars
outstanding are securitized.

\ldots.

Although residential-mortgage securitization transactions are complex and vary
somewhat depending on the type of entity undertaking the securitization, there
is still a  core standard  transaction. First, a financial  institution  (the
``sponsor''  or ``seller'') assembles a pool of mortgage loans either made
(``originated'')  by  an affiliate of the financial institution or
purchased from unaffiliated third-party originators. Second, the pool of
loans is sold by the sponsor to a special-purpose subsidiary (the
``depositor'') that has no other assets or liabilities and is little more than
a legal entity with a mailbox. This is done to segregate the loans from the
sponsor's assets and liabilities. Third, the depositor sells the loans to a
passive, specially created, single-purpose vehicle (SPV), typically a trust in
the case of residential-mortgage securitization. The trustee will then
typically convey the mortgage notes and security instruments to a document
custodian for safekeeping. The SPV issues certificated debt securities to raise
the funds to pay for the loans. As these debt securities are backed by the cash
flow from the mortgages, they are called mortgage-backed securities (MBS).
\end{quotation}
\sentence{levitin-securitization-foreclosure}.

It's not essential at this point that you understand the precise arrangement of
mortgage securitization. The key observation is that the mortgage, originally a
contractual transaction between a homebuyer and a bank, has been sold through a
number of hands into a trust or other entity. This is because the mortgage
transaction creates property interests---a future interest in the real estate
being mortgaged, and a right to collect payments on the promissory note. Because
they are property interests, they can be bought and sold. In much the same way
that one can acquire individual pieces of furniture and bundle them together
into a living room set, one can acquire multiple mortgages and bundle them
together into a mortgage-backed security.

