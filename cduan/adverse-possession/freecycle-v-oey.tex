\reading{Freecycle Network v. Oey}
\readingcite{505 F.3d 898 (9th Cir. 2007)}

\opinion \textsc{Hawkins}, Circuit Judge:

Tim Oey appeals a preliminary injunction preventing him from making any
comments that could be construed as to disparage upon The Freecycle Network's
possible trademark and logo and requiring that he remove all postings from
the Internet and any other public forums that he has previously made that
disparage The Freecycle Network's possible trademark and
logo. We have jurisdiction under 28 U.S.C. \S~1291 and, for the following
reasons, vacate the injunction and remand.



\readinghead{I.}

The Freecycle Network (``TFN'') is a nonprofit Arizona corporation ``dedicated
to encouraging and coordinating the reusing, recycling, and gifting of goods.''
Through its website, http://www.freecycle.org, TFN coordinates the efforts
of over 3,700 Freecycle groups worldwide. Via the local groups' webpages,
individuals can post goods they no longer want. If another member wants the item
offered, an exchange is arranged between the parties and the item thus avoids
the landfill.

Although TFN claims to have consistently used the marks FREECYCLE and THE
FREECYCLE NETWORK, and ``The Freecycle Network'' logo since May 2003 to refer to
TFN, it also admits that it initially used the term ``freecycle'' and its
various derivations (e.g., freecycling, freecycler) to refer more generally to
the act of recycling goods for free via the Internet. In 2004, based on the
advice of then-member  Oey, TFN decided to more actively police its use of the
term ``freecycle'' and to formally pursue trademark protection for it, filing a
trademark registration application on August 27, 2004. Shortly thereafter, TFN
instituted a strict usage policy, drafted by Oey, preventing use of the term
``freecycle'' in any sense other than to refer to TFN or TFN's services. On
January 17, 2006, TFN's proposed mark was published for opposition in the
Official Gazette. An opposition was filed the next day and the mark currently
remains unregistered.

A member of TFN since February 2004 and active in the corporation's early
development, Oey initially supported TFN's claim to the FREECYCLE mark.
Experiencing a change of heart and convinced that the term should remain in the
public domain, Oey later urged TFN to abandon its efforts to secure the mark,
conveying his feelings in an August 8, 2005, email to fellow TFN group
moderators.\readingfootnote{3}{In this email, Oey urged abandonment of TFN's
trademark pursuit, contending that forcing the term ``freecycle'' into the
public domain ``fits well with a `viral' marketing approach to freecycle\ldots
which will lead back to [TFN]\ldots [and] generate lots of goodwill.'' He also
recommended that TFN ``maintain the trademark on the full name `The Freecycle
Network'\ldots [and] take credit for birthing[the] freecycle [concept].''} In
the
following weeks, Oey made various statements on the Internet that TFN lacked
trademark rights in ``freecycle'' because it was a generic term, and he
encouraged others to use the term in its generic sense and to write letters to
the United States Patent and Trademark Office (``PTO'') opposing TFN's pending
registration.

Not surprisingly, TFN took issue with Oey's views and, on September 16, 2005,
asked him to sever ties with the company.\ldots

[The Ninth Circuit held that Oey's actions were not likely to constitute
trademark infringement.]

\readinghead{C) Genericide}

Although we do not reach the question of the validity of TFN's claimed mark, the
crux of TFN's complaint is that Oey should be prevented from using (or
encouraging the use of) TFN's claimed mark FREECYCLE in its generic sense.
However, TFN's asserted mark---like all marks---is always at risk of becoming
generic and thereby losing its ability to identify the trademark holder's goods
or services. \textit{See, e.g.}, \textit{Mattel, Inc.,} 296 F.3d at 900 (``Some
trademarks enter our public discourse and become an integral part of our
vocabulary.''); 2 J. Thomas McCarthy, \emph{McCarthy on Trademarks and Unfair
Competition} \S~12:1 (2007) (hereinafter ``McCarthy''). Where the majority of
the relevant public appropriates a trademark term as the name of a product (or
service), the mark is a victim of ``genericide'' and trademark rights generally
cease. McCarthy \S~12:1.

Such genericide can occur ``as a result of a trademark owner's failure to police
the mark, resulting in widespread usage by competitors leading to a perception
of genericness among the public, who sees many sellers using the same term.''
\textit{Id.} (footnotes omitted). Alternatively, ``a term intended by the seller
to be a trademark for a new product[can be] taken by the public as a generic
name because customers have no other word to use to name this new thing.''
\textit{Id.} Genericide has spelled the end for countless formerly trademarked
terms, including ``aspirin,'' ``escalator,'' ``brassiere,'' and ``cellophane.''
\textit{See id.} \S~12:18 (list of terms held to be generic).

``Although there is a social cost when a mark becomes generic---the trademark
owner has to invest in a new trademark to identify his brand---there is also a
social benefit, namely the addition to ordinary language.'' \textit{Ty Inc. v.
Perryman}, 306 F.3d 509, 514 (7th Cir.2002). Furthermore, when a trademark
becomes generic, ``it reduces the cost of communication by making it cheaper for
competitors to inform consumers that they are selling the same kind of product''
or providing the same  kind of service. McCarthy \S~12:2; \textit{see also}
\textit{Mattel, Inc.}, 296 F.3d at 900 (``Trademarks often fill in gaps in our
vocabulary and add a contemporary flavor to our expressions. Once imbued with
such expressive value, the trademark becomes a word in our language and assumes
a role outside the bounds of trademark law.'').

Of course, trademark owners are free (and perhaps wise) to take action to
prevent their marks from becoming generic and entering the public domain---e.g.,
through a public relations campaign or active policing of the mark's use. The
Lanham Act itself, however, contains no provision preventing the use of a
trademarked term in its generic sense. \textit{Cf.} \textit{Ty Inc.}, 306 F.3d
at 513-14 (rejecting
an attempt to extend the Lanham Act's antidilution provisions ``to enjoin uses
of their mark that, while not confusing, threaten to render the mark generic'').

Nor does the Act prevent an individual from expressing an opinion that a mark
should be considered generic or from encouraging others to use the mark in its
generic sense. Rather, the use of a mark in its generic sense is actionable
under the Lanham Act only when such use also satisfies the elements of a
specified cause of action---e.g., infringement, false designation of origin,
false advertising, or dilution. TFN's mere disagreement with Oey's opinion and
frustration with his activities cannot render Oey liable under the Lanham Act.



