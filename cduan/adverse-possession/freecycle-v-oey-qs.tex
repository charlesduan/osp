\item How many genericized trademarks can you think of? Why don't you google it?
Maybe you can make a powerpoint of the ones you find, even with some cleverly
photoshopped graphics. If you do, zoom me so I can see it, or you can write a
few examples onto post-its and rollerblade over to my office with them.

\defjrnart{heymann-grammar-trademarks}{
Laura A. Heymann, The Grammar of Trademarks, 14 Lewis and Clark Law Review 1313
(2010)
}

\defjrnart{linford-trademark-owner-adverse}{
Jake Linford, Trademark Owner as Adverse Possessor: Productive Use and Property
Acquisition, 63 Case Western Law Review 703 (2013)
}

\item The author of these notes is unaware of any other property textbook that
identifies trademark genericide as related to adverse possession, although the
idea is not unknown. \sentence{heymann-grammar-trademarks at 1318; cf.
linford-trademark-owner-adverse}. How do the two doctrines compare? Look at the
list of elements for adverse possession, and see if you can find an analogue (or
lack thereof) for trademarks.


\defcase{ty-v-perryman}{
parties=Ty Inc. v. Perryman,
cite=306 F.3d 509,
court=7th Cir.,
year=2002,
}


\defwebsite{johnson-please-help}{
author=Eric E. Johnson,
title={Please Help, if You Can},
date=june 29 2010,
journal=PrawfsBlawg,,
url=https://prawfsblawg.blogs.com/prawfsblawg/2010/06/please-help-if-you-can.html,
}
\item Quoting Judge Posner's opinion in \adjective{ty-v-perryman}\optclause,
Judge Hawkins observes that genericide has both a ``social cost'' and ``a social
benefit, namely the addition to ordinary language.'' Can you find a ``social
benefit'' in adverse possession of real property? Which doctrine do you find
more socially justified, and why?

\captionedgraphic{xerox-tm}{An advertisement by Xerox, run in the ABA Journal.
Via \protect\fullcite{johnson-please-help}\protect\clause{johnson-please-help}.}

\item The risk that a trademark will become generic leads some trademark holders
to campaign vigorously to protect their trademarks. Some, like Xerox, run
advertisements like the one shown in Figure~\ref{f:xerox-tm}. Others litigate
even the most minor uses of their trademarks, in order to show that they are
actively defending their rights---or, perhaps, using the risk of genericide as a
pretext for vigorous enforcement. Is this a desirable outcome? Would it be
better if it were harder for trademarks to become genericized?
