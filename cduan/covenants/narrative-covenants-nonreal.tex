Servitudes, easements, and restrictive covenants are creatures of real property
law. There are no doctrines of covenants for personal property or intellectual
property, for example. But the broader concept behind covenants---using
property-like agreements to restrict behavior---is potentially
useful and applicable beyond the domain of real estate. Indeed, clever lawyers
have attempted to devise ways of simulating the operation of servitudes in a
wide range of domains.

Here are a few examples of servitude-like arrangements. What do you think of
them? Do they strike you as achieving useful goals? Would the world be better
served if servitude law were recognized in other domains, rather than the
sometimes-hackish solutions below? Or do these situations strike you as
overreaching? Is there, perhaps, a good reason for servitudes to be limited to
land?


\paragraph{Personal property} The general view is that ``personal property
servitudes are seldom enforceable.'' \sentence{van-houweling at 906}.
Nevertheless, product manufacturers have long sought to control the ways in
which consumers use their products---that is, to impose servitude-like control
over personal property---so that those manufacturers can control markets for
repairs, resale, and parts. Keurig would prefer that coffee drinkers only buy
authorized (and more expensive) K-Cups; John Deere would prefer that farmers
repair their tractors only at John Deere service centers; Apple would prefer
that you buy a new phone every year or two.

\expected{impression-v-lexmark}

\defcase{vernor-v-autodesk}{
parties={Vernor v. Autodesk, Inc.},
cite=621 F.3d 1102,
court=9th Cir.,
year=2010,
}

To achieve this, manufacturers often turn to intellectual property rights as a
vehicle for controlling personal property. \sentence{see van-houweling at
924-927; perzanowski-schultz}. \mref{impression-v-lexmark} was an
attempt at imposing a covenant-like restriction on reselling toner cartridges
using patent law. Another strategy is to embed copyrighted software in a
product, and to include an End User License Agreement with the product so that
the consumer does not ``own'' the copy of the software, but rather only receives
a license to use it. \sentence{see vernor-v-autodesk at 1111}. Even if a
consumer sells the product downstream, every subsequent purchaser must have a
copyright license to use the software and therefore must comply with the EULA,
meaning that the license terms effectively ``run with the product.''

\defstatcode{dmca}{
cite=17 U.S.C. S 1201-1204,
}


Furthermore, the manufacturer can install software on the product that
programmatically enforces use restrictions. For example, car manufacturers often
lock vehicles down with software, such that the car owner cannot service the car
without authorization from the manufacturer or a licensed service dealer. Again,
this works even if the car is resold, so the repair restriction runs with the
car. If a car owner attempts to tamper with the software to overcome the
restriction, then the owner potentially violates the Digital Millennium
Copyright Act,
which prohibits circumvention of technological
protection measures that protect copyrighted works. \sentence{dmca at S 1201}.



\paragraph{Open-source works} Many software developers are willing to make
their computer programs freely available for others to use or modify. This
``open-source'' software is big business: Fundamental Internet and
communications technologies rely on it, and just about all of the largest
technology companies make open-source contributions. But these developers are
rightfully concerned about shady third parties taking their freely available
programs and selling them. In effect, open-source developers want a
covenant-like restriction on their software, preventing others from commercially
selling it.

To achieve this result in the absence of ``software covenants,'' open-source
communities have developed a number of standard license texts, such as the GNU
Public License, or GPL\@. These licenses exploit the same copyright-based
strategy described above. The license grants any recipient of the open-source
software a copyright license to copy or modify the software, so long as the
recipient complies with the terms of the license document. Typically this
requires the recipient to give attribution to the original author, and precludes
the recipient from selling the software commercially. Some open-source licenses,
like the GPL, go further, requiring as well that if the recipient modifies the
software, then the recipient must also make those modifications available under
the GPL\@. This is why open-source licenses are sometimes described as
``viral,'' and it makes them very covenant-like, binding a potentially enormous
range of downstream recipients of the licensed software.

The open-source license model has been adapted outside the software world, in
the form of the Creative Commons license. The original \emph{Open Source
Property} modules are governed by a Creative Commons license, which is why this
book is free.


\paragraph{Access to technologies} Patents on vital drugs or on critical
infrastructure can restrict the public's access to important technologies, and
so it is often useful to impose usage restrictions on patents to enhance public
access. Ideally, such usage restrictions continue to have force even if the
patent is sold to someone else---that is, the restrictions should run with the
patent.

\defstatcode{bayh-dole}{
cite=35 U.S.C. S 200-212,
}

One place this arises is in federal funding of research. Say the U.S. government
gives a grant to a university to do research on a disease. If the university
discovers a treatment for the disease, it can get a patent on the treatment.
\unskip\footnote{This is perhaps controversial---if the public paid for the
research, why should the grantee profit a second time through a patent? But
the current prevailing view is that patenting of funded research results is
appropriate because it gives the grantee the incentives to do the additional
work of bringing the product to market, such as getting regulatory approval.}
But the government, as grantor, presumably wants to ensure that the grantee
exploits the invention in the public interest. To achieve this,
the Bayh--Dole Act imposes a variety of limitations on patent rights obtained
through federal funding. \sentence{bayh-dole at S 202-204}. The most famous of
these are the government's ``march-in rights,'' which can force the grantee to
license their patents on federally funded inventions when certain conditions are
met, such as licensing being ``necessary to alleviate health or safety needs.''
\sentence{bayh-dole at S 203/a3}.

In the case of \inline{bayh-dole}, the ``covenants'' on federally funded patents
were to the benefit of the government, and they were achieved through targeted
legislation. What if the benefit runs to a private party that cannot simply
change patent law? Consider the Institute of Electrical and Electronics
Engineers, which develops the technical specifications for communication
technologies like Wi-Fi. Companies like Motorola come up with improved wireless
communication systems, and IEEE may incorporate those improvements into newer
versions of Wi-Fi. But if Motorola has patents on its improvements, then anyone
who uses Wi-Fi is potentially liable. Insofar as IEEE wants Wi-Fi to be broadly
adopted without fear of patent lawsuits, IEEE contracts with Motorola, with
Motorola agreeing to license its patents on reasonable terms.

\defadmincase{motorola}{
name=In re Motorola Mobility LLC,
cite=156 F.T.C. 147,
date=july 23 2013
}

\defcitecontainer{chao-motorola}{
author=Bernard Chao,
title={In re Motorola Mobility LLC, \& Google: FRAND, Injunctions and Unfair
Competition},
in=book: {
title=FRAND Cases in Context,
editor=Jorge L. Contreras,
year=2026,
forthcoming,
},
}

But what if another company, say Google, acquires Motorola's patents? Google has
no contract with IEEE\@, and so is theoretically free to sue all the Wi-Fi users
it wants. The solution from 2013: The Federal Trade Commission contended that
Google's failure to honor Motorola's reasonable-licensing contract was
anticompetitive behavior, and got Google to agree to honor it through a consent
decree. \sentence{see motorola; chao-motorola}.


\defjrnart{kesan-hayes-frands-forever}{
author=Jay P. Kesan,
author=Carol M. Hayes,
title={FRAND's Forever: Standards, Patent Transfers, and Licensing
Commitments},
date=2014,
issue=1,
opturl=https://www.repository.law.indiana.edu/ilj/vol89/iss1/10,
cite=89 Indiana Law Journal 10,
}

Would intellectual property covenants have been a better approach? And what
might they look like? \sentence{see kesan-hayes-frands-forever at 294-304}.
