\editfromrepo{base}

\replaceend{Notwithstanding this concern, English courts}{%
    This is the problem to be solved with the property interest called the
\term{restrictive covenant}, also called the ``real covenant.'' It looks much
like a simple contract between two parties, with a bunch of additional legal
requirements for formation. But being a property interest, the covenant is more
than a contract---it has the power to run with the land and bind future owners,
even ones who may never have wished to be bound by the terms of the covenant.

This chapter and the next will trace the history of the restrictive covenant. It
is a story of how an ancient legal device came to be adapted, modified, and
repurposed, from a tool for binding medieval tenant farmers to the legal
foundation of today's suburban landscape---the modern planned community.
}
\endedit
