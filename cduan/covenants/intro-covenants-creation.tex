\editfromrepo{base}

\replacerange{As courts became more amenable}{for such covenants to run with the
land.}{The traditional real covenant at law demanded compliance with a strict
set of requirements before an agreement between two landowners would run with
the land.}

\replaceend{635 (Tex. 1987).}{635 (Tex. 1987).

\expecting{neponsit-v-emigrant}

The first and third of these elements are straightforward. The
touch-and-concern element will be discussed in the \mref{neponsit-v-emigrant}
case presented later. As for the horizontal and vertical privity elements,
collectively called ``privity of estate,'' are strict technical tests relating
to the nature of the covenanting parties' property interests. The privity rules
are not given here because, as the cases to come will show, courts have been
willing to relax them from their original common-law
strictness.\having{narrative-leases-exit-tenant}{\footnote{Also, you already
somewhat know them. Remember privity of estate from subletting and assignment of
leases, in \mref{narrative-leases-exit-tenant}? It's the same thing. Conveyance
of a lease is one way of establishing horizontal privity, and assignments make
for vertical privity. There's more, of course, but the mechanics are essentially
the same.}}{}{}

More importantly, the result of this strict test was that common-law real
covenants were hard to enforce and of limited value. This ultimately led to the
creation of a new form of restrictive covenant: the \term{equitable servitude}.
}

\endedit
