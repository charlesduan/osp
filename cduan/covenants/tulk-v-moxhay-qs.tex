\editfromrepo{base}

\append{(2004) (footnotes omitted).}{
\item Although \emph{Tulk} is not explicit about this, the general view is that
the equitable servitude eliminates the privity requirements with a notice
requirement. In other words, the required elements for an equitable servitude
are (1) compliance with the statute of frauds and contract formation, (2) the
covenanting parties' intent to bind successors, (3) the covenant touching and
concerning the land, and (4) that successors-in-interest have constructive or
actual notice. Jurisdictions differ, of course, but insofar as courts typically
treat equitable servitudes and real covenants as largely interchangeable, this
is why the traditional, strict privity requirements are not worth dwelling on.
\item
\editrepofile{base}{covenants}{intro-covenants-creation}
\replacestart{significantly relaxes this approach.}{The \textsc{Third
Restatement}, adopted in some jurisdictions, provides yet another form of
restrictive covenant that differs from both the common-law real covenant and the
equitable servitude.}
\replacerange{The first is that}{is eliminated.}{First, the privity requirement
is eliminated.}
\endedit
}
\endedit
