Arguably the most quintessential feature of property is \term{alienability}: the
ability of ownership to change hands. Most obviously, property can be sold. A
property owner can also dispose of property by donative transfer: giving it away
as a gift, or by leaving the property to friends or relatives, either through a
written will or in accordance with state intestacy laws.

Just because a property owner has a right to alienate, however, does not mean
that the property owner's wishes control. An effective transfer of property
rights must follow rules created by law.  There are only certain ways people can
rearrange property relations.  Some rearrangements happen even if the people
involved don't want them, and some don't happen even if the people involved do
want them.  Knowing the rules is a way to understand which transfers work and
why.

The next two chapters will explore the rules that govern voluntary transfers of
property. This chapter will consider several types of \term{formalities}, namely
technical and procedural requirements that must be complied with for a property
transfer to be effective. The next chapter will consider ways in which a buyer
may question, or even invalidate, a property transfer.

Consider why these rules are necessary---why shouldn't the property owner's
intentions always control? One way of answering this question is by considering
the interests at stake:
\begin{itemize}
\item Buyers, who perhaps deserve protection from shady sellers
who misrepresent the property being sold---or who don't even own the property at
all.
\item Third parties, who might benefit from public records or evidence of
transactions in property.
\item Sellers, who might be deceived into unwittingly selling or giving away
their property.
\end{itemize}

As you read, pay close attention to the type of transfer (sale, gift, will) and
the type of property involved (real, personal). The problems that courts and
lawmakers are grappling with are often universal and cross-cutting, but the
legal doctrines are specific: Rules about gifts do not necessarily apply to
wills, and rules for recordation of real estate titles do not necessarily apply
to personal property, for example. When you observe a discrepancy, ask yourself
whether there is a good justification for the difference.
