\expected{intro-mortgage-background}

\item \label{mortgage-interests} As an initial matter, pay attention to the
property interests involved.
First, there is the promissory note itself, which gives the loan originator (the
bank) has the right to receive monthly payments.
\having{narrative-negotiability}{But recall the discussion of the doctrine of
\useterm{negotiability}.}{But we will later learn about the doctrine of
\term{negotiability}, which allows the holder of a promissory note to
transfer the note to another person, giving that other person the right to
collect the promised payments.}{But under the doctrine of \term{negotiability},
the holder of a promissory note may transfer the note to another person, giving
that other person to collect the promised payments.}
By that doctrine, the note is alienable---the originator can sell it to another
bank, or a loan servicer, or a financial institution. In that sense, the note
itself is a kind of property.

\expected{intro-estates}
\expected{narrative-estates-working}

Second, there are the property interests relating to the real property.
The mortgagor has a sort of possessory estate, insofar as the mortgagor gets to
live on the mortgaged land. The mortgagee has a kind of future interest.

You can, and should, notate these using the charts presented in the chapter on
Estates and Future Interests. Say that O owns a house in fee simple absolute, O
sells it to A, and A mortgages the house to bank B:
\begin{center}
\begin{tabular}{l|lll}
Event & O & A & B \\
\hline
Start & FSA & & \\
O to A & & FSA & \\
Mortgage & & \#1: MPE & MFI(\#1) \\
\end{tabular}
\end{center}
MPE stands for ``mortgage possessory estate,'' and MFI(\#1) stands for
``mortgage future interest following interest \#1.'' (These are just names I
made up; there aren't conventional terms for these interests.) You can draw a
similar chart for the interests relating to the promissory note.

Now consider events that can occur in the course of the mortgage, by adding them
to the chart above. (Treat each event separately, such that it occurs
immediately after the mortgage is formed without any intervening events.)
\begin{itemize}
\item A sells the house to C; there is no due-on-sale clause and C does not
assume the mortgage.
\item B goes bankrupt and all its assets go to D.
\item A pays off the mortgage in full.
\item A defaults.
\end{itemize}
Come up with more complex sequences of events, to test your grasp of the
operation of mortgages.

\item \textbf{The foreclosure crisis.} The original \emph{Open Source Property}
module on mortgages provides a
more detailed explanation of the 2007--2010 mortgage and foreclosure crisis in
the United States. But this overview of how mortgages work is enough to provide
the seeds for understanding what happened. Consider the following.

\item Foreclosure sales are supposed to recover the fair market value of the
mortgaged land, which ought to be enough to repay the mortgagor's debt and also
return additional equity that the mortgagor has built up through payments. These
sales are usually conducted by an auction. Do you believe that these auctions
actually recover the fair market value? Who shows up to these auctions?

\item When a bank offers a mortgage to a homebuyer, presumably the bank hopes
that the homebuyer will pay off the mortgage and not default. Foreclosure is a
costly, messy process. That's why credit ratings and background checks are so
important for getting mortgages. What might lead a bank to be willing to offer a
mortgage to a homebuyer who is at higher risk of default---a \term{subprime
mortgage}? Perhaps if housing prices are rising faster than expected, as they
were between 2001 and 2006?

\item A real estate mortgage is a useful security interest against a mortgage
debt because the real estate is presumably more valuable than the debt. (That's
also why a down payment around 20\% is required.) What happens if housing prices
fall so much that the real estate is worth less than the debt? This is called an
\term{underwater mortgage}. What are the incentives of the mortgagor and the
mortgagee?

\defjrnart{levitin-securitization}{Adam J. Levitin, The Paper Chase:
Securitization, Foreclosure, and the Uncertainty of Mortgage Title, 63 Duke Law
Journal 637 (2013)}

\item If mortgages are property that can be bought and sold, they can be turned
into investment vehicles. This
process\having{levitin-securitization-foreclosure}{, as described previously,}{,
described in more detail below,}{ (described in detail in
\clause{levitin-securitization})} is called
\term{securitization}. Similar to how
stocks for multiple companies can be bundled together to make a mutual
fund, where one can buy shares and receive a cut of all the companies'
dividends, multiple mortgages can be bundled together in a ``mortgage-backed
security,'' where shareholders in the security are entitled to a cut of the
profits (i.e., the interest payments) from the
foreclosures.\having{levitin-securitization-foreclosure}{}{}{ Typically, the
mortgages themselves are held by a legal entity such as a trust, which pays out
the interest payments as the trust proceeds.}

Who might buy these mortgage-backed securities? Investment bankers? Pension and
retirement funds? You? And what happens to these investments when the mortgages
go sour, for any of the reasons given above?

\expected{jackson-v-brownson}

\item As we have seen, the current possessory estate holder can owe duties to
future interest holders, under the doctrine of waste. What about the other way
around---can a future interest holder owe a duty to the possessory estate
holder? Consider the problem of mortgage servicing, described below.

\expectnext{intro-foreclosure-abuses}


