Money and property always seem to go together. A common way of joining the
two is the \term{security interest}, in which one person's property right is
used as security to guarantee a debt.

Consider the following: Alice needs to borrow money to buy a printing press (to
run her newspaper business), and Bob has cash to lend. In exchange for the
loan, Alice promises to pay Bob in monthly installments, with interest. But Bob
is worried---what if Alice skips town and stops making the payments? So Bob
wants to use the printing press as collateral for the loan. If Alice fails to
make a payment, that is, if she \term[default]{defaults}, Bob gets to keep the
printing press, which Bob can hopefully sell for enough money to recover the
value of the loan.

\expected{intro-good-faith-purchasers}

Alice and Bob could make these arrangements purely by contract, of course. But
what if Alice first sells the printing press to Charlie, and then skips town and
stops making the payments? Bob cannot sue Charlie for breach of a contract to
which Charlie was not a party. Bob's only option is to sue Alice for breach of
contract and hope that Alice can pay (if Bob can even find her).

So what Bob really wants is a property right in the printing press. Not a
current right to use it (that's what Alice needs), but a right to take it in the
event of a default. Like a reversion or remainder to a life estate that converts
into a possessory estate upon the life tenant's death, Bob's desired property
interest should convert into a possessory right to the printing press upon the
event of a default.

In other words, \emph{a security interest is simply a type of future interest},
and all the mechanics of the system of estates will help to explain the
mechanics of security interests. There will be complications, of course, which
this chapter will explore. But the basic framework will be the same: there will
be current possessory estates and future interests, and certain events will
change those interests by operation of law.

Security interests can arise in a variety of ways, either voluntarily (where a
property owner uses the property as collateral for a loan) or involuntarily
(for example, to collect on tax debts or tort judgments). A voluntary security
interest on real property is typically called a \term{mortgage}, and (somewhat
confusingly) the phrase \term{secured transaction} in the United States
generally
refers to voluntary security interests in personal property under Article 9 of
the Uniform Commercial Code. The term \term{lien} is typically used for
involuntary
security interests, though it is sometimes used interchangeably with ``security
interest.'' \unskip\footnote{Unfortunately, this terminology is not standardized
as a general matter, and different pockets of law may use these terms
differently.}

Our focus in this chapter will be less on the formation of these security
interests, and more on how they work: what happens when a default occurs, if the
underlying property is sold, and so on. Pay close attention to what interests
everyone holds, who has possession when, what events affect the parties'
property interests, and what legal procedures must be followed.


