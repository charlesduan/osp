\editfromrepo{base}

\replacerange{\ldots The bill very briefly charges}{issued should not be
dissolved.}{
\readinghead{[Facts]}

[The plaintiff is the publisher of the Chicago Daily Tribune, and since March
29, 1924 has operated a Chicago-area radio station with call sign WGN (``World's
Greatest Newspaper''). The station uses a frequency of 990 kilocycles per
second, what today we would call 990 kHz or 990 AM\@. According to the
complaint, the radio station has over 500,000 listeners, and is ``of a
high-class character'' that ``has built up a good will with the public, which is
of great value to the complainant.''

\defwebsite{bachrach}{
author=Julia Bachrach,
title=Chicago's Apartment Hotels of the Roaring Twenties,
date=Jan 6 2020,
journal=Julia Bachrach Consulting,
url=https://www.jbachrach.com/blog/2020/1/3/chicagos-apartment-hotels-of-the-roaring-twenties,
}

The defendants run Guyon's Paradise Ball Room, a Chicago dance hall, and
operate a radio station WGES out of the dance hall.\edfootnote{External sources
suggest that the dance hall was attached to a hotel, each room equipped with a
radio so the guest could tune into the hotel's station. \sentence{see
bachrach}.} On September 7, 1926, the defendants changed their broadcast
frequency to 950 kHz. According to the plaintiff, that change ``has interfered
with and destroyed complainant's broadcasting to the public in the city of
Chicago.'' The defendants responded that no such interference was occurring, and
to the extent that it was, ``it is because said complainant's broadcasting
station is improperly constructed and operated.''

As a general matter, interference occurs when two radio stations near each other
broadcast on similar frequencies. The degree of closeness necessary to cause
interference was disputed between the parties, with the plaintiff demanding at
least 50 kHz of separation but the defendants saying that 40 kHz was enough. The
1,200 kHz frequency was available to the defendants, but they found it ``not
desirable for the purpose of broadcasting and that its use would render WGES of
little or no value as a broadcasting station.''

The plaintiff sought an injunction requiring defendants to change their
broadcasting frequency. The defendants contended in their answer that ``they
have invested large sums of money in and about their plant and will suffer
damage'' from an injunction. But according to the plaintiffs, ``the defendants
have never enjoyed any considerable degree of the good will of the public, nor
was it popular with the users of radio receiving sets, but was comparatively
unknown in Chicago or its vicinity.'']

\readinghead{[Analysis]}
}

\endedit
