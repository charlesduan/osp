\expected{harms-v-sprague}

\item The result in \emph{Harms}, in which the mortgage disappears if the joint
tenant who granted it predeceases the other joint tenant, is the most common
result in ``lien theory'' states, which represent the vast majority of states
today. However, for the reasons discussed in Harms, the results in ``title
theory'' states are mixed. \footnote{\emph{Compare Downing v. Downing}, 606 A.2d
208 (Md. 1992) (no automatic severance although Maryland is a ``title'' state),
\emph{with Schaefer v. Peoples Heritage Savings Bank}, 669 A.2d 185 (Me. 1996)
(mortgage severs joint tenancy), \emph{and General Credit Co. v. Cleck}, 609
A.2d 553 (Pa. Sup. Ct. 1992) (same); Taylor Mattis, \emph{Severance of Joint
Tenancies by Mortgages: A Contextual Approach}, 1977 \textsc{S. Ill. U. L.J.}
27.}


Suppose we adopted an intent-based standard to determine whether the joint
tenancy was severed. How would we have determined John Harms' intent after his
death?



\item Is the result in \textit{Harms} fair? Suppose John had instead survived
William. Would the mortgage burden half the interest in the property, or the
whole interest? \footnote{\textit{See} \emph{People v. Nogarr}, 330 P.2d 858
(Cal. Ct. App. 1958) (if the mortgaging joint tenant survives the nonmortgaging
joint tenant, the lien attaches to the entire interest).} Wouldn't the
mortgagees get a windfall if the value of their secured interest suddently
jumped in value? On the other hand, isn't that just the flip side of the loss
they suffer if William survives John? Should we create a hybrid that would
protect the lender, and burden William's interest after John's death, even
without severing?


Suppose the mortgage had worked a severance. If John had paid the mortgage off
before dying, should the severance be undone and the joint tenancy restored?
What would the parties likely have expected?



\item Given the result in \textit{Harms}, how will lenders behave when one
co-owner seeks to take out a loan? Sophisticated lenders make mistakes,
\footnote{\emph{See Texas American Bank v. Morgan}, 733 P.2d 864 (N.M. 1987).}
but mostly the lenders at risk are ordinary people, often relatives or friends
of the borrower.


What about a creditor who has a judgment against one joint tenant---what should
she do to make sure she can get access to the property to satisfy the judgment?
In practice, the creditor must act during the debtor's life to attach a lien to
the property and foreclose on that lien. \footnote{\textit{See, e.g.},
\emph{Rembe v.
Stewart}, 387 N.W.2d 313 (Iowa 1986); \emph{Jamestown Terminal Elev., Inc. v.
Knopp}, 246 N.W.2d 612 (N.D. 1976) (judgment lien on joint tenancy property did
not survive when debtor cotenant died before execution sale); \emph{Jackson v.
Lacy}, 97 N.E.2d 839 (Ill. 1951) (severance doesn't occur at foreclosure, but
only on expiration of the redemption period after foreclosure sale); \textit{see
also} \emph{Harris v. Crowder}, 322 S.E.2d 854 (W. Va. 1984) (a creditor may do
what the debtor could do, so a creditor of one joint tenant could convert a
joint tenancy into a tenancy in common, as long as the other cotenant's interest
wouldn't be otherwise prejudiced; an example of prejudice would be the loss of a
favorable interest rate on a mortgage due to the timing of the creditor's act).}



\item According to Charles Sprague's lawyer, Charles and John were romantically
involved. If the events underlying the case occurred today, they could have
married before John's death. Would that have changed anything?


In \emph{Riccelli v. Forcinito}, 595 A.2d 1322 (Pa. Super. Ct. 1991), discussed
above, Sam Riccelli and Carmen Pirozek had a joint tenancy. Four years later,
Sam Riccelli married Rita Riccelli. Carmen Pirozek lived in the Riccelli-Pirozek
property until her death in 1984. Her son lived in the house until Sam Riccelli
died in 1987; Rita Riccelli then sued to kick him out, claiming to be the sole
owner because Sam had inherited the whole property by right of survivorship. Did
the marriage sever the joint tenancy? It might seem that the marriage, which
gave Rita at least a potential interest in the property, severed the unities of
time, title, interest, and possession. However, the court held that marriage of
one joint tenant did not sever the joint tenancy. What's the best argument
against severance? Is it the same as the argument in \textit{Harms} against
allowing a mortgage given by only one joint tenant to sever the joint tenancy?



Compare the case of \textit{Goldman v. Gelman}, 77 N.E.2d 200 (N.Y. 2000).
Before a divorce decree became final, the wife gave her divorce attorney a
mortgage on the marital home, which was owned by the entirety, in order to
secure her debt to her attorney. The husband was awarded exclusive title to
the whole marital home. New York's highest court held that the divorce did not
destroy the mortgage, because the wife's interest was valid until the final
divorce decree, which turned the tenancy by the entirety into a tenancy in
common. The mortgage still burdened the wife's interest, and survived when the
wife's interest was transferred to the husband. Who ultimately has to pay the
wife's divorce lawyer?



\item \textbf{Other acts that might work a severance}. Technical breaches of the
four unities are unlikely to work a severance. For example, when one joint
tenant is adjudged an incompetent and the legal title to the incompetent's
property is assigned to a guardian, courts hold that no severance occurred.
\footnote{\textit{See, e.g.}, \emph{Moses v. Butner} (\emph{In re Estate and
Guardianship of Wood}), 14
Cal. Rptr. 147 (Cal. Ct. App. 1961).} Cases are divided on whether the grant of
a lease by one joint tenant works a severance. \footnote{\textit{Compare Tenhet
v. Boswell}, 554 P.2d 330 (Cal. 1976) (lease by one joint tenant does not sever
joint tenancy, though lease is terminated by death of leasing joint tenant),
\textit{with Estate of Gulledge}, 673 A.2d 1278 (D.C. 1996) (lease to third
person severs joint tenancy); \textit{see also In re Estate of Johnson}, 739
N.W.2d 493 (Iowa 2007) (adopting intent-based approach to severance).} Some
cases even suggest that a lease only works a temporary severance, and the joint
tenancy is automatically reformed when the lease ends. Isn't that a ridiculous
rule? Are the four unities doing any real work here?


The traditional rule was that, when property is held jointly by spouses, divorce
did not sever the joint tenancy. Unlike entireties property, jointly held
property need not be held by spouses, so the four unities remain intact even
after divorce. Does this make sense? Some states now presume severance upon
divorce.\footnote{\emph{See e.g.}, \textsc{Ohio Rev. Code Ann.} \S~5302.20(c)(5)
(Anderson 1996).} Others require courts to deal with the status of property as
part of the divorce decree. \footnote{\textit{See, e.g.}, \emph{Johnson v.
Johnson}, 169
N.W.2d 595 (Minn. 1969).} The majority rule is that divorce works a severance,
though the cases are divided; Helmholz argues that the results turn not on the
four unities but on the courts' best understanding of the parties' intent. In a
divorce case, both parties are alive, so it may seem possible to determine that
intent. As Helmholz points out, matters get dicey when a divorce or a sale is
pending and one of the spouses dies:
\begin{quote}
Most of these disputes arose where the parties were not thinking at all about
what would happen if one of them died. Why would they? They assumed that the
divorce would be completed or that the contract for sale would be fulfilled. In
most situations that is exactly what did happen. But not all. Where the
unexpected does happen and one party dies, litigation all too easily ensues. In
it, the courts have been left with the task of discovering the intent of the
parties from what are very often the slenderest of indications.
\end{quote}
Helmholz, \textit{supra}, at 25. Given that ``intent'' may be an unworkable
standard, is a formalist approach looking only to the four unities preferable
in that it at least provides courts with an answer?



Finally, where joint tenants have sought partition but the partition hasn't yet
occurred, the almost universal rule is that there is no severance until a court
has granted the partition, or at least until only the barest formalities remain
to finalize it. \footnote{\textit{See, e.g.}, \emph{Heintz v. Hudkins}, 824
S.W.2d 139 (Mo. Ct. App. 1992).} Helmholz again:
\begin{quote}
Although it may be said in favor of this rule that the parties might always have
changed their mind before the final decree, that seems a poor justification in
the face of their clearly expressed intent to sever and the untimely death of
one of them. The true reason for the rule must be a formal one: the rule is
necessary in order to safeguard the integrity of the underlying action for
partition. Partition cannot be effective before it is obtained. One cannot
secure the results of a judicial action simply by asking for it.
\end{quote}
Helmholz, \textit{supra}, at 30.


\item \textbf{What shares exist after severance?} The general assumption is that
joint tenants have equal shares after severance---after all, the unity of
interest requires that all joint tenants have equal shares \textit{before}
severance. However, if the equities strongly favored unequal shares, courts
might well bend the rules. \footnote{\textit{Compare} \emph{Cunningham v.
Hastings}, 556 N.E.2d 12 (Ind. Ct. App. 1990) (though one cotenant paid the
purchase price, the creation of a joint tenancy entitles each party to an equal
share of the proceeds on partition; equitable adjustments to cotenants' equal
shares are allowed for tenancies in common, not joint tenancies), \textit{with}
\emph{Moat v. Ducharme}, 555 N.E.2d 897 (Mass. App. Ct. 1990) (presumption of
equal shares is rebuttable because partition must be equitable), \textit{and}
\emph{Jezo v. Jezo}, 127 N.W.2d 246 (Wis. 1964) (presumption of equal shares is
rebuttable).}


\item \textbf{Joint tenants who kill.} The general rule is that a person who
intentionally causes another's death loses any inheritance rights he otherwise
would have had from his victim's estate. In \textit{Estate of Castiglioni}, 47
Cal. Rptr. 2d 288 (Ct. App. 1995), the surviving spouse petitioned for half of
the property she held in joint tenancy with her deceased husband, of whose
murder she was subsequently convicted. California Probate Code Section 251
provides in part: ``A joint tenant who feloniously and intentionally kills
another joint tenant thereby effects a severance of the interest of the
decedent so that the share of the decedent passes as the decedent's property
and the killer has no rights by survivorship.'' Thus, there was no question
that she could not inherit the entire property through a right of survivorship;
her husband's share went to her husband's heir, a daughter.


However, years before the murder, the husband put his separate property in joint
tenancy with the wife. The question was therefore whether the husband's share
was an undivided half of the former joint tenancy property, or whether
equitable tracing rules should apply to increase that share. The court of
appeals held the latter, and that it was error to give the killer half of the
joint tenancy property. The court noted that, had the tenancy been severed by
divorce rather than by murder, the widow/murderer wouldn't have received any of
the property at issue, because under California's community property regime the
husband would have been reimbursed by tracing his contributions to their joint
property. \textsc{Cal. Family Code} \S~2640(b). Thus, equitable principles
dictated that she should not be allowed to benefit from her crime, and her share
would be reduced by the amount necessary to reflect his contribution.



What should have happened if the couple had lived in a state without community
property rules, the source of the court's equitable tracing principle? Suppose
section 251 instead read: ``If a joint tenant feloniously and intentionally
kills another joint tenant, the share of the decedent passes as though the
killer had predeceased the decedent.'' What would the result be in
\textit{Estate of Castiglioni} in that situation?



\item \textbf{Simultaneous death.} What happens when two joint tenants die in
the same accident, or the order of their death can't be determined? The
Uniform Simultaneous Death Act initially provided that, without sufficient
evidence of the order of death, half of the property should be distributed as
if the first joint tenant had died first, and the other half as if the other
joint tenant had died first. This rule led to some unpleasant litigation and
``gruesome'' attempts by heirs to prove that a specific joint tenant died
first. The 1993 revision of the USDA states that, unless a governing
instrument such as a will specifies otherwise, the half-and-half approach will
be used in the absence of ``clear and convincing'' evidence that one joint
tenant survived the other by 120 hours.


\item \textbf{Joint accounts with rights of survivorship.} ``Joint accounts''
are bank accounts generally held by couples, children and parents, or business
partners. Each account holder has the ability to draw on the account. Many
joint accounts come with a right of survivorship: If a joint account owner
dies, the survivor(s) get all the money---creating another way around the
delays involved in probating a will.


In many states, joint account-holders do not have the same undivided interest
and rights to the use and enjoyment of the deposits that joint owners of real
property do. That is, the donee/nondepositor isn't entitled to the funds
unless she survives the donor/depositor. \emph{See} \textsc{Uniform Probate
Code} \S~6-211
(2008). On the donor/depositor's death, the majority rule is that the
surviving joint tenant takes the balance in a joint account unless there is
clear and convincing evidence that the depositor's intent was to create a
``convenience account,'' that is, an account that was supposed to be used by
the nondepositor---usually a younger relative---to take care of the
depositor's business affairs. Some jurisdictions conclusively presume that the
surviving joint tenant should receive the balance. \footnote{\emph{See}
\emph{Wright v. Bloom}, 635 N.E.2d 31 (Ohio 1994).}



What should happen if Orlando deposits \$10,000 in a joint bank account with
Abbie, and Abbie then withdraws \$5000 from the account while Orlando is alive,
without his permission or later agreement? Orlando can force Abbie to return
the money. Why not presume that Orlando intended a present gift to Abbie? By
the same logic, her creditors can't reach all the money to satisfy their claims
against her unless and until she survives the donor/depositor. \footnote{N.
William Hines, \emph{Personal Property Joint Tenancies: More Law, Fact and
Fancy}, 54 \textsc{Minn. L. Rev.} 509 (1970).}



However, the presumption against a present gift can be overcome by clear and
convincing evidence. In a minority of jurisdictions, joint account owners have
equal shares in the account during their lifetimes, as in a joint tenancy in
land.



Joint accounts with a right of survivorship can be used as a will substitute,
but there are potential tax consequences, not to mention risks of dispute
during the time the person who put the money in the account is alive, or
disputes after death when alternate heirs argue that the account was never
intended to benefit the survivor. If the depositor's intent is to give
whatever money is in the account to the non-depositing joint account holder
when the depositor dies but not before, many states allow accounts to be
designated ``payable on death,'' preventing the non-depositing account holder
from withdrawing the money while the depositor is alive. In the alternative, a
revocable inter vivos trust will also provide the desired results. As for an
elderly parent who wants her child to use money for her care, a better solution
would be a power of attorney, making her child into her agent with the power to
act on her behalf. This power of attorney would end with the parent's death.



\item \textbf{Why not allow severance by will?} If a joint tenant can
sever without constraint during her lifetime, why not by will? Courts will not
recognize such a transfer. \footnote{\textit{See, e.g.}, \emph{Gladson v.
Gladson}, 800 S.W.2d 709 (Ark. 1990).} There is an easy formalist explanation:
by definition, the joint tenant's interest ends at her death and ownership
automatically passes to the survivor, so there is nothing for her to pass by
will. But isn't this just playing with definitions? A number of cases have
allowed severance by will when the joint owners make joint wills, indicating a
clear intent to sever at death, on the theory that it's the agreement to make
the joint will that severs the joint tenancy.


The best explanation for the ``no severance by will'' rule is that it is about
the operation of the system of wills, and preserves the use of joint tenancy as
a device to avoid probate, even if it frustrates the intent of the testator. In
addition, a joint tenant who severs by will is playing a no-lose game at the
other tenant's expense. If she dies first, her designated heir takes her share.
If she survives the other tenant, she takes all. If she has to sever during her
lifetime, the severance occurs, whether that ends up benefiting her or not.
This rule may not matter much given the cavalier way states allow secret
severances, but still, severance by will is so contrary to the sharing spirit
of joint tenancies that the rule requiring joint wills makes sense.

