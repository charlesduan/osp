\editfromrepo{base}

\replacerange{Though partition in kind is supposedly favored}{The Uniform
Partition of Heirs' Property Act}{\par
\item \textbf{Heirs' property.} When a property owner dies without a will, the
state intestacy laws often divide ownership across multiple relatives, giving
each a share as a tenant in common. Over multiple generations of intestacy,
ownership can become highly fractionated. This is unfortunately common for
families lacking access to legal resources, or those struck by disaster (say, in
New Orleans after Hurricane Katrina).

If this ``heirs' property'' becomes valuable for development, third parties
would often acquire the interest of a distant relative who has a fractional
share and petition for partition. Given the often hundreds of people who own
interests in a piece of heirs' property, courts generally hold that partition in
kind is impossible. The resulting sale can dispossess people who have lived on
or used the land for decades; family members who would like to keep the land are
rarely able to outbid developers, who nonetheless usually pay substantially
below‑market prices because of the forced nature of the sale.

The Uniform Partition of Heirs' Property Act}

\endedit
