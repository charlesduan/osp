\defcase{downing-v-downing}{Downing v. Downing, 606 A.2d 208 (Md. 1992)}
\defcase{taylor-v-taylor}{Taylor v. Taylor, 17 N.W.2d 745 (Mich. 1945)}
\defcase{hoover-v-smith}{Hoover v. Smith, 444 S.E.2d 546 (Va. 1994)}
\defcase{hunter-v-hunter}{Hunter v. Hunter, 320 S.W.2d 529 (Mo. 1959)}
\defcase{snover-v-snover}{Snover v. Snover, 502 N.W.2d 370 (Mich. Ct. App.
1993)}

\editfromrepo{base}

\replacerange{A conveyance ``to Alice and Beth}{tenancy in common with
remainder to survivor).}{A conveyance ``to Alice and Beth as joint tenants, and
not as tenants in common,'' will create a joint tenancy in most states.
``Joint'' alone, however, may not be enough insofar as it might indicate a
colloquial sense of ownership together, rather than the particular legal device.
\sentence{compare downing-v-downing (``to A and B as joint tenants'' creates
joint tenancy); with taylor-v-taylor (``jointly,'' absent further circumstantial
evidence, does not suffice to create joint tenancy)}. Some states require a
specific invocation of a right of survivorship. \sentence{see, e.g.,
hoover-v-smith}. But in other states, language like ``as joint
tenants with the right of survivorship'' might create life estates in the
supposedly joint tenants, with the remainder to the survivor after one dies.
\sentence{see hunter-v-hunter; snover-v-snover}.}

\replacerange{, is an example of how the law can lay traps}{therefore
meaningless in the context of estates.}{, for example, considered a deed
from a mother conveying property to three children ``jointly and severally, and
unto their heirs, assigns and successors forever.'' There was substantial
evidence of intent to create a joint tenancy with right of survivorship: one of
the children testified that her mother told her lawyer that she wanted the deed
drafted so that, if one of her children died, the property would belong to the
other two children, and so on. Nevertheless, the court of appeals
noted that the conveyance contained no express language indicating a right of
survivorship, so the state's default statutory policy, favoring tenancy in
common absent express language to the contrary, overrode the mother's intent.}

\endedit
