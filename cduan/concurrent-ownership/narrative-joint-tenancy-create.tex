The traditional test for the creation and continuation of a joint tenancy
depended upon the presence of the four ``unities'': (1) time---the joint
tenants' interests were all acquired at the same time; (2) title---the interests
were all acquired by the same document or by joint adverse possession; (3)
interest---the tenants' shares must all be equal and undivided; and (4)
possession---all joint tenants must have equal rights to possess the whole (in
the absence of an agreement to the contrary\footnote{At common law, joint
tenants could not hold unequal shares, and attempting to create such a tenancy
would create a tenancy in common. However, modern courts are increasingly
willing to accept a clearly shown intent to hold unequal shares. \emph{See}
\emph{Moat v. Ducharme}, 555 N.E.2d 897 (Mass. App. 1990) (unequal
contributions); \emph{Jezo v. Jezo}, 127 N.W.2d 246 (Wis. 1964) (evidence of
contrary intent can override presumption of equal shares).}):
\begin{quote}
Unless the unities existed at the tenancy's inception, or if they were broken at
any subsequent point, the joint tenancy was automatically severed, and the
owners became tenants in common. This requirement meant, for example, that the
owner of property could not create a joint tenancy in himself and others
without first making use of a straw man. Because all joint tenants had to
receive their interest in the property at the same time and by the same title,
the owner had first to convey to a third party, who would in turn convey the
property back to the grantor and the other tenants. They would then take in
joint tenancy. Without this purely formal step, however, they would be only
tenants in common.
\end{quote}
R. H. Helmholz, \textit{Realism and Formalism in the Severance of Joint
Tenancies}, 77 \textsc{Neb. L. Rev.} 1 (1998). Today (as was already largely
true in the 1950s), the necessity for using a straw man to create a joint
tenancy has been largely eliminated from American law, sometimes by judicial
decision but more often by statutory enactment. We will examine this issue
further below, when we discuss severance of a joint tenancy.

A conveyance ``to Alice and Beth as joint tenants, and not as tenants in
common,'' will create a joint tenancy in most states. \textit{See} \emph{Kurpiel
v. Kurpiel}, 271 N.Y.S.2d 114 (N.Y. Sup. Ct. 1966) (joint tenancy created). Most
states consider that this language confirms the grantor's intent---``joint''
alone might have been misunderstood by a layperson who thinks that tenants in
common are joint owners in a general sense, though some states accept ``to Alice
and Beth jointly'' as sufficient to create a joint tenancy.
\footnote{\textit{Compare}
\emph{Downing v. Downing}, 606 A.2d 208 (Md. 1992) (``to A and B as joint
tenants'' creates a joint tenancy where the state statute provides that a
tenancy in common is created unless a written instrument ``expressly provides
that the property granted is to be held in joint tenancy''), \textit{and}
\emph{Germaine v. Delaine}, 318 So. 2d 681 (Ala. 1975) (``jointly as tenants in
common'' created a joint tenancy where the deed indicated a clear intent for
survivorship), \textit{with} \emph{Taylor v. Taylor}, 17 N.W.2d 745 (Mich. 1945)
(``jointly,'' absent circumstantial evidence of intent to create the legal
effect of a joint tenancy, does not suffice to create a joint tenancy);
\emph{Montgomery v. Clarkson}, 585 S.W.2d 483 (Mo. 1979) (``jointly'' is not
``express declaration'' of joint tenancy, as required by state statute);
\emph{Overheiser v. Lackey}, 100 N.E. 738 (N.Y. 1913) (where the layman who
prepared a will used ``jointly,'' the will created a tenancy in common),
\textit{and} \emph{Householter v. Householter}, 164 P.2d 101 (Kan. 1945)
(``jointly,'' used five times in a will prepared by a person who had served as a
probate judge, created a joint tenancy).}

In some states, precedents require more, usually specific invocation of a right
of survivorship. \footnote{\textit{Compare} \emph{Germaine v. Delaine}, 318 So.
2d 681 (Ala. 1975) (deed to A and B ``jointly, as tenants in common and to the
survivor thereof'' created joint tenancy because of survivorship language),
\textit{with} \emph{Hoover v. Smith}, 444 S.E.2d 546 (Va. 1994) (``to A and B as
joint tenants, and not as tenants in common'' was insufficient to create a joint
tenancy because it was not explicit about the right of survivorship).}
In other states, however, use of that same language will cause problems.
\footnote{\textit{See, e.g.}, \emph{Hunter v. Hunter}, 320 S.W.2d 529 (Mo. 1959)
(will devising property to A and B ``as joint tenants with the right of
survivorship'' created life estates with remainder to the survivor);
\emph{Snover v. Snover}, 502 N.W.2d 370 (Mich. Ct. App. 1993) (``to A and B as
joint tenants with full rights of survivorship and not as tenants in common''
created life estate in tenancy in common with remainder to survivor).} Be sure
you understand what the problem is: under what circumstances will it make a
difference whether A and B have a joint tenancy, with right of survivorship, or
instead have a tenancy in common in life estate, with the remainder to the
survivor? Courts sometimes refer to the latter as an ``indestructible''
remainder, which is confusing language---the remainder can't be destroyed by the
\textit{other} cotenant, whereas a right of survivorship in a joint tenancy can
be unilaterally destroyed.

It is vitally important to consult your state's statutes and precedent before
drafting a conveyance to more than one owner. \emph{James v. Taylor}, 969 S.W.2d
672 (Ark. App. Ct. 1998), is an example of how the law can lay traps for the
well-intentioned but poorly advised. The issue in the case was whether a deed
conveyed property from a mother to her three children as tenants in common or as
joint tenants. The court of appeals reversed an initial ruling that the
conveyance created a joint tenancy. The deed named the three children ``jointly
and severally, and unto their heirs, assigns and successors forever,'' and the
mother retained a life estate. Two of the three children subsequently died, and
then the mother died. Melba Taylor, the surviving child, sought a declaration
that she was the sole owner, while the heirs of the other two children opposed
her. Arkansas, like most states, provides that every shared interest in land
``shall be in tenancy in common unless expressly declared in the grant or devise
to be a joint tenancy.'' \textsc{Ark. Code Ann.} \S~18-12-603 (1987).

The heirs argued that any ambiguity therefore pointed to a tenancy in common,
whereas Taylor argued that her mother's intent to create a joint tenancy could
be determined from the surrounding circumstances. The evidence of such intent
was relatively strong: Taylor testified that her mother told her lawyer that
she wanted the deed drafted so that, if one of her children died, the property
would belong to the other two children, and so on; and that her mother was
upset when she learned, just before her death, that there was a problem with
the deed. In addition, after the first child died, the mother drafted a new
will splitting her property between her two living children and giving nothing
to the dead child's heirs, and the mother deleted the names of each dead child
from bank accounts payable on death, leaving only Taylor's name.

The court of appeals nonetheless held that the policy of the statute, favoring
tenancy in common unless a joint tenancy was expressly granted, overrode any
inquiry into the mother's intent. While the words ``joint tenancy'' didn't
need to be used, some intent to convey a survivorship estate needed to appear
in the grant. The words ``jointly and severally'' were insufficient,
contradictory, and therefore meaningless in the context of estates.

Assuming a court looked for extrinsic evidence of the drafter's intent in a case
involving ambiguous language, what would constitute persuasive evidence of an
intent to create a joint tenancy?

