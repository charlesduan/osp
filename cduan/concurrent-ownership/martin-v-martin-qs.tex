\defcase{stylianopoulos}{
parties=Stylianopoulos v. Stylianopoulos,
cite=455 N.E.2d 477,
court=Mass. Ct. App.,
year=1983,
}

\defcase{reitmeier}{
parties=Reitmeier v. Kalinoski,
cite=631 F. Supp. 565,
court=D.N.J.,
year=1986,
}

\defcase{cohen-v-cohen}{Cohen v. Cohen, 746 N.Y.S.2d 22 (App. Div. 2002)}
\defcase{hare-v-chisman}{Hare v. Chisman, 101 N.E.2d 268 (Ind. 1951)}
\defcase{johnson-v-james}{Johnson v. James, 377 S.W.2d 44 (Ark. 1964)}


\editfromrepo{base}


\replacerange{What if the property is a single-family home and the
cotenants}{mortgagor was constructively ousted from a single-family home).}{Can
a ``constructive ouster'' occur without one of the cotenants explicitly barring
another from possession? Though this question can arise in a variety of contexts
(what if the property is physically too small for all the cotenants to live
in?), a common one is separation or divorce of a couple. Should the person
moving out be considered ``constructively ousted'' and entitled to rent from the
significant other retaining possession?
\sentence{compare stylianopoulos (yes); with reitmeier (no); see also
cohen-v-cohen (no right to rent for period during which a court protective order
barred cotenant from the property due to his assaultive conduct)}.}



\replacerange{Because each cotenant has the right to possession, it can be
difficult for one cotenant to possess adversely to another.}{(finding in favor
of claimant who'd been in possession for thirty years under a quitclaim deed
purporting to give title to the entire property).}{A cotenant can adversely
possess the share of another cotenant. But it is typically much more difficult
to show the elements of adverse possession (which ones?), because the ordinary
expectation is that each cotenant may possess the entire property.
\sentence{compare hare-v-chisman (husband's sole possession of house after wife
died was not adverse to his cotenants, her heirs, since it ``was not an
unnatural act of them to permit their father to occupy this property, collect
the income, pay the expense, and enjoy the surplus''); with johnson-v-james
(presumption against adversity is even stronger when cotenants are related,
though presumption was overcome through sole possession for 36 years, where
cotenants knew of a will purportedly granting occupant sole possession and said
nothing)}.}


\replacerange{In the U.S., ``joint authorship'' occurs}{for property that can't
be exclusively possessed?}{\having{erickson-v-trinity-theatre}{\emph{Erickson
v.~Trinity Theatre}, presented earlier, discussed joint ownership of
copyrights.}{\emph{Erickson v.~Trinity Theatre}, presented later in this book,
discusses joint ownership of copyrights.}{Intellectual property is also subject
to concurrent ownership. For example, co-authors of a work are treated as joint
owners of the copyright in that work.} (``Joint'' here does not refer to joint
tenancy; it is an unfortunately imprecise colloquialism in copyright law.)
\having{erickson-v-trinity-theatre}{Compare the rules of copyright ownership
among multiple authors in that case to the rules of tenancy in common
here.}{When you read that case, compare the rules of copyright ownership among
multiple authors with the rules of tenancy in common here.}{The typical rule is
that each author owns and equal share of the copyright, and has the full
unilateral power to license the work and permit others to use it---even against
the will of the co-authors. How does this compare with tenancy in common?}}


\endedit
