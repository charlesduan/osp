\editfromrepo{base}

\replace{property by virtue of state law.\dots}{property by virtue of state law.
In order to understand these rights, the tenancy by the entirety must first be
placed in some context.\ldots

A tenancy by the entirety is a unique sort of concurrent ownership that can only
exist between married persons. Because of the common-law fiction that the
husband and wife were one person at law (that person, practically speaking, was
the husband), Blackstone did not characterize the tenancy by the entirety as a
form of concurrent ownership at all. Instead, he thought that entireties
property was a form of single ownership by the marital unity. Neither spouse was
considered to own any individual interest in the estate; rather, it belonged to
the couple.\ldots

%Like joint tenants, tenants by the entirety enjoy the right of survivorship.
%Also like a joint tenancy, unilateral alienation of a spouse's interest in
%entireties property is typically not possible without severance. Unlike joint
%tenancies, however, tenancies by the entirety cannot easily be severed
%unilaterally. Typically, severance requires the consent
%of both spouses, or the ending of the marriage in divorce.
%At common law, all of the other rights associated with the
%entireties property belonged to the husband: as the head of the household, he
%could control the use of the property and the exclusion of others from it and
%enjoy all of the income produced from it. The husband's control of
%the property was so extensive that, despite the rules on alienation, the common
%law eventually provided that he could unilaterally alienate entireties property
%without severance subject only to the wife's survivorship interest.
%
%With the passage of the Married Women's Property Acts in the late 19th century
%granting women distinct rights with respect to marital property, most States
%either abolished the tenancy by the entirety or altered it significantly.
Michigan's version of the estate is typical of the modern tenancy by the
entirety. Following Blackstone, Michigan characterizes its tenancy by the
entirety as creating no individual rights whatsoever: ``It is well settled under
the law of this State that one tenant by the entirety has no interest separable
from that of the other\ldots. Each is vested with an entire title.'' And yet, in
Michigan, each tenant by the entirety possesses the right of survivorship. Each
spouse---the wife as well as the husband---may also use the property, exclude
third parties from it, and receive an equal share of the income produced by it.
Neither spouse may unilaterally alienate or encumber the property, although this
may be accomplished with mutual consent. Divorce ends the tenancy by the
entirety, generally giving each spouse an equal interest in the property as a
tenant in common, unless the divorce decree specifies otherwise.}

\endedit
