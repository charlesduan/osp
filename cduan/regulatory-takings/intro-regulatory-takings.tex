Traditional eminent domain actions involve the government intentionally
selecting a property right to appropriate---a parcel of land, for example. But
the Fifth Amendment also limits the ability of the state to regulate. Property
owners sometimes challenge property regulations as being so onerous that it is
as if the state has appropriated property and compensation is therefore due.
This chapter will review the Supreme Court's case law on these so-called
\term{regulatory takings}.

\defcase{mahon}{
Pennsylvania Coal Co. v. Mahon, 260 U.S. 393 (1922)
}

\Inline{mahon} is generally treated as the origin of the regulatory takings
doctrine. A mining company held property rights to mine coal in a certain area,
but sold the surface rights for private houses. Subsequently, Pennsylvania
enacted a law prohibiting coal mining in ways that might cause subsidence of a
house, rendering the mining company's rights worthless. Justice Holmes, for the
Supreme Court, recognized the tension in calling the mining regulation a taking:
\begin{quote}
Government hardly could go on if to some extent values incident to property
could not be diminished without paying for every such change in the general law.
As long recognized, some values are enjoyed under an implied limitation and must
yield to the police power.
\end{quote}
But does that mean that the regulated property owner must always eat the costs
of public-beneficial regulation? No:
\begin{quote}
In general it is not plain that a man's misfortunes or necessities will justify
his shifting the damages to his neighbor's shoulders. We are in danger of
forgetting that a strong public desire to improve the public condition is not
enough to warrant achieving the desire by a shorter cut than the constitutional
way of paying for the change.
\end{quote}
Accordingly, the Court concluded that the regulation was a taking:
\begin{quote}
The general rule at least is, that while property may be regulated to a certain
extent, if regulation goes too far it will be recognized as a taking.
\end{quote}
The three cases in this section will explore what ``too far'' means. The first
establishes the general balancing test used for regulatory takings. The other
cases explore two major exceptions in which a regulation is considered a
\term{per se
taking} without need for balancing: (1) where the property's economic value is
entirely wiped out, and (2) where the regulation causes a physical occupation of
property.

As you read these cases, consider the many different areas that the U.S.
government regulates: the environment, food quality, pharmaceuticals,
telecommunications, electricity, and more. What do these cases say about the
government's regulatory powers? How could you use the concepts and doctrines of
property, expansive over many types of subject matter, to advance or forestall
these regulatory objectives on behalf of your clients or in the public interest?
Besides being an especially controversial and debate-worthy area of law, the
regulatory takings doctrine demonstrates the rhetorical and doctrinal power of
this concept of ``property'' that we have explored.


