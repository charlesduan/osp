
\defcase{loretto}{
parties=Loretto v. Teleprompter Manhattan CATV Corp.,
cite=458 U.S. 419,
year=1982,
}

In \inline{loretto}\optclause, a New York statute required landlords to allow
cable television companies to install a cable and outlets on the leased
property. The installed components were tiny---a box four inches to the side,
plus wiring---and actually improved the value of the property, according to the
courts. Was such a regulation a taking? The Supreme Court held that it
categorically was a taking, short-circuiting the \emph{Penn Central} test,
because of the physical occupation:
\begin{quote}
[W]e have long considered a physical intrusion by government to be a property
restriction of an unusually serious character for purposes of the Takings
Clause.\ldots To borrow a metaphor, the government does not simply take asingle
``strand'' from the ``bundle'' of property rights: it chops through the bundle,
taking a slice from every strand.
\end{quote}
Among other things, the Court claimed that its per se rule ``presents relatively
few problems of proof'' and ``will rarely be subject to dispute.'' Is it so
easy? Consider this case.

\expectnext{cedar-point-v-hassid}
