\expected{fair-housing-act}

\item
You may have noticed the omission of ``handicap'' from \S~3604(a) and (b).
Paragraph (f) fills this gap, making it also unlawful to ``discriminate in the
sale or rental'' of a dwelling, or the terms thereof, ``because of on a handicap
of the buyer or renter,'' or someone related to the buyer or renter.
Discrimination includes a refusal to make policy accommodations or to permit
modifications to the premises, if those accommodations or modifications are
``reasonable'' and ``necessary'' to afford ``full enjoyment of the premises.''
Certain newly-constructed multifamily dwellings must also be designed according
to accessibility standards, per \S~3604(f)(3)(C).

\item There are several exceptions, such as for religious organizations and
``housing for older persons.'' \S~3607(a), (b)(1). The most controversial
of these, colloquially known as the ``Mrs.~Murphy exemption,'' is in \S~3603(b)
above. What does this exemption allow? If the act is intended to root out
pernicious discrimination, why include this provision?

It is crucial to note that the plain text of the Mrs.~Murphy exemption states
that it does not apply to \S~3604(c)---the subsection that prohibits
discriminatory advertising. Thus, although certain categories of landlords are
exempted from the statute’s basic framework, they are still not allowed to post
discriminatory advertisements.


\defcase{jones-v-alfred-mayer}{
parties=Jones v. Alfred H. Mayer Co.,
cite=392 U.S. 409,
year=1968,
}

\item The Civil Rights Act of 1866, codified at 42 U.S.C. \S~1982, ``bars all
racial discrimination, private as well as public, in the sale or rental of
property.'' \sentence{jones-v-alfred-mayer at 413}. Unlike the Fair Housing Act,
it has no exceptions or limitations, but it ``deals only with racial
discrimination,'' and not other forms. \sentence{jones-v-alfred-mayer at 413}.
