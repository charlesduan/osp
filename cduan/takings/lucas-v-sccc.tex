\reading{Lucas v. South Carolina Coastal Council}

\readingcite{505 U.S. 1003 (1992)}

\opinion Justice \textsc{Scalia} delivered the opinion of the Court.

In 1986, petitioner David H. Lucas paid \$975,000 for two residential lots on
the Isle of Palms in Charleston County, South Carolina, on which he intended to
build single-family homes. In 1988, however, the South Carolina Legislature
enacted the Beachfront Management Act,
which had the direct effect of barring petitioner from
erecting any permanent habitable structures on his two parcels.
A state trial court found that this prohibition rendered
Lucas's parcels ``valueless.'' This case requires us
to decide whether the Act's dramatic effect on the economic value of Lucas's
lots accomplished a taking of private property under the Fifth and Fourteenth
Amendments requiring the payment of ``just compensation.''


\readinghead{I}


\readinghead{A}

\ldots.
%South Carolina's expressed interest in intensively managing development
%activities in the so-called ``coastal zone'' dates from 1977 when, in the
%aftermath of Congress's passage of the federal Coastal Zone Management Act of
%1972, the
%legislature enacted a Coastal Zone Management Act of its own.
%In its original form, the South Carolina
%Act required owners of coastal zone land that qualified as a ``critical area''
%(defined in the legislation to include beaches and immediately adjacent sand
%dunes) to obtain a permit from the newly created South
%Carolina Coastal Council (Council) (respondent here) prior to committing the
%land to a ``use other than the use the critical area was devoted to on
%[September 28, 1977].''

In the late 1970's, Lucas and others began extensive residential development of
the Isle of Palms, a barrier island situated eastward of the city of Charleston.
Toward the close of the development cycle for one residential subdivision known
as ``Beachwood East,'' Lucas in 1986 purchased the two lots at issue in this
litigation for his own account. [At that time, South Carolina's Coastal Zone
Management Act did not impose restrictions on development of single-family
residences, which Lucas intended to build.]
%No portion of the lots, which were located
%approximately 300 feet from the beach, qualified as a ``critical area'' under
%the 1977 Act; accordingly, at the time Lucas acquired these parcels, he was not
%legally obliged to obtain a permit from the Council in advance of any
%development activity. His intention with respect to the lots was to do what the
%owners of the immediately adjacent parcels had already done: erect single-family
%residences. He commissioned architectural drawings for this purpose.

The Beachfront Management Act brought Lucas's plans to an abrupt end. Under that
1988 legislation, the Council was directed to establish a ``baseline''
connecting the landward-most ``point[s] of erosion\ldots during the past forty
years'' in the region of the Isle of Palms that includes Lucas's lots.
In action not challenged here, the
Council fixed this baseline landward of Lucas's parcels. That was significant,
for under the Act construction of occupiable improvements was flatly prohibited
seaward of a line drawn 20 feet landward of, and parallel to, the baseline.
The Act provided no exceptions.


\readinghead{B}

Lucas promptly filed suit in the South Carolina Court of Common Pleas,
contending that the Beachfront Management Act's construction bar effected a
taking of his property without just compensation. Lucas did not take issue with
the validity of the Act as a lawful exercise of South Carolina's police power,
but contended that the Act's complete extinguishment of his property's value
entitled him to compensation regardless of whether the legislature had acted in
furtherance of legitimate police power objectives.
%Following a bench trial, the
%court agreed. Among its factual determinations was the finding that ``at the
%time Lucas purchased the two lots, both were zoned for single-family residential
%construction and\ldots there were no restrictions imposed upon such use of the
%property by either the State of South Carolina, the County of Charleston, or the
%Town of the Isle of Palms.'' The trial court further found that the Beachfront
%Management Act decreed a permanent ban on construction insofar as Lucas's lots
%were concerned, and that this prohibition ``deprive[d] Lucas of any reasonable
%economic use of the lots,\ldots eliminated the unrestricted right of use, and
%render[ed] them valueless.'' The court thus concluded that Lucas's properties
%had been ``taken'' by operation of the Act, and it ordered respondent to pay
%``just compensation'' in the amount of \$1,232,387.50. 
%
[The trial court agreed and ordered just compensation of \$1.2 million.
The Supreme Court of South Carolina reversed, concluding that regulation ``to
prevent serious public harm'' is not a taking regardless no matter the effect on
property values.]


\readinghead{III}


\readinghead{A}

Prior to Justice Holmes's exposition in \textit{Pennsylvania Coal Co. v. Mahon},
260 U.S. 393 (1922), it was generally thought that the Takings Clause reached
only a ``direct appropriation'' of property
or the functional equivalent of a ``practical ouster of
[the owner's] possession.''
Justice Holmes recognized in \textit{Mahon}, however, that if the
protection against physical appropriations of private property was to be
meaningfully enforced, the government's power to redefine the range of interests
included in the ownership of property was necessarily constrained by
constitutional limits. If, instead, the uses of private property were subject to
unbridled, uncompensated qualification under the police power, ``the natural
tendency of human nature [would be] to extend the qualification more and more
until at last private property disappear[ed].'' These considerations gave birth
in that case to the oft-cited maxim that, ``while property may be regulated to a
certain extent, if regulation goes too far it will be recognized as a taking.'' 

Nevertheless, our decision in \textit{Mahon} offered little insight into when,
and under what circumstances, a given regulation would be seen as going ``too
far'' for purposes of the Fifth Amendment. In 70-odd years of succeeding
``regulatory takings'' jurisprudence, we have generally eschewed any ``set
formula'' for determining how far is too far, preferring to ``engag[e]
in\ldots essentially ad hoc, factual inquiries.'' \textit{Penn Central
Transportation Co. v. New York City}, 438 U.S. 104, 124 (1978).
We have, however,
described at least two discrete categories of regulatory action as compensable
without case-specific inquiry into the public interest advanced in support of
the restraint. The first encompasses regulations that compel the property owner
to suffer a physical ``invasion'' of his property.
\having{cedar-point-v-hassid}{[This is the \emph{Cedar Point} doctrine.]}{[This
will be discussed in the \emph{Cedar Point Nursery v. Hassid} case below.]}{In
general (at least with
regard to permanent invasions), no matter how minute the intrusion, and no
matter how weighty the public purpose behind it, we have required compensation.
For example, in \textit{Loretto v. Teleprompter Manhattan CATV Corp.}, 458 U.S.
419 (1982), we determined that New York's law requiring landlords to allow
television cable companies to emplace cable facilities in their apartment
buildings constituted a taking even though the facilities occupied at most only
1 {\textonehalf} cubic feet of the landlords' property.}

The second situation in which we have found categorical treatment appropriate is
where regulation denies all economically beneficial or productive use of
land.\ldots
%As we have said on numerous occasions, the Fifth Amendment is violated when
%land-use regulation ``does not substantially advance legitimate state interests
%\textit{or denies an owner economically viable use of his land.}''
%\textit{Agins, supra}, 447 U.S., at 260 (citations omitted) (emphasis
%added).
\unskip
\readingfootnote{7}{Regrettably, the rhetorical force of our
``deprivation of all economically feasible use'' rule is greater than its
precision, since the rule does not make clear the ``property interest'' against
which the loss of value is to be measured. When, for example, a regulation
requires a developer to leave 90\% of a rural tract in its natural state, it is
unclear whether we would analyze the situation as one in which the owner has
been deprived of all economically beneficial use of the burdened portion of the
tract, or as one in which the owner has suffered a mere diminution in value of
the tract as a whole.
%(For an extreme---and, we think, unsupportable---view of
%the relevant calculus, see \textit{Penn Central Transportation Co. v. New York
%City}, 42 N.Y.2d 324, 333--334, 366 N.E.2d 1271, 1276--1277 (1977), aff'd, 438
%U.S. 104 (1978), where the state court examined the diminution in a particular
%parcel's value produced by a municipal ordinance in light of total value of the
%takings claimant's other holdings in the vicinity.) Unsurprisingly, this
%uncertainty regarding the composition of the denominator in our ``deprivation''
%fraction has produced inconsistent pronouncements by the Court. Compare
%\textit{Pennsylvania Coal Co. v. Mahon}, 260 U.S. 393, 414 (1922) (law
%restricting subsurface extraction of coal held to effect a taking), with
%\textit{Keystone Bituminous Coal Assn. v. DeBenedictis}, 480 U.S. 470, 497--502
%(1987) (nearly identical law held not to effect a taking); see also
%\textit{id.}, at 515--520 (\textsc{Rehnquist}, C.J., dissenting); Rose,
%\textit{Mahon} Reconstructed: Why the Takings Issue is Still a Muddle, 57
%S.Cal.L.Rev. 561, 566--569 (1984). The answer to this difficult question may lie
%in how the owner's reasonable expectations have been shaped by the State's law
%of property---\textit{i.e.}, whether and to what degree the State's law has
%accorded legal recognition and protection to the particular interest in land
%with respect to which the takings claimant alleges a diminution in (or
%elimination of) value. In any event,
We avoid this difficulty in the present
case, since the ``interest in land'' that Lucas has pleaded (a fee simple
interest) is an estate with a rich tradition of protection at common law, and
since the South Carolina Court of Common Pleas found that the Beachfront
Management Act left each of Lucas's beachfront lots without economic value.}

We have never set forth the justification for this rule. Perhaps it is
simply\ldots
that total deprivation of beneficial use is, from
the landowner's point of view, the equivalent of a physical appropriation.
% See
%\textit{San Diego Gas \& Electric Co. v. San Diego}, 450 U.S., at 652
%(dissenting opinion).
``[F]or what is the land but the profits thereof[?]'' 1 E.
Coke, Institutes, ch. 1, \S~1 (1st Am. ed. 1812). Surely, at least, in the
extraordinary circumstance when \textit{no} productive or economically
beneficial use of land is permitted, it is less realistic to indulge our usual
assumption that the legislature is simply ``adjusting the benefits and burdens
of economic life,'' \emph{Penn Central},
in a manner that secures an ``average reciprocity of advantage'' to everyone
concerned, \emph{Mahon}. And the
\textit{functional} basis for permitting the government, by regulation, to
affect property values without compensation---that ``Government hardly could go
on if to some extent values incident to property could not be diminished without
paying for every such change in the general law''---does
not apply to the relatively rare situations where the government has deprived a
landowner of all economically beneficial uses.

On the other side of the balance, affirmatively supporting a compensation
requirement, is the fact that regulations that leave the owner of land without
economically beneficial or productive options for its use---typically, as here,
by requiring land to be left substantially in its natural state---carry with
them a heightened risk that private property is being pressed into some form of
public service under the guise of mitigating serious public harm.\ldots

We think, in short, that there are good reasons for our frequently expressed
belief that when the owner of real property has been called upon to sacrifice
\textit{all} economically beneficial uses in the name of the common good, that
is, to leave his property economically idle, he has suffered a
taking.\readingfootnote{8}{[In response to Justice Stevens's dissent:]
%Justice \textsc{Stevens} criticizes the ``deprivation
%of all economically beneficial use'' rule as ``wholly arbitrary,'' in that
%``[the] landowner whose property is diminished in value 95\% recovers nothing,''
%while the landowner who suffers a complete elimination of value ``recovers the
%land's full value.'' This analysis errs in its assumption that the landowner
%whose deprivation is one step short of complete is not entitled to compensation.
%Such an owner might not be able to claim the benefit of our categorical
%formulation, but, as we have acknowledged time and again, ``[t]he economic
%impact of the regulation on the claimant and\ldots the extent to which the
%regulation has interfered with distinct investment-backed expectations'' are
%keenly relevant to takings analysis generally. \textit{Penn Central
%Transportation Co. v. New York City}, 438 U.S. 104, 124 (1978).
It is true that
in at least \textit{some} cases the landowner with 95\% loss will get nothing,
while the landowner with total loss will recover in full. But that occasional
result is no more strange than the gross disparity between the landowner whose
premises are taken for a highway (who recovers in full) and the landowner whose
property is reduced to 5\% of its former value by the highway (who recovers
nothing). Takings law is full of these ``all-or-nothing'' situations.\ldots }


\readinghead{B}

%The trial court found Lucas's two beachfront lots to have been rendered
%valueless by respondent's enforcement of the coastal-zone construction
%ban.\readingfootnote{9}{This finding was the premise of the petition for
%certiorari, and since it was not challenged in the brief in opposition we
%decline to entertain the argument in respondent's brief on the merits that the
%finding was erroneous.} Under Lucas's theory of the case, which rested upon our
%``no economically viable use'' statements, that finding entitled him to
%compensation. Lucas believed it unnecessary to take issue with either the
%purposes behind the Beachfront Management Act, or the means chosen by the South
%Carolina Legislature to effectuate those purposes. The South Carolina Supreme
%Court, however, thought otherwise. In its view, the Beachfront Management Act
%was no ordinary enactment, but involved an exercise of South Carolina's ``police
%powers'' to mitigate the harm to the public interest that petitioner's use of
%his land might occasion.\ldots [and] within a long line of this Court's cases
%sustaining against Due Process and Takings Clause challenges the State's use of
%its ``police powers'' to enjoin a property owner from activities akin to public
%nuisances.
%See \textit{Mugler v. Kansas}, 123 U.S. 623 (1887) (law prohibiting
%manufacture of alcoholic beverages); \textit{Hadacheck v. Sebastian}, 239 U.S.
%394 (1915) (law barring operation of brick mill in residential area);
%\textit{Miller v. Schoene}, 276 U.S. 272 (1928) (order to destroy diseased cedar
%trees to prevent infection of nearby orchards); \textit{Goldblatt v. Hempstead},
%369 U.S. 590 (1962) (law effectively preventing continued operation of quarry in
%residential area).

[The Supreme Court does not explain this explicitly, but the South Carolina
Supreme Court reached its result of no taking based on a rule that ``a taking
has not been found when the regulation exists to prevent serious harm'' (i.e., a
nuisance). The U.S. Supreme Court responded as follows.]

%It is correct that many of our prior opinions have suggested that ``harmful or
%noxious uses'' of property may be proscribed by government regulation without
%the requirement of compensation. For a number of reasons, however, we think the
%South Carolina Supreme Court was too quick to conclude that that principle
%decides the present case. The ``harmful or noxious uses'' principle was the
%Court's early attempt to describe in theoretical terms why government may,
%consistent with the Takings Clause, affect property values by regulation without
%incurring an obligation to compensate---a reality we nowadays acknowledge
%explicitly with respect to the full scope of the State's police power.\ldots
%
%The transition from our early focus on control of ``noxious'' uses to our
%contemporary understanding of the broad realm within which government may
%regulate without compensation was an easy one, since
The distinction between
``harm-preventing'' and ``benefit-conferring'' regulation is often in the eye of
the beholder. It is quite possible, for example, to describe in \textit{either}
fashion the ecological, economic, and esthetic concerns that inspired the South
Carolina Legislature in the present case. One could say that imposing a
servitude on Lucas's land is necessary in order to prevent his use of it from
``harming'' South Carolina's ecological resources; or, instead, in order to
achieve the ``benefits'' of an ecological preserve.\ldots
%Whether one or the other of
%the competing characterizations will come to one's lips in a particular case
%depends primarily upon one's evaluation of the worth of competing uses of real
%estate. A given restraint will be seen as mitigating ``harm'' to the adjacent
%parcels or securing a ``benefit'' for them, depending upon the observer's
%evaluation of the relative importance of the use that the restraint favors.
%Whether Lucas's construction of single-family residences on his parcels should
%be described as bringing ``harm'' to South Carolina's adjacent ecological
%resources thus depends principally upon whether the describer believes that the
%State's use interest in nurturing those resources is so important that
%\textit{any} competing adjacent use must yield.
\unskip\readingfootnote{12}{In Justice
\textsc{Blackmun}'s view, even with respect to regulations that deprive an owner
of all developmental or economically beneficial land uses, the test for required
compensation is whether the legislature has recited a harm-preventing
justification for its action. Since such a justification can be formulated in
practically every case, this amounts to a test of whether the legislature has a
stupid staff. We think the Takings Clause requires courts to do more than insist
upon artful harm-preventing characterizations.}

%When it is understood that ``prevention of harmful use'' was merely our early
%formulation of the police power justification necessary to sustain (without
%compensation) \textit{any} regulatory diminution in value; and that the
%distinction between regulation that ``prevents harmful use'' and that which
%``confers benefits'' is difficult, if not impossible, to discern on an
%objective, value-free basis; it becomes self-evident that noxious-use logic
%cannot serve as a touchstone to distinguish regulatory ``takings''---which
%require compensation---from regulatory deprivations that do not require
%compensation. \textit{A fortiori}
\ldots The legislature's recitation of a noxious-use
justification cannot be the basis for departing from our categorical rule that
total regulatory takings must be compensated. If it were, departure would
virtually always be allowed. The South Carolina Supreme Court's approach would
essentially nullify \textit{Mahon}'s affirmation of limits to the noncompensable
exercise of the police power. Our cases provide no support for this: None of
them that employed the logic of ``harmful use'' prevention to sustain a
regulation involved an allegation that the regulation wholly eliminated the
value of the claimant's land.

Where the State seeks to sustain regulation that deprives land of all
economically beneficial use, we think it may resist compensation only if the
logically antecedent inquiry into the nature of the owner's estate shows that
the proscribed use interests were not part of his title to begin
with.\edfootnote{The remainder of the majority opinion has been reorganized and
heavily edited. Alterations are not marked.}
Any limitation so severe
cannot be newly legislated or decreed (without compensation), but must inhere in
the title itself, in the restrictions that background principles of the State's
law of property and nuisance already place upon land ownership.
A law or decree
with such an effect must, in other words, do no more than duplicate the result
that could have been achieved in the courts---by adjacent landowners (or other
uniquely affected persons) under the State's law of private nuisance, or by the
State under its complementary power to abate nuisances that affect the public
generally, or otherwise.
When, however, a regulation that declares ``off-limits'' all
economically productive or beneficial uses of land goes beyond what the relevant
background principles would dictate, compensation must be paid to sustain it.

In the case of personal property, by reason of the State's
traditionally high degree of control over commercial dealings, a property owner
ought to be aware of the possibility that new regulation might even render his
property economically worthless. See \textit{Andrus v. Allard},
444 U.S. 51, 66--67 (1979) (prohibition on sale of eagle feathers).
Or the owner of a lake-bed, for example, would not be entitled to
compensation when he is denied the requisite permit to engage in a landfilling
operation that would have the effect of flooding others' land. Nor the corporate
owner of a nuclear generating plant, when it is directed to remove all
improvements from its land upon discovery that the plant sits astride an
earthquake fault. Such regulatory action may well have the effect of eliminating
the land's only economically productive use, but it does not proscribe a
productive use that was previously permissible under relevant property and
nuisance principles. The use of these properties for what are now expressly
prohibited purposes was \textit{always} unlawful, and it was open to the State
at any point to make the
implication of those background principles of nuisance and property law
explicit.

It seems unlikely that common-law principles would have prevented the erection
of any habitable or productive improvements on petitioner's land; they rarely
support prohibition of the ``essential use'' of land. The question, however, is
one of state law to be dealt with on remand. We emphasize that to win its case,
South Carolina must identify background
principles of nuisance and property law that prohibit the uses he now intends in
the circumstances in which the property is presently found.

\opinion Justice \textsc{Kennedy}, concurring in the judgment.

%\ldots
%In my view, reasonable expectations must be understood in light of the
%whole of our legal tradition. The common law of nuisance is too narrow a confine
%for the exercise of regulatory power in a complex and interdependent society.
%The State should not be prevented from enacting new regulatory initiatives in
%response to changing conditions, and courts must consider all reasonable
%expectations whatever their source. The Takings Clause does not require a static
%body of state property law; it protects private expectations to ensure private
%investment.
\ldots I agree with the Court that nuisance prevention accords with the
most common expectations of property owners who face regulation, but I do not
believe this can be the sole source of state authority to impose severe
restrictions. Coastal property may present such unique concerns for a fragile
land system that the State can go further in regulating its development and use
than the common law of nuisance might otherwise permit.\ldots

%The Supreme Court of South Carolina erred, in my view, by reciting the general
%purposes for which the state regulations were enacted without a determination
%that they were in accord with the owner's reasonable expectations and therefore
%sufficient to support a severe restriction on specific parcels of
%property.\ldots

\opinion Justice \textsc{Blackmun}, dissenting.

Today the Court launches a missile to kill a mouse.\ldots
%
%The State of South Carolina prohibited petitioner Lucas from building a
%permanent structure on his property from 1988 to 1990. Relying on an unreviewed
%(and implausible) state trial court finding that this restriction left Lucas'
%property valueless, this Court granted review to determine whether compensation
%must be paid in cases where the State prohibits all economic use of real estate.
%According to the Court, such an occasion never has arisen in any of our prior
%cases, and the Court imagines that it will arise ``relatively rarely'' or only
%in ``extraordinary circumstances.'' Almost certainly it did not happen in this
%case.
%
%Nonetheless, the Court presses on to decide the issue, and as it does, it
%ignores its jurisdictional limits, remakes its traditional rules of review, and
%creates simultaneously a new categorical rule and an exception (neither of which
%is rooted in our prior case law, common law, or common sense). I protest not
%only the Court's decision, but each step taken to reach it. More fundamentally,
%I question the Court's wisdom in issuing sweeping new rules to decide such a
%narrow case.\ldots
%
My fear is that the Court's new policies will spread beyond the narrow confines
of the present case. For that reason, I, like the Court, will give far greater
attention to this case than its narrow scope suggests---not because I can
intercept the Court's missile, or save the targeted mouse, but because I hope
perhaps to limit the collateral damage.\ldots

%The South Carolina Supreme Court found that the Beachfront Management Act did
%not take petitioner's property without compensation. The decision rested on two
%premises that until today were unassailable---that the State has the power to
%prevent any use of property it finds to be harmful to its citizens, and that a
%state statute is entitled to a presumption of constitutionality.
%
%The Beachfront Management Act includes a finding by the South Carolina General
%Assembly that the beach/dune system serves the purpose of ``protect[ing] life
%and property by serving as a storm barrier which dissipates wave energy and
%contributes to shoreline stability in an economical and effective manner.'' The
%General Assembly also found that ``development unwisely has been sited too close
%to the [beach/dune] system. This type of development has jeopardized the
%stability of the beach/dune system, accelerated erosion, and endangered adjacent
%property.''
%
%If the state legislature is correct that the prohibition on building in front of
%the setback line prevents serious harm, then, under this Court's prior cases,
%the Act is constitutional. ``Long ago it was recognized that all property in
%this country is held under the implied obligation that the owner's use of it
%shall not be injurious to the community, and the Takings Clause did not
%transform that principle to one that requires compensation whenever the State
%asserts its power to enforce it.'' \textit{Keystone Bituminous Coal Assn. v.
%DeBenedictis}, 480 U.S. 470, 491--492 (1987) (internal quotation marks omitted).
%The Court consistently has upheld regulations imposed to arrest a significant
%threat to the common welfare, whatever their economic effect on the owner.\ldots
%
The Court creates its new takings jurisprudence based on the trial court's
finding that the property had lost all economic value. This finding is almost
certainly erroneous. Petitioner still can enjoy other attributes of ownership,
such as the right to exclude others, ``one of the most essential sticks in the
bundle of rights that are commonly characterized as property.''
Petitioner can picnic, swim,
camp in a tent, or live on the property in a movable trailer. State courts
frequently have recognized that land has economic value where the only residual
economic uses are recreation or camping. Petitioner also retains the right to
alienate the land, which would have value for neighbors and for those prepared
to enjoy proximity to the ocean without a house.\ldots

Clearly, the Court was eager to decide this case.\ldots

%The Court does not reject the South Carolina Supreme Court's decision simply on
%the basis of its disbelief and distrust of the legislature's findings. It also
%takes the opportunity to create a new scheme for regulations that eliminate all
%economic value. From now on, there is a categorical rule finding these
%regulations to be a taking unless the use they prohibit is a background
%common-law nuisance or property principle.
%
%I first question the Court's rationale in creating a category that obviates a
%``case-specific inquiry into the public interest advanced'' if all economic
%value has been lost. If one fact about the Court's takings jurisprudence can be
%stated without contradiction, it is that ``the particular circumstances of each
%case'' determine whether a specific restriction will be rendered invalid by the
%government's failure to pay compensation. \textit{United States v. Central
%Eureka Mining Co.}, 357 U.S. 155, 168 (1958). This is so because although we
%have articulated certain factors to be considered, including the economic impact
%on the property owner, the ultimate conclusion ``necessarily requires a weighing
%of private and public interests.'' \textit{Agins}, 447 U.S., at 261. When the
%government regulation prevents the owner from any economically valuable use of
%his property, the private interest is unquestionably substantial, but we have
%never before held that no public interest can outweigh it. Instead the Court's
%prior decisions ``uniformly reject the proposition that diminution in property
%value, standing alone, can establish a `taking.' `` \textit{Penn Central Transp.
%Co. v. New York City}, 438 U.S. 104, 131 (1978).\ldots
%
%The Court recognizes that ``our prior opinions have suggested that `harmful or
%noxious uses' of property may be proscribed by government regulation without the
%requirement of compensation,'' but seeks to reconcile them with its categorical
%rule by claiming that the Court never has upheld a regulation when the owner
%alleged the loss of all economic value. Even if the Court's factual premise were
%correct, its understanding of the Court's cases is distorted. In none of the
%cases did the Court suggest that the right of a State to prohibit certain
%activities without paying compensation turned on the availability of some
%residual valuable use. Instead, the cases depended on whether the government
%interest was sufficient to prohibit the activity, given the significant private
%cost.
%
%These cases rest on the principle that the State has full power to prohibit an
%owner's use of property if it is harmful to the public.\ldots
%
%Ultimately even the Court cannot embrace the full implications of its
%\textit{per se} rule: It eventually agrees that there cannot be a categorical
%rule for a taking based on economic value that wholly disregards the public need
%asserted. Instead, the Court decides that it will permit a State to regulate all
%economic value only if the State prohibits uses that would not be permitted
%under ``background principles of nuisance and property
%law.''\readingfootnote{15}{Although it refers to state nuisance and property
%law, the Court apparently does not mean just any state nuisance and property
%law. Public nuisance was first a common-law creation, see Newark, The Boundaries
%of Nuisance, 65 L.Q.Rev. 480, 482 (1949) (attributing development of nuisance to
%1535), but by the 1800's in both the United States and England, legislatures had
%the power to define what is a public nuisance, and particular uses often have
%been selectively targeted. See Prosser, Private Action for Public Nuisance, 52
%Va.L.Rev. 997, 999--1000 (1966); J. Stephen, A General View of the Criminal Law
%of England 105--107 (2d ed. 1890). The Court's references to ``common-law''
%background principles, however, indicate that legislative determinations do not
%constitute ``state nuisance and property law'' for the Court.}
%
%Until today, the Court explicitly had rejected the contention that the
%government's power to act without paying compensation turns on whether the
%prohibited activity is a common-law nuisance. The brewery closed in
%\textit{Mugler} itself was not a common-law nuisance, and the Court specifically
%stated that it was the role of the legislature to determine what measures would
%be appropriate for the protection of public health and safety.\ldots
%
%The Court rejects the notion that the State always can prohibit uses it deems a
%harm to the public without granting compensation because ``the distinction
%between `harm-preventing' and `benefit-conferring' regulation is often in the
%eye of the beholder.'' Since the characterization will depend ``primarily upon
%one's evaluation of the worth of competing uses of real estate,'' the Court
%decides a legislative judgment of this kind no longer can provide the desired
%``objective, value-free basis'' for upholding a regulation. The Court, however,
%fails to explain how its proposed common-law alternative escapes the same trap. 

The threshold inquiry for imposition of the Court's new rule, ``deprivation of
all economically valuable use,'' itself cannot be determined objectively. As the
Court admits, whether the owner has been deprived of all economic value of his
property will depend on how ``property'' is defined. The ``composition of the
denominator in our `deprivation' fraction'' is the dispositive inquiry. Yet
there is no ``objective'' way to define what that denominator should be.\ldots

%The Court's decision in \textit{Keystone Bituminous Coal} illustrates this
%principle perfectly. In \textit{Keystone}, the Court determined that the
%``support estate'' was ``merely a part of the entire bundle of rights possessed
%by the owner.'' 480 U.S., at 501. Thus, the Court concluded that the support
%estate's destruction merely eliminated one segment of the total property. The
%dissent, however, characterized the support estate as a distinct property
%interest that was wholly destroyed. The Court could agree on no ``value-free
%basis'' to resolve this dispute.

Even more perplexing, however, is the Court's reliance on common-law principles
of nuisance in its quest for a value-free takings jurisprudence. In determining
what is a nuisance at common law, state courts make exactly the decision that
the Court finds so troubling when made by the South Carolina General Assembly
today: They determine whether the use is harmful. Common-law public and private
nuisance law is simply a determination whether a particular use causes harm.
There is nothing magical in the reasoning of judges long dead. They determined a
harm in the same way as state judges and legislatures do today. If judges in the
18th and 19th centuries can distinguish a harm from a benefit, why not judges in
the 20th century, and if judges can, why not legislators? There simply is no
reason to believe that new interpretations of the hoary common-law nuisance
doctrine will be particularly ``objective'' or ``value free.'' Once one abandons
the level of generality of \textit{sic utere tuo ut alienum non laedas}, one
searches in vain, I think, for anything resembling a principle in the common law
of nuisance.

%Finally, the Court justifies its new rule that the legislature may not deprive a
%property owner of the only economically valuable use of his land, even if the
%legislature finds it to be a harmful use, because such action is not part of the
%long recognized understandings of our citizens. These
%``understandings'' permit such regulation only if the use is a nuisance under
%the common law. Any other course is ``inconsistent with the historical compact
%recorded in the Takings Clause.'' It is not clear from the Court's opinion where
%our ``historical compact'' or ``citizens' understanding'' comes from, but it
%does not appear to be history. [Justice Blackmun then reviewed historical
%sources to conclude:]
%
%The principle that the State should compensate individuals for property taken
%for public use was not widely established in America at the time of the
%Revolution.\ldots
%
%Even into the 19th century, state governments often felt free to take property
%for roads and other public projects without paying compensation to the
%owners.\ldots
%
%Nor does history indicate any common-law limit on the State's power to regulate
%harmful uses even to the point of destroying all economic value. Nothing in the
%discussions in Congress concerning the Takings Clause indicates that the Clause
%was limited by the common-law nuisance doctrine.\ldots
%
%In short,
%I find no clear and accepted ``historical compact'' or ``understanding
%of our citizens'' justifying the Court's new takings doctrine. Instead, the
%Court seems to treat history as a grab bag of principles, to be adopted where
%they support the Court's theory, and ignored where they do not.\ldots

%I dissent.

\opinion Justice \textsc{Stevens}, dissenting.

%\ldots In my opinion, the Court is doubly in error. The categorical rule the
%Court establishes is an unsound and unwise addition to the law and the Court's
%formulation of the exception to that rule is too rigid and too narrow.\ldots
%
%Although in dicta we have sometimes recited that a law ``effects a taking if
%[it]\ldots denies an owner economically viable use of his land,'' \textit{Agins
%v. City of Tiburon}, 447 U.S. 255, 260 (1980), our \textit{rulings} have
%rejected such an absolute position. We have frequently---and recently---held
%that, in some circumstances, a law that renders property valueless may
%nonetheless not constitute a taking.\ldots

\ldots
In addition to lacking support in past decisions, the Court's new rule is wholly
arbitrary. A landowner whose property is diminished in value 95\% recovers
nothing, while an owner whose property is diminished 100\% recovers the land's
full value. The case at hand illustrates this arbitrariness well. The Beachfront
Management Act not only prohibited the building of new dwellings in certain
areas, it also prohibited the rebuilding of houses that were ``destroyed beyond
repair by natural causes or by fire.'' 1988 S.C. Acts 634, \S~3. Thus, if the
homes adjacent to Lucas' lot were destroyed by a hurricane one day after the Act
took effect, the owners would not be able to rebuild, nor would they be assured
recovery. Under the Court's categorical approach, Lucas (who has lost the
opportunity to build) recovers, while his neighbors (who have lost \textit{both}
the opportunity to build \textit{and} their homes) do not recover. The
arbitrariness of such a rule is palpable.

Moreover, because of the elastic nature of property rights, the Court's new rule
will also prove unsound in practice. In response to the rule, courts may define
``property'' broadly and only rarely find regulations to effect total
takings.\ldots
%This is the approach the Court itself adopts in its revisionist reading of
%venerable precedents. We are told that---notwithstanding the Court's findings to
%the contrary in each case---the brewery in \textit{Mugler}, the brickyard in
%\textit{Hadacheck}, and the gravel pit in \textit{Goldblatt} all could be put to
%``other uses'' and that, therefore, those cases did not involve total regulatory
%takings.\readingfootnote{3}{Of course, the same could easily be said in this
%case: Lucas may put his land to ``other uses''---fishing or camping, for
%example---or may sell his land to his neighbors as a buffer. In either event,
%his land is far from ``valueless.''\par This highlights a fundamental weakness
%in the Court's analysis: its failure to explain why only the impairment of
%``\textit{economically} beneficial or productive use'' (emphasis added) of
%property is relevant in takings analysis. I should think that a regulation
%arbitrarily prohibiting an owner from continuing to use her property for bird
%watching or sunbathing might constitute a taking under some circumstances; and,
%conversely, that such uses are of value to the owner. Yet the Court offers no
%basis for its assumption that the only uses of property cognizable under the
%Constitution are developmental uses.}
%
On the other hand, developers and investors may market specialized estates to
take advantage of the Court's new rule. The smaller the estate, the more likely
that a regulatory change will effect a total taking.\ldots
% Thus, an investor may, for
%example, purchase the right to build a multifamily home on a specific lot, with
%the result that a zoning regulation that allows only single- family homes would
%render the investor's property interest ``valueless.'' In short, the categorical
%rule will likely have one of two effects: Either courts will alter the
%definition of the ``denominator'' in the takings ``fraction,'' rendering the
%Court's categorical rule meaningless, or investors will manipulate the relevant
%property interests, giving the Court's rule sweeping effect. To my mind, neither
%of these results is desirable or appropriate, and both are distortions of our
%takings jurisprudence.
%
%Finally, the Court's justification for its new categorical rule is remarkably
%thin. The Court mentions in passing three arguments in support of its rule; none
%is convincing. First, the Court suggests that ``total deprivation of feasible
%use is, from the landowner's point of view, the equivalent of a physical
%appropriation.'' This argument proves too much. From the ``landowner's point of
%view,'' a regulation that diminishes a lot's value by 50\% is as well ``the
%equivalent'' of the condemnation of half of the lot. Yet, it is well established
%that a 50\% diminution in value does not by itself constitute a taking. See
%\textit{Euclid v. Ambler Realty Co.}, 272 U.S. 365, 384 (1926) (75\% diminution
%in value). Thus, the landowner's perception of the regulation cannot justify the
%Court's new rule.
%
%Second, the Court emphasizes that because total takings are ``relatively rare''
%its new rule will not adversely affect the government's ability to ``go on.''
%This argument proves too little. Certainly it is true that defining a small
%class of regulations that are \textit{per se} takings will not greatly hinder
%important governmental functions---but this is true of \textit{any} small class
%of regulations. The Court's suggestion only begs the question of why regulations
%of \textit{this} particular class should always be found to effect takings.
%
%Finally, the Court suggests that ``regulations that leave the owner\ldots
%without economically beneficial\ldots use\ldots carry with them a heightened
%risk that private property is being pressed into some form of public
%service.''\ldots I agree that the risks of such singling out are of central
%concern in takings law. However, such risks do not justify a \textit{per se}
%rule for total regulatory takings. There is no necessary correlation between
%``singling out'' and total takings: A regulation may single out a property owner
%without depriving him of all of his property; and it may deprive him of all of
%his property without singling him out. What matters in such cases is not the
%degree of diminution of value, but rather the specificity of the expropriating
%act. For this reason, the Court's third justification for its new rule also
%fails.
%
%In short, the Court's new rule is unsupported by prior decisions, arbitrary and
%unsound in practice, and theoretically unjustified. In my opinion, a categorical
%rule as important as the one established by the Court today should be supported
%by more history or more reason than has yet been provided.

%\readinghead{The Nuisance Exception}
%
%Like many bright-line rules, the categorical rule established in this case is
%only ``categorical'' for a page or two in the U.S. Reports. No sooner does the
%Court state that ``total regulatory takings must be compensated,'' than it
%quickly establishes an exception to that rule.
%
%The exception provides that a regulation that renders property valueless is not
%a taking if it prohibits uses of property that were not ``previously permissible
%under relevant property and nuisance principles.'' The Court thus rejects the
%basic holding in \textit{Mugler v. Kansas}, 123 U.S. 623 (1887). There we held
%that a state-wide statute that prohibited the owner of a brewery from making
%alcoholic beverages did not effect a taking, even though the use of the property
%had been perfectly lawful and caused no public harm before the statute was
%enacted.\ldots
%
%Under our reasoning in \textit{Mugler}, a State's decision to prohibit or to
%regulate certain uses of property is not a compensable taking just because the
%particular uses were previously lawful. Under the Court's opinion today,
%however, if a State should decide to prohibit the manufacture of asbestos,
%cigarettes, or concealable firearms, for example, it must be prepared to pay for
%the adverse economic consequences of its decision. One must wonder if government
%will be able to ``go on'' effectively if it must risk compensation ``for every
%such change in the general law.'' \textit{Mahon}, 260 U.S., at 413.

[Regarding the nuisance exception:]
The Court's holding today effectively freezes the State's common law, denying
the legislature much of its traditional power to revise the law governing the
rights and uses of property. Until today, I had thought that we had long
abandoned this approach to constitutional law. More than a century ago we
recognized that ``the great office of statutes is to remedy defects in the
common law as they are developed, and to adapt it to the changes of time and
circumstances.'' \textit{Munn v. Illinois}, 94 U.S. 113 (1877).\ldots

Arresting the development of the common law is not only a departure from our
prior decisions; it is also profoundly unwise. The human condition is one of
constant learning and evolution---both moral and practical. Legislatures
implement that new learning; in doing so they must often revise the definition
of property and the rights of property owners. Thus, when the Nation came to
understand that slavery was morally wrong and mandated the emancipation of all
slaves, it, in effect, redefined ``property.'' On a lesser scale, our ongoing
self-education produces similar changes in the rights of property owners: New
appreciation of the significance of endangered species, the importance of
wetlands, and the vulnerability of coastal, shapes our evolving understandings
of property rights.

Of course, some legislative redefinitions of property will effect a taking and
must be compensated---but it certainly cannot be the case that every movement
away from common law does so. There is no reason, and less sense, in such an
absolute rule. We live in a world in which changes in the economy and the
environment occur with increasing frequency and importance.\ldots

%The Court's categorical approach rule will, I fear, greatly hamper the efforts
%of local officials and planners who must deal with increasingly complex problems
%in land-use and environmental regulation. As this case---in which the claims of
%an \textit{individual} property owner exceed \$1 million---well demonstrates,
%these officials face both substantial uncertainty because of the ad hoc nature
%of takings law and unacceptable penalties if they guess incorrectly about that
%law.\ldots

%In analyzing takings claims, courts have long recognized the difference between
%a regulation that targets one or two parcels of land and a regulation that
%enforces a statewide policy.\ldots
%
%In considering Lucas' claim, the generality of the Beachfront Management Act is
%significant. The Act does not target particular landowners, but rather regulates
%the use of the coastline of the entire State.\ldots Moreover, the Act did not
%single out owners of undeveloped land. The Act also prohibited owners of
%developed land from rebuilding if their structures were destroyed, and what is
%equally significant, from repairing erosion control devices, such as
%seawalls.\ldots In addition, in some situations, owners of developed land were
%required to ``renouris[h] the beach\ldots on a yearly basis with an amount\ldots
%of sand\ldots not\ldots less than one and one-half times the yearly volume of
%sand lost due to erosion.'' In short, the South Carolina Act imposed substantial
%burdens on owners of developed and undeveloped land alike. This generality
%indicates that the Act is not an effort to expropriate owners of undeveloped
%land.
%
%Admittedly, the economic impact of this regulation is dramatic and petitioner's
%investment-backed expectations are substantial. Yet, if anything, the costs to
%and expectations of the owners of developed land are even greater: I doubt,
%however, that the cost to owners of developed land of renourishing the beach and
%allowing their seawalls to deteriorate effects a taking. The costs imposed on
%the owners of undeveloped land, such as petitioner, differ from these costs only
%in degree, not in kind.\ldots
%
%In view of all of these factors, even assuming that petitioner's property was
%rendered valueless, the risk inherent in investments of the sort made by
%petitioner, the generality of the Act, and the compelling purpose motivating the
%South Carolina Legislature persuade me that the Act did not effect a taking of
%petitioner's property.\ldots

\opinion Statement of Justice \textsc{Souter}.

I would dismiss the writ of certiorari in this case as having been granted
improvidently. After briefing and argument it is abundantly clear that an
unreviewable assumption on which this case comes to us[, namely that Lucas had
been deprived of his entire economic interest,] is both questionable as a
conclusion of Fifth Amendment law and sufficient to frustrate the Court's
ability to render certain the legal premises on which its holding rests.\ldots

%The petition for review was granted on the assumption that the State by
%regulation had deprived the owner of his entire economic interest in the subject
%property. Such was the state trial court's conclusion, which the State Supreme
%Court did not review. It is apparent now that\ldots the trial court's conclusion
%is highly questionable. While the respondent now wishes to contest the point the
%Court is certainly right to refuse to take up the issue, which is not fairly
%included within the question presented, and has received only the most
%superficial and one-sided treatment before us.\ldots

