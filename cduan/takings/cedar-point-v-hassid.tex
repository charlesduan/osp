\reading{Cedar Point Nursery v. Hassid}

\readingcite{141 S. Ct. 2063 (2021)}

\opinion Chief Justice \textsc{Roberts} delivered the opinion of the Court.

A California regulation grants labor organizations a ``right to take access'' to
an agricultural employer's property in order to solicit support for
unionization. Agricultural employers must allow union organizers onto their
property for up to three hours per day, 120 days per year. The question
presented is whether the access regulation constitutes a \textit{per se}
physical taking under the Fifth and Fourteenth Amendments.


\readinghead{I}

[The California Agricultural Labor Relations Board enacted a regulation
requiring agricultural employers to permit union organizers onto their property
for up to 120 days a year. Per the regulation, a set number of organizers ``may
enter the employer's property for up to one hour before work, one hour during
the lunch break, and one hour after work. Organizers may not engage in
disruptive conduct, but are otherwise free to meet and talk with employees as
they wish.'']


%The California Agricultural Labor Relations Act of 1975 gives agricultural
%employees a right to self-organization and makes it an unfair labor practice for
%employers to interfere with that right. The state Agricultural Labor Relations
%Board has promulgated a regulation providing, in its current form, that the
%self-organization rights of employees include ``the right of access by union
%organizers to the premises of an agricultural employer for the purpose of
%meeting and talking with employees and soliciting their support.'' Under the
%regulation, a labor organization may ``take access'' to an agricultural
%employer's property for up to four 30-day periods in one calendar year. In order
%to take access, a labor organization must file a written notice with the Board
%and serve a copy on the employer. Two organizers per work crew (plus one
%additional organizer for every 15 workers over 30 workers in a crew) may enter
%the employer's property for up to one hour before work, one hour during the
%lunch break, and one hour after work. Organizers may not engage in disruptive
%conduct, but are otherwise free to meet and talk with employees as they wish.
%Interference with organizers' right of access may constitute an unfair labor
%practice, which can result in sanctions against the employer.
%
%
%Cedar Point Nursery is a strawberry grower in northern California. It employs
%over 400 seasonal workers and around 100 full-time workers, none of whom live on
%the property. According to the complaint, in October 2015, at five o'clock one
%morning, members of the United Farm Workers entered Cedar Point's property
%without prior notice. The organizers moved to the nursery's trim shed, where
%hundreds of workers were preparing strawberry plants. Calling through bullhorns,
%the organizers disturbed operations, causing some workers to join the organizers
%in a protest and others to leave the worksite altogether. Cedar Point filed a
%charge against the union for taking access without giving notice. The union
%responded with a charge of its own, alleging that Cedar Point had committed an
%unfair labor practice. [The case also involved the Fowler Packing Company, ``a
%Fresno-based grower and shipper of table grapes and citrus.'']
%

[Cedar Point Nursery is a strawberry grower, whose property was accessed by the
United Farm Workers union.]
Believing that the union would likely attempt to enter their property again in
the near future, the growers filed suit in Federal District Court against
several Board members in their official capacity. The growers argued that the
access regulation effected an unconstitutional \textit{per se} physical taking
under the Fifth and Fourteenth Amendments by appropriating without compensation
an easement for union organizers to enter their property.\ldots
% They requested
%declaratory and injunctive relief prohibiting the Board from enforcing the
%regulation against them.\ldots


%[The District Court ruled in favor of the Board, concluding that the regulation
%was not a \textit{per se} taking and that any takings claim would have to be
%made under the \textit{Penn Central} balancing test. A divided Court of Appeals
%panel affirmed. \textit{En banc} rehearing was denied over a dissent joined by
%seven other judges.]


\readinghead{II}


\readinghead{A}

\ldots When the government physically acquires private property for a public
use, the Takings Clause imposes a clear and categorical obligation to provide
the owner with just compensation.\ldots The government commits a physical taking
when it uses its power of eminent domain to formally condemn property. The same
is true when the government physically takes possession of property without
acquiring title to it.\ldots
% And the government likewise effects a physical taking
%when it occupies property---say, by recurring flooding as a result of building a
%dam. These sorts of physical appropriations constitute the ``clearest sort of
%taking,'' \textit{Palazzolo v. Rhode Island}, 533 U.S. 606, 617 (2001), and we
%assess them using a simple, \textit{per se} rule: The government must pay for
%what it takes.


When the government, rather than appropriating private property for itself or a
third party, instead imposes regulations that restrict an owner's ability to use
his own property, a different standard applies.\ldots
% Our jurisprudence governing such
%use restrictions has developed more recently. Before the 20th century, the
%Takings Clause was understood to be limited to physical appropriations of
%property. In \textit{Pennsylvania Coal Co. v. Mahon}, however, the Court
%established the proposition that ``while property may be regulated to a certain
%extent, if regulation goes too far it will be recognized as a taking.''\ldots
To
determine whether a use restriction effects a taking, this Court has generally
applied the flexible test developed in \textit{Penn Central}, balancing factors
such as the economic impact of the regulation, its interference with reasonable
investment-backed expectations, and the character of the government
action.\ldots


%Our cases have often described use restrictions that go ``too far'' as
%``regulatory takings.'' But that label can mislead. Government action that
%physically appropriates property is no less a physical taking because it arises
%from a regulation. That explains why we held that an administrative reserve
%requirement compelling raisin growers to physically set aside a percentage of
%their crop for the government constituted a physical rather than a regulatory
%taking. \textit{Horne}, 576 U.S. at 361. The essential question is not, as the
%Ninth Circuit seemed to think, whether the government action at issue comes
%garbed as a regulation (or statute, or ordinance, or miscellaneous decree). It
%is whether the government has physically taken property for itself or someone
%else---by whatever means---or has instead restricted a property owner's ability
%to use his own property. Whenever a regulation results in a physical
%appropriation of property, a \textit{per se} taking has occurred, and
%\textit{Penn Central} has no place.


\readinghead{B}

The access regulation appropriates a right to invade the growers' property and
therefore constitutes a \textit{per se} physical taking. The regulation grants
union organizers a right to physically enter and occupy the growers' land for
three hours per day, 120 days per year. Rather than restraining the growers' use
of their own property, the regulation appropriates for the enjoyment of third
parties the owners' right to exclude.


The right to exclude is ``one of the most treasured'' rights of property
ownership. \textit{Loretto v. Teleprompter Manhattan CATV Corp.}, 458 U.S. 419,
435 (1982). According to Blackstone, the very idea of property entails ``that
sole and despotic dominion which one man claims and exercises over the external
things of the world, in total exclusion of the right of any other individual in
the universe.'' 2 W. Blackstone, Commentaries on the Laws of England 2 (1766).
In less exuberant terms, we have stated that the right to exclude is
``universally held to be a fundamental element of the property right,'' and is
``one of the most essential sticks in the bundle of rights that are commonly
characterized as property.'' \textit{Kaiser Aetna v. United States}, 444 U.S.
164, 176, 179--180 (1979).


Given the central importance to property ownership of the right to exclude, it
comes as little surprise that the Court has long treated government-authorized
physical invasions as takings requiring just compensation. The Court has often
described the property interest taken as a servitude or an easement.

\ldots.

%For example, in \textit{United States v. Causby} we held that the
%invasion of private property by overflights effected a taking. 328 U.S. 256
%(1946). The government frequently flew military aircraft low over the Causby
%farm, grazing the treetops and terrorizing the poultry. The Court observed that
%ownership of the land extended to airspace that low, and that ``invasions of it
%are in the same category as invasions of the surface.'' Because the damages
%suffered by the Causbys ``were the product of a direct invasion of [their]
%domain,'' we held that ``a servitude has been imposed upon the land.'' 
%
%
%We similarly held that the appropriation of an easement effected a taking in
%\textit{Kaiser Aetna v. United States}. A real-estate developer dredged
%a pond, converted it into a marina, and connected it to a nearby bay and the
%ocean. The government asserted that the developer could not exclude the public
%from the marina because the pond had become a navigable water. We held that the
%right to exclude ``falls within [the] category of interests that the Government
%cannot take without compensation.''\ldots
%
In \textit{Loretto} v. \textit{Teleprompter Manhattan CATV Corp.}, we made clear
that a permanent physical occupation constitutes a \textit{per se} taking
regardless whether it results in only a trivial economic loss. New York adopted
a law requiring landlords to allow cable companies to install equipment on their
properties. Loretto alleged that the installation of a {\textonehalf}-inch
diameter cable and two 1{\textonehalf}-cubic-foot boxes on her roof caused a
taking. We agreed, stating that where government action results in a ``permanent
physical occupation of property, our cases uniformly have found a taking to the
extent of the occupation, without regard to whether the action achieves an
important public benefit or has only minimal economic impact on the owner.''


We reiterated that the appropriation of an easement constitutes a physical
taking in \textit{Nollan v.} \textit{California Coastal Commission}. The Nollans
sought a permit to build a larger home on their beachfront lot. The California
Coastal Commission issued the permit subject to the condition that the Nollans
grant the public an easement to pass through their property along the beach. As
a starting point to our analysis, we explained that, had the Commission simply
required the Nollans to grant the public an easement across their property, ``we
have no doubt there would have been a taking.'' 


More recently, in \textit{Horne v. Department of Agriculture}, we
observed that ``people still do not expect their property, real or personal, to
be actually occupied or taken away.'' The physical appropriation by the
government of the raisins in that case was a \textit{per se} taking, even if a
regulatory limit with the same economic impact would not have been. ``The
Constitution,'' we explained, ``is concerned with means as well as ends.'' 

The upshot of this line of precedent is that government-authorized invasions of
property---whether by plane, boat, cable, or beachcomber---are physical takings
requiring just compensation. As in those cases, the government here has
appropriated a right of access to the growers' property, allowing union
organizers to traverse it at will for three hours a day, 120 days a year. The
regulation appropriates a right to physically invade the growers' property---to
literally ``take access,'' as the regulation provides. It is therefore a
\textit{per se} physical taking under our precedents. Accordingly, the growers'
complaint states a claim for an uncompensated taking in violation of the Fifth
and Fourteenth Amendments.


\readinghead{C}

[The Court rejected arguments that the regulation was not a taking because the
unions only had access to agricultural property for a fraction of time.]

%\ldots The dissent\ldots concludes that the regulation cannot amount to a
%\textit{per se} taking because it allows ``access short of 365 days a year.''
%That position is insupportable as a matter of precedent and common sense. There
%is no reason the law should analyze an abrogation of the right to exclude in one
%manner if it extends for 365 days, but in an entirely different manner if it
%lasts for 364.
%
%
%To begin with, we have held that a physical appropriation is a taking whether it
%is permanent or temporary.\ldots The duration of an appropriation---just like
%the size of an appropriation---bears only on the amount of compensation.\ldots
%
%
%To be sure, \textit{Loretto} emphasized the heightened concerns associated with
%``[t]he permanence and absolute exclusivity of a physical occupation'' in
%contrast to ``temporary limitations on the right to exclude,'' and stated that
%``[n]ot every physical \textit{invasion} is a taking.'' The latter point is well
%taken, as we will explain. But \textit{Nollan} clarified that appropriation of a
%right to physically invade property may constitute a taking ``even though no
%particular individual is permitted to station himself permanently upon the
%premises.'' 
%
%
%Next, we have recognized that physical invasions constitute takings even if they
%are intermittent as opposed to continuous. \textit{Causby} held that overflights
%of private property effected a taking, even though they occurred on only 4\% of
%takeoffs and 7\% of landings at the nearby airport. And while \textit{Nollan}
%happened to involve a legally continuous right of access, we have no doubt that
%the Court would have reached the same conclusion if the easement demanded by the
%Commission had lasted for only 364 days per year.~.~.~. [W]hen the government
%physically takes an interest in property, it must pay for the right to do so.
%The fact that a right to take access is exercised only from time to time does
%not make it any less a physical taking.
%
%
%\ldots [The Board contends that the access regulation] fails to qualify as a
%\textit{per se} taking because it ``authorizes only limited and intermittent
%access for a narrow purpose.'' That position is little more defensible than the
%Ninth Circuit's. The fact that the regulation grants access only to union
%organizers and only for a limited time does not transform it from a physical
%taking into a use restriction. Saying that appropriation of a three hour per
%day, 120 day per year right to invade the growers' premises ``does not
%constitute a taking of a property interest but rather\ldots a mere restriction
%on its use, is to use words in a manner that deprives them of all their ordinary
%meaning.'' \textit{Nollan}, 483 U.S. at 831 (citation and internal quotation
%marks omitted).
%
%
%The Board also takes issue with the growers' premise that the access regulation
%appropriates an easement. In the Board's estimation, the regulation does not
%exact a true easement in gross under California law because the access right may
%not be transferred, does not burden any particular parcel of property, and may
%not be recorded. This, the Board says, reinforces its conclusion that the
%regulation does not take a constitutionally protected property interest from the
%growers. The dissent agrees, suggesting that the access right cannot effect a
%\textit{per se} taking because it does not require the growers to grant the
%union organizers an easement as defined by state property law. 
%
%
%These arguments misconstrue our physical takings doctrine. As a general matter,
%it is true that the property rights protected by the Takings Clause are
%creatures of state law. But no one disputes that, without the access regulation,
%the growers would have had the right under California law to exclude union
%organizers from their property. And no one disputes that the access regulation
%took that right from them. The Board cannot absolve itself of takings liability
%by appropriating the growers' right to exclude in a form that is a slight
%mismatch from state easement law.\ldots
%
%
%The Board and the dissent further contend that our decision in \textit{PruneYard
%Shopping Center v. Robins}, 447 U.S. 74 (1980), establishes that the access
%regulation cannot qualify as a \textit{per se} taking. There the California
%Supreme Court held that the State Constitution protected the right to engage in
%leafleting at the PruneYard, a privately owned shopping center. The shopping
%center argued that the decision had taken without just compensation its right to
%exclude. Applying the \textit{Penn Central} factors, we held that no compensable
%taking had occurred. cf. \textit{Heart of Atlanta Motel, Inc. v. United States},
%379 U.S. 241, 261 (1964) (rejecting claim that provisions of the Civil Rights
%Act of 1964 prohibiting racial discrimination in public accommodations effected
%a taking).
%
%
%The Board and the dissent argue that \textit{PruneYard} shows that limited
%rights of access to private property should be evaluated as regulatory rather
%than \textit{per se} takings. We disagree. Unlike the growers' properties, the
%PruneYard was open to the public, welcoming some 25,000 patrons a day.
%Limitations on how a business generally open to the public may treat individuals
%on the premises are readily distinguishable from regulations granting a right to
%invade property closed to the public. See \textit{Horne}, 576 U.S. at 364
%(distinguishing \textit{PruneYard} as involving ``an already publicly
%accessible'' business).
%
%
%The Board also relies on our decision in \textit{NLRB v. Babcock \& Wilcox Co}.
%But that reliance is misplaced. In \textit{Babcock}, the National Labor
%Relations Board found that several employers had committed unfair labor
%practices under the National Labor Relations Act by preventing union organizers
%from distributing literature on company property. We held that the statute did
%not require employers to allow organizers onto their property, at least outside
%the unusual circumstance where their employees were otherwise ``beyond the reach
%of reasonable union efforts to communicate with them.'' The Board contends that
%\textit{Babcock}'s approach of balancing property and organizational rights
%should guide our analysis here. But \textit{Babcock} did not involve a takings
%claim. Whatever specific takings issues may be presented by the highly
%contingent access right we recognized under the NLRA, California's access
%regulation effects a \textit{per se} physical taking under our precedents. 


\readinghead{D}

In its thoughtful opinion, the dissent advances a distinctive view of property
rights. The dissent encourages readers to consider the issue ``through the lens
of ordinary English,'' and contends that, so viewed, the ``regulation does not
\textit{appropriate} anything.'' Rather, the access regulation merely
``\textit{regulates}\ldots the owners' right to exclude,'' so it must be
assessed ``under \textit{Penn Central}'s fact-intensive test.''\ldots
%``A right to
%enter my woods only on certain occasions,'' the dissent elaborates, ``is a
%taking only if the regulation allowing it goes `too far.'\,''.\ldots According
%to the dissent, this kind of latitude toward temporary invasions is a practical
%necessity for governing in our complex modern world.


With respect, our own understanding of the role of property rights in our
constitutional order is markedly different. In ``ordinary English''
``appropriation'' means ``\textit{taking} as one's own,'' 1 Oxford English
Dictionary 587 (2d ed. 1989) (emphasis added), and the regulation expressly
grants to labor organizers the ``right to \textit{take} access,'' Cal. Code
Regs., tit. 8, \S~20900(e)(1)(C) (emphasis added). We cannot agree that the
right to exclude is an empty formality, subject to modification at the
government's pleasure. On the contrary, it is a ``fundamental element of the
property right,'' \textit{Kaiser Aetna}, 444 U.S. at 179--180, that cannot be
balanced away.
%Our cases establish that appropriations of a right to invade are
%\textit{per se} physical takings, not use restrictions subject to \textit{Penn
%Central}: ``[W]hen [government] planes use private airspace to approach a
%government airport, [the government] is required to pay for that share no matter
%how small.'' \textit{Tahoe-Sierra}, 535 U.S. at 322 (citing \textit{Causby}).
%And while \textit{Kaiser Aetna} may have referred to the test from \textit{Penn
%Central}, the Court concluded categorically that the government must pay just
%compensation for physical invasions. With regard to the complexities of modern
%society, we think they only reinforce the importance of safeguarding the basic
%property rights that help preserve individual liberty, as the Founders
%explained.\ldots


\readinghead{III}

The Board, seconded by the dissent, warns that treating the access regulation as
a \textit{per se} physical taking will endanger a host of state and federal
government activities involving entry onto private property. That fear is
unfounded.


\textit{First}, our holding does nothing to efface the distinction between
trespass and takings. Isolated physical invasions, not undertaken pursuant to a
granted right of access, are properly assessed as individual torts rather than
appropriations of a property right. This basic distinction is firmly grounded in
our precedent. See \textit{Portsmouth}, 260 U.S. at 329--330 (``[W]hile a single
act may not be enough, a continuance of them in sufficient number and for a
sufficient time may prove [the intent to take property]. Every successive
trespass adds to the force of the evidence.'')\ldots


%The distinction between trespass and takings accounts for our treatment of
%temporary government-induced flooding in \textit{Arkansas Game and Fish
%Commission v. United States}, 568 U.S. 23 (2012). There we held, ``simply and
%only,'' that such flooding ``gains no automatic exemption from Takings Clause
%inspection.'' Because this type of flooding can present complex questions of
%causation, we instructed lower courts evaluating takings claims based on
%temporary flooding to consider a range of factors including the duration of the
%invasion, the degree to which it was intended or foreseeable, and the character
%of the land at issue. Applying those factors on remand, the Federal Circuit
%concluded that the government had effected a taking in the form of a temporary
%flowage easement.\ldots


\textit{Second}, many government-authorized physical invasions will not amount
to takings because they are consistent with longstanding background restrictions
on property rights. As we explained in \textit{Lucas} v. \textit{South Carolina
Coastal Council}, the government does not take a property interest when it
merely asserts a ``pre-existing limitation upon the land owner's title.'' For
example, the government owes a landowner no compensation for requiring him to
abate a nuisance on his property, because he never had a right to engage in the
nuisance in the first place. 


These background limitations also encompass traditional common law privileges to
access private property[, such as necessity, to effect an arrest, or to conduct
a Fourth Amendment search].
%One such privilege allowed individuals to enter
%property in the event of public or private necessity. See Restatement (Second)
%of Torts \S~196 (1964) (entry to avert an imminent public disaster); \S~197
%(entry to avert serious harm to a person, land, or chattels); cf.
%\textit{Lucas}, 505 U.S. at 1029, n. 16. The common law also recognized a
%privilege to enter property to effect an arrest or enforce the criminal law
%under certain circumstances. Restatement (Second) of Torts {\S}\S~204--205.
%Because a property owner traditionally had no right to exclude an official
%engaged in a reasonable search, government searches that are consistent with the
%Fourth Amendment and state law cannot be said to take any property right from
%landowners. See generally \textit{Camara v. Municipal Court of City and County
%of San Francisco}, 387 U.S. 523, 538 (1967).


\textit{Third}, the government may require property owners to cede a right of
access as a condition of receiving certain benefits, without causing a taking.
%In \textit{Nollan}, we held that ``a permit condition that serves the same
%legitimate police-power purpose as a refusal to issue the permit should not be
%found to be a taking if the refusal to issue the permit would not constitute a
%taking.'' The inquiry, we later explained, is whether the permit condition bears
%an ``essential nexus'' and ``rough proportionality'' to the impact of the
%proposed use of the property. 
%
%
Under this framework, government health and safety inspection regimes will
generally not constitute takings. When the government conditions the grant of a
benefit such as a permit, license, or registration on allowing access for
reasonable health and safety inspections, both the nexus and rough
proportionality requirements of the constitutional conditions framework should
not be difficult to satisfy.
%See, \textit{e.g.}, 7 U.S.C. \S~136g(a)(1)(A)
%(pesticide inspections); 16 U.S.C. \S~823b(a) (hydroelectric project
%investigations); 21 U.S.C. \S~374(a)(1) (pharmaceutical inspections); 42 U.S.C.
%\S~2201(\textit{o}) (nuclear material inspections).


None of these considerations undermine our determination that the access
regulation here gives rise to a \textit{per se} physical taking. Unlike a mere
trespass, the regulation grants a formal entitlement to physically invade the
growers' land. Unlike a law enforcement search, no traditional background
principle of property law requires the growers to admit union organizers onto
their premises. And unlike standard health and safety inspections, the access
regulation is not germane to any benefit provided to agricultural employers or
any risk posed to the public.\ldots
% See \textit{Horne}, 576 U.S. at 366 (``basic and
%familiar uses of property'' are not a special benefit that ``the Government may
%hold hostage, to be ransomed by the waiver of constitutional protection''). The
%access regulation amounts to simple appropriation of private property.\ldots

\opinion Justice \textsc{Kavanaugh}, concurring.

I join the Court's opinion, which carefully adheres to constitutional text,
history, and precedent. I write separately to explain that, in my view, the
Court's precedent in \textit{NLRB v. Babcock \& Wilcox Co.}, 351 U.S. 105
(1956), also strongly supports today's decision.


In \textit{Babcock}, the National Labor Relations Board argued that the National
Labor Relations Act afforded union organizers a right to enter company property
to communicate with employees.\ldots
%Several employers responded that the Board's
%reading of the Act would infringe their Fifth Amendment property rights. The
%employers contended that Congress, ``even if it could constitutionally do so,
%has at no time shown any intention of destroying property rights secured by the
%\textit{Fifth Amendment}, in protecting employees' rights of collective
%bargaining under the Act. Until Congress should evidence such intention by
%specific legislative language, our courts should not construe the Act on such
%dangerous constitutional grounds.'' Brief for Respondent in \textit{NLRB} v.
%\textit{Babcock \& Wilcox Co.}, O. T. 1955, No. 250, pp. 18--19.
%
%
%This Court agreed with the employers' argument that the Act should be
%interpreted to avoid unconstitutionality. The Court reasoned that ``the National
%Government'' via the Constitution ``preserves property rights,'' including ``the
%right to exclude from property.''
Against the backdrop of the Constitution's
strong protection of property rights, the Court interpreted the Act to afford
access to union organizers only when ``needed''---that is, when the employees
live on company property and union organizers have no other reasonable means of
communicating with the employees, As I read it, \textit{Babcock} recognized that
employers have a basic Fifth Amendment right to exclude from their private
property, subject to a ``necessity'' exception similar to that noted by the
Court today. 

%\textit{Babcock} strongly supports the growers' position in today's case
%because the California union access regulation intrudes on the growers'
%property rights far more than \textit{Babcock} allows.\ldots

\opinion Justice \textsc{Breyer}, with whom Justice \textsc{Sotomayor} and
Justice \textsc{Kagan} join, dissenting.

%A California regulation provides that representatives of a labor organization
%may enter an agricultural employer's property for purposes of union organizing.
%They may do so during four months of the year, one hour before the start of
%work, one hour during an employee lunch break, and one hour after work. The
%question before us is how to characterize this regulation for purposes of the
%Constitution's Takings Clause.
%
%
%Does the regulation \textit{physically appropriate} the employers' property? If
%so, there is no need to look further; the Government must pay the employers
%``just compensation.'' U. S. Const., Amdt. 5. Or does the regulation simply
%\textit{regulate} the employers' property rights? If so, then there is every
%need to look further; the government need pay the employers ``just
%compensation'' only if the regulation ``goes too far.'' \textit{Pennsylvania
%Coal Co. v. Mahon}, 260 U.S. 393, 415 (1922) (\textsc{Holmes}, J., for the
%Court); see also \textit{Penn Central Transp. Co. v. New York City}, 438 U.S.
%104, 124 (1978) (determining whether a regulation is a taking by examining the
%regulation's ``economic impact,'' the extent of interference with
%``investment-backed expectations,'' and the ``character of the governmental
%action'').


The Court holds that the provision's ``access to organizers'' requirement
amounts to a physical appropriation of property.\ldots
%In its view, virtually every
%government-authorized invasion is an ``appropriation.''
But this regulation does
not ``appropriate'' anything; it regulates the employers' right to exclude
others. At the same time, our prior cases make clear that the regulation before
us allows only a \textit{temporary} invasion of a landowner's property and that
this kind of temporary invasion amounts to a taking only if it goes ``too far.''
%See, \textit{e.g.}, \textit{Loretto v. Teleprompter Manhattan CATV Corp.}, 458
%U.S. 419, 434 (1982).
In my view, the majority's conclusion threatens to make
many ordinary forms of regulation unusually complex or impractical.\ldots
%And though
%the majority attempts to create exceptions to narrow its rule, the law's need
%for feasibility suggests that the majority's framework is wrong. With respect, I
%dissent from the majority's conclusion that the regulation is a \textit{per se}
%taking.


%\readinghead{I}
%
%``In view of the nearly infinite variety of ways in which government actions or
%regulations can affect property interests, the Court has recognized few
%invariable rules in this area.'' \textit{Arkansas Game and Fish Comm'n}, 568
%U.S. at 31. Instead, most government action affecting property rights is
%analyzed case by case under \textit{Penn Central}'s fact-intensive test.
%Petitioners do not argue that the provision at issue is a ``regulatory taking''
%under that test.
%
%
%Instead, the question before us is whether the access regulation falls within
%one of two narrow categories of government conduct that are \textit{per se}
%takings. The first is when ``\,`the government directly appropriates private
%property for its own use.'\,'' \textit{Horne v. Department of
%Agriculture}, 576 U. S. 350, 357 (2015). The second is when the government
%causes a permanent physical occupation of private property. It does not.

[Justice Breyer's dissent is wide-ranging, and only a selection of its arguments
are presented here.]

%\readinghead{A}
%
%Initially it may help to look at the legal problem---a problem of
%characterization---through the lens of ordinary English. The word ``regulation''
%rather than ``appropriation'' fits this provision in both label and substance.
%It is contained in Title 8 of the California Code of Regulations. It was adopted
%by a state regulatory board, namely, the California Agricultural Labor Relations
%Board, in 1975. It is embedded in a set of related detailed regulations that
%describe and limit the access at issue. In addition to the hours of access just
%mentioned, it provides that union representatives can enter the property only
%``for the purpose of meeting and talking with employees and soliciting their
%support''; they have access only to ``areas in which employees congregate before
%and after working'' or ``at such location or locations as the employees eat
%their lunch''; and they cannot engage in ``conduct disruptive of the employer's
%property or agricultural operations, including injury to crops or machinery or
%interference with the process of boarding buses.'' \S\S~20900(e), (e)(3),
%(e)(4)(C) (2021). From the employers' perspective, it restricts when and where
%they can exclude others from their property.
%
%
%At the same time, the provision only awkwardly fits the terms ``physical
%taking'' and ``physical appropriation.'' The ``access'' that it grants union
%organizers does not amount to any traditional property interest in land. It does
%not, for example, take from the employers, or provide to the organizers, any
%freehold estate (\textit{e.g.}, a fee simple, fee tail, or life estate); any
%concurrent estate (\textit{e.g.}, a joint tenancy, tenancy in common, or tenancy
%by the entirety); or any leasehold estate (\textit{e.g.}, a term of years,
%periodic tenancy, or tenancy at will). Nor (as all now agree) does it provide
%the organizers with a formal easement or access resembling an easement, as the
%employers once argued, since it does not burden any particular parcel of
%property. 
%
%
%The majority concludes that the regulation nonetheless amounts to a physical
%taking of property because, the majority says, it ``appropriates'' a ``right to
%invade'' or a ``right to exclude'' others. It thereby likens this case to cases
%in which we have held that appropriation of property rights amounts to a
%physical \textit{per se} taking. 
%

It is important to understand, however, that, technically speaking, the majority
is wrong. The regulation does not \textit{appropriate} anything. It does not
take from the owners a right to invade (whatever that might mean). It does not
give the union organizations the right to exclude anyone. It does not give the
government the right to exclude anyone. What does it do? It gives union
organizers the right temporarily to invade a portion of the property owners'
land. It thereby limits the landowners' right to exclude certain others. The
regulation \textit{regulates} (but does not \textit{appropriate}) the owners'
right to exclude.


Why is it important to understand this technical point? Because only then can we
understand the issue before us. That issue is whether a regulation that
\textit{temporarily} limits an owner's right to exclude others from property
\textit{automatically} amounts to a Fifth Amendment taking. Under our cases, it
does not.



%\readinghead{B}
%
%Our cases draw a distinction between regulations that provide permanent rights
%of access and regulations that provide nonpermanent rights of access. They
%either state or hold that the first type of regulation is a taking \textit{per
%se}, but the second kind is a taking only if it goes ``too far.'' And they make
%this distinction for good reason.
%
%
%Consider the Court's reasoning in an important case in which the Court found a
%\textit{per se} taking. In \textit{Loretto}, the Court considered the status of
%a New York law that required landlords to permit cable television companies to
%install cable facilities on their property. We held that the installation
%amounted to a permanent physical occupation of the property and hence to a
%\textit{per se} taking. In reaching this holding we specifically said that
%``[n]ot every physical invasion is a taking.'' We explained that the
%``permanence and absolute exclusivity of a physical occupation distinguish it
%from temporary limitations on the right to exclude.'' And we provided an example
%of a federal statute that did \textit{not} effect a \textit{per se} taking---an
%example almost identical to the regulation before us. That statute provided
%``\,`access\ldots limited to (i) union organizers; (ii) prescribed non-working
%areas of the employer's premises; and (iii) the duration of the organization
%activity.'\,'' (quoting \textit{Central Hardware Co. v. NLRB}, 407 U.S. 539, 545
%(1972)).
%
%
%We also explained why permanent physical occupations are distinct from temporary
%limitations on the right to exclude. We said that, when the government
%permanently occupies property, it ``does not simply take a single `strand' from
%the `bundle' of property rights: it chops through the bundle, taking a slice of
%every strand,'' ``effectively destroy[ing]'' ``the rights `to possess, use and
%dispose of it.'\,'' \textit{Loretto}, 458 U.S. at 435.\ldots Thus, we concluded,
%a permanent physical occupation ``is perhaps the most serious form of invasion
%of an owner's property interests.'' 
%
%
%Now consider \textit{PruneYard Shopping Center v. Robins}, 447 U.S. 74 (1980).
%We there considered the status of a state constitutional requirement that a
%privately owned shopping center permit other individuals to enter upon, and to
%use, the property to exercise their rights to free speech and petition. We held
%that this requirement was not a \textit{per se} taking in part because (even
%though the individuals may have ``\,`physically invaded'\,'' the owner's
%property) ``[t]here [wa]s nothing to suggest that preventing [the owner] from
%prohibiting this sort of activity w[ould] unreasonably impair the value or use
%of th[e] property as a shopping center,'' and the owner could ``adop[t] time,
%place, and manner regulations that w[ould] minimize any interference with its
%commercial functions.'' 
%
%
%In \textit{Nollan v. California Coastal Comm'n}, 483 U.S. 825 (1987), we held
%that the State's taking of an easement across a landowner's property did
%constitute a \textit{per se} taking. But consider the Court's reason:
%``[I]ndividuals are given a \textit{permanent and continuous} right to pass to
%and fro.'' (emphasis added). We clarified that by ``permanent'' and
%``continuous'' we meant that the ``real property may continuously be traversed,
%even though no particular individual is permitted to station himself permanently
%upon the premises.''
%
%
%In \textit{Arkansas Game and Fish Comm'n}, 568 U.S. 23 we again said that
%permanent physical occupations are \textit{per se} takings, but temporary
%invasions are not. Rather, they ``are subject to a more complex balancing
%process to determine whether they are a taking.'' 
%
%
%As these cases have used the terms, the regulation here at issue provides access
%that is ``temporary,'' not ``permanent.'' Unlike the regulation in
%\textit{Loretto}, it does not place a ``fixed structure on land or real
%property.'' The employers are not ``forever denie[d]'' ``any power to control
%the use'' of any particular portion of their property. And it does not totally
%reduce the value of any section of the property. Unlike in \textit{Nollan}, the
%public cannot walk over the land whenever it wishes; rather a subset of the
%public may enter a portion of the land three hours per day for four months per
%year (about 4\% of the time). At bottom, the regulation here, unlike the
%regulations in \textit{Loretto} and \textit{Nollan}, is not ``functionally
%equivalent to the classic taking in which government directly appropriates
%private property or ousts the owner from his domain.'' \textit{Lingle}, 544 U.S.
%at 539.
%
%
%At the same time, \textit{PruneYard}'s holding that the taking was ``temporary''
%(and hence not a \textit{per se} taking) fits this case almost perfectly. There
%the regulation gave nonowners the right to enter privately owned property for
%the purpose of speaking generally to others, about matters of their choice,
%subject to reasonable time, place, and manner restrictions. The regulation
%before us grants a far smaller group of people the right to enter landowners'
%property for far more limited times in order to speak about a specific subject.
%Employers have more power to control entry by setting work hours, lunch hours,
%and places of gathering. On the other hand, as the majority notes, the shopping
%center in \textit{PruneYard} was open to the public generally. All these
%factors, however, are the stuff of which regulatory-balancing, not absolute
%\textit{per se}, rules are made.\ldots
%
%
%The majority refers to other cases. But those cases do not help its cause. That
%is because the Court in those cases\ldots did not apply a ``\textit{per se}
%takings'' approach. In \textit{United States v. Causby}, 328 U.S. 256, 259
%(1946), for example, the question was whether government flights over a piece of
%land constituted a taking. The flights amounted to 4\% of the takeoffs, and 7\%
%of the landings, at a nearby airport. But the planes flew ``in considerable
%numbers and rather close together.'' And the flights were ``so low and so
%frequent as to be a direct and immediate interference with the enjoyment and use
%of the land.'' Taken together, those flights ``destr[oyed] the use of the
%property as a commercial chicken farm.'' Based in part on that economic damage,
%the Court found that the rule allowing these overflights went ``too far.'' See
%\textit{id.}, at 266 (``\,`[I]t is the character of the invasion, not the amount
%of damage resulting from it, \textit{so long as the damage is substantial}, that
%determines the question whether it is a taking'\,'' (emphasis added)).\ldots
%
%
%If there is ambiguity in these cases, it concerns whether the Court considered
%the occupation at issue to be \textit{temporary} (requiring \textit{Penn
%Central}'s ``too far'' analysis) or \textit{permanent} (automatically requiring
%compensation). Nothing in them suggests the majority's view, namely, that
%compensation is automatically required for a \textit{temporary} right of access.
%Nor does anything in them support the distinction that the majority gleans
%between ``trespass'' and ``takings.'' 
%
%
%The majority also refers to \textit{Nollan} as support for its claim that the
%``fact that a right to take access is exercised only from time to time does not
%make it any less a physical taking.'' True. Here, however, unlike in
%\textit{Nollan}, the right taken is not a right to have access to the property
%at any time (which access different persons ``exercis[e]\ldots from time to
%time''). Rather here we have a right that does not allow access at any time. It
%allows access only from ``time to time.'' And that makes all the difference. A
%right to enter my woods whenever you wish is a right to use that property
%permanently, even if you exercise that right only on occasion. A right to enter
%my woods only on certain occasions is not a right to use the woods permanently.
%In the first case one might reasonably use the term \textit{per se} taking. It
%is as if my woods are yours. In the second case it is a taking only if the
%regulation allowing it goes ``too far,'' considering the factors we have laid
%out in \textit{Penn Central}. That is what our cases say.
%
%
%Finally, the majority says that \textit{Nollan} would have come out the same way
%had it involved, similar to the regulation here, access short of 365 days a
%year. Perhaps so. But, if so, that likely would be because the Court would have
%viewed the access as an ``easement,'' and therefore an appropriation. Or,
%perhaps, the Court would have viewed the regulation as going ``too far.'' I can
%assume, purely for argument's sake, that that is so. But the law is clear: A
%regulation that provides \textit{temporary}, not \textit{permanent}, access to a
%landowner's property, and that does not amount to a taking of a traditional
%property interest, is not a \textit{per se} taking. That is, it does not
%automatically require compensation. Rather, a court must consider whether it
%goes ``too far.''
%
%
%\readinghead{C}
%
%The persistence of the permanent/temporary distinction that I have described is
%not surprising.
[Justice Breyer reviewed takings precedents, concluding that they supported
distinguishing permanent and temporary rights of access.]
That distinction serves an important purpose. We live together
in communities. (Approximately 80\% of Americans live in urban areas.) Modern
life in these communities requires different kinds of regulation. Some, perhaps
many, forms of regulation require access to private property (for government
officials or others) for different reasons and for varying periods of time. Most
such temporary-entry regulations do not go ``too far.'' And it is impractical to
compensate every property owner for any brief use of their land. As we have
frequently said, ``[g]overnment hardly could go on if to some extent values
incident to property could not be diminished without paying for every such
change in the general law.'' \textit{Pennsylvania Coal Co.}, 260 U.S. at 413.
Thus, the law has not, and should not, convert all temporary-access-permitting
regulations into \textit{per se} takings automatically requiring compensation. 


Consider the large numbers of ordinary regulations in a host of different fields
that, for a variety of purposes, permit temporary entry onto (or an ``invasion
of'') a property owner's land. They include activities ranging from examination
of food products to inspections for compliance with preschool licensing
requirements.
%See, \textit{e.g.}, 29 U.S.C. \S~657(a) (authorizing inspections
%and investigations of ``any\ldots workplace or environment where work is
%performed'' during ``regular working hours and at other reasonable times''); 21
%U.S.C. \S~606(a) (authorizing ``examination and inspection of all meat food
%products\ldots at all times, by day or night''); 42 U.S.C. \S~5413(b)
%(authorizing inspections anywhere ``manufactured homes are manufactured, stored,
%or held for sale'' at ``reasonable times and without advance notice''); Miss.
%Code Ann. \S~49--27--63 (2012) (authorizing inspections of ``coastal wetlands''
%``from time to time''); Mich. Comp. Laws \S~208.1435(5) (2010) (authorizing
%inspections of any ``historic resource'' ``at any time during the rehabilitation
%process''); Mont. Code Ann. \S~81--22--304 (2019) (granting a ``right of
%entry\ldots [into] any premises where dairy products\ldots are produced,
%manufactured, [or] sold'' ``during normal business hours''); Neb. Rev. Stat.
%\S~43--1303(5) (2016) (authorizing visitation of ``foster care facilities in
%order to ascertain whether the individual physical, psychological, and
%sociological needs of each foster child are being met''); Va. Code Ann.
%\S~22.1--289.032(C)(8) (Cum. Supp. 2020) (authorizing ``annual inspection'' of
%``preschool programs of accredited private schools''); Cincinnati, Ohio,
%Municipal Code \S~603--1 (2021) (authorizing entry ``at any time'' for any place
%in which ``animals are slaughtered''); Dallas, Tex., Code of Ordinance
%\S~33--5(a) (2021) (authorizing inspection of ``assisted living facilit[ies]''
%``at reasonable times''); 6 N. Y. Rules \& Regs. \S~360.7 (Supp. 2020)
%(authorizing inspection of solid waste management facilities ``at all reasonable
%times, locations, whether announced or unannounced''); see also \textit{Boise
%Cascade Corp. v. United States}, 296 F.3d 1339, 1352 (C.A. Fed. 2002) (affirming
%an injunction requiring property owner to allow Government agents to enter its
%property to conduct owl surveys).

%The majority tries to deal with the adverse impact of treating these, and other,
%temporary invasions as if they were \textit{per se} physical takings by creating
%a series of exceptions from its \textit{per se} rule. It says: (1) ``Isolated
%physical invasions, not undertaken pursuant to a granted right of access, are
%properly assessed as individual torts rather than appropriations of a property
%right.'' It also would except from its \textit{per se} rule (2) government
%access that is ``consistent with longstanding background restrictions on
%property rights,'' including ``traditional common law privileges to access
%private property.'' And it adds that (3) ``the government may require property
%owners to cede a right of access as a condition of receiving certain benefits,
%without causing a taking.'' How well will this new system work? I suspect that
%the majority has substituted a new, complex legal scheme for a comparatively
%simpler old one.
%
%
%As to the first exception, what will count as ``isolated''? How is an ``isolated
%physical invasion'' different from a ``temporary'' invasion, sufficient under
%present law to invoke \textit{Penn Central}? And where should one draw the line
%between trespass and takings? Imagine a school bus that stops to allow public
%school children to picnic on private land. Do three stops a year place the stops
%outside the exception? One stop every week? Buses from one school? From every
%school? Under current law a court would know what question to ask. The stops are
%temporary; no one assumes a permanent right to stop; thus the court will ask
%whether the school district has gone ``too far.'' Under the majority's approach,
%the court must answer a new question (apparently about what counts as
%``isolated'').
%
%
%As to the second exception, a court must focus on ``traditional common law
%privileges to access private property.'' Just what are they? We have said before
%that the government can, without paying compensation, impose a limitation on
%land that ``inhere[s] in the title itself, in the restrictions that background
%principles of the State's law of property and nuisance already place upon land
%ownership.'' \textit{Lucas}, 505 U.S. at 1029. But we defined a very narrow set
%of such background principles. See \textit{ibid.}, and n. 16 (abatement of
%nuisances and cases of ``\,`actual necessity'\,'' or ``to forestall other grave
%threats to the lives and property of others''). To these the majority adds
%``public or private necessity,'' the enforcement of criminal law ``under certain
%circumstances,'' and reasonable searches. Do only those exceptions that existed
%in, say, 1789 count? Should courts apply those privileges as they existed at
%that time, when there were no union organizers? Or do we bring some exceptions
%(but not others) up to date, \textit{e.g.}, a necessity exception for preserving
%animal habitats?
%
%
%As to the third, what is the scope of the phrase ``certain benefits''? Does it
%include the benefit of being able to sell meat labeled ``inspected'' in
%interstate commerce? But see \textit{Horne}, 576 U.S. at 366 (concluding that
%``[s]elling produce in interstate commerce'' is ``not a special governmental
%benefit''). What about the benefit of having electricity? Of sewage collection?
%Of internet accessibility? Myriad regulatory schemes based on just these sorts
%of benefits depend upon intermittent, temporary government entry onto private
%property.

[The exceptions in section III of the majority opinion were, in Justice Breyer's
view, insufficient to address this problem. In particular, regarding the Court's
position that ``the government may require property owners to cede a right of
access as a condition of receiving certain benefits, without causing a
taking'':]

Labor peace (brought about through union organizing) is one such benefit, at
least in the view of elected representatives. They wrote laws that led to rules
governing the organizing of agricultural workers. Many of them may well have
believed that union organizing brings with it ``benefits,'' including community
health and educational benefits, higher standards of living, and (as I just
said) labor peace.
%See, \textit{e.g.}, 1975 Cal. Stats. ch. 1, \S~1 (stating
%that the purpose of the Agricultural Labor Relations Act was to ``ensure peace
%in the agricultural fields by guaranteeing justice for all agricultural workers
%and stability in labor relations'').
A landowner, of course, may deny the
existence of these benefits, but a landowner might do the same were a regulatory
statute to permit brief access to verify proper preservation of wetlands or the
habitat enjoyed by an endangered species or, for that matter, the safety of
inspected meat. So, if a regulation authorizing temporary access for purposes of
organizing agricultural workers falls outside of the Court's exceptions and is a
\textit{per se} taking, then to what other forms of regulation does the Court's
\textit{per se} conclusion also apply?


%\readinghead{II}
%
%Finally, I touch briefly on remedies, which the majority does not address. The
%Takings Clause prohibits the Government from taking private property for public
%use without ``just compensation.'' U. S. Const., Amdt. 5. But the employers do
%not seek compensation. They seek only injunctive and declaratory relief. Indeed,
%they did not allege any damages. On remand, California should have the choice of
%foreclosing injunctive relief by providing compensation. 
%
%\readinghead{* * *}
%
%I recognize that the Court's prior cases in this area are not easy to apply.
%Moreover, words such as ``temporary,'' ``permanent,'' or ``too far'' do not
%define themselves. But I do not believe that the Court has made matters clearer
%or better. Rather than adopt a new broad rule and indeterminate exceptions, I
%would stick with the approach that I believe the Court's case law sets forth.
%``Better the devil we know \ldots.'' A right of access such as the right at
%issue here, a nonpermanent right, is not automatically a ``taking.'' It is a
%regulation that falls within the scope of \textit{Penn Central}. Because the
%Court takes a different view, I respectfully dissent.

