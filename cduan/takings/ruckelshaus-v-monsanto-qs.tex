\expected{ruckelshaus-v-monsanto}


\item What exactly is the ``property'' that the government has taken in
\emph{Monsanto}? One answer is the data that Monsanto spent \$23 million to
develop. But then the government would ``take'' that data by requiring
Monsanto to disclose it merely for purposes of pesticide approval---something
the Court plainly does not hold. What rights or value has Monsanto lost? What
does that tell you about what property is?


\item \textbf{Other IP rights}.
Trade secrets exist under state law, to which courts may look in determining
whether a property interest exists. What about federal IP rights? The issue is a
debated.\footnote{\textit{Compare, e.g.}, Davida H. Isaacs, \textit{Not All
Property Is Created Equal: Why Modern Courts Resist Applying the Takings Clause
to Patents, and Why They Are Right to Do So}, 15 \textsc{Geo. Mason L. Rev.} 1
(2007), \textit{with} Adam Mossoff, \textit{Patents As Constitutional Private
Property: The Historical Protection of Patents Under the Takings Clause}, 87
\textsc{B.U. L. Rev}. 689, 689 (2007); \textit{see generally} Thomas F. Cotter,
\textit{Do Federal Uses of Intellectual Property Implicate the Fifth
Amendment?}, 50 \textsc{Fla. L. Rev}. 529 (1998). (``[T]he law of takings with
regard to intellectual property can only be characterized as a muddle within the
muddle.'').}


Should intellectual property receive takings protection? On the one hand, the
underlying statutes give them the attributes of property. 35 U.S.C. \S~261
(``Subject to the provisions of this title, patents shall have the attributes of
personal property.''). On the other, IP rights lack many of the traditional
attributes of property. Not only are they intangible, but they constitute a
government delegation to private parties of regulatory power over the actions of
others. To the extent the government wishes to curtail these rights---or
otherwise adjust the governing regime, introducing takings doctrine may upset
its ability to adjust a regulatory regime to changing circumstances. 


Moreover, the malleability of the concept of ``property'' complicates matters,
for the question whether an intangible interest is property may arise in a
context independent of any takings issues. Once the property switch is flipped,
however, the complexities of takings analysis kick in.
\having{kremen-v-cohen}{Recall, for example, the
issue covered earlier as to
whether domain names are property for purposes of
state law. \textit{Kremen v. Cohen}'s answer in the affirmative afforded a
remedy for a wronged party in a conversion action, but the classification could
ripple through other bodies of law.}{We will later cover whether domain names
are property for purposes of a conversion action in \emph{Kremen v. Cohen}. That
classification could ripple through other bodies of law.}{In \emph{Kremen v.
Cohen}, for example, the Ninth Circuit held that a domain name could be property
for purposes of a conversion action. \emph{See} 337 F.3d 1024 (9th Cir. 2003).
That classification could ripple through other bodies of law.}
For example, the Anticybersquatting Consumer
Protection Act (ACPA) allows trademark holders to claim domain names containing
the marks from those who registered them with a ``bad faith intent to profit.''
15 U.S.C. \S~1125(d). But if a domain name is property---one that one acquires
by registering it---how is ACPA's operation not a taking without just
compensation? Worse, how is it not taking from A and giving to B as prohibited
by the ``Public Use'' Clause? To date, courts have not been receptive to this
argument, \textit{DaimlerChrysler v. The Net Inc}., 388 F.3d 201 (6th Cir.
2004), but it suggests the difficulties with casually applying the label of
property to interests that exist outside the common law property tradition.


\defcase{james-v-campbell}{
parties=James v. Campbell,
cite=104 U.S. 356,
year=1882,
}

\defcase{horne-v-usda}{
parties=Horne v. Department of Agriculture,
cite=576 U.S. 350,
year=2015,
}

\defcase{schillinger-v-us}{
parties=Schillinger v. United States,
cite=155 U.S. 163,
year=1894,
}

\defcase{golden-v-us}{
parties=Golden v. US,
cite=955 F.3d 981,
court=Fed. Cir.,
year=2020,
}

\defjrnart{masur-mortara}{
Jonathan Masur & Adam Mortara, Patents, Property, and Prospectivity, 71 Stanford
Law Review 963 (2019)
}

\item If intellectual property is protected under the Takings Clause, what
government acts constitute a taking? In particular, is infringement of a patent
a taking? In \inline{james-v-campbell}, the Supreme Court in dicta held that a
patent ``confers upon the patentee an exclusive property in the patented
invention which cannot be appropriated or used by the government itself, without
just compensation.'' \sentence{james-v-campbell at 358}.
Recent cases have cited \inline{james-v-campbell} approvingly for this point,
leading some scholars to argue that the proposition is good law. \sentence{see,
e.g., horne-v-usda}. However, \inline{james-v-campbell} was followed just a few
years later by \inline{schillinger-v-us}, which rejected the argument that
government infringement of a patent is a taking, a view that courts continue to
follow. \sentence{see schillinger-v-us at 168; golden-v-us at 988}. As one
scholar has put it, patent infringement ``is described in terms of eminent
domain or takings when that characterization is irrelevant to the resolution of
the case.'' \sentence{masur-mortara at 991-992}.


\defcase{florida-prepaid}{
parties=Florida Prepaid Postsecondary Ed. Expense Bd. v. College Savings Bank,
cite=527 U.S. 627,
year=1999,
}


\defcase{goldberg}{
parties=Goldberg v. Kelly,
cite=397 U.S. 254,
year=1970,
}
\defcase{perry}{
parties=Perry v. Sindermann,
cite=408 U.S. 593,
year=1972,
}

\defcase{goss}{
parties=Goss v. Lopez,
cite=419 U.S. 565,
year=1975,
}
\item \textbf{Property under the Due Process Clause}. In
\inline{florida-prepaid}, the Supreme Court reasoned that patents ``are surely
included within the `property' of which no person may be deprived by a State
without due process of law.'' \sentence{florida-prepaid at 642}. In fact, many
things have been deemed property under the Due Process Clause: welfare benefits,
professorial tenure, and public education. \sentence{see goldberg at 262 note 8;
perry at 601; goss at 574}. Should these be property as well under the Takings
Clause? For other purposes? Can something to be property for one constitutional
provision but not another? Note that in these cases, all that the Due Process
Clause required was notice
and a hearing before deprivation of property; it does not require any monetary
compensation.


\defcase{bonito-boats}{
parties={Bonito Boats, Inc. v. Thunder Craft Boats, Inc.},
cite=489 U.S. 141,
year=1989,
}

\item \textbf{Preemption}. \emph{Monsanto} rejects an argument that federal
preemption of a state-law intellectual property right avoids the Takings Clause.
There are actually quite a few examples of federal preemption of state IP
rights. Federal patent law preempted a Florida statutory protection for boat
hull designs, \clause{see bonito-boats}, and the recently enacted Classics
Protection and Access Act, 17 U.S.C. \S~301(c), preempted state-level copyright
protection in certain old sound recordings. Is just compensation owed? What
value does preemption of state intellectual property rights serve?


\item \textbf{Regulatory exclusivities}. Under the 1978 FIFRA amendments
described in \emph{Monsanto}, if a pesticide manufacturer seeks approval for a
new active ingredient, the EPA cannot use that data in considering other
manufacturers' products for 10 years. The first manufacturer enjoys a
patent-like right over that active ingredient, insofar as competitors cannot win
approval for that ingredient (at least without doing expensive and duplicative
tests to reproduce the data). In the pharmaceutical industry, there are
similarly ``regulatory exclusivities'' that limit the Food and Drug
Administration's ability to approve competitor drugs for several years after
a first product is approved.

\defjrnart{feldman-regulatory-property}{
Robin Feldman, Regulatory Property: The New IP, 40 Columbia Journal of Law and
the Arts 53 (2016)
}

These regimes are much like intellectual property rights, and some commentators
have described them as such. \sentence{see, e.g., feldman-regulatory-property}.
To the extent that they are a kind of property, regulatory exclusivities are
often criticized as poorly designed. If you were advising lawmakers on designing
a regulatory exclusivity, what would you recommend they include? Consider
revisiting James Grimmelmann's list of themes of property law to answer this
question.


