We now address a final method of resolving incompatible property uses.
\textbf{Eminent domain}, also called \textbf{condemnation},
is the inherent power of the state to transfer title of
private property into state hands. In the United States, when the government
``takes'' land in this manner, it must comply with the Takings Clause of the
Fifth Amendment:
\begin{quote}
[N]or shall private property be taken for public use, without just compensation.
\end{quote}
This
brief constitutional provision encompasses three distinct issues that we will
deal with in this chapter (though not in this order): (1) has there been a
``taking'' of private property? (2) Is the taking for ``public use''; and (3)
has ``just compensation'' been provided?

[An administrative note: The Supreme Court has been especially active on takings
jurisprudence recently, which means that the opinions are often very long. In
preparing this and the next chapter's readings, I have very heavily edited these
cases for length. It is possible that I've accidentally omitted some information
necessary to understand the case, or some information that the notes and
questions rely on. If that has happened, please let me know so I can correct
it.]
