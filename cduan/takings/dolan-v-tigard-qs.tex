\expected{dolan-v-tigard}

\item Would this case have come out differently if the City of Tigard had made
more specific factual findings? Or do you think that the greenway and bicycle
path requirements could never avoid being an exaction? If you were a member of
the City Planning Commission, what would you have done once this case arrived
back on remand after the Supreme Court's decision?

\defcase{koontz-v-st-john}{
Koontz v. St. John's River Water Management District, 570 U.S. 595 (2013)
}
\defcase{eastern-v-apfel}{
Eastern Enterprises v. Apfel, 524 U.S. 498 (1998)
}

\expected{ruckelshaus-v-monsanto}

\item In \emph{Monsanto}, the government conditioned marketing approval of
pesticides on giving up intellectual property. Is that an exaction?

\item \textbf{Exactions of Money}. In \inline{koontz-v-st-john}\optclause, the
Supreme Court considered a local government's demand that, to receive a permit
for developing part of his land, the property owner pay for contractors to do
work on unrelated public lands. The local government argued that this was not an
exaction, because no interest in real property was being demanded. The Court
disagreed, holding that ``the government's demand for property from a land-use
permit applicant must satisfy the requirements of \emph{Nollan} and \emph{Dolan}
even when [the government's] demand is for money.''

While a general requirement to pay money is not a taking, \clause{see
eastern-v-apfel}, the \inline{koontz-v-st-john} Court held that a taking could
occur where
there is a ``direct link'' between the monetary demand and a property interest.
In the Court's view, the land at issue in the permitting process created that
direct link, rendering the Takings Clause applicable to the monetary exaction.
The dissenters, by contrast, reasoned that the \emph{Nollan}/\emph{Dolan}
framework applies only if the government's underlying demand would constitute a
taking, standing alone. Since a demand for monetary payment in isolation is not
a taking, the dissenters reasoned, a payment demand as a condition of permitting
cannot be either.

\item Building permits often require compliance with building codes that specify
construction materials, arrangement of exits, fire safety, and numerous other
matters. These requirements are conditions on obtaining the permit. Are these
building code requirements exactions? Try to make an argument in both
directions.

\item \textbf{Categorical Exclusions.} Just as some government acts are takings
as a categorical matter; others are categorically excluded. \textit{Koontz}
mentions that taxes and user fees are never takings. Why not? One possibility is
the idea that the private property protected by the Takings Clause only protects
discrete resources, and does not apply to legally obligated acts like the
payment of money. That was the logic of five Justices in
\inline{eastern-v-apfel}, which was discussed and distinguished in
\inline{koontz-v-st-john}.

But can we do more than provide a definitional exclusion? Eduardo Pe\~nalver
observes: 
\begin{quotation}
As Richard Epstein---one of the few scholars to focus substantial effort on the
issue---has noted, ``[t]he taxing power is placed in one compartment; the
takings power in another,'' and scholarly discussion of the conflict between the
two never really gets off the ground. In his book \textit{Takings}, Epstein
invited readers to view the conceptual similarity between takings and taxes as a
reason to dramatically curtail the state's power to tax. Specifically, Epstein
argued that the Takings Clause required the government to adopt a system of
proportional taxation, also known as a ``flat tax.'' This argument flew in the
face of settled constitutional orthodoxy, which since the founding era has
understood the state's power to tax as being virtually plenary.\ldots 

This cool response to Epstein's proposal is unsurprising. The constitutional
doctrine defining the state's power to tax is so entrenched that it is nearly
axiomatic. In contrast, Takings Clause jurisprudence is characterized by nothing
if not the confusion and intense disagreement it generates.\ldots
\end{quotation}
Eduardo Mois\'es Pe\~nalver, \textit{Regulatory Taxings}, 104 \textsc{Colum. L.
Rev}. 2182, 2185-86 (2004) (footnotes omitted). Pe\~nalver draws an opposite
conclusion from Epstein's, noting that the seeming conflict between the two
powers stems not from the reach of the taxing power, but from the fact that
courts have applied the Takings Clause beyond its original understanding as a
simple requirement of compensation when the power of eminent domain is
exercised. If the clause were read more narrowly, the apparent tension would
disappear. On this view, ``Takings are the state's direct appropriation of
parcels of property from individuals through the power of eminent domain, and
taxes are generally applicable measures, enacted under the state's power to tax,
requiring individuals to make payments to the state. Each corresponds to
different and nonoverlapping governmental powers.'' \textit{Id.} at 2188.

There are also government actions that do affect specific pieces of property
that are nonetheless excluded from operation of the Takings Clause. We have
already seen one example in the rule\having{lucas-v-sccc}{We have already seen
one example in the rule---discussed in the opinions in \textit{Lucas}---}{One
example is the rule }{One example is the rule }that regulation of a common law
nuisance is never a taking.
Other examples include government forfeitures, federal control of navigable
waterways, and the state's right to destroy property to contain the spread of
fire. \textit{See generally} David A. Dana \& Thomas W. Merrill,
\textsc{Takings} 110-120 (Foundation Press 2002); \emph{AmeriSource Corp. v.
United States}, 525 F.3d 1149, 1153 (Fed. Cir.2008) (``Property seized and
retained pursuant to the police power is not taken for a `public use' in the
context of the Takings Clause.''). What explains these exceptions? Perhaps they,
too, may be understood as simply categorically different government powers
(i.e., if the Takings Clause is read as simply applying to eminent domain, the
existence of regulatory takings notwithstanding). Dana and Merrill suggest that
we might understand these exceptions similarly to the nuisance exclusion---the
powers are within traditional conceptions of the state's police powers, and they
have a long historical pedigree, long enough that property owners may be said to
be on imputed notice that they may be exercised. 

