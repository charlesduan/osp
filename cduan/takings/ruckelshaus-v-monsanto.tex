\reading{Ruckelshaus v. Monsanto Co.}
\readingcite{467 U.S. 986 (1984)}

\opinion Justice \textsc{Blackmun} delivered the opinion of the Court.

In this case, we are asked to review a United States District Court's
determination that several provisions of the Federal Insecticide, Fungicide, and
Rodenticide Act (FIFRA)
are unconstitutional. The provisions at issue authorize the Environmental
Protection Agency (EPA) to use data submitted by an applicant for registration
of a pesticide in evaluating the application of a subsequent applicant,
and to disclose publicly some of the submitted data.



\readinghead{I}

Over the past century, the use of pesticides to control weeds and minimize crop
damage caused by insects, disease, and animals has become increasingly more
important for American agriculture.
While pesticide use has led to improvements in productivity, it has also led to
increased risk of harm to humans and the environment.\ldots

[The legislative history of FIFRA is complex but important. Here are the
relevant parts.]
As first enacted, FIFRA was primarily a licensing and labeling statute.
[Applicants wishing to sell pesticides were required to provide the government
with testing data about the product, which the Department of Agriculture (later
the EPA) would use to verify the pesticide's labeling information.]


Congress undertook a comprehensive revision of FIFRA through
the adoption of the Federal Environmental Pesticide Control Act of 1972.
The amendments transformed FIFRA from a labeling law into a
comprehensive regulatory statute. [Relevant to the case, the law permitted the
EPA to use submitted testing data for two purposes: (1)
during consideration of a second manufacturer's application for approval
of a pesticide similar active ingredients, and
(data-consideration); and
(2) disclosure to the public as part of the application process
(data-disclosure).
However, the applicant could designate parts of the data as ``trade secrets or
commercial or financial information.'' Under the 1972 amendments, the EPA was
not allowed to use correctly designated data for either of the two above
purposes.]

Congress enacted other amendments to FIFRA in 1978. [Regarding the
data-consideration provision],
applicants are granted a 10-year period of exclusive use for data on new active
ingredients contained in pesticides registered after September 30, 1978. All
other data submitted after December 31, 1969, may be cited and considered in
support of another application for 15 years after the original submission if the
applicant offers to compensate the original submitter[, with the amount of
compensation determined by binding arbitration.] Data that do not qualify for
either the 10-year period of exclusive use or the 15-year period of compensation
may be considered by EPA without limitation.

Also in 1978, Congress added a new subsection,
that provides for disclosure of all health,  safety, and environmental data to
qualified requesters, notwithstanding the prohibition against disclosure of
trade secrets[, with limited exceptions].


\readinghead{II}

Appellee Monsanto Company (Monsanto) is an inventor, developer, and producer of
various kinds of chemical products, including pesticides.\ldots
The District Court found that
Monsanto had incurred costs in excess of \$23.6 million in developing the
health, safety, and environmental data submitted by it under FIFRA.\ldots
Monsanto also uses this information to develop additional end-use
products and to expand the uses of its registered products. The information
would also be valuable to Monsanto's competitors. For that reason, Monsanto has
instituted stringent security measures to ensure the secrecy of the data.

\ldots Monsanto alleged that [the data-consideration and data-disclosure]
provisions effected a ``taking'' of property without just compensation, in
violation of the Fifth Amendment. [The district court agreed.]
We noted probable jurisdiction.



\readinghead{III}

In deciding this case, we are faced with [three] questions: (1) Does Monsanto have
a property interest protected by the Fifth Amendment's Taking Clause in the
health, safety, and environmental data it has submitted to EPA? (2) If so, does
EPA's use of the data to evaluate the applications of others or EPA's disclosure
of the data to qualified members of the public effect a taking of that property
interest? (3) If there  is a taking, is it a taking for a public
use?\ldots\edfootnote{The Court also considered the means of providing just
compensation for any taking, and concluded that the Tucker Act gave Monsanto a
sufficient means of obtaining compensation.}

This Court never has squarely addressed the applicability of the protections of
the Taking Clause of the Fifth Amendment to commercial data of the kind involved
in this case. In answering the question now, we are mindful of the basic axiom
that ``\,`[p]roperty interests\ldots are not created by the Constitution.
Rather,
they are created and their dimensions are defined by existing rules or
understandings that stem from an independent source such as state law.'\,''
Monsanto asserts that the health, safety, and environmental data it
has submitted to EPA are property under Missouri law, which recognizes trade
secrets, as defined in \S~757, Comment \textit{b}, of the Restatement of Torts,
as property.
The Restatement defines a trade secret as ``any formula,
pattern, device or compilation of information which is used in one's business,
and which gives him an opportunity to obtain an advantage over competitors who
do not know or use it.'' And the parties have
stipulated that much of the information, research, and test data that Monsanto
has submitted under  FIFRA to EPA ``contains or relates to trade secrets as
defined by the Restatement of Torts.''

Because of the intangible nature of a trade secret, the extent of the property
right therein is defined by the extent to which the owner of the secret protects
his interest from disclosure to others.
Information that is public knowledge or that is generally known in an
industry cannot be a trade secret. If an
individual discloses his trade secret to others who are under no obligation to
protect the confidentiality of the information, or otherwise publicly discloses
the secret, his property right is extinguished.

Trade secrets have many of the characteristics of more tangible forms of
property. A trade secret is assignable.
A trade secret can form the res of a trust,
and it passes to a trustee in bankruptcy.

Even the manner in which Congress referred to trade secrets in the legislative
history of FIFRA supports the general perception of their property-like nature.
In discussing the 1978 amendments to FIFRA, Congress recognized that data
developers like Monsanto have a ``proprietary interest'' in their data.
Further, Congress reasoned that submitters of data are
``entitled'' to ``compensation'' because they ``have legal ownership of the
data.'' This general
perception of trade secrets as property is consonant with a notion of
``property'' that extends beyond land and tangible goods and includes the
products of an individual's ``labour and invention.'' 2 W. Blackstone,
Commentaries *405; see generally J. Locke, The Second Treatise of Civil
Government, ch. 5 (J. Gough ed. 1947).

Although this Court never has squarely addressed the question whether a person
can have a property interest in a trade secret, which is admittedly intangible,
the Court has found other kinds of intangible interests to be property for
purposes of the Fifth Amendment's Taking Clause. See, \textit{e.g.},
\textit{Armstrong} v. \textit{United States}, 364 U. S. 40, 44, 46 (1960)
(materialman's lien provided for under Maine law protected by Taking Clause);
\textit{Louisville Joint Stock Land Bank} v. \textit{Radford}, 295 U. S. 555,
596-602 (1935) (real estate lien protected); \textit{Lynch} v. \textit{United
States}, 292 U. S. 571, 579 (1934) (valid contracts are property within meaning
of the Taking Clause). That intangible property rights protected by state law
are deserving of the protection of the Taking Clause has long been implicit in
the thinking of this Court:
\begin{quote}
It is conceivable that [the term ``property'' in the Taking
Clause] was used in its vulgar and untechnical sense of the physical thing with
respect to which the citizen exercises rights recognized by law. On the other
hand, it may have been employed in a more accurate sense to denote the group of
rights inhering in the citizen's relation to the physical thing, as the right to
possess, use and dispose of it. In point of fact, the construction given the
phrase has been the latter.
\end{quote}
\textit{United States} v. \textit{General Motors
Corp.}, 323 U. S. 373, 377-378 (1945).

We therefore hold that to the
extent that Monsanto has an interest in its health, safety, and environmental
data cognizable as a trade-secret property right under Missouri law,  that
property right is protected by the Taking Clause of the Fifth
Amendment.

\readinghead{IV}

Having determined that Monsanto has a property interest in the data it has
submitted to EPA, we confront the difficult question whether a ``taking'' will
occur when EPA discloses those data or considers the data in evaluating another
application for registration.\ldots

The inquiry into whether a taking has occurred is essentially an ad
hoc, factual inquiry. The Court,
however, has identified several factors that should be taken into account when
determining whether a governmental action has gone beyond ``regulation'' and
effects a ``taking.'' Among those factors are: the character of the
governmental action, its economic impact, and its interference with reasonable
investment-backed expectations.
It is to the last of these three
factors that we now direct our attention, for we find that the force of this
factor is so overwhelming, at least with respect to certain of the data
submitted by Monsanto to EPA, that it disposes of the taking question regarding
those data.


\readinghead{A}

A ``reasonable investment-backed expectation'' must be more than a ``unilateral
expectation or an abstract need.''\ldots
With respect to any data submitted to EPA on or after October 1, 1978,
Monsanto knew that\ldots once
the 10-year period had expired, EPA could use the data [in considering another's
application] without Monsanto's
permission.\ldots The statute also
gave Monsanto notice that much of the health, safety, and efficacy data provided
by it could be disclosed to the general public at any time. If,
despite the data-consideration and data-disclosure provisions in the statute,
Monsanto chose to submit the requisite data in order to receive a registration,
it can hardly argue that its reasonable investment-backed  expectations are
disturbed when EPA acts to use or disclose the data in a manner that was
authorized by law at the time of the submission.

Monsanto argues that the statute's requirement that a submitter give up its
property interest in the data constitutes placing an unconstitutional condition
on the right to a valuable Government benefit. But
Monsanto has not challenged the ability of the Federal Government to regulate
the marketing and use of pesticides. Nor could Monsanto successfully make such a
challenge, for such restrictions are the burdens we all must bear in exchange
for the advantage of living and doing business in a civilized community.\ldots
Thus, as long as Monsanto is aware of the conditions under which the data are
submitted, and the conditions are rationally related to a legitimate Government
interest, a voluntary submission of data by an applicant in exchange for the
economic advantages of a registration can hardly be called a taking.


\readinghead{B}

[For data prior to the 1972 FIFRA amendments, the Court concluded that federal
law offered no promise that the government would protect trade secrets.]
Absent an express promise, Monsanto had no reasonable,
investment-backed expectation that its information would remain inviolate in the
hands of EPA.\ldots


\readinghead{C}

The situation may be different, however, with respect to data submitted by
Monsanto to EPA during the period from October 22, 1972, through September 30,
1978. Under the statutory scheme then in effect, a submitter was given an
opportunity to protect its trade secrets from disclosure by designating them as
trade secrets at the time of submission.\ldots The statute gave Monsanto
explicit assurance that EPA was prohibited from disclosing publicly, or
considering in connection with the application of another, any data submitted by
an applicant if both the applicant and EPA determined the data to constitute
trade secrets.\ldots This
explicit governmental guarantee formed the basis of a reasonable
investment-backed expectation. If EPA, consistent with the authority granted it
by the 1978 FIFRA amendments, were now to disclose trade-secret data or consider
those data in evaluating the application of a subsequent applicant in a manner
not authorized by the version of FIFRA in effect between 1972 and 1978, EPA's
actions would frustrate Monsanto's reasonable investment-backed expectation with
respect to its control over the use and dissemination of the data it had
submitted.

The right to exclude others is generally one of the most essential sticks in
the bundle of rights that are commonly characterized as property.
With respect to a trade secret, the
right to exclude others is central to the very definition of the property
interest. Once the data that constitute a trade secret are disclosed to others,
or others are allowed to use those data, the holder of the trade secret has lost
his property interest in the data.\readingfootnote{16}{We emphasize that the
value of a trade secret lies in the competitive advantage it gives its owner
over competitors. Thus, it is the fact that operation of the data-consideration
or data-disclosure provisions will allow a competitor to register more easily
its product or to use the disclosed data to improve its own technology that may
constitute a taking. If, however, a public disclosure of data reveals, for
example, the harmful side effects of the submitter's product and causes the
submitter to suffer a decline in the potential profits from sales of the
product, that decline in profits stems from a decrease in the value of the
pesticide to consumers, rather than from the destruction of an edge the
submitter had over its competitors, and cannot constitute the taking of a trade
secret.}  That the data retain usefulness for Monsanto even after they are
disclosed---for example, as bases from which to develop new products or refine
old products, as marketing and advertising tools, or as information necessary to
obtain registration in foreign countries---is irrelevant to the determination of
the economic impact of the EPA action on Monsanto's property right. The economic
value of that property right lies in the competitive advantage over others that
Monsanto enjoys by virtue of its exclusive access to the data, and disclosure or
use by others of the data would destroy that competitive edge.

EPA encourages us to view the situation not as a taking of Monsanto's property
interest in the trade secrets, but as a ``pre-emption'' of whatever property
rights Monsanto may have had in those trade secrets.\edfootnote{If you are not
familiar with it, preemption is the doctrine that where a state law conflicts
with a federal law, the federal law overrides the state law. The EPA was in
essence arguing that FIFRA eliminated state trade secret protection by virtue of
preemption.}\ldots
This argument proves too much. If Congress can ``pre-empt'' state property
law in the manner advocated by EPA, then the Taking Clause has lost all
vitality. This Court has stated that a sovereign, ``by \textit{ipse dixit}, may
not transform private property into public property without compensation\ldots.
This is the very kind of thing that the Taking Clause of the Fifth Amendment was
meant to prevent.''

\ldots.

In summary, we hold that EPA's consideration or disclosure of data submitted by
Monsanto to the agency prior to October 22, 1972, or after September 30, 1978,
does not effect a taking. We further hold that EPA consideration or disclosure
of health, safety, and environmental data will constitute a taking if Monsanto
submitted the data to EPA between October 22, 1972, and September 30,
1978\ldots.



\readinghead{V}

We must next consider whether any taking of private property that may occur by
operation of the data-disclosure and data-consideration provisions of FIFRA is a
taking for a ``public use.'' We have recently stated that the scope of the
``public use'' requirement of the Taking Clause is ``coterminous with the scope
of a sovereign's police powers.''
The role of the courts in second-guessing the legislature's
judgment of what constitutes a public use is extremely narrow.

\ldots Here, the public purpose behind the data-consideration provisions
is clear from  the legislative history. Congress believed that the provisions
would eliminate costly duplication of research and streamline the registration
process, making new end-use products available to consumers more quickly.
Allowing applicants for registration, upon payment of compensation, to use data
already accumulated by others, rather than forcing them to go through the
time-consuming process of repeating the research, would eliminate a significant
barrier to entry into the pesticide market, thereby allowing greater competition
among producers of end-use products. Such a procompetitive
purpose is well within the police power of Congress.

\ldots
The court found that the
data-disclosure provisions served no use [because the EPA-approved pesticide
label] provided the public with all the assurance
it needed that the product is safe and effective.
It is enough for us to state that the optimum amount of disclosure to the
public is for Congress, not the courts, to decide, and that the statute embodies
Congress'  judgment on that question.
We further observe, however, that public disclosure can provide an
effective check on the decisionmaking processes of EPA and allows members of the
public to determine the likelihood of individualized risks peculiar to their use
of the product.

We therefore hold that any taking of private property that may occur in
connection with EPA's use or disclosure of data submitted to it by Monsanto
between October 22, 1972, and September 30, 1978, is a taking for a public use.

[\textsc{Justice O'Connor} dissented in part. In her view, the government had a
confidentiality obligation in the pre-1972 data, so a disclosure of it would
have been a taking.]

