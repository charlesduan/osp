\item Initially, consider Lexmark's business model that gave rise to this case.
Lexmark discounts its printers heavily, sometimes selling them at or below cost.
It then marks up the prices of consumable supplies like toner and ink,
recovering any losses on the printer and making the company's profits. This
is known as the ``razor and blades business model'' (sell the razor handles
cheaply, and then mark up the blades), and companies use it for a wide variety
of products. (Single-serve coffee pods are another classic example.)

This business model is why Lexmark pursued the toner refillers so vigorously.
Competitors can supply the consumable parts at much cheaper prices, because the
original manufacturer is overpricing those parts as part of the business model.
But if consumers buy from those competitors, then the original manufacturer
never recovers the initial loss. So the razor and blades model depends on some
mechanism of exclusion---some property right, perhaps---that keeps competitors
out.

Why use this business model? Couldn't Lexmark just charge more for the printers?

\defcase{mppc}{
parties=Motion Picture Patents Co. v. Universal Film Manufacturing Co.,
cite=243 U.S. 502,
year=1917,
}
\defjrnart{cassady}{
Ralph Cassady, Jr., Monopoly in Motion Picture Production and Distribution:
1908--1915, 32 Southern California Law Review 325 (1959)
}



\item What else might post-sale restrictions be used for, besides preventing
resale or repair? In \inline{mppc}, the patent holder held patents on movie
projectors, and imposed a condition on theaters that they only use licensed
projectors on the patent holder's terms. \sentence{see mppc at 506-507}. The
patent holder, a licensing firm created and run by Thomas Edison, wielded
extraordinary power over the motion picture industry during the early 1900s,
unilaterally deciding what films would be made, which actors would be promoted,
and which theaters would be allowed to operate. \sentence{see cassady}.

Should a patent's right to exclude entail this level of industry control?


\item \emph{Impression} does not just pit two types of property against each
other---it pits two specific rights of property against each other. The toner
cartridge owner enjoys a right to alienate to a refiller or anyone else.
Lexmark, on the other hand, enjoys a right to subdivide its patent interest, in
the same way that a landlord can lease one room of a house and retain the rest
of it.

The Court holds that the right to alienate a chattel overrides the right to
subdivide a patent. Do you agree? Can you think of a basis for prioritizing one
right over the other? One point to consider: The right to subdivide is not
absolute, as the \emph{\useterm{numerus clausus}} principle and menu of estates
in land
demonstrate. But neither is the right to alienate---regulations such as drug
approval can prohibit sales of products.


\defcase{kirtsaeng}{
parties={Kirtsaeng v. John Wiley and Sons, Inc.},
cite=568 U.S. 519,
year=2013,
}

\defbook{perzanowski-schultz}{
Aaron Perzanowski & Jason Schultz, The End of Ownership: Personal Property in
the Digital Economy (2016)
}

\defjrnart{van-houweling}{
author=Molly Shaffer {Van Houweling},
title=The New Servitudes,
jcite=96 Georgetown Law Journal 885,
year=2008
}


\item Patents are far from the only vehicle for imposing post-sale restraints on
consumer goods. Copyright holders have sought to use their copyrights to prevent
resale of books or to enforce minimum retail prices. The Supreme Court held such
copyright-based restraints unenforceable, in a case about resale of used
textbooks. \sentence{see kirtsaeng}. Other statutes, including the Digital
Millennium Copyright Act and the Computer Fraud and Abuse Act, have been used to
restrict consumers from reselling their purchased goods or using those goods in
ways contrary to the manufacturers' wishes. \sentence{see generally
perzanowski-schultz; van-houweling}.


\item If you were representing Lexmark, how would you advise the company to
proceed after this decision? Can you come up with another legal arrangement that
prevents refilling? Look back through the property materials you've learned so
far.
