\reading{Impression Products, Inc. v. Lexmark International, Inc.}
\readingcite{137 S.Ct. 1523 (2017)}


\opinion Chief Justice \textsc{Roberts} delivered the opinion of the Court.

A United States patent entitles the patent holder (the ``patentee''), for a
period of 20 years, to ``exclude others from making, using, offering for sale,
or selling [its] invention throughout the United States or importing the
invention into the United States.'' Whoever engages in one
of these acts ``without authority'' from the patentee may face liability for
patent infringement.

When a patentee sells one of its products, however, the patentee can no longer
control that item through the patent laws---its patent rights are said to
``exhaust.'' The purchaser and all subsequent owners are free to use or resell
the product just like any other item of personal property, without fear of an
infringement lawsuit.

[The question in this case is]
%This case presents two questions about the scope of the patent exhaustion
%doctrine: First,
whether a patentee that sells an item under an express
restriction on the purchaser's right to reuse or resell the product may enforce
that restriction through an infringement lawsuit.\edfootnote{The case considered
a second question regarding sales outside the United States, not reproduced
here.}
%And second, whether a patentee
%exhausts its patent rights by selling its product outside the United States,
%where American patent laws do not apply. We conclude that a patentee's decision
%to sell a product exhausts all of its patent rights in that item, regardless of
%any restrictions the patentee purports to impose or the location of the sale.



\readinghead{I}

The underlying dispute in this case is about laser printers---or, more
specifically, the cartridges that contain the powdery substance, known as toner,
that laser printers use to make an image appear on paper. Respondent Lexmark
International, Inc. designs, manufactures, and sells toner cartridges to
consumers in the United States and around the globe. It owns a number of patents
that cover components of those cartridges and the manner in which they are used.

When toner cartridges run out of toner they can be refilled and used again. This
creates an opportunity for other companies---known as remanufacturers---to
acquire empty Lexmark cartridges from purchasers in the United States and
abroad, refill them with toner, and then resell them at a lower price than the
new ones Lexmark puts on the shelves.

Not blind to this business problem, Lexmark structures its sales in a way that
encourages customers to return spent cartridges. It gives purchasers two
options: One is to buy a toner cartridge at full price, with no strings
attached. The other is to buy a cartridge at roughly 20-percent  off through
Lexmark's ``Return Program.'' A customer who buys through the Return Program
still owns the cartridge but, in exchange for the lower price, signs a contract
agreeing to use it only once and to refrain from transferring the empty
cartridge to anyone but Lexmark. To enforce this single-use/no-resale
restriction, Lexmark installs a microchip on each Return Program cartridge that
prevents reuse once the toner in the cartridge runs out.

Lexmark's strategy just spurred remanufacturers to get more creative. Many kept
acquiring empty Return Program cartridges and developed methods to counteract
the effect of the microchips. With that technological obstacle out of the way,
there was little to prevent the remanufacturers from using the Return Program
cartridges in their resale business. After all, Lexmark's contractual
single-use/no-resale agreements were with the initial customers, not with
downstream purchasers like the remanufacturers.

Lexmark, however, was not so ready to concede that its plan had been foiled. In
2010, it sued a number of remanufacturers, including petitioner Impression
Products, Inc., for patent infringement with respect to two groups of
cartridges. One group consists of Return Program cartridges that Lexmark sold
within the United States. Lexmark argued that, because it expressly prohibited
reuse and resale of these cartridges, the remanufacturers infringed the Lexmark
patents when they refurbished and resold them. The other group consists of all
toner cartridges that Lexmark sold abroad and that remanufacturers imported into
the country. Lexmark claimed that it never gave anyone authority to import these
cartridges, so the remanufacturers ran afoul of its patent rights by doing just
that.

Eventually, the lawsuit was whittled down to one defendant, Impression Products,
and one defense: that Lexmark's sales, both in the United States and abroad,
exhausted its patent rights in the cartridges, so Impression Products was free
to refurbish and resell them, and to import them if acquired abroad. [The
district court held that Lexmark's patent rights were exhausted; the Federal
Circuit reversed.]
%Impression
%Products filed separate motions to dismiss with respect to both groups of
%cartridges. The District Court granted the motion as to the domestic Return
%Program cartridges, but denied the motion as to the cartridges Lexmark sold
%abroad. Both parties appealed.

%The Federal Circuit considered the appeals en banc and ruled for Lexmark with
%respect to both groups of cartridges. The court began with the Return Program
%cartridges that Lexmark sold in the United States. Relying on its decision in
%\textit{Mallinckrodt, Inc. v. Medipart, Inc.}, 976 F.2d 700 (1992), the Federal
%Circuit held that a patentee may sell an item and retain the right to enforce,
%through patent infringement lawsuits, ``clearly communicated, ... lawful
%restriction[s] as to post-sale use or resale.'' 816 F.3d 721, 735 (2016). The
%exhaustion doctrine, the court reasoned, derives from the prohibition on making,
%using, selling, or importing items ``without authority.'' \textit{Id.}, at 734
%(quoting 35 U.S.C. \S~271(a)). When you purchase an item you presumptively also
%acquire the authority to use or resell the item freely, but that is just a
%presumption; the same authority does not run with the item when the seller
%restricts post-sale use or resale. 816 F.3d, at 742. Because the parties agreed
%that Impression Products knew about Lexmark's restrictions and that those
%restrictions did not violate any laws, the Federal Circuit concluded that
%Lexmark's sales had not exhausted all of its patent rights, and that the company
%could sue for infringement when Impression Products refurbished  and resold
%Return Program cartridges.
%
%As for the cartridges that Lexmark sold abroad, the Federal Circuit once again
%looked to its precedent. In \textit{Jazz Photo Corp. v. International Trade
%Commission}, 264 F.3d 1094 (2001), the court had held that a patentee's decision
%to sell a product abroad did not terminate its ability to bring an infringement
%suit against a buyer that ``import[ed] the article and [sold] ... it in the
%United States.'' 816 F.3d, at 726-727. That rule, the court concluded, makes
%good sense: Exhaustion is justified when a patentee receives ``the reward
%available from [selling in] American markets,'' which does not occur when the
%patentee sells overseas, where the American patent offers no protection and
%therefore cannot bolster the price of the patentee's goods. \textit{Id.}, at
%760-761. As a result, Lexmark was free to exercise its patent rights to sue
%Impression Products for bringing the foreign-sold cartridges to market in the
%United States.
%
%Judge Dyk, joined by Judge Hughes, dissented. In their view, selling the Return
%Program cartridges in the United States exhausted Lexmark's patent rights in
%those items because any ``authorized sale of a patented article ... free[s] the
%article from any restrictions on use or sale based on the patent laws.''
%\textit{Id.}, at 775-776. As for the foreign cartridges, the dissenters would
%have held that a sale abroad also results in exhaustion, unless the seller
%``explicitly reserve[s] [its] United States patent rights'' at the time of sale.
%\textit{Id.}, at 774, 788. Because Lexmark failed to make such an express
%reservation, its foreign sales exhausted its patent rights.

We granted certiorari to consider the Federal Circuit's decisions\ldots
%with respect
%to both domestic and international exhaustion, 580 U.S. \_\_\_, 137 S.Ct. 546,
%196 L.Ed.2d 442 (2016),
and now reverse.



\readinghead{II}



\readinghead{A}

%First up are the Return Program cartridges that Lexmark sold in the United
%States.
We conclude that Lexmark exhausted its patent rights in [the Return Program]
cartridges
the moment it sold them. The single-use/no-resale restrictions in Lexmark's
contracts with customers may have been clear and enforceable under contract law,
but they do not entitle Lexmark to retain patent rights in an item that it has
elected to sell.

The Patent Act grants patentees the ``right to exclude others from making,
using, offering for sale, or selling [their] invention[s].''
For over 160 years, the doctrine of patent exhaustion has imposed a
limit on that right to exclude. See \textit{Bloomer v. McQuewan}, 14 How. 539,
14 L.Ed. 532 (1853). The limit functions automatically: When a patentee chooses
to sell an item, that product ``is no longer within the limits of the monopoly''
and instead becomes the ``private, individual property'' of the purchaser, with
the rights and benefits that come along with ownership.
A patentee is free to set the price and negotiate contracts with
purchasers, but may not, ``\textit{by virtue of his patent}, control the use or
disposition'' of the product after ownership passes to the purchaser.
The sale ``terminates all patent rights to
that item.''

This well-established exhaustion rule marks the point where patent rights yield
to the common law principle against restraints on alienation. The Patent Act
``promote[s] the progress of science and the useful arts by granting to
[inventors] a limited monopoly'' that allows them to ``secure the financial
rewards'' for their inventions. But once a patentee sells an item, it has
``enjoyed all the rights secured'' by that limited monopoly. Because ``the
purpose of the patent law is fulfilled\ldots when the patentee has received his
reward for the use of his invention,'' that law furnishes ``no basis for
restraining the use and enjoyment of the thing sold.''

We have explained in the context of copyright law that exhaustion has ``an
impeccable historic pedigree,'' tracing its lineage back to the ``common law's
refusal to permit restraints on the alienation of chattels.'' \textit{Kirtsaeng
v. John Wiley \& Sons, Inc.}, 568 U.S. 519, 538
(2013). As Lord Coke put it in the 17th century, if an owner restricts the
resale or use of an item after selling it, that restriction ``is voide,
because\ldots
it is against Trade and Traffique, and bargaining and contracting betweene
man and man.'' 1 E. Coke, Institutes of the Laws of England \S~360, p. 223
(1628); see J. Gray, Restraints on the Alienation of Property \S~27, p. 18 (2d
ed. 1895) (``A condition or conditional limitation on alienation attached to a
transfer of the entire interest in personalty is as void as if attached to a fee
simple in land'').

This venerable principle is not, as the Federal Circuit dismissively viewed it,
merely ``one common-law jurisdiction's general judicial policy at one time
toward anti-alienation restrictions.'' Congress enacted and
has repeatedly revised the Patent Act against the backdrop of the hostility
toward restraints on alienation. That enmity is reflected in the exhaustion
doctrine. The patent laws do not include the right to ``restrain[]\ldots further
alienation'' after an initial sale; such conditions have been ``hateful to the
law from Lord Coke's day to ours'' and are ``obnoxious to the public interest.''
\textit{Straus v. Victor Talking Machine Co.}, 243 U.S. 490, 501
(1917). ``The inconvenience and annoyance to the public that an
opposite conclusion would occasion are too obvious to require illustration.''

But an illustration never hurts. Take a shop that restores and sells used cars.
The business works because the shop can rest assured that, so long as those
bringing in the cars own them, the shop is free to repair and resell those
vehicles. That smooth flow of commerce would sputter if companies that make the
thousands of parts that go into a vehicle could keep their patent rights after
the first sale. Those companies might, for instance, restrict resale rights and
sue the shop owner for patent infringement. And even if they refrained from
imposing such restrictions, the very threat of patent liability would force the
shop to invest in efforts to protect itself from hidden lawsuits. Either way,
extending the patent rights beyond the first sale would clog the channels of
commerce, with little benefit from the extra control that the patentees retain.
And advances in technology, along with increasingly complex supply chains,
magnify the problem.

This Court accordingly has long held that, even when a patentee sells an item
under an express restriction, the patentee does not retain patent rights in that
product.\ldots
% In \textit{Boston Store of Chicago v. American Graphophone Co.}, for
%example, a manufacturer sold graphophones---one of the earliest devices for
%recording and reproducing sounds---to retailers under contracts requiring those
%stores to resell at a specific price. 246 U.S. 8
%(1918). When the manufacturer brought a patent infringement suit against a
%retailer who sold for less, we concluded that there was ``no room for
%controversy'' about the result: By selling the item, the manufacturer placed it
%``beyond the confines of the patent law, [and] could not, by qualifying
%restrictions as to use, keep [it] under the patent monopoly.''
%
%Two decades later, we confronted a similar arrangement in \textit{United States
%v. Univis Lens Co.} There, a company that made eyeglass lenses
%authorized an agent to sell its products to wholesalers and retailers only if
%they promised to market the lenses at fixed prices. The Government filed an
%antitrust lawsuit, and the company defended its arrangement on the ground that
%it was exercising authority under the Patent Act. We held that the initial sales
%``relinquish[ed]\ldots the patent monopoly with respect to the article[s]
%sold,'' so the ``stipulation\ldots fixing resale prices derive[d] no support
%from the patent and must stand on the same footing'' as restrictions on
%unpatented goods.
%
%It is true that \textit{Boston Store} and \textit{Univis} involved resale price
%restrictions that, at the time of those decisions, violated the antitrust laws.
%But in both cases it was the sale of the items, rather than the illegality of
%the restrictions, that prevented the patentees from enforcing those resale price
%agreements through patent infringement suits. And if there were any lingering
%doubt that patent exhaustion applies even when a sale is subject to an express,
%otherwise lawful restriction,
Our recent decision in \textit{Quanta Computer,
Inc. v. LG Electronics, Inc.} settled the matter. In that case, a
technology company---with authorization from the patentee---sold microprocessors
under contracts requiring purchasers to use those processors with other parts
that the company manufactured. One buyer disregarded the restriction, and the
patentee sued for infringement. Without so much as mentioning the lawfulness of
the contract, we held that the patentee could not bring an infringement suit
because the ``authorized sale\ldots took its products outside the scope of the
patent monopoly.'' 553 U.S., at 638.

Turning to the case at hand, we conclude that this well-settled line of
precedent allows for only one answer: Lexmark cannot bring a patent infringement
suit against Impression Products to enforce the single-use/no-resale provision
accompanying its Return Program cartridges. Once sold, the Return Program
cartridges passed outside of the patent monopoly, and whatever rights Lexmark
retained are a matter of the contracts with its purchasers, not the patent law.



\readinghead{B}

The Federal Circuit reached a different result largely because it got off on the
wrong foot. The ``exhaustion doctrine,'' the court believed, ``must be
understood as an interpretation of'' the infringement statute, which prohibits
anyone from using or selling a patented article ``without authority'' from the
patentee. Exhaustion reflects a
default rule that a patentee's decision to sell an item ``\textit{presumptively}
grant[s] `authority' to the purchaser to use it and resell it.''
But, the Federal Circuit explained, the patentee does not have to hand over
the full ``bundle  of rights'' every time.
If the patentee expressly withholds a stick from the
bundle---perhaps by restricting the purchaser's resale rights---the buyer never
acquires that withheld authority, and the patentee may continue to enforce its
right to exclude that practice under the patent laws.

The misstep in this logic is that the exhaustion doctrine is not a presumption
about the authority that comes along with a sale; it is instead a limit on ``the
scope of the \textit{patentee's rights.}''
The right to use, sell, or import an item exists independently of the
Patent Act. What a patent adds---and grants exclusively to the patentee---is a
limited right to prevent others from engaging in those practices.
Exhaustion extinguishes that exclusionary power.
As a result, the sale
transfers the right to use, sell, or import because those are the rights that
come along with ownership, and the buyer is free and clear of an infringement
lawsuit because there is no exclusionary right left to enforce.

%The Federal Circuit also expressed concern that preventing patentees from
%reserving patent rights when they sell goods would create an artificial
%distinction between such sales and sales by licensees. Patentees, the court
%explained, often license others to make and sell their products, and may place
%restrictions on those licenses. A computer developer could, for instance,
%license a manufacturer to make its patented devices and sell them only for
%non-commercial use by individuals. If a licensee breaches the license by selling
%a computer for commercial use, the patentee can sue the licensee for
%infringement. And, in the Federal Circuit's view, our decision in
%\textit{General Talking Pictures Corp. v. Western Elec. Co.}, 304 U.S. 175, 58
%S.Ct. 849, 82 L.Ed. 1273, aff'd on reh'g, 305 U.S. 124, 59 S.Ct. 116, 83 L.Ed.
%81 (1938), established that — when a patentee grants a license ``under clearly
%stated restrictions on post-sale activities'' of those who purchase products
%from the licensee — the patentee can \textit{also} sue for infringement those
%purchasers who knowingly violate the restrictions. 816 F.3d, at 743-744. If
%patentees can employ licenses to impose post-sale restrictions on purchasers
%that are enforceable through infringement suits, the court concluded, it would
%make little sense to prevent patentees from doing so when they sell directly to
%consumers.
%
%The Federal Circuit's concern is misplaced. A patentee can impose restrictions
%on licensees because a license does not implicate the same concerns about
%restraints on alienation as a sale. Patent exhaustion reflects the principle
%that, when an item passes into commerce, it should not be shaded by a legal
%cloud on title as it moves through the marketplace. But a license is not about
%passing title to a product, it is about changing the contours of the patentee's
%monopoly: The patentee agrees not to exclude a licensee from making or selling
%the patented invention, expanding the club of authorized producers and sellers.
%See \textit{General Elec. Co.}, 272 U.S., at 489-490, 47 S.Ct. 192. Because the
%patentee is exchanging rights, not goods, it is free to relinquish only a
%portion of its bundle of patent protections.
%
%A patentee's authority to limit \textit{licensees} does not, as the Federal
%Circuit thought, mean that patentees can use licenses  to impose post-sale
%restrictions on \textit{purchasers} that are enforceable through the patent
%laws. So long as a licensee complies with the license when selling an item, the
%patentee has, in effect, authorized the sale. That licensee's sale is treated,
%for purposes of patent exhaustion, as if the patentee made the sale itself. The
%result: The sale exhausts the patentee's rights in that item. See \textit{Hobbie
%v. Jennison}, 149 U.S. 355, 362-363, 13 S.Ct. 879, 37 L.Ed. 766 (1893). A
%license may require the licensee to impose a restriction on purchasers, like the
%license limiting the computer manufacturer to selling for non-commercial use by
%individuals. But if the licensee does so — by, perhaps, having each customer
%sign a contract promising not to use the computers in business — the sale
%nonetheless exhausts all patent rights in the item sold. See \textit{Motion
%Picture Patents Co. v. Universal Film Mfg. Co.}, 243 U.S. 502, 506-507, 516, 37
%S.Ct. 416, 61 L.Ed. 871 (1917). The purchasers might not comply with the
%restriction, but the only recourse for the licensee is through contract law,
%just as if the patentee itself sold the item with a restriction.
%
%\textit{General Talking Pictures} involved a fundamentally different situation:
%There, a licensee ``knowingly ma[de] ... sales ... \textit{outside} the scope of
%its license.'' 304 U.S., at 181-182, 58 S.Ct. 849 (emphasis added). We treated
%the sale ``as if no license whatsoever had been granted'' by the patentee, which
%meant that the patentee could sue both the licensee and the purchaser — who knew
%about the breach — for infringement. \textit{General Talking Pictures Corp. v.
%Western Elec. Co.}, 305 U.S. 124, 127, 59 S.Ct. 116, 83 L.Ed. 81 (1938). This
%does not mean that patentees can use licenses to impose post-sale restraints on
%purchasers. Quite the contrary: The licensee infringed the patentee's rights
%because it did \textit{not} comply with the terms of its license, and the
%patentee could bring a patent suit against the purchaser only because the
%purchaser participated in the licensee's infringement. \textit{General Talking
%Pictures}, then, stands for the modest principle that, if a patentee has not
%given authority for a licensee to make a sale, that sale cannot exhaust the
%patentee's rights.

In sum, patent exhaustion is uniform and automatic. Once a patentee decides to
sell---whether on its own or through a licensee---that sale exhausts its patent
rights, regardless of any post-sale restrictions the patentee purports to
impose, either directly or through a license.

%\readinghead{III}
%
%Our conclusion that Lexmark exhausted its patent rights when it sold the
%domestic Return Program cartridges goes only halfway to resolving this case.
%Lexmark also sold toner cartridges abroad and sued Impression Products for
%patent infringement for ``importing [Lexmark's] invention into the United
%States.'' 35 U.S.C. \S~154(a). Lexmark contends that it may sue for infringement
%with respect to all of the imported cartridges — not just those in the Return
%Program — because a foreign sale does not trigger patent exhaustion unless the
%patentee ``expressly or implicitly transfer[s] or license[s]'' its rights. Brief
%for Respondent 36-37. The Federal Circuit agreed, but we do not. An authorized
%sale outside the United States, just as one within the United States, exhausts
%all rights under the Patent Act.
%
%This question about international exhaustion of intellectual property rights has
%also arisen in the context of copyright law. Under the ``first sale doctrine,''
%which is codified at 17 U.S.C. \S~109(a), when a copyright owner sells a
%lawfully made copy of its work, it loses the power to restrict the purchaser's
%freedom ``to sell or otherwise dispose of ... that copy.'' In  \textit{Kirtsaeng
%v. John Wiley \& Sons, Inc}\textit{.}, we held that this ``\,`first sale' [rule]
%applies to copies of a copyrighted work lawfully made [and sold] abroad.'' 568
%U.S., at 525, 133 S.Ct. 1351. We began with the text of \S~109(a), but it was
%not decisive: The language neither ``restrict[s] the scope of [the] `first sale'
%doctrine geographically,'' nor clearly embraces international exhaustion.
%\textit{Id.}, at 528-533, 133 S.Ct. 1351. What helped tip the scales for global
%exhaustion was the fact that the first sale doctrine originated in ``the common
%law's refusal to permit restraints on the alienation of chattels.''
%\textit{Id.}, at 538, 133 S.Ct. 1351. That ``common-law doctrine makes no
%geographical distinctions.'' \textit{Id.}, at 539, 133 S.Ct. 1351. The lack of
%any textual basis for distinguishing between domestic and international sales
%meant that ``a straightforward application'' of the first sale doctrine required
%the conclusion that it applies overseas. \textit{Id.}, at 540, 133 S.Ct. 1351
%(internal quotation marks omitted).
%
%Applying patent exhaustion to foreign sales is just as straightforward. Patent
%exhaustion, too, has its roots in the antipathy toward restraints on alienation,
%see \textit{supra}, at 1528 - 1533, and nothing in the text or history of the
%Patent Act shows that Congress intended to confine that borderless common law
%principle to domestic sales. In fact, Congress has not altered patent exhaustion
%at all; it remains an unwritten limit on the scope of the patentee's monopoly.
%See \textit{Astoria Fed. Sav. \& Loan Assn. v. Solimino}, 501 U.S. 104, 108, 111
%S.Ct. 2166, 115 L.Ed.2d 96 (1991) (``[W]here a common-law principle is well
%established, ... courts may take it as given that Congress has legislated with
%an expectation that the principle will apply except when a statutory purpose to
%the contrary is evident'' (internal quotation marks omitted)). And
%differentiating the patent exhaustion and copyright first sale doctrines would
%make little theoretical or practical sense: The two share a ``strong similarity
%... and identity of purpose,'' \textit{Bauer \& Cie v. O'Donnell}, 229 U.S. 1,
%13, 33 S.Ct. 616, 57 L.Ed. 1041 (1913), and many everyday products —
%``automobiles, microwaves, calculators, mobile phones, tablets, and personal
%computers'' — are subject to both patent and copyright protections, see
%\textit{Kirtsaeng}, 568 U.S., at 545, 133 S.Ct. 1351; Brief for Costco Wholesale
%Corp. et al. as \textit{Amici Curiae} 14-15. There is a ``historic kinship
%between patent law and copyright law,'' \textit{Sony Corp. of America v.
%Universal City Studios, Inc.}, 464 U.S. 417, 439, 104 S.Ct. 774, 78 L.Ed.2d 574
%(1984), and the bond between the two leaves no room for a rift on the question
%of international exhaustion.
%
%Lexmark sees the matter differently. The Patent Act, it points out, limits the
%patentee's ``right to exclude others'' from making, using, selling, or importing
%its products to acts that occur in the United States. 35 U.S.C. \S~154(a). A
%domestic sale, it argues, triggers exhaustion because the sale compensates the
%patentee for ``surrendering [those] \textit{U.S.} rights.'' Brief for Respondent
%38. A foreign sale is different: The Patent Act does not give patentees
%exclusionary powers abroad. Without those powers, a patentee selling in a
%foreign market may not be able to sell its product for the same price that it
%could in the United States, and therefore is not sure to receive ``the reward
%guaranteed by U.S. patent law.'' \textit{Id.}, at 39 (internal quotation marks
%omitted). Absent that reward, says Lexmark, there should be no exhaustion. In
%short, there is no patent exhaustion from sales abroad because there are no
%patent rights abroad to exhaust.
%
%The territorial limit on patent rights is, however, no basis for distinguishing
%copyright protections; those protections  ``do not have any extraterritorial
%operation'' either. 5 M. Nimmer \& D. Nimmer, Copyright \S~17.02, p. 17-26
%(2017). Nor does the territorial limit support the premise of Lexmark's
%argument. Exhaustion is a separate limit on the patent grant, and does not
%depend on the patentee receiving some undefined premium for selling the right to
%access the American market. A purchaser buys an item, not patent rights. And
%exhaustion is triggered by the patentee's decision to give that item up and
%receive whatever fee it decides is appropriate ``for the article and the
%invention which it embodies.'' \textit{Univis}, 316 U.S., at 251, 62 S.Ct. 1088.
%The patentee may not be able to command the same amount for its products abroad
%as it does in the United States. But the Patent Act does not guarantee a
%particular price, much less the price from selling to American consumers.
%Instead, the right to exclude just ensures that the patentee receives one reward
%— of whatever amount the patentee deems to be ``satisfactory compensation,''
%\textit{Keeler}, 157 U.S., at 661, 15 S.Ct. 738 — for every item that passes
%outside the scope of the patent monopoly.
%
%This Court has addressed international patent exhaustion in only one case,
%\textit{Boesch v. Graff}\textit{}, decided over 125 years ago. All that case
%illustrates is that a sale abroad does not exhaust a patentee's rights when the
%patentee had nothing to do with the transaction. \textit{Boesch} — from the days
%before the widespread adoption of electrical lighting — involved a retailer who
%purchased lamp burners from a manufacturer in Germany, with plans to sell them
%in the United States. The manufacturer had authority to make the burners under
%German law, but there was a hitch: Two individuals with no ties to the German
%manufacturer held the American patent to that invention. These patentees sued
%the retailer for infringement when the retailer imported the lamp burners into
%the United States, and we rejected the argument that the German manufacturer's
%sale had exhausted the American patentees' rights. The German manufacturer had
%no permission to sell in the United States from the American patentees, and the
%American patentees had not exhausted their patent rights in the products because
%they had not sold them to anyone, so ``purchasers from [the German manufacturer]
%could not be thereby authorized to sell the articles in the United States.'' 133
%U.S. 697, 703, 10 S.Ct. 378, 33 L.Ed. 787 (1890).
%
%Our decision did not, as Lexmark contends, exempt all foreign sales from patent
%exhaustion. See Brief for Respondent 44-45. Rather, it reaffirmed the basic
%premise that only the patentee can decide whether to make a sale that exhausts
%its patent rights in an item. The American patentees did not do so with respect
%to the German products, so the German sales did not exhaust their rights.
%
%Finally, the United States, as an \textit{amicus}, advocates what it views as a
%middle-ground position: that ``a foreign sale authorized by the U.S. patentee
%exhausts U.S. patent rights unless those rights are expressly reserved.'' Brief
%for United States 7-8. Its position is largely based on policy rather than
%principle. The Government thinks that an overseas ``buyer's legitimate
%expectation'' is that a ``sale conveys all of the seller's interest in the
%patented article,'' so the presumption should be that a foreign sale triggers
%exhaustion. \textit{Id.}, at 32-33. But, at the same time, ``lower courts long
%ago coalesced around'' the rule that ``a patentee's express reservation of U.S.
%patent rights at the time of a foreign sale will be given effect,'' so that
%option should remain open to the patentee. \textit{Id.}, at 22 (emphasis
%deleted).
%
%The Government has little more than ``long ago'' on its side. In the 1890s, two
%circuit courts — in cases involving the same company — did hold that patentees
%may use express restrictions to reserve their patent rights in connection with
%foreign sales. See \textit{Dickerson v. Tinling}, 84 F. 192, 194-195 (C.A.8
%1897); \textit{Dickerson v. Matheson}, 57 F. 524, 527 (C.A.2 1893). But no
%``coalesc[ing]'' ever took place: Over the following hundred-plus years, only a
%smattering of lower court decisions mentioned this express-reservation rule for
%foreign sales. See, \textit{e.g.}, \textit{Sanofi, S.A. v. Med-Tech Veterinarian
%Prods., Inc.}, 565 F.Supp. 931, 938 (D.N.J.1983). And in 2001, the Federal
%Circuit adopted its blanket rule that foreign sales do not trigger exhaustion,
%even if the patentee fails to expressly reserve its rights. \textit{Jazz Photo},
%264 F.3d, at 1105. These sparse and inconsistent decisions provide no basis for
%any expectation, let alone a settled one, that patentees can reserve patent
%rights when they sell abroad.
%
%The theory behind the Government's express-reservation rule also wrongly focuses
%on the likely expectations of the patentee and purchaser during a sale.
%Exhaustion does not arise because of the parties' expectations about how sales
%transfer patent rights. More is at stake when it comes to patents than simply
%the dealings between the parties, which can be addressed through contract law.
%Instead, exhaustion occurs because, in a sale, the patentee elects to give up
%title to an item in exchange for payment. Allowing patent rights to stick
%remora-like to that item as it flows through the market would violate the
%principle against restraints on alienation. Exhaustion does not depend on
%whether the patentee receives a premium for selling in the United States, or the
%type of rights that buyers expect to receive. As a result, restrictions and
%location are irrelevant; what matters is the patentee's decision to make a sale.
%
%
%
%\readinghead{* * *}
%
%The judgment of the United States Court of Appeals for the Federal Circuit is
%reversed, and the case is remanded for further proceedings consistent with this
%opinion.
%
%\textit{It is so ordered.}
%
%Justice GORSUCH took no part in the consideration or decision of this case.
%
%\vskip\baselineskip
%
%\textbf{Justice GINSBURG, concurring in part and dissenting in part.}
%
%I concur in the Court's holding regarding domestic exhaustion — a patentee who
%sells a product with an express restriction on reuse or resale may not enforce
%that restriction through an infringement lawsuit, because the U.S. sale exhausts
%the U.S. patent rights in the product sold. See \textit{ante}, at 1531 - 1536. I
%dissent, however, from the Court's holding on international exhaustion. A
%foreign sale, I would hold, does not exhaust a U.S. inventor's U.S. patent
%rights.
%
%Patent law is territorial. When an inventor receives a U.S. patent, that patent
%provides no protection abroad. See \textit{Deepsouth Packing Co. v. Laitram
%Corp.}, 406 U.S. 518, 531, 92 S.Ct. 1700, 32 L.Ed.2d 273 (1972) (``Our patent
%system makes no claim to extraterritorial effect.''). See also 35 U.S.C.
%\S~271(a) (establishing liability for acts of patent infringement ``within the
%United States'' and for ``import[ation] into the United States [of] any patented
%invention''). A U.S. patentee must apply to each country in which she seeks the
%exclusive right to sell her invention. \textit{Microsoft Corp. v. AT \& T
%Corp.}, 550 U.S. 437, 456, 127 S.Ct. 1746, 167 L.Ed.2d 737 (2007) (``[F]oreign
%law alone, not United States law, currently governs the manufacture and sale of
%components of patented inventions in foreign countries.''). See also Convention
%at Brussels, An Additional Act  Modifying the Paris Convention for the
%Protection of Industrial Property of Mar. 20, 1883, Dec. 14, 1900, Art. I, 32
%Stat. 1940 (``Patents applied for in the different contracting States ... shall
%be independent of the patents obtained for the same invention in the other
%States.''). And patent laws vary by country; each country's laws ``may embody
%different policy judgments about the relative rights of inventors, competitors,
%and the public in patented inventions.'' \textit{Microsoft}, 550 U.S., at 455,
%127 S.Ct. 1746 (internal quotation marks omitted).
%
%Because a sale abroad operates independently of the U.S. patent system, it makes
%little sense to say that such a sale exhausts an inventor's U.S. patent rights.
%U.S. patent protection accompanies none of a U.S. patentee's sales abroad — a
%competitor could sell the same patented product abroad with no U.S.-patent-law
%consequence. Accordingly, the foreign sale should not diminish the protections
%of U.S. law in the United States.
%
%The majority disagrees, in part because this Court decided, in \textit{Kirtsaeng
%v. John Wiley \& Sons, Inc.}, 568 U.S. 519, 525, 133 S.Ct. 1351, 185 L.Ed.2d 392
%(2013), that a foreign sale exhausts U.S. \textit{copyright} protections.
%Copyright and patent exhaustion, the majority states, ``share a strong
%similarity.'' \textit{Ante}, at 1536 (internal quotation marks omitted). I
%dissented from our decision in \textit{Kirtsaeng} and adhere to the view that a
%foreign sale should not exhaust U.S. copyright protections. See 568 U.S., at
%557, 133 S.Ct. 1351.
%
%But even if I subscribed to \textit{Kirtsaeng}'s reasoning with respect to
%copyright, that decision should bear little weight in the patent context.
%Although there may be a ``historical kinship'' between patent law and copyright
%law, \textit{Sony Corp. of America v. Universal City Studios, Inc.}, 464 U.S.
%417, 439, 104 S.Ct. 774, 78 L.Ed.2d 574 (1984), the two ``are not identical
%twins,'' \textit{id.}, at 439, n. 19, 104 S.Ct. 774. The Patent Act contains no
%analogue to 17 U.S.C. \S~109(a), the Copyright Act first-sale provision analyzed
%in \textit{Kirtsaeng.} See \textit{ante}, at 1535 - 1536. More importantly,
%copyright protections, unlike patent protections, are harmonized across
%countries. Under the Berne Convention, which 174 countries have
%joined,\readingfootnote{2}{See WIPO-Administered Treaties: Contracting Parties:
%Berne Convention, www.wipo. int/treaties/en/ShowResults.jsp?lang=en\&
%treaty\_id=5 (as last visited May 25, 2017).} members ``agree to treat authors
%from other member countries as well as they treat their own.'' \textit{Golan v.
%Holder}, 565 U.S. 302, 308, 132 S.Ct. 873, 181 L.Ed.2d 835 (2012) (citing Berne
%Convention for the Protection of Literary and Artistic Works, Sept. 9, 1886, as
%revised at Stockholm on July 14, 1967, Arts. 1, 5(1), 828 U.N.T.S. 225,
%231-233). The copyright protections one receives abroad are thus likely to be
%similar to those received at home, even if provided under each country's
%separate copyright regime.
%
%For these reasons, I would affirm the Federal Circuit's judgment with respect to
%foreign exhaustion.
%
