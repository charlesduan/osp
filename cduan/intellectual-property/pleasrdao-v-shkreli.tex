\reading{PleasrDAO v. Shkreli}
\readingcite{No. 24-cv-4126 (E.D.N.Y. Sept. 25, 2025)}



\opinion \textsc{Pamela K. Chen}, District Judge.

In 2015, Defendant Martin Shkreli purchased the
most expensive musical work ever sold: the only copy of Wu-Tang Clan's album
\textit{Once Upon a Time in Shaolin}.
His ownership of the Album, however, was short-lived. In 2017, Shkreli,
a former pharmaceutical executive, was convicted on federal securities fraud
charges and was subsequently sentenced to seven years in prison and forced to
forfeit approximately \$7.4 million in assets including
the Album. Plaintiff PleasrDAO
thereafter purchased the Album, and now alleges that Shkreli
unlawfully retained and distributed copies of it.
Defendant moves to dismiss the
Complaint under Federal Rule of Civil Procedure 12.\ldots





\readinghead{I. Factual Background\readingfootnote{1}{The facts are derived from
Plaintiff's Complaint [and documents referenced therein].}}


\readinghead{A. The Wu-Tang Clan Album: \textit{Once Upon a Time in Shaolin}}

Between 2007 and 2013, Wu-Tang Clan, a world-famous hip-hop group, recorded
\textit{Once Upon a Time in Shaolin}. The Album has
31 tracks and includes guest appearances from notable musicians, actors, and
athletes. Rather than a conventional commercial album
release, Wu-Tang Clan produced only one hard copy of the Album, which has never
been publicly released.\ldots

In 2015, the Producers sold the Album to Shkreli for \$2 million\ldots.
The sale was executed in the Original Purchase
Agreement (``OPA'') dated September 3, 2015.\ldots
The OPA
bound Shkreli by several usage restrictions for a period of 88 years.
It states that the buyer ``may duplicate or replicate the
Work for private use, but shall not duplicate, replicate,
and/or exploit the Work for any commercial or other non-commercial purposes''
other than certain permitted uses limited to: ``the private or public exhibition
or playing of the Work, with or without charge, in locations such as Buyer's
home, museums, art galleries, restaurants, bars, exhibition spaces, or other
similar spaces not customarily used as venues for large musical concerts, as
well as the advertising and/or promotion of such exhibition or playing of the
Work.''\ldots



\readinghead{B. Shkreli's Conviction and Forfeiture Order}

In 2017, Shkreli was convicted by a federal jury in the United States District
Court for the Eastern District of New York on two counts of securities fraud and
one count of conspiracy to commit securities fraud.\ldots
The Forfeiture Order [resulting from the conviction] required
Shkreli to turn over to the United States his interest in certain assets
including ``the album `Once Upon a
Time in Shaolin' by the Wu-Tang Clan.''\ldots

\readinghead{C. PleasrDAO's Purchase of the Album}

PleasrDAO is an exempted foundation company established in the Cayman
Islands ``that
collects and publicly displays culturally significant media and materials with
the intent of creating ecosystem experiences that encourage participation and
interaction throughout the United States and other countries.''
In July 2021, PleasrDAO purchased the Album---specifically, ``the physical asset
and exclusive right to play the audio tracks''---for approximately
\$4 million. The Complaint does not identify from whom
Plaintiff purchased the Album; rather, the Complaint refers to and attaches an
Asset Purchase Agreement dated July 19, 2021, executed by the
United States Marshals Service on behalf of the United States
memorializing the United States' sale of the Album to undisclosed buyers for an
undisclosed amount. In January 2024, PleasrDAO purchased
the ``copyrights in and exclusive right to exploit the recordings for
approximately \$750,000.''



\readinghead{D. Shkreli's Retention and Distribution of the Album}

In May 2022, Shkreli was released from prison and began serving a three-year
term of supervised release. Since his release from
prison, Shkreli has frequently participated in online ``live stream'' activity
on various social media platforms. During a live stream
on June 18, 2022, Shkreli admitted that he had played the Album for his
followers, stating, ``Yeah, that's the Wu-Tang album for all you crazy streamer
people.'' During a live stream on YouTube on June 22, 2022,
Shkreli was asked if he had still retained a copy of the Album and responded,
``I do. I was playing it on YouTube the other night even though somebody paid
\$4 million for it.'' [He said other things to similar effect.]\ldots




\readinghead{II. Trade Secret Claims}

Plaintiff alleges that Defendant
misappropriated Plaintiff's trade secret---the Album---in violation of [the
federal trade secrets statute] the DTSA, and New York common law. Defendant
seeks to dismiss these claims on the ground that Plaintiff has failed to plead
that the Album is a ``trade secret.'' The parties dispute whether the Album is
sufficiently ``secret'' to warrant protection.

The elements required to establish misappropriation of trade secrets under the
DTSA and New York law are ``fundamentally the same''\ldots.
``To succeed on a claim for the
misappropriation of trade secrets under New York law, a party must
demonstrate,'' first, ``that it possessed a trade secret, and'' second, ``that
the defendants used that trade secret in breach of an agreement, confidential
relationship or duty, or as a result of discovery by improper means.''

``The DTSA defines `trade secret' to include `all forms and types of financial,
business, scientific, technical, economic, or engineering information, including
patterns, plans, compilations, program devices, formulas, designs, prototypes,
methods, techniques, processes, procedures, programs, or codes,' so long as: (1)
`the owner thereof has taken reasonable measures to keep such information
secret'; and (2) `the information derives independent economic value\ldots from
not being generally known to, and not being readily ascertainable through proper
means by, another person who can obtain economic value from the disclosure or
use of the information.'\,''

New York courts typically consider the following six factors in determining
whether information qualifies as a trade secret under either the DTSA or state
law:
\begin{quote} (1) The extent to which the information is known outside of the
business; (2) the extent to which it is known by employees and others involved
in the business; (3) the extent of measures taken by the business to guard the
secrecy of the information; (4) the value of the information to the business and
to its competitors; (5) the amount of effort or money expended by the business
in developing the information; (6) the ease or difficulty with which the
information could be properly acquired or duplicated by others. \end{quote}
\textit{Id.} (cleaned up) (quoting \textit{Integrated Cash Mgmt. Servs., Inc. v.
Digit. Transactions, Inc.}, 920 F.2d 171, 173 (2d Cir. 1990)) (hereinafter the
``\textit{Integrated Cash} factors'').\ldots

First, though neither party addresses this, the Court notes that this is a
somewhat unusual application of the trade secret doctrine or statutes. Plaintiff
alleges that the trade secret at issue is the unreleased Album, comprising of
the only copy of \textit{Once Upon a Time in Shaolin}, including its data and
files. However, the Album does not fit squarely within a
category of business information or data that is traditionally protectable as
trade secrets, such as an internal customer list.
Nor does it clearly resemble a secret
recipe or formula used to make a product, such as the formula for Coca-Cola.
The Album's data and files arguably fall somewhere
between information ``used in \textit{running} [Plaintiff's] business'' and
information that is ``its \textit{product}.'' Still, ``[t]rade
secrets include \textit{all forms and types} of business information'' so long
as the criteria described above pertaining to ``reasonable measures'' to
maintain secrecy and the ``independent economic value'' from that secrecy are
met. Given this broad
definition and Defendant's seeming concession that the trade secret statutes
might apply to this type of information, the Court will apply the
\textit{Integrated Cash} factors to determine whether Plaintiff has sufficiently
pleaded that the Album is a ``trade secret'' under the DTSA and New York law.


[The court first determined that the Album was not known inside or outside the
business, and that PleasrDAO had plausibly alleged that it had taken reasonable
measures to guard the Album's secrecy.]


\readinghead{C. The Value of the Album to Plaintiff and its Competitors}

The fourth relevant factor is ``the value of the information to [the business]
and [its] competitors.''
Under the DTSA, for information to be a trade secret it must ``derive[]
independent economic value, actual or potential, from not being generally known
to\ldots another person who can obtain economic value from the disclosure or use
of the information.'' Plaintiff argues that the factual
allegations in the Complaint are sufficient to show that the Album's data and
files are of significant value to Plaintiff's business, and that ``the value
rests on it having \textit{not} been heard by the public.''
Though the Court agrees with Plaintiff on this factor, it cannot be
understated that the application of trade secret doctrine to the unique facts of
this case is unchartered territory.

\ldots Plaintiff must adequately allege that it is the
\textit{secrecy} of the Album that makes the information so valuable to its
business. Here, Wu-Tang Clan ``produced only one copy of'' the Album, which
``has never been publicly released.'' The OPA bound
Defendant and any subsequent purchasers of the Album to certain confidentiality
and usage restrictions for a period of 88 years. Though it permits duplicating
or replicating the contents of the
Album ``for private use,'' it prohibits duplicating, replicating, or exploiting
the Album for ``any commercial or other non-commercial purposes'' other than
certain permitted uses, including ``the private or public exhibition or playing
of the Work,'' in spaces ``not customarily used as venues for large musical
concerts.'' Plaintiff has alleged that the OPA
expresses the Producers' intention to create only one copy of the Album ``as a
protest to what they saw as the devaluation of music in the digital era,'' and
``to keep ownership of the Album in one person's hands at a time.'' In other
words, the secret and exclusive nature of the
Album is a large part of its intrinsic value.

On the other hand, some courts have found that trade secret protection does not,
or is unlikely to, extend to certain unreleased musical works. In
\textit{Paisley Park Enters., Inc. v. Boxill}, for example, the plaintiffs
sought trade secret protection for five unreleased recordings of Prince songs.
The District of Minnesota held
that the plaintiffs' trade secrets claim under Minnesota law was unlikely to
succeed on the merits because the ``only economic value of the recordings
derives from the right to sell the recordings to the public,'' and thus the
plaintiffs could not ``realize any independent economic value by keeping the
contents of the recordings secret.'' Similarly, in
\textit{Anderson v. Jackson}, the Central District of California held that an
unreleased Janet Jackson song was not a trade secret under California law
because, among other reasons, the ``[p]laintiffs fail[ed] to identify anything
in or about the song that derived `independent economic value' by virtue of its
secrecy.''

It certainly can be argued that the difference between the Album and the musical
works at issue in \textit{Paisley Park} and \textit{Jackson} is one of degree,
rather than a difference in kind. The Court accordingly has considered the
question: if the contents of the Album \textit{always} remain secret, will it
still hold economic value? Arguably, the value of the Album to PleasrDAO's
business is only realized by sharing the Album's contents with others, at least
in some form. But unlike \textit{Paisley Park}, where the plaintiffs ``[did] not
derive any competitive advantage in the marketplace from the secrecy'' of the
musical works because ``[n]o other artist or record company could take market
share from [the plaintiffs] by discovering the contents of the disputed
recordings,'' PleasrDAO's business model is unique.
PleasrDAO ``collects\ldots culturally significant media and materials'' to
create ``ecosystem experiences.'' The independent
economic value of the Album comes from Plaintiff's ability to exploit its
exclusivity to create an ``experience'' that its competitors cannot, rather than
from a public commercial release or from traditional forms of music distribution
that the courts in \textit{Paisley Park} and \textit{Jackson} seemingly
considered. As explained \textit{supra}, the Album is still subject to
significant usage and distribution restrictions under the OPA, which preserve
the Wu-Tang Clan Producers' intention ``to keep ownership of the Album in one
person's hands at a time,'' ``as a protest to what they saw as the devaluation
of music in the digital era.'' For these
reasons, the Court finds that the unique facts alleged in the Complaint
sufficiently distinguish the Album from typical recorded musical works that
might not otherwise be afforded trade secret protection. On these facts, at the
pleading stage, Plaintiff has adequately alleged that the Album derives
independent economic value from its secrecy.


[Regarding the last \emph{Integrated Cash} factor, the court held that the OPA
limited Shkreli from easily duplicating the Album.]

Ultimately, whether the Album, and the information contained therein, is a trade
secret is a question of fact.
Given the weight of the \textit{Integrated Cash} factors and the unique facts of
this case, the Court finds that Plaintiff has adequately pleaded a ``trade
secret'' under the DTSA and New York law. Accordingly, Defendant's motion to
dismiss Counts II and III for failure to plead a ``secret'' is denied.

