\item The Supreme Court calls patents a ``franchise,'' which it agrees are a
type of property right but not one that get private-rights treatment like land.
What's the difference? In what ways do you expect the law to treat
``franchises'' differently from ``property'' beyond the public/private rights
distinction here?

\item
\having{us-v-turoff}{In \emph{Turoff}, above, the court}{We will later read
\emph{United States v. Turoff}, in which a court considered whether
misappropriation of taxicab medallions (authorizing the bearer to operate a
taxicab in New York City) was ``property'' for purposes of federal mail fraud.
The court}{In \emph{United States v. Turoff}, 701 F. Supp. 981 (E.D.N.Y. 1988),
a court considered whether misappropriation of taxicab medallions (authorizing
the bearer to operate a taxicab in New York City) was property for purposes of
the federal mail fraud statute. The court} deemed a government-issued taxicab
medallion to be a franchise, and thus property. In \emph{Oil States}, though,
the Supreme Court deemed a government-issued patent to be a franchise, and thus
not property. Are these decisions just inconsistent? Here are some possible
explanations:
\begin{itemize}
\item The courts meant different things when they said ``franchise'' between the
two cases.
\item Something can be property for one statute, and not property for another.
\item Taxicab medallions and patents grant different rights, and therefore
receive different legal treatment even if both are ``property'' (or
``franchises'').
\item ``Property'' is just a legal conclusion. Courts first determine whether
the law gives rights to the holder of a thing, and then calls those rights
``property'' where the rights exist.
\end{itemize}
How satisfying are these explanations to you?

\item What else might you call a ``franchise''? Here are some ideas:
\begin{itemize}
\item Approval by the U.S. Food and Drug Administration to sell a pharmaceutical
on the market.

\item \emph{Exclusive} FDA approval to sell a pharmaceutical on the market, such
that the FDA is prohibited from approving any other sellers for a period of
time.

\item A health inspection grade of ``A'' from the local health inspector.

\item A tax credit for installing solar panels.

\item A research grant from the National Institutes of Health.
\end{itemize}
What difference does it make if the above are property, franchises, or simply
government benefits?

\item Consider the real property analogy to this case. A politically appointed
city administrator decides that your title to your house is defective, and takes
away your house with no compensation. On the assumption that your title
\emph{is} defective (for example, the person who sold you the land didn't
actually own it), how do you feel? Would you feel differently if a court had
adjudicated your house's title (with the same result)? What systems should be in
place to prevent this result?

Now translate this back to patents. Do you expect that patent holders feel
similarly when the PTO cancels their patents? (Maybe you're a patent holder, how
would you feel?) Is there something different about patents?

\item Here's one possible difference. In the early 2000s, there was a wave of
patent lawsuits, often over basic technologies like scanning documents to email
or displaying electronic restaurant menus. The defendants were often small
businesses like restaurants and florists, who could not afford to litigate these
patents, and often paid nuisance settlements. Larger companies could afford
litigation and in fact invalidated many of these patents, but only after
millions of dollars of litigation fees. Congress created inter partes review in
large part to provide a lower-cost, more efficient pathway for dealing with
this activity, disparagingly called ``patent trolling.''

Does this background affect whether you think patents should be property? Can
one engage in land trolling?

\item Here's another possible difference. If you write a computer program, build
a machine, or do any other sort of activity, you may infringe a patent even
without knowing that the patent exists. As patent lawyers like to say,
independent invention is no defense. By contrast, generally when you're walking
on someone else's land, you know it. Does this make a difference in whether
patents are property?

\expecting{takings}

\item Justice Thomas's opinion ends with a slew of caveats. In particular, note
the discussion about ``property for purposes of the Due Process Clause or the
Takings Clause.'' We'll consider these constitutional issues in greater detail
in our unit on Takings, but for now, just consider that something like a patent
can be property for one legal doctrine but not for others. Does that strike you
as odd? What else might be property in some situations but not others? What does
that tell you about the concept of ``property'' as a unified whole?
\having{hohfeld-merrill-smith-qs}{Does it inform the debate between Hohfeld's
bundle-of-rights, and Merrill and Smith's property as law of things?}{}{}

\expected{intro-property-rationales}

\item Justice Gorsuch draws a distinction between early English patents as
``feudal favors'' on the one hand, and U.S. patents that ``the grantee is
entitled to\ldots as a matter of right.'' Putting aside the legal accuracy of
that statement,\footnote{The Constitution permits Congress
to grant patents, but is generally not understood to require Congress to grant
them as a matter of right.} do inventors have a natural right in their
inventions? Which of the justifications for property rights strike you as
applicable or inapplicable to patents? What are the pros and cons of treating
patents as a ``matter of grace and favor'' from the government?

For that matter, is all property a ``matter of grace and favor'' from the
government?

