\item As an initial matter, observe a consequence of calling patents (or
anything else) ``property.'' The government itself is required to respect
multiple constitutional rights of property owners. This gives the ``property''
label even further significance, and also makes property law uniquely a subject
of both private and public law.



\item The majority opinion describes the patent grant as something that ``takes
from the public rights of immense value, and bestows them upon the patentee.''
How does a patent take rights away from the public? Consider two cases: (1) A
patent is granted on a wholly new invention, that no one in the public has ever
seen before. (2) A technology is well-known already, but a patent is erroneously
issued on that technology anyway.

Can property ownership of land take ``rights of immense value'' away from the
public? If a country gives a private space entrepreneur exclusive rights to
colonize Mars, has the public lost something?



\item
\having{us-v-turoff}{In \emph{Turoff}, above, the court}{We will later read
\emph{United States v. Turoff}, in which a court considered whether
misappropriation of taxicab medallions (authorizing the bearer to operate a
taxicab in New York City) was ``property'' for purposes of federal mail fraud.
The court}{In \emph{United States v. Turoff}, 701 F. Supp. 981 (E.D.N.Y. 1988),
a court considered whether misappropriation of taxicab medallions (authorizing
the bearer to operate a taxicab in New York City) was property for purposes of
the federal mail fraud statute. The court} deemed a government-issued taxicab
medallion to be a franchise, and thus property. In \emph{Oil States}, though,
the Supreme Court deemed a government-issued patent to be a franchise, and thus
not property. Are these decisions inconsistent? Here are some possible
explanations:
\begin{itemize}
\item The courts meant different things when they said ``franchise'' between the
two cases.
\item Something can be property for one statute, and not property for another.
\item Taxicab medallions and patents grant different rights, and therefore
receive different legal treatment even if both are ``property'' (or
``franchises'').
\item ``Property'' is just a legal conclusion. Courts first determine whether
the law gives rights to the holder of a thing, and then calls those rights
``property'' where the rights exist.
\end{itemize}
How satisfying are these explanations to you?

\expected{intro-property-rationales}

\item Justice Gorsuch's dissent appears to rely on a Lockean natural-rights
labor desert theory of patents---that inventors are entitled to patents ``as a
matter of right,'' not ``as a matter of grace and favor,'' in view of the ``much
hard work and no little investment'' that inventors put in. By contrast, Justice
Thomas takes a more utilitarian, total-welfare approach, focusing on ``the
public's paramount interest in seeing that patent monopolies are kept within
their legitimate scope.'' Which perspective strikes you as more appealing with
regard to inventions and new technology? What about with respect to land and
personal property?


\item Patents are an incentive for inventors to invent, but they are not the
only possible incentive. The most obvious alternative is simply to reward
inventors with prizes---cash awards, medals, research grants, and so on. In many
respects, prizes are far superior to patents, because they enable the entire
public to use the invention immediately, avoiding a 20-year period of monopoly
pricing. But there is a tremendous difficulty with prizes---can you think of it?
Think back in particular to the discussion of efficient allocation (the flute
problem, on page~\pageref{flutes-efficient-allocation}). Does this give you an
additional perspective on whether patents should be treated like property?


