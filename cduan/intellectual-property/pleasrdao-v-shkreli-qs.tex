\item To reiterate what the court said: \emph{This is not a normal trade secrecy
case}. (It's not a normal case of any kind.) Ordinarily, trade secrets are
unpatented inventions, secret recipes, technical drawings, business data, or the
like, and protection is asserted to prevent competitors from exploiting such
information that they obtained illicitly. Intellectual property law is often an
area that rewards creative lawyering of the sort that the plaintiff's attorneys
did here.

\item What exactly was the ``thing'' that PleasrDAO obtained, that enabled it to
bring this trade secrets claim? Whose hands did it pass through to get there?
This activity of tracing out who owns what will come up throughout property law.

\defjrnart{hrdy-value-secrecy}{
Camilla A. Hrdy, The Value in Secrecy, 91 Fordham Law Review 557 (2022)
}
\defcase{moussoris-v-microsoft}{
parties={Moussoris v. Microsoft Corp.},
court=W.D. Wash.,
docket=No. 15-cv-1483,
url=https://storage.courtlistener.com/recap/gov.uscourts.wawd.220713/gov.uscourts.wawd.220713.351.0.pdf,
date=feb 16 2018,
}

\defjrnart{morten-publicizing-corporate}{
Christopher J. Morten, Publicizing Corporate Secrets, 171 University of
Pennsylvania Law Review 1319 (2023)
}


\item A large part of trade secrecy protection turns on whether information
``derives independent economic value'' from its secrecy. What exactly is
``economic value''? \sentence{see hrdy-value-secrecy}. Does a company's employee
diversity data derive economic value from secrecy, because revelation of the
numbers could be embarrassing to the company? \sentence{see hrdy-value-secrecy
at 600-601 (discussing: moussoris-v-microsoft at 25)}. What about information
showing that a drug product is contaminated and dangerous? \sentence{see
morten-publicizing-corporate}.
