\reading[Oil States v. Greene's Energy]{Oil States Energy Services, LLC v.
Greene's Energy Group, LLC}
\readingcite{138 S. Ct. 1365 (2018)}

\opinion Justice \textsc{Thomas} delivered the opinion of the Court.

[The constitutional issue is not the focus of this excerpt of the case, but here
is a brief summary. To get a patent, an inventor files an application with the
U.S. Patent and Trademark Office, describing the invention. The agency examines
the application, and if it determines that the invention is patentable, it
issues a patent to the inventor. Of course, the Patent Office makes mistakes
during examination, so what happens if it issues a patent wrongly? Typically a
court would decide this, when the patent holder sues someone for infringement,
and the accused infringer argues to the court that the patent is invalid.

In 2011, Congress enacted the America Invents Act, creating an administrative
proceeding in which an arm of the Patent Office, called the ``Patent Trial and
Appeal Board,'' had the power to reconsider granted patents and effectively
revoke those that it deemed wrongly granted. The purpose of this new procedure
was to create a faster, cheaper way of reviewing the correctness of issued
patents, in part by avoiding the costliness of federal court litigation.

This case is a constitutional challenge to that new proceeding, called ``inter
partes review.'' The basic outline of the challenge is as follows. Article III
of the Constitution, which vests judicial power in the courts, implies that
Congress and the executive branch cannot conduct adjudication. Standing alone,
that makes inter partes review unconstitutional---along with a huge swath of
federal agency powers. But the Supreme Court has recognized an exception:
Administrative agencies are constitutionally allowed to adjudicate ``public
rights.''

What counts as a ``public right''? As Justice Thomas acknowledges, Supreme Court
``precedents applying the public-rights doctrine have not been entirely
consistent.'' Very roughly speaking, though, public rights are meant to be
``new'' benefits created by the government, like social security benefits.
``Private'' rights, by contrast, are traditional rights from common law. This
hopefully makes some intuitive sense. Congress is under no obligation to create
programs like social security, so when it does create them, it is free to create
them with strings attached including administrative adjudication. But it would
be really weird for Congress to create the ``federal tort agency'' to decide
private lawsuits---those cases seem like they belong in courts.

Private property is, of course, the subject of traditional common law. So if
revoking a patent is like revoking private property, then it must be adjudicated
in court, and inter partes review is unconstitutional. But if revoking a patent
is more like denying a government benefit, then there is nothing wrong with an
agency deciding patent revocation. Which is it?]


\readinghead{1}

This Court has long recognized that the grant of a patent is a ``matter
involving public rights.'' \textit{United States v. Duell}, 172 U.S. 576,
582-583 (1899) (quoting \textit{Murray's Lessee v. Hoboken Land \& Improvement
Co.}, 18 How. 272 (1856)). It has the key features to fall within this Court's
longstanding formulation of the public-rights doctrine.

\textit{Ab initio}, the grant of a patent involves a matter arising between the
government and others. As this Court has long recognized, the grant of a patent
is a matter between the public, who are the grantors, and\ldots the patentee. By
issuing patents, the PTO takes from the public rights of immense value, and
bestows them upon the patentee. Specifically, patents are ``public franchises''
that the Government grants ``to the inventors of new and useful improvements.''
The franchise gives the patent owner the right to exclude
others from making, using, offering for sale, or selling the invention
throughout the United States. That right did not exist at common law. Rather, it
is a creature of statute law.\ldots



\readinghead{2}

Inter partes review involves the same basic matter as the grant of a patent. So
it, too, falls on the public-rights side of the line.

Inter partes review is a second look at an earlier administrative grant of a
patent. The Board considers the same statutory requirements that the PTO
considered when granting the patent. Those statutory requirements prevent the
``issuance of patents whose effects are to remove existent knowledge from the
public domain.'' \textit{Graham v. John Deere Co. of Kansas City}, 383 U.S. 1, 6
(1966). So, like the PTO's initial review, the Board's inter partes review
protects the public's paramount interest in seeing that patent monopolies are
kept within their legitimate scope. Thus, inter partes review involves the same
interests as the determination to grant a patent in the first instance.

The primary distinction between inter partes review and the initial grant of a
patent is that inter partes review occurs \textit{after} the patent has issued.
But that distinction does not make a difference here. Patent claims\edfootnote{A
``claim'' is the legally operative part of a patent, which specifies the class
of inventions that the patent covers. Because patents can contain more than one
claim, each giving rise to independent legal rights, courts that are being very
precise will refer to determinations about ``patent claims'' rather than
``patents'' overall.} are granted subject to the qualification that the PTO has
the authority to reexamine---and perhaps cancel---a patent claim in an inter
partes review.

This Court has recognized that franchises can be qualified in this manner. For
example, Congress can grant a franchise that permits a company to erect a toll
bridge, but qualify the grant by reserving its authority to revoke or amend the
franchise. See, \textit{e.g., }\textit{Louisville Bridge Co. v. United States},
242 U.S. 409, 421, (1917) (collecting cases). Even after the bridge is built,
the Government can exercise its reserved authority through legislation or an
administrative proceeding. See, \textit{e.g., id.}, at 420-421. The same is true
for franchises that permit companies to build railroads or telegraph lines. See,
\textit{e.g.}, \textit{United States v. Union Pacific R. Co.}, 160 U.S. 1, 24-25
(1895).

Thus, the public-rights doctrine covers the matter resolved in inter partes
review. The Constitution does not prohibit the Board from resolving it outside
of an Article III court.


\readinghead{B}

Oil States challenges this conclusion, citing three decisions that recognize
patent rights as the ``private property of the patentee.'' \textit{United States
v.~American Bell Telephone Co.}, 128 U.S. 315, 370 (1888); see also
\textit{McCormick Harvesting Machine Co. v. Aultman}, 169 U.S. 606, 609 (1898)
(``[A granted patent] has become the property of the patentee''); \textit{Brown
v. Duchesne}, 19 How. 183 (1857) (``[T]he rights of a party
under a patent are his private property''). But those cases do not contradict
our conclusion.

Patents convey only a specific form of property right---a public franchise.
And patents are ``entitled to
protection as any other property, \textit{consisting of a franchise.}''
\textit{Seymour}, 11 Wall. at 533 (emphasis added). As a public franchise, a
patent can confer only the rights that the statute prescribes.
It is noteworthy that one of
the precedents cited by Oil States acknowledges that the patentee's rights are
``derived altogether'' from statutes, ``are to be regulated and measured by
these laws, and cannot go beyond them.'' \textit{Brown, supra}, at
195.\readingfootnote{3}{This Court has also recognized this dynamic for
state-issued
franchises. For instance, States often reserve the right to alter or revoke a
corporate charter either in the act of incorporation or in some general law of
the State which was in operation at the time the charter was granted.
That reservation remains effective even after the corporation comes into
existence, and such alterations do not offend the Contracts Clause of Article I,
\S~10.}

One such regulation is inter partes review.
The Patent Act provides that, subject to the
provisions of this title, patents shall have the attributes of personal
property. This provision qualifies any property rights that
a patent owner has in an issued patent, subjecting them to the express
provisions of the Patent Act. Those provisions
include inter partes review.

Nor do the precedents that Oil States cites foreclose the kind of
post-issuance administrative review that Congress has authorized here. To be
sure, two of the cases make broad declarations that ``[t]he only authority
competent to set a patent aside, or to annul it, or to correct it for any reason
whatever, is vested in the courts of the United States, and not in the
department which issued the patent.'' \textit{McCormick Harvesting Machine Co.,
supra}, at 609; accord, \textit{American Bell Telephone Co.}, 128
U.S., at 364. But those cases were decided under the Patent Act of
1870.
That version of the Patent Act did not
include any provision for post-issuance administrative review. Those precedents,
then, are best read as a description of the statutory scheme that existed at
that time. They do not resolve Congress' authority under the Constitution to
establish a different scheme.\readingfootnote{4}{The dissent points to
\textit{McCormick}'s statement that the Patent Office Commissioner could not
invalidate the patent at issue because it would ``deprive the applicant of his
property without due process of law, and would be in fact an invasion of the
judicial branch.'' But that statement followed naturally from the Court's
determination that, under the Patent Act of 1870, the Commissioner ``was
\textit{functus officio}'' and ``had no power to revoke, cancel, or annul'' the
patent at issue.

Nor is it significant that the \textit{McCormick} Court ``equated invention
patents with land patents'' [quoting the dissent]. \textit{McCormick} itself
makes clear that the analogy between the two depended on the particulars of the
Patent Act of 1870. Modern invention
patents, by contrast, are meaningfully different from land patents. The
land-patent cases invoked by the dissent involved a transaction in which
all authority or control' over the lands has passed from `the Executive
Department.
Their holdings do not apply when the Government continues
to possess some measure of control over the right in question.
And that is true of modern invention patents under the current Patent Act, which
gives the PTO continuing authority to review and potentially cancel patents
after they are issued.}

\ldots.

\readinghead{E}

We emphasize the narrowness of our holding. We address the constitutionality of
inter partes review only. We do not address whether other patent matters, such
as infringement actions, can be heard in a non-Article III forum. And because
the Patent Act provides for judicial review by the Federal Circuit,
we need not consider whether inter partes review would be
constitutional without any sort of intervention by a court at any stage of the
proceedings.
Moreover, we address only the precise constitutional challenges that Oil States
raised here. Oil States does not challenge the retroactive application of inter
partes review, even though that procedure was not in place when its patent
issued. Nor has Oil States raised a due process challenge. Finally, our decision
should not be misconstrued as suggesting that patents are not property for
purposes of the Due Process Clause or the Takings Clause.

\ldots.


\readinghead{V}

Because inter partes review does not violate Article III or the Seventh
Amendment, we affirm the judgment of the Court of Appeals.

\textit{It is so ordered.}

[A concurrence by Justice Breyer, joined by Justices Ginsburg and Sotomayor, is
omitted.]

\opinion Justice \textsc{Gorsuch}, with whom \textsc{The Chief Justice} joins,
dissenting.

After much hard work and no little investment you devise something you think
truly novel. Then you endure the further cost and effort of applying for a
patent, devoting maybe \$30,000 and two years to that process alone. At the end
of it all, the Patent Office agrees your invention is novel and issues a patent.
The patent affords you exclusive rights to the fruits of your labor for two
decades. But what happens if someone later emerges from the woodwork, arguing
that it was all a mistake and your patent should be canceled? Can a political
appointee and his administrative agents, instead of an independent judge,
resolve the dispute? The Court says yes. Respectfully, I disagree.


\ldots.


Patents began as little
more than feudal favors. The crown both issued and
revoked them. And they often permitted the lucky
recipient the exclusive right to do very ordinary things, like operate a toll
bridge or run a tavern. But by the 18th century, inventors were
busy in Britain and invention patents came to be seen in a different light. They
came to be viewed not as endowing accidental and anticompetitive monopolies on
the fortunate few but as a procompetitive means to secure to individuals the
fruits of their labor and ingenuity; encourage others to emulate them; and
promote public access to new technologies that would not otherwise exist.
The Constitution itself reflects this new
thinking, authorizing the issuance of patents precisely because of their
contribution to the ``Progress of Science and useful Arts.'' Art. I, \S~8, cl.
8. In essence, there was a change in perception---from viewing a patent as a
contract between the crown and the patentee to viewing it as a ``social
contract'' between the patentee and society.
And as invention patents came to be seen so differently, it is no
surprise courts came to treat them more solicitously.


\ldots.

Any lingering doubt about English law is resolved for me by looking to our own.
While the Court is correct that the Constitution's Patent Clause was written
against the backdrop of English practice,
it's also true that the Clause sought to \textit{reject}
some of early English practice. Reflecting the growing sentiment that patents
shouldn't be used for anticompetitive monopolies over goods or businesses
which had long before been enjoyed by the public, the framers wrote the Clause
to protect only procompetitive invention patents that are the product of hard
work and insight and ``add to the sum of useful knowledge.
In light of the Patent Clause's restrictions on this score, courts
took the view that when the federal government grants a patent
the grantee is entitled to it \textit{as a matter of right}, and does not
receive it, as was originally supposed to be the case in England, as a matter of
grace and favor. \textit{James v. Campbell}, 104 U.S. 356
(1882) (emphasis added). As Chief Justice Marshall explained, courts treated
American invention patents as recognizing an ``inchoate property'' that exists
``from the moment of invention.'' \textit{Evans v. Jordan}, 8 F.Cas. 872, 873
(No. 4,564) (C.C.D.Va.1813). American patent holders thus were thought to
hold a property in their inventions by as good a title as the farmer
holds his farm and flock.
And just as with farm and flock, it was
widely accepted that the government could divest patent owners of their rights
only through proceedings before independent judges.

\ldots.

With so much in the relevant history and precedent against it, the Court invites
us to look elsewhere. Instead of focusing on the revocation of patents, it asks
us to abstract the level of our inquiry and focus on their issuance. Because the
job of issuing invention patents traditionally belonged to the Executive, the
Court proceeds to argue, the job of revoking them can be left there too.
But that doesn't follow. Just because you give a
gift doesn't mean you forever enjoy the right to reclaim it. And, as we've seen,
just because the Executive could \textit{issue} an invention (or land) patent
did not mean the Executive could \textit{revoke} it. To reward those who had
proven the social utility of their work (and to induce others to follow suit),
the law long afforded patent holders more protection than that against the
threat of governmental intrusion and dispossession. The law requires us to honor
those historical rights, not diminish them.

Still, the Court asks us to look away in yet another direction. At the founding,
the Court notes, the Executive could sometimes both dispense and revoke public
franchises. And because, it says, invention patents are a species of public
franchises, the Court argues the Executive should be allowed to dispense and
revoke them too. But labels aside, by the time of the founding the law treated
patents protected by the Patent Clause quite differently from ordinary public
franchises. Many public franchises amounted to little more than favors
resembling the original royal patents the framers expressly refused to protect
in the Patent Clause. The Court points to a good example: the state-granted
exclusive right to operate a toll bridge. By the founding, courts in this
country (as in England) had come to view anticompetitive monopolies like that
with disfavor, narrowly construing the rights they conferred. By contrast,
courts routinely applied to invention patents protected by the Patent Clause the
liberal common sense construction that applies to other instruments creating
private property rights, like land deeds. As Justice Story explained, invention
patents protected by the Patent Clause were not to be treated as mere monopolies
odious in the eyes of the law, and therefore not to be favored. For precisely
these reasons and as we've seen, the law traditionally treated patents issued
under the Patent Clause very differently than monopoly franchises when it came
to governmental invasions. Patents alone required independent judges. Nor can
simply invoking a mismatched label obscure that fact. The people's historic
rights to have independent judges decide their disputes with the government
should not be a constitutional Maginot Line, easily circumvented by such simple
maneuvers.

Today's decision may not represent a rout but it at least signals a retreat from
Article III's guarantees. Ceding to the political branches ground they wish to
take in the name of efficient government may seem like an act of judicial
restraint. But enforcing Article III isn't about protecting judicial authority
for its own sake. It's about ensuring the people today and tomorrow enjoy no
fewer rights against governmental intrusion than those who came before. And the
loss of the right to an independent judge is never a small thing. It's for that
reason Hamilton warned the judiciary to take ``all possible care\ldots to defend
itself against'' intrusions by the other branches. The Federalist No. 78, at
466. It's for that reason I respectfully dissent.

