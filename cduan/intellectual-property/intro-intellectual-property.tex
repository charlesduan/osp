\editfromrepo{base}

\replacerange{It is common, and in some respects}{all of copyright law's other
requirements.}{In many ways, intellectual property rights are closely analogous
to rights in real or personal property. But there are critical
differences. For one thing, not all information is reasonably the subject of
ownership---no one can, or should, own the fact that five plus seven equals
twelve. Each intellectual property system must therefore have defined subject
matter, namely the kinds of information that can be protected under the system.

Second, how do rights in information come to be in the first place? Unlike land,
which simply exists, new information arises out of vague creative though
processes (or perhaps by other means). New ideas crop up all the time, and ideas
evolve out of older ideas. As a result, intellectual property laws must grapple
with questions of initial ownership.

Third, not all uses of information can be the subject of exclusivity. Law cannot
stop people from thinking about information once it is in their minds---that is
the stuff of science fiction. More importantly, information is
``non-rivalrous'': If you obtain information that I have, that has no effect on
my ability to use the same information. (Compare---if you're standing on my
land, I can't use the spot where your feet are.) This suggests a need for
detailed rules on what constitutes infringement.}

\replacerange{three such bodies}{few of the most important ones.}{some exemplary
cases from different bodies of intellectual property law. These cases are
intended to give you a taste for the different ways in which law protects
information. But they are also intended to prompt questions in your mind, about
how far this concept of ``property'' can extend into worlds like information
that are distinctly not like ordinary property.}

\endedit
