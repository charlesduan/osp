A patent is a federally granted right given to inventors for new and useful
inventions. During the period of approximately 20 years when a patent is in
force, the holder has the right to sue others who make, use, or sell the
invention without the patent holder's permission.

The traditional U.S. justification for patents is an
economic incentives argument. Invention is hard work, but it is often easy
to copy someone else's invention. So absent some sort of reward, inventors would
either not invest time and money into developing new inventions, or they would
keep their inventions secret. Insofar as a patent restricts competition by
allowing the patent holder to sue competitors, the patent holder is able to
profit off of the invention and thus recoup the costs of development.
Additionally, obtaining a patent requires revealing the inner workings of the
invention in a public document, thereby guaranteeing that the public has free
access to the invented technology after the patent expires.

In many ways, this seems like a property scheme. The patent holder has a
right to sue infringers who use the patented invention, much like how a
landowner can sue trespassers. On the other hand,
inventions don't seem much like land, and land ownership usually doesn't expire
after 20 years, right?

The following case presents competing perspectives on whether patents are a form
of property. The doctrinal setup is somewhat convoluted (and not relevant to
property law), so a brief summary is given here. Under the constitutional
doctrine of separation of powers, administrative agencies under the executive
branch cannot adjudicate disputes; that power is reserved to the courts under
Article III\@. However, under the ``public rights''
exception to the separation of powers doctrine, agencies are free to revoke
government benefits that they hand out, like taxi licenses, and thus are free as
well to adjudicate the benefit without a court.

In 2011, Congress created a new procedure called ``inter partes review'' by
which an agency, the Patent Trial and Appeal Board of the U.S. Patent and
Trademark Office, could adjudicate whether patents had been wrongly granted and
revoke them if so. This new procedure is unconstitutional, since the Board is
not an Article III court, unless it fits within the public rights exception.
Does it? The answer turns, in large part, on whether patents are more like
private property or government benefits.

In reading this case, focus on the justices' visions of patents, either as
property or not. What purposes do they see patents serving? How do these line up
with the theories of why society protects property generally? To what extent do
inventions merit different treatment from land? And what does this case tell you
more generally about the relationship between government and holders of
property?




