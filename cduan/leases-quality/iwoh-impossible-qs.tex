\defbook{rurlta}{
title=Revised Uniform Residential Landlord and Tenant Act,
year=2015,
instauth=National Conference of Commissioners on Uniform State Laws,
url=https://www.uniformlaws.org/viewdocument/final-act-119,
hereinafter=RURLTA,
}

\item What if the landlord is unable to remedy a violation within a reasonable
time, due to external circumstances? For example, in Flint, Michigan, a
municipal utility planning error left residents with hazardous tap water in
2014. Assuming that water filtration or other in-home solutions were not
options, would a landlord be on the hook for the city's error?

The \textsc{Revised Uniform Residential Landlord and Tenant Act}, proposed in
2015 but not yet adopted in any state (as of 2024), recognized that this could
be a problem. The model code adds Section 403, which ``limits the landlord's
liabilities in cases where it is impossible for the landlord to remedy a
noncompliance.'' \sentence{rurlta at 2}. Specifically, that section provides
that in cases where remediation is impossible, the tenant may either terminate
the lease or ``recover actual damages limited to the diminution in the value of
the dwelling unit.'' \sentence{rurlta at S 403(c)}. The landlord may also
terminate the lease upon 30 days' notice, and a landlord who does so may not
rent the unit for 90 days thereafter. \sentence{rurlta at S 403(d)}.

Given that the RURLTA has not been adopted, how should state courts treat these
situations? Are there doctrines from contract law that might help?


