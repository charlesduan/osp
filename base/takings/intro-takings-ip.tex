Property rights may reach intangible things, and the Takings Clause may apply to
these rights. \textit{See generally} 2-5 \textsc{Nichols on Eminent Domain}
\S~5.03 (listing examples). What about ``intellectual property''? In
\textit{Ruckelshaus v. Monsanto Co.}, 467 U.S. 986 (1984), the Court ruled that
to the extent state law recognized a property right in trade secrets, they were
protected by the Takings Clause. 
\begin{quote}
Although this Court never has squarely addressed the question whether a person
can have a property interest in a trade secret, which is admittedly intangible,
the Court has found other kinds of intangible interests to be property for
purposes of the Fifth Amendment's Taking Clause. \textit{See, e.g., Armstrong v.
United States}, 364 U.S. 40, 44, 46 (1960) (materialman's lien provided for
under Maine law protected by Taking Clause); \textit{Louisville Joint Stock Land
Bank v. Radford}, 295 U.S. 555, 596--602, (1935) (real estate lien protected);
\textit{Lynch v. United States}, 292 U.S. 571, 579 (1934) (valid contracts are
property within meaning of the Taking Clause). That intangible property rights
protected by state law are deserving of the protection of the Taking Clause has
long been implicit in the thinking of this Court.
\end{quote}
\textit{Id.} at 1003 (1984). The \textit{Monsanto} plaintiff claimed that
disclosure requirements of the Federal Insecticide, Fungicide, and Rodenticide
Act would destroy its trade secrets. The Court held that the absence of
reasonable investment-backed expectations precluded some of these claims,
concluding that the plaintiff had submitted its data under a regulatory scheme
that required eventual disclosure. \textit{Id.} at 1007 (``Thus, as long as
Monsanto is aware of the conditions under which the data are submitted\ldots a
voluntary submission of data by an applicant in exchange for the economic
advantages of a registration can hardly be called a taking.'').

Trade secrets exist under state law, to which courts may look in determining
whether a property interest exists. What about federal IP rights? The issue is a
debated. \textit{Compare, e.g.}, Davida H. Isaacs, \textit{Not All Property Is
Created Equal: Why Modern Courts Resist Applying the Takings Clause to Patents,
and Why They Are Right to Do So}, 15 \textsc{Geo. Mason L. Rev.} 1 (2007),
\textit{with} Adam Mossoff, \textit{Patents As Constitutional Private Property:
The Historical Protection of Patents Under the Takings Clause}, 87 \textsc{B.U.
L. Rev}. 689, 689 (2007); \textit{see generally} Thomas F. Cotter, \textit{Do
Federal Uses of Intellectual Property Implicate the Fifth Amendment?}, 50
\textsc{Fla. L. Rev}. 529 (1998). (``[T]he law of takings with regard to
intellectual property can only be characterized as a muddle within the
muddle.'').

Should intellectual property receive takings protection? On the one hand, the
underlying statutes give them the attributes of property. 35 U.S.C. \S~261
(``Subject to the provisions of this title, patents shall have the attributes of
personal property.''). On the other, IP rights lack many of the traditional
attributes of property. Not only are they intangible, but they constitute a
government delegation to private parties of regulatory power over the actions of
others. To the extent the government wishes to curtail these rights---or
otherwise adjust the governing regime, introducing takings doctrine may upset
its ability to adjust a regulatory regime to changing circumstances. 

\expected{kremen-v-cohen}

Moreover, the malleability of the concept of ``property'' complicates matters,
for the question whether an intangible interest is property may arise in a
context independent of any takings issues. Once the property switch is flipped,
however, the complexities of takings analysis kick in. Recall, for example, the
issue covered earlier as to whether domain names are property for purposes of
state law. \textit{Kremen v. Cohen}'s answer in the affirmative afforded a
remedy for a wronged party in a conversion action, but the classification could
ripple through other bodies of law. For example, the Anticybersquatting Consumer
Protection Act (ACPA) allows trademark holders to claim domain names containing
the marks from those who registered them with a ``bad faith intent to profit.''
15 U.S.C. \S~1125(d). But if a domain name is property---one that one acquires
by registering it---how is ACPA's operation not a taking without just
compensation? Worse, how is it not taking from A and giving to B as prohibited
by the ``Public Use'' Clause? To date, courts have not been receptive to this
argument, \textit{DaimlerChrysler v. The Net Inc}., 388 F.3d 201 (6th Cir.
2004), but it suggests the difficulties with casually applying the label of
property to interests that exist outside the common law property tradition.

