\reading{Tahoe-Sierra Preservation Council, Inc. v. Tahoe Regional Planning
Agency}

\readingcite{535 U.S. 302 (2002)}

\opinion Justice \textsc{Stevens} delivered the opinion of the Court.

The question presented is whether a moratorium on development imposed during the
process of devising a comprehensive land-use plan constitutes a \textit{per se}
taking of property requiring compensation under the Takings Clause of the United
States Constitution. This case actually involves two moratoria ordered by
respondent Tahoe Regional Planning Agency (TRPA) to maintain the status quo
while studying the impact of development on Lake Tahoe and designing a strategy
for environmentally sound growth. The first, Ordinance 81--5, was effective from
August 24, 1981, until August 26, 1983, whereas the second more restrictive
Resolution 83--21 was in effect from August 27, 1983, until April 25, 1984. As a
result of these two directives, virtually all development on a substantial
portion of the property subject to TRPA's jurisdiction was prohibited for a
period of 32 months. Although the question we decide relates only to that
32--month period, a brief description of the events leading up to the moratoria
and a comment on the two permanent plans that TRPA adopted thereafter will
clarify the narrow scope of our holding.


\readinghead{I}

The relevant facts are undisputed. The Court of Appeals, while reversing the
District Court on a question of law, accepted all of its findings of fact, and
no party challenges those findings. All agree that Lake Tahoe is ``uniquely
beautiful,'' 34 F.Supp.2d 1226, 1230 (D.Nev.1999), that President Clinton was
right to call it a ``\,`national treasure that must be protected and
preserved,'\,'' \textit{ibid.}, and that Mark Twain aptly described the clarity
of its waters as ``\,`not \textit{merely} transparent, but dazzlingly,
brilliantly so,'\,'' \textit{ibid.} (emphasis added) (quoting M. Twain, Roughing
It 174--175 (1872)).

Lake Tahoe's exceptional clarity is attributed to the absence of algae that
obscures the waters of most other lakes. Historically, the lack of nitrogen and
phosphorous, which nourish the growth of algae, has ensured the transparency of
its waters. Unfortunately, the lake's pristine state has deteriorated rapidly
over the past 40 years; increased land development in the Lake Tahoe Basin
(Basin) has threatened the ``noble sheet of blue water'' beloved by Twain
and countless others. As the District Court found, ``[d]ramatic decreases in
clarity first began to be noted in the late 1950's/early 1960's, shortly after
development at the lake began in earnest.'' The lake's unsurpassed beauty, it
seems, is the wellspring of its undoing.

The upsurge of development in the area has caused ``increased nutrient loading
of the lake largely because of the increase in impervious coverage of land in
the Basin resulting from that development.''\ldots 

Given this trend, the District Court predicted that ``unless the process is
stopped, the lake will lose its clarity and its trademark blue color, becoming
green and opaque for eternity.''

Those areas in the Basin that have steeper slopes produce more runoff;
therefore, they are usually considered ``high hazard'' lands. Moreover, certain
areas near streams or wetlands known as ``Stream Environment Zones'' (SEZs) are
especially vulnerable to the impact of development because, in their natural
state, they act as filters for much of the debris that runoff carries. Because
``[t]he most obvious response to this problem\ldots is to restrict development
around the lake---especially in SEZ lands, as well as in areas already naturally
prone to runoff,'' conservation efforts have focused on controlling growth in
these high hazard areas.

In the 1960's, when the problems associated with the burgeoning development
began to receive significant attention, jurisdiction over the Basin, which
occupies 501 square miles, was shared by the States of California and Nevada,
five counties, several municipalities, and the Forest Service of the Federal
Government. In 1968, the legislatures of the two States adopted the Tahoe
Regional Planning Compact. The compact set goals for the protection and
preservation of the lake and created TRPA\ldots.

The 1980 Tahoe Regional Planning Compact (Compact) redefined the structure,
functions, and voting procedures of TRPA and directed it to develop regional
``environmental threshold carrying capacities''---a term that embraced
``standards for air quality, water quality, soil conservation, vegetation
preservation and noise.''\ldots The Compact also contained a finding by the
legislatures of California and Nevada ``that in order to make effective the
regional plan as revised by [TRPA], it is necessary to halt temporarily works of
development in the region which might otherwise absorb the entire capability of
the region for further development or direct it out of harmony with the ultimate
plan.'' Accordingly, for the period prior to the adoption of the final plan
(``or until May 1, 1983, whichever is earlier''), the Compact itself prohibited
the development of new subdivisions, condominiums, and apartment buildings, and
also prohibited each city and county in the Basin from granting any more permits
in 1981, 1982, or 1983 than had been granted in 1978.

During this period TRPA was also working on the development of a regional water
quality plan to comply with the Clean Water Act, 33 U.S.C. \S~1288 (1994 ed.).
[Because it could not meet the Compact's timetables,] ``[o]n June 25, 1981, it
therefore enacted Ordinance 81--5 imposing the first of the two moratoria on
development that petitioners challenge in this proceeding. The ordinance
provided that it would become effective on August 24, 1981, and remain in effect
pending the adoption of the permanent plan required by the Compact. 

The District Court made a detailed analysis of the ordinance, noting that it
might even prohibit hiking or picnicking on SEZ lands, but construed it as
essentially banning any construction or other activity that involved the removal
of vegetation or the creation of land coverage on all SEZ lands, as well as on
class 1, 2, and 3 lands in California. Some permits could be obtained for such
construction in Nevada if certain findings were made. It is undisputed, however,
that Ordinance 81--5 prohibited the construction of any new residences on SEZ
lands in either State and on class 1, 2, and 3 lands in California.\ldots

[TRPA later] adopted Resolution 83--21, ``which completely suspended all project
reviews and approvals, including the acceptance of new proposals,'' and which
remained in effect until a new regional plan was adopted on April 26, 1984.
Thus, Resolution 83--21 imposed an 8--month moratorium prohibiting all
construction on high hazard lands in either State. In combination, Ordinance
81--5 and Resolution 83--21 effectively prohibited all construction on sensitive
lands in California and on all SEZ lands in the entire Basin for 32 months, and
on sensitive lands in Nevada (other than SEZ lands) for eight months. It is
these two moratoria that are at issue in this case.\ldots


\readinghead{II}

Approximately two months after the adoption of the 1984 plan, petitioners filed
parallel actions against TRPA and other defendants in federal courts in Nevada
and California that were ultimately consolidated for trial in the District of
Nevada. The petitioners include the Tahoe--Sierra Preservation Council, Inc., a
nonprofit membership corporation representing about 2,000 owners of both
improved and unimproved parcels of real estate in the Lake Tahoe Basin, and a
class of some 400 individual owners of vacant lots located either on SEZ lands
or in other parts of districts 1, 2, or 3. Those individuals purchased their
properties prior to the effective date of the 1980 Compact primarily for the
purpose of constructing ``at a time of their choosing'' a single-family home
``to serve as a permanent, retirement or vacation residence.'' When they made
those purchases, they did so with the understanding that such construction was
authorized provided that ``they complied with all reasonable requirements for
building.''

Petitioners' complaints gave rise to protracted litigation that has produced
four opinions by the Court of Appeals for the Ninth Circuit and several
published District Court opinions. For present purposes, however, we need only
describe those courts' disposition of the claim that three actions taken by
TRPA---Ordinance 81--5, Resolution 83--21, and the 1984 regional
plan---constituted takings of petitioners' property without just compensation.
Indeed, the challenge to the 1984 plan is not before us\ldots. Thus, we limit
our discussion to the lower courts' disposition of the claims based on the
2--year moratorium (Ordinance 81--5) and the ensuing 8--month moratorium
(Resolution 83--21).

The District Court began its constitutional analysis by identifying the
distinction between a direct government appropriation of property without just
compensation and a government regulation that imposes such a severe restriction
on the owner's use of her property that it produces ``nearly the same result as
a direct appropriation.'' The court noted that all of the claims in this case
``are of the `regulatory takings' variety.''\ldots [The District Court concluded
that there was no taking under the \textit{Penn Central} factors.]

The District Court had more difficulty with the ``total taking'' issue. Although
it was satisfied that petitioners' property did retain some value during the
moratoria, it found that they had been temporarily deprived of ``all
economically viable use of their land.'' The court concluded that those actions
therefore constituted ``categorical'' takings under our decision in
\textit{Lucas v. South Carolina Coastal Council}, 505 U.S. 1003 (1992). It
rejected TRPA's response that Ordinance 81--5 and Resolution 83--21 were
``reasonable temporary planning moratoria'' that should be excluded from
\textit{Lucas'} categorical approach. The court thought it ``fairly clear'' that
such interim actions would not have been viewed as takings prior to our
decisions in \textit{Lucas} and \textit{First English Evangelical Lutheran
Church of Glendale v. County of Los Angeles}, 482 U.S. 304 (1987)\ldots . After
expressing uncertainty as to whether those cases required a holding that
moratoria on development automatically effect takings, the court concluded that
TRPA's actions did so, partly because neither the ordinance nor the resolution,
even though intended to be temporary from the beginning, contained an express
termination date. Accordingly, it ordered TRPA to pay damages to most
petitioners for the 32--month period from August 24, 1981, to April 25, 1984,
and to those owning class 1, 2, or 3 property in Nevada for the 8--month period
from August 27, 1983, to April 25, 1984.

Both parties appealed. TRPA successfully challenged the District Court's takings
determination\ldots . Petitioners did not, however, challenge the District
Court's findings or conclusions concerning its application of \textit{Penn
Central.\ldots } Accordingly, the only question before the court was ``whether
the rule set forth in \textit{Lucas} applies---that is, whether a categorical
taking occurred because Ordinance 81--5 and Resolution 83--21 denied the
plaintiffs `all economically beneficial or productive use of land.' `` Moreover,
because petitioners brought only a facial challenge, the narrow inquiry before
the Court of Appeals was whether the mere enactment of the regulations
constituted a taking.

Contrary to the District Court, the Court of Appeals held that because the
regulations had only a temporary impact on petitioners' fee interest in the
properties, no categorical taking had occurred.\ldots


\readinghead{III}

Petitioners make only a facial attack on Ordinance 81--5 and Resolution 83--21.
They contend that the mere enactment of a temporary regulation that, while in
effect, denies a property owner all viable economic use of her property gives
rise to an unqualified constitutional obligation to compensate her for the value
of its use during that period. Hence, they ``face an uphill battle,''
\textit{Keystone Bituminous Coal Assn. v. DeBenedictis}, 480 U.S. 470, 495
(1987), that is made especially steep by their desire for a categorical rule
requiring compensation whenever the government imposes such a moratorium on
development. Under their proposed rule, there is no need to evaluate the
landowners' investment-backed expectations, the actual impact of the regulation
on any individual, the importance of the public interest served by the
regulation, or the reasons for imposing the temporary restriction. For
petitioners, it is enough that a regulation imposes a temporary deprivation---no
matter how brief---of all economically viable use to trigger a \textit{per se}
rule that a taking has occurred. Petitioners assert that our opinions in
\textit{First English} and \textit{Lucas} have already endorsed their view, and
that it is a logical application of the principle that the Takings Clause was
``designed to bar Government from forcing some people alone to bear burdens
which, in all fairness and justice, should be borne by the public as a whole.''
\textit{Armstrong v. United States}, 364 U.S. 40, 49 (1960).

We shall first explain why our cases do not support their proposed categorical
rule---indeed, fairly read, they implicitly reject it. Next, we shall explain
why the \textit{Armstrong} principle requires rejection of that rule as well as
the less extreme position advanced by petitioners at oral argument. In our view
the answer to the abstract question whether a temporary moratorium effects a
taking is neither ``yes, always'' nor ``no, never''; the answer depends upon the
particular circumstances of the case.\ldots


\readinghead{IV}

\ldots When the government physically takes possession of an interest in
property for some public purpose, it has a categorical duty to compensate the
former owner regardless of whether the interest that is taken constitutes an
entire parcel or merely a part thereof. Thus, compensation is mandated when a
leasehold is taken and the government occupies the property for its own
purposes, even though that use is temporary.\ldots But a government regulation
that merely prohibits landlords from evicting tenants unwilling to pay a higher
rent, \textit{Block v. Hirsh}, 256 U.S. 135 (1921); that bans certain private
uses of a portion of an owner's property, \textit{Village of Euclid v. Ambler
Realty Co.}, 272 U.S. 365 (1926); or that forbids the private use of certain
airspace, \textit{Penn Central Transp. Co. v. New York City}, 438 U.S. 104
(1978), does not constitute a categorical taking. ``The first category of cases
requires courts to apply a clear rule; the second necessarily entails complex
factual assessments of the purposes and economic effects of government
actions.'' \textit{Yee v. Escondido}, 503 U.S. 519, 523 (1992).

This longstanding distinction between acquisitions of property for public use,
on the one hand, and regulations prohibiting private uses, on the other, makes
it inappropriate to treat cases involving physical takings as controlling
precedents for the evaluation of a claim that there has been a ``regulatory
taking,'' and vice versa. For the same reason that we do not ask whether a
physical appropriation advances a substantial government interest or whether it
deprives the owner of all economically valuable use, we do not apply our
precedent from the physical takings context to regulatory takings claims.
Land-use regulations are ubiquitous and most of them impact property values in
some tangential way---often in completely unanticipated ways. Treating them all
as \textit{per se} takings would transform government regulation into a luxury
few governments could afford. By contrast, physical appropriations are
relatively rare, easily identified, and usually represent a greater affront to
individual property rights. ``This case does not present the `classi[c] taking'
in which the government directly appropriates private property for its own
use,'' \textit{Eastern Enterprises v. Apfel}, 524 U.S. 498, 522 (1998); instead
the interference with property rights ``arises from some public program
adjusting the benefits and burdens of economic life to promote the common
good,'' \textit{Penn Central}, 438 U.S., at 124.

Perhaps recognizing this fundamental distinction, petitioners wisely do not
place all their emphasis on analogies to physical takings cases. Instead, they
rely principally on our decision in \textit{Lucas v. South Carolina Coastal
Council}, 505 U.S. 1003 (1992)---a regulatory takings case that, nevertheless,
applied a categorical rule---to argue that the \textit{Penn Central} framework
is inapplicable here.\ldots 

As we noted in \textit{Lucas}, it was Justice Holmes' opinion in
\textit{Pennsylvania Coal Co. v. Mahon}, 260 U.S. 393 (1922), that gave birth to
our regulatory takings jurisprudence. In subsequent opinions we have repeatedly
and consistently endorsed Holmes' observation that ``if regulation goes too far
it will be recognized as a taking.''\ldots

In the decades following that decision, we have ``generally eschewed'' any set
formula for determining how far is too far, choosing instead to engage in
``\,`essentially ad hoc, factual inquiries.'\,'' \textit{Lucas}, 505 U.S., at
1015 (quoting \textit{Penn Central}, 438 U.S., at 124). Indeed, we still resist
the temptation to adopt \textit{per se} rules in our cases involving partial
regulatory takings, preferring to examine ``a number of factors'' rather than a
simple ``mathematically precise'' formula. Justice Brennan's opinion for the
Court in \textit{Penn Central} did, however, make it clear that even
though multiple factors are relevant in the analysis of regulatory takings
claims, in such cases we must focus on ``the parcel as a whole'':
\begin{quote}
``Taking'' jurisprudence does not divide a single parcel into discrete segments
and attempt to determine whether rights in a particular segment have been
entirely abrogated. In deciding whether a particular governmental action has
effected a taking, this Court focuses rather both on the character of the action
and on the nature and extent of the interference with rights in the parcel as a
whole---here, the city tax block designated as the ``landmark site.''
\end{quote}

This requirement that ``the aggregate must be viewed in its entirety'' explains
why, for example, a regulation that prohibited commercial transactions in eagle
feathers, but did not bar other uses or impose any physical invasion or
restraint upon them, was not a taking. \textit{Andrus v. Allard}, 444 U.S. 51
(1979). It also clarifies why restrictions on the use of only limited portions
of the parcel, such as setback ordinances, \textit{Gorieb v. Fox}, 274 U.S. 603
(1927), or a requirement that coal pillars be left in place to prevent mine
subsidence, \textit{Keystone Bituminous Coal Assn. v. DeBenedictis}, 480 U.S.,
at 498, were not considered regulatory takings. In each of these cases, we
affirmed that ``where an owner possesses a full `bundle' of property rights, the
destruction of one `strand' of the bundle is not a taking.'' \textit{Andrus},
444 U.S., at 65--66.

While the foregoing cases considered whether particular regulations had ``gone
too far'' and were therefore invalid, none of them addressed the separate
remedial question of how compensation is measured once a regulatory taking is
established. In his dissenting opinion in \textit{San Diego Gas \& Elec. Co. v.
San Diego}, 450 U.S. 621, 636 (1981), Justice Brennan identified that question
and explained how he would answer it:
\begin{quote}
``The constitutional rule I propose requires that, once a court finds that a
police power regulation has effected a `taking,' the government entity must pay
just compensation for the period commencing on the date the regulation first
effected the `taking,' and ending on the date the government entity chooses to
rescind or otherwise amend the regulation.'' 
\end{quote}

Justice Brennan's proposed rule was subsequently endorsed by the Court in
\textit{First English}, 482 U.S., at 315, 318, 321. \textit{First English} was
certainly a significant decision, and nothing that we say today qualifies its
holding. Nonetheless, it is important to recognize that we did not address in
that case the quite different and logically prior question whether the temporary
regulation at issue had in fact constituted a taking.

In \textit{First English}, the Court unambiguously and repeatedly characterized
the issue to be decided as a ``compensation question'' or a ``remedial
question.'' And the Court's statement of its holding was equally unambiguous:
``We merely hold that where the government's activities \textit{have already
worked a taking} of all use of property, no subsequent action by the government
can relieve it of the duty to provide compensation for the period during which
the taking was effective.'' (emphasis added). In fact, \textit{First English}
expressly disavowed any ruling on the merits of the takings issue because the
California courts had decided the remedial question on the assumption that a
taking had been alleged.\ldots

Similarly, our decision in \textit{Lucas} is not dispositive of the question
presented. Although \textit{Lucas} endorsed and applied a categorical rule, it
was not the one that petitioners propose. Lucas purchased two residential lots
in 1988 for \$975,000. These lots were rendered ``valueless'' by a statute
enacted two years later. The trial court found that a taking had occurred and
ordered compensation of \$1,232,387.50, representing the value of the fee simple
estate, plus interest. As the statute read at the time of the trial, it effected
a taking that ``was unconditional and permanent.''\ldots

The categorical rule that we applied in \textit{Lucas} states that compensation
is required when a regulation deprives an owner of ``\textit{all} economically
beneficial uses'' of his land. Under that rule, a statute that ``wholly
eliminated the value'' of Lucas' fee simple title clearly qualified as a taking.
But our holding was limited to ``the extraordinary circumstance when \textit{no}
productive or economically beneficial use of land is permitted.'' The emphasis
on the word ``no'' in the text of the opinion was, in effect, reiterated in a
footnote explaining that the categorical rule would not apply if the diminution
in value were 95\% instead of 100\%. Anything less than a ``complete elimination
of value,'' or a ``total loss,'' the Court acknowledged, would require the kind
of analysis applied in \textit{Penn Central.}

Certainly, our holding that the permanent ``obliteration of the value'' of a fee
simple estate constitutes a categorical taking does not answer the question
whether a regulation prohibiting any economic use of land for a 32--month period
has the same legal effect. Petitioners seek to bring this case under the rule
announced in \textit{Lucas} by arguing that we can effectively sever a 32--month
segment from the remainder of each landowner's fee simple estate, and then ask
whether that segment has been taken in its entirety by the moratoria. Of course,
defining the property interest taken in terms of the very regulation being
challenged is circular. With property so divided, every delay would become a
total ban; the moratorium and the normal permit process alike would constitute
categorical takings. Petitioners' ``conceptual severance'' argument is
unavailing because it ignores \textit{Penn Central's} admonition that in
regulatory takings cases we must focus on ``the parcel as a whole.'' We have
consistently rejected such an approach to the ``denominator'' question. Thus,
the District Court erred when it disaggregated petitioners' property into
temporal segments corresponding to the regulations at issue and then analyzed
whether petitioners were deprived of all economically viable use during each
period. The starting point for the court's analysis should have been to ask
whether there was a total taking of the entire parcel; if not, then \textit{Penn
Central} was the proper framework.

An interest in real property is defined by the metes and bounds that describe
its geographic dimensions and the term of years that describes the temporal
aspect of the owner's interest. See Restatement of Property \S\S~7--9 (1936).
Both dimensions must be considered if the interest is to be viewed in its
entirety. Hence, a permanent deprivation of the owner's use of the entire area
is a taking of ``the parcel as a whole,'' whereas a temporary restriction that
merely causes a diminution in value is not. Logically, a fee simple estate
cannot be rendered valueless by a temporary prohibition on economic use, because
the property will recover value as soon as the prohibition is lifted.\ldots


\readinghead{V}

Considerations of ``fairness and justice'' arguably could support the conclusion
that TRPA's moratoria were takings of petitioners' property based on any of
seven different theories. First, even though we have not previously done so, we
might now announce a categorical rule that, in the interest of fairness and
justice, compensation is required whenever government temporarily deprives an
owner of all economically viable use of her property. Second, we could craft a
narrower rule that would cover all temporary land-use restrictions except those
``normal delays in obtaining building permits, changes in zoning ordinances,
variances, and the like'' which were put to one side in our opinion in
\textit{First English}. Third, we could adopt a rule like the one suggested by
an \textit{amicus} supporting petitioners that would ``allow a short fixed
period for deliberations to take place without compensation---say maximum one
year---after which the just compensation requirements'' would ``kick in.''
Fourth, with the benefit of hindsight, we might characterize the successive
actions of TRPA as a ``series of rolling moratoria'' that were the functional
equivalent of a permanent taking. Fifth, were it not for the findings of the
District Court that TRPA acted diligently and in good faith, we might have
concluded that the agency was stalling in order to avoid promulgating the
environmental threshold carrying capacities and regional plan mandated by the
1980 Compact. Sixth, apart from the District Court's finding that TRPA's actions
represented a proportional response to a serious risk of harm to the lake,
petitioners might have argued that the moratoria did not substantially advance a
legitimate state interest. Finally, if petitioners had challenged the
application of the moratoria to their individual parcels, instead of making a
facial challenge, some of them might have prevailed under a \textit{Penn
Central} analysis.

As the case comes to us, however, none of the last four theories is available.
The ``rolling moratoria'' theory was presented in the petition for certiorari,
but our order granting review did not encompass that issue; the case was tried
in the District Court and reviewed in the Court of Appeals on the theory that
each of the two moratoria was a separate taking, one for a 2--year period and
the other for an 8--month period. And, as we have already noted, recovery on
either a bad faith theory or a theory that the state interests were
insubstantial is foreclosed by the District Court's unchallenged findings of
fact. Recovery under a \textit{Penn Central} analysis is also foreclosed both
because petitioners expressly disavowed that theory, and because they did not
appeal from the District Court's conclusion that the evidence would not support
it. Nonetheless, each of the three \textit{per se} theories is fairly
encompassed within the question that we decided to answer.

With respect to these theories, the ultimate constitutional question is whether
the concepts of ``fairness and justice'' that underlie the Takings Clause will
be better served by one of these categorical rules or by a \textit{Penn Central}
inquiry into all of the relevant circumstances in particular cases. From that
perspective, the extreme categorical rule that any deprivation of all economic
use, no matter how brief, constitutes a compensable taking surely cannot be
sustained. Petitioners' broad submission would apply to numerous ``normal delays
in obtaining building permits, changes in zoning ordinances, variances, and the
like,'' [\textit{First English},] 482 U.S., at 321, as well as to orders
temporarily prohibiting access to crime scenes, businesses that violate health
codes, fire-damaged buildings, or other areas that we cannot now foresee. Such a
rule would undoubtedly require changes in numerous practices that have long been
considered permissible exercises of the police power. As Justice Holmes warned
in \textit{Mahon}, ``[g]overnment hardly could go on if to some extent values
incident to property could not be diminished without paying for every such
change in the general law.'' A rule that required compensation for every delay
in the use of property would render routine government processes prohibitively
expensive or encourage hasty decisionmaking. Such an important change in the law
should be the product of legislative rulemaking rather than adjudication.\ldots

In rejecting petitioners' \textit{per se} rule, we do not hold that the
temporary nature of a land-use restriction precludes finding that it effects a
taking; we simply recognize that it should not be given exclusive significance
one way or the other.

A narrower rule that excluded the normal delays associated with processing
permits, or that covered only delays of more than a year, would certainly have a
less severe impact on prevailing practices, but it would still impose serious
financial constraints on the planning process.\ldots [M]oratoria like Ordinance
81--5 and Resolution 83--21 are used widely among land-use planners to preserve
the status quo while formulating a more permanent development strategy.\ldots 

The interest in facilitating informed decisionmaking by regulatory agencies
counsels against adopting a \textit{per se} rule that would impose such severe
costs on their deliberations. Otherwise, the financial constraints of
compensating property owners during a moratorium may force officials to rush
through the planning process or to abandon the practice altogether.\ldots

It may well be true that any moratorium that lasts for more than one year should
be viewed with special skepticism. But given the fact that the District Court
found that the 32 months required by TRPA to formulate the 1984 Regional Plan
was not unreasonable, we could not possibly conclude that every delay of over
one year is constitutionally unacceptable. Formulating a general rule of this
kind is a suitable task for state legislatures. In our view, the duration of the
restriction is one of the important factors that a court must consider in the
appraisal of a regulatory takings claim .\ldots We conclude, therefore, that the
interest in ``fairness and justice'' will be best served by relying on the
familiar \textit{Penn Central} approach when deciding cases like this, rather
than by attempting to craft a new categorical rule.

[The dissenting opinions of Chief Justice Rehnquist (joined by Justice Scalia
and Justice Thomas) and of Justice Thomas (joined by Justice Scalia) are
omitted.]

