\reading{Koontz v. St. Johns River Water Management Dist.}

\readingcite{570 U.S. 595 (2013)}

\opinion Justice \textsc{Alito} delivered the opinion of the Court.

Our decisions in \textit{Nollan v. California Coastal Comm'n}, 483 U.S.
825 (1987), and \textit{Dolan v. City of Tigard}, 512 U.S. 374 (1994), provide
important protection against the misuse of the power of land-use regulation. In
those cases, we held that a unit of government may not condition the approval of
a land-use permit on the owner's relinquishment of a portion of his property
unless there is a ``nexus'' and ``rough proportionality'' between the
government's demand and the effects of the proposed land use. In this case, the
St. Johns River Water Management District (District) believes that it
circumvented \textit{Nollan} and \textit{Dolan} because of the way in which it
structured its handling of a permit application submitted by Coy Koontz, Sr.,
whose estate is represented in this Court by Coy Koontz, Jr. The District did
not approve his application on the condition that he surrender an interest in
his land. Instead, the District, after suggesting that he could obtain approval
by signing over such an interest, denied his application because he refused to
yield. The Florida Supreme Court blessed this maneuver and thus effectively
interred those important decisions. Because we conclude that \textit{Nollan} and
\textit{Dolan} cannot be evaded in this way, the Florida Supreme Court's
decision must be reversed. 


\readinghead{I}


\readinghead{A}

In 1972, petitioner purchased an undeveloped 14.9--acre tract of land on the
south side of Florida State Road 50, a divided four-lane highway east of
Orlando. The property is located less than 1,000 feet from that road's
intersection with Florida State Road 408, a tolled expressway that is one of
Orlando's major thoroughfares.

A drainage ditch runs along the property's western edge, and high-voltage power
lines bisect it into northern and southern sections. The combined effect of the
ditch, a 100--foot wide area kept clear for the power lines, the highways, and
other construction on nearby parcels is to isolate the northern section of
petitioner's property from any other undeveloped land. Although largely
classified as wetlands by the State, the northern section drains well; the most
significant standing water forms in ruts in an unpaved road used to access the
power lines. The natural topography of the property's southern section is
somewhat more diverse, with a small creek, forested uplands, and wetlands that
sometimes have water as much as a foot deep. A wildlife survey found evidence of
animals that often frequent developed areas: raccoons, rabbits, several species
of bird, and a turtle. The record also indicates that the land may be a suitable
habitat for opossums.

[Florida law regulates construction that affect state waters. Landowners with
construction plans that might affect state waters must obtain a Management and
Storage of Surface Water (MSSW) permit, which may impose conditions to protect
local water resources. In addition, state law prohibits dredging or filling
surface waters without a Wetlands Resource Management (WRM) permit, which is to
be granted only if the construction is not against the public interest. To that
end, the St. Johns River Water Management District, which regulated Koontz's
land, required construction in the wetlands to be offset by activities that
benefitted wetlands in other locations.]

Petitioner decided to develop the 3.7--acre northern section of his property,
and in 1994 he applied to the District for MSSW and WRM permits. Under his
proposal, petitioner would have raised the elevation of the northernmost section
of his land to make it suitable for a building, graded the land from the
southern edge of the building site down to the elevation of the high-voltage
electrical lines, and installed a dry-bed pond for retaining and gradually
releasing stormwater runoff from the building and its parking lot. To mitigate
the environmental effects of his proposal, petitioner offered to foreclose any
possible future development of the approximately 11--acre southern section of
his land by deeding to the District a conservation easement on that portion of
his property.

The District considered the 11--acre conservation easement to be inadequate, and
it informed petitioner that it would approve construction only if he agreed to
one of two concessions. First, the District proposed that petitioner reduce the
size of his development to 1 acre and deed to the District a conservation
easement on the remaining 13.9 acres. To reduce the development area, the
District suggested that petitioner could eliminate the dry-bed pond from his
proposal and instead install a more costly subsurface stormwater management
system beneath the building site. The District also suggested that petitioner
install retaining walls rather than gradually sloping the land from the building
site down to the elevation of the rest of his property to the south.

In the alternative, the District told petitioner that he could proceed with the
development as proposed, building on 3.7 acres and deeding a conservation
easement to the government on the remainder of the property, if he also agreed
to hire contractors to make improvements to District-owned land several miles
away. Specifically, petitioner could pay to replace culverts on one parcel or
fill in ditches on another. Either of those projects would have enhanced
approximately 50 acres of District-owned wetlands. When the District asks permit
applicants to fund offsite mitigation work, its policy is never to require any
particular offsite project, and it did not do so here. Instead, the District
said that it ``would also favorably consider'' alternatives to its suggested
offsite mitigation projects if petitioner proposed something ``equivalent.'' 

Believing the District's demands for mitigation to be excessive in light of the
environmental effects that his building proposal would have caused, petitioner
filed suit in state court. Among other claims, he argued that he was entitled to
relief under Fla. Stat. \S~373.617(2), which allows owners to recover ``monetary
damages'' if a state agency's action is ``an unreasonable exercise of the
state's police power constituting a taking without just compensation.''


\readinghead{B}

\ldots [T]he State Circuit Court held a 2--day bench trial. After considering
testimony from several experts who examined petitioner's property, the trial
court found that the property's northern section had already been ``seriously
degraded'' by extensive construction on the surrounding parcels. In light of
this finding and petitioner's offer to dedicate nearly three-quarters of his
land to the District, the trial court concluded that any further mitigation in
the form of payment for offsite improvements to District property lacked both a
nexus and rough proportionality to the environmental impact of the proposed
construction. It accordingly held the District's actions unlawful under our
decisions in \textit{Nollan} and \textit{Dolan}.

The Florida District Court affirmed, but the State Supreme Court reversed. A
majority of that court distinguished \textit{Nollan} and \textit{Dolan} on two
grounds. First, the majority thought it significant that in this case, unlike
\textit{Nollan} or \textit{Dolan}, the District did not approve petitioner's
application on the condition that he accede to the District's demands; instead,
the District denied his application because he refused to make concessions.
Second, the majority drew a distinction between a demand for an interest in real
property (what happened in \textit{Nollan} and \textit{Dolan}) and a demand for
money.\ldots

Recognizing that the majority opinion rested on a question of federal
constitutional law on which the lower courts are divided, we granted the
petition for a writ of certiorari and now reverse.


\readinghead{II}

[The Court held that ``[t]he principles that undergird our decisions in
\textit{Nollan} and \textit{Dolan} do not change depending on whether the
government \textit{approves} a permit on the condition that the applicant turn
over property or \textit{denies} a permit because the applicant refuses to do
so.'']


\readinghead{III}

We turn to the Florida Supreme Court's alternative holding that petitioner's
claim fails because respondent asked him to spend money rather than give up an
easement on his land. A predicate for any unconstitutional conditions claim is
that the government could not have constitutionally ordered the person asserting
the claim to do what it attempted to pressure that person into doing. For that
reason, we began our analysis in both \textit{Nollan} and \textit{Dolan} by
observing that if the government had directly seized the easements it sought to
obtain through the permitting process, it would have committed a \textit{per se}
taking. The Florida Supreme Court held that petitioner's claim fails at this
first step because the subject of the exaction at issue here was money rather
than a more tangible interest in real property. Respondent and the dissent take
the same position, citing the concurring and dissenting opinions in
\textit{Eastern Enterprises v. Apfel}, 524 U.S. 498 (1998), for the proposition
that an obligation to spend money can never provide the basis for a takings
claim.

We note as an initial matter that if we accepted this argument it would be very
easy for land-use permitting officials to evade the limitations of
\textit{Nollan} and \textit{Dolan}. Because the government need only provide a
permit applicant with one alternative that satisfies the nexus and rough
proportionality standards, a permitting authority wishing to exact an easement
could simply give the owner a choice of either surrendering an easement or
making a payment equal to the easement's value.\ldots


\readinghead{A}

In \textit{Eastern Enterprises}, \textit{supra}, the United States retroactively
imposed on a former mining company an obligation to pay for the medical benefits
of retired miners and their families. A four-Justice plurality concluded that
the statute's imposition of retroactive financial liability was so arbitrary
that it violated the Takings Clause. Although Justice KENNEDY concurred in the
result on due process grounds, he joined four other Justices in dissent in
arguing that the Takings Clause does not apply to government-imposed financial
obligations that ``d[o] not operate upon or alter an identified property
interest.'' \textit{Id.}, at 540 (opinion concurring in judgment and dissenting
in part); see \textit{id.}, at 554--556 (BREYER, J., dissenting) (``The `private
property' upon which the [Takings] Clause traditionally has focused is a
specific interest in physical or intellectual property''). Relying on the
concurrence and dissent in \textit{Eastern Enterprises}, respondent argues that
a requirement that petitioner spend money improving public lands could not give
rise to a taking.

Respondent's argument rests on a mistaken premise. Unlike the financial
obligation in \textit{Eastern Enterprises}, the demand for money at issue here
did ``operate upon\ldots an identified property interest'' by directing the
owner of a particular piece of property to make a monetary payment. In this
case, unlike \textit{Eastern Enterprises}, the monetary obligation burdened
petitioner's ownership of a specific parcel of land. In that sense, this case
bears resemblance to our cases holding that the government must pay just
compensation when it takes a lien---a right to receive money that is secured by
a particular piece of property. The fulcrum this case turns on is the direct
link between the government's demand and a specific parcel of real property.
Because of that direct link, this case implicates the central concern of
\textit{Nollan} and \textit{Dolan}: the risk that the government may use its
substantial power and discretion in land-use permitting to pursue governmental
ends that lack an essential nexus and rough proportionality to the effects of
the proposed new use of the specific property at issue, thereby diminishing
without justification the value of the property.

In this case, moreover, petitioner does not ask us to hold that the government
can commit a \textit{regulatory} taking by directing someone to spend money. As
a result, we need not apply \textit{Penn Central}'s ``essentially ad hoc,
factual inquir[y],'' 438 U.S., at 124, at all, much less extend that ``already
difficult and uncertain rule'' to the ``vast category of cases'' in which
someone believes that a regulation is too costly. Instead, petitioner's claim
rests on the more limited proposition that when the government commands the
relinquishment of funds linked to a specific, identifiable property interest
such as a bank account or parcel of real property, a ``\textit{per se} [takings]
approach'' is the proper mode of analysis under the Court's precedent.
\textit{Brown v. Legal Foundation of Wash.}, 538 U.S. 216 (2003).\ldots 


\readinghead{B}

Respondent and the dissent argue that if monetary exactions are made subject to
scrutiny under \textit{Nollan} and \textit{Dolan}, then there will be no
principled way of distinguishing impermissible land-use exactions from property
taxes. We think they exaggerate both the extent to which that problem is unique
to the land-use permitting context and the practical difficulty of
distinguishing between the power to tax and the power to take by eminent domain.

It is beyond dispute that ``[t]axes and user fees\ldots are not `takings.'\,''
\textit{Brown, supra}, at 243, n. 2 (SCALIA, J., dissenting).\ldots This case
therefore does not affect the ability of governments to impose property taxes,
user fees, and similar laws and regulations that may impose financial burdens on
property owners.

At the same time, we have repeatedly found takings where the government, by
confiscating financial obligations, achieved a result that could have been
obtained by imposing a tax. Most recently, in \textit{Brown, supra}, at 232, we
were unanimous in concluding that a State Supreme Court's seizure of the
interest on client funds held in escrow was a taking despite the unquestionable
constitutional propriety of a tax that would have raised exactly the same
revenue. Our holding in \textit{Brown} followed from \textit{Phillips v.
Washington Legal Foundation}, 524 U.S. 156 (1998), and \textit{Webb's Fabulous
Pharmacies, Inc. v. Beckwith}, 449 U.S. 155 (1980), two earlier cases in which
we treated confiscations of money as takings despite their functional similarity
to a tax. Perhaps most closely analogous to the present case, we have repeatedly
held that the government takes property when it seizes liens, and in so ruling
we have never considered whether the government could have achieved an
economically equivalent result through taxation.

Two facts emerge from those cases. The first is that the need to distinguish
taxes from takings is not a creature of our holding today that monetary
exactions are subject to scrutiny under \textit{Nollan} and \textit{Dolan}.
Rather, the problem is inherent in this Court's long-settled view that property
the government could constitutionally demand through its taxing power can also
be taken by eminent domain.

Second, our cases show that teasing out the difference between taxes and takings
is more difficult in theory than in practice. \textit{Brown} is illustrative.
Similar to respondent in this case, the respondents in \textit{Brown} argued
that extending the protections of the Takings Clause to a bank account would
open a Pandora's Box of constitutional challenges to taxes. But also like
respondent here, the \textit{Brown} respondents never claimed that they were
exercising their power to levy taxes when they took the petitioners' property.
Any such argument would have been implausible under state law; in Washington,
taxes are levied by the legislature, not the courts. 

The same dynamic is at work in this case because Florida law greatly
circumscribes respondent's power to tax. If respondent had argued that its
demand for money was a tax, it would have effectively conceded that its denial
of petitioner's permit was improper under Florida law. Far from making that
concession, respondent has maintained throughout this litigation that it
considered petitioner's money to be a substitute for his deeding to the public a
conservation easement on a larger parcel of undeveloped land.

This case does not require us to say more. We need not decide at precisely what
point a land-use permitting charge denominated by the government as a ``tax''
becomes ``so arbitrary\ldots that it was not the exertion of taxation but a
confiscation of property.'' \textit{Brushaber v. Union Pacific R. Co.}, 240 U.S.
1, 24--25 (1916).\ldots


\readinghead{C}

Finally, we disagree with the dissent's forecast that our decision will work a
revolution in land use law by depriving local governments of the ability to
charge reasonable permitting fees. Numerous courts---including courts in many of
our Nation's most populous States---have confronted constitutional challenges to
monetary exactions over the last two decades and applied the standard from
\textit{Nollan} and \textit{Dolan} or something like it. Yet the ``significant
practical harm'' the dissent predicts has not come to pass. That is hardly
surprising, for the dissent is correct that state law normally provides an
independent check on excessive land use permitting fees.\ldots

We hold that the government's demand for property from a land-use permit
applicant must satisfy the requirements of \textit{Nollan} and \textit{Dolan}
even when the government denies the permit and even when its demand is for
money. The Court expresses no view on the merits of petitioner's claim that
respondent's actions here failed to comply with the principles set forth in this
opinion and those two cases. The Florida Supreme Court's judgment is reversed,
and this case is remanded for further proceedings not inconsistent with this
opinion.\ldots

\opinion Justice \textsc{Kagan}, with whom Justice \textsc{Ginsburg}, Justice
\textsc{Breyer}, and Justice \textsc{Sotomayor} join, dissenting.

In the paradigmatic case triggering review under \textit{Nollan} [and]
\textit{Dolan\ldots }, the government approves a building permit on the
condition that the landowner relinquish an interest in real property, like an
easement. The significant legal questions that the Court resolves today are
whether \textit{Nollan} and \textit{Dolan} also apply when that case is varied
in two ways. First, what if the government does not approve the permit, but
instead demands that the condition be fulfilled before it will do so? Second,
what if the condition entails not transferring real property, but simply paying
money? This case also raises other, more fact-specific issues I will address:
whether the government here imposed any condition at all, and whether petitioner
Coy Koontz suffered any compensable injury.

I think the Court gets the first question it addresses right. The
\textit{Nollan}--\textit{Dolan} standard applies not only when the government
approves a development permit conditioned on the owner's conveyance of a
property interest (\textit{i.e.}, imposes a condition subsequent), but also when
the government denies a permit until the owner meets the condition
(\textit{i.e.}, imposes a condition precedent).\ldots So far, we all agree.

Our core disagreement concerns the second question the Court addresses. The
majority extends \textit{Nollan} and \textit{Dolan} to cases in which the
government conditions a permit not on the transfer of real property, but instead
on the payment or expenditure of money. That runs roughshod over \textit{Eastern
Enterprises v. Apfel}, 524 U.S. 498 (1998), which held that the government may
impose ordinary financial obligations without triggering the Takings Clause's
protections. The boundaries of the majority's new rule are uncertain. But it
threatens to subject a vast array of land-use regulations, applied daily in
States and localities throughout the country, to heightened constitutional
scrutiny. I would not embark on so unwise an adventure, and would affirm the
Florida Supreme Court's decision.\ldots


\readinghead{I}

\ldots [T]he \textit{Nollan}--\textit{Dolan} test applies only when the property
the government demands during the permitting process is the kind it otherwise
would have to pay for---or, put differently, when the appropriation of that
property, outside the permitting process, would constitute a taking.\ldots Even
the majority acknowledges this basic point about \textit{Nollan} and
\textit{Dolan}: It too notes that those cases rest on the premise that ``if the
government had directly seized the easements it sought to obtain through the
permitting process, it would have committed a \textit{per se} taking.'' Only if
that is true could the government's demand for the property force a landowner to
relinquish his constitutional right to just compensation.

Here, Koontz claims that the District demanded that he spend money to improve
public wetlands, not that he hand over a real property interest. I assume for
now that the District made that demand (although I think it did not, see
\textit{infra}) The key question then is: Independent of the permitting process,
does requiring a person to pay money to the government, or spend money on its
behalf, constitute a taking requiring just compensation? Only if the answer is
yes does the \textit{Nollan}--\textit{Dolan} test apply.

But we have already answered that question no. [Discussion of \textit{Eastern
Enterprises v. Apfel} omitted.].\ldots

The majority's approach, on top of its analytic flaws, threatens significant
practical harm. By applying \textit{Nollan} and \textit{Dolan} to permit
conditions requiring monetary payments---with no express limitation except as to
taxes---the majority extends the Takings Clause, with its notoriously
``difficult'' and ``perplexing'' standards, into the very heart of local
land-use regulation and service delivery. 524 U.S., at 541. Cities and towns
across the nation impose many kinds of permitting fees every day. Some enable a
government to mitigate a new development's impact on the community, like
increased traffic or pollution---or destruction of wetlands. Others cover the
direct costs of providing services like sewage or water to the development.
Still others are meant to limit the number of landowners who engage in a certain
activity, as fees for liquor licenses do. All now must meet \textit{Nollan} and
\textit{Dolan}'s nexus and proportionality tests. The Federal Constitution thus
will decide whether one town is overcharging for sewage, or another is setting
the price to sell liquor too high. And the flexibility of state and local
governments to take the most routine actions to enhance their communities will
diminish accordingly.

That problem becomes still worse because the majority's distinction between
monetary ``exactions'' and taxes is so hard to apply. The majority acknowledges,
as it must, that taxes are not takings. But once the majority decides that a
simple demand to pay money---the sort of thing often viewed as a tax---can count
as an impermissible ``exaction,'' how is anyone to tell the two apart? The
question, as Justice BREYER's opinion in \textit{Apfel} noted, ``bristles with
conceptual difficulties.'' And practical ones, too: How to separate orders to
pay money from\ldots well, orders to pay money, so that a locality knows what it
can (and cannot) do. State courts sometimes must confront the same question, as
they enforce restrictions on localities' taxing power. And their
decisions---contrary to the majority's blithe assertion---struggle to draw a
coherent boundary.\ldots Nor does the majority's opinion provide any help with
that issue: Perhaps its most striking feature is its refusal to say even a word
about how to make the distinction that will now determine whether a given fee is
subject to heightened scrutiny.

Perhaps the Court means in the future to curb the intrusion into local affairs
that its holding will accomplish; the Court claims, after all, that its opinion
is intended to have only limited impact on localities' land-use authority. The
majority might, for example, approve the rule, adopted in several States, that
\textit{Nollan} and \textit{Dolan} apply only to permitting fees that are
imposed ad hoc, and not to fees that are generally applicable.\ldots Maybe
today's majority accepts that distinction; or then again, maybe not. At the
least, the majority's refusal ``to say more'' about the scope of its new rule
now casts a cloud on every decision by every local government to require a
person seeking a permit to pay or spend money. 

At bottom, the majority's analysis seems to grow out of a yen for a prophylactic
rule: Unless \textit{Nollan} and \textit{Dolan} apply to monetary demands, the
majority worries, ``land-use permitting officials'' could easily ``evade the
limitations'' on exaction of real property interests that those decisions
impose. But that is a prophylaxis in search of a problem. No one has presented
evidence that in the many States declining to apply heightened scrutiny to
permitting fees, local officials routinely short-circuit \textit{Nollan} and
\textit{Dolan} to extort the surrender of real property interests having no
relation to a development's costs. And if officials were to impose a fee as a
contrivance to take an easement (or other real property right), then a court
could indeed apply \textit{Nollan} and \textit{Dolan}. That situation does not
call for a rule extending, as the majority's does, to \textit{all} monetary
exactions. Finally, a court can use the \textit{Penn Central} framework, the Due
Process Clause, and (in many places) state law to protect against monetary
demands, whether or not imposed to evade \textit{Nollan} and \textit{Dolan},
that simply ``go[ ] too far.'' \textit{Mahon}, 260 U.S., at
415.\readingfootnote{3}{Our \textit{Penn Central} test protects against
regulations that unduly burden an owner's use of his property: Unlike the
\textit{Nollan}--\textit{Dolan} standard, that framework fits to a T a complaint
(like Koontz's) that a permitting condition makes it inordinately expensive to
develop land. And the Due Process Clause provides an additional backstop against
excessive permitting fees .\ldots My argument is that our prior caselaw struck
the right balance: heightened scrutiny when the government uses the permitting
process to demand property that the Takings Clause protects, and lesser
scrutiny, but a continuing safeguard against abuse, when the government's demand
is for something falling outside that Clause's scope.}

In sum, \textit{Nollan} and \textit{Dolan} restrain governments from using the
permitting process to do what the Takings Clause would otherwise
prevent---\textit{i.e.}, take a specific property interest without just
compensation. Those cases have no application when governments impose a general
financial obligation as part of the permitting process, because under
\textit{Apfel} such an action does not otherwise trigger the Takings Clause's
protections. By extending \textit{Nollan} and \textit{Dolan}'s heightened
scrutiny to a simple payment demand, the majority threatens the heartland of
local land-use regulation and service delivery, at a bare minimum depriving
state and local governments of ``necessary predictability.'' \textit{Apfel}, 524
U.S., at 542 (opinion of KENNEDY, J.). That decision is unwarranted---and deeply
unwise. I would keep \textit{Nollan} and \textit{Dolan} in their intended sphere
and affirm the Florida Supreme Court.


\readinghead{II}

I also would affirm the judgment below for two independent reasons, even
assuming that a demand for money can trigger \textit{Nollan} and \textit{Dolan}.
First, the District never demanded that Koontz give up anything (including
money) as a condition for granting him a permit. And second, because (as
everyone agrees) no actual taking occurred, Koontz cannot claim just
compensation even had the District made a demand. The majority nonetheless
remands this case on the theory that Koontz might still be entitled to money
damages. I cannot see how, and so would spare the Florida courts.


\readinghead{A}

\textit{Nollan} and \textit{Dolan} apply only when the government makes a
``demand[ ]'' that a landowner turn over property in exchange for a permit.
\textit{Lingle}, 544 U.S., at 546. I understand the majority to agree with that
proposition .\ldots

And unless \textit{Nollan} and \textit{Dolan} are to wreck land-use permitting
throughout the country---to the detriment of both communities and property
owners---that demand must be unequivocal. If a local government risked a lawsuit
every time it made a suggestion to an applicant about how to meet permitting
criteria, it would cease to do so; indeed, the government might desist
altogether from communicating with applicants. That hazard is to some extent
baked into \textit{Nollan} and \textit{Dolan}; observers have wondered whether
those decisions have inclined some local governments to deny permit applications
outright, rather than negotiate agreements that could work to both sides'
advantage. But that danger would rise exponentially if something less than a
clear condition---if each idea or proposal offered in the back-and-forth of
reconciling diverse interests---triggered \textit{Nollan}--\textit{Dolan}
scrutiny. At that point, no local government official with a decent lawyer would
have a conversation with a developer. Hence the need to reserve \textit{Nollan}
and \textit{Dolan}, as we always have, for reviewing only what an official
demands, not all he says in negotiations.

With that as backdrop, consider how this case arose. To arrest the loss of the
State's rapidly diminishing wetlands, Florida law prevents landowners from
filling or draining any such property without two permits. Koontz's property
qualifies as a wetland, and he therefore needed the permits to embark on
development. His applications, however, failed the District's preliminary
review: The District found that they did not preserve wetlands or protect fish
and wildlife to the extent Florida law required. At that point, the District
could simply have denied the applications; had it done so, the \textit{Penn
Central} test---not \textit{Nollan} and \textit{Dolan}---would have governed any
takings claim Koontz might have brought. 

Rather than reject the applications, however, the District suggested to Koontz
ways he could modify them to meet legal requirements. The District proposed
reducing the development's size or modifying its design to lessen the impact on
wetlands. Alternatively, the District raised several options for ``off-site
mitigation'' that Koontz could undertake in a nearby nature preserve, thus
compensating for the loss of wetlands his project would cause. The District
never made any particular demand respecting an off-site project (or anything
else); as Koontz testified at trial, that possibility was presented only in
broad strokes, ``[n]ot in any great detail.'' And the District made clear that
it welcomed additional proposals from Koontz to mitigate his project's damage to
wetlands. Even at the final hearing on his applications, the District asked
Koontz if he would ``be willing to go back with the staff over the next month
and renegotiate this thing and try to come up with'' a solution. But Koontz
refused, saying (through his lawyer) that the proposal he submitted was ``as
good as it can get.'' The District therefore denied the applications, consistent
with its original view that they failed to satisfy Florida law.

In short, the District never made a demand or set a condition---not to cede an
identifiable property interest, not to undertake a particular mitigation
project, not even to write a check to the government. Instead, the District
suggested to Koontz several non-exclusive ways to make his applications conform
to state law. The District's only hard-and-fast requirement was that Koontz do
something---anything---to satisfy the relevant permitting criteria. Koontz's
failure to obtain the permits therefore did not result from his refusal to
accede to an allegedly extortionate demand or condition; rather, it arose from
the legal deficiencies of his applications, combined with his unwillingness to
correct them \textit{by any means.} \textit{Nollan} and \textit{Dolan} were
never meant to address such a run-of-the-mill denial of a land-use permit. As
applications of the unconstitutional conditions doctrine, those decisions
require a condition; and here, there was none.

Indeed, this case well illustrates the danger of extending \textit{Nollan} and
\textit{Dolan} beyond their proper compass. Consider the matter from the
standpoint of the District's lawyer. The District, she learns, has found that
Koontz's permit applications do not satisfy legal requirements. It can deny the
permits on that basis; or it can suggest ways for Koontz to bring his
applications into compliance. If every suggestion could become the subject of a
lawsuit under \textit{Nollan} and \textit{Dolan}, the lawyer can give but one
recommendation: Deny the permits, without giving Koontz any advice---even if he
asks for guidance.\ldots Nothing in the Takings Clause requires that folly. I
would therefore hold that the District did not impose an unconstitutional
condition---because it did not impose a condition at all.


\readinghead{B}

And finally, a third difficulty: Even if (1) money counted as ``specific and
identified propert[y]'' under \textit{Apfel} (though it doesn't), and (2) the
District made a demand for it (though it didn't), (3) Koontz never paid a cent,
so the District took nothing from him. As I have explained, that third point
does not prevent Koontz from suing to invalidate the purported demand as an
unconstitutional condition. But it does mean, as the majority agrees, that
Koontz is not entitled to just compensation under the Takings Clause. He may
obtain monetary relief under the Florida statute he invoked only if it
authorizes damages \textit{beyond} just compensation for a taking.

The majority remands that question to the Florida Supreme Court, and given how
it disposes of the other issues here, I can understand why. As the majority
indicates, a State could decide to create a damages remedy not only for a
taking, but also for an unconstitutional conditions claim predicated on the
Takings Clause. And that question is one of state law, which we usually do well
to leave to state courts.\ldots


\readinghead{III}

\textit{Nollan} and \textit{Dolan} are important decisions, designed to curb
governments from using their power over land-use permitting to extract for free
what the Takings Clause would otherwise require them to pay for. But for no
fewer than three independent reasons, this case does not present that problem.
First and foremost, the government commits a taking only when it appropriates a
specific property interest, not when it requires a person to pay or spend money.
Here, the District never took or threatened such an interest; it tried to
extract from Koontz solely a commitment to spend money to repair public
wetlands. Second, \textit{Nollan} and \textit{Dolan} can operate only when the
government makes a demand of the permit applicant; the decisions' prerequisite,
in other words, is a condition. Here, the District never made such a demand: It
informed Koontz that his applications did not meet legal requirements; it
offered suggestions for bringing those applications into compliance; and it
solicited further proposals from Koontz to achieve the same end. That is not the
stuff of which an unconstitutional condition is made. And third, the Florida
statute at issue here does not, in any event, offer a damages remedy for
imposing such a condition. It provides relief only for a consummated taking,
which did not occur here.

The majority's errors here are consequential. The majority turns a broad array
of local land-use regulations into federal constitutional questions. It deprives
state and local governments of the flexibility they need to enhance their
communities---to ensure environmentally sound and economically productive
development. It places courts smack in the middle of the most everyday local
government activity. As those consequences play out across the country, I
believe the Court will rue today's decision. I respectfully dissent.

