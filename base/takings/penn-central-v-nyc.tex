\reading{Penn Cent. Transp. Co. v. City of New York}

\readingcite{438 U.S. 104 (1978)}

\opinion Mr. Justice \textsc{Brennan} delivered the opinion of the Court.

The question presented is whether a city may, as part of a comprehensive program
to preserve historic landmarks and historic districts, place restrictions on the
development of individual historic landmarks---in addition to those imposed by
applicable zoning ordinances---without effecting a ``taking'' requiring the
payment of ``just compensation.'' Specifically, we must decide whether the
application of New York City's Landmarks Preservation Law to the parcel of land
occupied by Grand Central Terminal has ``taken'' its owners' property in
violation of the Fifth and Fourteenth Amendments.


\readinghead{I}


\readinghead{A}

Over the past 50 years, all 50 States and over 500 municipalities have enacted
laws to encourage or require the preservation of buildings and areas with
historic or aesthetic importance. These nationwide legislative efforts have been
precipitated by two concerns. The first is recognition that, in recent years,
large numbers of historic structures, landmarks, and areas have been destroyed
without adequate consideration of either the values represented therein or the
possibility of preserving the destroyed properties for use in economically
productive ways. The second is a widely shared belief that structures with
special historic, cultural, or architectural significance enhance the quality of
life for all. Not only do these buildings and their workmanship represent the
lessons of the past and embody precious features of our heritage, they serve as
examples of quality for today. ``[H]istoric conservation is but one aspect of
the much larger problem, basically an environmental one, of enhancing---or
perhaps developing for the first time---the quality of life for people.''

New York City, responding to similar concerns and acting pursuant to a New York
State enabling Act, adopted its Landmarks Preservation Law in 1965. See N.Y.C.
Admin. Code, ch. 8--A, \S~205--1.0 \textit{et seq}. (1976). The city acted from
the conviction that ``the standing of [New York City] as a world-wide tourist
center and world capital of business, culture and government'' would be
threatened if legislation were not enacted to protect historic landmarks and
neighborhoods from precipitate decisions to destroy or fundamentally alter their
character. \S~205--1.0(a). The city believed that comprehensive measures to
safeguard desirable features of the existing urban fabric would benefit its
citizens in a variety of ways: \textit{e. g.}, fostering ``civic pride in the
beauty and noble accomplishments of the past''; protecting and enhancing ``the
city's attractions to tourists and visitors''; ``support[ing] and stimul [ating]
business and industry''; ``strengthen[ing] the economy of the city''; and
promoting ``the use of historic districts, landmarks, interior landmarks and
scenic landmarks for the education, pleasure and welfare of the people of the
city.'' \S~205--1.0(b).

The New York City law is typical of many urban landmark laws in that its primary
method of achieving its goals is not by acquisitions of historic
properties,\readingfootnote{6}{The consensus is that widespread public ownership
of historic properties in urban settings is neither feasible nor wise. Public
ownership reduces the tax base, burdens the public budget with costs of
acquisitions and maintenance, and results in the preservation of public
buildings as museums and similar facilities, rather than as economically
productive features of the urban scene. See Wilson \& Winkler, The Response of
State Legislation to Historic Preservation, 36 Law \& Contemp. Prob. 329,
330--331, 339--340 (1971).} but rather by involving public entities in land-use
decisions affecting these properties and providing services, standards,
controls, and incentives that will encourage preservation by private owners and
users. While the law does place special restrictions on landmark properties as a
necessary feature to the attainment of its larger objectives, the major theme of
the law is to ensure the owners of any such properties both a ``reasonable
return'' on their investments and maximum latitude to use their parcels for
purposes not inconsistent with the preservation goals.

The operation of the law can be briefly summarized. The primary responsibility
for administering the law is vested in the Landmarks Preservation Commission
(Commission), a broad based, 11-member agency assisted by a technical staff. The
Commission first performs the function, critical to any landmark preservation
effort, of identifying properties and areas that have ``a special character or
special historical or aesthetic interest or value as part of the development,
heritage or cultural characteristics of the city, state or nation.'' If the
Commission determines, after giving all interested parties an opportunity to be
heard, that a building or area satisfies the ordinance's criteria, it will
designate a building to be a ``landmark,'' situated on a particular ``landmark
site,'' or will designate an area to be a ``historic district.'' After the
Commission makes a designation, New York City's Board of Estimate, after
considering the relationship of the designated property ``to the master plan,
the zoning resolution, projected public improvements and any plans for the
renewal of the area involved,'' may modify or disapprove the designation, and
the owner may seek judicial review of the final designation decision. Thus far,
31 historic districts and over 400 individual landmarks have been finally
designated, and the process is a continuing one.

Final designation as a landmark results in restrictions upon the property
owner's options concerning use of the landmark site. First, the law imposes a
duty upon the owner to keep the exterior features of the building ``in good
repair'' to assure that the law's objectives not be defeated by the landmark's
falling into a state of irremediable disrepair. Second, the Commission must
approve in advance any proposal to alter the exterior architectural features of
the landmark or to construct any exterior improvement on the landmark site, thus
ensuring that decisions concerning construction on the landmark site are made
with due consideration of both the public interest in the maintenance of the
structure and the landowner's interest in use of the property. 

In the event an owner wishes to alter a landmark site, three separate procedures
are available through which administrative approval may be obtained. First, the
owner may apply to the Commission for a ``certificate of no effect on protected
architectural features'': that is, for an order approving the improvement or
alteration on the ground that it will not change or affect any architectural
feature of the landmark and will be in harmony therewith. Denial of the
certificate is subject to judicial review.

Second, the owner may apply to the Commission for a certificate of
``appropriateness.'' Such certificates will be granted if the Commission
concludes---focusing upon aesthetic, historical, and architectural values---that
the proposed construction on the landmark site would not unduly hinder the
protection, enhancement, perpetuation, and use of the landmark. Again, denial of
the certificate is subject to judicial review. Moreover, the owner who is denied
either a certificate of no exterior effect or a certificate of appropriateness
may submit an alternative or modified plan for approval. The final
procedure---seeking a certificate of appropriateness on the ground of
``insufficient return,''---provides special mechanisms, which vary depending on
whether or not the landmark enjoys a tax exemption, to ensure that designation
does not cause economic hardship.

Although the designation of a landmark and landmark site restricts the owner's
control over the parcel, designation also enhances the economic position of the
landmark owner in one significant respect. Under New York City's zoning laws,
owners of real property who have not developed their property to the full extent
permitted by the applicable zoning laws are allowed to transfer development
rights to contiguous parcels on the same city block. A 1968 ordinance gave the
owners of landmark sites additional opportunities to transfer development rights
to other parcels. Subject to a restriction that the floor area of the transferee
lot may not be increased by more than 20\% above its authorized level, the
ordinance permitted transfers from a landmark parcel to property across the
street or across a street intersection. In 1969, the law governing the
conditions under which transfers from landmark parcels could occur was
liberalized, apparently to ensure that the Landmarks Law would not unduly
restrict the development options of the owners of Grand Central Terminal. The
class of recipient lots was expanded to include lots ``across a street and
opposite to another lot or lots which except for the intervention of streets or
street intersections f[or]m a series extending to the lot occupied by the
landmark building [, provided that] all lots [are] in the same ownership.'' New
York City Zoning Resolution 74--79 (emphasis deleted). In addition, the 1969
amendment permits, in highly commercialized areas like midtown Manhattan, the
transfer of all unused development rights to a single parcel. 


\readinghead{B}

This case involves the application of New York City's Landmarks Preservation Law
to Grand Central Terminal (Terminal). The Terminal, which is owned by the Penn
Central Transportation Co. and its affiliates (Penn Central), is one of New York
City's most famous buildings. Opened in 1913, it is regarded not only as
providing an ingenious engineering solution to the problems presented by urban
railroad stations, but also as a magnificent example of the French beaux-arts
style.

The Terminal is located in midtown Manhattan. Its south facade faces 42d Street
and that street's intersection with Park Avenue. At street level, the Terminal
is bounded on the west by Vanderbilt Avenue, on the east by the Commodore Hotel,
and on the north by the Pan-American Building. Although a 20-story office tower,
to have been located above the Terminal, was part of the original design, the
planned tower was never constructed. The Terminal itself is an eight-story
structure which Penn Central uses as a railroad station and in which it rents
space not needed for railroad purposes to a variety of commercial interests. The
Terminal is one of a number of properties owned by appellant Penn Central in
this area of midtown Manhattan.\ldots At least eight of these are eligible to be
recipients of development rights afforded the Terminal by virtue of landmark
designation.

On August 2, 1967, following a public hearing, the Commission designated the
Terminal a ``landmark'' and designated the ``city tax block'' it occupies a
``landmark site.'' The Board of Estimate confirmed this action on September 21,
1967. Although appellant Penn Central had opposed the designation before the
Commission, it did not seek judicial review of the final designation decision.

On January 22, 1968, appellant Penn Central, to increase its income, entered
into a renewable 50-year lease and sublease agreement with appellant UGP
Properties, Inc. (UGP), a wholly owned subsidiary of Union General Properties,
Ltd., a United Kingdom corporation. Under the terms of the agreement, UGP was to
construct a multistory office building above the Terminal. UGP promised to pay
Penn Central \$1 million annually during construction and at least \$3 million
annually thereafter. The rentals would be offset in part by a loss of some
\$700,000 to \$1 million in net rentals presently received from concessionaires
displaced by the new building.

Appellants UGP and Penn Central then applied to the Commission for permission to
construct an office building atop the Terminal. Two separate plans, both
designed by architect Marcel Breuer and both apparently satisfying the terms of
the applicable zoning ordinance, were submitted to the Commission for approval.
The first, Breuer I, provided for the construction of a 55-story office
building, to be cantilevered above the existing facade and to rest on the roof
of the Terminal. The second, Breuer II Revised, called for tearing down a
portion of the Terminal that included the 42d Street facade, stripping off some
of the remaining features of the Terminal's facade, and constructing a 53-story
office building. The Commission denied a certificate of no exterior effect on
September 20, 1968. Appellants then applied for a certificate of
``appropriateness'' as to both proposals. After four days of hearings at which
over 80 witnesses testified, the Commission denied this application as to both
proposals.

\captionedgraphic{takings-img001}{Reproductions of the proposals. From
\protect\url{http://www.architakes.com/?p=13036}.}

The Commission's reasons for rejecting certificates respecting Breuer II Revised
are summarized in the following statement: ``To protect a Landmark, one does not
tear it down. To perpetuate its architectural features, one does not strip them
off.'' Breuer I, which would have preserved the existing vertical facades of the
present structure, received more sympathetic consideration. The Commission first
focused on the effect that the proposed tower would have on one desirable
feature created by the present structure and its surroundings: the dramatic view
of the Terminal from Park Avenue South. Although appellants had contended that
the Pan-American Building had already destroyed the silhouette of the south
facade and that one additional tower could do no further damage and might even
provide a better background for the facade, the Commission disagreed, stating
that it found the majestic approach from the south to be still unique in the
city and that a 55-story tower atop the Terminal would be far more detrimental
to its south facade than the Pan-American Building 375 feet away. Moreover, the
Commission found that from closer vantage points the Pan Am Building and the
other towers were largely cut off from view, which would not be the case of the
mass on top of the Terminal planned under Breuer I. In conclusion, the
Commission stated:
\begin{quotation}
[We have] no fixed rule against making additions to designated buildings---it
all depends on how they are done . . . . But to balance a 55-story office tower
above a flamboyant Beaux-Arts facade seems nothing more than an aesthetic joke.
Quite simply, the tower would overwhelm the Terminal by its sheer mass. The
`addition' would be four times as high as the existing structure and would
reduce the Landmark itself to the status of a curiosity.

Landmarks cannot be divorced from their settings---particularly when the
setting is a dramatic and integral part of the original concept. The Terminal,
in its setting, is a great example of urban design. Such examples are not so
plentiful in New York City that we can afford to lose any of the few we have.
And we must preserve them in a meaningful way---with alterations and additions
of such character, scale, materials and mass as will protect, enhance and
perpetuate the original design rather than overwhelm it.
\end{quotation}

Appellants did not seek judicial review of the denial of either
certificate.\ldots Further, appellants did not avail themselves of the
opportunity to develop and submit other plans for the Commission's consideration
and approval. Instead, appellants filed suit in New York Supreme Court, Trial
Term, claiming, \textit{inter alia}, that the application of the Landmarks
Preservation Law had ``taken'' their property without just compensation in
violation of the Fifth and Fourteenth Amendments and arbitrarily deprived them
of their property without due process of law in violation of the Fourteenth
Amendment. Appellants sought a declaratory judgment, injunctive relief barring
the city from using the Landmarks Law to impede the construction of any
structure that might otherwise lawfully be constructed on the Terminal site, and
damages for the ``temporary taking'' that occurred between August 2, 1967, the
designation date, and the date when the restrictions arising from the Landmarks
Law would be lifted. The trial court granted the injunctive and declaratory
relief, but severed the question of damages for a ``temporary taking.'' [The New
York Supreme Court, Appellate Division, reversed, and this ruling was affirmed
by the state Court of Appeals.]


\readinghead{II}

The issues presented by appellants are (1) whether the restrictions imposed by
New York City's law upon appellants' exploitation of the Terminal site effect a
``taking'' of appellants' property for a public use within the meaning of the
Fifth Amendment, which of course is made applicable to the States through the
Fourteenth Amendment, and, (2), if so, whether the transferable development
rights afforded appellants constitute ``just compensation'' within the meaning
of the Fifth Amendment. We need only address the question whether a ``taking''
has occurred.


\readinghead{A}

\ldots The question of what constitutes a ``taking'' for purposes of the Fifth
Amendment has proved to be a problem of considerable difficulty. While this
Court has recognized that the ``Fifth Amendment's guarantee . . . [is] designed
to bar Government from forcing some people alone to bear public burdens which,
in all fairness and justice, should be borne by the public as a whole,''
\textit{Armstrong v. United States}, 364 U.S. 40, 49 (1960), this Court, quite
simply, has been unable to develop any ``set formula'' for determining when
``justice and fairness'' require that economic injuries caused by public action
be compensated by the government, rather than remain disproportionately
concentrated on a few persons. Indeed, we have frequently observed that whether
a particular restriction will be rendered invalid by the government's failure to
pay for any losses proximately caused by it depends largely ``upon the
particular circumstances [in that] case.'' \textit{United States v. Central
Eureka Mining Co.}, 357 U.S. 155, 168 (1958).

In engaging in these essentially ad hoc, factual inquiries, the Court's
decisions have identified several factors that have particular significance. The
economic impact of the regulation on the claimant and, particularly, the extent
to which the regulation has interfered with distinct investment-backed
expectations are, of course, relevant considerations. So, too, is the character
of the governmental action. A ``taking'' may more readily be found when the
interference with property can be characterized as a physical invasion by
government, than when interference arises from some public program adjusting the
benefits and burdens of economic life to promote the common good.

``Government hardly could go on if to some extent values incident to property
could not be diminished without paying for every such change in the general
law,'' \textit{Pennsylvania Coal Co. v. Mahon}, 260 U.S. 393, 413 (1922), and
this Court has accordingly recognized, in a wide variety of contexts, that
government may execute laws or programs that adversely affect recognized
economic values. Exercises of the taxing power are one obvious example. A second
are the decisions in which this Court has dismissed ``taking'' challenges on the
ground that, while the challenged government action caused economic harm, it did
not interfere with interests that were sufficiently bound up with the reasonable
expectations of the claimant to constitute ``property'' for Fifth Amendment
purposes. See, \textit{e. g., United States v. Willow River Power Co.}, 324 U.S.
499 (1945) (interest in high-water level of river for runoff for tailwaters to
maintain power head is not property); \textit{United States v. Chandler-Dunbar
Water Power Co.}, 229 U.S. 53 (1913).

More importantly for the present case, in instances in which a state tribunal
reasonably concluded that ``the health, safety, morals, or general welfare''
would be promoted by prohibiting particular contemplated uses of land, this
Court has upheld land-use regulations that destroyed or adversely affected
recognized real property interests. Zoning laws are, of course, the classic
example, see \textit{Euclid v. Ambler Realty Co.}, 272 U.S. 365 (1926)
(prohibition of industrial use); \textit{Gorieb v. Fox}, 274 U.S. 603, 608
(1927) (requirement that portions of parcels be left unbuilt); \textit{Welch v.
Swasey}, 214 U.S. 91 (1909) (height restriction), which have been viewed as
permissible governmental action even when prohibiting the most beneficial use of
the property. 

Zoning laws generally do not affect existing uses of real property, but
``taking'' challenges have also been held to be without merit in a wide variety
of situations when the challenged governmental actions prohibited a beneficial
use to which individual parcels had previously been devoted and thus caused
substantial individualized harm. \textit{Miller v. Schoene}, 276 U.S. 272
(1928), is illustrative. In that case, a state entomologist, acting pursuant to
a state statute, ordered the claimants to cut down a large number of ornamental
red cedar trees because they produced cedar rust fatal to apple trees cultivated
nearby. Although the statute provided for recovery of any expense incurred in
removing the cedars, and permitted claimants to use the felled trees, it did not
provide compensation for the value of the standing trees or for the resulting
decrease in market value of the properties as a whole. A unanimous Court held
that this latter omission did not render the statute invalid. The Court held
that the State might properly make ``a choice between the preservation of one
class of property and that of the other'' and since the apple industry was
important in the State involved, concluded that the State had not exceeded ``its
constitutional powers by deciding upon the destruction of one class of property
[without compensation] in order to save another which, in the judgment of the
legislature, is of greater value to the public.'' 

Again, \textit{Hadacheck v. Sebastian}, 239 U.S. 394 (1915), upheld a law
prohibiting the claimant from continuing his otherwise lawful business of
operating a brickyard in a particular physical community on the ground that the
legislature had reasonably concluded that the presence of the brickyard was
inconsistent with neighboring uses.\ldots

\textit{Pennsylvania Coal Co. v. Mahon}, 260 U.S. 393 (1922), is the leading
case for the proposition that a state statute that substantially furthers
important public policies may so frustrate distinct investment-backed
expectations as to amount to a ``taking.'' There the claimant had sold the
surface rights to particular parcels of property, but expressly reserved the
right to remove the coal thereunder. A Pennsylvania statute, enacted after the
transactions, forbade any mining of coal that caused the subsidence of any
house, unless the house was the property of the owner of the underlying coal and
was more than 150 feet from the improved property of another. Because the
statute made it commercially impracticable to mine the coal, and thus had nearly
the same effect as the complete destruction of rights claimant had reserved from
the owners of the surface land, the Court held that the statute was invalid as
effecting a ``taking'' without just compensation. 

Finally, government actions that may be characterized as acquisitions of
resources to permit or facilitate uniquely public functions have often been held
to constitute ``takings.'' \textit{United States v. Causby}, 328 U.S. 256
(1946), is illustrative. In holding that direct overflights above the claimant's
land, that destroyed the present use of the land as a chicken farm, constituted
a ``taking,'' \textit{Causby} emphasized that Government had not ``merely
destroyed property [but was] using a part of it for the flight of its planes.''
\textit{Id.}, 328 U.S., at 262--263, n. 7. 


\readinghead{B}

\ldots Because this Court has recognized, in a number of settings, that States
and cities may enact land-use restrictions or controls to enhance the quality of
life by preserving the character and desirable aesthetic features of a city,
appellants do not contest that New York City's objective of preserving
structures and areas with special historic, architectural, or cultural
significance is an entirely permissible governmental goal. They also do not
dispute that the restrictions imposed on its parcel are appropriate means of
securing the purposes of the New York City law. Finally, appellants do not
challenge any of the specific factual premises of the decision below. They
accept for present purposes both that the parcel of land occupied by Grand
Central Terminal must, in its present state, be regarded as capable of earning a
reasonable return, and that the transferable development rights afforded
appellants by virtue of the Terminal's designation as a landmark are valuable,
even if not as valuable as the rights to construct above the Terminal. In
appellants' view none of these factors derogate from their claim that New York
City's law has effected a ``taking.''

They first observe that the airspace above the Terminal is a valuable property
interest, citing \textit{United States v. Causby, supra.} They urge that the
Landmarks Law has deprived them of any gainful use of their ``air rights'' above
the Terminal and that, irrespective of the value of the remainder of their
parcel, the city has ``taken'' their right to this superadjacent airspace, thus
entitling them to ``just compensation'' measured by the fair market value of
these air rights.

Apart from our own disagreement with appellants' characterization of the effect
of the New York City law, the submission that appellants may establish a
``taking'' simply by showing that they have been denied the ability to exploit a
property interest that they heretofore had believed was available for
development is quite simply untenable. Were this the rule, this Court would have
erred not only in upholding laws restricting the development of air rights, see
\textit{Welch v. Swasey, supra}, but also in approving those prohibiting both
the subjacent, see \textit{Goldblatt v. Hempstead}, 369 U.S. 590 (1962), and the
lateral, see \textit{Gorieb v. Fox}, 274 U.S. 603 development of particular
parcels.\readingfootnote{27}{These cases dispose of any contention that might be
based on \emph{Pennsylvania Coal Co. v. Mahon}, 260 U.S. 393 (1922), that full
use of air rights is so bound up with the investment-backed expectations of
appellants that governmental deprivation of these rights invariably---i. e.,
irrespective of the impact of the restriction on the value of the parcel as a
whole---constitutes a ``taking.'' Similarly, \textit{Welch}, \textit{Goldblatt},
and \textit{Gorieb} illustrate the fallacy of appellants' related contention
that a ``taking'' must be found to have occurred whenever the land-use
restriction may be characterized as imposing a ``servitude'' on the claimant's
parcel.} ``Taking'' jurisprudence does not divide a single parcel into discrete
segments and attempt to determine whether rights in a particular segment have
been entirely abrogated. In deciding whether a particular governmental action
has effected a taking, this Court focuses rather both on the character of the
action and on the nature and extent of the interference with rights in the
parcel as a whole---here, the city tax block designated as the ``landmark
site.''

Secondly, appellants, focusing on the character and impact of the New York City
law, argue that it effects a ``taking'' because its operation has significantly
diminished the value of the Terminal site. Appellants concede that the decisions
sustaining other land-use regulations, which, like the New York City law, are
reasonably related to the promotion of the general welfare, uniformly reject the
proposition that diminution in property value, standing alone, can establish a
``taking,'' see \textit{Euclid v. Ambler Realty Co.}, 272 U.S. 365 (1926) (75\%
diminution in value caused by zoning law); \textit{Hadacheck v. Sebastian}, 239
U.S. 394 (1915) (87 1/2 \% diminution in value), and that the ``taking'' issue
in these contexts is resolved by focusing on the uses the regulations
permit.\ldots [B]ut appellants argue that New York City's regulation of
individual landmarks is fundamentally different from zoning or from
historic-district legislation because the controls imposed by New York City's
law apply only to individuals who own selected properties.

Stated baldly, appellants' position appears to be that the only means of
ensuring that selected owners are not singled out to endure financial hardship
for no reason is to hold that any restriction imposed on individual landmarks
pursuant to the New York City scheme is a ``taking'' requiring the payment of
``just compensation.'' Agreement with this argument would, of course, invalidate
not just New York City's law, but all comparable landmark legislation in the
Nation. We find no merit in it.

It is true, as appellants emphasize, that both historic-district legislation and
zoning laws regulate all properties within given physical communities whereas
landmark laws apply only to selected parcels. But, contrary to appellants'
suggestions, landmark laws are not like discriminatory, or ``reverse spot,''
zoning: that is, a land-use decision which arbitrarily singles out a particular
parcel for different, less favorable treatment than the neighboring ones. In
contrast to discriminatory zoning, which is the antithesis of land-use control
as part of some comprehensive plan, the New York City law embodies a
comprehensive plan to preserve structures of historic or aesthetic interest
wherever they might be found in the city, and as noted, over 400 landmarks and
31 historic districts have been designated pursuant to this plan.

Equally without merit is the related argument that the decision to designate a
structure as a landmark ``is inevitably arbitrary or at least subjective,
because it is basically a matter of taste,'' Reply Brief for Appellants 22, thus
unavoidably singling out individual landowners for disparate and unfair
treatment. The argument has a particularly hollow ring in this case. For
appellants not only did not seek judicial review of either the designation or of
the denials of the certificates of appropriateness and of no exterior effect,
but do not even now suggest that the Commission's decisions concerning the
Terminal were in any sense arbitrary or unprincipled. But, in any event, a
landmark owner has a right to judicial review of any Commission decision, and,
quite simply, there is no basis whatsoever for a conclusion that courts will
have any greater difficulty identifying arbitrary or discriminatory action in
the context of landmark regulation than in the context of classic zoning or
indeed in any other context. 

Next, appellants observe that New York City's law differs from zoning laws and
historic-district ordinances in that the Landmarks Law does not impose identical
or similar restrictions on all structures located in particular physical
communities. It follows, they argue, that New York City's law is inherently
incapable of producing the fair and equitable distribution of benefits and
burdens of governmental action which is characteristic of zoning laws and
historic-district legislation and which they maintain is a constitutional
requirement if ``just compensation'' is not to be afforded. It is, of course,
true that the Landmarks Law has a more severe impact on some landowners than on
others, but that in itself does not mean that the law effects a ``taking.''
Legislation designed to promote the general welfare commonly burdens some more
than others. The owners of the brickyard in \textit{Hadacheck}, of the cedar
trees in \textit{Miller v. Schoene}, and of the gravel and sand mine in
\textit{Goldblatt v. Hempstead}, were uniquely burdened by the legislation
sustained in those cases.\readingfootnote{30}{Appellants attempt to distinguish
these cases on the ground that, in each, government was prohibiting a
``noxious'' use of land and that in the present case, in contrast, appellants'
proposed construction above the Terminal would be beneficial. We observe that
the uses in issue in \textit{Hadacheck}, \textit{Miller}, and \textit{Goldblatt}
were perfectly lawful in themselves. They involved no ``blameworthiness, . . .
moral wrongdoing or conscious act of dangerous risk-taking which induce[d
society] to shift the cost to a pa[rt]icular individual.'' Sax, \textit{Takings
and the Police Power}, 74 Yale L.J. 36, 50 (1964). These cases are better
understood as resting not on any supposed ``noxious'' quality of the prohibited
uses but rather on the ground that the restrictions were reasonably related to
the implementation of a policy---not unlike historic preservation---expected to
produce a widespread public benefit and applicable to all similarly situated
property.\par Nor, correlatively, can it be asserted that the destruction or
fundamental alteration of a historic landmark is not harmful. The suggestion
that the beneficial quality of appellants' proposed construction is established
by the fact that the construction would have been consistent with applicable
zoning laws ignores the development in sensibilities and ideals reflected in
landmark legislation like New York City's.} 30 Similarly, zoning laws often
affect some property owners more severely than others but have not been held to
be invalid on that account. For example, the property owner in \textit{Euclid}
who wished to use its property for industrial purposes was affected far more
severely by the ordinance than its neighbors who wished to use their land for
residences.

In any event, appellants' repeated suggestions that they are solely burdened and
unbenefited is factually inaccurate. This contention overlooks the fact that the
New York City law applies to vast numbers of structures in the city in addition
to the Terminal---all the structures contained in the 31 historic districts and
over 400 individual landmarks, many of which are close to the Terminal. Unless
we are to reject the judgment of the New York City Council that the preservation
of landmarks benefits all New York citizens and all structures, both
economically and by improving the quality of life in the city as a whole---which
we are unwilling to do---we cannot conclude that the owners of the Terminal have
in no sense been benefited by the Landmarks Law. Doubtless appellants believe
they are more burdened than benefited by the law, but that must have been true,
too, of the property owners in \textit{Miller, Hadacheck, Euclid}, and
\textit{Goldblatt}.

Appellants' final broad-based attack would have us treat the law as an instance,
like that in \textit{United States v. Causby}, in which government, acting in an
enterprise capacity, has appropriated part of their property for some strictly
governmental purpose. Apart from the fact that \textit{Causby} was a case of
invasion of airspace that destroyed the use of the farm beneath and this New
York City law has in nowise impaired the present use of the Terminal, the
Landmarks Law neither exploits appellants' parcel for city purposes nor
facilitates nor arises from any entrepreneurial operations of the city. The
situation is not remotely like that in \textit{Causby} where the airspace above
the property was in the flight pattern for military aircraft. The Landmarks
Law's effect is simply to prohibit appellants or anyone else from occupying
portions of the airspace above the Terminal, while permitting appellants to use
the remainder of the parcel in a gainful fashion. This is no more an
appropriation of property by government for its own uses than is a zoning law
prohibiting, for ``aesthetic'' reasons, two or more adult theaters within a
specified area, or a safety regulation prohibiting excavations below a certain
level. 


\readinghead{C}

Rejection of appellants' broad arguments is not, however, the end of our
inquiry, for all we thus far have established is that the New York City law is
not rendered invalid by its failure to provide ``just compensation'' whenever a
landmark owner is restricted in the exploitation of property interests, such as
air rights, to a greater extent than provided for under applicable zoning laws.
We now must consider whether the interference with appellants' property is of
such a magnitude that ``there must be an exercise of eminent domain and
compensation to sustain [it].'' \textit{Pennsylvania Coal Co. v. Mahon}, 260
U.S., at 413. That inquiry may be narrowed to the question of the severity of
the impact of the law on appellants' parcel, and its resolution in turn requires
a careful assessment of the impact of the regulation on the Terminal site.

\ldots [T]he New York City law does not interfere in any way with the present
uses of the Terminal. Its designation as a landmark not only permits but
contemplates that appellants may continue to use the property precisely as it
has been used for the past 65 years: as a railroad terminal containing office
space and concessions. So the law does not interfere with what must be regarded
as Penn Central's primary expectation concerning the use of the parcel. More
importantly, on this record, we must regard the New York City law as permitting
Penn Central not only to profit from the Terminal but also to obtain a
``reasonable return'' on its investment.

Appellants, moreover, exaggerate the effect of the law on their ability to make
use of the air rights above the Terminal in two respects. First, it simply
cannot be maintained, on this record, that appellants have been prohibited from
occupying \textit{any} portion of the airspace above the Terminal. While the
Commission's actions in denying applications to construct an office building in
excess of 50 stories above the Terminal may indicate that it will refuse to
issue a certificate of appropriateness for any comparably sized structure,
nothing the Commission has said or done suggests an intention to prohibit
\textit{any} construction above the Terminal. The Commission's report emphasized
that whether any construction would be allowed depended upon whether the
proposed addition ``would harmonize in scale, material and character with [the
Terminal].'' Since appellants have not sought approval for the construction of a
smaller structure, we do not know that appellants will be denied any use of any
portion of the airspace above the Terminal.

Second, to the extent appellants have been denied the right to build above the
Terminal, it is not literally accurate to say that they have been denied
\textit{all} use of even those pre-existing air rights. Their ability to use
these rights has not been abrogated; they are made transferable to at least
eight parcels in the vicinity of the Terminal, one or two of which have been
found suitable for the construction of new office buildings. Although appellants
and others have argued that New York City's transferable development-rights
program is far from ideal, the New York courts here supportably found that, at
least in the case of the Terminal, the rights afforded are valuable. While these
rights may well not have constituted ``just compensation'' if a ``taking'' had
occurred, the rights nevertheless undoubtedly mitigate whatever financial
burdens the law has imposed on appellants and, for that reason, are to be taken
into account in considering the impact of regulation. 

On this record, we conclude that the application of New York City's Landmarks
Law has not effected a ``taking'' of appellants' property. The restrictions
imposed are substantially related to the promotion of the general welfare and
not only permit reasonable beneficial use of the landmark site but also afford
appellants opportunities further to enhance not only the Terminal site proper
but also other properties.

\textit{Affirmed}.

\opinion Mr. Justice \textsc{Rehnquist}, with whom \textsc{The Chief Justice}
and Mr. Justice \textsc{Stevens} join, dissenting.

Of the over one million buildings and structures in the city of New York,
appellees have singled out 400 for designation as official landmarks. The owner
of a building might initially be pleased that his property has been chosen by a
distinguished committee of architects, historians, and city planners for such a
singular distinction. But he may well discover, as appellant Penn Central
Transportation Co. did here, that the landmark designation imposes upon him a
substantial cost, with little or no offsetting benefit except for the honor of
the designation. The question in this case is whether the cost associated with
the city of New York's desire to preserve a limited number of ``landmarks''
within its borders must be borne by all of its taxpayers or whether it can
instead be imposed entirely on the owners of the individual properties.

Only in the most superficial sense of the word can this case be said to involve
``zoning.'' Typical zoning restrictions may, it is true, so limit the
prospective uses of a piece of property as to diminish the value of that
property in the abstract because it may not be used for the forbidden purposes.
But any such abstract decrease in value will more than likely be at least
partially offset by an increase in value which flows from similar restrictions
as to use on neighboring properties. All property owners in a designated area
are placed under the same restrictions, not only for the benefit of the
municipality as a whole but also for the common benefit of one another. In the
words of Mr. Justice Holmes, speaking for the Court in \textit{Pennsylvania Coal
Co. v. Mahon}, there is ``an average reciprocity of advantage.''

Where a relatively few individual buildings, all separated from one another, are
singled out and treated differently from surrounding buildings, no such
reciprocity exists. The cost to the property owner which results from the
imposition of restrictions applicable only to his property and not that of his
neighbors may be substantial---in this case, several million dollars---with no
comparable reciprocal benefits. And the cost associated with landmark
legislation is likely to be of a completely different order of magnitude than
that which results from the imposition of normal zoning restrictions. Unlike the
regime affected by the latter, the landowner is not simply prohibited from using
his property for certain purposes, while allowed to use it for all other
purposes. Under the historic-landmark preservation scheme adopted by New York,
the property owner is under an affirmative duty to \textit{preserve} his
property \textit{as a landmark} at his own expense. To suggest that because
traditional zoning results in some limitation of use of the property zoned, the
New York City landmark preservation scheme should likewise be upheld, represents
the ultimate in treating as alike things which are different. The rubric of
``zoning'' has not yet sufficed to avoid the well-established proposition that
the Fifth Amendment bars the ``Government from forcing some people alone to bear
public burdens which, in all fairness and justice, should be borne by the public
as a whole.'' \textit{Armstrong v. United States}, 364 U.S. 40, 49 (1960).\ldots


\readinghead{I}

The Fifth Amendment provides in part: ``nor shall private property be taken for
public use, without just compensation.'' In a very literal sense, the actions of
appellees violated this constitutional prohibition. Before the city of New York
declared Grand Central Terminal to be a landmark, Penn Central could have used
its ``air rights'' over the Terminal to build a multistory office building, at
an apparent value of several million dollars per year. Today, the Terminal
cannot be modified in \textit{any} form, including the erection of additional
stories, without the permission of the Landmark Preservation Commission, a
permission which appellants, despite good-faith attempts, have so far been
unable to obtain. Because the Taking Clause of the Fifth Amendment has not
always been read literally, however, the constitutionality of appellees' actions
requires a closer scrutiny of this Court's interpretation of the three key words
in the Taking Clause---``property,'' ``taken,'' and ``just compensation.''


\readinghead{A}

Appellees do not dispute that valuable property rights have been destroyed. And
the Court has frequently emphasized that the term ``property'' as used in the
Taking Clause includes the entire ``group of rights inhering in the citizen's
[ownership].'' \textit{United States v. General Motors Corp.}, 323 U.S. 373
(1945).\ldots

While neighboring landowners are free to use their land and ``air rights'' in
any way consistent with the broad boundaries of New York zoning, Penn Central,
absent the permission of appellees, must forever maintain its property in its
present state. The property has been thus subjected to a nonconsensual servitude
not borne by any neighboring or similar properties.


\readinghead{B}

\ldots [A]n examination of the two exceptions where the destruction of property
does \textit{not} constitute a taking demonstrates that a compensable taking has
occurred here.


\readinghead{1}

As early as 1887, the Court recognized that the government can prevent a
property owner from using his property to injure others without having to
compensate the owner for the value of the forbidden use.\ldots

The nuisance exception to the taking guarantee is not coterminous with the
police power itself. The question is whether the forbidden use is dangerous to
the safety, health, or welfare of others. Thus, in \textit{Curtin v. Benson},
222 U.S. 78 (1911), the Court held that the Government, in prohibiting the owner
of property within the boundaries of Yosemite National Park from grazing cattle
on his property, had taken the owner's property. The Court assumed that the
Government could constitutionally require the owner to fence his land or take
other action to prevent his cattle from straying onto others' land without
compensating him.\ldots

Appellees are not prohibiting a nuisance. The record is clear that the proposed
addition to the Grand Central Terminal would be in full compliance with zoning,
height limitations, and other health and safety requirements. Instead, appellees
are seeking to preserve what they believe to be an outstanding example of
beaux-arts architecture. Penn Central is prevented from further developing its
property basically because \textit{too good} a job was done in designing and
building it. The city of New York, because of its unadorned admiration for the
design, has decided that the owners of the building must preserve it unchanged
for the benefit of sightseeing New Yorkers and tourists.

Unlike land-use regulations, appellees' actions do not merely \textit{prohibit}
Penn Central from using its property in a narrow set of noxious ways. Instead,
appellees have placed an \textit{affirmative} duty on Penn Central to maintain
the Terminal in its present state and in ``good repair.'' Appellants are not
free to use their property as they see fit within broad outer boundaries but
must strictly adhere to their past use except where appellees conclude that
alternative uses would not detract from the landmark. While Penn Central may
continue to use the Terminal as it is presently designed, appellees otherwise
``exercise complete dominion and control over the surface of the land,''
\textit{United States v. Causby}, 328 U.S. 256, 262 (1946), and must compensate
the owner for his loss. ``Property is taken in the constitutional sense when
inroads are made upon an owner's use of it to an extent that, as between private
parties, a servitude has been acquired.'' \textit{United States v. Dickinson},
331 U.S. 745, 748 (1947).


\readinghead{2}

Even where the government prohibits a noninjurious use, the Court has ruled that
a taking does not take place if the prohibition applies over a broad cross
section of land and thereby ``secure[s] an average reciprocity of advantage.''
\textit{Pennsylvania Coal Co. v. Mahon}, 260 U.S., at 415. It is for this reason
that zoning does not constitute a ``taking.'' While zoning at times reduces
\textit{individual} property values, the burden is shared relatively evenly and
it is reasonable to conclude that on the whole an individual who is harmed by
one aspect of the zoning will be benefited by another.

Here, however, a multimillion dollar loss has been imposed on appellants; it is
uniquely felt and is not offset by any benefits flowing from the preservation of
some 400 other ``landmarks'' in New York City. Appellees have imposed a
substantial cost on less than one one-tenth of one percent of the buildings in
New York City for the general benefit of all its people. It is exactly this
imposition of general costs on a few individuals at which the ``taking''
protection is directed.\ldots

As Mr. Justice Holmes pointed out in \textit{Pennsylvania Coal Co. v. Mahon},
``the question at bottom'' in an eminent domain case ``is upon whom the loss of
the changes desired should fall.'' The benefits that appellees believe will flow
from preservation of the Grand Central Terminal will accrue to all the citizens
of New York City. There is no reason to believe that appellants will enjoy a
substantially greater share of these benefits. If the cost of preserving Grand
Central Terminal were spread evenly across the entire population of the city of
New York, the burden per person would be in cents per year---a minor cost
appellees would surely concede for the benefit accrued. Instead, however,
appellees would impose the entire cost of several million dollars per year on
Penn Central. But it is precisely this sort of discrimination that the Fifth
Amendment prohibits.

Appellees in response would argue that a taking only occurs where a property
owner is denied \textit{all} reasonable value of his property. The Court has
frequently held that, even where a destruction of property rights would not
\textit{otherwise} constitute a taking, the inability of the owner to make a
reasonable return on his property requires compensation under the Fifth
Amendment. But the converse is not true. A taking does not become a
noncompensable exercise of police power simply because the government in its
grace allows the owner to make some ``reasonable'' use of his property.\ldots


\readinghead{C}

Appellees, apparently recognizing that the constraints imposed on a landmark
site constitute a taking for Fifth Amendment purposes, do not leave the property
owner empty-handed. As the Court notes, the property owner may theoretically
``transfer'' his previous right to develop the landmark property to adjacent
properties if they are under his control. Appellees have coined this system
``Transfer Development Rights,'' or TDR's.

Of all the terms used in the Taking Clause, ``just compensation'' has the
strictest meaning. The Fifth Amendment does not allow simply an approximate
compensation but requires ``a full and perfect equivalent for the property
taken.'' \textit{Monongahela Navigation Co. v. United States}, 148 U.S., at
326.\ldots

Appellees contend that, even if they have ``taken'' appellants' property, TDR's
constitute ``just compensation.'' Appellants, of course, argue that TDR's are
highly imperfect compensation. Because the lower courts held that there was no
``taking,'' they did not have to reach the question of whether or not just
compensation has already been awarded.\ldots

Because the record on appeal is relatively slim, I would remand to the Court of
Appeals for a determination of whether TDR's constitute a ``full and perfect
equivalent for the property taken.''


\readinghead{II}

Over 50 years ago, Mr. Justice Holmes, speaking for the Court, warned that the
courts were ``in danger of forgetting that a strong public desire to improve the
public condition is not enough to warrant achieving the desire by a shorter cut
than the constitutional way of paying for the change.'' \textit{Pennsylvania
Coal Co. v. Mahon}, 260 U.S., at 416. The Court's opinion in this case
demonstrates that the danger thus foreseen has not abated. The city of New York
is in a precarious financial state, and some may believe that the costs of
landmark preservation will be more easily borne by corporations such as Penn
Central than the overburdened individual taxpayers of New York. But these
concerns do not allow us to ignore past precedents construing the Eminent Domain
Clause to the end that the desire to improve the public condition is, indeed,
achieved by a shorter cut than the constitutional way of paying for the change.

