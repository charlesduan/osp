\expected{cedar-point-v-hassid}

\item \textbf{The baseline scope of the \textit{per se} rule}.
Both the majority and dissent appear to agree that state-mandated intrusions
into the right to exclude are common and that many (most? some?) of these
regulations are not takings. So what makes \textit{this} intrusion a taking as a
categorical matter while, say, a mandatory health inspection is (presumably)
not? 

Let's begin with the apparent scope of the \textit{per se} rule. The
majority declares that the access regulation is a \textit{per se} physical
taking because it ``appropriates a right to invade the growers' property.'' What
does that mean, precisely? Does the ``right to invade'' parallel any property
interest we have discussed in this course? (And does it matter if it doesn't?)
Could any regulation of the right to exclude (e.g., a rent control law) be
similarly characterized as an appropriation of a right to invade? The Court also
states that ``the regulation appropriates for the enjoyment of third parties the
owners' right to exclude.'' Why isn't the same true for the mandate to a
restaurant to admit a health inspector?

As addressed in following notes, the majority proffers arguments for why
health inspection regimes are \textit{not} takings (at least if reasonable in
the Court's eyes) and that concerns to the contrary are ``unfounded.'' Before we
turn to them, there is a question of their purpose. Does the survival of any
access regulation now \textit{depend} on the regulation's ability to fit itself
into one of these exceptions? Stated another way, have we now flipped the
baseline on regulations that regulate the right to exclude from being
presumptively constitutional to presumptively suspect?

\item \textbf{The trespass/takings distinction.} First, the majority observes
that ``our holding does nothing to efface the distinction between trespass and
takings. Isolated physical invasions, not undertaken pursuant to a granted right
of access, are properly assessed as individual torts rather than appropriations
of a property right.'' Justice Breyer's dissent raises the critique that it is
uncertain what will distinguish ``isolated physical invasions'' from
``appropriations.'' In any case, this provision likely has little applicability
to regulatory programs, as inspectors would seem to have a ``granted right of
access.''

\expected{lucas-v-sccc}

\item \textbf{Background principles.} Second, the majority argues that ``many
government-authorized physical invasions will not amount to takings because they
are consistent with longstanding background restrictions on property rights.''
This justification is taken from the discussion in \textit{Lucas} of
``pre-existing limitation upon the land owner's title.'' There, it was invoked
to note a potential exception to the ``wipeout'' categorical rule.
\textit{Lucas} focused on the prospect of a regulation preventing a nuisance,
which, being a nuisance, the landowner never had a right to maintain in the
first place. In \textit{Lucas}, the majority indicated that this class of cases
is small, provoking the Justice Blackmun to observe in dissent that, ``There is
nothing magical in the reasoning of judges long dead. They determined a harm in
the same way as state judges and legislatures do today. If judges in the 18th
and 19th centuries can distinguish a harm from a benefit, why not judges in the
20th century, and if judges can, why not legislators?'' Justice Breyer's dissent
echoes this concern, asking if ``only those exceptions that existed in, say,
1789 count?'' 

For the majority, the relevant background principles are broader than
preventing nuisance and include the expectation that some government activities
may require grants of access. These include matters of public or private
necessity (concerning the need to prevent disaster or ``serious harm''), entry
to effect an arrest, and the conduct of a search. Here, the Court invokes the
body of case law concerning the Fourth Amendment's applicability to
administrative searches. In \textit{Camara v. Municipal Court of City and County
of San Francisco}, 387 U.S. 523 (1967), cited by the majority, the Court held
that a warrant was required before a building inspector could enter premises
without consent. \textit{Camera} explains, however, that the requirements for
probable cause in the administrative context are weaker than in the criminal
setting.
\begin{quote}
This is not to suggest that a health official need show the same kind of proof
to a magistrate to obtain a warrant as one must who would search for the fruits
or instrumentalities of crime. Where considerations of health and safety are
involved, the facts that would justify an inference of ``probable cause'' to
make
an inspection are clearly different from those that would justify such an
inference where a criminal investigation has been undertaken. Experience may
show the need for periodic inspections of certain facilities without a further
showing of cause to believe that substandard conditions dangerous to the public
are being maintained. The passage of a certain period without inspection might
of itself be sufficient in a given situation to justify the issuance of warrant.
\end{quote}
\textit{Id.} at 538. The ins and outs of what the Fourth Amendment
requires in the administrative search context constitute, as might be expected,
a complicated body of law. \textit{See, e.g.}, \textit{New York v. Burger}, 482
U.S. 691, 702-03 (1987) (warrantless inspections of businesses in ``pervasively
regulated'' allowed if industries may be searched without a warrant when they
are justified by a substantial government interest, necessary to further the
regulatory scheme and provide protections to the searched that effectively
substitute for the protections of a warrant). 

\item \textbf{Exactions.} Third, the majority notes that ``the government may
require property owners to cede a right of access as a condition of receiving
certain benefits, without causing a taking.'' Note that this is an invocation of
exactions doctrine, in which the state demands, as a condition to receiving a
government benefit, that the property owner give up an interest that the state
could not take without paying compensation. But \textit{Cedar Point} is not a
review of an exaction, but rather of a state regulation. Normally, economic
regulations are evaluated under rational basis review, but here the Court seems
to suggest that regulations involving access must meet the rough proportionality
test of \textit{Dolan} (at least if they do not want to run afoul of the
\textit{Cedar Point per se} rule). In effect, therefore, a certain class of
regulations must now meet heightened scrutiny. Alternatively, has the Court
greatly expanded the scope of state action considered to be an exaction?

Of course, regulatory regimes will often not reflect the exactions fact
pattern. This can be seen here, as the majority notes that ``the access
regulation is not germane to any benefit provided to agricultural employers or
any risk posed to the public.'' But is that always true of the regulations that
the majority says are safe from \textit{Cedar Point}? And how direct must
``public risk'' be? Are measures to protect endangered species a sufficient
necessity, or will later opinions see them as too attenuated? This ties to the
dissent's question of why ``labor peace'' is not a sufficiently important goal?
Are courts in a better position to make these calls than democratically
accountable actors? Likewise, why cannot the state make certain labor practices
a condition of being able to practice agriculture on an industrial scale? 

To be sure, the majority may just have determined that these interests
are insufficiently important given the relative intrusiveness of the regulation.
If so, that sounds an awful lot like balancing, doesn't it? The very balancing
that the majority says is inappropriate under its perhaps-not-so-categorical
approach.

\item \textbf{A \textit{Cedar Point} hypothetical.} Consider
this law pertaining to land surveyors:
\begin{quote}
A professional land surveyor, or persons under his or her direct
supervision, together with his or her survey party, who, in the course of making
a survey, finds it necessary to go upon the land of a party or parties other
than the one for whom the survey is being made is not liable for civil or
criminal trespass and is liable only for any actual damage done to the land or
property.
\end{quote}
225 \textsc{Ill. Comp. Stat. Ann.} 330/45. Many states have similar measures. Is
this a taking under \textit{Cedar Point}? Is this a regulation about health or
safety? Does it confer a benefit to the owner of the burdened property? What
would be just compensation if this were adjudicated to be a taking?

\defcase{pruneyard}{
parties=PruneYard Shopping Center v. Robins,
cite=447 U.S. 74,
year=1980,
}

\having{loretto-v-teleprompter-qs}{As noted above,
\textit{PruneYard}}{\Inline{pruneyard}\optclause}{\Inline{pruneyard}\optclause}
ruled that it was not a taking for state law to require a
shopping mall to permit speech and petition activities by visitors, and the
opinion evaluated the question using the \textit{Penn Central} factors.
\textit{Cedar Point} distinguishes \textit{PruneYard} as being about space open
to the public. Why does openness to the public matter? What kind of openness to
the public is necessary to be out from under the \textit{Cedar Point per se}
rule? Or is it the significance of the intruder's interest that matters? 

Likewise, what is the precise doctrinal effect of openness to the
public? Does it mean that the property is not subject to the \textit{Cedar
Point} rule---that is that the case does not involve an appropriation of a right
to invade? Or does public openness flag a situation in which one of the
majority's limitations to \textit{Cedar Point}'s scope comes into effect (e.g.,
the openness of the mall to the public allows the state to demand an exaction)?

\item \textbf{How far is too far?} For the dissent, short of a permanent
occupation, state access regulation should be evaluated under \textit{Penn
Central} unless the state's access rises to the level of a discrete property
interest (e.g., being the equivalent of an easement, lease, or some such). If
something allows access but doesn't rise to the level of a recognized property
right, then \textit{Penn Central} should apply. Would that formalist approach be
sufficiently protective of property owners? To what extent might the majority be
reacting to the difficulty to property owners of prevailing under the
\textit{Penn Central} analysis? Do considerations of landowner autonomy play a
role insofar as having to tolerate an unwanted visitor might be seen as a
particularly significant intrusion? On that note, consider this exchange from
oral argument between Justice Barrett and the California Solicitor General:
\begin{quotation}
\textsc{Justice Barrett}:\ldots Let's imagine [my house is] situated on the
corner of two busy streets and a city decides that it would be beneficial to
allow people to protest on my lawn because it's so highly visible to the traffic
that's passing by. But exactly like this one, you know, it says you can do it
120 days a year and three hours at a time just during rush hour. I take it,
under your theory, that's not a per se taking, that would be subject to
\emph{Penn Central}. 

\textsc{Mr.~Mongan}: Yes, that would be a powerful \emph{Penn Central} case.

\textsc{Justice Barrett}: Okay, but why would it be a powerful \emph{Penn
Central}? I mean, in the reply brief, your friends on the other side point out
that the Ninth Circuit and the Federal Circuit couldn't identify any \emph{Penn
Central} cases in which a court has found a taking where the diminution in value
is less than 50 percent. And, surely, my property value hasn't decreased more
than 50 percent as a result of the regulation I just described.\ldots

\textsc{Mr.~Mongan}: [\emph{Penn Central}] says that if there is a regulation
authorizing a physical intrusion, courts should be more likely to find a
taking.\ldots And if there's a concern that courts are not properly applying
\emph{Penn Central} to this type of situation, then the solution would be to
take that type of case, as I mentioned, and clarify how it should apply.\ldots

\textsc{Justice Barrett}:\ldots \emph{Penn Central} is deliberately designed to
be permissive towards regulations given the pervasiveness of regulations on
property use in modern life. And so\ldots it's stacked in favor of regulations.
But\ldots you're saying that physical occupations are different. So, if physical
occupations are different, why isn't the easier way to handle them the rule that
we announced in \emph{Loretto}, which is to say they're subject to a per se
rule?
\end{quotation}

\item \textbf{Just compensation.} Suppose California wants the access regulation
to remain in force. What should it (or the union) have to pay to allow union
representatives on the land under the terms of the regulation? How should the
amount be calculated? Is there a market for such matters that can provide data
on just compensation? If the value is low, does that lower the stakes?
\having{loretto-v-teleprompter-qs}{Recall
\textit{Loretto}, in which the ultimate compensation required proved to be quite
low, indeed.}{}{}

\having{state-v-shack}{%
\item \textbf{What about \textit{State v. Shack}?} Recall
\textit{State v. Shack}. Is the New Jersey Supreme Court's
dictate that}{Later in this book, we will read \emph{State v. Shack}, 58 N.J.
297 (1971), in which the New Jersey Supreme Court wrote that}{In \emph{State v.
Shack}, 58 N.J. 297 (1971), the New Jersey Supreme Court wrote that}
``Title to real property cannot include dominion over the destiny
of persons the owner permits to come upon the premises'' a reflection of New
Jersey background principles that the \textit{Cedar Point} majority would
accept? Note that \textit{Shack} uses the language of necessity (``Hence it has
long been true that necessity, private or public, may justify entry upon the
lands of another\ldots .''). Does the fact that the farmers lived on the land
fit the case within the distinction drawn by Justice Kavanaugh's concurrence? 
