\expected{koontz-v-st-johns}

\item Why aren't taxes takings? Does it make a difference whether there is an
individualized determination about a particular use, and which way should an
individualized determination cut? That is, suppose in order to deal with global
warming, the Miami legislature imposes a new tax of 10\% of the assessed value
of a parcel every time a new building permit for that parcel is granted. The
money will go into a fund to help the city become more flood-resistant. Is this
an unconstitutional exaction? If your answer is yes, what about a new tax of
10\% of the assessed value of \textit{every} parcel, regardless of whether
there's new building on it or not? 

\item What if the condition isn't monetary? Suppose the zoning authority says
``you may build your building, but only if you comply with building codes that
specify a minimum number of exits, minimum width of doors, and multiple other
details.'' Is that an exaction? If not, why not?

\item \textbf{Categorical Exclusions.} Just as some government acts are takings
as a categorical matter; others are categorically excluded. \textit{Koontz}
mentions that taxes and user fees are never takings. Why not? One possibility is
the idea that the private property protected by the Takings Clause only protects
discrete resources, and does not apply to legally obligated acts like the
payment of money. That was the logic of five Justices in \textit{Eastern
Enterprises v. Apfel}, which was discussed and distinguished in \textit{Koontz}.
\emph{E. Enterprises v. Apfel}, 524 U.S. 498, 540 (1998) (Kennedy, J.,
concurring in the judgment and dissenting in part); \textit{id.} at 554 (Breyer,
J., dissenting with three other Justices). 


But can we do more than provide a definitional exclusion? Eduardo Pe\~nalver
observes: 
\begin{quotation}
As Richard Epstein---one of the few scholars to focus substantial effort on the
issue---has noted, ``[t]he taxing power is placed in one compartment; the
takings power in another,'' and scholarly discussion of the conflict between the
two never really gets off the ground. In his book \textit{Takings}, Epstein
invited readers to view the conceptual similarity between takings and taxes as a
reason to dramatically curtail the state's power to tax. Specifically, Epstein
argued that the Takings Clause required the government to adopt a system of
proportional taxation, also known as a ``flat tax.'' This argument flew in the
face of settled constitutional orthodoxy, which since the founding era has
understood the state's power to tax as being virtually plenary.\ldots 

This cool response to Epstein's proposal is unsurprising. The constitutional
doctrine defining the state's power to tax is so entrenched that it is nearly
axiomatic. In contrast, Takings Clause jurisprudence is characterized by nothing
if not the confusion and intense disagreement it generates.\ldots
\end{quotation}
Eduardo Mois\'es Pe\~nalver, \textit{Regulatory Taxings}, 104 \textsc{Colum. L.
Rev}. 2182, 2185-86 (2004) (footnotes omitted). Pe\~nalver draws an opposite
conclusion from Epstein's, noting that the seeming conflict between the two
powers stems not from the reach of the taxing power, but from the fact that
courts have applied the Takings Clause beyond its original understanding as a
simple requirement of compensation when the power of eminent domain is
exercised. If the clause were read more narrowly, the apparent tension would
disappear. On this view, ``Takings are the state's direct appropriation of
parcels of property from individuals through the power of eminent domain, and
taxes are generally applicable measures, enacted under the state's power to tax,
requiring individuals to make payments to the state. Each corresponds to
different and nonoverlapping governmental powers.'' \textit{Id.} at 2188.

There are also government actions that do affect specific pieces of property
that are nonetheless excluded from operation of the Takings Clause. We have
already seen one example in the rule---discussed in the opinions in
\textit{Lucas}---that regulation of a common law nuisance is never a taking.
Other examples include government forfeitures, federal control of navigable
waterways, and the state's right to destroy property to contain the spread of
fire. \textit{See generally} \textsc{David A. Dana \& Thomas W. Merrill,
Takings} 110-120 (Foundation Press 2002); \emph{AmeriSource Corp. v.
United States}, 525 F.3d 1149, 1153 (Fed. Cir. 2008) (``Property seized and
retained pursuant to the police power is not taken for a `public use' in the
context of the Takings Clause.''). What explains these exceptions? Perhaps they,
too, may be understood as simply categorically different government powers
(i.e., if the Takings Clause is read as simply applying to eminent domain, the
existence of regulatory takings notwithstanding). Dana and Merrill suggest that
we might understand these exceptions similarly to the nuisance exclusion---the
powers are within traditional conceptions of the state's police powers, and they
have a long historical pedigree, long enough that property owners may be said to
be on imputed notice that they may be exercised. 

