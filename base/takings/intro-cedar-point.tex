Many property regulations burden the right to exclude without causing a
permanent occupation or otherwise taking a discrete interest in the land (like
an easement). When might regulations of this sort be takings? 

In
\having{loretto-v-teleprompter}{\textit{Loretto}}{\emph{Loretto v. Teleprompter
Manhattan CATV Corp.}, 458 U.S. 419 (1982)}{\emph{Loretto v. Teleprompter
Manhattan CATV Corp.}, 458 U.S. 419 (1982)}, the Supreme Court indicated that
this question should be
evaluated under the \textit{Penn Central} test and not the physical occupation
\textit{per se} rule. The opinion distinguishes government activities resulting
in a \textit{permanent} occupation from those producing a \textit{temporary}
one. In enunciating the distinction, the Court explicitly identified government
rules that might require a landowner to allow unwanted third parties onto the
property as \textit{not} being covered by the \textit{Loretto} categorical rule,
at least in cases where the land was otherwise open to the general public.
\begin{quote}
Another recent case underscores the constitutional distinction between a
permanent occupation and a temporary physical invasion. In \textit{PruneYard
Shopping Center v. Robins}, 447 U.S. 74 (1980), the Court upheld a state
constitutional requirement that shopping center owners permit individuals to
exercise free speech and petition rights on their property, to which they had
already invited the general public. The Court emphasized that the State
Constitution does not prevent the owner from restricting expressive activities
by imposing reasonable time, place, and manner restrictions to minimize
interference with the owner's commercial functions. Since the invasion was
temporary and limited in nature, and since the owner had not exhibited an
interest in excluding all persons from his property, ``the fact that [the
solicitors] may have `physically invaded' [the owners'] property cannot be
viewed as determinative.'' Id., at 84.
\end{quote}
\emph{Loretto}, 458 U.S. at 434.
Notably, in \textit{PruneYard}, the Court applied the \textit{Penn Central}
factors. \emph{PruneYard Shopping Ctr. v. Robins}, 447 U.S. 74, 83 (1980).

In \textit{Cedar Point Nursery v. Hassid}, 141 S. Ct. 2063 (2021), the Court
returned to the question of regulations limiting the exercise of the right to
exclude. California law required agricultural employers to allow union
organizers onto their land to contact workers. In \textit{Loretto}, the Court
appeared to categorize regulations of this sort as being outside its categorical
rule, distinguishing earlier cases that discussed such rules: 
\begin{quotation}
Teleprompter's reliance on labor cases requiring companies to permit access to
union organizers, \textit{see, e.g., Hudgens v. NLRB}, 424 U.S. 507 (1976);
\textit{Central Hardware Co. v. NLRB}, 407 U.S. 539 (1972); \textit{NLRB v.
Babcock \& Wilcox Co}., 351 U.S. 105 (1956), is similarly misplaced. As we
recently explained:

``[T]he allowed intrusion on property rights is limited to that necessary to
facilitate the exercise of employees' \S~7 rights [to organize under the
National Labor Relations Act]. After the requisite need for access to the
employer's property has been shown, the access is limited to (i) union
organizers; (ii) prescribed non-working areas of the employer's premises; and
(iii) the duration of the organization activity. In short, the principle of
accommodation announced in \textit{Babcock} is limited to labor organization
campaigns, and the `yielding' of property rights it may require is both
temporary and limited.'' \textit{Central Hardware Co}., \textit{supra}, at 545.
\end{quotation}
\emph{Loretto}, 458 U.S. at 434 n.11
(1982). \textit{Cedar Point} revisits those precedents and a good deal more.

\expectnext{cedar-point-v-hassid}

