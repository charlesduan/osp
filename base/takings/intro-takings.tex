We now address a final method of resolving incompatible property uses.
\term[eminent domain]{Eminent domain} is the inherent power of the state to
transfer title of
private property into state hands. In the United States, when the government
\term[taking]{takes} land in this manner, it must pay the owner \term{just
compensation}. This
is a constitutional requirement, as the Fifth Amendment provides, ``nor shall
private property be taken for public use, without just compensation.'' This
brief constitutional provision encompasses three distinct issues that we will
deal with in this chapter (though not in this order): (1) has there been a
``taking'' of private property? (2) Is the taking for \term{public use}? And (3)
has ``just compensation'' been provided?

Precedent under the Takings Clause regulates the manner in which the state
directly exercises its eminent domain power. As we will see, however, the clause
also limits the ability of the state to regulate. Property owners sometimes
challenge property regulations as being so onerous that it is as if the state
has appropriated property and compensation is therefore due. Much of the Supreme
Court's takings caselaw concerns these so-called ``regulatory takings.''

