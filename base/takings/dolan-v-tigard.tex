\reading{Dolan v. City of Tigard}

\readingcite{512 U.S. 374 (1994)}

\opinion Chief Justice \textsc{Rehnquist} delivered the opinion of the Court.

Petitioner challenges the decision of the Oregon Supreme Court which held that
the city of Tigard could condition the approval of her building permit on the
dedication of a portion of her property for flood control and traffic
improvements. We granted certiorari to resolve a question left open by our
decision in \textit{Nollan v. California Coastal Comm'n}, 483 U.S. 825 (1987),
of what is the required degree of connection between the exactions imposed by
the city and the projected impacts of the proposed development.


\readinghead{I}

The State of Oregon enacted a comprehensive land use management program in
1973.\ldots Pursuant to the State's requirements, the city of Tigard, a
community of some 30,000 residents on the southwest edge of Portland, developed
a comprehensive plan and codified it in its Community Development Code (CDC).
The CDC requires property owners in the area zoned Central Business District to
comply with a 15\% open space and landscaping requirement, which limits total
site coverage, including all structures and paved parking, to 85\% of the
parcel.\ldots

The city also adopted a Master Drainage Plan (Drainage Plan). The Drainage Plan
noted that flooding occurred in several areas along Fanno Creek, including areas
near petitioner's property. The Drainage Plan also established that the increase
in impervious surfaces associated with continued urbanization would exacerbate
these flooding problems.\ldots

Petitioner Florence Dolan owns a plumbing and electric supply store located on
Main Street in the Central Business District of the city. The store covers
approximately 9,700 square feet on the eastern side of a 1.67-acre parcel, which
includes a gravel parking lot. Fanno Creek flows through the southwestern corner
of the lot and along its western boundary. The year-round flow of the creek
renders the area within the creek's 100-year floodplain virtually unusable for
commercial development. The city's comprehensive plan includes the Fanno Creek
floodplain as part of the city's greenway system.

Petitioner applied to the city for a permit to redevelop the site. Her proposed
plans called for nearly doubling the size of the store to 17,600 square feet and
paving a 39-space parking lot. The existing store, located on the opposite side
of the parcel, would be razed in sections as construction progressed on the new
building. In the second phase of the project, petitioner proposed to build an
additional structure on the northeast side of the site for complementary
businesses and to provide more parking. The proposed expansion and intensified
use are consistent with the city's zoning scheme in the Central Business
District.

The City Planning Commission (Commission) granted petitioner's permit
application subject to conditions imposed by the city's CDC. The CDC establishes
the following standard for site development review approval:
\begin{quote}
Where landfill and/or development is allowed within and adjacent to the
100-year floodplain, the City shall require the dedication of sufficient open
land area for greenway adjoining and within the floodplain. This area shall
include portions at a suitable elevation for the construction of a
pedestrian/bicycle pathway within the floodplain in accordance with the adopted
pedestrian/bicycle plan.
\end{quote}

Thus, the Commission required that petitioner dedicate the portion of her
property lying within the 100-year floodplain for improvement of a storm
drainage system along Fanno Creek and that she dedicate an additional 15-foot
strip of land adjacent to the floodplain as a pedestrian/bicycle pathway. The
dedication required by that condition encompasses approximately 7,000 square
feet, or roughly 10\% of the property. In accordance with city practice,
petitioner could rely on the dedicated property to meet the 15\% open space and
landscaping requirement mandated by the city's zoning scheme. The city would
bear the cost of maintaining a landscaped buffer between the dedicated area and
the new store. 

Petitioner requested variances from the CDC standards. Variances are granted
only where it can be shown that, owing to special circumstances related to a
specific piece of the land, the literal interpretation of the applicable zoning
provisions would cause ``an undue or unnecessary hardship'' unless the variance
is granted.\ldots The Commission denied the request.

The Commission made a series of findings concerning the relationship between the
dedicated conditions and the projected impacts of petitioner's project. First,
the Commission noted that ``[i]t is reasonable to assume that customers and
employees of the future uses of this site could utilize a pedestrian/bicycle
pathway adjacent to this development for their transportation and recreational
needs.'' The Commission noted that the site plan has provided for bicycle
parking in a rack in front of the proposed building and ``[i]t is reasonable to
expect that some of the users of the bicycle parking provided for by the site
plan will use the pathway adjacent to Fanno Creek if it is constructed.'' In
addition, the Commission found that creation of a convenient, safe
pedestrian/bicycle pathway system as an alternative means of transportation
``could offset some of the traffic demand on [nearby] streets and lessen the
increase in traffic congestion.'' 

The Commission went on to note that the required floodplain dedication would be
reasonably related to petitioner's request to intensify the use of the site
given the increase in the impervious surface. The Commission stated that the
``anticipated increased storm water flow from the subject property to an already
strained creek and drainage basin can only add to the public need to manage the
stream channel and floodplain for drainage purposes.'' Based on this anticipated
increased storm water flow, the Commission concluded that ``the requirement of
dedication of the floodplain area on the site is related to the applicant's plan
to intensify development on the site.'' The Tigard City Council approved the
Commission's final order, subject to one minor modification; the city council
reassigned the responsibility for surveying and marking the floodplain area from
petitioner to the city's engineering department. 

Petitioner appealed to the Land Use Board of Appeals (LUBA) on the ground that
the city's dedication requirements were not related to the proposed development,
and, therefore, those requirements constituted an uncompensated taking of her
property under the Fifth Amendment. In evaluating the federal taking claim, LUBA
assumed that the city's findings about the impacts of the proposed development
were supported by substantial evidence. Given the undisputed fact that the
proposed larger building and paved parking area would increase the amount of
impervious surfaces and the runoff into Fanno Creek, LUBA concluded that ``there
is a `reasonable relationship' between the proposed development and the
requirement to dedicate land along Fanno Creek for a greenway.'' With respect to
the pedestrian/bicycle pathway, LUBA noted the Commission's finding that a
significantly larger retail sales building and parking lot would attract larger
numbers of customers and employees and their vehicles. It again found a
``reasonable relationship'' between alleviating the impacts of increased traffic
from the development and facilitating the provision of a pedestrian/bicycle
pathway as an alternative means of transportation. 

[The Oregon Court of Appeals and the Oregon Supreme Court both affirmed.]


\readinghead{II}

The Takings Clause of the Fifth Amendment of the United States Constitution,
made applicable to the States through the Fourteenth Amendment, provides:
``[N]or shall private property be taken for public use, without just
compensation.'' One of the principal purposes of the Takings Clause is ``to bar
Government from forcing some people alone to bear public burdens which, in all
fairness and justice, should be borne by the public as a whole.''
\textit{Armstrong v. United States}, 364 U.S. 40, 49 (1960). Without question,
had the city simply required petitioner to dedicate a strip of land along Fanno
Creek for public use, rather than conditioning the grant of her permit to
redevelop her property on such a dedication, a taking would have occurred. Such
public access would deprive petitioner of the right to exclude others, ``one of
the most essential sticks in the bundle of rights that are commonly
characterized as property.'' \textit{Kaiser Aetna v. United States}, 444 U.S.
164, 176 (1979).

On the other side of the ledger, the authority of state and local governments to
engage in land use planning has been sustained against constitutional challenge
as long ago as our decision in \textit{Village of Euclid v. Ambler Realty Co.},
272 U.S. 365 (1926). ``Government hardly could go on if to some extent values
incident to property could not be diminished without paying for every such
change in the general law.'' \textit{Pennsylvania Coal Co. v. Mahon}, 260 U.S.
393, 413 (1922).\ldots

The sort of land use regulations discussed in the cases just cited, however,
differ in two relevant particulars from the present case. First, they involved
essentially legislative determinations classifying entire areas of the city,
whereas here the city made an adjudicative decision to condition petitioner's
application for a building permit on an individual parcel. Second, the
conditions imposed were not simply a limitation on the use petitioner might make
of her own parcel, but a requirement that she deed portions of the property to
the city. In \textit{Nollan, supra}, we held that governmental authority to
exact such a condition was circumscribed by the Fifth and Fourteenth Amendments.
Under the well-settled doctrine of ``unconstitutional conditions,'' the
government may not require a person to give up a constitutional right-here the
right to receive just compensation when property is taken for a public use-in
exchange for a discretionary benefit conferred by the government where the
benefit sought has little or no relationship to the property. 

Petitioner contends that the city has forced her to choose between the building
permit and her right under the Fifth Amendment to just compensation for the
public easements. Petitioner does not quarrel with the city's authority to exact
some forms of dedication as a condition for the grant of a building permit, but
challenges the showing made by the city to justify these exactions.\ldots


\readinghead{III}

In evaluating petitioner's claim, we must first determine whether the
``essential nexus'' exists between the ``legitimate state interest'' and the
permit condition exacted by the city. \textit{Nollan}, 483 U.S., at 837. If we
find that a nexus exists, we must then decide the required degree of connection
between the exactions and the projected impact of the proposed development. We
were not required to reach this question in \textit{Nollan}, because we
concluded that the connection did not meet even the loosest standard. Here,
however, we must decide this question.


\readinghead{A}

We addressed the essential nexus question in \textit{Nollan.}\ldots The
California Coastal Commission demanded a lateral public easement across the
Nollans' beachfront lot in exchange for a permit to demolish an existing
bungalow and replace it with a three-bedroom house.\ldots

We agreed that the Coastal Commission's concern with protecting visual access to
the ocean constituted a legitimate public interest.\ldots We resolved, however,
that the Coastal Commission's regulatory authority was set completely adrift
from its constitutional moorings when it claimed that a nexus existed between
visual access to the ocean and a permit condition requiring lateral public
access along the Nollans' beachfront lot.\ldots The absence of a nexus left the
Coastal Commission in the position of simply trying to obtain an easement
through gimmickry\ldots.

No such gimmicks are associated with the permit conditions imposed by the city
in this case. Undoubtedly, the prevention of flooding along Fanno Creek and the
reduction of traffic congestion in the Central Business District qualify as the
type of legitimate public purposes we have upheld. It seems equally obvious that
a nexus exists between preventing flooding along Fanno Creek and limiting
development within the creek's 100-year floodplain. Petitioner proposes to
double the size of her retail store and to pave her now-gravel parking lot,
thereby expanding the impervious surface on the property and increasing the
amount of storm water runoff into Fanno Creek.

The same may be said for the city's attempt to reduce traffic congestion by
providing for alternative means of transportation. In theory, a
pedestrian/bicycle pathway provides a useful alternative means of transportation
for workers and shoppers\ldots.


\readinghead{B}

The second part of our analysis requires us to determine whether the degree of
the exactions demanded by the city's permit conditions bears the required
relationship to the projected impact of petitioner's proposed development.\ldots

The city required that petitioner dedicate ``to the City as Greenway all
portions of the site that fall within the existing 100-year floodplain [of Fanno
Creek]\ldots and all property 15 feet above [the floodplain] boundary.'' In
addition, the city demanded that the retail store be designed so as not to
intrude into the greenway area. The city relies on the Commission's rather
tentative findings that increased storm water flow from petitioner's property
``can only add to the public need to manage the [floodplain] for drainage
purposes'' to support its conclusion that the ``requirement of dedication of the
floodplain area on the site is related to the applicant's plan to intensify
development on the site.'' 

The city made the following specific findings relevant to the pedestrian/bicycle
pathway:
\begin{quote}
In addition, the proposed expanded use of this site is anticipated to generate
additional vehicular traffic thereby increasing congestion on nearby collector
and arterial streets. Creation of a convenient, safe pedestrian/bicycle pathway
system as an alternative means of transportation could offset some of the
traffic demand on these nearby streets and lessen the increase in traffic
congestion.
\end{quote}

The question for us is whether these findings are constitutionally sufficient to
justify the conditions imposed by the city on petitioner's building permit.
Since state courts have been dealing with this question a good deal longer than
we have, we turn to representative decisions made by them.

In some States, very generalized statements as to the necessary connection
between the required dedication and the proposed development seem to suffice. We
think this standard is too lax to adequately protect petitioner's right to just
compensation if her property is taken for a public purpose.

Other state courts require a very exacting correspondence, described as the
``specifi[c] and uniquely attributable'' test. The Supreme Court of Illinois
first developed this test in \textit{Pioneer Trust \& Savings Bank v. Mount
Prospect}, 22 Ill.2d 375, 380, 176 N.E.2d 799, 802 (1961). Under this standard,
if the local government cannot demonstrate that its exaction is directly
proportional to the specifically created need, the exaction becomes ``a veiled
exercise of the power of eminent domain and a confiscation of private property
behind the defense of police regulations.'' \textit{Id.}, at 381, 176 N.E.2d, at
802. We do not think the Federal Constitution requires such exacting scrutiny,
given the nature of the interests involved.

A number of state courts have taken an intermediate position, requiring the
municipality to show a ``reasonable relationship'' between the required
dedication and the impact of the proposed development. Typical is the Supreme
Court of Nebraska's opinion in \textit{Simpson v. North Platte}, 206 Neb. 240,
245, 292 N.W.2d 297, 301 (1980), where that court stated:
\begin{quote}
The distinction, therefore, which must be made between an appropriate exercise
of the police power and an improper exercise of eminent domain is whether the
requirement has some reasonable relationship or nexus to the use to which the
property is being made or is merely being used as an excuse for taking property
simply because at that particular moment the landowner is asking the city for
some license or permit.
\end{quote}

Thus, the court held that a city may not require a property owner to dedicate
private property for some future public use as a condition of obtaining a
building permit when such future use is not ``occasioned by the construction
sought to be permitted.'' \textit{Id.}, at 248, 292 N.W.2d, at 302.

Some form of the reasonable relationship test has been adopted in many other
jurisdictions. Despite any semantical differences, general agreement exists
among the courts ``that the dedication should have some reasonable relationship
to the needs created by the [development].'' 

We think the ``reasonable relationship'' test adopted by a majority of the state
courts is closer to the federal constitutional norm than either of those
previously discussed. But we do not adopt it as such, partly because the term
``reasonable relationship'' seems confusingly similar to the term ``rational
basis'' which describes the minimal level of scrutiny under the Equal Protection
Clause of the Fourteenth Amendment. We think a term such as ``rough
proportionality'' best encapsulates what we hold to be the requirement of the
Fifth Amendment. No precise mathematical calculation is required, but the city
must make some sort of individualized determination that the required dedication
is related both in nature and extent to the impact of the proposed
development.\ldots

\ldots We turn now to analysis of whether the findings relied upon by the city
here, first with respect to the floodplain easement, and second with respect to
the pedestrian/bicycle path, satisfied these requirements.

It is axiomatic that increasing the amount of impervious surface will increase
the quantity and rate of storm water flow from petitioner's property. Therefore,
keeping the floodplain open and free from development would likely confine the
pressures on Fanno Creek created by petitioner's development. In fact, because
petitioner's property lies within the Central Business District, the CDC already
required that petitioner leave 15\% of it as open space and the undeveloped
floodplain would have nearly satisfied that requirement. But the city demanded
more---it not only wanted petitioner not to build in the floodplain, but it also
wanted petitioner's property along Fanno Creek for its greenway system. The city
has never said why a public greenway, as opposed to a private one, was required
in the interest of flood control.

The difference to petitioner, of course, is the loss of her ability to exclude
others. As we have noted, this right to exclude others is ``one of the most
essential sticks in the bundle of rights that are commonly characterized as
property.'' \textit{Kaiser Aetna}, 444 U.S., at 176. It is difficult to see why
recreational visitors trampling along petitioner's floodplain easement are
sufficiently related to the city's legitimate interest in reducing flooding
problems along Fanno Creek, and the city has not attempted to make any
individualized determination to support this part of its request.\ldots

If petitioner's proposed development had somehow encroached on existing greenway
space in the city, it would have been reasonable to require petitioner to
provide some alternative greenway space for the public either on her property or
elsewhere. But that is not the case here. We conclude that the findings upon
which the city relies do not show the required reasonable relationship between
the floodplain easement and the petitioner's proposed new building.

With respect to the pedestrian/bicycle pathway, we have no doubt that the city
was correct in finding that the larger retail sales facility proposed by
petitioner will increase traffic on the streets of the Central Business
District. The city estimates that the proposed development would generate
roughly 435 additional trips per day. Dedications for streets, sidewalks, and
other public ways are generally reasonable exactions to avoid excessive
congestion from a proposed property use. But on the record before us, the city
has not met its burden of demonstrating that the additional number of vehicle
and bicycle trips generated by petitioner's development reasonably relate to the
city's requirement for a dedication of the pedestrian/bicycle pathway easement.
The city simply found that the creation of the pathway ``could offset some of
the traffic demand\ldots and lessen the increase in traffic congestion.''

As Justice Peterson of the Supreme Court of Oregon explained in his dissenting
opinion, however, ``[t]he findings of fact that the bicycle pathway system
`\textit{could} offset some of the traffic demand' is a far cry from a finding
that the bicycle pathway system \textit{will}, or is \textit{likely to}, offset
some of the traffic demand.'' 317 Ore., at 127, 854 P.2d, at 447 (emphasis in
original). No precise mathematical calculation is required, but the city must
make some effort to quantify its findings in support of the dedication for the
pedestrian/bicycle pathway beyond the conclusory statement that it could offset
some of the traffic demand generated.


\readinghead{IV}

Cities have long engaged in the commendable task of land use planning, made
necessary by increasing urbanization, particularly in metropolitan areas such as
Portland. The city's goals of reducing flooding hazards and traffic congestion,
and providing for public greenways, are laudable, but there are outer limits to
how this may be done. ``A strong public desire to improve the public condition
[will not] warrant achieving the desire by a shorter cut than the constitutional
way of paying for the change.'' \textit{Pennsylvania Coal}, 260 U.S., at 416.

The judgment of the Supreme Court of Oregon is reversed, and the case is
remanded for further proceedings not inconsistent with this opinion.

\textit{It is so ordered.}

[The dissenting opinions of Justices Stevens (joined by Justices Blackmun and
Ginsburg) and of Justice Souter are omitted.]

