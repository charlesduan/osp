\reading{Kelo v. City of New London}

\readingcite{545 U.S. 469 (2005)}

\opinion Justice \textsc{Stevens} delivered the opinion of the Court.

In 2000, the city of New London approved a development plan that, in the words
of the Supreme Court of Connecticut, was ``projected to create in excess of
1,000 jobs, to increase tax and other revenues, and to revitalize an
economically distressed city, including its downtown and waterfront areas.'' In
assembling the land needed for this project, the city's development agent has
purchased property from willing sellers and proposes to use the power of eminent
domain to acquire the remainder of the property from unwilling owners in
exchange for just compensation. The question presented is whether the city's
proposed disposition of this property qualifies as a ``public use'' within the
meaning of the Takings Clause of the Fifth Amendment to the Constitution.


\readinghead{I}

The city of New London (hereinafter City) sits at the junction of the Thames
River and the Long Island Sound in southeastern Connecticut. Decades of economic
decline led a state agency in 1990 to designate the City a ``distressed
municipality.'' In 1996, the Federal Government closed the Naval Undersea
Warfare Center, which had been located in the Fort Trumbull area of the City and
had employed over 1,500 people. In 1998, the City's unemployment rate was nearly
double that of the State, and its population of just under 24,000 residents was
at its lowest since 1920.

These conditions prompted state and local officials to target New London, and
particularly its Fort Trumbull area, for economic revitalization. To this end,
respondent New London Development Corporation (NLDC), a private nonprofit entity
established some years earlier to assist the City in planning economic
development, was reactivated. In January 1998, the State authorized a \$5.35
million bond issue to support the NLDC's planning activities and a \$10 million
bond issue toward the creation of a Fort Trumbull State Park. In February, the
pharmaceutical company Pfizer Inc. announced that it would build a \$300 million
research facility on a site immediately adjacent to Fort Trumbull; local
planners hoped that Pfizer would draw new business to the area, thereby serving
as a catalyst to the area's rejuvenation. After receiving initial approval from
the city council, the NLDC continued its planning activities and held a series
of neighborhood meetings to educate the public about the process.\ldots Upon
obtaining state-level approval, the NLDC finalized an integrated development
plan focused on 90 acres of the Fort Trumbull area.

The Fort Trumbull area is situated on a peninsula that juts into the Thames
River. The area comprises approximately 115 privately owned properties, as well
as the 32 acres of land formerly occupied by the naval facility (Trumbull State
Park now occupies 18 of those 32 acres). The development plan encompasses seven
parcels. Parcel 1 is designated for a waterfront conference hotel at the center
of a ``small urban village'' that will include restaurants and shopping. This
parcel will also have marinas for both recreational and commercial uses. A
pedestrian ``riverwalk'' will originate here and continue down the coast,
connecting the waterfront areas of the development. Parcel 2 will be the site of
approximately 80 new residences organized into an urban neighborhood and linked
by public walkway to the remainder of the development, including the state park.
This parcel also includes space reserved for a new U.S. Coast Guard Museum.
Parcel 3, which is located immediately north of the Pfizer facility, will
contain at least 90,000 square feet of research and development office space.
Parcel 4A is a 2.4-acre site that will be used either to support the adjacent
state park, by providing parking or retail services for visitors, or to support
the nearby marina. Parcel 4B will include a renovated marina, as well as the
final stretch of the riverwalk. Parcels 5, 6, and 7 will provide land for office
and retail space, parking, and water-dependent commercial uses. 

The NLDC intended the development plan to capitalize on the arrival of the
Pfizer facility and the new commerce it was expected to attract. In addition to
creating jobs, generating tax revenue, and helping to ``build momentum for the
revitalization of downtown New London,'' the plan was also designed to make the
City more attractive and to create leisure and recreational opportunities on the
waterfront and in the park.

The city council approved the plan in January 2000, and designated the NLDC as
its development agent in charge of implementation. The city council also
authorized the NLDC to purchase property or to acquire property by exercising
eminent domain in the City's name. The NLDC successfully negotiated the purchase
of most of the real estate in the 90-acre area, but its negotiations with
petitioners failed. As a consequence, in November 2000, the NLDC initiated the
condemnation proceedings that gave rise to this case.


\readinghead{II}

Petitioner Susette Kelo has lived in the Fort Trumbull area since 1997. She has
made extensive improvements to her house, which she prizes for its water view.
Petitioner Wilhelmina Dery was born in her Fort Trumbull house in 1918 and has
lived there her entire life. Her husband Charles (also a petitioner) has lived
in the house since they married some 60 years ago. In all, the nine petitioners
own 15 properties in Fort Trumbull---4 in parcel 3 of the development plan and
11 in parcel 4A. Ten of the parcels are occupied by the owner or a family
member; the other five are held as investment properties. There is no allegation
that any of these properties is blighted or otherwise in poor condition; rather,
they were condemned only because they happen to be located in the development
area.

In December 2000, petitioners brought this action in the New London Superior
Court. They claimed, among other things, that the taking of their properties
would violate the ``public use'' restriction in the Fifth Amendment. After a
7-day bench trial, the Superior Court granted a permanent restraining order
prohibiting the taking of the properties located in parcel 4A (park or marina
support). It, however, denied petitioners relief as to the properties located in
parcel 3 (office space). 

After the Superior Court ruled, both sides took appeals to the Supreme Court of
Connecticut. That court held, over a dissent, that all of the City's proposed
takings were valid. It began by upholding the lower court's determination that
the takings were authorized by chapter 132, the State's municipal development
statute. That statute expresses a legislative determination that the taking of
land, even developed land, as part of an economic development project is a
``public use'' and in the ``public interest.'' Next, relying on cases such as
\textit{Hawaii Housing Authority v. Midkiff}, 467 U.S. 229 (1984), and
\textit{Berman v. Parker}, 348 U.S. 26 (1954), the court held that such economic
development qualified as a valid public use under both the Federal and State
Constitutions. 

Finally, adhering to its precedents, the court went on to determine, first,
whether the takings of the particular properties at issue were ``reasonably
necessary'' to achieving the City's intended public use and, second, whether the
takings were for ``reasonably foreseeable needs.'' The court upheld the trial
court's factual findings as to parcel 3, but reversed the trial court as to
parcel 4A, agreeing with the City that the intended use of this land was
sufficiently definite and had been given ``reasonable attention'' during the
planning process. 

The three dissenting justices would have imposed a ``heightened'' standard of
judicial review for takings justified by economic development. Although they
agreed that the plan was intended to serve a valid public use, they would have
found all the takings unconstitutional because the City had failed to adduce
``clear and convincing evidence'' that the economic benefits of the plan would
in fact come to pass. 

We granted certiorari to determine whether a city's decision to take property
for the purpose of economic development satisfies the ``public use'' requirement
of the Fifth Amendment. 


\readinghead{III}

Two polar propositions are perfectly clear. On the one hand, it has long been
accepted that the sovereign may not take the property of \textit{A} for the sole
purpose of transferring it to another private party \textit{B}, even though
\textit{A} is paid just compensation. On the other hand, it is equally clear
that a State may transfer property from one private party to another if future
``use by the public'' is the purpose of the taking; the condemnation of land for
a railroad with common-carrier duties is a familiar example. Neither of these
propositions, however, determines the disposition of this case.

As for the first proposition, the City would no doubt be forbidden from taking
petitioners' land for the purpose of conferring a private benefit on a
particular private party. Nor would the City be allowed to take property under
the mere pretext of a public purpose, when its actual purpose was to bestow a
private benefit. The takings before us, however, would be executed pursuant to a
``carefully considered'' development plan. The trial judge and all the members
of the Supreme Court of Connecticut agreed that there was no evidence of an
illegitimate purpose in this case.\ldots

On the other hand, this is not a case in which the City is planning to open the
condemned land---at least not in its entirety---to use by the general public.
Nor will the private lessees of the land in any sense be required to operate
like common carriers, making their services available to all comers. But
although such a projected use would be sufficient to satisfy the public use
requirement, this ``Court long ago rejected any literal requirement that
condemned property be put into use for the general public.'' [\textit{Midkiff},
467 U.S.] at 244. Indeed, while many state courts in the mid-19th century
endorsed ``use by the public'' as the proper definition of public use, that
narrow view steadily eroded over time. Not only was the ``use by the public''
test difficult to administer (\textit{e.g.}, what proportion of the public need
have access to the property? at what price?), but it proved to be impractical
given the diverse and always evolving needs of society.\ldots Thus, in a case
upholding a mining company's use of an aerial bucket line to transport ore over
property it did not own, Justice Holmes' opinion for the Court stressed ``the
inadequacy of use by the general public as a universal test.'' \textit{Strickley
v. Highland Boy Gold Mining Co.}, 200 U.S. 527, 531 (1906). We have repeatedly
and consistently rejected that narrow test ever since.

The disposition of this case therefore turns on the question whether the City's
development plan serves a ``public purpose.'' Without exception, our cases have
defined that concept broadly, reflecting our longstanding policy of deference to
legislative judgments in this field.

In \textit{Berman v. Parker}, 348 U.S. 26 (1954), this Court upheld a
redevelopment plan targeting a blighted area of Washington, D. C., in which most
of the housing for the area's 5,000 inhabitants was beyond repair. Under the
plan, the area would be condemned and part of it utilized for the construction
of streets, schools, and other public facilities. The remainder of the land
would be leased or sold to private parties for the purpose of redevelopment,
including the construction of low-cost housing.

The owner of a department store located in the area challenged the condemnation,
pointing out that his store was not itself blighted and arguing that the
creation of a ``better balanced, more attractive community'' was not a valid
public use. Writing for a unanimous Court, Justice Douglas refused to evaluate
this claim in isolation, deferring instead to the legislative and agency
judgment that the area ``must be planned as a whole'' for the plan to be
successful. The Court explained that ``community redevelopment programs need
not, by force of the Constitution, be on a piecemeal basis---lot by lot,
building by building.'' The public use underlying the taking was unequivocally
affirmed:
\begin{quote}
We do not sit to determine whether a particular housing project is or is not
desirable. The concept of the public welfare is broad and inclusive\ldots. The
values it represents are spiritual as well as physical, aesthetic as well as
monetary. It is within the power of the legislature to determine that the
community should be beautiful as well as healthy, spacious as well as clean,
well-balanced as well as carefully patrolled. In the present case, the Congress
and its authorized agencies have made determinations that take into account a
wide variety of values. It is not for us to reappraise them. If those who govern
the District of Columbia decide that the Nation's Capital should be beautiful as
well as sanitary, there is nothing in the Fifth Amendment that stands in the
way.
\end{quote}

In \textit{Hawaii Housing Authority v. Midkiff}, 467 U.S. 229 (1984), the Court
considered a Hawaii statute whereby fee title was taken from lessors and
transferred to lessees (for just compensation) in order to reduce the
concentration of land ownership. We unanimously upheld the statute and rejected
the Ninth Circuit's view that it was ``a naked attempt on the part of the state
of Hawaii to take the property of A and transfer it to B solely for B's private
use and benefit.'' Reaffirming \textit{Berman's} deferential approach to
legislative judgments in this field, we concluded that the State's purpose of
eliminating the ``social and economic evils of a land oligopoly'' qualified as a
valid public use. Our opinion also rejected the contention that the mere fact
that the State immediately transferred the properties to private individuals
upon condemnation somehow diminished the public character of the taking. ``[I]t
is only the taking's purpose, and not its mechanics,'' we explained, that
matters in determining public use.\ldots

Viewed as a whole, our jurisprudence has recognized that the needs of society
have varied between different parts of the Nation, just as they have evolved
over time in response to changed circumstances.\ldots For more than a century,
our public use jurisprudence has wisely eschewed rigid formulas and intrusive
scrutiny in favor of affording legislatures broad latitude in determining what
public needs justify the use of the takings power.


\readinghead{IV}

Those who govern the City were not confronted with the need to remove blight in
the Fort Trumbull area, but their determination that the area was sufficiently
distressed to justify a program of economic rejuvenation is entitled to our
deference. The City has carefully formulated an economic development plan that
it believes will provide appreciable benefits to the community, including---but
by no means limited to---new jobs and increased tax revenue. As with other
exercises in urban planning and development, the City is endeavoring to
coordinate a variety of commercial, residential, and recreational uses of land,
with the hope that they will form a whole greater than the sum of its parts. To
effectuate this plan, the City has invoked a state statute that specifically
authorizes the use of eminent domain to promote economic development. Given the
comprehensive character of the plan, the thorough deliberation that preceded its
adoption, and the limited scope of our review, it is appropriate for us, as it
was in \textit{Berman}, to resolve the challenges of the individual owners, not
on a piecemeal basis, but rather in light of the entire plan. Because that plan
unquestionably serves a public purpose, the takings challenged here satisfy the
public use requirement of the Fifth Amendment.

To avoid this result, petitioners urge us to adopt a new bright-line rule that
economic development does not qualify as a public use. Putting aside the
unpersuasive suggestion that the City's plan will provide only purely economic
benefits, neither precedent nor logic supports petitioners' proposal. Promoting
economic development is a traditional and long-accepted function of government.
There is, moreover, no principled way of distinguishing economic development
from the other public purposes that we have recognized. In our cases upholding
takings that facilitated agriculture and mining, for example, we emphasized the
importance of those industries to the welfare of the States in question\ldots .
It would be incongruous to hold that the City's interest in the economic
benefits to be derived from the development of the Fort Trumbull area has less
of a public character than any of those other interests. Clearly, there is no
basis for exempting economic development from our traditionally broad
understanding of public purpose.

Petitioners contend that using eminent domain for economic development
impermissibly blurs the boundary between public and private takings. Again, our
cases foreclose this objection. Quite simply, the government's pursuit of a
public purpose will often benefit individual private parties.\ldots The owner of
the department store in \textit{Berman} objected to ``taking from one
businessman for the benefit of another businessman,'' referring to the fact that
under the redevelopment plan land would be leased or sold to private developers
for redevelopment. Our rejection of that contention has particular relevance to
the instant case: ``The public end may be as well or better served through an
agency of private enterprise than through a department of government---or so the
Congress might conclude. We cannot say that public ownership is the sole method
of promoting the public purposes of community redevelopment projects.'' 

It is further argued that without a bright-line rule nothing would stop a city
from transferring citizen \textit{A}'s property to citizen \textit{B} for the
sole reason that citizen \textit{B} will put the property to a more productive
use and thus pay more taxes. Such a one-to-one transfer of property, executed
outside the confines of an integrated development plan, is not presented in this
case. While such an unusual exercise of government power would certainly raise a
suspicion that a private purpose was afoot, the hypothetical cases posited by
petitioners can be confronted if and when they arise. They do not warrant the
crafting of an artificial restriction on the concept of public use.

Alternatively, petitioners maintain that for takings of this kind we should
require a ``reasonable certainty'' that the expected public benefits will
actually accrue. Such a rule, however, would represent an even greater departure
from our precedent.\ldots The disadvantages of a heightened form of review are
especially pronounced in this type of case. Orderly implementation of a
comprehensive redevelopment plan obviously requires that the legal rights of all
interested parties be established before new construction can be commenced. A
constitutional rule that required postponement of the judicial approval of every
condemnation until the likelihood of success of the plan had been assured would
unquestionably impose a significant impediment to the successful consummation of
many such plans.

Just as we decline to second-guess the City's considered judgments about the
efficacy of its development plan, we also decline to second-guess the City's
determinations as to what lands it needs to acquire in order to effectuate the
project.\ldots

In affirming the City's authority to take petitioners' properties, we do not
minimize the hardship that condemnations may entail, notwithstanding the payment
of just compensation. We emphasize that nothing in our opinion precludes any
State from placing further restrictions on its exercise of the takings power.
Indeed, many States already impose ``public use'' requirements that are stricter
than the federal baseline. Some of these requirements have been established as a
matter of state constitutional law, while others are expressed in state eminent
domain statutes that carefully limit the grounds upon which takings may be
exercised. As the submissions of the parties and their \textit{amici} make
clear, the necessity and wisdom of using eminent domain to promote economic
development are certainly matters of legitimate public debate. This Court's
authority, however, extends only to determining whether the City's proposed
condemnations are for a ``public use'' within the meaning of the Fifth Amendment
to the Federal Constitution. Because over a century of our case law interpreting
that provision dictates an affirmative answer to that question, we may not grant
petitioners the relief that they seek.\ldots

\opinion Justice \textsc{Kennedy}, concurring.

\ldots This Court has declared that a taking should be upheld as consistent with
the Public Use Clause, U.S. Const., Amdt. 5, as long as it is ``rationally
related to a conceivable public purpose.'' \textit{Hawaii Housing Authority v.
Midkiff}, 467 U.S. 229, 241 (1984). This deferential standard of review echoes
the rational-basis test used to review economic regulation under the Due Process
and Equal Protection Clauses. The determination that a rational-basis standard
of review is appropriate does not, however, alter the fact that transfers
intended to confer benefits on particular, favored private entities, and with
only incidental or pretextual public benefits, are forbidden by the Public Use
Clause.

A court applying rational-basis review under the Public Use Clause should strike
down a taking that, by a clear showing, is intended to favor a particular
private party, with only incidental or pretextual public benefits, just as a
court applying rational-basis review under the Equal Protection Clause must
strike down a government classification that is clearly intended to injure a
particular class of private parties, with only incidental or pretextual public
justifications.\ldots

A court confronted with a plausible accusation of impermissible favoritism to
private parties should treat the objection as a serious one and review the
record to see if it has merit, though with the presumption that the government's
actions were reasonable and intended to serve a public purpose. [Justice Kennedy
went on to observe that the trial court made findings that supported the
conclusion ``that benefiting Pfizer was not `the primary motivation or effect of
this development plan'\,''.]\ldots This case, then, survives the meaningful
rational-basis review that in my view is required under the Public Use
Clause.\ldots

\ldots There may be private transfers in which the risk of undetected
impermissible favoritism of private parties is so acute that a presumption
(rebuttable or otherwise) of invalidity is warranted under the Public Use
Clause. This demanding level of scrutiny, however, is not required simply
because the purpose of the taking is economic development.\ldots

\opinion Justice O'\textsc{Connor}, with whom \textsc{The Chief Justice},
Justice \textsc{Scalia}, and Justice \textsc{Thomas} join, dissenting.

\ldots Under the banner of economic development, all private property is now
vulnerable to being taken and transferred to another private owner, so long as
it might be upgraded---\textit{i.e.}, given to an owner who will use it in a way
that the legislature deems more beneficial to the public---in the process. To
reason, as the Court does, that the incidental public benefits resulting from
the subsequent ordinary use of private property render economic development
takings ``for public use'' is to wash out any distinction between private and
public use of property---and thereby effectively to delete the words ``for
public use'' from the Takings Clause of the Fifth Amendment. Accordingly I
respectfully dissent.\ldots

\ldots Where is the line between ``public'' and ``private'' property use? We
give considerable deference to legislatures' determinations about what
governmental activities will advantage the public. But were the political
branches the sole arbiters of the public-private distinction, the Public Use
Clause would amount to little more than hortatory fluff. An external, judicial
check on how the public use requirement is interpreted, however limited, is
necessary if this constraint on government power is to retain any meaning. 

Our cases have generally identified three categories of takings that comply with
the public use requirement, though it is in the nature of things that the
boundaries between these categories are not always firm. Two are relatively
straightforward and uncontroversial. First, the sovereign may transfer private
property to public ownership---such as for a road, a hospital, or a military
base. Second, the sovereign may transfer private property to private parties,
often common carriers, who make the property available for the public's
use---such as with a railroad, a public utility, or a stadium. But ``public
ownership'' and ``use-by-the-public'' are sometimes too constricting and
impractical ways to define the scope of the Public Use Clause. Thus we have
allowed that, in certain circumstances and to meet certain exigencies, takings
that serve a public purpose also satisfy the Constitution even if the property
is destined for subsequent private use. See, \textit{e.g., Berman v. Parker},
348 U.S. 26 (1954); \textit{Hawaii Housing Authority v. Midkiff}, 467 U.S. 229
(1984).\ldots

\ldots We are guided by two precedents about the taking of real property by
eminent domain. In \textit{Berman}, we upheld takings within a blighted
neighborhood of Washington, D.C. The neighborhood had so deteriorated that, for
example, 64.3\% of its dwellings were beyond repair. It had become burdened with
``overcrowding of dwellings,'' ``lack of adequate streets and alleys,'' and
``lack of light and air.'' Congress had determined that the neighborhood had
become ``injurious to the public health, safety, morals, and welfare'' and that
it was necessary to ``eliminat[e] all such injurious conditions by employing all
means necessary and appropriate for the purpose,'' including eminent domain. Mr.
Berman's department store was not itself blighted. Having approved of Congress'
decision to eliminate the harm to the public emanating from the blighted
neighborhood, however, we did not second-guess its decision to treat the
neighborhood as a whole rather than lot-by-lot. 

In \textit{Midkiff}, we upheld a land condemnation scheme in Hawaii whereby
title in real property was taken from lessors and transferred to lessees. At
that time, the State and Federal Governments owned nearly 49\% of the State's
land, and another 47\% was in the hands of only 72 private landowners.
Concentration of land ownership was so dramatic that on the State's most
urbanized island, Oahu, 22 landowners owned 72.5\% of the fee simple titles. The
Hawaii Legislature had concluded that the oligopoly in land ownership was
``skewing the State's residential fee simple market, inflating land prices, and
injuring the public tranquility and welfare,'' and therefore enacted a
condemnation scheme for redistributing title.\ldots

In moving away from our decisions sanctioning the condemnation of harmful
property use, the Court today significantly expands the meaning of public use.
It holds that the sovereign may take private property currently put to ordinary
private use, and give it over for new, ordinary private use, so long as the new
use is predicted to generate some secondary benefit for the public---such as
increased tax revenue, more jobs, maybe even esthetic pleasure. But nearly any
lawful use of real private property can be said to generate some incidental
benefit to the public. Thus, if predicted (or even guaranteed) positive side
effects are enough to render transfer from one private party to another
constitutional, then the words ``for public use'' do not realistically exclude
\textit{any} takings, and thus do not exert any constraint on the eminent domain
power.\ldots

\opinion Justice \textsc{Thomas}, dissenting.

Long ago, William Blackstone wrote that ``the law of the land\ldots postpone[s]
even public necessity to the sacred and inviolable rights of private property.''
1 Commentaries on the Laws of England 134--135 (1765) (hereinafter Blackstone).
The Framers embodied that principle in the Constitution, allowing the government
to take property not for ``public necessity,'' but instead for ``public use.''
Amdt. 5. Defying this understanding, the Court replaces the Public Use Clause
with a `` `[P]ublic [P]urpose' '' Clause (or perhaps the ``Diverse and Always
Evolving Needs of Society'' Clause (capitalization added)), a restriction that
is satisfied, the Court instructs, so long as the purpose is ``legitimate'' and
the means ``not irrational.'' This deferential shift in phraseology enables the
Court to hold, against all common sense, that a costly urban-renewal project
whose stated purpose is a vague promise of new jobs and increased tax revenue,
but which is also suspiciously agreeable to the Pfizer Corporation, is for a
``public use.''

I cannot agree. If such ``economic development'' takings are for a ``public
use,'' any taking is, and the Court has erased the Public Use Clause from our
Constitution, as Justice \textsc{O'Connor} powerfully argues in dissent. I do
not believe
that this Court can eliminate liberties expressly enumerated in the Constitution
and therefore join her dissenting opinion. Regrettably, however, the Court's
error runs deeper than this. Today's decision is simply the latest in a string
of our cases construing the Public Use Clause to be a virtual nullity, without
the slightest nod to its original meaning. In my view, the Public Use Clause,
originally understood, is a meaningful limit on the government's eminent domain
power. Our cases have strayed from the Clause's original meaning, and I would
reconsider them.\ldots 

The consequences of today's decision are not difficult to predict, and promise
to be harmful. So-called ``urban renewal'' programs provide some compensation
for the properties they take, but no compensation is possible for the subjective
value of these lands to the individuals displaced and the indignity inflicted by
uprooting them from their homes. Allowing the government to take property solely
for public purposes is bad enough, but extending the concept of public purpose
to encompass any economically beneficial goal guarantees that these losses will
fall disproportionately on poor communities. Those communities are not only
systematically less likely to put their lands to the highest and best social
use, but are also the least politically powerful.\ldots 

\ldots In the 1950's, no doubt emboldened in part by the expansive understanding
of ``public use'' this Court adopted in \textit{Berman}, cities ``rushed to draw
plans'' for downtown development. B. Frieden \& L. Sagalyn, Downtown, Inc. How
America Rebuilds Cities 17 (1989). ``Of all the families displaced by urban
renewal from 1949 through 1963, 63 percent of those whose race was known were
nonwhite, and of these families, 56 percent of nonwhites and 38 percent of
whites had incomes low enough to qualify for public housing, which, however, was
seldom available to them.'' Public works projects in the 1950's and 1960's
destroyed predominantly minority communities in St. Paul, Minnesota, and
Baltimore, Maryland. In 1981, urban planners in Detroit, Michigan, uprooted the
largely ``lower-income and elderly'' Poletown neighborhood for the benefit of
the General Motors Corporation. J. Wylie, Poletown: Community Betrayed 58
(1989). Urban renewal projects have long been associated with the displacement
of blacks; ``[i]n cities across the country, urban renewal came to be known as
`Negro removal.'\,'' Pritchett, The ``Public Menace'' of Blight: Urban Renewal
and the Private Uses of Eminent Domain, 21 Yale L. \& Pol'y Rev. 1, 47 (2003).
Over 97 percent of the individuals forcibly removed from their homes by the
``slum-clearance'' project upheld by this Court in \textit{Berman} were black.
Regrettably, the predictable consequence of the Court's decision will be to
exacerbate these effects.\ldots

