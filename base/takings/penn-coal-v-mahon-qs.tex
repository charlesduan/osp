\expected{penn-coal-v-mahon}

\item \textbf{Nuisances.} Justice Brandeis's dissent objects that the Kohler Act
simply prohibits a ``noxious use.'' A number of prior precedents, Brandeis
argues, established that the state may enjoin such uses even if doing so
``deprives the owner of the only use to which the property can then be
profitably put.'' In \textit{Hadacheck v. Sebastian}, 239 U.S. 394 (1915), for
example, the Court found no taking where an ordinance prohibiting brickyards
largely destroyed the value of an existing facility. The land was alleged to be
worth \$800,000 as a brickyard and \$60,000 otherwise. Nonetheless, the Court
deemed it within the state's police power to declare previously lawful
activities to be nuisances and enjoin them. \textit{Id.} at 410 (``[T]here must
be progress, and if in its march private interests are in the way, they must
yield to the good of the community.''). The principle, that regulating nuisances
is never a taking, has been referred to as a second categorical rule in takings
law. As we will see below (in our discussion of the \textit{Lucas} case), the
actual doctrine is not so simple.

\item \textbf{Diminution of value.} How far is too far depends on how one
defines the property interest at stake. For Holmes, the Kohler Act ``purports to
abolish\ldots an estate in land,'' by preventing the exercise of the mining
company's bargained-for rights. On this logic, the diminution of value is total.
Brandeis, by contrast, objected that ``[t]he rights of an owner as against the
public are not increased by dividing the interests in his property into surface
and subsoil. The sum of the rights in the parts can not be greater than the
rights in the whole.'' Analyzing the takings question by looking at the property
as a composition of discrete ``estates,'' rather than as an integrated whole has
been called ``\textbf{conceptual severance}.'' Margaret Jane Radin, \textit{The
Liberal Conception of Property: Cross Currents in the Jurisprudence of Takings},
88 \textsc{Colum. L. Rev}. 1667, 1676 (1988) (``[T]his strategy hypothetically
or conceptually ``severs'' from the whole bundle of rights just those strands
that are interfered with by the regulation, and then hypothetically or
conceptually construes those strands in the aggregate as a separate whole
thing.''). 


The issue is also sometimes referred to as the ``\textbf{denominator problem}.''
Suppose I have a parcel of land that I could sell for \$200,000, but I could
also sell the mining rights alone for \$100,000. Suppose further that the state
enacts a ban on mining, which reduces the market value of the land to \$100,000.
How do we evaluate the diminution of value? Is it 50\% (\$100,000/\$200,000)? Or
is the denominator the mining rights alone, making the diminution 100\%
(\$100,000/\$100,000)? If we were to permit conceptual severance, how should the
relevant estates be identified? In \textit{Pennsylvania Coal}, Holmes noted that
the mining interest at issue was an established one under state law. Is that a
satisfactory basis? Can state law define federal rights in this way? What if an
anti-regulatory state legislature took advantage of its time in power to create
broad new ``estates'' (e.g., one for oil drilling, one for factory smoke, etc.)?


\item \textbf{Support estates revisited.} On this question, note that the Court
revisited the takings implications of Pennsylvania statutes designed to protect
surface structures from mining. \textit{Keystone Bituminous Coal Ass'n v.
DeBenedictis}, 480 U.S. 470 (1987), upheld a statute whose implementing
regulations required coal companies to leave approximately 50\% of coal in the
ground beneath protected buildings. The Court did so notwithstanding
Pennsylvania law's ``unique'' approach of treating the ``support estate'' as a
discrete interest in land. By a 5-4 vote, the Court concluded that the interest
is part and parcel of other mining interests (thus expanding the denominator at
issue in considering diminution of value). ``Because petitioners retain the
right to mine virtually all of the coal in their mineral estates, the burden the
Act places on the support estate does not constitute a taking. Petitioners may
continue to mine coal profitably even if they may not destroy or damage surface
structures at will in the process.'' \textit{Id.} at 501. 


This result may seem at odds with \textit{Pennsylvania Coal}. The dissent
certainly thought so. The majority read \textit{Pennsylvania Coal} narrowly as
reaching only a specific application of the Kohler Act to bargained-for rights
to mine under a particular house. The rest, pertaining to the general
applicability of the Kohler Act was described as an ``uncharacteristically''
advisory opinion on Justice Holmes's part. \textit{Id.} at 484. In any case, the
majority viewed the Subsidence Act as different than the earlier law in two key
respects. First, the Court read the history of the statute as disclosing a
public purpose. ``None of the indicia of a statute enacted solely for the
benefit of private parties identified in Justice Holmes' opinion are present
here.'' \textit{Id.} at 486. That some private parties \textit{did} benefit was
seen as incidental. Second, as noted above, the Court viewed the challengers as
retaining valuable mining rights. Unlike ``the Kohler Act[, which] made mining
of ``certain coal'' commercially impracticable,'' the Subsidence Act was not
shown to have worked a similar harm, at least for purposes of a facial
challenge.



\item \textbf{Baseline Games.} Is Justice Brandeis's distinction between
``confer[ring] benefits on property owners'' and ``protect[ing] the public from
detriment and danger'' persuasive? What Justice Brandeis views as prevention of
a harm---preventing the collapse of surface structures overlying coal formations
owned by mining interests---Justice Holmes views as conferral of an
unbargained-for benefit---a support estate that was willingly bargained away. Is
one of them wrong? What is the baseline against which the economic effects of a
regulation ought to be evaluated?

\item \textbf{``Reciprocity of advantage.''} Reciprocity of advantage refers
to a sort of implicit compensation of regulation. Suppose you own land in a part
of town zoned for residential use. You may not build a factory on your property,
but neither can your neighbors. Your property's residential value is enhanced
accordingly. ``Under our system of government, one of the State's primary ways
of preserving the public weal is restricting the uses individuals can make of
their property. While each of us is burdened somewhat by such restrictions, we,
in turn, benefit greatly from the restrictions that are placed on others.''
\textit{Keystone Bituminous Coal Ass'n}, 480 U.S. at 491. The principle is often
invoked to argue that regulations should not ``single out'' anyone for
disproportionate burdens. That does not mean that everything comes out even.
``The Takings Clause has never been read to require the States or the courts to
calculate whether a specific individual has suffered burdens\ldots in excess of
the benefits received. Not every individual gets a full dollar return in
benefits for the taxes he or she pays; yet, no one suggests that an individual
has a right to compensation for the difference between taxes paid and the dollar
value of benefits received.'' \textit{Id.} at 491 n.21.

