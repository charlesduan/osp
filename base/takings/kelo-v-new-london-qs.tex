\expected{kelo-v-new-london}

\item If the state pays compensation and bears the political costs, what is
wrong with taking from A and giving to B? Suppose the state wants land to be
used for a particular purpose. Is it sensible to require the state to conduct
operations or might turning them over to private actors enhance efficiency? Or
is a ``public use'' requirement more about policing local political processes,
deterring corruption or special interest capture? If so, is this an efficient
mechanism? 

\item \textit{Kelo} provoked a strong public reaction and a flurry of state
legislative activity designed to control abuses of eminent domain. By 2009, 43
states had enacted eminent domain restrictions. Does this mean that democracy
works? Are there advantages to the Supreme Court's setting limits on eminent
domain? \textit{Compare} Alberto B. Lopez, \textit{Revisiting Kelo and Eminent
Domain's ``Summer of Scrutiny,''} 59 \textsc{Ala. L. Rev.} 561, 565 (2008)
(``[P]ost-\textit{Kelo} legislation symbolizes the government's effort to remedy
the breach of the public's trust caused by \textit{Kelo} regardless of one's
substantive view of those legislative measures. Furthermore, the robust
post-\textit{Kelo} legislative response is a testament to the strength of one of
the core principles of our government---federalism.''), \textit{with} Ilya
Somin, \textit{The Limits of Backlash: Assessing the Political Response to}
Kelo, 93 \textsc{Minn. L. Rev}. 2100, 2105 (2009) (``Only seven states that had
recently engaged in significant numbers of economic development and blight
condemnations have enacted post-Kelo legislative reforms with any real
teeth.''). Can one's answer be independent of one's prior views on the
legitimate uses of eminent domain?

\item As Justice Thomas's dissent notes, one criticism of the eminent domain
power has been that it has been used in either a discriminatory or racially
disproportionate manner. Which way does this consideration cut in \textit{Kelo}?
After all, the practice of labeling of minority communities as ``blighted'' is a
matter of historical record. Might the Court's approval of eminent domain's use
on \textit{Kelo}'s facts improve the politics of eminent domain law by making
clear that anyone could be on the receiving end of a condemnation? And to the
extent the problem with eminent domain is discriminatory application, why isn't
the Constitution's Equal Protection Clause a preferable safeguard? Or does the
history cited by Justice Thomas answer that question?

\item Most of the affected homeowners in New London negotiated a purchase price
with the New London Development Corporation (NLDC). For her part, Kelo
reportedly turned down a purchase offer that would have netted her a \$22,000
profit on her home. The decision to litigate, while not letting her keep her
property, did lead to a higher purchase price. The public outcry in the wake of
the \textit{Kelo} ruling led to favorable settlements for the holdout
landowners. For example,
\begin{quote}
Kelo agreed in June 2006 to sell for \$442,000 (\$392,000 plus a pay-off of her
\$50,000 mortgage); not too bad for a place she had purchased in August 1997 for
\$53,500, and NLDC had appraised for condemnation at \$123,000 in November 2000.
She only sold the lot. Avner Gregory, the same preservationist who had
refurbished the house after moving it from its original location to the site
where Kelo found it, relocated the house a second time to a vacant parcel with a
pre-existing foundation, in a modest neighborhood several miles away, on the
other side of the Amtrak rail line from Fort Trumbull. A plaque identifies the
house as ``The Kelo House.'' 
\end{quote}
George Lefcoe, \textit{Jeff Benedict's Little Pink House: The Back Story of the}
Kelo \textit{Case}, 42 \textsc{Conn. L. Rev}. 925, 954-55 (2010) (footnotes
omitted). In 2009 Pfizer announced it would leave New London to cut costs,
taking its jobs to its facility in Groton, Connecticut. Patrick McGeehan,
\emph{Pfizer to Leave City That Won Land-Use Case}, \textsc{N.Y. Times},
November 13, 2009, at A1,
\url{http://www.nytimes.com/2009/11/13/nyregion/13pfizer.html}. 

