\reading{Pennsylvania Coal Co. v. Mahon}

\readingcite{260 U.S. 393 (1922)}

\opinion Mr. Justice \textsc{Holmes} delivered the opinion of the Court.

This is a bill in equity brought by the defendants in error to prevent the
Pennsylvania Coal Company from mining under their property in such way as to
remove the supports and cause a subsidence of the surface and of their house.
The bill sets out a deed executed by the Coal Company in 1878, under which the
plaintiffs claim. The deed conveys the surface but in express terms reserves the
right to remove all the coal under the same and the grantee takes the premises
with the risk and waives all claim for damages that may arise from mining out
the coal. But the plaintiffs say that whatever may have been the Coal Company's
rights, they were taken away by an Act of Pennsylvania, approved May 27, 1921
(P. L. 1198), commonly known there as the Kohler Act. The Court of Common Pleas
found that if not restrained the defendant would cause the damage to prevent
which the bill was brought but denied an injunction, holding that the statute if
applied to this case would be unconstitutional. On appeal the Supreme Court of
the State agreed that the defendant had contract and property rights protected
by the Constitution of the United States, but held that the statute was a
legitimate exercise of the police power and directed a decree for the
plaintiffs. A writ of error was granted bringing the case to this Court.

The statute forbids the mining of anthracite coal in such way as to cause the
subsidence of, among other things, any structure used as a human habitation,
with certain exceptions, including among them land where the surface is owned by
the owner of the underlying coal and is distant more than one hundred and fifty
feet from any improved property belonging to any other person. As applied to
this case the statute is admitted to destroy previously existing rights of
property and contract. The question is whether the police power can be stretched
so far.

Government hardly could go on if to some extent values incident to property
could not be diminished without paying for every such change in the general law.
As long recognized some values are enjoyed under an implied limitation and must
yield to the police power. But obviously the implied limitation must have its
limits or the contract and due process clauses are gone. One fact for
consideration in determining such limits is the extent of the diminution. When
it reaches a certain magnitude, in most if not in all cases there must be an
exercise of eminent domain and compensation to sustain the act. So the question
depends upon the particular facts. The greatest weight is given to the judgment
of the legislature but it always is open to interested parties to contend that
the legislature has gone beyond its constitutional power.

This is the case of a single private house. No doubt there is a public interest
even in this, as there is in every purchase and sale and in all that happens
within the commonwealth. Some existing rights may be modified even in such a
case. But usually in ordinary private affairs the public interest does not
warrant much of this kind of interference. A source of damage to such a house is
not a public nuisance even if similar damage is inflicted on others in different
places. The damage is not common or public. The extent of the public interest is
shown by the statute to be limited, since the statute ordinarily does not apply
to land when the surface is owned by the owner of the coal. Furthermore, it is
not justified as a protection of personal safety. That could be provided for by
notice. Indeed the very foundation of this bill is that the defendant gave
timely notice of its intent to mine under the house. On the other hand the
extent of the taking is great. It purports to abolish what is recognized in
Pennsylvania as an estate in land-a very valuable estate-and what is declared by
the Court below to be a contract hitherto binding the plaintiffs. If we were
called upon to deal with the plaintiffs' position alone we should think it clear
that the statute does not disclose a public interest sufficient to warrant so
extensive a destruction of the defendant's constitutionally protected rights.

But the case has been treated as one in which the general validity of the act
should be discussed. The Attorney General of the State, the City of Scranton and
the representatives of other extensive interests were allowed to take part in
the argument below and have submitted their contentions here. It seems,
therefore, to be our duty to go farther in the statement of our opinion, in
order that it may be known at once, and that further suits should not be brought
in vain.

It is our opinion that the act cannot be sustained as an exercise of the police
power, so far as it affects the mining of coal under streets or cities in places
where the right to mine such coal has been reserved. As said in a Pennsylvania
case, ``For practical purposes, the right to coal consists in the right to mine
it.'' \emph{Commonwealth v. Clearview Coal Co.}, 256 Pa. 328, 331, 100 Atl. 820.
What makes the right to mine coal valuable is that it can be exercised with
profit. To make it commercially impracticable to mine certain coal has very
nearly the same effect for constitutional purposes as appropriating or
destroying it. This we think that we are warranted in assuming that the statute
does.

It is true that in \emph{Plymouth Coal Co. v. Pennsylvania}, 232 U. S. 531, it
was held competent for the legislature to require a pillar of coal to the left
along the line of adjoining property, that with the pillar on the other side of
the line would be a barrier sufficient for the safety of the employees of either
mine in case the other should be abandoned and allowed to fill with water. But
that was a requirement for the safety of employees invited into the mine, and
secured an average reciprocity of advantage that has been recognized as a
justification of various laws.

The rights of the public in a street purchased or laid out by eminent domain are
those that it has paid for. If in any case its representatives have been so
short sighted as to acquire only surface rights without the right of support we
see no more authority for supplying the latter without compensation than there
was for taking the right of way in the first place and refusing to pay for it
because the public wanted it very much. The protection of private property in
the Fifth Amendment presupposes that it is wanted for public use, but provides
that it shall not be taken for such use without compensation. A similar
assumption is made in the decisions upon the Fourteenth Amendment.
\emph{Hairston v. Danville \& Western Ry. Co.}, 208 U. S. 598, 605. When this
seemingly absolute protection is found to be qualified by the police power, the
natural tendency of human nature is to extend the qualification more and more
until at last private property disappears. But that cannot be accomplished in
this way under the Constitution of the United States.

The general rule at least is that while property may be regulated to a certain
extent, if regulation goes too far it will be recognized as a taking. It may be
doubted how far exceptional cases, like the blowing up of a house to stop a
conflagration, go-and if they go beyond the general rule, whether they do not
stand as much upon tradition as upon principle. In general it is not plain that
a man's misfortunes or necessities will justify his shifting the damages to his
neighbor's shoulders. We are in danger of forgetting that a strong public desire
to improve the public condition is not enough to warrant achieving the desire by
a shorter cut than the constitutional way of paying for the change. As we
already have said this is a question of degree-and therefore cannot be disposed
of by general propositions. But we regard this as going beyond any of the cases
decided by this Court.\ldots

We assume, of course, that the statute was passed upon the conviction that an
exigency existed that would warrant it, and we assume that an exigency exists
that would warrant the exercise of eminent domain. But the question at bottom is
upon whom the loss of the changes desired should fall. So far as private persons
or communities have seen fit to take the risk of acquiring only surface rights,
we cannot see that the fact that their risk has become a danger warrants the
giving to them greater rights than they bought.

Decree reversed.

\opinion Mr. Justice \textsc{Brandeis} dissenting.

The Kohler Act prohibits, under certain conditions, the mining of anthracite
coal within the limits of a city in such a manner or to such an extent ``as to
cause the\ldots subsidence of\ldots any dwelling or other structure used as a
human habitation, or any factory, store, or other industrial or mercantile
establishment in which human labor is employed.'' Coal in place is land, and the
right of the owner to use his land is not absolute. He may not so use it as to
create a public nuisance, and uses, once harmless, may, owing to changed
conditions, seriously threaten the public welfare. Whenever they do, the
Legislature has power to prohibit such uses without paying compensation; and the
power to prohibit extends alike to the manner, the character and the purpose of
the use. Are we justified in declaring that the Legislature of Pennsylvania has,
in restricting the right to mine anthracite, exercised this power so arbitrarily
as to violate the Fourteenth Amendment?

Every restriction upon the use of property imposed in the exercise of the police
power deprives the owner of some right theretofore enjoyed, and is, in that
sense, an abridgment by the state of rights in property without making
compensation. But restriction imposed to protect the public health, safety or
morals from dangers threatended is not a taking. The restriction here in
question is merely the prohibition of a noxious use. The property so restricted
remains in the possession of its owner. The state does not appropriate it or
make any use of it. The state merely prevents the owner from making a use which
interferes with paramount rights of the public. Whenever the use prohibited
ceases to be noxious-as it may because of further change in local or social
conditions-the restriction will have to be removed and the owner will again be
free to enjoy his property as heretofore.

The restriction upon the use of this property cannot, of course, be lawfully
imposed, unless its purpose is to protect the public. But the purpose of a
restriction does not cease to be public, because incidentally some private
persons may thereby receive gratuitously valuable special benefits. Thus, owners
of low buildings may obtain, through statutory restrictions upon the height of
neighboring structures, benefits equivalent to an easement of light and air.
\emph{Welch v. Swasey}, 214 U. S. 91. Furthermore, a restriction, though imposed
for a public purpose, will not be lawful, unless the restriction is an
appropriate means to the public end. But to keep coal in place is surely an
appropriate means of preventing subsidence of the surface; and ordinarily it is
the only available means. Restriction upon use does not become inappropriate as
a means, merely because it deprives the owner of the only use to which the
property can then be profitably put. The liquor and the oleomargine cases
settled that. \emph{Mugler v. Kansas}, 123 U. S. 623, 668, 669; \emph{Powell v.
Pennsylvania}, 127 U. S. 678, 682. See also \emph{Hadacheck v. Los Angeles}, 239
U. S. 394; \emph{Pierce Oil Corporation v. City of Hope}, 248 U. S. 498. Nor is
a restriction imposed through exercise of the police power inappropriate as a
means, merely because the same end might be effected through exercise of the
power of eminent domain, or otherwise at public expense. Every restriction upon
the height of buildings might be secured through acquiring by eminent domain the
right of each owner to build above the limiting height; but it is settled that
the state need not resort to that power. If by mining anthracite coal the owner
would necessarily unloose poisonous gases, I suppose no one would doubt the
power of the state to prevent the mining, without buying his coal fields. And
why may not the state, likewise, without paying compensation, prohibit one from
digging so deep or excavating so near the surface, as to expose the community to
like dangers? In the latter case, as in the former, carrying on the business
would be a public nuisance.

It is said that one fact for consideration in determining whether the limits of
the police power have been exceeded is the extent of the resulting diminution in
value, and that here the restriction destroys existing rights of property and
contract. But values are relative. If we are to consider the value of the coal
kept in place by the restriction, we should compare it with the value of all
other parts of the land. That is, with the value not of the coal alone, but with
the value of the whole property. The rights of an owner as against the public
are not increased by dividing the interests in his property into surface and
subsoil. The sum of the rights in the parts can not be greater than the rights
in the whole. The estate of an owner in land is grandiloquently described as
extending \emph{ab orco usque ad coelum}. But I suppose no one would contend
that by selling his interest above 100 feet from the surface he could prevent
the state from limiting, by the police power, the height of structures in a
city. And why should a sale of underground rights bar the state's power? For
aught that appears the value of the coal kept in place by the restriction may be
negligible as compared with the value of the whole property, or even as compared
with that part of it which is represented by the coal remaining in place and
which may be extracted despite the statute. Ordinarily a police regulation,
general in operation, will not be held void as to a particular property,
although proof is offered that owing to conditions peculiar to it the
restriction could not reasonably be applied. But even if the particular facts
are to govern, the statute should, in my opinion be upheld in this case. For the
defendant has failed to adduce any evidence from which it appears that to
restrict its mining operations was an unreasonable exercise of the police power.
Where the surface and the coal belong to the same person, self-interest would
ordinarily prevent mining to such an extent as to cause a subsidence. It was,
doubtless, for this reason that the Legislature, estimating the degrees of
danger, deemed statutory restriction unnecessary for the public safety under
such conditions.

It is said that this is a case of a single dwelling house, that the restriction
upon mining abolishes a valuable estate hitherto secured by a contract with the
plaintiffs, and that the restriction upon mining cannot be justified as a
protection of personal safety, since that could be provided for by notice. The
propriety of deferring a good deal to tribunals on the spot has been repeatedly
recognized. May we say that notice would afford adequate protection of the
public safety where the Legislature and the highest court of the state, with
greater knowledge of local conditions, have declared, in effect, that it would
not? If the public safety is imperiled, surely neither grant, nor contract, can
prevail against the exercise of the police power.\ldots Nor can existing
contracts between private individuals preclude exercise of the police
power.\ldots The fact that this suit is brought by a private person is, of
course, immaterial. To protect the community through invoking the aid, as
litigant, of interested private citizens is not a novelty in our law. That it
may be done in Pennsylvania was decided by its Supreme Court in this case. And
it is for a state to say how its public policy shall be enforced.

This case involves only mining which causes subsidence of a dwelling house. But
the Kohler Act contains provisions in addition to that quoted above; and as to
these, also, an opinion is expressed. These provisions deal with mining under
cities to such an extent as to cause subsidence of---
\begin{itemize}
\item[(a)] Any public building or any structure customarily used by the public
as a place of resort, assemblage, or amusement, including, but not limited to,
churches, schools, hospitals, theaters, hotels, and railroad stations.

\item[(b)] Any street, road, bridge, or other public passageway, dedicated to
public use or habitually used by the public.

\item[(c)] Any track, roadbed, right of way, pipe, conduit, wire, or other
facility, used in the service of the public by any municipal corporation or
public service company as defined by the Public Service Law, section 1.
\end{itemize}
A prohibition of mining which causes subsidence of such structures and
facilities is obviously enacted for a public purpose; and it seems, likewise,
clear that mere notice of intention to mine would not in this connection secure
the public safety. Yet it is said that these provisions of the act cannot be
sustained as an exercise of the police power where the right to mine such coal
has been reserved. The conclusion seems to rest upon the assumption that in
order to justify such exercise of the police power there must be `an average
reciprocity of advantage' as between the owner of the property restricted and
the rest of the community; and that here such reciprocity is absent. Reciprocity
of advantage is an important consideration, and may even be an essential, where
the state's power is exercised for the purpose of conferring benefits upon the
property of a neighborhood, as in drainage projects; or upon adjoining owners,
as by party wall. But where the police power is exercised, not to confer
benefits upon property owners but to protect the public from detriment and
danger, there is in my opinion, no room for considering reciprocity of
advantage. There was no reciprocal advantage to the owner prohibited from using
his oil tanks in 248 U. S. 498; his brickyard, in 239 U. S. 394; his livery
stable, in 237 U. S. 171; his billiard hall, in 225 U. S. 623; his oleomargarine
factory, in 127 U. S. 678; his brewery, in 123 U. S. 623; unless it be the
advantage of living and doing business in a civilized community. That reciprocal
advantage is given by the act to the coal operators.

