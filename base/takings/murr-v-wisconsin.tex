\reading{Murr v. Wisconsin}

\readingcite{137 S. Ct. 1933 (2017)}

\opinion Justice \textsc{Kennedy} delivered the opinion of the Court.

[Petitioners, the Murrs, owned two adjacent lots, Lot E and Lot F, that were
subject to state and local regulations that limited development of lots with
less than one acre of suitable land. Neither lot met the size requirement
individually. The regulations contained a grandfather clause allowing
development of preexisting undersized lots, but a merger provision prohibited
undersized adjacent lots under common ownership from sale or development as
separate lots. Lots E and F came under the common ownership of the Murrs in
1995, making them subject to the merger provision. The regulations thus
interfered with the Murrs' plan to move a cabin on Lot F and sell Lot E to pay
the costs of doing so. They filed a regulatory takings claim. In rejecting it,
the Wisconsin Court of Appeals concluded that the relevant parcel for takings
analysis was the combination of Lots E and F, refusing to consider the
regulations' effect on Lot E individually.]

\ldots.

This case presents a question that is linked to the ultimate determination
whether a regulatory taking has occurred: What is the proper unit of property
against which to assess the effect of the challenged governmental action? Put
another way, ``[b]ecause our test for regulatory taking requires us to compare
the value that has been taken from the property with the value that remains in
the property, one of the critical questions is determining how to define the
unit of property `whose value is to furnish the denominator of the
fraction.'\,'' \textit{Keystone Bituminous Coal Assn. v. DeBenedictis}, 480 U.S.
470, 497 (1987) (quoting Michelman, Property, Utility, and Fairness, 80 Harv. L.
Rev. 1165, 1992 (1967)).

As commentators have noted, the answer to this question may be outcome
determinative. This Court, too, has explained that the question is important to
the regulatory takings inquiry. ``To the extent that any portion of property is
taken, that portion is always taken in its entirety; the relevant question,
however, is whether the property taken is all, or only a portion of, the parcel
in question.'' \textit{Concrete Pipe \& Products of Cal., Inc. v. Construction
Laborers Pension Trust for Southern Cal.}, 508 U.S. 602, 644 (1993).\ldots

While the Court has not set forth specific guidance on how to identify the
relevant parcel for the regulatory taking inquiry, there are two concepts which
the Court has indicated can be unduly narrow.

First, the Court has declined to limit the parcel in an artificial manner to the
portion of property targeted by the challenged regulation. In \textit{Penn
Central}, for example, the Court rejected a challenge to the denial of a permit
to build an office tower above Grand Central Terminal. The Court refused to
measure the effect of the denial only against the ``air rights'' above the
terminal\ldots.

The second concept about which the Court has expressed caution is the view that
property rights under the Takings Clause should be coextensive with those under
state law. Although property interests have their foundations in state law, the
\textit{Palazzolo} Court reversed a state-court decision that rejected a takings
challenge to regulations that predated the landowner's acquisition of title. The
Court explained that States do not have the unfettered authority to ``shape and
define property rights and reasonable investment-backed expectations,'' leaving
landowners without recourse against unreasonable regulations.

By the same measure, defining the parcel by reference to state law could defeat
a challenge even to a state enactment that alters permitted uses of property in
ways inconsistent with reasonable investment-backed expectations. For example, a
State might enact a law that consolidates nonadjacent property owned by a single
person or entity in different parts of the State and then imposes development
limits on the aggregate set. If a court defined the parcel according to the
state law requiring consolidation, this improperly would fortify the state law
against a takings claim, because the court would look to the retained value in
the property as a whole rather than considering whether individual holdings had
lost all value.


\readinghead{III}


\readinghead{A}

As the foregoing discussion makes clear, no single consideration can supply the
exclusive test for determining the denominator. Instead, courts must consider a
number of factors. These include the treatment of the land under state and local
law; the physical characteristics of the land; and the prospective value of the
regulated land. The endeavor should determine whether reasonable expectations
about property ownership would lead a landowner to anticipate that his holdings
would be treated as one parcel, or, instead, as separate tracts. The inquiry is
objective, and the reasonable expectations at issue derive from background
customs and the whole of our legal tradition. 

First, courts should give substantial weight to the treatment of the land, in
particular how it is bounded or divided, under state and local law. The
reasonable expectations of an acquirer of land must acknowledge legitimate
restrictions affecting his or her subsequent use and dispensation of the
property. A reasonable restriction that predates a landowner's acquisition,
however, can be one of the objective factors that most landowners would
reasonably consider in forming fair expectations about their property. In a
similar manner, a use restriction which is triggered only after, or because of,
a change in ownership should also guide a court's assessment of reasonable
private expectations.

Second, courts must look to the physical characteristics of the landowner's
property. These include the physical relationship of any distinguishable tracts,
the parcel's topography, and the surrounding human and ecological environment.
In particular, it may be relevant that the property is located in an area that
is subject to, or likely to become subject to, environmental or other
regulation.

Third, courts should assess the value of the property under the challenged
regulation, with special attention to the effect of burdened land on the value
of other holdings. Though a use restriction may decrease the market value of the
property, the effect may be tempered if the regulated land adds value to the
remaining property, such as by increasing privacy, expanding recreational space,
or preserving surrounding natural beauty. A law that limits use of a landowner's
small lot in one part of the city by reason of the landowner's nonadjacent
holdings elsewhere may decrease the market value of the small lot in an
unmitigated fashion. The absence of a special relationship between the holdings
may counsel against consideration of all the holdings as a single parcel, making
the restrictive law susceptible to a takings challenge. On the other hand, if
the landowner's other property is adjacent to the small lot, the market value of
the properties may well increase if their combination enables the expansion of a
structure, or if development restraints for one part of the parcel protect the
unobstructed skyline views of another part. That, in turn, may counsel in favor
of treatment as a single parcel and may reveal the weakness of a regulatory
takings challenge to the law.\ldots


\readinghead{B}

The State of Wisconsin and petitioners each ask this Court to adopt a
formalistic rule to guide the parcel inquiry. Neither proposal suffices to
capture the central legal and factual principles that inform reasonable
expectations about property interests.

Wisconsin would tie the definition of the parcel to state law, considering the
two lots here as a single whole due to their merger under the challenged
regulations. That approach, as already noted, simply assumes the answer to the
question: May the State define the relevant parcel in a way that permits it to
escape its responsibility to justify regulation in light of legitimate property
expectations? It is, of course, unquestionable that the law must recognize those
legitimate expectations in order to give proper weight to the rights of owners
and the right of the State to pass reasonable laws and regulations. . . . ~

Petitioners propose a different test that is also flawed. They urge the Court to
adopt a presumption that lot lines define the relevant parcel in every instance,
making Lot E the necessary denominator. Petitioners' argument, however, ignores
the fact that lot lines are themselves creatures of state law, which can be
overridden by the State in the reasonable exercise of its power. In effect,
petitioners ask this Court to credit the aspect of state law that favors their
preferred result (lot lines) and ignore that which does not (merger provision).

This approach contravenes the Court's case law, which recognizes that reasonable
land-use regulations do not work a taking.\ldots

The merger provision here is likewise a legitimate exercise of government power,
as reflected by its consistency with a long history of state and local merger
regulations that originated nearly a century ago.\ldots

When States or localities first set a minimum lot size, there often are existing
lots that do not meet the new requirements, and so local governments will strive
to reduce substandard lots in a gradual manner. The regulations here represent a
classic way of doing this: by implementing a merger provision, which combines
contiguous substandard lots under common ownership, alongside a grandfather
clause, which preserves adjacent substandard lots that are in separate
ownership. Also, as here, the harshness of a merger provision may be ameliorated
by the availability of a variance from the local zoning authority for landowners
in special circumstances. 

Petitioners' insistence that lot lines define the relevant parcel ignores the
well-settled reliance on the merger provision as a common means of balancing the
legitimate goals of regulation with the reasonable expectations of landowners.
Petitioners' rule would frustrate municipalities' ability to implement minimum
lot size regulations by casting doubt on the many merger provisions that exist
nationwide today. 

Petitioners' reliance on lot lines also is problematic for another reason. Lot
lines have varying degrees of formality across the States, so it is difficult to
make them a standard measure of the reasonable expectations of property owners.
Indeed, in some jurisdictions, lot lines may be subject to informal adjustment
by property owners, with minimal government oversight. The ease of modifying lot
lines also creates the risk of gamesmanship by landowners, who might seek to
alter the lines in anticipation of regulation that seems likely to affect only
part of their property.


\readinghead{IV}

Under the appropriate multifactor standard, it follows that for purposes of
determining whether a regulatory taking has occurred here, petitioners' property
should be evaluated as a single parcel consisting of Lots E and F together.

First, the treatment of the property under state and local law indicates
petitioners' property should be treated as one when considering the effects of
the restrictions. As the Wisconsin courts held, the state and local regulations
merged Lots E and F. The decision to adopt the merger provision at issue here
was for a specific and legitimate purpose, consistent with the widespread
understanding that lot lines are not dominant or controlling in every case.
Petitioners' land was subject to this regulatory burden, moreover, only because
of voluntary conduct in bringing the lots under common ownership after the
regulations were enacted. As a result, the valid merger of the lots under state
law informs the reasonable expectation they will be treated as a single
property.

Second, the physical characteristics of the property support its treatment as a
unified parcel. The lots are contiguous along their longest edge. Their rough
terrain and narrow shape make it reasonable to expect their range of potential
uses might be limited. The land's location along the river is also significant.
Petitioners could have anticipated public regulation might affect their
enjoyment of their property, as the Lower St. Croix was a regulated area under
federal, state, and local law long before petitioners possessed the land.

Third, the prospective value that Lot E brings to Lot F supports considering the
two as one parcel for purposes of determining if there is a regulatory taking.
Petitioners are prohibited from selling Lots E and F separately or from building
separate residential structures on each. Yet this restriction is mitigated by
the benefits of using the property as an integrated whole, allowing increased
privacy and recreational space, plus the optimal location of any improvements. 

The special relationship of the lots is further shown by their combined
valuation. Were Lot E separately saleable but still subject to the development
restriction, petitioners' appraiser would value the property at only \$40,000.
We express no opinion on the validity of this figure. We also note the number is
not particularly helpful for understanding petitioners' retained value in the
properties because Lot E, under the regulations, cannot be sold without Lot F.
The point that is useful for these purposes is that the combined lots are valued
at \$698,300, which is far greater than the summed value of the separate
regulated lots (Lot F with its cabin at \$373,000, according to respondents'
appraiser, and Lot E as an undevelopable plot at \$40,000, according to
petitioners' appraiser). The value added by the lots' combination shows their
complementarity and supports their treatment as one parcel.\ldots

Considering petitioners' property as a whole, the state court was correct to
conclude that petitioners cannot establish a compensable taking in these
circumstances. Petitioners have not suffered a taking under \textit{Lucas}, as
they have not been deprived of all economically beneficial use of their
property. They can use the property for residential purposes, including an
enhanced, larger residential improvement. The property has not lost all economic
value, as its value has decreased by less than 10 percent. 

Petitioners furthermore have not suffered a taking under the more general test
of \textit{Penn Central}. The expert appraisal relied upon by the state courts
refutes any claim that the economic impact of the regulation is severe.
Petitioners cannot claim that they reasonably expected to sell or develop their
lots separately given the regulations which predated their acquisition of both
lots. Finally, the governmental action was a reasonable land-use regulation,
enacted as part of a coordinated federal, state, and local effort to preserve
the river and surrounding land.\ldots

Justice \textsc{Gorsuch} took no part in the consideration or decision of this
case.

\opinion Chief Justice \textsc{Roberts}, with whom Justice \textsc{Thomas} and
Justice \textsc{Alito} join, dissenting.

The Murr family owns two adjacent lots along the Lower St. Croix River. Under a
local regulation, those two properties may not be ``sold or developed as
separate lots'' because neither contains a sufficiently large area of buildable
land. Wis. Admin. Code \S~NR 118.08(4)(a)(2) (2017). The Court today holds that
the regulation does not effect a taking that requires just compensation. This
bottom-line conclusion does not trouble me; the majority presents a fair case
that the Murrs can still make good use of both lots, and that the ordinance is a
commonplace tool to preserve scenic areas, such as the Lower St. Croix River,
for the benefit of landowners and the public alike.

Where the majority goes astray, however, is in concluding that the definition of
the ``private property'' at issue in a case such as this turns on an elaborate
test looking not only to state and local law, but also to (1) ``the physical
characteristics of the land,'' (2) ``the prospective value of the regulated
land,'' (3) the ``reasonable expectations'' of the owner, and (4) ``background
customs and the whole of our legal tradition.'' Our decisions have, time and
again, declared that the Takings Clause protects private property rights as
state law creates and defines them. By securing such \textit{established}
property rights, the Takings Clause protects individuals from being forced to
bear the full weight of actions that should be borne by the public at large. The
majority's new, malleable definition of ``private property''---adopted solely
``for purposes of th[e] takings inquiry''---undermines that protection.

I would stick with our traditional approach: State law defines the boundaries of
distinct parcels of land, and those boundaries should determine the ``private
property'' at issue in regulatory takings cases. Whether a regulation effects a
taking of that property is a separate question, one in which common ownership of
adjacent property may be taken into account. Because the majority departs from
these settled principles, I respectfully dissent.\ldots

Because a regulation amounts to a taking if it completely destroys a property's
productive use, there is an incentive for owners to define the relevant
``private property'' narrowly. This incentive threatens the careful balance
between property rights and government authority that our regulatory takings
doctrine strikes: Put in terms of the familiar ``bundle'' analogy, each
``strand'' in the bundle of rights that comes along with owning real property is
a distinct property interest. If owners could define the relevant ``private
property'' at issue as the specific ``strand'' that the challenged regulation
affects, they could convert nearly all regulations into \textit{per se}
takings.

And so we do not allow it. In \textit{Penn Central Transportation Co. v. New
York City}, we held that property owners may not ``establish a `taking' simply
by showing that they have been denied the ability to exploit a property
interest.'' In that case, the owner of Grand Central Terminal in New York City
argued that a restriction on the owner's ability to add an office building atop
the station amounted to a taking of its air rights. We rejected that narrow
definition of the ``property'' at issue, concluding that the correct unit of
analysis was the owner's ``rights in the parcel as a whole.''\ldots

The question presented in today's case concerns the ``parcel as a whole''
language from \textit{Penn Central}. This enigmatic phrase has created confusion
about how to identify the relevant property in a regulatory takings case when
the claimant owns more than one plot of land. Should the impact of the
regulation be evaluated with respect to each individual plot, or with respect to
adjacent plots grouped together as one unit? According to the majority, a court
should answer this question by considering a number of facts about the land and
the regulation at issue. The end result turns on whether those factors ``would
lead a landowner to anticipate that his holdings would be treated as one parcel,
or, instead, as separate tracts.'' 

I think the answer is far more straightforward: State laws define the boundaries
of distinct units of land, and those boundaries should, in all but the most
exceptional circumstances, determine the parcel at issue. Even in regulatory
takings cases, the first step of the Takings Clause analysis is still to
identify the relevant ``private property.'' States create property rights with
respect to particular ``things.'' And in the context of real property, those
``things'' are horizontally bounded plots of land. States may define those plots
differently---some using metes and bounds, others using government surveys,
recorded plats, or subdivision maps. But the definition of property draws the
basic line between, as P.G. Wodehouse would put it, \textit{meum} and
\textit{tuum}. The question of who owns what is pretty important: The rules must
provide a readily ascertainable definition of the land to which a particular
bundle of rights attaches that does not vary depending upon the purpose at
issue. 

Following state property lines is also entirely consistent with \textit{Penn
Central}. Requiring consideration of the ``parcel as a whole'' is a response to
the risk that owners will strategically pluck one strand from their bundle of
property rights---such as the air rights at issue in \textit{Penn Central}---and
claim a complete taking based on that strand alone. That risk of strategic
unbundling is not present when a legally distinct parcel is the basis of the
regulatory takings claim. State law defines all of the interests that come along
with owning a particular parcel, and both property owners and the government
must take those rights as they find them.

The majority envisions that relying on state law will create other opportunities
for ``gamesmanship'' by landowners and States: The former, it contends, ``might
seek to alter [lot] lines in anticipation of regulation,'' while the latter
might pass a law that ``consolidates\ldots property'' to avoid a successful
takings claim. But such obvious attempts to alter the legal landscape in
anticipation of a lawsuit are unlikely and not particularly difficult to detect
and disarm. We rejected the strategic splitting of property rights in
\textit{Penn Central}, and courts could do the same if faced with an attempt to
create a takings-specific definition of ``private property.''

Once the relevant property is identified, the real work begins. To decide
whether the regulation at issue amounts to a ``taking,'' courts should focus on
the effect of the regulation on the ``private property'' at issue. Adjacent land
under common ownership may be relevant to that inquiry. The owner's possession
of such a nearby lot could, for instance, shed light on how the owner reasonably
expected to use the parcel at issue before the regulation.\ldots

In sum, the ``parcel as a whole'' requirement prevents a property owner from
identifying a single ``strand'' in his bundle of property rights and claiming
that interest has been taken. Allowing that strategic approach to defining
``private property'' would undermine the balance struck by our regulatory
takings cases. Instead, state law creates distinct parcels of land and defines
the rights that come along with owning those parcels. Those established bundles
of rights should define the ``private property'' in regulatory takings cases.
While ownership of contiguous properties may bear on whether a person's plot has
been ``taken,'' \textit{Penn Central} provides no basis for disregarding state
property lines when identifying the ``parcel as a whole.''


\readinghead{II}

The lesson that the majority draws from \textit{Penn Central} is that defining
``the proper parcel in regulatory takings cases cannot be solved by any simple
test.'' Following through on that stand against simplicity, the majority lists a
complex set of factors theoretically designed to reveal whether a hypothetical
landowner might expect that his property ``would be treated as one parcel, or,
instead, as separate tracts.'' Those factors, says the majority, show that Lots
E and F of the Murrs' property constitute a single parcel and that the local
ordinance requiring the Murrs to develop and sell those lots as a pair does not
constitute a taking.

In deciding that Lots E and F are a single parcel, the majority focuses on the
importance of the ordinance at issue and the extent to which the Murrs may have
been especially surprised, or unduly harmed, by the application of that
ordinance to their property. But these issues should be considered when deciding
if a regulation constitutes a ``taking.'' Cramming them into the definition of
``private property'' undermines the effectiveness of the Takings Clause as a
check on the government's power to shift the cost of public life onto private
individuals.

The problem begins when the majority loses track of the basic structure of
claims under the Takings Clause. While it is true that we have referred to
regulatory takings claims as involving ``essentially ad hoc, factual
inquiries,'' we have conducted those wide-ranging investigations when assessing
``the question of what constitutes a `\textit{taking}'\,'' under \textit{Penn
Central}. \textit{Ruckelshaus}, 467 U.S., at 1004 (emphasis added). And even
then, we reach that ``ad hoc'' \textit{Penn Central} framework only after
determining that the regulation did not deny all productive use of the parcel.
Both of these inquiries presuppose that the relevant ``private property'' has
already been identified. There is a simple reason why the majority does not cite
a single instance in which we have made that identification by relying on
anything other than state property principles---we have never done so.

In departing from state property principles, the majority authorizes governments
to do precisely what we rejected in \textit{Penn Central} : create a
litigation-specific definition of ``property'' designed for a claim under the
Takings Clause. Whenever possible, governments in regulatory takings cases will
ask courts to aggregate legally distinct properties into one ``parcel,'' solely
for purposes of resisting a particular claim. And under the majority's test,
identifying the ``parcel as a whole'' in such cases will turn on the
reasonableness of the regulation as applied to the claimant. The result is that
the government's regulatory interests will come into play not once, but
twice---first when identifying the relevant parcel, and again when determining
whether the regulation has placed too great a public burden on that property.

Regulatory takings, however---by their very nature---pit the common good against
the interests of a few. There is an inherent imbalance in that clash of
interests. The widespread benefits of a regulation will often appear far
weightier than the isolated losses suffered by individuals. And looking at the
bigger picture, the overall societal good of an economic system grounded on
private property will appear abstract when cast against a concrete regulatory
problem. In the face of this imbalance, the Takings Clause ``prevents the public
from loading upon one individual more than his just share of the burdens of
government,'' \textit{Monongahela Nav. Co. v. United States}, 148 U.S. 312, 325
(1893), by considering the effect of a regulation on specific property rights as
they are established at state law. But the majority's approach undermines that
protection, defining property only after engaging in an ad hoc, case-specific
consideration of individual and community interests. The result is that the
government's goals shape the playing field before the contest over whether the
challenged regulation goes ``too far'' even gets underway.

Suppose, for example, that a person buys two distinct plots of land---known as
Lots A and B---from two different owners. Lot A is landlocked, but the
neighboring Lot B shares a border with a local beach. It soon comes to light,
however, that the beach is a nesting habitat for a species of turtle. To protect
this species, the state government passes a regulation preventing any
development or recreation in areas abutting the beach---including Lot B. If that
lot became the subject of a regulatory takings claim, the purchaser would have a
strong case for a \textit{per se} taking: Even accounting for the owner's
possession of the other property, Lot B had no remaining economic value or
productive use. But under the majority's approach, the government can argue
that---based on all the circumstances and the nature of the regulation---Lots A
and B should be considered one ``parcel.'' If that argument succeeds, the
owner's \textit{per se} takings claim is gone, and he is left to roll the dice
under the \textit{Penn Central} balancing framework, where the court will, for a
second time, throw the reasonableness of the government's regulatory action into
the balance.

The majority assures that, under its test, ``[d]efining the property\ldots
should not \textit{necessarily} preordain the outcome in \textit{every} case.''
(emphasis added). The underscored language cheapens the assurance. The framework
laid out today provides little guidance for identifying whether ``expectations
about property ownership would lead a landowner to anticipate that his holdings
would be treated as one parcel, or, instead, as separate tracts.'' Instead, the
majority's approach will lead to definitions of the ``parcel'' that have far
more to do with the reasonableness of applying the challenged regulation to a
particular landowner. The result is clear double counting to tip the scales in
favor of the government: Reasonable government regulation should have been
anticipated by the landowner, so the relevant parcel is defined consistent with
that regulation. In deciding whether there is a taking under the second step of
the analysis, the regulation will seem eminently reasonable given its impact on
the pre-packaged parcel. Not, as the Court assures us, ``necessarily'' in
``every'' case, but surely in most.

Moreover, given its focus on the particular challenged regulation, the
majority's approach must mean that two lots might be a single ``parcel'' for one
takings claim, but separate ``parcels'' for another. This is just another
opportunity to gerrymander the definition of ``private property'' to defeat a
takings claim.\ldots

Put simply, today's decision knocks the definition of ``private property'' loose
from its foundation on stable state law rules and throws it into the maelstrom
of multiple factors that come into play at the second step of the takings
analysis. The result: The majority's new framework compromises the Takings
Clause as a barrier between individuals and the press of the public interest.


\readinghead{III}

Staying with a state law approach to defining ``private property'' would make
our job in this case fairly easy. The Murr siblings acquired Lot F in 1994 and
Lot E a year later. Once the lots fell into common ownership, the challenged
ordinance prevented them from being ``sold or developed as separate lots''
because neither contained a sufficiently large area of buildable land. The Murrs
argued that the ordinance amounted to a taking of Lot E, but the State of
Wisconsin and St. Croix County proposed that both lots together should count as
the relevant ``parcel.''\ldots

As I see it, the Wisconsin Court of Appeals was wrong to apply a
takings-specific definition of the property at issue. Instead, the court should
have asked whether, under general state law principles, Lots E and F are legally
distinct parcels of land. I would therefore vacate the judgment below and remand
for the court to identify the relevant property using ordinary principles of
Wisconsin property law.

After making that state law determination, the next step would be to determine
whether the challenged ordinance amounts to a ``taking.'' If Lot E is a legally
distinct parcel under state law, the Court of Appeals would have to perform the
takings analysis anew, but could still consider many of the issues the majority
finds important. The majority, for instance, notes that under the ordinance the
Murrs can use Lot E as ``recreational space,'' as the ``location of any
improvements,'' and as a valuable addition to Lot F. These facts could be
relevant to whether the ``regulation denies all economically beneficial or
productive use'' of Lot E. Similarly, the majority touts the benefits of the
ordinance and observes that the Murrs had little use for Lot E independent of
Lot F and could have predicted that Lot E would be regulated. These facts speak
to ``the economic impact of the regulation,'' interference with
``investment-backed expectations,'' and the ``character of the governmental
action''---all things we traditionally consider in the \textit{Penn Central}
analysis.

I would be careful, however, to confine these considerations to the question
whether the regulation constitutes a taking. As Alexander Hamilton explained,
``the security of Property'' is one of the ``great object[s] of government.'' 1
Records of the Federal Convention of 1787, p. 302 (M. Farrand ed. 1911). The
Takings Clause was adopted to ensure such security by protecting property rights
as they exist under state law. Deciding whether a regulation has gone so far as
to constitute a ``taking'' of one of those property rights is, properly enough,
a fact-intensive task that relies ``as much on the exercise of judgment as on
the application of logic.'' \textit{MacDonald, Sommer \& Frates v. Yolo County},
477 U.S. 340, 349 (1986) (alterations and internal quotation marks omitted). But
basing the definition of ``property'' on a judgment call, too, allows the
government's interests to warp the private rights that the Takings Clause is
supposed to secure.

I respectfully dissent.

\opinion Justice \textsc{Thomas}, dissenting.

I join \textsc{The Chief Justice}'s dissent because it correctly applies this
Court's regulatory takings precedents, which no party has asked us to
reconsider. The Court, however, has never purported to ground those precedents
in the Constitution as it was originally understood. In \textit{Pennsylvania
Coal Co. v. Mahon}, 260 U.S. 393, 415 (1922), the Court announced a ``general
rule'' that ``if regulation goes too far it will be recognized as a taking.''
But we have since observed that, prior to \textit{Mahon}, ``it was generally
thought that the Takings Clause reached only a `direct appropriation' of
property, \textit{Legal Tender Cases}, 12 Wall. 457, 551 (1871), or the
functional equivalent of a `practical ouster of [the owner's] possession,'
\textit{Transportation Co. v. Chicago}, 99 U.S. 635, 642 (1879).'' \textit{Lucas
v. South Carolina Coastal Council}, 505 U.S. 1003, 1014 (1992). In my view, it
would be desirable for us to take a fresh look at our regulatory takings
jurisprudence, to see whether it can be grounded in the original public meaning
of the Takings Clause of the Fifth Amendment or the Privileges or Immunities
Clause of the Fourteenth Amendment. See generally Rappaport, Originalism and
Regulatory Takings: Why the Fifth Amendment May Not Protect Against Regulatory
Takings, but the Fourteenth Amendment May, 45 San Diego L. Rev. 729 (2008)
(describing the debate among scholars over those questions).

