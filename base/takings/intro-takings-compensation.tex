What is just compensation? The standard approach is fair market value.
\textit{See, e.g.}, 735 Ill. Comp. Stat. Ann. \S~30/10-5-60 (``[T]he fair cash
market value of property in a proceeding in eminent domain shall be the amount
of money that a purchaser, willing, but not obligated, to buy the property,
would pay to an owner willing, but not obliged, to sell in a voluntary sale.'').
This amount may include costs directly attributable to the condemnation.
\textit{See id.} \S~735 Ill. Comp. Stat. Ann. 30/10-5-62 (providing for
compensation of reasonable relocation costs).

Evidentiary difficulties aside, the fair market value metric potentially
understates the value of the home from the perspective of the property owner in
at least three ways. First, fair market value ignores subjective values. A
property owner often values it more than the market (as reflected by the fact
that it has not yet been sold for the market price). If the property is a home,
it may have high sentimental value (e.g., if it is where one raised children) or
offer idiosyncratic amenities that cannot be easily duplicated but are not
reflected in market price (e.g., proximity to friends, work, etc.). Second,
eminent domain is a forced transaction. The landowner may experience the
transaction as a violation of personal autonomy. Third, to the extent the
project produces a surplus, the displaced landowner does not get a share. In
other words, suppose five lots are each individually worth \$10,000, but they
can be assembled into a park that confers \$100,000 of benefits on the
surrounding area. The owners of the condemned lots do not share in the surplus,
they still receive only \$10,000. \textit{See, e.g.}, 735 \textsc{Ill. Comp.
Stat. Ann.} \S~30/10-5-60 (``In the condemnation of property for a public
improvement, there shall be excluded from the fair cash market value of the
property any appreciation in value proximately caused by the improvement and any
depreciation in value proximately caused by the improvement'').

What happens when only part of a parcel is taken? The general approach is to
allow compensation for the effect of the severance on the land retained by the
condemnee. Imagine O owns Blackacre and Whiteacre as one parcel with a combined
value of \$100,000. If Blackacre is taken for a fair market value of \$50,000,
and the severance leaves Whiteacre worth only \$40,000, O is entitled to
compensation for the lost \$10,000. Note, however, that if O owned \textit{only}
Whiteacre, and its value was reduced by \$10,000 due to the next-door
condemnation of Blackacre, O would receive nothing. 13-79F \textsc{Powell on
Real Property} \S~79F.04. 

What if a partial taking \textit{enhances} the value of the remainder?
\textit{See, e.g.}, 735 \textsc{Ill. Comp. Stat. Ann.} \S~30/10-5-55 (``In
assessing damages or compensation for any taking or property acquisition under
this Act, due consideration shall be given to any special benefit that will
result to the property owner from any public improvement to be erected on the
property.''); \emph{Illinois State Toll Highway Auth. v. Am. Nat. Bank \& Trust
Co. of Chicago}, 642 N.E.2d 1249, 1255 (Ill. 1994) (``[S]pecial benefits are any
benefits to the property that enhance its market value and are not conjectural
or speculative.''). 

This mix of rules leads to results that may strike you as unfair. Imagine a
government project to build a subway station, and three affected landowners,
Alice, Bob, and Charles. Alice's parcel is condemned in its entirety; half of
Bob's land is condemned; and Charles's land is untouched. Suppose further that
the transit station leads to a doubling in the property values of the
surrounding land. On these facts, Alice receives the pre-project value of her
land. Bob receives nothing (assuming the appreciation of his retained half
matches the pre-project value of the condemned portion); and Charles receives a
windfall. Is there any way to avoid these difficulties?

Holders of future interests are also entitled to compensation. \textit{See
generally} 2-5 \textsc{Nichols on Eminent Domain} \S~5.02; \textit{see, e.g.},
\textsc{Cal. Code Civ. Proc.} \S~1265.420 (``Where property acquired for public
use is subject to a life tenancy, upon petition of the life tenant or any other
person having an interest in the property, the court may order any of the
following: (a) An apportionment and distribution of the award based on the value
of the interest of life tenant and remainderman; (b) The compensation to be used
to purchase comparable property to be held subject to the life tenancy; (c) The
compensation to be held in trust and invested and the income (and, to the extent
the instrument that created the life tenancy permits, principal) to be
distributed to the life tenant for the remainder of the tenancy; (d) Such other
arrangement as will be equitable under the circumstances.''). 

