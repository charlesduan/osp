Local governments carry out condemnations in a variety of ways. There is no
standard eminent domain regime. Some states require some sort of
pre-condemnation activity (e.g., formal findings that a condemnation is
necessary or efforts to negotiate with the landowner); others do not. Some
jurisdictions require the condemning authority to initiate a judicial action;
others allow an administrative procedure, giving the landowner the right to
challenge the taking in court. Some states provide for expedited procedures,
``quick take'' provisions, either as an independent cause of action or by motion
within an ongoing proceeding. 13-79F \textsc{Powell on Real Property} \S~79F.06.

In Illinois, for example, the condemning authority files an eminent domain
action in the circuit court for the county of the property. The complaint
details: ``(i) the complainant's authority in the premises, (ii) the purpose for
which the property is sought to be taken or damaged, (iii) a description of the
property, and (iv) the names of all persons interested in the property as owners
or otherwise, as appearing of record, if known.'' 735 \textsc{Ill. Comp. Stat.
Ann.} 30/10-5-10. Either the condemning authority or the property owner may
request a jury trial. Expedited procedures (called a ``quick take'' procedure)
are also available upon motion. \textit{Id.} \S~30/20-5-5.

