\expected{lucas-v-sccc}

\item For some before-and-after photos of the Lucas lot, visit
\url{http://www.dartmouth.edu/~wfischel/lucasupdate.html}. Writing about
\textit{Lucas}, Carol Rose observes that much of what made the case seem unfair
to the reviewing courts---the ``singling out'' of Lucas's lot---was a byproduct
of an effort to limit political opposition to the state's coastal preservation
program by curtailing its regulatory reach. It also limited the ability of the
regulations to combat the problems of development. Carol M. Rose, \textit{The
Story of Lucas: Environmental Land Use Regulation Between Developers and the
Deep Blue Sea} at 24, in \textsc{Environmental Stories} (Richard J. Lazarus and
Oliver A. Houck, eds., Foundation Press, 2005),
\url{http://ssrn.com/abstract=706637}. Bad optics notwithstanding, Rose notes
that the impacts of development do not accumulate in a linear manner. It may
very well make sense to impose restrictions after a period of unchecked growth.
\begin{quotation}
\ldots Environmental resources typically have some threshold below which use is
not harmful, but beyond which marginal costs rise not just additively but
exponentially. Bodies of water, for example, can tolerate some organic
materials, but over a threshold, each increment of additional waste is not just
additively but exponentially more damaging to wildlife, vegetation, and water
quality. The smoke from an old-fashioned house furnace or two will dissipate
without damage, but if you burn enough, you run the risk of a killer fog.
Beachfront management is another clear example of this pattern of exponentially
rising costs. A single revetment or seawall would have had little impact on
South Carolina's beaches or their ability to replenish themselves; what
threatened to become devastating was the accumulation of ever more armored
structures\ldots.

That is why a conventional notion of equality is inadequate with respect to
environmental uses, including land uses. If early uses are relatively harmless,
it would be pointless and overly intrusive to try to regulate them. But
something has to be done when later uses slice far enough out on the salami. At
that later point, it can be an invitation to environmental disaster to look
around at pre-existing uses, and to say that new users should all receive the
same old lax treatment, as Scalia suggested in Lucas.
\end{quotation}
\textit{Id.} at 38.

\item What if someone ``comes to'' the regulation by purchasing a property
\textit{after} the objected-to regulation has been imposed. Does that preclude a
takings challenge? As noted previously, the Court held that takings claims
remain available lest the state ``put an expiration date on the Takings
Clause.'' \textit{Palazzolo v. Rhode Island}, 533 U.S. 606, 627 (2001).

\item \textbf{Can judges take?} On the question of nuisance definition, what if
the state actor declaring/redefining property interests is a court? In
\textit{Stop the Beach Renourishment, Inc. v. Florida Dep't of Envtl. Prot.}, a
four-Justice plurality opinion would have recognized judicial takings as a
viable claim (though in the case at hand it would have found no taking). 560
U.S. 702, 715 (2010) (plurality) (``[T]he Takings Clause bars the State from
taking private property without paying for it, no matter which branch is the
instrument of the taking.''). But don't judges adjust the contours of property
law all the time? How could this basic function of the courts continue if
challengeable as a taking? Some of these issues were taken up in the
concurrences in \textit{Stop the Beach} and the academic commentary that
followed.

\item \textbf{``Inverse condemnation'' procedures}. In regulatory takings
cases, the government typically denies that a taking has occurred, so there is
no condemnation proceeding. Instead, the property owner brings suit seeking
relief. The Tucker Act provides an avenue for federal claimants. The statute
waives United States sovereign immunity for claims founded on the Constitution,
a statute, a regulation, or an express or implied-in-fact contract. 28 U.S.C.
\S~1491(a)(1). The ``Little Tucker Act,'' \S~1346(a)(2), establishes concurrent
jurisdiction in the district courts for claims of less than \$10,000. If a state
government is the offending regulator, the property owner may look to available
state remedies, but may also proceed under the federal civil rights statute. 42
U.S.C. \S~1983. A litigant pursuing a \S~1983 action need not pursue state
remedies first. \emph{Knick v. Twp. of Scott, Pennsylvania}, 139 S. Ct. 2162,
2177 (2019) (``[A] government violates the Takings Clause when it takes property
without compensation, and that a property owner may bring a Fifth Amendment
claim under \S~1983 at that time.\ldots Given the availability of post-taking
compensation, barring the government from acting will ordinarily not be
appropriate. But because the violation is complete at the time of the taking,
pursuit of a remedy in federal court need not await any subsequent state
action.'').

\item \textit{First English Evangelical Lutheran Church of Glendale v. County of
Los Angeles}, 482 U.S. 304 (1987), held that compensation is required for
temporary takings. This opened the door to the argument that regulations
temporarily suspending certain land uses are takings under the \textit{Lucas}
categorical rule. The Court addressed the claim in the following case.
\expectnext{tspc-v-trpa}

