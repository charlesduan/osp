\expected{loretto-v-teleprompter}

\item \having{gruen-v-gruen}{Remember Michael Gruen of \textit{Gruen v. Gruen}
fame? He became a lawyer and argued the case for Loretto.}{We'll later read
\emph{Gruen v.~Gruen}, a case about transfers of property by gift. Loretto's
lawyer was Michael Gruen from that case.}{The original Gifts module of
\emph{Open Source Property} contains a case \emph{Gruen v. Gruen}. Michael
Gruen, a party to that case, became a lawyer and argued this case for Loretto.}

\item Loretto's victory at the Supreme Court amounted to little. The Commission
on Cable Television decided that \$1 sufficed as compensation because cable
television access enhances property values, and the Court of Appeals held it was
permissible for the compensation to be set by the commission, subject to later
judicial review, rather than a court. \emph{Loretto v. Teleprompter Manhattan
CATV Corp.}, 446 N.E.2d 428, 434 (N.Y. 1983).

\item \textbf{Categorical Rules.} One debate between the majority and the
dissent concerns the merits of rules versus standards. As we will discuss in
greater detail, the Court had developed a balancing test for determining whether
government regulation goes ``too far'' and becomes a taking. The question thus
arose whether that balancing test applies to \textit{all} takings inquiries.
Even if \textit{Loretto} had gone the dissent's way, it still would be the case
that physical invasions would generally be takings. The dispute was over whether
courts have the discretion to treat certain minor intrusions sufficiently
\textit{de minimis} as not to require compensation. The Court rejected this
approach, clarifying that any permanent physical occupation by or authorized by
the government is a taking as a categorical, per se, matter. 

\item A consequence of the rule is that certain minor intrusions merit
compensation, while more costly regulations may pass muster under the balancing
test. That problem aside, Justice Blackmun claims that the \textit{per se}
occupations rule lacks the compensating benefit of ease of application, pointing
to the difficulty of distinguishing permanent from temporary occupations. Do you
agree?

\item Another point of contention between the majority and dissent is whether it
is sensible to allow the state to require by regulation the installation of
cable (or other) facilities, but prohibit it from directly authorizing their
installation. At some point, might regulation become so extensive that it
constitutes a \textit{de facto} occupation? \textit{Yee v. City of Escondido},
503 U.S. 519 (1992), rejects the argument that rent control laws fall under
\textit{Loretto}'s categorical rule, concluding that the decision of the
landlord to lease the premises negates the claim of any forced physical
occupation. 

\item \textbf{Personal property.} How does \textit{Loretto} apply to personal
property? \textit{Horne v. Department of Agriculture}, 576 U.S. 350 (2015),
addressed a challenge to a Department of Agriculture program intended to promote
stability in the raisin market. The program issued marketing orders that
required raisin farmers to set aside a certain percentage of their annual crop.
The government took title to the reserved raisins and disposed of them in a
variety of ways, including sales in non-competitive markets, returning any net
profits to the growers. The Court held this to be a taking under
\textit{Loretto.} 
\begin{quote}
Raisin growers subject to the reserve requirement thus lose the entire
``bundle'' of property rights in the appropriated raisins---``the rights to
possess, use and dispose of'' them, \textit{Loretto}, 458 U.S., at 435 (internal
quotation marks omitted)---with the exception of the speculative hope that some
residual proceeds may be left when the Government is done with the raisins and
has deducted the expenses of implementing all aspects of the marketing order.
The Government's ``actual taking of possession and control'' of the reserve
raisins gives rise to a taking as clearly ``as if the Government held full title
and ownership,'' id., at 431 (internal quotation marks omitted), as it
essentially does. The Government's formal demand that the [farmers] turn over a
percentage of their raisin crop without charge, for the Government's control and
use, is ``of such a unique character that it is a taking without regard to other
factors that a court might ordinarily examine.'' Id., at 432.
\end{quote}
576 U.S. at 361-62. As in \textit{Loretto}, the Court rejected the argument that
the reserve requirement was permissible given that the government could achieve
the same end by simply prohibiting the farmers from selling a portion of their
crop. 
\begin{quote}
[T]hat distinction flows naturally from the settled difference in our takings
jurisprudence between appropriation and regulation. A physical taking of raisins
and a regulatory limit on production may have the same economic impact on a
grower. The Constitution, however, is concerned with means as well as ends. 
\end{quote}
\textit{Id.} The Court likewise determined that the farmers' retention of a
contingent monetary interest in the sale of the reserved raisins did not negate
the physical taking. Dissenting, Justice Sotomayor argued that
\textit{Loretto}'s \textit{per se} rule applies only when \textit{all} property
rights have been taken, and the farmers' contingent interest negates use of the
per se rule.

\item \textbf{\textit{Cedar Point} and ``rights of access''---expanding
the \textit{per se} rule}. \textit{Loretto} draws a
distinction between ``permanent'' and ``temporary'' occupations. \emph{Loretto
v. Teleprompter Manhattan CATV Corp.}, 458 U.S. 419, 428 (1982) (``[T]his Court
has consistently distinguished between flooding cases involving a
\textit{permanent} physical occupation, on the one hand, and cases involving a
more \textit{temporary} invasion, or government action outside the owner's
property that causes consequential damages within, on the other. A taking has
always been found \textit{only} in the former situation.'' (emphases added)). In
so doing, the Court cited \textit{PruneYard Shopping Center v. Robins}, 447 U.S.
74 (1980), which upheld California law requiring a shopping center to allow
visitors to exercise free speech and petition rights on its property.
\textit{Id.} at 434. On this logic, permanent occupations receive \textit{per
se} treatment, while temporary occupations are evaluated under the balancing
test---discussed in the next section---to determine whether they rise to the
level of a taking.

In \textit{Cedar Point} \textit{Nursery v. Hassid}, 141 S. Ct. 2063
(2021), the Court greatly expanded the class of cases subject to a \textit{per
se} rule. A California regulation required agricultural employers to permit
union organizers to come onto their property. The Court held that this
regulation was a \textit{per se} taking. 
\begin{quote}
The access regulation appropriates a right to invade the growers' property and
therefore constitutes a per se physical taking. The regulation grants union
organizers a right to physically enter and occupy the growers' land for three
hours per day, 120 days per year. Rather than restraining the growers' use of
their own property, the regulation appropriates for the enjoyment of third
parties the owners' right to exclude.
\end{quote}
141 S. Ct. at 2072. \textit{Cedar Point} raises a host of significant
questions, not only about the scope of the per se rule, but also about the
state's ability to issue regulations that depend on the ability of non-owners to
access property (e.g., the conduct of health inspections).
\having{cedar-point-v-hassid}{}{It is difficult to
discuss \textit{Cedar Point}'s treatment of these issues without first exploring
the Court's jurisprudence on regulatory takings and exactions, both of which are
covered below. We therefore hold off on a full treatment of \textit{Cedar Point}
until after those units.}{}

