\reading{Loretto v. Teleprompter Manhattan CATV Corp.}

\readingcite{458 U.S. 419 (1982)}

\opinion Justice \textsc{Marshall} delivered the opinion of the Court.

This case presents the question whether a minor but permanent physical
occupation of an owner's property authorized by government constitutes a
``taking'' of property for which just compensation is due under the Fifth and
Fourteenth Amendments of the Constitution. New York law provides that a landlord
must permit a cable television company to install its cable facilities upon his
property. In this case, the cable installation occupied portions of appellant's
roof and the side of her building. The New York Court of Appeals ruled that this
appropriation does not amount to a taking. Because we conclude that such a
physical occupation of property is a taking, we reverse.


\readinghead{I}

Appellant Jean Loretto purchased a five-story apartment building located at 303
West 105th Street, New York City, in 1971. The previous owner had granted
appellees Teleprompter Corp. and Teleprompter Manhattan CATV (collectively
Teleprompter) permission to install a cable on the building and the exclusive
privilege of furnishing cable television (CATV) services to the tenants. The New
York Court of Appeals described the installation as follows:
\begin{quote}
``On June 1, 1970 TelePrompter installed a cable slightly less than one-half
inch in diameter and of approximately 30 feet in length along the length of the
building about 18 inches above the roof top, and directional taps, approximately
4 inches by 4 inches by 4 inches, on the front and rear of the roof. By June 8,
1970 the cable had been extended another 4 to 6 feet and cable had been run from
the directional taps to the adjoining building at 305 West 105th Street.''
\end{quote}

Teleprompter also installed two large silver boxes along the roof cables. The
cables are attached by screws or nails penetrating the masonry at approximately
two-foot intervals, and other equipment is installed by bolts.

Initially, Teleprompter's roof cables did not service appellant's building. They
were part of what could be described as a cable ``highway'' circumnavigating the
city block, with service cables periodically dropped over the front or back of a
building in which a tenant desired service. Crucial to such a network is the use
of so-called ``crossovers''---cable lines extending from one building to another
in order to reach a new group of tenants. Two years after appellant purchased
the building, Teleprompter connected a ``noncrossover'' line---\textit{i.e.},
one that provided CATV service to appellant's own tenants---by dropping a line
to the first floor down the front of appellant's building.

Prior to 1973, Teleprompter routinely obtained authorization for its
installations from property owners along the cable's route, compensating the
owners at the standard rate of 5\% of the gross revenues that Teleprompter
realized from the particular property. To facilitate tenant access to CATV, the
State of New York enacted \S~828 of the Executive Law, effective January 1,
1973. Section 828 provides that a landlord may not ``interfere with the
installation of cable television facilities upon his property or premises,'' and
may not demand payment from any tenant for permitting CATV, or demand payment
from any CATV company ``in excess of any amount which the [State Commission on
Cable Television] shall, by regulation, determine to be reasonable.'' The
landlord may, however, require the CATV company or the tenant to bear the cost
of installation and to indemnify for any damage caused by the installation.
Pursuant to \S~828(1)(b), the State Commission has ruled that a one-time \$1
payment is the normal fee to which a landlord is entitled. The Commission ruled
that this nominal fee, which the Commission concluded was equivalent to what the
landlord would receive if the property were condemned pursuant to New York's
Transportation Corporations Law, satisfied constitutional requirements ``in the
absence of a special showing of greater damages attributable to the taking.''

Appellant did not discover the existence of the cable until after she had
purchased the building. She brought a class action against Teleprompter in 1976
on behalf of all owners of real property in the State on which Teleprompter has
placed CATV components, alleging that Teleprompter's installation was a trespass
and, insofar as it relied on \S~828, a taking without just compensation. She
requested damages and injunctive relief. Appellee City of New York, which has
granted Teleprompter an exclusive franchise to provide CATV within certain areas
of Manhattan, intervened. The Supreme Court, Special Term, granted summary
judgment to Teleprompter and the city, upholding the constitutionality of \S~828
in both crossover and noncrossover situations. The Appellate Division affirmed
without opinion. 

On appeal, the Court of Appeals, over dissent, upheld the statute.\ldots The
court\ldots ruled that the law serves a legitimate police power
purpose---eliminating landlord fees and conditions that inhibit the development
of CATV, which has important educational and community benefits. Rejecting the
argument that a physical occupation authorized by government is necessarily a
taking, the court stated that the regulation does not have an excessive economic
impact upon appellant when measured against her aggregate property rights, and
that it does not interfere with any reasonable investment-backed expectations.
Accordingly, the court held that \S~828 does not work a taking of appellant's
property. Chief Judge Cooke dissented, reasoning that the physical appropriation
of a portion of appellant's property is a taking without regard to the balancing
analysis courts ordinarily employ in evaluating whether a regulation is a
taking.

In light of its holding, the Court of Appeals had no occasion to determine
whether the \$1 fee ordinarily awarded for a noncrossover installation was
adequate compensation for the taking. Judge Gabrielli, concurring, agreed with
the dissent that the law works a taking but concluded that the \$1 presumptive
award, together with the procedures permitting a landlord to demonstrate a
greater entitlement, affords just compensation. We noted probable jurisdiction. 


\readinghead{II}

The Court of Appeals determined that \S~828 serves the legitimate public purpose
of ``rapid development of and maximum penetration by a means of communication
which has important educational and community aspects,'' and thus is within the
State's police power. We have no reason to question that determination. It is a
separate question, however, whether an otherwise valid regulation so frustrates
property rights that compensation must be paid. We conclude that a permanent
physical occupation authorized by government is a taking without regard to the
public interests that it may serve. Our constitutional history confirms the
rule, recent cases do not question it, and the purposes of the Takings Clause
compel its retention.


\readinghead{A}

In \textit{Penn Central Transportation Co. v. New York City} the Court surveyed
some of the general principles governing the Takings Clause. The Court noted
that no ``set formula'' existed to determine, in all cases, whether compensation
is constitutionally due for a government restriction of property. Ordinarily,
the Court must engage in ``essentially ad hoc, factual inquiries.'' But the
inquiry is not standardless. The economic impact of the regulation, especially
the degree of interference with investment-backed expectations, is of particular
significance. ``So, too, is the character of the governmental action. A `taking'
may more readily be found when the interference with property can be
characterized as a physical invasion by government, than when interference
arises from some public program adjusting the benefits and burdens of economic
life to promote the common good.'' 

As \textit{Penn Central} affirms, the Court has often upheld substantial
regulation of an owner's use of his own property where deemed necessary to
promote the public interest. At the same time, we have long considered a
physical intrusion by government to be a property restriction of an unusually
serious character for purposes of the Takings Clause. Our cases further
establish that when the physical intrusion reaches the extreme form of a
permanent physical occupation, a taking has occurred. In such a case, ``the
character of the government action'' not only is an important factor in
resolving whether the action works a taking but also is determinative.

When faced with a constitutional challenge to a permanent physical occupation of
real property, this Court has invariably found a taking. As early as 1872, in
\textit{Pumpelly v. Green Bay Co.}, 13 Wall. (80 U.S.) 166, this Court held that
the defendant's construction, pursuant to state authority, of a dam which
permanently flooded plaintiff's property constituted a taking. A unanimous Court
stated, without qualification, that ``where real estate is actually invaded by
superinduced additions of water, earth, sand, or other material, or by having
any artificial structure placed on it, so as to effectually destroy or impair
its usefulness, it is a taking, within the meaning of the Constitution.''
\textit{Id.}, at 181. Seven years later, the Court reemphasized the importance
of a physical occupation by distinguishing a regulation that merely restricted
the use of private property. In \textit{Northern Transportation Co. v. Chicago},
99 U.S. 635 (1879), the Court held that the city's construction of a temporary
dam in a river to permit construction of a tunnel was not a taking, even though
the plaintiffs were thereby denied access to their premises, because the
obstruction only impaired the use of plaintiffs' property. The Court
distinguished earlier cases in which permanent flooding of private property was
regarded as a taking, \textit{e.g., Pumpelly, supra}, as involving ``a physical
invasion of the real estate of the private owner, and a practical ouster of his
possession.'' In this case, by contrast, ``[n]o entry was made upon the
plaintiffs' lot.'' 

Since these early cases, this Court has consistently distinguished between
flooding cases involving a permanent physical occupation, on the one hand, and
cases involving a more temporary invasion, or government action outside the
owner's property that causes consequential damages within, on the other. A
taking has always been found only in the former situation.

In \textit{St. Louis v. Western Union Telegraph Co.}, 148 U.S. 92 (1893), the
Court applied the principles enunciated in \textit{Pumpelly} to a situation
closely analogous to the one presented today. In that case, the Court held that
the city of St. Louis could exact reasonable compensation for a telegraph
company's placement of telegraph poles on the city's public streets.\ldots 

Similarly, in \textit{Western Union Telegraph Co. v. Pennsylvania R. Co.}, 195
U.S. 540 (1904), a telegraph company constructed and operated telegraph lines
over a railroad's right of way. In holding that federal law did not grant the
company the right of eminent domain or the right to operate the lines absent the
railroad's consent, the Court assumed that the invasion of the telephone lines
would be a compensable taking. \textit{Id.}, at 570 (the right-of-way ``cannot
be appropriated in whole or in part except upon the payment of compensation'').
Later cases, relying on the character of a physical occupation, clearly
establish that permanent occupations of land by such installations as telegraph
and telephone lines, rails, and underground pipes or wires are takings even if
they occupy only relatively insubstantial amounts of space and do not seriously
interfere with the landowner's use of the rest of his land.

More recent cases confirm the distinction between a permanent physical
occupation, a physical invasion short of an occupation, and a regulation that
merely restricts the use of property.\ldots

Although this Court's most recent cases have not addressed the precise issue
before us, they have emphasized that physical \textit{invasion} cases are
special and have not repudiated the rule that any permanent physical
\textit{occupation} is a taking. The cases state or imply that a physical
invasion is subject to a balancing process, but they do not suggest that a
permanent physical occupation would ever be exempt from the Takings
Clause.\ldots


\readinghead{B}

The historical rule that a permanent physical occupation of another's property
is a taking has more than tradition to commend it. Such an appropriation is
perhaps the most serious form of invasion of an owner's property interests. To
borrow a metaphor, the government does not simply take a single ``strand'' from
the ``bundle'' of property rights: it chops through the bundle, taking a slice
of every strand.

Property rights in a physical thing have been described as the rights ``to
possess, use and dispose of it.'' \textit{United States v. General Motors
Corp.}, 323 U.S. 373, 378 (1945). To the extent that the government permanently
occupies physical property, it effectively destroys \textit{each} of these
rights. First, the owner has no right to possess the occupied space himself, and
also has no power to exclude the occupier from possession and use of the space.
The power to exclude has traditionally been considered one of the most treasured
strands in an owner's bundle of property rights. Second, the permanent physical
occupation of property forever denies the owner any power to control the use of
the property; he not only cannot exclude others, but can make no nonpossessory
use of the property. Although deprivation of the right to use and obtain a
profit from property is not, in every case, independently sufficient to
establish a taking, it is clearly relevant. Finally, even though the owner may
retain the bare legal right to dispose of the occupied space by transfer or
sale, the permanent occupation of that space by a stranger will ordinarily empty
the right of any value, since the purchaser will also be unable to make any use
of the property.

Moreover, an owner suffers a special kind of injury when a \textit{stranger}
directly invades and occupies the owner's property. As Part II--A,
\textit{supra}, indicates, property law has long protected an owner's
expectation that he will be relatively undisturbed at least in the possession of
his property. To require, as well, that the owner permit another to exercise
complete dominion literally adds insult to injury. \emph{See} Michelman,
\emph{Property,
Utility, and Fairness: Comments on the Ethical Foundations of ``Just
Compensation'' Law}, 80 Harv. L. Rev. 1165, 1228, and n. 110 (1967).
Furthermore,
such an occupation is qualitatively more severe than a regulation of the
\textit{use} of property, even a regulation that imposes affirmative duties on
the owner, since the owner may have no control over the timing, extent, or
nature of the invasion. 

The traditional rule also avoids otherwise difficult line-drawing problems. Few
would disagree that if the State required landlords to permit third parties to
install swimming pools on the landlords' rooftops for the convenience of the
tenants, the requirement would be a taking. If the cable installation here
occupied as much space, again, few would disagree that the occupation would be a
taking. But constitutional protection for the rights of private property cannot
be made to depend on the size of the area permanently occupied. Indeed, it is
possible that in the future, additional cable installations that more
significantly restrict a landlord's use of the roof of his building will be
made. Section 828 requires a landlord to permit such multiple installations.

Finally, whether a permanent physical occupation has occurred presents
relatively few problems of proof. The placement of a fixed structure on land or
real property is an obvious fact that will rarely be subject to dispute. Once
the fact of occupation is shown, of course, a court should consider the
\textit{extent} of the occupation as one relevant factor in determining the
compensation due. For that reason, moreover, there is less need to consider the
extent of the occupation in determining whether there is a taking in the first
instance.


\readinghead{C}

Teleprompter's cable installation on appellant's building constitutes a taking
under the traditional test. The installation involved a direct physical
attachment of plates, boxes, wires, bolts, and screws to the building,
completely occupying space immediately above and upon the roof and along the
building's exterior wall.

In light of our analysis, we find no constitutional difference between a
crossover and a noncrossover installation. The portions of the installation
necessary for both crossovers and noncrossovers permanently appropriate
appellant's property. Accordingly, each type of installation is a taking.

Appellees raise a series of objections to application of the traditional rule
here. Teleprompter notes that the law applies only to buildings used as rental
property, and draws the conclusion that the law is simply a permissible
regulation of the use of real property. We fail to see, however, why a physical
occupation of one type of property but not another type is any less a physical
occupation. Insofar as Teleprompter means to suggest that this is not a
permanent physical invasion, we must differ. So long as the property remains
residential and a CATV company wishes to retain the installation, the landlord
must permit it.\readingfootnote{17}{It is true that the landlord could avoid
the requirements of \S~828 by ceasing to rent the building to tenants. But a
landlord's ability to rent his property may not be conditioned on his forfeiting
the right to compensation for a physical occupation.\ldots }\ldots

Finally, we do not agree with appellees that application of the physical
occupation rule will have dire consequences for the government's power to adjust
landlord-tenant relationships. This Court has consistently affirmed that States
have broad power to regulate housing conditions in general and the
landlord-tenant relationship in particular without paying compensation for all
economic injuries that such regulation entails. See, \textit{e.g., Heart of
Atlanta Motel, Inc. v. United States}, 379 U.S. 241 (1964) (discrimination in
places of public accommodation); \textit{Queenside Hills Realty Co. v. Saxl},
328 U.S. 80 (1946) (fire regulation); \textit{Bowles v. Willingham}, 321 U.S.
503 (1944) (rent control); \textit{Home Building \& Loan Assn. v. Blaisdell},
290 U.S. 398 (1934) (mortgage moratorium); \textit{Edgar A. Levy Leasing Co. v.
Siegel}, 258 U.S. 242 (1922) (emergency housing law); \textit{Block v. Hirsh},
256 U.S. 135 (1921) (rent control). In none of these cases, however, did the
government authorize the permanent occupation of the landlord's property by a
third party. Consequently, our holding today in no way alters the analysis
governing the State's power to require landlords to comply with building codes
and provide utility connections, mailboxes, smoke detectors, fire extinguishers,
and the like in the common area of a building. So long as these regulations do
not require the landlord to suffer the physical occupation of a portion of his
building by a third party, they will be analyzed under the multifactor inquiry
generally applicable to nonpossessory governmental activity. See \textit{Penn
Central Transportation Co. v. New York City}, 438 U.S. 104
(1978).\readingfootnote{19}{If \S~828 required landlords to provide cable
installation if a tenant so desires, the statute might present a different
question from the question before us, since the landlord would own the
installation. Ownership would give the landlord rights to the placement, manner,
use, and possibly the disposition of the installation. The fact of ownership is,
contrary to the dissent, not simply ``incidental''; it would give a landlord
(rather than a CATV company) full authority over the installation except only as
government specifically limited that authority. The landlord would decide how to
comply with applicable government regulations concerning CATV and therefore
could minimize the physical, esthetic, and other effects of the installation.
Moreover, if the landlord wished to repair, demolish, or construct in the area
of the building where the installation is located, he need not incur the burden
of obtaining the CATV company's cooperation in moving the cable.\par In this
case, by contrast, appellant suffered injury that might have been obviated if
she had owned the cable and could exercise control over its installation. The
drilling and stapling that accompanied installation apparently caused physical
damage to appellant's building. Appellant, who resides in her building, further
testified that the cable installation is ``ugly.'' Although \S~828 provides that
a landlord may require ``reasonable'' conditions that are ``necessary'' to
protect the appearance of the premises and may seek indemnity for damage, these
provisions are somewhat limited. Even if the provisions are effective, the
inconvenience to the landlord of initiating the repairs remains a cognizable
burden.}


\readinghead{III}

Our holding today is very narrow. We affirm the traditional rule that a
permanent physical occupation of property is a taking. In such a case, the
property owner entertains a historically rooted expectation of compensation, and
the character of the invasion is qualitatively more intrusive than perhaps any
other category of property regulation. We do not, however, question the equally
substantial authority upholding a State's broad power to impose appropriate
restrictions upon an owner's \textit{use} of his property.

Furthermore, our conclusion that \S~828 works a taking of a portion of
appellant's property does not presuppose that the fee which many landlords had
obtained from Teleprompter prior to the law's enactment is a proper measure of
the value of the property taken. The issue of the amount of compensation that is
due, on which we express no opinion, is a matter for the state courts to
consider on remand.\readingfootnote{20}{In light of
our disposition of appellant's takings claim, we do not address her contention
that \S~828 deprives her of property without due process of law.}\ldots

\opinion Justice \textsc{Blackmun}, with whom Justice \textsc{Brennan} and
Justice \textsc{White} join, dissenting.

\ldots In my view, the Court's approach ``reduces the constitutional issue to a
formalistic quibble'' over whether property has been ``permanently occupied'' or
``temporarily invaded.'' Sax, Takings and the Police Power, 74 Yale L.J. 36, 37
1964). The Court's application of its formula to the facts of this case vividly
illustrates that its approach is potentially dangerous as well as
misguided.\ldots

Before examining the Court's new takings rule, it is worth reviewing what was
``taken'' in this case. At issue are about 36 feet of cable one-half inch in
diameter and two 4$''$ x 4$''$ x 4$''$ metal boxes. Jointly, the cable
and boxes occupy only about one-eighth of a cubic foot of space on the roof of
appellant's Manhattan apartment building. When appellant purchased that building
in 1971, the ``physical invasion'' she now challenges had already
occurred.\ldots

The Court argues that a \textit{per se} rule based on ``permanent physical
occupation'' is both historically rooted, and jurisprudentially sound. I
disagree in both respects. The 19th-century precedents relied on by the Court
lack any vitality outside the agrarian context in which they were decided. But
if, by chance, they have any lingering vitality, then, in my view, those cases
stand for a constitutional rule that is uniquely unsuited to the modern urban
age. Furthermore, I find logically untenable the Court's assertion that \S~828
must be analyzed under a \textit{per se} rule because it ``effectively
destroys'' three of ``the most treasured strands in an owner's bundle of
property rights.''

The Court's recent Takings Clause decisions teach that \textit{nonphysical}
government intrusions on private property, such as zoning ordinances and other
land-use restrictions, have become the rule rather than the exception. Modern
government regulation exudes intangible ``externalities'' that may diminish the
value of private property far more than minor physical touchings.\ldots 

Precisely because the extent to which the government may injure private
interests now depends so little on whether or not it has authorized a ``physical
contact,'' the Court has avoided \textit{per se} takings rules resting on
outmoded distinctions between physical and nonphysical intrusions. As one
commentator has observed, a takings rule based on such a distinction is
inherently suspect because ``its capacity to distinguish, even crudely, between
significant and insignificant losses is too puny to be taken seriously.''
Michelman, \emph{Property, Utility, and Fairness: Comments on the Ethical
Foundations of ``Just Compensation'' Law}, 80 Harv. L. Rev. 1165, 1227 (1967).

Surprisingly, the Court draws an even finer distinction today---between
``temporary physical invasions'' and ``permanent physical occupations.'' When
the government authorizes the latter type of intrusion, the Court would find ``a
taking without regard to the public interests'' the regulation may serve. Yet an
examination of each of the three words in the Court's ``permanent physical
occupation'' formula illustrates that the newly-created distinction is even less
substantial than the distinction between physical and nonphysical intrusions
that the Court already has rejected.

First, what does the Court mean by ``permanent''? Since all ``temporary
limitations on the right to exclude'' remain ``subject to a more complex
balancing process to determine whether they are a taking,'' the Court presumably
describes a government intrusion that lasts forever. But as the Court itself
concedes, \S~828 does not require appellant to permit the cable installation
forever, but only ``[s]o long as the property remains residential and a CATV
company wishes to retain the installation.'' This is far from ``permanent.''

The Court reaffirms that ``States have broad power to regulate housing
conditions in general and the landlord-tenant relationship in particular without
paying compensation for all economic injuries that such regulation entails.''
Thus, \S~828 merely defines one of the many statutory responsibilities that a
New Yorker accepts when she enters the rental business. If appellant occupies
her own building, or converts it into a commercial property, she becomes
perfectly free to exclude Teleprompter from her one-eighth cubic foot of roof
space. But once appellant chooses to use her property for rental purposes, she
must comply with all reasonable government statutes regulating the
landlord-tenant relationship. If \S~828 authorizes a ``permanent'' occupation,
and thus works a taking ``without regard to the public interests that it may
serve,'' then all other New York statutes that require a landlord to make
physical attachments to his rental property also must constitute takings, even
if they serve indisputably valid public interests in tenant protection and
safety.

The Court denies that its theory invalidates these statutes, because they ``do
not require the landlord to suffer the physical occupation of a portion of his
building by a third party.'' But surely this factor cannot be determinative,
since the Court simultaneously recognizes that temporary invasions by third
parties are not subject to a \textit{per se} rule. Nor can the qualitative
difference arise from the incidental fact that, under \S~828, Teleprompter,
rather than appellant or her tenants, owns the cable installation. If anything,
\S~828 leaves appellant better off than do other housing statutes, since it
ensures that her property will not be damaged esthetically or physically,
without burdening her with the cost of buying or maintaining the cable.

In any event, under the Court's test, the ``third party'' problem would remain
even if appellant herself owned the cable. So long as Teleprompter continuously
passed its electronic signal through the cable, a litigant could argue that the
second element of the Court's formula---a ``physical touching'' by a
stranger---was satisfied and that \S~828 therefore worked a taking. Literally
read, the Court's test opens the door to endless metaphysical struggles over
whether or not an individual's property has been ``physically'' touched.\ldots

Third, the Court's talismanic distinction between a continuous ``occupation''
and a transient ``invasion'' finds no basis in either economic logic or Takings
Clause precedent. In the landlord-tenant context, the Court has upheld against
takings challenges rent control statutes permitting ``temporary'' physical
invasions of considerable economic magnitude. Moreover, precedents record
numerous other ``temporary'' officially authorized invasions by third parties
that have intruded into an owner's enjoyment of property far more deeply than
did Teleprompter's long-unnoticed cable. While, under the Court's balancing
test, some of these ``temporary invasions'' have been found to be takings, the
Court has subjected none of them to the inflexible \textit{per se} rule now
adapted to analyze the far less obtrusive ``occupation'' at issue in the present
case. 

In sum, history teaches that takings claims are properly evaluated under a
multifactor balancing test. By directing that all ``permanent physical
occupations'' automatically are compensable, ``without regard to whether the
action achieves an important public benefit or has only minimal economic impact
on the owner,'' the Court does not further equity so much as it encourages
litigants to manipulate their factual allegations to gain the benefit of its
\textit{per se} rule. I do not relish the prospect of distinguishing the
inevitable flow of certiorari petitions attempting to shoehorn insubstantial
takings claims into today's ``set formula.''

Setting aside history, the Court also states that the permanent physical
occupation authorized by \S~828 is a \textit{per se} taking because it uniquely
impairs appellant's powers to dispose of, use, and exclude others from, her
property. In fact, the Court's discussion nowhere demonstrates how \S~828
impairs these private rights in a manner \textit{qualitatively} different from
other garden-variety landlord-tenant legislation.

The Court first contends that the statute impairs appellant's legal right to
dispose of cable-occupied space by transfer and sale. But that claim dissolves
after a moment's reflection. If someone buys appellant's apartment building, but
does not use it for rental purposes, that person can have the cable removed, and
use the space as he wishes. In such a case, appellant's right to dispose of the
space is worth just as much as if \S~828 did not exist.

Even if another landlord buys appellant's building for rental purposes, \S~828
does not render the cable-occupied space valueless. As a practical matter, the
regulation ensures that tenants living in the building will have access to cable
television for as long as that building is used for rental purposes, and thereby
likely increases both the building's resale value and its attractiveness on the
rental market.

In any event, \S~828 differs little from the numerous other New York statutory
provisions that require landlords to install physical facilities ``permanently
occupying'' common spaces in or on their buildings. As the Court acknowledges,
the States traditionally---and constitutionally---have exercised their police
power ``to require landlords to\ldots provide utility connections, mailboxes,
smoke detectors, fire extinguishers, and the like in the common area of a
building.'' Like \S~828, these provisions merely ensure tenants access to
services the legislature deems important, such as water, electricity, natural
light, telephones, intercommunication systems, and mail service. A landlord's
dispositional rights are affected no more adversely when he sells a building to
another landlord subject to \S~828, than when he sells that building subject
only to these other New York statutory provisions.

The Court also suggests that \S~828 unconstitutionally alters appellant's right
to control the \textit{use} of her one-eighth cubic foot of roof space. But
other New York multiple dwelling statutes not only oblige landlords to surrender
significantly larger portions of common space for their tenants' use, but also
compel the \textit{landlord}---rather than the tenants or the private
installers---to pay for and to maintain the equipment. For example, New York
landlords are required by law to provide and pay for mailboxes that occupy more
than five times the volume that Teleprompter's cable occupies on appellant's
building. If the State constitutionally can insist that appellant make this
sacrifice so that her tenants may receive mail, it is hard to understand why the
State may not require her to surrender less space, \textit{filled at another's
expense}, so that those same tenants can receive television signals.

For constitutional purposes, the relevant question cannot be solely
\textit{whether} the State has interfered in some minimal way with an owner's
use of space on her building. Any intelligible takings inquiry must also ask
whether the \textit{extent} of the State's interference is so severe as to
constitute a compensable taking in light of the owner's alternative uses for the
property. Appellant freely admitted that she would have had no other use for the
cable-occupied space, were Teleprompter's equipment not on her building. 

The Court's third and final argument is that \S~828 has deprived appellant of
her ``power to exclude the occupier from possession and use of the space''
occupied by the cable. This argument has two flaws. First, it unjustifiably
assumes that appellant's tenants have no countervailing property interest in
permitting Teleprompter to use that space. Second, it suggests that the New York
Legislature may not exercise its police power to affect appellant's common-law
right to exclude Teleprompter even from one-eighth cubic foot of roof space. But
this Court long ago recognized that new social circumstances can justify
legislative modification of a property owner's common-law rights, without
compensation, if the legislative action serves sufficiently important public
interests.\ldots

In the end, what troubles me most about today's decision is that it represents
an archaic judicial response to a modern social problem. Cable television is a
new and growing, but somewhat controversial, communications medium. The New York
Legislature not only recognized, but also responded to, this technological
advance by enacting a statute that sought carefully to balance the interests of
all private parties. New York's courts in this litigation, with only one jurist
in dissent, unanimously upheld the constitutionality of that considered
legislative judgment.

This Court now reaches back in time for a \textit{per se} rule that disrupts
that legislative determination. Like Justice Black, I believe that ``the
solution of the problems precipitated by\ldots technological advances and new
ways of living cannot come about through the application of rigid constitutional
restraints formulated and enforced by the courts.'' \textit{United States v.
Causby}, 328 U.S., at 274 (dissenting opinion). I would affirm the judgment and
uphold the reasoning of the New York Court of Appeals.

