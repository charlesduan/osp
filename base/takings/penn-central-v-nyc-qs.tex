\expected{penn-central-v-nyc}
\expected{penn-coal-v-mahon}

\item \textbf{The \textit{Penn Central} test}. The
\textit{Penn Central} factors are generally listed as an inquiry into ``[1] the
regulation's economic effect on the landowner, [2] the extent to which the
regulation interferes with reasonable investment-backed expectations, and [3]
the character of the government action.'' \emph{Palazzolo v. Rhode Island}, 533
U.S. 606, 617 (2001). The first factor concerns diminution of value, an issue
raised by \textit{Pennsylvania Coal}. As you see, the Court resisted the
conceptual severance claim, rejecting the notion that ``air rights'' were
something to be evaluated independently of the property as a whole.

\item \textbf{Distinct Investment-Backed Expectations}. The meaning of the
second factor as something distinct from the first is a matter of debate.
Unhelpfully, the Court later described the question as being one of
``reasonable'' investment-backed expectations in \textit{Kaiser Aetna v. United
States}, 444 U.S. 164, 175 (1979). 


The idea is frequently credited to an article by Frank Michelman, who argued
that the principle more accurately captures what may rise to the level of a
taking than simple diminution of value:
\begin{quotation}
The customary labels---magnitude of the harm test, or diminution of value
test---obscure the test's foundations by conveying the idea that it calls for an
arbitrary pinpointing of a critical proportion (probably lying somewhere between
fifty and one hundred percent). More sympathetically perceived, however, the
test poses not nearly so loose a question of degree; it does not ask ``how
much,'' but rather (like the physical-occupation test) it asks ``whether or
not'': whether or not the measure in question can easily be seen to have
practically deprived the claimant of some distinctly perceived, sharply
crystallized, investment-backed expectation.

The nature and relevance of this inquiry may emerge more clearly if we notice
one other familiar line of doctrine\ldots when a new zoning scheme is
instituted, for ``established'' uses which would be violations were the scheme
applied with full retrospective vigor. The standard practice of granting
dispensations for such ``nonconforming uses'' seems to imply an understanding
that simply to ban them without payment of compensation, thus seriously reducing
the property's market value, would be wrong and perhaps unconstitutional. But a
ban on potential uses not yet established may destroy market value as
effectively as does a ban on activity already in progress. The ban does not shed
its retrospective quality simply because it affects only prospective uses. What
explains, then, the universal understanding that only those nonconforming uses
are protected which were demonstrably afoot by the time the regulation was
adopted? The answer seems to be that actual establishment of the use
demonstrates that the prospect of continuing it is a discrete twig out of his
fee simple bundle to which the owner makes explicit reference in his own
thinking, so that enforcement of the restriction would, as he looks at the
matter, totally defeat a distinctly crystallized expectation.
\end{quotation}
Frank I. Michelman, \textit{Property, Utility, and Fairness: Comments on the
Ethical Foundations of ``Just Compensation'' Law}, 80 \textsc{Harv. L. Rev}.
1165, 1232-34 (1967) (footnotes omitted); \emph{Ruckelshaus v. Monsanto Co.},
467 U.S. 986, 1005-06 (1984) (``A `reasonable investment-backed expectation'
must be more than a ``unilateral expectation or an abstract need.'') (citing
\textit{Webb's Fabulous Pharmacies, Inc. v. Beckwith}, 449 U.S. 155, 161
(1980)). As the excerpted text notes, the principle of nonconforming uses in
zoning law reflects the importance of property owner expectations in uses that
preexist the arrival of new zoning rules.

Michelman's argument, and some precedent, suggests that investment-backed
expectations are less likely to be found where the property in question is
purchased against a backdrop of regulation. Does that mean that takings
challenges are doomed whenever the property is acquired after the offending
regulations are in place? In \textit{Palazzolo v. Rhode Island}, 533 U.S. 606
(2001), the Court held in the negative. Ever straining for eloquence, Justice
Kennedy concluded that ``[t]he State may not put so potent a Hobbesian stick
into the Lockean bundle.\ldots Were we to accept the State's rule, the
postenactment transfer of title would absolve the State of its obligation to
defend any action restricting land use, no matter how extreme or unreasonable. A
State would be allowed, in effect, to put an expiration date on the Takings
Clause. This ought not to be the rule. Future generations, too, have a right to
challenge unreasonable limitations on the use and value of land.'' \textit{Id.}
at 627.

\expected{loretto-v-teleprompter}

\item \textbf{Character of the Governmental Action.} Here, too, the Court is
less than clear, as its example of how this factor might be weighed in the
property owner's favor, a permanent physical invasion, was later held to be a
taking as a categorical matter in
\having{loretto-v-teleprompter}{\textit{Loretto}}{\emph{Loretto v. Teleprompter
Manhattan CATV Corp.}, 458 U.S. 419 (1982)}{\emph{Loretto v. Teleprompter
Manhattan CATV Corp.}, 458 U.S. 419 (1982)}. That sort of invasion is
juxtaposed against an interference ``from some public program adjusting the
benefits and burdens of economic life to promote the common good,'' suggesting
room for judgment when a program falls short (e.g., when someone is unfairly
singled out for the burdens, whether there is a reciprocity of advantage, etc.).
\textit{See, e.g.}, Thomas W. Merrill, \textit{The Character of the Governmental
Action}, 36 \textsc{Vt. L. Rev}. 649, 664 (2012) (``Several lower courts have
picked up on the idea that the character factor is designed to measure the
distributional impact of the challenged governmental action. These courts favor
broad-based laws that offer reciprocity of advantage and find suspect laws that
single out particular owners for severe burdens while conferring benefits on
others.'').

\item \textbf{Takings and Due Process inquiries distinguished.} The question
whether a regulation amounts to a taking is distinct from the issue of whether
it violates a liberty or property interest under the Due Process Clause. The
latter asks whether the government may impose the challenged regulation at all.
The former identifies a subset of cases in which the government regulation is
such an intrusion as to require compensation.


In takings cases, you may encounter citations to \textit{Agins v. City of
Tiburon} for the proposition that ``[t]he application of a general zoning law to
particular property effects a taking if the ordinance does not substantially
advance legitimate state interests.'' 447 U.S. 255, 260 (1980). Does this mean
that compensation must be paid if the state cannot meet a higher burden than the
one required for regulation under the Due Process Clause? No. In \textit{Lingle
v. Chevron U.S.A. Inc}., 544 U.S. 528, 540-42 (2005), the Court observed the
phrase was ``regrettably imprecise'' and clarified that ``it has no proper place
in our takings jurisprudence.''



\item Several articles report that the government generally prevails under the
Penn Central test in the lower courts. F. Patrick Hubbard et al., \emph{Do
Owners Have
A Fair Chance of Prevailing Under the Ad Hoc Regulatory Takings Test of Penn
Central Transportation Company?}, 14 \textsc{Duke Envtl. L. \& Pol'y F}. 121
(2003); Basil H. Mattingly, \emph{Forum Over Substance: The Empty Ritual of
Balancing
in Regulatory Takings Jurisprudence}, 36 \textsc{Willamette L. Rev}. 695 (2000).
One such study argues that calling the factors a balancing test misstates what
is actually going on.
\begin{quote}
The analysis reveals that the Courts of Appeals for the First, Ninth, and
Federal Circuits, and the trial courts within the Ninth Circuit, all decided
\textit{Penn Central} cases utilizing fewer than three factors in a majority of
the cases reaching the merits: on average, the circuit courts of appeals
utilized three factors only slightly more than one-third of the time (37.8\%).
Complementing these findings is data on how often the courts actually applied
\textit{Penn Central} as a balancing test. The data shows that applying
\textit{Penn Central} as a balancing test is statistically rare. Averaging the
cases that reached the merits of a takings claim, the courts applied
\textit{Penn} as a balancing test less than 7\% of the time. As an average
percentage of cases applying all three \textit{Penn Central} factors (cases that
themselves are less than half of all cases reaching the merits), courts applied
it as a balancing test less than 14\% of the time. Together this data indicates
that the predominant practice of the federal courts is not to use \textit{Penn
Central} as a balancing test.
\end{quote}
Adam R. Pomeroy, \textit{Penn Central After 35 Years: A Three Part Balancing
Test or A One Strike Rule?}, 22 \textsc{Fed. Cir. B.J.} 677, 704 (2013).
Pomeroy argues that regulatory takings claims prevail only when the court
concludes that the regulation looks like an act that is normally a taking as a
categorical matter. \textit{Id.} at 696 (``It seems that instead of balancing
factual situations, the courts of appeals have found regulatory takings under
\textit{Penn Central} only when a claim falls barely short being a taking under
one of the categorical rules.''). We have already discussed one such categorical
rule in \textit{Loretto}. We now turn to the second.

\expected{loretto-v-teleprompter}

\expectnext{lucas-v-sccc}


