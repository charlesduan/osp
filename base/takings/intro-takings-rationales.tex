The power to take property is recognized (but not granted) by the Constitution
and long historical practice, but what justifies it? Simply calling it an
attribute of sovereignty does not provide a reason for its use. Property
ownership usually encompasses the right to say no. If I want to ship a mobile
home across your field, but we don't agree on a price, it's my duty to stay out.
I cannot declare your property mine in exchange for a judicially determined
measure of ``just compensation.'' What makes the state different?

One traditional explanation concerns the transaction costs of government
enterprises. In a normal market, buyers can choose from among competing sellers.
If houses in town A are too expensive, you can look for one in town B, and if
you are priced out of the market, so be it. The state is often more constrained.
Imagine a planned road that will connect two cities. Building the road requires
assembling multiple, connected parcels. The number of plausible routes is
finite, and increasingly constrained as plans progress. Owners along the planned
route therefore may hold out for higher sale values, knowing the state has few
alternatives. The absence of a functioning market depletes the social surplus of
the road and may kill the project altogether. Eminent domain enables the
government to engage in projects like these without the risk that a single
property owner might exercise a veto.\footnote{And courts sometimes \textit{do}
require property owners to take the money and bear an intrusion. For example,
private condemnation statutes allow landlocked owners to obtain access to public
roads so long as they pay compensation. Likewise,
\having{boomer-v-atlantic-cement}{recall that \textit{Boomer} required}{we will
see in \emph{Boomer v. Atlantic Cement Co.} that courts can require}{courts can
require} nuisance plaintiffs to accept a de facto servitude on their land upon
payment of permanent damages by the defendant\having{boomer-v-atlantic-cement}{
cement plant.}{.}{. \emph{See} \emph{Boomer v. Atlantic Cement Co.}, 257 N.E.2d
870 (N.Y. 1970).} Both situations may
be described as involving high transaction costs either in the form of bilateral
monopoly or problems of coordinating numerous parties.} Of course private
entities sometimes undertake large projects. Why might they succeed despite
lacking the eminent domain power? For one argument, \textit{see} Daniel B.
Kelly, \textit{The ``Public Use'' Requirement in Eminent Domain Law: A Rationale
Based on Secret Purchases and Private Influence}, 92 \textsc{Cornell L. Rev}. 1,
5 (2006) (``[T]akings for the benefit of private parties are generally
unnecessary---even if a private project potentially also has a public
benefit---because private parties can avoid the holdout problem using secret
buying agents. These undisclosed agents overcome the holdout problem by
purchasing property without revealing the identity of the assembler or the
nature of the assembly project to existing owners.'').

A second question concerns the requirement of compensation. Why do you think it
is required? Fairness? Perhaps, but life is unfair. Moreover, we have insurance
to protect against life's calamities. Why couldn't we insure against government
takings? Might the answer have something to do with the nature of government
action? Unlike forces of nature, it is susceptible to outside influence. Can you
think of other rationales? For a discussion, \textit{see} Steve P. Calandrillo,
\textit{Eminent Domain Economics: Should ``Just Compensation'' Be Abolished, and
Would ``Takings Insurance'' Work Instead?}, 64 \textsc{Ohio St}. L.J. 451
(2003). If the government did not have a duty to compensate, how would its
behavior change?

