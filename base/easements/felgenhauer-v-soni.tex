\reading{Felgenhauer v. Soni}

\readingcite{17 Cal.Rptr.3d 135 (Cal. App. 2004).}

\opinion \textsc{Gilbert}, P.J.

Here we hold that to establish a claim of right to a prescriptive easement, the
claimant need not believe he or she is legally entitled to use of the easement.
Jerry and Kim Felgenhauer brought this action to quiet title to prescriptive
easements over neighboring property owned by Ken and Jennifer Soni. A jury made
special findings that established a prescriptive easement for deliveries. We
affirm.

\readinghead{Facts}

In November of 1971, the Felgenhauers purchased a parcel of property consisting
of the front portion of two contiguous lots on Spring Street in Paso Robles. The
parcel is improved with a restaurant that faces Spring Street. The back portion
of the lots is a parking lot that was owned by a bank. The parking lot is
between a public alley and the back of the Felgenhauers' restaurant.

From the time the Felgenhauers opened their restaurant in 1974, deliveries were
made through the alley by crossing over the parking lot to the restaurant's back
door. The Felgenhauers never asked permission of the bank to have deliveries
made over its parking lot. The Felgenhauers operated the restaurant until the
spring of 1978. Thereafter, until 1982, the Felgenhauers leased their property
to various businesses.

The Felgenhauers reopened their restaurant in June of 1982. Deliveries resumed
over the bank's parking lot to the restaurant's back door. In November of 1984,
the Felgenhauers sold their restaurant business, but not the real property, to
James and Ann Enloe. The Enloes leased the property from the Felgenhauers.
Deliveries continued over the bank's parking lot.

James Enloe testified he did not believe he had the right to use the bank's
property and never claimed the right. Enloe said that during his tenancy, he saw
the bank manager in the parking lot. The manager told him the bank planned to
construct a fence to define the boundary between the bank's property and the
Felgenhauers' property. Enloe asked the manager to put in a gate so that he
could continue to receive deliveries and have access to a trash dumpster. The
manager agreed. Enloe ``guess[ed]'' the fence and gate were constructed about
three years into his term. He said, ``[Three years] could be right, but it's a
guess.'' In argument to the jury, the Sonis' counsel said the fence and gate
were constructed in January of 1988.

The Enloes sold the restaurant to Brett Butterfield in 1993. Butterfield sold it
to William DaCossee in March of 1998. DaCossee was still operating the
restaurant at the time of trial. During all this time, deliveries continued
across the bank's parking lot.

The Sonis purchased the bank property, including the parking lot in dispute in
1998. In 1999, the Sonis told the Felgenhauers' tenant, DaCossee, that they were
planning to cut off access to the restaurant from their parking lot.

The jury found the prescriptive period was from June of 1982 to January of
1988.

\readinghead{Discussion}

\readinghead{I}

The Sonis contend there is no substantial evidence to support a prescriptive
easement for deliveries across their property. They claim the uncontroverted
evidence is that the use of their property was not under ``a claim of
right.''\ldots

At common law, a prescriptive easement was based on the fiction that a person
who openly and continuously used the land of another without the owner's
consent, had a lost grant. California courts have rejected the fiction of the
lost grant. Instead, the courts have adopted language from adverse possession in
stating the elements of a prescriptive easement. The two are like twins, but not
identical. Those elements are open and notorious use that is hostile and
adverse, continuous and uninterrupted for the five-year statutory period under a
claim of right. Unfortunately, the language used to state the elements of a
prescriptive easement or adverse possession invites misinterpretation. This is a
case in point.

The Sonis argue the uncontroverted evidence is that the use of their property
was not under a claim of right. They rely on the testimony of James Enloe that
he never claimed he had a right to use the bank property for any purpose.

Claim of right does not require a belief or claim that the use is legally
justified. It simply means that the property was used without permission of the
owner of the land. As the American Law of Property states in the context of
adverse possession: ``In most of the cases asserting [the requirement of a claim
of right], it means no more than that possession must be hostile, which in turn
means only that the owner has not expressly consented to it by lease or license
or has not been led into acquiescing in it by the denial of adverse claim on the
part of the possessor.'' (3 Casner, American Law of Property (1952) Title by
Adverse Possession, \S~5.4, p. 776.)\dots Enloe testified that he had no
discussion with the bank about deliveries being made over its property. The jury
could reasonably conclude the Enloes used the bank's property without its
permission. Thus they used it under a claim of right.

The Sonis attempt to make much of the fence the bank constructed between the
properties and Enloe's request to put in a gate. But Enloe was uncertain when
the fence and gate were constructed. The Sonis' attorney argued it was
constructed in January of 1988. The jury could reasonably conclude that by then
the prescriptive easement had been established.

The Sonis argue the gate shows the use of their property was not hostile. They
cite \textit{Myran v. Smith} (1931) 117 Cal.App. 355, 362, 4 P.2d 219, for the
proposition that to effect a prescriptive easement the adverse user ``\ldots
must unfurl his flag on the land, and keep it flying, so that the owner may see,
if he will, that an enemy has invaded his domains, and planted the standard of
conquest.''

But \textit{Myran} made the statement in the context of what is necessary to
create a prescriptive easement. Here, as we have said, the jury could reasonably
conclude the prescriptive easement was established prior to the erection of the
fence and gate. The Sonis cite no authority for the proposition that even after
the easement is created, the user must keep the flag of hostility flying. To the
contrary, once the easement is created, the use continues as a matter of legal
right, and it is irrelevant whether the owner of the servient estate purports to
grant permission for its continuance.\dots

