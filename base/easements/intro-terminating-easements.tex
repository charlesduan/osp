Easements can be terminated in a variety of ways.

\paragraph{Unity of ownership} When the dominant and servient estates of an
easement appurtenant unite under one owner, the easement ends. Likewise an
easement in gross ends if the owner acquires an interest in the servient
tenement that would have provided independent authority to exercise the rights
of the easement.


\paragraph{Release by the easement holder} The \textsc{First Restatement}
would require a written instrument under seal for an inter vivos release, while
the modern \textsc{Restatement} simply requires compliance with the Statute of
Frauds.


\paragraph{Abandonment} Abandonment resembles a release. The \textsc{First
Restatement} treats them separately, however, and distinguishes the two by
describing abandonment as intent by the easement holder to give up the easement,
while a release is an act done on behalf of the owner of the burdened property.
Abandonment may be inferred by actions. \textsc{Restatement (First) of Property}
\S~504 (1944).


\paragraph{Estoppel} Estoppel may terminate an easement when (1) the owner of
the servient tenement acts in a manner that is inconsistent with the easement's
continuation; (2) the acts are in foreseeable reasonable reliance on conduct by
the easement holder; and (3) allowing the easement to continue would work an
unreasonable harm to the owner of the servient property. \textit{Id.} \S~505.


\paragraph{Prescription} Just as an easement may be gained by prescription,
so too may it be lost by open and notorious adverse acts by the owner of the
servient tenement that interrupt the exercise of the easement for the
prescription period.


\paragraph{Condemnation} The exercise of the eminent domain power to take
the servient estate creates the possibility of compensation for the easement
owner.


\paragraph{A tax deed} Section 509 of the \textsc{First Restatement}
provides that a tax deed will extinguish an easement in gross, but not an
easement appurtenant. 



\paragraph{Expiration} If the interest was for a particular time.


\paragraph{Recording Acts} Being property interests, easements are subject
to the recording acts, and unrecorded interests may be defeated by transferees
without notice. The modern restatement provides for exceptions for certain
easements not subject to the Statute of Frauds and generally for servitudes that
``would be discovered by reasonable inspection or inquiry.'' \textsc{Restatement
(Third) of Property (Servitudes)} \S~7.14 (2000).
