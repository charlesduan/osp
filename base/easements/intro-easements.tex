Easements are interests in land. Unlike fee simple ownership, they are
nonpossessory. Rather, they allow the easement holder to use or control someone
else's land. Suppose Anna owns Blackacre, and Brad owns Whiteacre, which borders
Blackacre. Anna would like to cross Whiteacre to reach Blackacre. She could ask
Brad for permission to cross, but even if he says yes, permission can be
revoked. Brad might also convey Whiteacre to a less welcoming owner. Anna may
therefore wish to acquire a property interest that gives her an
\textit{irrevocable} right to cross over Whiteacre. If Brad conveys her this
interest (by sale or grant), Anna now owns an \textbf{easement of access}, which
is a right to enter and cross through someone's land on the way to someplace
else.

\paragraph{Terminology}
Easements come in multiple flavors. The first distinction is between affirmative
and negative easements. An \textbf{affirmative easement} lets the owner do
something on (or affecting) the land of another, known as the \textbf{servient
estate} (or \textbf{servient tenement}). The right is the \textbf{benefit} of
the easement, and the obligation on the servient estate is its \textbf{burden}.


As noted above, a common affirmative easement is an \textbf{easement of access}
(also known as an \textbf{easement of way}), which requires the owner of the
servient estate to allow the easement holder to travel on the land to reach
another location. In the example above, Anna has an affirmative easement to
cross Whiteacre, the servient estate, to access Blackacre.\footnote{If the
easement holder is allowed to take something from the land (suppose Anna has the
right to harvest wheat from Whiteacre while in transit to Blackacre), the right
is called a \textbf{profit a prendre} or \textbf{profit}. Profits were
traditionally classified as distinct from easements, though their legal
treatment is typically similar. \textit{See, e.g.}, \emph{Figliuzzi v. Carcajou
Shooting Club of Lake Koshkonong}, 516 N.W.2d 410, 414 (Wis. 1994) (``[W]e can
find no distinction between easements and profits relevant to recording the
property interest[.]''). The \textsc{Restatement} characterizes the profit as a
kind of easement. \S~1.2.} A \textbf{negative easement} prohibits the owner of
the servient estate from engaging in some action on the land. For example, if
Anna has a solar panel on her property, she might acquire a solar easement from
Brad that would prohibit the construction of any structures on Whiteacre that
might block the sun from Anna's panel on Blackacre. 

Another distinction is between \textbf{easements appurtenant} and
\textbf{easements in gross}. An easement appurtenant benefits another piece of
land, the \textbf{dominant estate}. The owner of the dominant estate exercises
the rights of the easement. If ownership of the dominant estate changes, the new
owner exercises the powers of the easement; the prior owner retains no interest.
So if Anna's easement to cross Whiteacre to reach Blackacre is an easement
appurtenant, Blackacre is the dominant estate. If she conveys Blackacre to
Charlie, Charlie becomes the owner of the easement. 

In an easement in gross, the easement benefits a specific person, who exercises
the rights of the easement rights regardless of land ownership. If Anna's
easement to cross Whiteacre to reach Blackacre is an easement in gross, she
keeps her easement even if she conveys Blackacre. In general, the presumption is
in favor of an easement appurtenant over an easement in gross. Why do you think
that is?

Easements are part of the larger law of \textbf{servitudes}, which include real
covenants and equitable servitudes.  A servitude is a legal device that creates
a right or obligation that \textbf{runs with the land}. A right runs with the
land when it is enjoyed not only by its initial owner but also by all successors
to that owner's benefited property interest. A burden runs with the land when it
binds not only its initial obligor but also all successors to that obligor's
burdened property interest. A servitude can be, among other things, an easement,
profit, or covenant. These interests overlap, and the \textsc{Restatement
(Third) of Property (Servitudes)} (2000) seeks to unify them.
\having{covenants}{}{\footnote{A
covenant is a servitude if either its benefit or its burden runs with the land;
otherwise it is merely a contract enforceable only as between the original
contracting parties (or perhaps a gratuitous promise enforceable by nobody at
all). When a covenant is a servitude, it may equivalently be described as either
a ``servitude'' or ``a covenant running with the land.'' We will discuss
covenants in a later chapter.}}{\footnote{A
covenant is a servitude if either its benefit or its burden runs with the land;
otherwise it is merely a contract enforceable only as between the original
contracting parties (or perhaps a gratuitous promise enforceable by nobody at
all). When a covenant is a servitude, it may equivalently be described as either
a ``servitude'' or ``a covenant running with the land.''}} As a matter of
history, however, easement law
developed as a distinct set of doctrines, and this chapter gives them separate
treatment.\footnote{Moreover, the \textsc{Third Restatement} is somewhat
notorious for
the extent to which it seeks not only to ``restate'' the common law, but to push
it in a particular direction. While the \textsc{Third Restatement} does tend to
provide
the modern approach to most servitudes issues, it has a tendency to advocate
against traditional, formalist rules that are often still good law in many
American jurisdictions. We will not thoroughly explore these distinctions here;
you should however be aware of the importance of thoroughly investigating the
applicable law in your jurisdiction if you ever encounter servitudes in
practice.} 


