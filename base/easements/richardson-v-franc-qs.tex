\expected{richardson-v-franc}
\expected{intro-prescriptive-easements}

\item The \textsc{Restatement} characterizes irrevocable license situations as a
servitude created by estoppel. \textsc{Restatement} \S~2.10. Is there any
difference, then, between an irrevocable license and an easement by
prescription? Is there any reason to treat them differently?


\item Is landscaping important enough to justify the intrusion into property
ownership interests? What do you think would have happened had the appellants
won?


\item How well does \textit{Richardson} track your intuitions about everyday
behavior? Would you ask permission before engaging in the landscaping at issue
here? Would you advise a client to? Suppose you asked your neighbor for an
easement of way to enable you to build on an adjoining property? You're friends,
and he says yes. But you know a thing or two about the law, so you know that if
your relations turn sour you would have to rely on an irrevocable license claim.
Would you push for a formal grant in writing? Is that a neighborly thing to do?
For one view, \textit{see} \emph{Shepard v. Purvine}, 248 P.2d 352, 361-62 (Or.
1952) (``Under the circumstances, for plaintiffs to have insisted upon a deed
would have been embarrassing; in effect, it would have been expressing a doubt
as to their friend's integrity.''). Does it make a difference that you know to
ask? What about those without legal training? Should the law accommodate private
ordering or funnel property holders into formal arrangements? Do the interests
of third parties, including possible future purchasers of each of the affected
properties, matter to your analysis?

