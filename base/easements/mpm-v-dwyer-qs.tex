\expected{mpm-v-dwyer}

\item
The ability to adapt long-term easements to new uses often depends on formal
labels. Suppose a railroad acquires the right to conduct rail service over a
stretch of land. Decades pass, and the railroad seeks to abandon the line and
turn the tracks over to a local government that will tear them out and create a
system of nature trails. Can it? If the railroad had acquired a fee simple,
sure. But if it only had an easement of way for railroad operations, the change
would exceed the easement's scope, giving the owner of the underlying land a
claim. \textsc{Restatement (First) of Property} \S~482 (1944) (``The extent of
an easement created by a conveyance is fixed by the conveyance.''). This is so
even if the easement gave the railroad exclusive access to the land in question
while the easement was active. \textit{See, e.g.}, \emph{Preseault v. United
States}, 100 F.3d 1525 (Fed. Cir. 1996). 

