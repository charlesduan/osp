\expected{intro-implied-easements}

\item \textbf{Common Ownership.} Are easements implied by prior existing use
fair to owners of subdivided land? Why shouldn't we require purchasers of
subdivided lots to ``get it in writing''---that is, to bargain for easements to
obvious and necessary amenities when accepting a parcel carved out from a larger
plot of land? For that matter, why don't we require the original owner to
bargain for the right to continue to use land that they are purporting to sell?
Who do we think is in a better position to identify the need for such an
easement, the prior owner of the undivided parcel, or the purchaser of the
carved-out portion of that parcel? Should the answer matter in determining
whether to imply an easement or not?

The common law did draw distinctions between implied \textit{reservation} of an
easement (to the owner of the original undivided lot) and implied \textit{grant}
of an easement (to the first purchaser of the separated parcel). The latter
required a lesser showing of necessity than the former, which would only be
recognized upon a showing of \textit{strict} necessity. The theory was that the
deed that first severed the parcels from one another should be construed against
its grantor, who was in a better position to know of the need for an easement to
property she already owned, and to write such an easement into the deed she was
delivering. Indeed, a minority of jurisdictions still follow this rule.

The modern \textsc{Restatement}, in contrast, makes no distinction as to whether
the
easement is sought by the grantor or the grantee, providing simply that the use
will continue if the parties had reasonable grounds to so expect. Factors
tending to show that expectation are that: ``(1) the prior use was not merely
temporary or casual, and (2) continuance of the prior use was reasonably
necessary to enjoyment of the parcel, estate, or interest previously benefited
by the use, and(3) existence of the prior use was apparent or known to the
parties, or (4) the prior use was for underground utilities serving either
parcel.'' \textsc{Restatement (Third) of Property (Servitudes)} \S~2.12 (2000).
The commentary allows for the possibility that the balance of hardships and
grantor knowledge might justify a court's refusing to imply a servitude in favor
of the grantor when it would have for the grantee. \textit{Id.} cmt. a. But the
general approach is to accept and accommodate the fact that grantors do not
always protect themselves as well as they perhaps should. \textit{Id.}
(``Although grantors might be expected to know that they should expressly
reserve any use rights they intend to retain after severance, experience has
shown that too often they do not.''). 

\item \textbf{Reasonable necessity}. Reasonable necessity is something less than
absolute necessity. \textit{See, e.g.}, \emph{Rinderer v. Keeven}, 412 N.E.2d
1015, 1026 (Ill. App. 1980) (``It is well established that one who claims an
easement by implication need not show absolute necessity in order to prevail; it
is sufficient that such an easement be reasonable, highly convenient and
beneficial to the dominant estate.'' (internal quotation and citation omitted)).
Does this leave courts with too much discretion to impose easements? A minority
of jurisdictions make a formal distinction between implied easements in favor of
grantees and grantors, requiring strict necessity in the case of the latter.
\textsc{Restatement} \S~2.12. \textit{But see} \emph{Tortoise Island
Communities, Inc. v. Moorings Ass'n, Inc.}, 489 So. 2d 22, 22 (Fla. 1986)
(concluding that an absolute necessity is required in all cases). 


\item \textbf{What is apparent?} Should home purchasers be expected to
investigate the state of utility lines upon making a purchase? The
\textsc{Restatement (Third) of Property (Servitudes)} reports that most cases to
consider the question imply the easement when underground utilities are at
issue. \S~2.12 (Reporter's Note) (such easements ``will be implied without
regard to their visibility or the parties' knowledge of their existence if the
utilities serve either parcel''). Are such uses plausibly apparent? Or is this
simply a case of the law implying terms that the parties likely would have
bargained for had they thought to consider the matter?

