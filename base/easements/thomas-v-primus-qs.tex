\expected{thomas-v-primus}

\item As \textit{Thomas} indicates, there are two traditional rationales for
easements by necessity. The first considers it an implied term of a conveyance,
assuming that the parties would not intend for land to be conveyed without a
means for access. The second simply treats the issue as one of public policy
favoring land use. \textit{See} \textsc{Restatement (Third) of Property
(Servitudes)} \S~2.15 cmt. a (2000).


\item \textit{Thomas}'s implication to the contrary aside, the traditional
view is that the necessity giving rise to an easement by necessity must exist at
the time the property is severed. \textsc{Restatement (Third) of Property
(Servitudes)}
\S~2.15 (2000) (``Servitudes by necessity arise only on severance of rights held
in a unity of ownership.''); \emph{Roy v. Euro-Holland Vastgoed, B.V.}, 404 So.
2d 410, 412 (Fla. Dist. Ct. App. 1981) (``[I]n order for the owner of a dominant
tenement to be entitled to a way of necessity over the servient tenement both
properties must at one time have been owned by the same party .\ldots In
addition, the common source of title must have created the situation causing the
dominant tenement to become landlocked. A further requirement is that at the
time the common source of title created the problem the servient tenement must
have had access to a public road.'').


\item Easements by necessity are typically about access, but other kinds of uses
may be necessary to the reasonable enjoyment of property. For example, suppose O
conveys mineral rights to Blackacre to A. A would have both an easement of
access to Blackacre and the right to engage in the mining necessary to reach the
minerals. Likewise, an express easement of way may require rights to maintain
and improve the easement. Access for utilities may also give rise to an easement
by necessity, creating litigation over which utilities are ``necessary'':
\begin{quote}
When questioned by defendants as to why he could not use a cellular phone on his
property, plaintiff testified he ran a home business and a cellular phone was
not adequate to handle his business needs; for example, a computer cannot access
the Internet over a cellular phone. Plaintiff also testified solar power and gas
generators were unable to produce enough electricity to make his home
habitable.
\end{quote}
\emph{Smith v. Heissinger}, 745 N.E.2d 666, 672 (Ill. App. 2001) (affirming
finding of necessity of easement for underground utilities). 


Courts often describe the degree of necessity required to find an easement by
necessity as being ``strict.'' \textit{See, e.g.}, \emph{Ashby v. Maechling},
229 P.3d 1210, 1214 (Mont. 2010) ``Two essential elements of an easement by
necessity are unity of ownership and strict necessity.''). It is certainly
higher than that needed for an easement implied by existing use. That said,
considerable precedent indicates that the necessity need not be absolute.
\textit{See, e.g.}, \emph{Cale v. Wanamaker}, 121 N.J. Super. 142, 148, 296 A.2d
329, 333 (Ch. Div. 1972) (``Although some courts have held that access to a
piece of property by navigable waters negates the `necessity' required for a way
of necessity, the trend since the 1920's has been toward a more liberal attitude
in allowing easements despite access by water, reflecting a recognition that
most people today think in terms of `driving' rather than `rowing' to work or
home.'').



\item Several states provide owners of landlocked property a statutory right to
obtain access through neighboring land by means of a \textbf{private
condemnation} action. Some courts have held that the availability of private
condemnation actions negate the necessity prong of a common law easement by
necessity claim. \textit{See, e.g.}, \emph{Ferguson Ranch, Inc. v. Murray}, 811
P.2d 287, 290 (Wyo. 1991) (``[A] civil action for a common law way of necessity
is not available because of the existence of W.S. 24--9--101.''). Private
condemnation actions may also extend to contexts beyond those covered by the
common law easement by necessity. \textit{See, e.g.}, Cal. Civ. Code \S~1001
(utilities).

