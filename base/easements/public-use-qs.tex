\expected{matthews-v-bay-head}

\item Do the rights covered by the public trust doctrine preexist the state, or
are they pure creatures of law? When may courts change public trust rules? When
they do so, are the rules changing or is the court explaining that the rule
``always'' thus, but is only now being announced? Does anything turn on this
distinction? As we will see, how we define such changes has implications on
whether a property owner may claim that the state is committing a constitutional
violation by ``taking'' land without just compensation. 


\item When a court alters preexisting conceptions of the right to exclude should
anything be due to the property owner? Does your conception of what the public
trust doctrine is help determine your answer to this question?


\item \textbf{Other theories of expanding public access rights.} Courts have
used other doctrines to expand public access to private lands, including
theories of prescriptive easements, ``implied dedication,'' and customary uses.
\textit{See generally} 4-34 \textsc{Powell on Real Property} \S~34.11. As an
example of implied dedication, the California Supreme Court declared:
\begin{quote}
Although ``No Trespassing'' signs may be sufficient when only an occasional
hiker
traverses an isolated property, the same action cannot reasonably be expected to
halt a continuous influx of beach users to an attractive seashore property. If
the fee owner proves that he has made more than minimal and ineffectual efforts
to exclude the public, then the trier of fact must decide whether the owner's
activities have been adequate. If the owner has not attempted to halt public use
in any significant way, however, it will be held as a matter of law that he
intended to dedicate the property or an easement therein to the public, and
evidence that the public used the property for the prescriptive period is
sufficient to establish dedication.
\end{quote}
\emph{Gion v. City of Santa Cruz}, 2 Cal. 3d 29, 41, 465 P.2d 50, 58 (1970). On
custom, \textit{see, e.g.}, \emph{City of Daytona Beach v. Tona-Rama, Inc.}, 294
So. 2d 73, 78 (Fla. 1974) (``The general public may continue to use the dry sand
area for their usual recreational activities, not because the public has any
interest in the land itself, but because of a right gained through custom to use
this particular area of the beach as they have without dispute and without
interruption for many years.'').



\item \textbf{Politics!} Do not overlook the role of the political process in
questions of beach access. Following the \textit{Gion} ruling noted above, the
California legislature added Cal. Civ. Code \S~1009, which opines that
``[o]wners of private real property are confronted with the threat of loss of
rights in their property if they allow or continue to allow members of the
public to use, enjoy or pass over their property for recreational purposes'' and
that the ``stability and marketability of record titles is clouded by such
public use, thereby compelling the owner to exclude the public from his
property.'' It therefore provides that ``no use of such property by the public
after the effective date of this section shall ever ripen to confer upon the
public or any governmental body or unit a vested right to continue to make such
use permanently, in the absence of an express written irrevocable offer of
dedication of such property to such use.'' Does the availability of a
legislative remedy if landowners organize and convince the legislature to act
suffice to address the concerns about cases like \textit{Matthews}? 


\item \textbf{Conflicting uses.} Once the public has the right of access to
private land, what other limits on private ownership follow? \textit{See, e.g.},
\emph{City of Daytona Beach v. Tona-Rama, Inc.}, 294 So. 2d 73, 78 (Fla. 1974)
(private landowner's construction of tower on beach did not interfere with
customary public rights).


\item \textbf{Public Policy.} Are expansions of public access rights by the
courts beneficial? What kinds of incentives do they create? Consider the
following criticism:
\begin{quotation}
Commentators were severe in their criticism of \textit{Gion-Dietz}, noting not
only departure from precedent, the failure to consider total loss to the owner,
and the prohibition of taking property without compensation, but also that the
case created an obvious inequity and would prove counterproductive to the public
policy espoused. [Citations of critical commentary omitted.]

The inequity addressed by commentators appears when weighing penalties against
rewards to landowners having no immediate use for their property so that
permitting public use poses no interference or impairment. Those landowners who
were neighborly and hospitable in permitting public use were penalized by
\textit{Gion-Dietz} by loss of their land, while those excluding the public by
fencing or other means were rewarded by retention of their exclusive use. While
virtue is usually its own reward, the law does not usually penalize the
virtuous. The decision was asserted to be counterproductive because landowners
to avoid prescriptive dedication would now exclude the public from using open
and unimproved property for recreation purposes. Thus the very policy sought to
be furthered would be defeated. (\emph{County of Orange v. Chandler-Sherman
Corp.}
(1976) 54 Cal.App.3d 561, 564, 126 Cal.Rptr. 765, 767, points out that one of
the reactions to \emph{Gion-Dietz} was ``soaring sales of chain link fences.'')
\end{quotation}
\emph{Cnty. of Los Angeles v. Berk}, 26 Cal. 3d 201, 228-31, 605 P.2d 381,
398-401 (1980) (Clark, J., dissenting). But expanding access offers benefits of
its own:
\begin{quote}
The law of beach access in Hawaii has an enormous, incalculable impact on social
life. Though the law limits the property rights of beachfront owners as they are
defined elsewhere, it increases the wealth of every single person in the state
by giving them a right to go to the beach anywhere in the state. Everyone, no
matter how poor, has a backyard on the beach. Individuals and families go the
shore in the morning to swim or surf before work. Families gather to watch the
sun go down in the evening. Even if they only have a small apartment inland,
they have a right to sit outside on the beach wherever they please. It affects
the range of options people have, their daily routine, and the sense of
satisfaction of almost everyone.
\end{quote}
Joseph William Singer, \textit{Property as the Law of Democracy}, 63
\textsc{Duke L.J.} 1287, 1329 (2014).

\item Many European nations recognize (either by tradition or statute) a ``right
to roam'' on private lands (excluding homestead or cultivated areas). Heidi
Gorovitz Robertson, \emph{Public Access to Private Land for Walking:
Environmental and
Individual Responsibility As Rationale for Limiting the Right to Exclude}, 23
\textsc{Geo. Int'l Envtl. L. Rev}. 211 (2011). The right to roam often
encompasses the picking of berries, mushrooms, and the like. Open access used to
be the norm for unenclosed land in the United States until the late 1800s; open
range laws allowed cattle grazing on unimproved lands. Brian Sawers, \emph{The
Right to Exclude from Unimproved Land}, 83 \textsc{Temp. L. Rev}. 665, 674
(2011); \emph{Nashville \& C.R. Co. v. Peacock}, 25 Ala. 229, 232 (1854) (``Our
present
Code contains similar provisions, which show conclusively that the unenclosed
lands of this State are to be treated as common pasture for the cattle and stock
of every citizen.''). Pressure to close the range and forbid the crossing of
uncultivated or unenclosed land came from three sources: farmers, who were
relying less on free range livestock; railroads, who wished to avoid liability
for cattle collisions; and southern planters, who viewed closed range laws as a
mechanism for limiting the independence of newly emancipated African-American
farmers. Sawers, \emph{supra}, at 681-84; R. Ben Brown, \textit{Free Men and
Free Pigs: Closing the Southern Range and the American Property Tradition}, 108
\textsc{Radical Hist. Rev.} 117, 119 (Fall 2010) (``When the most important
political
and economic project of the post-Reconstruction era became recapturing the labor
of African Americans to produce staple crops, restricting African American
access to open range resources became a priority.'').

