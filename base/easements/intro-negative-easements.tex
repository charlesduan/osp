In the United States, most of the work that could have been done by
\useterm{negative easement}s is largely performed by real covenants or equitable
servitudes\having{intro-covenants}{, which we previously covered}{, which we
take up in a future reading}{}. \emph{See} \textsc{Restatement (Third) of
Property (Servitudes)} \S~1.2 (``A `negative' easement, the obligation not to
use land in one's possession in specified ways, has become indistinguishable
from a restrictive covenant, and is treated as such in this Restatement.'').
Nineteenth century English law gave negative easements a narrow domain. They
were available only to prevent the servient estate from restricting light, air,
support, or the flow of water of an artificial stream to the dominant estate.
\textit{Id.} \S~1.2 cmt. h. Such easements were likewise not widely embraced in
the United States, where equitably enforced negative covenants held in gross
were disfavored. 

For the most part, negative easements only arise by agreement or grant. U.S.
courts therefore consistently reject the English ``doctrine of ancient lights,''
which recognizes a right to light from a neighbor's land after the passage of
time under certain circumstances. 4-34 \textsc{Powell on Real Property}
\S~34.11.

The limitations of negative easements complicated efforts to create conservation
and preservation easements. Such easements tend to be held in gross (e.g., by a
conservation organization), and the common law prohibited equitable enforcement
of negative covenants held in gross. The law likewise was skeptical about
expanding the categories for which negative easements were available.
\textsc{Restatement (Third) of Property (Servitudes)} \S~1.6 cmt. a (2000). The problem
was addressed by the Uniform Conservation Easement Act, which has now been
adopted by every state.  4-34A \textsc{Powell on Real Property} \S~34A.01.


