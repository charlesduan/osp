\reading{Lawrence v. Clark County}

\readingcite{254 P.3d 606, 608-09 (Nev. 2011)}

The public trust doctrine is an ancient principle thought to be traceable to
Roman law and the works of Emperor Justinian. \textit{See State v. Sorensen}, 
436 N.W.2d 358, 361 (Iowa 1989). Justinian derived the doctrine from the
principle that the public possesses inviolable rights to certain natural
resources, noting that ``[b]y the law of nature these things are common to
mankind---the air, running water, the sea, and consequently the shores of the
sea.'' The Institutes of Justinian, Lib. II, Tit. I, \S~1 (Thomas Collett
Sandars trans. 5th London ed. 1876). He also stated that ``rivers and ports are
public; hence the right of fishing in a port, or in rivers, is common to all
men.'' \textit{Id.} \S~2.

The doctrine was thereafter adopted by the common law courts of England, which
espoused the similar principle that ``title in the soil of the sea, or of arms
of the sea, below ordinary high-water mark, is in the King'' and that such title
``is held subject to the public right.'' \textit{Shively v. Bowlby},  152 U.S.
1, 13, 14 S.Ct. 548, 38 L.Ed. 331 (1894).{\dots}

Courts in this country have readily embraced the public trust doctrine. In 1821,
in the first notable American case to express public trust principles, the
Supreme Court of New Jersey observed that citizens have a common right to
sovereign-controlled waterways:
\begin{quote}
The sovereign power itself\ldots cannot, consistently with the principles of the
law of nature and the constitution of a well ordered society, make a direct and
absolute grant of the waters of the state, divesting all the citizens of their
common right. It would be a grievance which never could be long borne by a free
people.
\end{quote}
\textit{Arnold v. Mundy},  6 N.J.L. 1, 78 (N.J.1821).

Thereafter, the United States Supreme Court similarly recognized that ``when the
Revolution took place, the people of each state became themselves sovereign; and
in that character hold the absolute right to all their navigable waters and the
soils under them for their own common use.'' \textit{Martin et al. v. Waddell}, 
41 U.S. 367, 410, 16 Pet. 367, 10 L.Ed. 997 (1842).

Fifty years later, in what has become the seminal public trust doctrine case,
the Supreme Court decided \textit{Illinois Central Railroad v. Illinois},  146
U.S. 387 (1892). In \textit{Illinois Central} the Court noted that because the
State of Illinois was admitted to the United States on ``equal footing'' with
the original 13 colonies, it, like the colonies, was granted title to the
navigable waters and the lands underneath them. For Illinois, that meant that
upon its admission, it held title to its portion of the waters of and lands
beneath Lake Michigan. However, the waters and lands underneath Lake Michigan
were not freely alienable by the State of Illinois---its title to those areas
was ``different in character from that which the State holds in lands intended
for sale.'' More specifically, it possessed only ``title held in trust for the
people of the State that they may enjoy the navigation of the waters, carry on
commerce over them, and have liberty of fishing therein freed from the
obstruction or interference of private parties.'' As a result, the Court
concluded that the Illinois Legislature's attempted relinquishment of such trust
property to the Illinois Central Railroad
\begin{quote}
is not consistent with the exercise of that trust which requires the government
of the State to preserve such waters for the use of the public\ldots. The State
can no more abdicate its trust over property in which the whole people are
interested than it can abdicate its police powers in the administration of
government and the preservation of the peace.
\end{quote}

While the Court noted that such lands need not, under all circumstances, be
perpetually held in trust, it recognized that in effecting transfers, the public
interest is always paramount, providing that ``[t]he control of the State for
the purposes of the trust can never be lost, except as to such parcels as are
used in promoting the interests of the public therein, or can be disposed of
without any substantial impairment of the public interest in the lands and
waters remaining.'' \textit{Id.}

