\paragraph{Easements appurtenant}
Transferring easements appurtenant is simple; when the dominant estate is
conveyed, the rights of the easement come along. This is a natural consequence
of the principle that servitudes (such as easements) run with the land. A more
complicated problem concerns the division of the dominant estate into smaller
parcels. The default approach is to allow each parcel to enjoy the benefit of
the easement. \textsc{Restatement (First) of Property} \S~488 (1944) (``Except
as limited
by the terms of its transfer, or by the manner or terms of the creation of the
easement appurtenant, those who succeed to the possession of each of the parts
into which a dominant tenement may be subdivided thereby succeed to the
privileges of use of the servient tenement authorized by the easement.''). Here,
however, foreseeability and the extent of the added burden matters. \textit{See
generally} R. W. Gascoyne, \textit{Right of owners of parcels into which
dominant tenement is or will be divided to use right of way},  10 A.L.R.3d 960
(Originally published in 1966) (collecting cases).

\paragraph{Easements in gross}
The modern view is that easements in gross are transferable, assuming no
contrary intent in their creation (e.g., that the benefit was intended to be
personal to the recipient). \textsc{Restatement (Third) of Property
(Servitudes)} \S~4.6 cmt. (2000) (``Although historically courts have often
stated that benefits in gross are not transferable, American courts have long
carved out an exception for profits and easements in gross that serve commercial
purposes. Under the rule stated in this section, the exception has now become
the rule.''); \textsc{Restatement (First) of Property} \S~489 (1944) (commercial
easements in gross, as distinct from easements for personal satisfaction, are
transferable); \S~491 (noncommercial easements in gross ``determined by the
manner or the terms of their creation''). 

Another issue concerns the divisibility of an easement in gross. Here, too, the
danger is that divisibility may lead to excessive burdens on the servient
estate. Section 493 of the \textsc{First Restatement of Property} provides that
whether
divisibility is permitted depends on the circumstances surrounding the
easement's creation. The facts giving rise to a prescriptive easement, for
example, may give a landowner fair notice that a single trespasser may acquire
an easement, but not that the easement may then be shared by many others once
the prescription period passes. In contrast, an exclusive easement might lead to
a presumption of divisibilty, for ``the fact that [the owner of the servient
tenement] is excluded from making the use authorized by the easement, plus the
fact that apportionability increases the value of the easement to its owner,
tends to the inference in the usual case that the easement was intended in its
creation to be apportionable.'' \textit{Id.} cmt. c. Where the grant is
non-exclusive a clearer indication of intended divisibility may be required.
\textit{Id.} cmt. d. Section 5.9 of the modern \textsc{Restatement} goes further
by
making divisibility the default assumption unless contrary to the parties intent
or where divisibility would place unreasonable burdens on the servient estate.

