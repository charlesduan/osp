\expected{felgenhauer-v-soni}

\item \textbf{Fiction of the lost grant.} \textit{Felgenhauer} refers to the
fiction of the lost grant. The principle traces back to English law. 4-34
\textsc{Powell on Real Property} \S~34.10 (``In early England the enjoyment had
to have been `from time immemorial,' and this date came to be fixed by statute
as the year 1189. Towards the close of the medieval period, this theory was
rephrased and an easement of this type was said to arise from a grant,
presumably made in favor of the claimant before the time of legal memory, but
since lost.''). The usual American approach is to ignore the fiction and simply
apply rules of prescription that largely track those of adverse possession.
\textit{See id.} 


\item How do the elements of a prescriptive easement differ from the elements of
adverse possession? Why do you think they differ in this way? How do the
resulting interests differ?


\item \textbf{Easements acquired by the public.} What happens if city
pedestrians routinely cut across a private parking lot? May an easement by
prescription be claimed by the public at large? Does it matter that the right
asserted is not in the hands of any one person? Here, too, the fiction of the
lost grant may play a role in the willingness of courts to entertain the
possibility.
\begin{quote}
There is a split of authority as to whether a public highway may be created by
prescription. A number of older cases hold that the public cannot acquire a road
by prescription because the doctrine of prescription is based on the theory of a
lost grant, and such a grant cannot be made to a large and indefinite body such
as the public. See II American Law of Property \S~9.50 (J. Casner ed.1952). The
lost grant theory, however, has been discarded. W. Burby, Real Property \S~31,
at 77 (1965). In its place, courts have resorted to the justifications that
underlie statutes of limitations: ``[The] functional utility in helping to cause
prompt termination of controversies before the possible loss of evidence and in
stabilizing long continued property uses.'' 3 R. Powell, supra note 5, {\P} 413,
at 34--103--04; W. Burby, supra, \S~31, at 77; Restatement of Property ch. 38,
Introductory Note, at 2923 (1944). These reasons apply equally to the
acquisition of prescriptive easements by public use. The majority view now is
that a public easement may be acquired by prescription. 2 J. Grimes, Thompson on
Real Property \S~342, at 209 (1980). 
\end{quote}
\emph{Dillingham Commercial Co. v. City of Dillingham}, 705 P.2d 410, 416
(Alaska 1985).

\captionedgraphic{easements-img003}{Image by the Greater Southwestern
Exploration Company, available under a Creative Commons Attribution 2.0 Generic
license, \protect\url{https://www.flickr.com/photos/gsec/10784349106}.}

What then should the owner of a publicly accessible location do? The owners of
Rockefeller Center reportedly block off its streets one day per year in order to
prevent the loss of any rights to exclude. David W. Dunlap, \emph{Closing for a
Spell, Just to Prove It's Ours}, \textsc{N.Y. Times} (Oct. 28, 2011),
\url{http://www.nytimes.com/2011/10/30/nyregion/lever-house-closes-once-a-year-to-maintain-its-ownership-rights.html}
(``But there is another significant hybrid: purely private space to which the
public is customarily welcome, at the owners' implicit discretion. These spaces
include Lever House, Rockefeller Plaza and College Walk at Columbia University,
which close for part of one day every year.''). Another option is to post a sign
granting permission to enter (thus negating any element of adversity). Some
states approve this approach by statute. \textsc{Cal. Civ. Code} \S~1008 (``No
use by any
person or persons, no matter how long continued, of any land, shall ever ripen
into an easement by prescription, if the owner of such property posts at each
entrance to the property or at intervals of not more than 200 feet along the
boundary a sign reading substantially as follows: `Right to pass by permission,
and subject to control, of owner: Section 1008, Civil Code.'\,'').

