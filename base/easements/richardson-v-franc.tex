\reading{Richardson v. Franc}

\readingcite{182 Cal.Rptr.3d 853 (Cal. App. 2015)}

\opinion \textsc{Ruvolo}, P.J.

In order to access their home in Novato, California, James Scott Richardson and
Lisa Donetti (respondents) had to traverse land belonging to their neighbors,
Greg and Terrie Franc (appellants) on a 150-foot long road which was authorized
by an easement for ``access and public utility purposes.'' Over a 20-year
period, both respondents and their predecessors-in-interest maintained
landscaping, irrigation, and lighting appurtenant to both sides of the road
within the easement area without any objection. Six years after purchasing the
property burdened by the easement, appellants demanded that respondents remove
the landscaping, irrigation, and lighting on the ground that respondents' rights
in the easement area were expressly limited to access and utility purposes, and
the landscaping and other improvements exceeded the purpose for which the
easement was granted. Respondents brought this lawsuit seeking, among other
things, to establish their right to an irrevocable license which would grant
them an uninterrupted right to continue to maintain the landscaping and other
improvements.\ldots

\ldots. In 1989, Karen and Tom Poksay began building their home on undeveloped
property at 2513 Laguna Vista Drive in Novato, California. The project included
constructing and landscaping a 150-foot long driveway within the 30-foot wide
easement running down to the site of their new home, which was hidden from the
street. The driveway was constructed pursuant to an easement over 2515 Laguna
Vista Drive, which was then owned by [appellants' predecessors in interest]. The
easement was for access and utility purposes only.

Landscaping along the driveway was important to the Poksays.\ldots They hired a
landscaper, who dug holes for plants and trees. Ms. Poksay then added plants and
trees along both sides of the driveway in the easement area---hawthorn trees,
Australian tea trees, daylilies, Mexican sage, breath of heaven, flowering pear
trees, and evergreen shrubs.

The landscaper installed a drip irrigation system.\ldots Water fixtures were
also installed along the driveway for fire safety. The Poksays also added
electrical lighting along the driveway, later replacing the electrical lighting
with solar lighting.

During the decade that the Poksays resided at the property Ms. Poksay regularly
tended to the landscaped area, including trimming and weeding, ensuring the
irrigation system was working properly, and replacing plants and trees as
necessary. In addition to Ms. Poskay's own labor, the Poksays paid their
landscaper to perform general maintenance\ldots.

Respondents purchased the property in late 2000.\ldots Over the years,
respondents added new plants and trees, including oleanders, an evergreen tree,
another tea tree, Mexican sage, lavender, rosemary, and a potato bush.
Respondent Donetti testified that landscapers came weekly or every other week,
and the landscapers spent 40 to 50 percent of their time in the easement
area.\ldots During her testimony, respondent Donetti explained, ``we've paid a
lot of money to nurture it and grow it. It's beautiful. It has privacy. It's
absolutely tied to our house value. It's our curb appeal.''

Appellants purchased 2515 Laguna Vista Drive in 2004. [Appellant Greg Franc
admitted he knew about the landscaping in the easement area, as well as the
hiring of landscapers.] He even admitted that the trees were ``beautiful and
provide a lot of color and [were] just all-around attractive.'' From 2004 to
August 2010, appellants and respondents lived in relative harmony\ldots. It was
not until late 2010---approximately six years after appellants bought the
property and two decades after the landscaping and other improvements
began---that appellants first raised a concern about the landscaping and other
improvements. Prior to that date, no one had ever objected.

In late September or early October 2010, without any notice, appellant Greg
Franc cut the irrigation and electrical lines on both sides of the driveway. He
cut not only the lines irrigating the landscaping on the easement, but also
those irrigating respondents' own property. The water valve pumps leading to the
irrigation lines were disassembled as well. As part of these proceedings, the
trial court granted respondents' motion for preliminary injunction and the
irrigation system was restored.\ldots Following a bench trial and an on-site
visit to the property, the court\ldots granted respondents' request for an
irrevocable license.\dots

\ldots [A]s appellants acknowledge, the grant of an irrevocable license is
``based in equity.'' After the trial court has exercised its equitable powers,
the appellate court reviews the judgment under the abuse of discretion
standard.\dots

Before we address the specific issues appellants raise on appeal, it is helpful
to review the law governing the grant of an irrevocable license. ``A license
gives authority to a licensee to perform an act or acts on the property of
another pursuant to the express or implied permission of the owner.'' (6 Miller
\& Starr, Cal. Real Estate (3d ed. 2000) Easements, \S~15:2, p. 15--10.) ``A
licensor generally can revoke a license at any time without excuse or without
consideration to the licensee. In addition, a conveyance of the property
burdened with a license revokes the license\ldots.'' (\textit{Id}. at pp
15--10--15--11, fns. omitted.)

However, a license may become irrevocable when a landowner knowingly permits
another to repeatedly perform acts on his or her land, and the licensee, in
reasonable reliance on the continuation of the license, has expended time and a
substantial amount of money on improvements with the licensor's knowledge. Under
such circumstances, it would be inequitable to terminate the license. In that
case, the licensor is said to be estopped from revoking the license, and the
license becomes the equivalent of an easement, commensurate in its extent and
duration with the right to be enjoyed. A trial court's factual finding that a
license is irrevocable is reviewed for substantial evidence.

In the paradigmatic case, a landowner allows his neighbor the right to use some
portion of his property---often a right of way or water from a creek---knowing
that the neighbor needs the right to develop his property. The neighbor then
builds a house, digs an irrigation ditch, paves the right of way, plants an
orchard, or farms the land in reliance on the landowner's acquiescence. Later,
after failing to make a timely objection, the landowner or his successor
suddenly raises legal objections and seeks to revoke the neighbor's permissive
usage.\dots

In the instant case\ldots the statement of decision states: ``Because
[respondents] adduced sufficient evidence at trial concerning their substantial
expenditures in the easement area for landscaping, maintenance, care, and
physical labor, and because sufficient evidence was presented at trial to
support that [respondents'] predecessor-in-interest, Ms. Poksay, also expended
substantial sums in the easement area for landscaping, maintenance, care, and
physical labor, and because, as the evidence and testimony at trial showed, that
no objection was made to any of this by either [appellants] or [appellants']
predecessor-in-interest, Mr. Schaefer, over the course of more than 20 years,
[respondents] have sufficiently met the requirements for an irrevocable parol
license for both [respondents], and [respondents'] successors-in-interest. Both
law and equity dictate this result.''

\ldots [Appellants] contend the trial court erred in finding the evidence
supported the creation of an irrevocable license because respondents' reliance
on continued permission to landscape and make other improvements in the easement
area was not reasonable as a matter of law. Appellants point out the evidence at
trial revealed that throughout the history of the ownership of the property,
there was never an actual request for permission to make and maintain these
improvements and express consent was never given. In essence, appellants contend
that tacit permission by silence is insufficient to create an irrevocable
license and that respondents were required to show an express grant of
permission induced them into undertaking the improvements within the easement
area.

Permission sufficient to establish a license can be express or implied.\ldots A
license may also arise by implication from the acts of the parties, from their
relations, or from custom. When a landowner knowingly permits another to perform
acts on his land, a license may be implied from his failure to object.\ldots

\ldots Here, the undisputed evidence revealed appellants failed to object to the
landscaping and other improvements for \textit{6 years} before appellants first
made their demand that the landscaping and other improvements be removed. Thus,
with full knowledge that the road providing ingress and egress to respondents'
property was landscaped, irrigated, and lit, and with full knowledge that
respondents were maintaining these improvements on an ongoing basis, appellants
said nothing to respondents. When coupled with the previous 14 years appellants'
predecessors-in-interest acquiesced in these improvements, this constituted a
total of \textit{20 years} of uninterrupted permissive use of the easement area
for the landscaping and other improvements. Therefore, we find the court had
ample evidence to conclude that adequate and sufficient permission was granted
to respondents by appellants to maintain the extensive landscaping improvements
on either side of the roadway.

Appellants next stress that for the license to be irrevocable, there must be
substantial expenditures in reliance on the license. In this regard, the trial
court made the necessary findings that respondents ``have expended substantial
monetary sums to improve, maintain, landscape, and care for the easement area,
including the retention of professional landscapers on a regular basis\ldots.'' 

Appellants next challenge ``the unlimited physical scope and duration of the
license'' granted by the trial court. They claim ``the trial court, in
derogation of equity and the law, decided that [r]espondents\ldots should have
sole and absolute discretion to decide what will happen on property that is
owned by [appellants].'' In making this argument, appellants ignore the fact
that the trial court was vested with broad discretion in framing an equitable
result under the facts of this case.\ldots As it was empowered to do, the trial
court exercised its broad equitable discretion and fashioned relief to fit the
specific facts of this case. The court found ``by a preponderance of the
evidence that [respondents] hold an irrevocable parol license for themselves and
their successors-in-interest to maintain and improve landscaping, irrigation,
and lighting within the 30$'$ wide and 150$'$ long easement.''

Appellants assert ``it is wholly erroneous and grossly unfair to make the
license \textit{irrevocable in perpetuity.}'' (Original italics.) Appellants
argue that a proper ruling in this case would be to grant respondents an
irrevocable license but ``with the license to landscape and garden limited in
duration until [respondents] transfer title to anyone else or no longer reside
on the property\ldots.''

The principles relating to the duration of an irrevocable license were stated by
our Supreme Court over a century ago, and these principles are still valid
today. An otherwise revocable license becomes irrevocable when the licensee,
acting in reasonable reliance either on the licensor's representations or on the
terms of the license, makes substantial expenditures of money or labor in the
execution of the license; and the license will continue ``for so long a time as
the nature of it calls for.'' As explained in a leading treatise, ``A license
remains irrevocable for a period sufficient to enable the licensee to capitalize
on his or her investment. He can continue to use it only as long as justice and
equity require its use.'' (6 Miller \& Starr, \textit{supra},  \S~15:2, p.
15--15.)

The evidence adduced at trial indicates respondents and their predecessors in
interest expended significant money and labor when they planted and nurtured the
landscaping abutting the roadway, installed sophisticated irrigation equipment
throughout the easement area, and constructed lighting along the roadway. Under
such circumstances the trial court did not abuse its discretion in concluding it
would be inequitable to require respondents to remove these improvements when
the property is transferred, given the substantial investment in time and money
and the permanent nature of these improvements.\dots

Lastly, we reject appellants' hyperbolic claim that in fashioning the scope and
duration of the irrevocable license granted in this case, ``the trial court,
without exercising caution, took property that rightfully belonged to
[appellants] and ceded it to [r]espondents---and their successors---forever.''

This argument ignores that a license does not create or convey any interest in
the real property; it merely makes lawful an act that otherwise would constitute
a trespass.\ldots Far from granting respondents ``an exclusive easement
amounting to fee title'' as appellants' claim, the court's decision simply
maintains the status quo that has existed for over 20 years and was obvious to
appellants when they purchased the property a decade ago. 

