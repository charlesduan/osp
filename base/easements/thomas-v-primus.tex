\reading{Thomas v. Primus}

\readingcite{84 A.3d 916 (Conn. App. 2014)}

\opinion \textsc{Mihalakos}, J.

The plaintiffs, William Thomas, Craig B. Thomas and Andrea Thomas Jabs, appeal
from the trial court's declaratory judgment granting an easement by necessity
and implication in favor of the defendant, Bruno Primus. On appeal, the
plaintiffs claim that the court erred in finding an easement by
necessity.\readingfootnote{1}{The plaintiffs also claim that the court erred in
finding an easement by implication. Because we conclude that the court properly
found an easement by necessity, we need not consider this claim.} The plaintiffs
also claim that the defendant's claim for an easement should have been barred by
the defense of laches. We affirm the judgment of the trial court.

The following facts, as found by the court, are relevant to this appeal. The
plaintiffs own property located at 460 Camp Street in Plainville. The defendant
owns one and one-quarter acres of undeveloped land abutting the eastern boundary
of the plaintiffs' property. The dispute at issue here concerns the northernmost
portion of the plaintiffs' property, a twenty-five feet wide by three hundred
feet long strip of land known as the ``passway,'' which stretches from the
public road on the western boundary of the plaintiffs' property to the
defendant's property to the east.

Both the plaintiffs' and the defendant's properties originally were part of a
single lot owned by Martha Thomas, the grandmother of the plaintiffs. In 1959,
Martha Thomas conveyed the one and one-quarter acres of landlocked property,
currently owned by the defendant, to Arthur Primus, the defendant's brother. At
the conveyance, which the defendant attended, Martha Thomas and Arthur Primus
agreed that access to the landlocked property would be through the passway,
which until that time had been used by Martha Thomas to access the eastern
portions of her property. In 1969, the defendant took possession of the land. In
2002, the plaintiffs took possession of the western portion of Martha Thomas'
property, including the passway.

In 2008, the plaintiffs decided to sell their property. When the defendant
learned of their intention, he sent a letter to the plaintiffs asserting his
right to use the passway to access his land. In 2009, the plaintiffs signed a
contract to sell their property, but the prospective purchasers cancelled the
contract when they learned of the defendant's claimed right to use the passway.
The plaintiffs then brought the action to quiet title that is the subject of
this appeal, seeking, among other things, a declaratory judgment that the
defendant had no legal interest in the property. The defendant brought a
counterclaim asking the court to establish his right to use the passway
uninterrupted by the plaintiffs.\ldots In response to the defendant's
counterclaim, the plaintiffs asserted the special defense of laches.

A trial was held on June 5 and 6, 2012. On August 31, 2012, the court issued its
decision, finding in favor of the defendant on the plaintiffs' complaint and on
his counterclaim, and concluding that the defendant had an easement by necessity
and an easement by implication over the passway. Specifically, the court found
an easement by necessity was created when Martha Thomas conveyed a landlocked
parcel to Arthur Primus, as it was absolutely necessary in order to access the
property.\ldots

\readinghead{I}

On appeal, the plaintiffs claim that the court erred in finding an easement by
necessity because (1) the defendant's predecessor in title had the right to buy
reasonable alternative access to the street, (2) the defendant failed to present
full title searches of all adjoining properties, and (3) Martha Thomas and
Arthur Primus did not intend for an easement to exist.\ldots

Originating in the common law, easements by necessity are premised on the
conception that ``the law will not presume, that it was the intention of the
parties, that one should convey land to the other, in such manner that the
grantee could derive no benefit from the conveyance\ldots.'' \textit{Collins v.
Prentice},  15 Conn. 39, 44 (1842). An easement by necessity is ``imposed where
a conveyance by the grantor leaves the grantee with a parcel inaccessible save
over the lands of the grantor\ldots.'' \textit{Hollywyle Assn., Inc. v.
Hollister},  164 Conn. 389, 398, 324 A.2d 247 (1973). The party seeking an
easement by necessity has the burden of showing that the easement is reasonably
necessary for the use and enjoyment of the party's property.

\readinghead{A}

First, the plaintiffs claim that an easement by necessity does not exist because
the defendant's predecessor in title had the right to buy reasonable alternative
access to the street. We disagree.

In considering whether an easement by necessity exists, ``the law may be
satisfied with less than the absolute need of the party claiming the right of
way. The necessity need only be a reasonable one.'' \textit{Hollywyle Assn.,
Inc. v. Hollister},  supra, 164 Conn. at 399, 324 A.2d 247.

In this case, the plaintiffs presented evidence at trial that, at the time he
purchased the property from Martha Thomas in 1959, Arthur Primus maintained
bonds for deed that allowed him to purchase access to Camp Street through a
different piece of property for \$900. Although he did not exercise this right,
the plaintiffs contend that the fact that Arthur Primus held this option
establishes that the defendant's use of the passway is not reasonably
necessary.

The plaintiffs correctly note that the ability of a party to create alternative
access through his or her own property at a reasonable cost can preclude the
finding of reasonable necessity required to establish an easement by necessity.
Nonetheless, we are aware of nothing in our case law that suggests that a party
is required to purchase \textit{additional} property in order to create
alternative access, even at a reasonable price.\readingfootnote{2}{The
plaintiffs' sole authority in support of their position; \emph{Griffeth v. Eid},
573 N.W.2d 829 (N.D.1998); is distinguishable from the case before us. In that
case, the North Dakota Supreme Court upheld a trial court's ruling that a party
seeking an easement by necessity had not met his burden of establishing
reasonable necessity because potential alternate access existed, including the
possibility of purchasing an easement over another abutting property, and the
party had not provided evidence that he had pursued these options and found them
unavailing. In this case, there is no evidence in the record that the defendant
had the opportunity to purchase alternate access.}

Furthermore, easements by necessity need not be created at the time of
conveyance. See \textit{D'Addario v. Truskoski},  57 Conn.App. 236, 247, 749
A.2d 38 (2000) (recognizing easement by necessity created by state taking and
natural disaster). Even if we were to assume, arguendo, that Arthur Primus'
bonds for deed made use of the passway unnecessary at the time he owned the
property, those bonds for deed expired in 1962, several years before the
defendant owned the property, and provide no reasonable alternative access
today. Thus, we see no reason to disturb the court's finding that use of the
passway is currently necessary for the use and enjoyment of the defendant's
property.\ldots

\readinghead{C}

Finally, the plaintiffs argue that an easement by necessity does not exist
because Martha Thomas and Arthur Primus did not intend for the easement to
exist. We disagree.

The seminal case in this state on easements by necessity recognized that ``the
law will not presume, that it was the intention of the parties, that one should
convey land to the other, in such manner that the grantee could derive no
benefit from the conveyance\ldots. The law, under such circumstances, will give
effect to the grant according to the presumed intent of the parties.''
\textit{Collins v. Prentice},  supra, 15 Conn. at 44, 15 Conn. 39. This
rationale does not, as the plaintiffs suggest, establish intent as an element of
an easement by necessity. Instead, ``[t]he presumption as to the intent of the
parties is a fiction of law\ldots and merely disguises the public policy that no
land should be left inaccessible or incapable of being put to profitable use.''
(Citation omitted.) \textit{Hollywyle Assn., Inc. v. Hollister},  supra, 164
Conn. at 400, 324 A.2d 247. Thus, absent an explicit agreement by the grantor
and grantee that an easement does \textit{not} exist, a court need not consider
intent in establishing an easement by necessity. See \textit{O'Brien v. Coburn},
 46 Conn.App. 620, 633, 700 A.2d 81 (holding that ``the intention of the parties
[was] irrelevant'' in case establishing easement by necessity), cert. denied,
243 Conn. 938, 702 A.2d 644 (1997).

In this case, the court found that the defendant's property was landlocked and
that access over the pass-way was reasonably necessary for the use and enjoyment
of the defendant's property. Therefore, the court found an easement by necessity
to exist over the pass-way. This conclusion was supported by the record and
there is no legal deficiency in the court's analysis.\ldots


