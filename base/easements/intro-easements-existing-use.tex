An \term{easement implied by existing use} may arise when a parcel of land is
divided
and amenities once enjoyed by the whole parcel are now split up, such that in
order to enjoy the amenity (a utility line, or a driveway, for example), one of
the divided lots requires access to the other. Imagine, for example, a home
connected to a city sewer line via a privately owned drainpipe, on a parcel that
is later divided by carving out a portion of the lot between the original house
and the sewer line connection, as shown in
Figure~\ref{f:easements-existing-use}.

\begin{figure}
\begin{center}
\begin{tikzpicture}
\draw (0, 0) rectangle (6, 3);
\draw[dashed] (4, 0) -- (4, 3);
\node[
draw, single arrow, shape border rotate=90,
single arrow tip angle=120,
single arrow head extend=1pt,
minimum width=2cm,
minimum height=1.5cm,
] (house) at (2, 1.5) {};
\node[
draw, cylinder, shape border rotate=90,
minimum width=0.5cm, minimum height=3cm,
aspect=0.6, anchor=west,
] (pipe) at (7, 1.5) {};
\draw[double, double distance=6pt, rounded corners] (house.south) |- (7, 0.5);
\end{tikzpicture}
\end{center}
\caption{A divided parcel of land with a sewer line running through both lots.}
\label{f:easements-existing-use}
\end{figure}


In such a situation, courts will frequently find an easement implied by prior
existing use, allowing the owner of the house to continue using the drainpipe
even though it is now under someone else's land. \textit{See, e.g.},  \emph{Van
Sandt v. Royster}, 83 P.2d 698 (Kan. 1938). There are, however, some limits to
the circumstances that will justify the implication of such an easement:
\begin{quote}
[T]he easement implied from a preexisting use, [is] also characterized as a
quasi-easement. Such an easement arises where, during the unity of title, an
apparently permanent and obvious servitude is imposed on one part of  an estate
in favor of another part. The servitude must be in use at the time of severance
and necessary for the reasonable enjoyment of the severed part. A grant of a
right to continue such use arises by implication of law. An implied easement
from a preexisting use is established by proof of three elements: (1) common
ownership of the claimed dominant and servient parcels and a subsequent
conveyance or transfer separating that ownership; (2) before severance, the
common owner used part of the united parcel for the benefit of another part, and
this use was apparent and obvious, continuous, and permanent; and (3) the
claimed easement is necessary and beneficial to the enjoyment of the parcel
conveyed or retained by the grantor or transferrer. 
\end{quote}
\emph{Dudley v. Neteler}, 924 N.E.2d 1023, 1027-28 (Ill. App. 2009) (internal
citations and quotations omitted). The following notes consider each of these
elements.


