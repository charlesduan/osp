\reading{M.P.M. Builders, LLC v. Dwyer}

\readingcite{809 N.E.2d 1053 (Mass. 2004)}

\opinion \textsc{Cowin}, J.

We are asked to decide whether the owner of a servient estate may change the
location of an easement without the consent of the easement holder. We conclude
that, subject to certain limitations, described below, the servient estate owner
may do so.

1. \textit{Facts.} The essential facts are not in dispute. The defendant, Leslie
Dwyer, owns a parcel of land in Raynham abutting property owned by the
plaintiff, M.P.M. Builders, L.L.C. (M.P.M.). Dwyer purchased his parcel in 1941,
and, in the deed, he was also conveyed an easement, a ``right of way along the
cartway to Pine Street,'' across M.P.M.'s land. The cartway branches so that it
provides Dwyer access to his property at three separate points. The deed
describes the location of the easement and contains no language concerning its
relocation.

In July, 2002, M.P.M. received municipal approval for a plan to subdivide and
develop its property into seven house lots. Because Dwyer's easement cuts across
and interferes with construction on three of M.P.M.'s planned lots, M.P.M.
offered to construct two new access easements to Dwyer's property. The proposed
easements would continue to provide unrestricted access from the public street
(Pine Street) to Dwyer's parcel in the same general areas as the existing
cartway. The relocation of the easement would allow unimpeded construction by
M.P.M. on its three house lots. M.P.M. has agreed to clear and construct the new
access ways, at its own expense, so ``that they are as convenient [for the
defendant] as the existing cartway[ ].'' Dwyer objected to the proposed easement
relocation, ``preferring to maintain [his] right of way in the same place that
it has been and has been used by [him] for the past 62 years.''

2. \textit{Procedural history.} M.P.M. sought a declaration, pursuant to G.L. c.
231A, that it has a right unilaterally to relocate Dwyer's easement. When M.P.M.
moved for summary judgment, a Land Court judge found that there were no material
issues of fact in dispute, denied M.P.M.'s motion for summary judgment, entered
summary judgment against M.P.M., and dismissed the case.

The judge recognized that this case was ``a clear example of an increasingly
common situation where a dominant tenant is able to block development on the
servient land because of the common-law rule which\ldots may well be the result
of unreflective repetition of a misapplied rationale.'' He noted that the rule
conflicts with the ``right of a servient tenant to use his land in any lawful
manner that does not interfere with the purpose of the easement.'' Nevertheless,
he concluded that under the ``settled'' common law, once the location of an
easement has been fixed it cannot be changed except by agreement of the estate
owners. The judge concluded that, unless this court decides ``to dispel the
uncertainty that now exists and adapt the common law to present-day
circumstances,'' he was bound to apply the law currently in effect. We granted
M.P.M.'s application for direct appellate review to decide whether our law
should permit the owner of a servient estate to change the location of an
easement without the easement holder's consent.

3. \textit{Discussion.}\dots 

The parties disagree whether our common law permits the servient estate owner to
relocate an easement without the easement holder's consent. Dwyer, citing
language in our cases, contends that, once the location of an easement has been
defined, it cannot be changed except by agreement of the parties.\ldots M.P.M.
claims that our common law permits the servient estate owner to relocate an
easement as long as such relocation would not materially increase the cost of,
or inconvenience to, the easement holder's use of the easement for its intended
purpose. M.P.M. urges us to clarify the law by expressly adopting the modern
rule proposed by the American Law Institute in the Restatement (Third) of
Property (Servitudes) \S~4.8(3) (2000).

This section provides that:
\begin{quote}
Unless expressly denied by the terms of an easement, as defined in \S~1.2,
the owner of the servient estate is entitled to make reasonable changes in the
location or dimensions of an easement, at the servient owner's expense, to
permit normal use or development of the servient estate, but only if the changes
do not (a) significantly lessen the utility of the easement, (b) increase the
burdens on the owner of the easement in its use and enjoyment, or (c) frustrate
the purpose for which the easement was created.
\end{quote}

Section 4.8(3) is a default rule, to apply only in the absence of an express
prohibition against relocation in the instrument creating the easement and only
to changes made by the servient, not the dominant, estate
owner.\readingfootnote{1}{We previously have concluded that the dominant estate
owner, that is, the easement holder, may not unilaterally relocate an easement.
According to the Restatement, many jurisdictions have erroneously expanded that
sensible restriction into one that prevents the owner of the servient estate
from relocating the easement without the consent of the easement holder.
Restatement (Third) of Property (Servitudes) \S~4.8(3) comment f, at 563
(2000).} It ``is designed to permit development of the servient estate to the
extent it can be accomplished without unduly interfering with the legitimate
interests of the easement holder.'' \textit{Id.} at comment f, at 563. Section
4.8(3) maximizes the over-all property utility by increasing the value of the
servient estate without diminishing the value of the dominant estate; minimizes
the cost associated with an easement by reducing the risk that the easement will
prevent future beneficial development of the servient estate; and encourages the
use of easements. Regardless of what heretofore has been the common law, we
conclude that \S~4.8(3) of the Restatement is a sensible development in the law
and now adopt it as the law of the Commonwealth.

We are persuaded that \S~4.8(3) strikes an appropriate balance between the
interests of the respective estate owners by permitting the servient owner to
develop his land without unreasonably interfering with the easement holder's
rights. The rule permits the servient owner to relocate the easement subject to
the stated limitations as a ``fair tradeoff for the vulnerability of the
servient estate to increased use of the easement to accommodate changes in
technology and development of the dominant estate.'' Restatement (Third) of
Property (Servitudes), \textit{supra} at comment f, at 563. Therefore, under
\S~4.8(3), the owner of the servient estate is ``able to make the fullest use of
his or her property allowed by law, subject only to the requirement that he or
she not damage other vested rights holders.'' \textit{Roaring Fork Club, L.P. v.
St. Jude's Co.},  [36 P.3d 1229, ] 1237 [(Colo. 2001)].

It is a long-established rule in the Commonwealth that the owner of real estate
may make any and all beneficial uses of his property consistent with the
easement.\ldots We conclude that \S~4.8(3) is consistent with these principles
in its protection of the interests of the easement holder: a change may not
significantly lessen the utility of the easement, increase the burden on the use
and enjoyment by the owner of the easement, or frustrate the purpose for which
the easement was created. The servient owner must bear the entire expense of the
changes in the easement. 

Dwyer urges us to reject the Restatement approach. He argues that adoption of
\S~4.8(3) will devalue easements, create uncertainty in property interests, and
lead to an increase in litigation over property rights. Our adoption of
\S~4.8(3) will neither devalue easements nor place property interests in an
uncertain status. An easement is by definition a limited, nonpossessory interest
in realty.\ldots An easement is created to serve a particular objective, not to
grant the easement holder the power to veto other uses of the servient estate
that do not interfere with that purpose.

The limitations embodied in \S~4.8(3) ensure a relocated easement will continue
to serve the purpose for which it was created. So long as the easement continues
to serve its intended purpose, reasonably altering the location of the easement
does not destroy the value of it. For the same reason, a relocated easement is
not any less certain as a property interest. The only uncertainty generated by
\S~4.8(3) is in the easement's location. A rule that permits the easement holder
to prevent any reasonable changes in the location of an easement would render an
access easement virtually a possessory interest rather than what it is, merely a
right of way. Finally, parties retain the freedom to contract for greater
certainty as to the easement's location by incorporating consent requirements
into their agreement.

``Clearly, the best course is for the [owners] to agree to alterations that
would accommodate both parties' use of their respective properties to the
fullest extent possible.'' \textit{Roaring Fork Club, L.P. v. St. Jude's Co.,
supra} at 1237. In some cases, the parties will be unable to reach a meeting of
the minds on the location of an easement. In the absence of agreement between
the owners of the dominant and servient estates concerning the relocation of an
easement, the servient estate owner should seek a declaration from the court
that the proposed changes meet the criteria in \S~4.8(3). Such an action gives
the servient owner an opportunity to demonstrate that relocation comports with
the Restatement requirements and the dominant owner an opportunity to
demonstrate that the proposed alterations will cause damage. The servient owner
may not resort to self-help remedies, and, as M.P.M. did here, should obtain a
declaratory judgment before making any alterations.

Although Dwyer may be correct that increased litigation could result as a
consequence of adopting \S~4.8(3), we do not reject desirable developments in
the law solely because such developments may result in disputes spurring
litigation. Section 4.8(3) ``imposes upon the easement holder the burden and
risk of bringing suit against an unreasonable relocation,'' but this {}``far
surpasses in utility and fairness the traditional rule that left the servient
land owner remediless against an unreasonable easement holder.'' \textit{Roaring
Fork Club, L.P. v. St. Jude's Co., supra} at 1237, quoting Note, Balancing the
Equities: Is Missouri Adopting a Progressive Rule for Relocation of Easements?,
61 Mo. L.Rev. 1039, 1060 (1996). We trust that, over time, uncertainties will
diminish and litigation will subside as easement holders realize that in some
circumstances unilateral changes to an easement, paid for by the servient estate
owner, will be enforced by courts. Dominant and servient estate owners will have
an incentive to negotiate a result rather than having a court impose one on
them. 

We return to the facts of this case. The Land Court judge ruled correctly under
existing law. But we conclude that \S~4.8(3) of the Restatement best complies
with present-day realities. The deed creating Dwyer's easement does not
expressly prohibit relocation. Therefore, M.P.M. may relocate the easement at
its own expense if the proposed change in location does not significantly lessen
the utility of the easement, increase the burdens on Dwyer's use and enjoyment
of the easement, or frustrate the purpose for which the easement was created.
M.P.M. shall pay for all the costs of relocating the easement.

Because we cannot determine from the present record whether the proposed
relocation of the easement meets the aforementioned criteria, we vacate the
judgment and remand the case to the Land Court for further proceedings
consistent with this opinion.

