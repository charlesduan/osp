\reading{Brown v. Voss}

\readingcite{715 P.2d 514 (Wash. 1986)}

\opinion \textsc{Brachtenbach}, Justice.

The question posed is to what extent, if any, the holder of a private road
easement can traverse the servient estate to reach not only the original
dominant estate, but a subsequently acquired parcel when those two combined
parcels are used in such a way that there is no increase in the burden on the
servient estate. The trial court denied the injunction sought by the owners of
the servient estate. The Court of Appeals reversed. We reverse the Court of
Appeals and reinstate the judgment of the trial court.

A portion of an exhibit [in Figure~\ref{f:easements-img004}] depicts the
involved parcels.

\captionedgraphic{easements-img004}{Exhibit from \emph{Brown v. Voss}.}

In 1952 the predecessors in title of parcel A granted to the predecessor owners
of parcel B a private road easement across parcel A for ``ingress to and egress
from'' parcel B. Defendants acquired parcel A in 1973. Plaintiffs bought parcel
B on April 1, 1977 and parcel C on July 31, 1977, but from two different owners.
Apparently the previous owners of parcel C were not parties to the easement
grant.

When plaintiffs acquired parcel B a single family dwelling was situated thereon.
They intended to remove that residence and replace it with a single family
dwelling which would straddle the boundary line common to parcels B and C.

Plaintiffs began clearing both parcels B and C and moving fill materials in
November 1977. Defendants first sought to bar plaintiff's use of the easement in
April 1979 by which time plaintiffs had spent more than \$11,000 in developing
their property for building.

Defendants placed logs, a concrete sump and a chain link fence within the
easement. Plaintiffs sued for removal of the obstructions, an injunction against
defendant's interference with their use of the easement and damages. Defendants
counterclaimed for damages and an injunction against plaintiffs using the
easement other than for parcel B.

The trial court awarded each party \$1 in damages. The award against the
plaintiffs was for a slight inadvertent trespass outside the easement.

The trial court made the following findings of fact:

\begin{quotation}
\readinghead{VI}

The plaintiffs have made no unreasonable use of the easement in the development
of their property. There have been no complaints of unreasonable use of the
roadway to the south of the properties of the parties by other neighbors who
grant easements to the parties to this action to cross their properties to gain
access to the property of the plaintiffs. Other than the trespass there is no
evidence of any damage to the defendants as a result of the use of the easement
by the plaintiffs. There has been no increase in volume of travel on the
easement to reach a single family dwelling whether built on tract B or on Tacts
[sic] B and C. There is no evidence of any increase in the burden on the
subservient estate from the use of the easement by the plaintiffs for access to
parcel C.

\readinghead{VIII}

If an injunction were granted to bar plaintiffs access to tract C across the
easement to a single family residence, Parcel C would become landlocked;
plaintiffs would not be able to make use of their property; they would not be
able to build their single family residence in a manner to properly enjoy the
view of the Hood Canal and the surrounding area as originally anticipated at the
time of their purchase and even if the single family residence were constructed
on parcel B, if the injunction were granted, plaintiffs would not be able to use
the balance of their property in parcel C as a yard or for any other use of
their property in conjunction with their home. Conversely, there is and will be
no appreciable hardship or damage to the defendants if the injunction is
denied.

\readinghead{IX}

If an injunction were to be granted to bar the plaintiffs access to tract C, the
framing and enforcing of such an order would be impractical. Any violation of
the order would result in the parties back in court at great cost but with
little or no damages being involved.

\readinghead{X}

Plaintiffs have acted reasonable \textit{[sic]} in the development of their
property. Their trespass over a ``little'' corner of the defendants' property
was inadvertent, and \textit{de minimis.} The fact that the defendants counter
claim seeking an injunction to bar plaintiffs access to parcel C was filed as
leverage against the original plaintiffs' claim for an interruption of their
easement rights, may be considered in determining whether equitable relief by
way of an injunction should be granted.
\end{quotation}

Relying upon these findings of fact, the court denied defendant's request for an
injunction and granted the plaintiffs the right to use the easement for access
to parcels B \& C ``as long as plaintiffs [sic] properties (B and C) are
developed and used solely for the purpose of a single family residence.''

The Court of Appeals reversed\ldots.

The easement in this case was created by express grant. Accordingly, the extent
of the right acquired is to be determined from the terms of the grant properly
construed to give effect to the intention of the parties.  By the express terms
of the 1952 grant, the predecessor owners of parcel B acquired a private road
easement across parcel A and the right to use the easement for ingress to and
egress from parcel B. Both plaintiffs and defendants agree that the 1952 grant
created an easement appurtenant to parcel B as the dominant estate. Thus,
plaintiffs, as owners of the dominant estate, acquired rights in the use of the
easement for ingress to and egress from parcel B.

However, plaintiffs have no such easement rights in connection with their
ownership of parcel C, which was not a part of the original dominant estate
under the terms of the 1952 grant. As a general rule, an easement appurtenant to
one parcel of land may not be extended by the owner of the dominant estate to
other parcels owned by him, whether adjoining or distinct tracts, to which the
easement is not appurtenant. 

Plaintiffs, nonetheless, contend that extension of the use of the easement for
the benefit of nondominant property does not constitute a misuse of the
easement, where as here, there is no evidence of an increase in the burden on
the servient estate. We do not agree. If an easement is appurtenant to a
particular parcel of land, any extension thereof to other parcels is a misuse of
the easement.\ldots Under the express language of the 1952 grant, plaintiffs
only have rights in the use of the easement for the benefit of parcel B.
Although, as plaintiffs contend, their planned use of the easement to gain
access to a single family residence located partially on parcel B and partially
on parcel C is perhaps no more than technical misuse of the easement, we
conclude that it is misuse nonetheless.

However, it does not follow from this conclusion alone that defendants are
entitled to injunctive relief. Since the awards of \$1 in damages were not
appealed, only the denial of an injunction to defendants is in issue. Some
fundamental principles applicable to a request for an injunction must be
considered. (1) The proceeding is equitable and addressed to the sound
discretion of the trial court. (2) The trial court is vested with a broad
discretionary power to shape and fashion injunctive relief to fit the
\textit{particular facts, circumstances, and equities of the case before it.}
Appellate courts give great weight to the trial court's exercise of that
discretion. (3) One of the essential criteria for injunctive relief is actual
and substantial injury sustained by the person seeking the injunction.  

The trial court found as facts, upon substantial evidence, that plaintiffs have
acted reasonably in the development of their property, that there is and was no
damage to the defendants from plaintiffs' use of the easement, that there was no
increase in the volume of travel on the easement, that there was no increase in
the burden on the servient estate, that defendants sat by for more than a year
while plaintiffs expended more than \$11,000 on their project, and that
defendants' counterclaim was an effort to gain ``leverage'' against plaintiffs'
claim. In addition, the court found from the evidence that plaintiffs would
suffer considerable hardship if the injunction were granted whereas no
appreciable hardship or damages would flow to defendants from its denial.
Finally, the court limited plaintiffs' use of the combined parcels solely to the
same purpose for which the original parcel was used---\textit{i.e.},  for a
single family residence.

\ldots Based upon the equities of the case, as found by the trial court, we are
persuaded that the trial court acted within its discretion. The Court of Appeals
is reversed and the trial court is affirmed.

\opinion \textsc{Dore}, Justice (dissenting).

The majority correctly finds that an extension of this easement to nondominant
property is a misuse of the easement. The majority, nonetheless, holds that the
owners of the servient estate are not entitled to injunctive relief. I dissent.

The comments and illustrations found in the Restatement of Property \S~478
(1944) address the precise issue before this court. Comment \textit{e} provides
in pertinent part that ``if one who has an easement of way over Whiteacre
appurtenant to Blackacre uses the way with the purpose of going to Greenacre,
the use is improper even though he eventually goes to Blackacre rather than to
Greenacre.'' Illustration 6 provides:
\begin{quote}
6. By prescription, A has acquired, as the owner and possessor of Blackacre, an
easement of way over an alley leading from Blackacre to the street. He buys
Whiteacre, an adjacent lot, to which the way is not appurtenant, and builds a
public garage one-fourth of which is located on Blackacre and three-fourths of
which is located on Whiteacre. A wishes to use the alley as a means of ingress
and egress to and from the garage. He has no privilege to use the alley to go to
that part of the garage which is built on Whiteacre, and he may not use the
alley until that part of the garage built on Blackacre is so separated from the
part built on Whiteacre that uses for the benefit of Blackacre are
distinguishable from those which benefit Whiteacre.
\end{quote}
The majority grants the privilege to extend the agreement to nondominant
property on the basis that the trial court found no appreciable hardship or
damage to the servient owners. However, as conceded by the majority, any
extension of the use of an easement to benefit a nondominant estate constitutes
a misuse of the easement. Misuse of an easement is a trespass. The Brown's use
of the easement to benefit parcel C, especially if they build their home as
planned, would involve a continuing trespass for which damages would be
difficult to measure. Injunctive relief is the appropriate remedy under these
circumstances.\ldots

The Browns are responsible for the hardship of creating a landlocked parcel.
They knew or should have known from the public records that the easement was not
appurtenant to parcel C. In encroachment cases this factor is
significant.\dots

In addition, an injunction would not interfere with the Brown's right to use the
easement as expressly granted, \textit{i.e.},  for access to parcel B. An
injunction would merely require the Browns to acquire access to parcel C if they
want to build a home that straddles parcels B and C. One possibility would be to
condemn a private way of necessity over their existing easement in an action
under RCW 8.24.010. \textit{See Brown v. McAnally},  97 Wash.2d 360, 644 P.2d
1153 (1982).\dots

