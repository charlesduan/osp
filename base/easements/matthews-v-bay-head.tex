\reading{Matthews v. Bay Head Imp. Ass'n}

\readingcite{471 A.2d 355 (N.J. 1984)}

\dots In order to exercise these rights guaranteed by the public trust
doctrine, the public must have access to municipally-owned dry sand areas as
well as the foreshore. The extension of the public trust doctrine to include
municipally-owned dry sand areas was necessitated by our conclusion that
enjoyment of rights in the foreshore is inseparable from use of dry sand
beaches.\ldots We [previously] held that where a municipal beach is dedicated to
public use, the public trust doctrine ``dictates that the beach and the ocean
waters must be open to all on equal terms and without preference and that any
contrary state or municipal action is impermissible.'' 61 N.J. at 309,
294 A.2d 47.\dots

We now address the extent of the public's interest in privately-owned dry sand
beaches. This interest may take one of two forms. First, the public may have a
right to cross privately owned dry sand beaches in order to gain access to the
foreshore. Second, this interest may be of the sort enjoyed by the public in
municipal beaches\dots namely, the right to sunbathe and generally enjoy
recreational activities.

Beaches are a unique resource and are irreplaceable. The public demand for
beaches has increased with the growth of population and improvement of
transportation facilities.\dots

Exercise of the public's right to swim and bathe below the mean high water mark
may depend upon a right to pass across the upland beach. Without some means of
access the public right to use the foreshore would be meaningless. To say that
the public trust doctrine entitles the public to swim in the ocean and to use
the foreshore in connection therewith without assuring the public of a feasible
access route would seriously impinge on, if not effectively eliminate, the
rights of the public trust doctrine. This does not mean the public has an
unrestricted right to cross at will over any and all property bordering on the
common property. The public interest is satisfied so long as there is reasonable
access to the sea.\dots

The bather's right in the upland sands is not limited to passage. Reasonable
enjoyment of the foreshore and the sea cannot be realized unless some enjoyment
of the dry sand area is also allowed. The complete pleasure of swimming must be
accompanied by intermittent periods of rest and relaxation beyond the water's
edge. The unavailability of the physical situs for such rest and relaxation
would seriously curtail and in many situations eliminate the right to the
recreational use of the ocean. This was a principal reason why in [earlier
cases] we held that municipally-owned dry sand beaches ``must be open to all on
equal terms \ldots.'' We see no reason why rights under the public trust
doctrine to use of the upland dry sand area should be limited to
municipally-owned property. It is true that the private owner's interest in the
upland dry sand area is not identical to that of a municipality. Nonetheless,
where use of dry sand is essential or reasonably necessary for enjoyment of the
ocean, the doctrine warrants the public's use of the upland dry sand area
subject to an accommodation of the interests of the owner.

We perceive no need to attempt to apply notions of prescription, \textit{City of
Daytona Beach v. Tona-Rama, Inc.},  294 So.2d 73 (Fla.1974),
dedication, \textit{Gion v. City of Santa Cruz},  2 Cal.3d 29, 465
P.2d 50, 84 Cal.Rptr. 162 (1970), or custom, \textit{State ex
rel. Thornton v. Hay},  254 Or. 584, 462 P.2d 671 (1969), as
an alternative to application of the public trust doctrine. Archaic judicial
responses are not an answer to a modern social problem. Rather, we perceive the
public trust doctrine not to be ``fixed or static,'' but one to ``be molded and
extended to meet changing conditions and needs of the public it was created to
benefit.'' \textit{Avon},  61 N.J. at 309, 294 A.2d 47.

Precisely what privately-owned upland sand area will be available and required
to satisfy the public's rights under the public trust doctrine will depend on
the circumstances. Location of the dry sand area in relation to the foreshore,
extent and availability of publicly-owned upland sand area, nature and extent of
the public demand, and usage of the upland sand land by the owner are all
factors to be weighed and considered in fixing the contours of the usage of the
upper sand.

Today, recognizing the increasing demand for our State's beaches and the dynamic
nature of the public trust doctrine, we find that the public must be given both
access to and use of privately-owned dry sand areas as reasonably necessary.
While the public's rights in private beaches are not co-extensive with the
rights enjoyed in municipal beaches, private landowners may not in all instances
prevent the public from exercising its rights under the public trust doctrine.
The public must be afforded reasonable access to the foreshore as well as a
suitable area for recreation on the dry sand.

