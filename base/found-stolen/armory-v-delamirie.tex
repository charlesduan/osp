\reading{Armory v. Delamirie}
\readingcite{(1722) 1 Strange 505, 93 Eng. Rep. 664 (K.B.)}

The plaintiff being a chimney sweeper's boy found a jewel and carried it to the
defendant's shop (who was a goldsmith) to know what it was, and delivered it
into the hands of the apprentice, who under pretence of weighing it, took out
the stones, and calling to the master to let him know it came to three
halfpence, the master offered the boy the money, who refused to take it, and
insisted to have the thing again; whereupon the apprentice delivered him back
the socket without the stones. And now in trover against the master these
points were ruled: 

1. That the finder of a jewel, though he does not by such finding acquire an
absolute property or ownership, yet he has such a property as will enable him
to keep it against all but the rightful owner, and consequently may maintain
trover. 

2. That the action well lay against the master, who gives a credit to his
apprentice, and is answerable for his neglect. 

3. As to the value of the jewel several of the trade were examined to prove what
a jewel of the finest water that would fit the socket would be worth; and the
Chief Justice directed the jury, that unless the defendant did produce the
jewel, and shew it not to be of the finest water, they should presume the
strongest against him, and make the value of the best jewels the measure of
their damages: which they accordingly did. 

