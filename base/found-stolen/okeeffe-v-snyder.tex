\reading{O'Keeffe v. Snyder}
\readingcite{416 A.2d 862 (N.J. 1980)}

The opinion of the Court was delivered by POLLOCK, J.

This is an appeal from an order of the Appellate Division granting summary
judgment to plaintiff, Georgia O'Keeffe, against defendant, Barry Snyder, d/b/a
Princeton Gallery of Fine Art, for replevin of three small pictures painted by
O'Keeffe. In her complaint, filed in March, 1976, O'Keeffe alleged she was the
owner of the paintings and that they were stolen from a New York art gallery in
1946. Snyder asserted he was a purchaser for value of the paintings, he had
title by adverse possession, and O'Keeffe's action was barred by the expiration
of the six-year period of limitations provided by N.J.S.A. 2A:14-1 pertaining
to an action in replevin. Snyder impleaded third party defendant, Ulrich A.
Frank, from whom Snyder purchased the paintings in 1975 for \$35,000. 

The trial court granted summary judgment for Snyder on the ground that
O'Keeffe's action was barred because it was not commenced within six years of
the alleged theft. The Appellate Division reversed and entered judgment for
O'Keeffe. A majority of that court concluded that the paintings were stolen,
the defenses of expiration of the statute of limitations and title by adverse
possession were identical, and Snyder had not proved the elements of adverse
possession. Consequently, the majority ruled that O'Keeffe could still enforce
her right to possession of the paintings. 

The dissenting judge stated that the appropriate measurement of the period of
limitation was not by analogy to adverse possession, but by application of the
``discovery rule'' pertaining to some statutes of limitation. He concluded that
the six-year period of limitations commenced when O'Keeffe knew or should have
known who unlawfully possessed the paintings, and that the matter should be
remanded to determine if and when that event had occurred.

We granted certification\ldots

\readinghead{I}

The record, limited to pleadings, affidavits, answers to interrogatories, and
depositions, is fraught with factual conflict. Apart from the creation of the
paintings by O'Keeffe and their discovery in Snyder's gallery in 1976, the
parties agree on little else.

O'Keeffe contended the paintings were stolen in 1946 from a gallery, An American
Place. The gallery was operated by her late husband, the famous photographer
Alfred Stieglitz.

An American Place was a cooperative undertaking of O'Keeffe and some other
American artists\ldots. In 1946, Stieglitz arranged an exhibit which included
an O'Keeffe painting, identified as Cliffs. According to O'Keeffe, one day in
March, 1946, she and Stieglitz discovered Cliffs was missing from the wall of
the exhibit. O'Keeffe estimates the value of the painting at the time of the
alleged theft to have been about \$150.

About two weeks later, O'Keeffe noticed that two other paintings, Seaweed and
Fragments, were missing from a storage room at An American Place. She did not
tell anyone, even Stieglitz, about the missing paintings, since she did not
want to upset him.

Before the date when O'Keeffe discovered the disappearance of Seaweed, she had
already sold it (apparently for a string of amber beads) to a Mrs. Weiner, now
deceased. Following the grant of the motion for summary judgment by the trial
court in favor of Snyder, O'Keeffe submitted a release from the legatees of
Mrs. Weiner purportedly assigning to O'Keeffe their interest in the sale.

O'Keeffe testified on depositions that at about the same time as the
disappearance of her paintings, 12 or 13 miniature paintings by Marin also were
stolen from An American Place. According to O'Keeffe, a man named Estrick took
the Marin paintings and ``maybe a few other things.'' Estrick distributed the
Marin paintings to members of the theater world who, when confronted by
Stieglitz, returned them. However, neither Stieglitz nor O'Keeffe confronted
Estrick with the loss of any of the O'Keeffe paintings.

There was no evidence of a break and entry at An American Place on the dates
when O'Keeffe discovered the disappearance of her paintings. Neither Stieglitz
nor O'Keeffe reported them missing to the New York Police Department or any
other law enforcement agency. Apparently the paintings were uninsured, and
O'Keeffe did not seek reimbursement from an insurance company. Similarly,
neither O'Keeffe nor Stieglitz advertised the loss of the paintings in Art News
or any other publication. Nonetheless, they discussed it with associates in the
art world and later O'Keeffe mentioned the loss to the director of the Art
Institute of Chicago, but she did not ask him to do anything because ``it
wouldn't have been my way.'' O'Keeffe does not contend that Frank or Snyder had
actual knowledge of the alleged theft.

Stieglitz died in the summer of 1946, and O'Keeffe explains she did not pursue
her efforts to locate the paintings because she was settling his estate. In
1947, she retained the services of Doris Bry to help settle the estate. Bry
urged O'Keeffe to report the loss of the paintings, but O'Keeffe declined
because ``they never got anything back by reporting it.'' Finally, in 1972,
O'Keeffe authorized Bry to report the theft to the Art Dealers Association of
America, Inc., which maintains for its members a registry of stolen paintings.
The record does not indicate whether such a registry existed at the time the
paintings disappeared.

In September, 1975, O'Keeffe learned that the paintings were in the Andrew
Crispo Gallery in New York on consignment from Bernard Danenberg Galleries. On
February 11, 1976, O'Keeffe discovered that Ulrich A. Frank had sold the
paintings to Barry Snyder, d/b/a Princeton Gallery of Fine Art. She demanded
their return and, following Snyder's refusal, instituted this action for
replevin.

Frank traces his possession of the paintings to his father, Dr. Frank, who died
in 1968. He claims there is a family relationship by marriage between his
family and the Stieglitz family, a contention that O'Keeffe disputes. Frank
does not know how his father acquired the paintings, but he recalls seeing them
in his father's apartment in New Hampshire as early as 1941-1943, a period that
precedes the alleged theft. Consequently, Frank's factual contentions are
inconsistent with O'Keeffe's allegation of theft. Until 1965, Dr. Frank
occasionally lent the paintings to Ulrich Frank. In 1965, Dr. and Mrs. Frank
formally gave the paintings to Ulrich Frank, who kept them in his residences in
Yardley, Pennsylvania and Princeton, New Jersey. In 1968, he exhibited
anonymously Cliffs and Fragments in a one day art show in the Jewish Community
Center in Trenton. All of these events precede O'Keeffe's listing of the
paintings as stolen with the Art Dealers Association of America, Inc. in 1972.

Frank claims continuous possession of the paintings through his father for over
thirty years and admits selling the paintings to Snyder. Snyder and Frank do
not trace their provenance, or history of possession of the paintings, back to
O'Keeffe.

As indicated, Snyder moved for summary judgment on the theory that O'Keeffe's
action was barred by the statute of limitations and title had vested in Frank
by adverse possession. For purposes of his motion, Snyder conceded that the
paintings had been stolen. On her cross motion, O'Keeffe urged that the
paintings were stolen, the statute of limitations had not run, and title to the
paintings remained in her.

\readinghead{II}

[The court held that there was a genuine factual dispute whether the paintings
had been stolen.]

\readinghead{III}

On the limited record before us, we cannot determine now who has title to the
paintings. That determination will depend on the evidence adduced at trial.
Nonetheless, we believe it may aid the trial court and the parties to resolve
questions of law that may become relevant at trial.

Our decision begins with the principle that, generally speaking, if the
paintings were stolen, the thief acquired no title and could not transfer good
title to others regardless of their good faith and ignorance of the theft.
Proof of theft would advance O'Keeffe's right to possession of the paintings
absent other considerations such as expiration of the statute of limitations.

On this appeal, the critical legal question is when O'Keeffe's cause of action
accrued. The fulcrum on which the outcome turns is the statute of limitations
in N.J.S.A. 2A:14-1, which provides that an action for replevin of goods or
chattels must be commenced within six years after the accrual of the cause of
action.\ldots

Since the alleged theft occurred in New York, a preliminary question is whether
the statute of limitations of New York or New Jersey applies. The New York
statute, N.Y. Civ. Prac. Law {\S} 214 (McKinney), has been interpreted so that
the statute of limitations on a cause of action for replevin does not begin to
run until after refusal upon demand for the return of the goods. Here, O'Keeffe
demanded return of the paintings in February, 1976. If the New York statute
applied, her action would have been commenced within the period of limitations.

The traditional rule to determine which of two statutes of limitations is
applicable is that the statute of the forum governs unless the limitation is a
condition of the cause of action. However, this Court has discarded the
mechanical rule that the statute of limitations of the forum must be employed
in every suit on a foreign cause of action. \textit{Heavner v. Uniroyal, Inc.},
63 N.J. 130, 140-141 (1973). Heavner set out five requirements for barring an
action by applying a statute of limitations other than the appropriate New
Jersey statute: (1) the cause of action arose in the other state; (2) the
parties are all present in and amenable to jurisdiction in the other state; (3)
New Jersey has no substantial interest in the matter; (4) the substantive law
of the other jurisdiction is applicable, and (5) the limitations' period of the
other jurisdiction has expired at the time of the commencement of the suit in
New Jersey. The Heavner rule provides a limited and special exception to the
general rule that the rule of the forum determines the applicable period of
limitations. In the present case, none of the parties resides in New York and
the paintings are located in New Jersey. On the facts before us, it would
appear that the appropriate statute of limitations is the law of the forum,
N.J.S.A. 2A:14-1. On remand, the trial court may reconsider this issue if the
parties present other relevant facts.

\readinghead{IV}

On the assumption that New Jersey law will apply, we shall consider significant
questions raised about the interpretation of N.J.S.A. 2A:14-1. The purpose of a
statute of limitations is to stimulate to activity and punish negligence and
promote repose by giving security and stability to human affairs. A statute of
limitations achieves those purposes by barring a cause of action after the
statutory period. In certain instances, this Court has ruled that the literal
language of a statute of limitations should yield to other considerations.

To avoid harsh results from the mechanical application of the statute, the
courts have developed a concept known as the discovery rule. The discovery rule
provides that, in an appropriate case, a cause of action will not accrue until
the injured party discovers, or by exercise of reasonable diligence and
intelligence should have discovered, facts which form the basis of a cause of
action. The rule is essentially a principle of equity, the purpose of which is
to mitigate unjust results that otherwise might flow from strict adherence to a
rule of law.

This Court first announced the discovery rule in \textit{Fernandi}, supra, 35
N.J. at 434. In \textit{Fernandi}, a wing nut was left in a patient's abdomen
following surgery and was not discovered for three years. The majority held
that fairness and justice mandated that the statute of limitations should not
have commenced running until the plaintiff knew or had reason to know of the
presence of the foreign object in her body. \ldots

Similarly, we conclude that the discovery rule applies to an action for replevin
of a painting under N.J.S.A. 2A:14-1. O'Keeffe's cause of action accrued when
she first knew, or reasonably should have known through the exercise of due
diligence, of the cause of action, including the identity of the possessor of
the paintings. \ldots

In determining whether O'Keeffe is entitled to the benefit of the discovery
rule, the trial court should consider, among others, the following issues: (1)
whether O'Keeffe used due diligence to recover the paintings at the time of the
alleged theft and thereafter; (2) whether at the time of the alleged theft
there was an effective method, other than talking to her colleagues, for
O'Keeffe to alert the art world; and (3) whether registering paintings with the
Art Dealers Association of America, Inc. or any other organization would put a
reasonably prudent purchaser of art on constructive notice that someone other
than the possessor was the true owner.

\readinghead{V}

The acquisition of title to real and personal property by adverse possession is
based on the expiration of a statute of limitations. R. Brown, The Law of
Personal Property (3d ed. 1975), {\S} 4.1 at 33 (Brown). Adverse possession
does not create title by prescription apart from the statute of limitations.

To establish title by adverse possession to chattels, the rule of law has been
that the possession must be hostile, actual, visible, exclusive, and
continuous. \ldots

[T]here is an inherent problem with many kinds of personal property that will
raise questions whether their possession has been open, visible, and notorious.
\ldots For example, if jewelry is stolen from a municipality in one county in
New Jersey, it is unlikely that the owner would learn that someone is openly
wearing that jewelry in another county or even in the same municipality. Open
and visible possession of personal property, such as jewelry, may not be
sufficient to put the original owner on actual or constructive notice of the
identity of the possessor.

The problem is even more acute with works of art. Like many kinds of personal
property, works of art are readily moved and easily concealed. O'Keeffe argues
that nothing short of public display should be sufficient to alert the true
owner and start the statute running. Although there is merit in that contention
from the perspective of the original owner, the effect is to impose a heavy
burden on the purchasers of paintings who wish to enjoy the paintings in the
privacy of their homes. \ldots

The problem is serious. According to an affidavit submitted in this matter by
the president of the International Foundation for Art Research, there has been
an ``explosion in art thefts'' and there is a ``worldwide phenomenon of art
theft which has reached epidemic proportions''.

The limited record before us provides a brief glimpse into the arcane world of
sales of art, where paintings worth vast sums of money sometimes are bought
without inquiry about their provenance. There does not appear to be a
reasonably available method for an owner of art to record the ownership or
theft of paintings. Similarly, there are no reasonable means readily available
to a purchaser to ascertain the provenance of a painting. It may be time for
the art world to establish a means by which a good faith purchaser may
reasonably obtain the provenance of a painting. An efficient registry of
original works of art might better serve the interests of artists, owners of
art, and bona fide purchasers than the law of adverse possession with all of
its uncertainties. Although we cannot mandate the initiation of a registration
system, we can develop a rule for the commencement and running of the statute
of limitations that is more responsive to the needs of the art world than the
doctrine of adverse possession.

We are persuaded that the introduction of equitable considerations through the
discovery rule provides a more satisfactory response than the doctrine of
adverse possession. The discovery rule shifts the emphasis from the conduct of
the possessor to the conduct of the owner. The focus of the inquiry will no
longer be whether the possessor has met the tests of adverse possession, but
whether the owner has acted with due diligence in pursuing his or her personal
property.

For example, under the discovery rule, if an artist diligently seeks the
recovery of a lost or stolen painting, but cannot find it or discover the
identity of the possessor, the statute of limitations will not begin to run.
The rule permits an artist who uses reasonable efforts to report, investigate,
and recover a painting to preserve the rights of title and possession.

Properly interpreted, the discovery rule becomes a vehicle for transporting
equitable considerations into the statute of limitations for replevin. In
determining whether the discovery rule should apply, a court should identify,
evaluate, and weigh the equitable claims of all parties. If a chattel is
concealed from the true owner, fairness compels tolling the statute during the
period of concealment. That conclusion is consistent with tolling the statute
of limitations in a medical malpractice action where the physician is guilty of
fraudulent concealment.

It is consistent also with the law of replevin as it has developed apart from
the discovery rule. In an action for replevin, the period of limitations
ordinarily will run against the owner of lost or stolen property from the time
of the wrongful taking, absent fraud or concealment. Where the chattel is
fraudulently concealed, the general rule is that the statute is tolled \ldots.

The discovery rule will fulfill the purposes of a statute of limitations and
accord greater protection to the innocent owner of personal property whose
goods are lost or stolen. \ldots

By diligently pursuing their goods, owners may prevent the statute of
limitations from running. The meaning of due diligence will vary with the facts
of each case, including the nature and value of the personal property. For
example, with respect to jewelry of moderate value, it may be sufficient if the
owner reports the theft to the police. With respect to art work of greater
value, it may be reasonable to expect an owner to do more. In practice, our
ruling should contribute to more careful practices concerning the purchase of
art.

The considerations are different with real estate, and there is no reason to
disturb the application of the doctrine of adverse possession to real estate.
Real estate is fixed and cannot be moved or concealed. The owner of real
property knows or should know where his property is located and reasonably can
be expected to be aware of open, notorious, visible, hostile, continuous acts
of possession on it.

Our ruling not only changes the requirements for acquiring title to personal
property after an alleged unlawful taking, but also shifts the burden of proof
at trial. Under the doctrine of adverse possession, the burden is on the
possessor to prove the elements of adverse possession. Under the discovery
rule, the burden is on the owner as the one seeking the benefit of the rule to
establish facts that would justify deferring the beginning of the period of
limitations.

\readinghead{VI}

Read literally, the effect of the expiration of the statute of limitations under
N.J.S.A. 2A:14-1 is to bar an action such as replevin. The statute does not
speak of divesting the original owner of title. By its terms the statute cuts
off the remedy, but not the right of title. Nonetheless, the effect of the
expiration of the statute of limitations, albeit on the theory of adverse
possession, has been not only to bar an action for possession, but also to vest
title in the possessor. There is no reason to change that result although the
discovery rule has replaced adverse possession. History, reason, and common
sense support the conclusion that the expiration of the statute of limitations
bars the remedy to recover possession and also vests title in the possessor.
\ldots

Before the expiration of the statute, the possessor has both the chattel and the
right to keep it except as against the true owner. The only imperfection in the
possessor's right to retain the chattel is the original owner's right to
repossess it. Once that imperfection is removed, the possessor should have good
title for all purposes. Ames, \textit{The Disseisin of Chattels}, 3
\textsc{Harv. L. Rev}. 313, 321 (1890) (Ames). As Dean Ames wrote: ``An
immortal right to bring an eternally prohibited action is a metaphysical
subtlety that the present writer cannot pretend to understand.'' \textit{Id}.
at 319.

Recognizing a metaphysical notion of title in the owner would be of little
benefit to him or her and would create potential problems for the possessor and
third parties. The expiration of the six-year period of N.J.S.A. 2A:14-1 should
vest title as effectively under the discovery rule as under the doctrine of
adverse possession. \ldots

\readinghead{VII}

We next consider the effect of transfers of a chattel from one possessor to
another during the period of limitation under the discovery rule. Under the
discovery rule, the statute of limitations on an action for replevin begins to
run when the owner knows or reasonably should know of his cause of action and
the identity of the possessor of the chattel. Subsequent transfers of the
chattel are part of the continuous dispossession of the chattel from the
original owner. The important point is not that there has been a substitution
of possessors, but that there has been a continuous dispossession of the former
owner.

Professor Ballantine explains:
\begin{quote}
Where the same claim of title has been consistently asserted for the statutory
period by persons in privity with each other, there is the same reason to quiet
and establish the title as where one person has held. The same flag has been
kept flying for the whole period. It is the same ouster and disseisin. If the
statute runs, it quiets a title which has been consistently asserted and
exercised as against the true owner, and the possession of the prior holder
justly enures to the benefit of the last. [H. Ballantine, \textit{Title by
Adverse Possession}, 32 \textsc{Harv. L. Rev}. 135, 158 (1919)] \ldots
\end{quote}
For the purpose of evaluating the due diligence of an owner, the dispossession
of his chattel is a continuum not susceptible to separation into distinct acts.
Nonetheless, subsequent transfers of the chattel may affect the degree of
difficulty encountered by a diligent owner seeking to recover his goods. To
that extent, subsequent transfers and their potential for frustrating diligence
are relevant in applying the discovery rule. An owner who diligently seeks his
chattel should be entitled to the benefit of the discovery rule although it may
have passed through many hands. Conversely an owner who sleeps on his rights
may be denied the benefit of the discovery rule although the chattel may have
been possessed by only one person.

We reject the alternative of treating subsequent transfers of a chattel as
separate acts of conversion that would start the statute of limitations running
anew.  \ldots

Treating subsequent transfers as separate acts of conversion could lead to
absurd results. As explained by Dean Ames:
\begin{quote}
\ldots If a converter were to sell the chattel, five years after its
conversion, to one ignorant of the seller's tort, the disposed owner's right to
recover the chattel from the purchaser would continue five years longer than
his right to recover from the converter would have lasted if there had been no
sale. In other words, an innocent purchaser from a wrong-doer would be in a
worse position than the wrong-doer himself,---a conclusion as shocking in
point of justice as it would be anomalous in law. [Ames, \textit{supra} at 323,
footnotes omitted]
\end{quote}
It is more sensible to recognize that on expiration of the period of
limitations, title passes from the former owner by operation of the statute.
Needless uncertainty would result from starting the statute running anew merely
because of a subsequent transfer. \ldots

We reverse the judgment of the Appellate Division in favor of O'Keeffe and
remand the matter for trial in accordance with this opinion.

