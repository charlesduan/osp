\reading{Other Variations on \emph{Armory}}

Just how far does the holding of \textit{Armory v. Delamirie} (``That the finder
of [property], though he does not by such finding acquire an absolute property
or ownership, yet he has such a property as will enable him to keep it against
all but the rightful owner'') go? Consider three nineteenth-century cases about
lost lumber. Are they required by \textit{Armory}? Consistent with
\textit{Armory}? Consistent with each other? Which is most persuasive?

In \textit{Clark v. Maloney}, 3 Del. 68 (1840), the plaintiff found ten logs
floating in a bay after a storm. He tied them up in the mouth of a creek, but
they (apparently) got free again and the defendants (apparently) found them
floating up the creek. \textit{Held}, the plaintiffs were entitled to the logs:
\begin{quote}
Possession is certainly prima facie evidence of property. It is called
\textit{prima facie} evidence because it may be rebutted by evidence of better
title, but in the absence of better title it is as effective a support of title
as the most conclusive evidence could be. It is for this reason, that
\textit{the finder of a chattel, though he does not acquire an absolute
property in it, yet has such a property, as will enable him to keep it against
all but the rightful owner}. The defence consists, not in showing that the
defendants are the rightful owners, or claim under the rightful owner; but that
the logs were found by them adrift in Mispillion creek, having been loosened
from their fastening either by accident or design, and they insist that their
title is as good as that\textbf{\textit{} }of the plaintiff. But it is a well
settled rule of law that the loss of a chattel does not change the right of
property; and for the same reason that the original loss of these logs by the
rightful owner, did not change his absolute property in them, but he might have
maintained trover against the plaintiff upon refusal to deliver them, so the
subsequent loss did not divest the \textit{special} property of the plaintiff.
It follows, therefore, that as the plaintiff has shown a special property in
these logs, which he never abandoned, and which enabled him to keep them
against all the world but the rightful owner, he is entitled to a verdict.
\end{quote}

In \textit{Anderson v. Gouldberg}, 53 N.W. 636 (Minn. 1892), the defendants took
ninety-three logs from the plaintiff's mill. The defendants claimed that the
plaintiff had cut the logs on their land, but the plaintiff replied (and a jury
agreed) that he had actually cut the logs by trespassing on the land of a third
party. \textit{Held}: the plaintiff was entitled to the logs:
\begin{quote}
Therefore the only question is whether bare possession of property, though
wrongfully obtained, is sufficient title to enable the party enjoying it to
maintain replevin against a mere stranger, who takes it from him. We had
supposed that this was settled in the affirmative as long ago, at least, as the
early case of \textit{Armory v. Delamirie}, so often cited on that point. When
it is said that to maintain replevin the plaintiff's possession must have been
lawful, it means merely that it must have been lawful as against the person who
deprived him of it; and possession is good title against all the world except
those having a better title. Counsel says that possession only raises a
presumption of title, which, however, may be rebutted. Rightly understood, this
is correct; but counsel misapplies it. One who takes property from the
possession of another can only rebut this presumption by showing a superior
title in himself, or in some way connecting himself with one who has. One who
has acquired the possession of property, whether by finding, bailment, or by
mere tort, has a right to retain that possession as against a mere wrongdoer
who is a stranger to the property. Any other rule would lead to an endless
series of unlawful seizures and reprisals in every case where property had once
passed out of the possession of the rightful owner.
\end{quote}
\textit{Anderson} states what is overwhelmingly the majority rule. Seven years
after \textit{Anderson}, North Carolina took the opposite course. In
\textit{Russell v. Hill}, 34 S.E. 640 (N.C. 1899), two different people held
what appeared to be state grants to the same tract of land, and the plaintiff
cut timber on the land with the wrong one's permission. While the logs were
floating in a river, the defendants---unconnected with either of the purported
landowners---took them away and sold them. \textit{Held}: the defendants were
entitled to the logs (internal quotation marks omitted):
\begin{quote}
In some of the English books, and in some of the Reports of our sister states,
cases might be found to the contrary, but that those cases were all founded
upon a misapprehension of the principle laid down in the case of \textit{Armory
v. Delamirie}. There a chimney sweep found a lost jewel. He took it into his
possession, as he had a right to do, and was the owner, because of having it in
possession, unless the true owner should become known. That owner was not
known, and it was properly decided that trover would lie in favor of the finder
against the defendant, to whom he had handed it for inspection, and who refused
to restore it. But the court said the case would have been very different if
the owner had been known.
\end{quote}
Is this an accurate reading of \textit{Armory}? The court also expressed concern
about the defendant's potential liability to the true owner:
\begin{quote}
It is true that, as possession is the strongest evidence of the ownership,
property may be presumed from possession. \dots{} But if it appears on the
trial that the plaintiff, although in possession, is not in fact the owner, the
presumption of title inferred from the possession is rebutted, and it would be
manifestly wrong to allow the plaintiff to recover the value of the property;
for the real owner may forthwith bring trover against the defendant, and force
him to pay the value the second time, and the fact that he paid it in a former
suit would be no defense. Consequently trover can never be maintained unless a
satisfaction of the judgment will have the effect of vesting a good title in
the defendant.
\end{quote}
Is the fear of double liability sufficient reason to allow the defendant to
escape liability entirely? Based on a review of the court records in the case,
John V. Orth writes that the true owner in \textit{Russell v. Hill} was ``no
bodiless abstraction but had in fact a name and identity: [Fabius Haywood]
Busbee, one of the state's leading lawyers, a man well known to every member of
the supreme court that decided the case.'' John V. Orth, Russell v.
Hill \textit{(N.C. 1899): Misunderstood Lessons}, 73 \textsc{N.C. L. Rev.} 2031,
2034 (1995). Does this help explain \textit{Russell}? 

Professor Orth, arguing for a middle ground between \textit{Anderson} and
\textit{Russell}, argues that \textit{Armory} should protect only prior
possessors who took the property in good faith: ``A technical wrongdoing, such
as an innocent trespass, as the source of possession should not disable the
possessor from securing judicial protection against an unauthorized taking, but
a willful trespass at the root of title should. Plaintiff in \textit{Russell},
in other words, deserved a new trial at which to show, not his title, but his
\textit{bona fides}.'' \textit{Id.} at 2060. Is this a better rule?

