\expected{armory-v-delamirie}

\item One way of describing the holding of \textit{Armory} is that it sets out
the rights of finders. Suppose that the ``rightful owner'' of the jewel, Lord
Hobnob, had shown up in the shop while the chimney-sweep and the apprentice
were arguing over the jewel. Who would have been entitled to the jewel? If the
chimney-sweep is not the ``rightful owner,'' why does he still win the case?
What kind of interest does he have in the jewel?

\item A second way of describing of describing the holding of \textit{Armory} is
that it illustrates ``relativity of title.'' As between the plaintiff and the
defendant, the party with the relatively better claim to title wins, even if
their title is in some sense defective in an absolute sense. Relativity of
title is intimately connected to the idea of ``chains of title'': competing
claimants to a piece of property each do their best to trace their claims back
to a rightful source. What is the source of the chimney-sweep's claim to the
jewel? And the jeweler's? Does this explain the outcome of the case? What
result if the jeweler had proven that he had signed a contract to purchase the
jewel from Lord Hobnob but that Lord Hobnob had lost the jewel before
delivering it?

\item A third way of describing the holding of \textit{Armory} is that it
rejects the jeweler's attempt to assert a \textit{jus tertii} (Latin for
``right of a third party'') defense. The defendant cannot defeat the
plaintiff's otherwise-valid claim to the jewel by arguing that a third party --
Lord Hobnob -- has an even better claim. Put differently, we might say that
``as against a wrongdoer, possession is title.'' \textit{Jeffries v. Great W.
Ry. Co.}, (1856) 119 Eng. Rep. 680, 681 (Q.B.). Does this narrowing of focus to
the parties before the court make sense? 

Here is one way to think about it. Suppose that Lord Hobnob shows up in court
while \textit{Armory} is being argued and explains that the jewel slipped from
his finger while he was strolling in Lincoln's Inn Fields. Who is entitled to
the jewel? What if Lord Hobnob shows up and explains that he tossed the jewel
aside in the mud, saying ``I have become tired of this bauble; it bores me and
I no longer wish to have it.'' What if he explains that he handed it to the
chimney-sweep, saying ``I wish you to have this jewel; may it serve you better
than it has me.'' But recall that in the actual case, Lord Hobnob was nowhere
to be found; no one even knew his identity. Does it matter to the outcome of
\textit{Armory v. Delamirie} how the jewel passed from Lord Hobnob's hands to
the chimney-sweep's? 

If you are still not convinced, consider this. If the jeweler could set up Lord
Hobnob's title to show that the chimney-sweep's title was defective, would the
chimney-sweep be entitled to present evidence that Lord Hobnob's title was
defective, say because Lord Hobnob stole the jewel from a visiting Frenchman in
1693? Cutting off inquiry into third parties' claims also helps cut off inquiry
into old claims. Can you see why this might be an appealing choice for a system
of property law?

\item We are not quite done with Lord Hobnob. Consider the remedy the plaintiff
obtains: an award of the value of the jewel, rather than the jewel itself. This
is in effect a forced sale of the jewel, which the defendant can keep after
paying the plaintiff's damage award. \textit{Now} who owns the jewel? What if
Lord Hobnob shows up now? Can he also bring trover, and if so, will the jeweler
be forced to pay out a second time? In fact, why is Paul de Lamerie, the
goldsmith whose name the court mangles, on the hook for his apprentice's
wrongdoing? What if the apprentice pocketed the jewel and never turned it over
to the master?

\item About that damage award. Why is the jury instructed to presume that the
jewel was ``of the finest water?'' (i.e. highest quality)?

