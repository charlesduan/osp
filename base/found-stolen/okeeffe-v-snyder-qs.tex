\expected{okeeffe-v-snyder}

\item What did O'Keeffe do to locate the paintings? At least according to the
court, what more could she have done? What more should she have done? What did
the Franks and Snyder do to make their possession of the paintings clear? What
more could they have done? What more should they have done? Did the court
properly balance the parties' interests? Did it give the right incentives to
future parties in their positions?


\item The court argues that switching from adverse possession to the discovery
rule ``shifts the emphasis from the conduct of the possessor to the conduct of
the owner.'' Is this a good description of the difference between the two
tests? Is the court's explanation of its reasons for the change persuasive?


\item Notice \textit{O'Keeffe}'s discussion of the choice-of-law problem.
O'Keeffe's suit was timely under the New York statute of limitations but may
not have been under New Jersey's. In theory, choice of law is simple for
property: the law of the property's ``situs'' (i.e. location)
controls.\footnote{Relatedly, courts have \textit{in rem} jurisdiction over
property located within their state's borders, and the traditional rule has
been that courts have no jurisdiction at all over real property outside their
state's borders. Why might these rules have developed? Do they seem likely to
simplify litigation or complicate it?} The rule is easy enough to apply to real
property, although even there hard cases are possible.\footnote{\textit{See,
e.g.}, \textit{Durfee v. Duke}, 375 U.S. 106 (1963), in which the Missouri
River, which forms the boundary between Nebraska and Missouri, had shifted its
channel from the east of the land in question to the west of it. If the river
had shifted suddenly (by ``avulsion''), the boundary stayed where it was and
the land was legally in Nebraska. But if the river had shifted course slowly
(by ``accretion''), the boundary moved with the river and the land was in
Missouri. Since the plaintiff claimed title under a Nebraska foreclosure
proceeding and the defendant claimed title under a Missouri swamp land patent,
the case turned on which state the land was in.} But personal property can move
around, generating contacts with multiple states. Suppose the contacts had been
flipped, so that the painting was stolen in New Jersey but was currently in New
York. Should New York law have applied? Another possible rule is that the law
of the place where the property is now applies. What incentives would that rule
create? How about a rule selecting the law of the place where the property was
at the time of the relevant events? (Wait. What \textit{are} the ``relevant
events'' in a replevin case involving the statute of limitations?) Another
layer of difficulty in choice of law comes from the characterization problem:
is the validity of a mortgage securing a loan with an illegally high rate of
interest a ``property'' issue (governed by the situs rule) or a ``contract''
issue (governed by the place the contract was made or the place of residence of
the parties)? The characterization question puts \textit{O'Keeffe}'s use of
medical-malpractice tort principles in a replevin case in a new light, doesn't
it?

