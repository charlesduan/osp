\expected{mcavoy-v-medina}

\item In \textit{Lawrence v. State}, on which \textit{McAvoy} relies, the
customer did come back for his lost pocketbook containing \$480 in bank notes,
which he had left on a table while the barber went out to make change. To quote
the court: ``The barber left the shop to get the bill changed, and, a fight
occurring in the streets, the [customer's] attention was arrested thereat and
he left the shop, his pocket-book lying on the table.'' When he returned, the
barber ``denied all knowledge of the pocket-book'' but then ``expended [the
bank notes] in the purchase of confections, etc.'' A criminal prosecution for
grand larceny followed, and the barber argued that the pocketbook had been lost
because larceny only applies when the defendant takes property from the
possession of the victim. The court held that because the pocketbook on a table
was merely \textit{mislaid}, rather than ``lost,'' it was still within the
customer's ``constructive possession.'' First of all, is this plausible? And
second, is this a good fit for the facts of \textit{McAvoy}?


\item By way of contrast, in \textit{Bridges v. Hawkesworth}, which
\textit{McAvoy} distinguishes, the plaintiff found a small parcel on the floor
of the defendant's shop and immediately showed it to the defendant's employee.
The parcel contained bank notes; the plaintiff ``requested the defendant to
deliver them to the owner.'' Three years later, with no owner having returned,
the court held the plaintiff as finder was entitled to the notes. ``If the
notes had been accidentally kicked into the street, and then found by someone
passing by, could it be contended that the defendant was entitled to them, from
the mere fact of their having been dropped in his shop? \dots{} Certainly not.
The notes were never in the custody of the defendant, nor within the protection
of his house before they were found, as they would have had they been
intentionally deposited there, and the defendant has come under no
responsibility.'' First, what do you make of the \textit{Bridges} court's
argument that the shopkeeper's entitlement to the notes should turn on whether
he would have been held responsible to the true owner for losing them? And
second, is this any better a fit for the facts of \textit{McAvoy}?


\item What do you make of the argument that awarding the pocket-book to the
shopkeeper is ``one better adapted to secure the rights of the true owner?''


\item In addition to lost and mislaid property, there is also abandoned
property: property which the owner has voluntarily relinquished with no intent
to reclaim. Since abandoned property is again unowned, the usual rules of first
possession apply. (\having{pierson-v-post}{As you have seen, these}{As you will see in \emph{Pierson v.~Post}, these}{These} rules themselves are not as simple
as ``first possessor wins.''). How easy is it to tell the three apart? Why?


\item In \textit{Benjamin v. Lindner Aviation}, 534 N.W.2d 400 (Iowa
1995)\textbf{} in which an airplane inspector found \$18,000 in cash inside the
wing of an airplane in 1992 while the plane was parked in his employer's hangar
for maintenance. The money, which consisted primarily of \$20 bills dating to
the 1950s and 1960s, was in two four-inch packets wrapped in handkerchiefs and
tied with string and then wrapped again in aluminum foil. The packets were
inserted behind a panel on the underside of the plane's wing; the panel was
secured with rusty screws that had not been removed in several years. The
inspector, the employer, and the bank that owned the plane (after repossessing
it from a prior owner who had defaulted on a loan) all made claims to the
money. Was it lost, mislaid, or abandoned, and who was entitled to it?


\item Another category sometimes mentioned in the found-property caselaw is
treasure trove: money, gold, or silver intentionally placed underground, which
is found long enough later that it is likely the owner is dead or will never
return for it. At common law in England, treasure trove belonged to the King.
Most American states now treat treasure trove like any other found property. Is
this a sensible rejection of an archaic and pointless quirk of the common-law,
or was there something to the doctrine?

\hyphenation{gwern-hay-lod}

\item In \textit{Hannah v. Peel}, [1945] K.B. 509, the British government
requisitioned Gwernhaylod House in 1940 for use during World War II and paid
the owner, Major Hugh Edward Ethelston Peel {\pounds}250 per year. The house
had been conveyed to Major Peel in 1938 but it was unoccupied from then until
when it was requisitioned. Duncan Hannah, a lance-corporal with the Royal
Artillery, was stationed in the house and was adjusting a blackout curtain in
August 1940 when he found something loose in a crevice on top of the
window-frame. It turned out to be a brooch covered in cobwebs and dirt; he
informed his commanding officer and then turned it over to the police. Two
years later, the police gave it to Major Peel, who sold it for {\pounds}66.
Lance-Corporal Hannah sued and was awarded the value of the brooch. The court
discussed numerous cases, including \textit{Bridges v. Hawkesworth} and
\textit{South Staffordshire Water Co. v. Sharman}, [1896] 2 Q.B. 44, which
awarded two rings found by a workman embedded in the mud at the bottom of a
pool to the company that owned the land. From them, it extracted a rule that
``a man possesses everything which is attached to or under his land.'' Since
Major Peel ``was never physically in possession of these premises'' and hence
had no ``prior possession'' of the brooch, Lance-Corporal Hannah was entitled
to it as a finder. Is this possession-based approach a better way of analyzing
found-property cases than the categorical lost-vs-mislaid American approach
exemplified by \textit{McAvoy}? Or is \textit{Hannah} an oddball outlier driven
by the court's desire to do right by a wartime serviceman ``whose conduct was
commendable and meritorious,'' especially as against an absentee landlord from
the local gentry?

