\reading{McAvoy v. Medina}
\readingcite{93 Mass. (11 Allen) 548 (1866)}

\textsc{Tort} to recover a sum of money found by the plaintiff in the shop of
the
defendant.

[I]t appeared that the defendant was a barber, and the plaintiff, being a
customer in the defendant's shop, saw and took up a pocket-book which was lying
upon a table there, and said, ``See what I have found.'' The defendant came to
the table and asked where he found it. The plaintiff laid it back in the same
place and said, ``I found it right there.'' The defendant then took it and
counted the money, and the plaintiff told him to keep it, and if the owner
should come to give it to him; and otherwise to advertise it; which the
defendant promised to do. Subsequently the plaintiff made three demands for the
money, and the defendant never claimed to hold the same till the last demand.
It was agreed that the pocket-book was placed upon the table by a transient
customer of the defendant and accidentally left there, and was first seen and
taken up by the plaintiff, and that the owner had not been found.\ldots

\opinion \textsc{Dewey}, J.

It seems to be the settled law that the finder of lost property has a valid
claim to the same against all the world except the true owner, and generally
that the place in which it is found creates no exception to this rule.

But this property is not, under the circumstances, to be treated as lost
property in that sense in which a finder has a valid claim to hold the same
until called for by the true owner. This property was voluntarily placed upon a
table in the defendant's shop by a customer of his who accidentally left the
same there and has never called for it. The plaintiff also came there as a
customer, and first saw the same and took it up from the table. The plaintiff
did not by this acquire the right to
take % sic; checked as accurate
the property from the shop, but it
was rather the duty of the defendant, when the fact became thus known to him,
to use reasonable care for the safe keeping of the same until the owner should
call for it. In the case of \textit{Bridges v. Hawkesworth}, 7 Eng. Law \& Eq.
R. 424, the property, although found in a shop, was found on the floor of the
same, and had not been placed there voluntarily by the owner, and the court
held that the finder was entitled to the possession of the same, except as to
the owner. But the present case more resembles that of \textit{Lawrence v. The
State}, 1 Humph. (Tenn.) 228, and is indeed very similar in its facts. The
court there take a distinction between the case of property thus placed by the
owner and neglected to be removed, and property lost. It was there held that
``to place a pocket-book upon a table and to forget to take it away is not to
lose it, in the sense in which the authorities referred to speak of lost
property.''

We accept this as the better rule, and especially as one better adapted to
secure the rights of the true owner.

In view of the facts of this case, the plaintiff acquired no original right to
the property, and the defendant's subsequent acts in receiving and holding the
property in the manner he did does not create any.

