
\reading{The New York Mess}

\having{okeeffe-v-snyder}{As noted in \textit{O'Keeffe}, }{As we will see in
\emph{O'Keeffe v.~Snyder}, }{}New York has a three-year statute of limitations
for personal-property actions, and traditionally applied a demand-and-refusal
rule to start the statute running. Two further doctrines complicate the
picture. One is that the demand-and-refusal rule only applies against
good-faith purchasers; the limitations period in a suit against a thief starts
at the time of the theft. (Do you see how this result, illogical as it may
sound, follows from the logic that the good-faith purchaser is not
considered a wrongdoer until she refuses a demand for return of the property?)
The other is that an owner who unreasonably delays making a demand for the
return of property, at least where she knows the identity of the possessor, may
find her suit barred by the equitable doctrine of laches.

In \textit{DeWeerth v. Baldinger}, 836 F.2d 103 (2d Cir. 1987), a landscape by
Claude Monet owned by Gerda DeWeerth disappeared from a castle in Southern
Germany where American soldiers were quartered during World War II. It turned
up on the art market in the mid-1950s and was eventually sold by a New York
gallery to Edith Baldinger, who kept it in her apartment in New York. In 1981,
DeWeerth's nephew tracked the painting to the gallery's sale to Baldinger and
made a demand for its return which was refused. The Second Circuit, sitting in
diversity, held that New York ``would impose a duty of reasonable diligence in
attempting to locate stolen property,'' not just a duty to demand its return in
a reasonable time after the property is located:
\begin{quote}
For if demand is delayed, then so is accrual of the cause of action, and the
good-faith purchaser will remain exposed to suit long after an action against a
thief or even other innocent parties would be time-barred. \dots{} In this
case, plaintiff's proposed exception to the rule would rob it of all of its
salutary effect: The thief would be immune from suit after three years, while
the good-faith purchaser would remain exposed as long as his identity did not
fortuitously come to the property owner's attention. A construction of the rule
requiring due diligence in making a demand to include an obligation to make a
reasonable effort to locate the property will prevent unnecessary hardship to
the good-faith purchaser, the party intended to be protected. \dots{} A rule
requiring reasonable diligence in attempting to locate stolen property is
especially appropriate with respect to stolen art. Much art is kept in private
collections, unadvertised and unavailable to the public. An owner seeking to
recover such property will almost never learn of its whereabouts by chance. Yet
the location of stolen art may frequently be discovered through investigation.
\end{quote}
The court concluded that DeWeerth's efforts were ``minimal'' before her nephew
took up the case in 1981, so Baldinger kept the Monet.

That was 1987. Shortly thereafter, the same issue came up through the New York
state court system. In \textit{Guggenheim Foundation v. Lubell}, 569 N.E.2d 426
(N.Y. 1991), a painting by Marc Chagall was stolen from the Guggenheim Museum
by a mailroom employee in the 1960s. The Lubells bought the painting from a
reputable dealer in 1967; the museum demanded it back in 1986. The court
rejected \textit{O'Keeffe}, repudiated \textit{DeWeerth}{}'s interpretation of
New York law, and reaffirmed the New York demand-and-refusal rule. It
specifically rejected the discovery rule with its requirement of reasonable
diligence by the owner:
\begin{quotation}
Our case law already recognizes that the true owner, having discovered the
location of its lost property, cannot unreasonably delay making demand upon the
person in possession of that property. \dots{} Further, the facts of this case
reveal how difficult it would be to specify the type of conduct that would be
required for a showing of reasonable diligence. Here, the parties hotly contest
whether publicizing the theft would have turned up the gouache. According to
the museum, some members of the art community believe that publicizing a theft
exposes gaps in security and can lead to more thefts; the museum also argues
that publicity often pushes a missing painting further underground. In light of
the fact that members of the art community have apparently not reached a
consensus on the best way to retrieve stolen art, it would be particularly
inappropriate for this Court to spell out arbitrary rules of conduct that all
true owners of stolen art work would have to follow to the letter if they
wanted to preserve their right to pursue a cause of action in replevin. All
owners of stolen property should not be expected to behave in the same way and
should not be held to a common standard. The value of the property stolen, the
manner in which it was stolen, and the type of institution from which it was
stolen will all necessarily affect the manner in which a true owner will search
for missing property. We conclude that it would be difficult, if not
impossible, to craft a reasonable diligence requirement that could take into
account all of these variables and that would not unduly burden the true owner.

Further, our decision today is in part influenced by our recognition that New
York enjoys a worldwide reputation as a preeminent cultural center. To place
the burden of locating stolen artwork on the true owner and to foreclose the
rights of that owner to recover its property if the burden is not met would, we
believe, encourage illicit trafficking in stolen art. Three years after the
theft, any purchaser, good faith or not, would be able to hold onto stolen art
work unless the true owner was able to establish that it had undertaken a
reasonable search for the missing art. This shifting of the burden onto the
wronged owner is inappropriate. In our opinion, the better rule gives the owner
relatively greater protection and places the burden of investigating the
provenance of a work of art on the potential purchaser.
\end{quotation}
Armed with the New York Court of Appeals's holding in \textit{Guggenheim}, Gerda
DeWeerth filed a Rule 60(b) motion for relief from judgment, arguing that the
decision against her rested on a misinterpretation of New York law and that she
should not have been subjected to a diligent-search requirement. The District
Court agreed with her, but the Second Circuit reversed in \textit{DeWeerth v.
Baldinger,} 38 F.3d 1266 (2d Cir. 1994), emphasizing the need for finality in
litigation:
\begin{quote}
We conclude that the prior \textit{DeWeerth} panel conscientiously satisfied its
duty to predict how New York courts would decide the due diligence question,
and that \textit{Erie} and its progeny require no more than this. The fact that
the New York Court of Appeals subsequently reached a contrary conclusion in
\textit{Guggenheim} does not constitute an ``extraordinary circumstance'' that
would justify reopening this case in order to achieve a similar result.
\end{quote}

Finally, consider \textit{SongByrd Inc. v. Estate of Grossman}, 206 F.3d 172 (2d
Cir. 2000), another diversity case under New York law. Henry Byrd, who recorded
under the name ``Professor Longhair,'' was a celebrated jazz musician. He went
to Woodstock, New York for a studio recording session for Bearsville Records in
the 1970s. The session was considered unsatisfactory at the time, so the tapes
were never released. Instead, Arthur Davis, the record-store owner who
discovered Byrd, sent the tapes to Bearsville ``as demonstration tapes only,
without any intent for either Albert Grossman or Bearsville Records Inc. to
possess these aforementioned tapes as owner.'' Byrd's attorney wrote two
letters to Bearsville requesting the return of the tapes in 1975, but there was
no evidence in the record that the letters were even received. Byrd died in
1980, and after Bearsville's founder died in 1985, his estate licensed the
recordings to two record companies. One of the resulting albums won Byrd a
(posthumous) Grammy for Best Traditional Blues Album of 1987. SongByrd, the
successor-in-interest to Byrd's rights, sued in 1995. The Second Circuit held
that the conversion claim was barred, because the defendant ``began using the
master tapes as its own when it licensed portions of them to Rounder in 1986.''
\begin{quote}
The conversion alleged by SongByrd occurred no later than that date. The
demand-and-refusal rule, which functioned to delay accrual of the claim in
[\textit{Guggenheim}] \dots{} for the benefit of the true owner, normally
provides some benefit to the good-faith possessor by precipitating its
awareness that continued possession will be regarded as wrongful by the true
owner. New York has not required a demand and refusal for the accrual of a
conversion claim against a possessor who openly deals with the property as its
own. 
\end{quote}
As an alternative basis for its holding, the court added that the plaintiff had
unreasonably delayed making its demand.
\begin{quote}
Even if a demand were required for accrual of SongByrd's claim,
[\textit{Guggenheim}] instructs that a plaintiff may not unreasonably delay in
making a demand for property whose location is known. Byrd, either
independently or through his agents, had known since the 1970s that the master
tapes were in Grossman's possession, and the unanswered letters to Grossman in
1975 for return of the master tapes probably sufficed to alert him to
Grossman's disregard of his ownership claim, thereby rendering any demand
thereafter unreasonably delayed. In any event, his successors' delay in not
making a demand in 1987, when Bearsville's licensing of the master tapes became
well known in the music world as a result of the Grammy Award for Byrd's
recordings, was clearly unreasonable.
\end{quote}

After \textit{DeWeerth}, \textit{Guggenheim}, and \textit{SongByrd}, does New
York have a coherent approach to the statute of limitations in personal
property cases? Does it depend whether the case is brought in state or federal
court? Has New York done better or worse than New Jersey at balancing the
competing interests at stake?


