\reading{Matthew v. Smith}

\readingcite{707 S.W.2d 411 (Mo. 1986)}

\opinion \textsc{Welliver}, Judge.

This is an appeal from a circuit court judgment affirming the Board of Zoning
Adjustment's decision to grant Jim and Susan Brandt a variance. The Brandts
purchased a residential lot containing two separate houses upon a tract of land
zoned for a single-family use. The court of appeals reversed the circuit court
judgment, and the case was then certified to this Court by a dissenting judge.
We reverse and remand. 

The Brandts own a tract of land comprising one and one-half plotted lots. When
they purchased the property in March of 1980, there already were two houses on
the land, one toward the front of Erie Street and one in the rear. Each of the
buildings is occupied by one residential family as tenants of the Brandts. The
two houses apparently have been used as separate residences for the past thirty
years, with only intermittent vacancies. The property is zoned for Single Family
Residences. At the suggestion of a city official, the Brandts applied for a
variance which would allow them to rent both houses with a single family in each
house. After some delay, including two hearings by the Board of Zoning
Adjustment of Kansas City, the Board granted the application. Appellant, Jon
Matthew, a neighboring landowner challenged the grant of the variance and sought
a petition for certiorari from the Board's action. The circuit court affirmed
the Board's order; on appeal, the court of appeals held that the Board was
without authority to grant the requested variance. A dissenting judge certified
the case to this Court\ldots .

Under most zoning acts, these boards have the authority to grant variances from
the strict letter of the zoning ordinance. The variance procedure ``fulfil[s] a
sort of `escape hatch' or `safety valve' function for individual landowners who
would suffer special hardship from the literal application of the\ldots zoning
ordinance.'' It is often said that ``[t]he variance provides an administrative
alternative for individual relief that can avoid the damage that can occur to a
zoning ordinance as a result of as applied taking litigation.'' The general rule
is that the authority to grant a variance should be exercised sparingly and only
under exceptional circumstances. 

Both the majority of courts and the commentators recognize two types of
variances: an area (nonuse) variance and a use variance.

The two types of variances with which cases are customarily concerned are
``use'' variances and ``nonuse variances.'' The latter consist mostly of
variances of bulk restrictions, of area, height, density, setback, side line
restrictions, and restrictions coverning miscellaneous subjects, including the
right to enlarge nonconforming uses or to alter nonconforming structures.

As the name indicates, a use variance is one which permits a use other than one
of those prescribed by the zoning ordinance in the particular district; it
permits a use which the ordinance prohibits. A nonuse variance authorizes
deviations from restrictions which relate to a permitted use, rather than
limitations on the use itself, that is, restrictions on the bulk of buildings,
or relating to their height, size, and extent of lot coverage, or minimum
habitable area therein, or on the placement of buildings and structures on the
lot with respect to required yards. Variances made necessary by the physical
characteristics of the lot itself are nonuse variances of a kind commonly termed
``area variances.''

Many zoning acts or ordinances expressly distinguish between the two types of
variances. When the distinction is not statutory, ``the courts have always
distinguished use from area variances.'' Some jurisdictions, whether by express
statutory directive or by court interpretation, do not permit the grant of a use
variance. 

[The Brandts] seek a variance to use the property in a manner not permitted
under the permissible uses established by the ordinance. The ordinance clearly
permits only the use of the property for a single family residence. The
applicant is not seeking a variance from the area and yard restrictions which
are no doubt violated because of the existence of the second residence. Such an
area variance is not necessary because the applicant has a permissible
nonconforming structure under the ordinance. 

\ldots [T]he express language of \S~89.090, RSMo 1978,\ldots grants the Board
the ``power to vary or modify the application of any of the regulations or
provisions of such ordinance relating to the \textit{use}, construction or
alteration of buildings or structures, or the use of land'' (emphasis added).
We, therefore, hold that under the proper circumstances an applicant may obtain
a use variance. 

Section 89.090, RSMo 1978 delegates to the Board of Adjustment the power to
grant a variance when the applicant establishes ``practical difficulties or
unnecessary hardship in the way of carrying out the strict letter of such
ordinance\ldots so that the spirit of the ordinance shall be observed, public
safety and welfare secured and substantial justice done.''\ldots.

Almost all jurisdictions embellished the general concepts of ``unnecessary
hardship'' or ``practical difficulties'' by further defining the conditions an
applicant must satisfy before obtaining a variance\ldots .

Unfortunately, any attempt to set forth a unified structure illustrating how all
the courts have treated these conditions would, according to Professor Williams,
prove unsuccessful. Williams observes that the law of variances is in ``great
confusion'' and that aside from general themes any further attempt at unifying
the law indicates ``either (a) [one] has not read the case law, or (b) [one] has
simply not understood it. Here far more than elsewhere in American planning law,
muddle reigns supreme.'' Yet, four general themes can be distilled from variance
law and indicate what an applicant for a variance must prove:
\begin{itemize}
\item[(1)] relief is necessary because of the unique character of the property
rather than for personal considerations; and

\item[(2)] applying the strict letter of the ordinance would result in
unnecessary hardship; and the

\item[(3)] imposition of such a hardship is not necessary for the preservation
of the plan; and

\item[(4)] granting the variance will result in substantial justice to all.
\end{itemize}
Although all the requirements must be satisfied, it is generally held that
``\,`[u]nnecessary hardship' is the principal basis on which a variance is
granted.'' 

Before further examining the contours of unnecessary hardship, jurisdictions
such as Missouri that follow the New York model rather than the Standard Act
need to address the significance of the statutory dual standard of ``unnecessary
hardship'' or ``practical difficulties.'' Generally, this dual standard has been
treated in one of two ways. On the one hand, many courts view the two terms as
interchangeable. On the other hand, a number of jurisdictions follow the
approach of New York, the jurisdiction where the language originated, and hold
that ``practical difficulties'' is a slightly lesser standard than ``unnecessary
hardship'' and only applies to the granting of an area variance and not a use
variance. The rationale for this approach is that an area variance is a
relaxation of one or more incidental limitations to a permitted use and does not
alter the character of the district as much as a use not permitted by the
ordinance.

In light of our decision to permit the granting of a use variance, we are
persuaded that the New York rule reflects the sound approach for treating the
distinction between area and use variances. To obtain a use variance, an
applicant must demonstrate, inter alia, unnecessary hardship; and, to obtain an
area variance, an applicant must establish, inter alia, the existence of
conditions slightly less rigorous than unnecessary hardship.

\ldots It is generally said that \emph{Otto v. Steinhilber}, 282 N.Y. 71, 24
N.E.2d 851, 853 (1939) contains the classic definition of unnecessary hardship: 
\begin{quote}
Before the Board may exercise its discretion and grant a variance upon the
ground of unnecessary hardship, the record must show that (1) the land in
question cannot yield a reasonable return if used only for a purpose allowed in
that zone; (2) that the plight of the owner is due to unique circumstances and
not to the general conditions in the neighborhood which may reflect the
unreasonableness of the zoning ordinance itself; and (3) that the use to be
authorized by the variance will not alter the essential character of the
locality.
\end{quote}

Quite often the existence of unnecessary hardship depends upon whether the
landowner can establish that without the variance the property cannot yield a
reasonable return. ``Reasonable return is not maximum return.'' Rather, the
landowner must demonstrate that he or she will be deprived of all beneficial use
of the property under any of the permitted uses:
\begin{quote}
A zoning regulation imposes unnecessary hardship if property to which it applies
cannot yield a reasonable return from any permitted use. Lack of a reasonable
return may be shown by proof that the owner has been deprived of all beneficial
use of his land. All beneficial use is said to have been lost where the land is
not suitable for any use permitted by the zoning ordinance.
\end{quote}

Most courts agree that mere conclusory and lay opinion concerning the lack of
any reasonable return is not sufficient; there must be actual proof, often in
the form of dollars and cents evidence. In a well-reasoned opinion, Judge Meyer
of the New York Court of Appeals stated:
\begin{quote}
Whether the existing zoning permits of a reasonable return requires proof from
which can be determined the rate of return earned by like property in the
community and proof in dollars and cents form of the owner's investment in the
property as well as the return that the property will produce from the various
uses permissible under the existing classification.
\end{quote}
\emph{N. Westchester Prof. Park v. Town of Bedford}, 458 N.E.2d 809 (N.Y. 1983).
Such pronouncements and requirements of the vast majority of jurisdictions
illustrate that, if the law of variances is to have any viability, only in the
exceptional case will a use variance be justified. 

\ldots [T]he record is without sufficient evidence to establish unnecessary
hardship. The only evidence in the record is the conclusory opinion of Brandt
that they would be deprived of a reasonable return if not allowed to rent both
houses. No evidence of land values was offered; and, no dollars and cents proof
was presented to demonstrate that they would be deprived of all beneficial use
of their property. Appellant, in fact, was not permitted to introduce such
evidence. The Board, therefore, was without authority to grant a use variance
upon this record. 

The record, however, indicates that the Brandts may be entitled to a
nonconforming use under the ordinance.\ldots 

\opinion \textsc{Robertson}, Judge, concurring in result.

[Judge Robertson
concurred on the ground that the Brandts sought an area variance, not a use
variance, but, under the zoning ordinance, they still needed to demonstrate that
the property couldn't earn a reasonable return without the variance.] [A
separate concurrence is omitted.]

