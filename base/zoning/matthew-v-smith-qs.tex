\expected{matthew-v-smith}

\item Was this a use variance or an area variance? 

\item
Note that the prior nonconforming use alternative is both more stringent and
more relaxed: it requires the use to predate the zoning, but it also requires no
showing of hardship once that priority is established.

\item
Although the standard of review is supposed to be deferential, reversals of
zoning board decisions are not uncommon. \textit{See, e.g.}, \emph{Housing
Authority v. Board of Adjustment},
941 S.W.2d 725 (Mo. Ct. App. 1997) (board abused discretion in denying
variances for lot size and setbacks where unusual size of parcel, which was laid
out before zoning was enacted, meant that no conforming building could be
erected, and where numerous other nearby properties had similar lot sizes and
setbacks); \emph{State ex rel. Klawuhn v. Board of Zoning Adjustment},
952 S.W.2d 725 (Mo. Ct. App. 1997) (board wrongly granted three
variances to allow owners to build a storage building on a vacant lot and store
various vehicles and equipment in it; asserted hardship was personal to owners,
``namely the large quantity of vehicles and equipment they wished to store
inside the proposed storage building,'' even though housing the vehicles inside
a structure might be more aesthetically appealing to neighbors than keeping them
in open view; when asked whether he could get by with a smaller storage shed,
owner responded, ``Not and put what\ldots I have to put in it''). 

\item
\textbf{Mistakes}. Is a good-faith mistake a self-inflicted hardship? The answer
is usually yes. \textit{See, e.g.}, \emph{Wehrle v. Cassor}, 708 SW 2d 788 (Mo.
Ct. App. 1986) (board erred in granting variance where violation, and hardship
involved in curing violation, resulted from builders' measurement errors).

\item
\textbf{Purchase with knowledge of the problem}. Suppose undeveloped land is
purchased by someone who knows or should know that the land can't be developed
in accordance with current restrictions without a variance. Does purchase with
knowledge of a hardship count as a self-inflicted harm, disentitling the owner
to a variance? \emph{See, e.g.}, \emph{Conley v. Town of Brookhaven Zoning Bd.
of Appeals}, 40 N.Y.2d 309 (N.Y. 1976) (self-imposed hardship through purchase
with notice of restrictions didn't preclude the zoning board from granting an
area variance); \emph{Somol v. Board of Adjustment}, 649 A.2d 422 (N.J. Super.
Ct. Law Div. 1994) (as long as a prior owner didn't create the hardship,
purchase with knowledge of the restrictions is no barrier to a variance);
\emph{In re Gregor}, 627 A.2d 308 (Pa. Commw. Ct. 1993) (``The right to develop
a nonconforming lot is not personal to the owner of property at the time of
enactment of the zoning ordinance but runs with the land, and a purchaser's
knowledge of zoning restrictions alone is insufficient to preclude the grant of
a variance unless the purchase itself gives rise to the hardship.''). In what
way could a prior owner or a purchaser create the hardship? 

For use variances, by contrast to area variances, purchase with knowledge
precludes a claim for a variance. Why distinguish area variances from use
variances in this context? 

\item
\textbf{Can refusal to sell be a self-inflicted hardship?} In \textit{Wolfner v.
Board of Adjustment of City of Warson Woods}, 114 S.W.3d 298 (Mo. Ct. App.
2003), the owners bought one lot in 1939 and built a house on it, before zoning
began in 1941, thus creating a prior nonconforming use. After 1941, they
acquired an adjacent lot that was too small to be built on under the 1941
zoning. Until 1995, the owners used the adjacent lot as a sideyard. The
surviving owner then sold the main lot, but not the adjacent lot. The buyer of
the main lot tried to buy the adjacent lot, but the owner rejected the offer,
along with other offers from surrounding property owners. She requested a
variance allowing a home to be built on the adjacent lot---it was only 7,500
square feet and 60 feet wide, less than the required 8,750 square feet and
70-foot width. The Board denied her request, and that of subsequent purchasers,
the Wolfners, whose purchase was conditional on getting the variance. The
Wolfners agreed to pay \$80,000 for the lot on the hope they could build on it;
the Board found that this was not the kind of harm that merited a variance.

The court upheld the denial, noting that it was still possible that neighboring
owners would be interested in buying the lot at its fair market value as a side
yard. Is this fair? Note that if the original owners had \textit{not} owned an
adjacent lot, they would almost certainly have been entitled to the variance
because their property was otherwise unbuildable. \textit{Compare, e.g.},
\emph{Detwiler v. Zoning Hearing Board}, 596 A.2d 1156 (Pa. Comm. Ct. 1991)
(holding owners of oddly shaped parcel entitled to variance even though they
bought after the zoning began); \emph{Commons v. Westwood Zoning Board of
Adjustment.}, 410 A.2d 1138 (N.J. 1980) (similar result; although neighbors
might be entitled to denial of variance if they were willing to buy the
undersized parcel at fair market value, fair market value was to be calculated
according to the value of the parcel with the variance, not the much lower value
of the parcel without it).

\item
\textbf{The law in action}. The legal standards governing variances are fairly
easy to state, but doctrine doesn't necessarily control outcomes; facts on the
ground are much more important. See Kathryn Moore, \textit{The Lexington-Fayette
Urban County Board of Adjustment: Fifty Years Later}, 100 \textsc{Ky. L.J.} 435
(2011-2012) (law professor who served on zoning board commented on ``the Board's
tendency to make decisions that seem fair and practical rather than technically
legally correct. Indeed, I am not sure that it is possible or even reasonable to
expect a lay body to prefer technically legally correct decisions to practical
and fair decisions, especially when the staff recommends the practical decision
over the legally correct decision.''). The conventional wisdom is that courts
reverse the grant of variances more often than their denial. Do you share the
judicial intuition that an issued variance is more likely to be problematic than
a denied one? The individual entity seeking a variance usually has a more
focused interest in getting it than the rest of the neighbors have in blocking
it. Some people who seek variances have even bribed zoning authorities.

