There are a variety of other refinements or complications in the zoning process
that provide flexibility. In theory, they should all have to conform to the
general development plan or the plan itself should have to be changed; practice
is somewhat more messy. This section provides only a brief introduction to the
relevant concepts. A class in land use law or local government will provide
substantially more detail.

\paragraph{Special exceptions/special uses/conditional uses}
A special exception (varyingly known as a special use or conditional use in
different states) is a ban on particular types of uses, such as apartment
buildings, unless certain criteria are met. One might wonder how they differ
from variances. The basic idea is that variances are necessary though not
desirable, designed to deal with unexpected situations in which land uses that
are otherwise banned should be allowed, usually for parcel-specific and
therefore unpredictable reasons. We know that there is, in general, a need for
the ability to grant variances, but we don't know which variances we will need.
So the standards for variances are worded generally. 

By contrast, special exceptions are authorized when the zoning body anticipates
that particular uses will be appropriate, but should be carefully scrutinized.
When a special exception is authorized by the zoning code, that reflects a
determination that the use is generally appropriate for the zone. As a result,
the zoning board must not be left with only vague criteria that do not constrain
its discretion when assessing whether a particular application should be
granted. With variances, the risk of arbitrary decisions has to be borne to
provide the necessary flexibility. But when the zoning authority can anticipate
the issues that will predictably arise with a particular use---apartments, for
example, are likely to raise questions about how many parking spaces are
needed---then there is no need to take the risk of arbitrary or biased
enforcement. ``The issuing of a permit is a ministerial act, not a discretionary
act, which may not be refused if the requirements of the applicable ordinance
have been met.'' State ex rel. \emph{Kugler v. City of Maryland Heights}, 817
S.W.2d 931 (Mo. Ct. App. 1991); \textit{see also} \emph{Curry Inv. Co. v. Board
of Zoning Adjustment of Kansas City}, 399 S.W.3d 106 (Mo. Ct. App. 2013)
(finding that the zoning board unlawfully made approval of a special use permit
conditional on the removal of two nonconforming signs; signs were lawful as
prior nonconforming uses, and the board's staff concluded that all the criteria
for a special use permit were met); \emph{Waeckerle v. Board of Zoning
Adjustment}, 525 S.W.2d 351 (Mo. Ct. App. 1975) (allowing the zoning board to
treat a conditional use application as requiring a variance ``would amount to
permitting the Board to exercise legislative power,'' conflicting with its
administrative role; zoning board cannot repeal authorization for uses given by
legislature). Relatedly, no special showing of hardship is required to grant a
special use permit, unlike a variance. The inevitable legal debate over when
rules are preferable to standards, or vice versa, is actualized in zoning by
using both. 

When a state is concerned about equalizing the burden of particular uses, it may
mandate that a sub-state jurisdiction provide for them through special
exceptions. Missouri law, for example, requires municipalities with more than
500 persons to allow substance abuse treatment facilities as a permitted,
conditional special use. Municipalities may establish density standards and
require that exterior appearance conform to area standards. Section 89.143 RSMo.

\paragraph{Floating zones} Floating zones are something like special exceptions,
in that they contemplate that a particular use or combination of uses will be
appropriate for an area under certain circumstances, but it's not yet clear
exactly where that use should be. Once a development plan is proposed by a
developer and accepted by the zoning authority, the floating zone ``lands.''
\textit{See} \emph{Treme v. St. Louis County}, 609 S.W.2d 706 (Mo. Ct. App.
1980) (accepting floating zones so long as the determination to rezone a
particular piece of property in a floating zone is not arbitrary, capricious or
unreasonable). Floating zones are useful for extensively planned developments
that may need more flexibility in use than the current zoning allows. The plan
can also be overlaid onto an existing zoning district if there's a proposal with
no need to ``float''; either way, the rezoning usually only takes place once a
plan is approved. \emph{See, e.g.}, \emph{Heidrich v. City of Lee's Summit}, 916
S.W.2d 242 (Mo. Ct. App. 1995) (dealing with a planned district); \emph{McCarty
v. City of Kansas City}, 671 S.W.2d 790 (Mo. Ct. App. 1984) (approval of plan is
a legislative act).

\paragraph{Planned Unit Development (PUD)} A PUD is a self-contained
development, often with a mixture of housing types and densities, in which the
subdivision and zoning controls are applied to the project as a whole rather
than to individual lots. Densities are thus calculated for the entire
development, which allows clustering of houses and common open spaces.
\textit{See} \emph{Turner v. City of Independence}, 186 S.W.3d 786 (Mo. Ct. App.
W.D. 2006) (upholding high density residential mixed use planned unit
development rezoning ordinance enacted by City as lawful and reasonable). Within
a PUD, the number of uses expressly permitted is limited and the number of
conditional uses is expanded, allowing the zoning authority more control over
the development of the land. Developers may use a PUD to get more flexibility in
terms of open space, parking, and setback requirements, in return for giving
zoning authorities more control than they would normally have in matters of
building appearance and landscaping. \textit{See, e.g.}, \emph{State ex rel.
Helujon, Ltd. v. Jefferson County}, 964 S.W.2d 531 (Mo. Ct. App. 1998)
(accepting PUD as legitimate legislative rezoning technique).
Ladue has now provided for a PUD in its zoning ordinance:
\begin{quotation}
This section is intended to enable the creation of a Planned Unit Development
(P.U.D.) District on properties with a minimum size of twelve (12) acres that
abut a City border.

The purpose of the Planned Unit Development District overlay is to provide a
means of achieving greater flexibility in development of land in a manner not
possible in the underlying zoning district; to encourage development of
downsized luxury housing; to encourage a more environmentally sustainable
development; to promote a more desirable community environment; and to maintain
maximum control over both the structure and future operation of the development.

A Planned Unit Development District overlay is not a rezoning of the property;
only those uses permitted in the underlying zoning classification shall be
allowed\ldots. Lot area, yard setbacks, lot frontage, lot width, and other
requirements and regulations contained in the underlying zoning districts may be
altered or amended as set forth in the authorized Planned Unit Development
District. There shall be no increase in unit density in residentially zoned
districts.
\end{quotation}
Ladue, Missouri's Zoning Ordinance, Ordinance 1175, as amended through
Feb.~2016.

\paragraph{Rezoning} Rezoning more generally is exactly what it sounds like. As
long as it is part of a comprehensive plan, it is usually acceptable, even if it
changes the rules substantially (and doesn't just exclude specific businesses,
the way the rezoning in prior nonconforming use cases often does). 

\begin{quotation}
[Under Missouri law, t]he requirement for passage of the rezoning ordinance is a
simple majority. It takes a two-thirds vote, however, if the owners of thirty
percent or more of the land within 185 feet of the boundaries of the area of
land (exclusive of streets and alleys) that is being rezoned sign and
acknowledge (before a notary public) a written protest against the rezoning.

In some cities there are additional self-imposed limitations on rezoning
amendments. These limitations state that, if the planning commission recommends
against the proposed amendment, then it will take a three-fourths vote of the
council to overturn that action.
\end{quotation}
Missouri Municipal League, \emph{Planning and Zoning Procedures for Missouri
Municipalities} (Sept. 2004).

Should we treat rezoning as legislative in nature, and thus entitled to very
deferential judicial review the way the initial adoption of a zoning plan is
treated under \textit{Euclid}, or rather as quasi-judicial like a variance and
subject to less deference? The courts are divided on this question. 

\paragraph{Contract zoning} This is an often derogatory term for a rezoning in
which a developer promises to provide certain benefits to the zoning
jurisdiction in return for zoning that allows the developer to accomplish its
goals. In theory, it should not be allowed, because it makes the idea of general
planning seem like a sick joke. In practice, it is hard to distinguish from
acceptable rezoning, and courts have increasingly tolerated it, perhaps
reflecting the commodification of all other values. Christopher Serkin, Local
Property Law: Adjusting the Scale of Property Protection, 107 Colum. L. Rev. 883
(2007). Nonetheless, most suburban communities have not accepted contract
zoning, as a political matter.

\paragraph{Spot zoning} This is another kind of rezoning, in which a particular
parcel is rezoned (rather than being given a variance, for which the standard
would be much higher). Because it can be used as a variance workaround when the
zoning board is on the owner's side, some courts are skeptical of spot zoning.
The classic scenario involves a parcel that is zoned to ``higher'' use, often
single-family residential, but abuts a less restrictive zone. The developer
wishes to use the parcel for apartments, and argues that the neighborhood is
already transitional in character and that another apartment building will be
consistent with the overall area. What responses can you imagine the residential
neighbors making? 

Because of the potential for collusion between a zoning board and the owner of a
benefitted parcel, spot zoning is more often the legal conclusion of a court
striking down a zoning change than a characterization adopted by a zoning board
to describe what it is doing. Courts tend to be particularly suspicious when a
change confers unique benefits on a specific parcel, making it distinctly more
valuable than its neighbors. It is not necessary that the new use cause
hardships to the neighbors; the problem is one of unjustified favoritism.

\paragraph{Upzoning and downzoning} You may expect that rezoning often favors
developers trying to take advantage of desirable locations. In fact,
``downzoning''---making it harder to build at higher densities, which are the
most profitable for developers---may often be more successful than upzoning.
Homevoters, it seems, are likely to have the political power to protect new
housing from coming in and diluting the value of prized locations, or attracting
the ``wrong'' sorts of residents. \textit{See} Vicki Been, Josiah Madar \& Simon
McDonnell, \textit{Urban Land-Use Regulation: Are Homevoters Overtaking the
Growth Machine?}, 11 \textsc{J. Empir. Leg. Stud.} 227 (2014) (finding, in study
of New York City, that areas in proximity to high-quality infrastructure and
services were more likely to have zoning changes than other areas, but almost
always in the direction of downzoning, so that parcels in high-performing school
districts were 43\% more likely than the typical parcel to be upzoned but 392\%
more likely to be downzoned; downzoning was also highly correlated with race,
with parcels in areas that were 80\% white more than seven times more likely to
be downzoned than parcels in areas that were under 20\% white).

