\readingnote{Excerpts reprinted with permission. Prof.~Fischel, an economist,
studies zoning; he also sat on a zoning board for several years in order to
better understand its workings.}

\reading[Fischel, The Evolution of Zoning Since the 1980s]{William A. Fischel,
\textit{The Evolution of Zoning Since the 1980s: The Persistence of Localism}}
\readingcite{In \textsc{Property in Land and Other Resources} 259 (Daniel H.
Cole \& Elinor Ostrom eds., 2012)}

An observation about zoning boards that might be useful to scholars is that
visiting the site in question is essential. Site visits can change the views of
the case enormously. An applicant may show charming pictures of his antique-car
hobby and seek a variance only to park some storage trailers. A visit might
reveal that he actually harbors a private junkyard. Local knowledge is important
because there is a literature on zoning boards, most often by attorneys, that
finds fault with their decisions. One early and well-known critique is the
article by Dukeminier and Stapleton (1962). A more recent study was conducted by
an attorney who statistically examined variance decisions in five New Hampshire
towns, one of which was Hanover, during the years 1987--1992, when I was on the
zoning board. His chief finding was that variances are disproportionately
granted if abutters do not object (Kent 1993; Ellickson and Been 2000). To which
most board members would say, ``Who knows better whether the variance will have
an adverse effect?'' The practice of granting variances if abutters do not
object illustrates the recurrence of an early, grassroots approach to land use
regulation, which required nonconforming uses to obtain permission of local
property owners. It was struck down as unlawful delegation of the police power
in several early cases, such as \emph{Eubank v. City of Richmond}, 226 U.S. 137
(1912), but most local zoning boards informally operate as if it were still in
effect.

Kent (1993) neglected to point out that four of the five towns in his sample
have administrative officers who could discourage applicants with weak cases
(Hanover's certainly did), but none of the other ``misrule-by-variance'' studies
worry much about selection bias either. Kent also accurately reported that
during the period he examined, the New Hampshire Supreme Court overturned the
decisions of all ten towns whose opponents appealed their granting of variances.
This seems to support his conclusion that local boards were prodigal in this
regard. However, the decision in \emph{Simplex v. Newington}, 145 N.H. 727
(2001), changed the court's previous zoning variance criteria, on which Kent had
relied as the source of proper variances, to a less exacting standard that more
closely reflected actual practice.

Legal error is not practical error, much less economic harm. Although the
articles critical of boards mention the possibility that variances will degrade
the neighborhood, even anecdotal evidence in support of that contention is
scarce. Without visiting the site in question, it is often extremely difficult
to tell whether the variance was warranted by legal, practical, or economic
criteria. An underappreciated study by David Bryden (1977) established this more
systematically. Bryden examined scores of Minnesota lakeshore building and
septic variances (which he had no part in granting) and concluded that what
looked like a travesty from the legal record in almost all cases made perfectly
good sense to local board members who were acquainted with the details of the
sites in question. For example, building setback variances, which by themselves
seemed to have been issued with little regard to the state's standard criteria,
were granted most often to allow septic systems to be even farther from the lake
than the state required. The local officials knew the sites and made what Bryden
inferred were appropriate trade-offs between the serious risk of septic-tank
pollution of water bodies and the less consequential aesthetic concerns of
building setbacks.

This is not to say that zoning boards are faultless. Some members can be
inclined to promote a political agenda. Favoritism and score settling can
influence some members' votes. But even the least sophisticated zoning boards
have an asset that is almost never available to appellate judges or to
statistical analysts: they know at least the neighborhood and usually the
specific site from personal experience. This makes a big difference that critics
of boards need to take into account.
