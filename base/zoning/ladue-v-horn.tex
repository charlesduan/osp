\reading{City of Ladue v. Horn}

\readingcite{720 S.W.2d 745 (Mo. Ct. App. 1986)}

Defendants, Joan Horn and E. Terrence Jones, appeal from the judgment of the
trial court in favor of plaintiff, City of Ladue (Ladue), which enjoined
defendants from occupying their home in violation of Ladue's zoning ordinance
and which dismissed defendants' counterclaim. We affirm.

The case was submitted to the trial court on stipulated facts. Ladue's Zoning
Ordinance No. 1175 was in effect at all times pertinent to the present action.
Certain zones were designated as one-family residential. The zoning ordinance
defined family as: ``One or more persons related by blood, marriage or adoption,
occupying a dwelling unit as an individual housekeeping organization.'' The only
authorized accessory use in residential districts was for ``[a]ccommodations for
domestic persons employed and living on the premises and home occupations.'' The
purpose of Ladue's zoning ordinance was broadly stated as to promote ``the
health, safety, morals and general welfare'' of Ladue.

In July, 1981, defendants purchased a seven-bedroom, four-bathroom house which
was located in a single-family residential zone in Ladue. Residing in
defendants' home were Horn's two children (aged 16 and 19) and Jones's one child
(age 18). The two older children attended out-of-state universities and lived in
the house only on a part-time basis. Although defendants were not married, they
shared a common bedroom, maintained a joint checking account for the household
expenses, ate their meals together, entertained together, and disciplined each
other's children. Ladue made demands upon defendants to vacate their home
because their household did not comprise a family, as defined by Ladue's zoning
ordinance, and therefore they could not live in an area zoned for single-family
dwellings. When defendants refused to vacate, Ladue sought to enjoin defendants'
continued violation of the zoning ordinance. Defendants counterclaimed, seeking
a declaration that the zoning ordinance was constitutionally void. They also
sought attorneys' fees and costs. The trial court entered a permanent injunction
in favor of Ladue and dismissed defendants' counterclaim. Enforcement of the
injunction was stayed pending this appeal.

\ldots In Missouri, the scope of appellate review in zoning matters is limited;
and the reviewing court may not substitute its judgment for that of the zoning
authority. A zoning ordinance is presumed valid. The legislative body is vested
with broad discretion and the appellate court cannot interfere unless it is
shown that the legislative body has acted arbitrarily. ``If the council's action
is fairly debatable, the court cannot substitute its opinion.'' 

\ldots Capsulated, defendants' attack on Ladue's ordinance is three-pronged.
First, the zoning limitations foreclose them from exercising their right to
associate freely with whomever they wish. Second, their right to privacy is
violated by the zoning restrictions. Third, the zoning classification
distinguishes between related persons and unrelated persons. Defendants allege
that the United States and Missouri Constitutions grant each of them the right
to share his or her residence with whomever he or she chooses. They assert that
Ladue has not demonstrated a compelling, much less rational, justification for
the overly proscriptive blood or legal relationship requirement in its zoning
ordinance.

Defendants posit that the term ``family'' is susceptible to several meanings.
They contend that, since their household is the ``functional and factual
equivalent of a natural family,'' the ordinance may not preclude them from
living in a single-family residential Ladue neighborhood. Defendants argue in
their brief as follows:
\begin{quote}
The record amply demonstrates that the private, intimate interests of Horn and
Jones are substantial. Horn, Jones, and their respective children have
historically lived together as a single family unit. They use and occupy their
home for the identical purposes and in the identical manners as families which
are biologically or maritally related.
\end{quote}

To bolster this contention, defendants elaborate on their shared duties, as set
forth earlier in this opinion. Defendants acknowledge the importance of viewing
themselves as a family unit, albeit a ``conceptual family'' as opposed to a
``true non-family,'' in order to prevent the application of the
ordinance.\readingfootnote{3}{The distinction between ``conceptual'' or
``non-traditional'' families and true non-families may well be a distinction
without a difference, the distinction resting in speculation and sterotypical
presumptions. Further, recognition of the conceptual family suffers from the
defect of commanding inquiry into who are the users rather than focusing on the
use itself. \emph{See generally} Note, \emph{City of Santa Barbara v. Adamson:
An Associational Right of Privacy and the End of Family Zones}, 69 Calif. L.
Rev. 1052, 1068--70 (1981).}

The fallacy in defendants' syllogism is that the stipulated facts do not compel
the conclusion that defendants are living as a family. A man and woman living
together, sharing pleasures and certain responsibilities, does not per se
constitute a family in even the conceptual sense. To approximate a family
relationship, there must exist a commitment to a permanent relationship and a
perceived reciprocal obligation to support and to care for each other. Only when
these characteristics are present can the conceptual family, perhaps, equate
with the traditional family. In a traditional family, certain of its inherent
attributes arise from the legal relationship of the family members. In a
non-traditional family, those same qualities arise in fact, either by explicit
agreement or by tacit understanding among the parties.

While the stipulated facts could arguably support an inference by the trial
court that defendants and their children comprised a non-traditional family,
they do not compel that inference.\ldots We assume, arguendo, that the sole
basis for the judgment entered by the trial court was that defendants were not
related by blood, marriage or adoption, as required by Ladue's ordinance.

We first consider whether the ordinance violates any federally protected rights
of the defendants. Generally, federal court decisions hold that a zoning
classification based upon a biological or a legal relationship among household
members is justifiable under constitutional police powers to protect the public
health, safety, morals or welfare of the community. 

More specifically, the United States Supreme Court has developed a two-tiered
approach by which to examine legislation challenged as violative of the equal
protection clause. If the personal interest affected by the ordinance is
fundamental, ``strict scrutiny'' is applied and the ordinance is sustained only
upon a showing that the burden imposed is necessary to protect a compelling
governmental interest. If the ordinance does not contain a suspect class or
impinge upon a fundamental interest, the more relaxed ``rational basis'' test is
applied and the classification imposed by the ordinance is upheld if any facts
can reasonably justify it. Defendants urge this court to recognize that their
interest in choosing their own living arrangement inexorably involves their
fundamental rights of freedom of association and of privacy.

In \textit{Village of Euclid v. Ambler Realty Co.}, 272 U.S. 365 (1926) and in
\textit{Nectow v. City of Cambridge}, 277 U.S. 183 (1928), the United States
Supreme Court also established the due process parameters of permissible
legislation. The ordinance in question must have a ``foundation in reason'' and
bear a ``substantial relation to the public health, the public morals, the
public safety or the public welfare in its proper sense.'' 

In the \textit{Village of Belle Terre v. Boraas}, 416 U.S. 1 (1974), the court
addressed a zoning regulation of the type at issue in this case. The court held
that the Village of Belle Terre ordinance involved no fundamental right, but was
typical of economic and social legislation which is upheld if it is reasonably
related to a permissible governmental objective. The challenged zoning ordinance
of the Village of Belle Terre defined family as:
\begin{quote}
One or more persons related by blood, adoption or marriage, living and cooking
together as a single housekeeping unit [or] a number of persons but not
exceeding two (2) living and cooking together as a single housekeeping unit
though not related by blood, adoption, or marriage\ldots.
\end{quote}

The court upheld the ordinance, reasoning that the ordinance constituted valid
land use legislation reasonably designed to maintain traditional family values
and patterns.

The importance of the family was reaffirmed in \textit{Moore v. City of East
Cleveland}, 431 U.S. 494 (1977), wherein the United States Supreme Court was
confronted with a housing ordinance which defined a ``family'' as only certain
closely related individuals. Consequently, a grandmother who lived with her son
and two grandsons was convicted of violating the ordinance because her two
grandsons were first cousins rather than brothers. The United States Supreme
Court struck down the East Cleveland ordinance for violating the freedom of
personal choice in matters of marriage and family life. The court distinguished
\textit{Belle Terre} by stating that the ordinance in that case allowed all
individuals related by blood, marriage or adoption to live together; whereas
East Cleveland, by restricting the number of related persons who could live
together, sought ``to regulate the occupancy of its housing by slicing deeply
into the family itself.'' The court pointed out that the institution of the
family is protected by the Constitution precisely because it is so deeply rooted
in the American tradition and that ``[o]urs is by no means a tradition limited
to respect for the bonds uniting the members of the nuclear family.'' 

Here, because we are dealing with economic and social legislation and not with a
fundamental interest or a suspect classification, the test of constitutionality
is whether the ordinance is reasonable and not arbitrary and bears a rational
relationship to a permissible state objective. ``[E]very line drawn by a
legislature leaves some out that might well have been included. That exercise of
discretion, however, is a legislative, not a judicial, function.'' 

Ladue has a legitimate concern with laying out guidelines for land use addressed
to family needs. ``It is ample to lay out zones where family values, youth
values, and the blessings of quiet seclusion and clean air make the area a
sanctuary for people.'' The question of whether Ladue could have chosen more
precise means to effectuate its legislative goals is immaterial. Ladue's zoning
ordinance is rationally related to its expressed purposes and violates no
provisions of the Constitution of the United States. Further, defendants'
assertion that they have a constitutional right to share their residence with
whomever they please amounts to the same argument that was made and found
unpersuasive by the court in \textit{Belle Terre}.

We next consider whether the Ladue ordinance violates any rights of defendants
protected by the Missouri Constitution. \ldots

For purposes of its zoning code, Ladue has in precise language defined the term
family. It chose the definition which comports with the historical and
traditional notions of family; namely, those people related by blood, marriage
or adoption. That definition of family has been upheld in numerous Missouri
decisions. See, e.g., \emph{London v. Handicapped Facilities Board of St.
Charles County}, 637 S.W.2d 212 (Mo. App.1982) (group home not a ``family'' as
used in restrictive covenant); \emph{Feely v. Birenbaum}, 554 S.W.2d 432
(Mo.App.1977) (two unrelated males not a ``family'' as used in restrictive
covenant); \emph{Cash v. Catholic Diocese}, 414 S.W.2d 346 (Mo.App. 1967) (nuns
not a ``family'' as used in a restrictive covenant).

Decisions from other state jurisdictions have addressed identical constitutional
challenges to zoning ordinances similar to the ordinance in the instant case.
The reviewing courts have upheld their respective ordinances on the ground that
maintenance of a traditional family environment constitutes a reasonable basis
for excluding uses that may impair the stability of that environment and erode
the values associated with traditional family
life.\readingfootnote{4}{\textit{See, e.g.}, \emph{City of White Plains v.
Ferraioli}, 34 N.Y.2d 300, 357 N.Y.S.2d 449, 313 N.E.2d 756 (1974) (married
couple, their two children and 10 foster children not a family under city's
ordinance); \emph{Rademan v. City and County of Denver}, 186 Colo. 250, 526 P.2d
1325 (1974) (two married couples living as a ``communal family'' not a family);
\emph{Town of Durham v. White Enterprises, Inc.}, 115 N.H. 645, 348 A.2d 706
(1975) (student renters not a family); \emph{Prospect Gardens Convalescent Home,
Inc. v. City of Norwalk}, 32 Conn.Sup. 214, 347 A.2d 637 (1975) (nursing home
employees living together not a family). See generally Annot., 12 A.L.R. 4th 238
(1985). A number of jurisdictions have found restrictive zoning ordinances
invalid. \textit{See, e.g.}, \emph{City of Des Plaines v. Trottner}, 34 Ill.2d
432, 216 N.E.2d 116 (1970) (ordinance with restrictive definition of family
violates authority delegated by state legislature in the enabling statute);
\emph{City of Santa Barbara v. Adamson}, 27 Cal.3d 123, 164 Cal. Rptr. 539, 610
P.2d 436 (1982) (zoning ordinance limiting the number of unrelated persons who
could live together, but not related persons, did not further legislative
goals); \emph{Charter Township of Delta v. Dinolfo}, 419 Mich. 253, 351 N.W.2d
831 (1984) (restrictive definition of family not rationally related to achieving
township's goals).}

The essence of zoning is selection; and, if it is not invidious or
discriminatory against those not selected, it is proper. There is no doubt that
there is a governmental interest in marriage and in preserving the integrity of
the biological or legal family. There is no concomitant governmental interest in
keeping together a group of unrelated persons, no matter how closely they
simulate a family. Further, there is no state policy which commands that groups
of people may live under the same roof in any section of a municipality they
choose.

The stated purpose of Ladue's zoning ordinance is the promotion of the health,
safety, morals and general welfare in the city. Whether Ladue could have adopted
less restrictive means to achieve these same goals is not a controlling factor
in considering the constitutionality of the zoning ordinance. Rather, our focus
is on whether there exists some reasonable basis for the means actually
employed. In making such a determination, if any state of facts either known or
which could reasonably be assumed is presented in support of the ordinance, we
must defer to the legislative judgment. We find that Ladue has not acted
arbitrarily in enacting its zoning ordinance which defines family as those
related by blood, marriage or adoption. Given the fact that Ladue has so defined
family, we defer to its legislative judgment.

The judgment of the trial court is affirmed.

