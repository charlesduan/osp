\captionedgraphic{zoning-img003}{Bob Bawell, Pond at St. Louis Country Club
(Ladue, Mo.), Oct. 28, 2012, BY-NC. Seventy percent of Ladue's acreage is
comprised of open space.}


\reading{City of Ladue, Missouri, \textit{Comprehensive Plan Update}}
\readingcite{September 27, 2006,
\url{https://web.archive.org/web/20161201182531/http://www.cityofladue-mo.gov/mm/files/LadueComprehensivePlan.pdf}}

In 1936, several villages officially consolidated as the City of Ladue. At the
time it was the largest municipality in St. Louis County, with 4,553 acres of
land. Its first comprehensive plan, the \textit{Preliminary Report Upon a City
Plan}, was completed in 1939.\ldots The plan articulated the following
imperative which is equally applicable today:
\begin{quote}
It should be recognized that cities now are judged more by the character or
quality than they are by their size. This factor will be increasingly important
in the future with the entire country approaching a stabilized population. The
areas that will grow are those that provide desirable living conditions and
reasonable tax rates, and such areas will probably grow at the expense of some
other area having less favorable conditions. Thus the protection and
perpetuation of the present advantages are not only essential for the welfare of
the citizens, but are important measures of insuring continued healthy growth.
\end{quote}

\ldots Large residential lots predominated, with 13\% of all residences situated
on lots of at least five acres. The plan noted ``no other large suburban town in
the St. Louis region contains such a low population density or such a spacious
character of development.''\ldots

In accordance with the residential character objective, the 1939 plan proposed
five residential districts with largely overlapping uses, but with differences
in lot area and yard regulations. Permitted minimum lot sizes ranged from 10,000
square feet to three acres. Industry was confined to grandfathered areas. The
commercial district was expanded to only 15.2 acres with a neighborhood focus,
and this was deemed adequate for the target population of 10,000, given the fact
that commercial areas were available in adjoining communities.

Significantly, the ordinance did not make provision for apartments. The plan was
clear and consistent regarding the Commission's residential character objective.

[From the 1939 report: ``The opening of any section of the city for this use
would invite speculation, result in undue concentration of population, and make
it extremely difficult to prevent the spreading of this use throughout the
entire city. Apartment development would especially overburden the school
facilities, which are now adequate and have been planned for a continuation of
the present type of development. If apartment construction would be permitted in
the City of Ladue, it would enhance the value of the property of a few
individual owners, but, on the other hand, it would seriously depreciate
surrounding property, overtax school and sewer systems, and necessitate many
additional governmental services, all of which would unduly increase taxes\ldots
.'']

\ldots Ladue's character can be described as follows:
\begin{itemize}
\item ``Spacious'' (an attribute that was already defined in the City's 1939
plan)

\item ``Spacious residential character'' (as stated by the City's first Zoning
Commission)


\item A substantial legacy of fine estates, large homes, and elegant cottages

\item Predominant single family residential land use

\item Rolling hills

\item Countryside setting overlain with an extensive blanket of mature
vegetation

\item Architectural quality and diversity

\item Contained commercial areas

\item A network of old country-type roads that frame and help to define the
city's historic roots

\item A demographically concentrated community of civically prominent and active
residents


\item A multigenerational family heritage

\item Premium land values
\end{itemize}

[The report notes that Ladue, like most inner ring communities in the St. Louis
region, is shrinking, but not by very much.] Even with substantial demographic
shifts in St. Louis County that result in slow growth, the County is expected to
retain its central position of economic power both within the region as well as
in the State of Missouri. Approximately half of the jobs in the entire St. Louis
region are located in St. Louis County. Moreover, considerable wealth is
concentrated here, where one-fourth of all state sales tax revenue and over
one-third of all income tax revenue are generated. This is despite the fact that
the county represents only 19\% of the state's population. Its disproportionate
role in the state's income tax base results directly from a high concentration
of affluent households. Given the county's continued economic prominence in the
region as well as the sustained affluence of county residents in general, Ladue
seems to be particularly well positioned to retain its role as one of the
leading affluent cities not only within the county but also in the region and
the entire state\ldots .

\readinghead{1. Issues}

\begin{itemize}
\item The need to retain Ladue's existing housing character and general
densities as infill occurs.
\item The challenge of infills built to the maximum allowable
footprint---``McMansions''---which are frequently out of scale to surrounding
structures, negatively affect the visual quality of the blockface, and reduce
the open space and landscapes that are such an important part of Ladue's
character.
\item The desire of older residents to have downsized high-end housing options
available in Ladue, and the nature of such housing\ldots .
\item The need to maintain existing retail areas at present levels of
development.
\item The corresponding need for commercial development within existing
commercial districts as a tax-generating entity to meet rising municipal
costs\ldots .
\end{itemize}


\readinghead{A. Goals and Objectives}

\ldots

\readinghead{1. Maintain, Preserve and Improve the City's Present Residential
Character Within Already-Developed Areas.}

\begin{enumerate}
\item[a.] Maintain present low densities within already-developed areas to
preserve the characteristic of spaciousness.

\item[b.] Guide and direct land use activity within the estate residential
districts to retain their position of visual prominence in the City's housing
stock.

\item[c.] Preserve Ladue's predominantly single-family characteristics in
existing neighborhoods and developments.

\item[d.] Promote architectural quality and diversity.

\item[e.] Preserve and foster the City's countryside setting of rolling hills,
mature trees and extensive vegetation.\ldots
\end{enumerate}

\paragraph{Downsized Luxury Housing Opportunities} The demand for downsized
luxury housing in Ladue appears to be increasing, based on comments heard from
Ladue residents as well as by general market trends and regional development
activity. The City recognizes the need to consider this type of housing for
residents who seek it and who prefer to continue residing in the City rather
than move to another community. However, the City also recognizes the need to
maintain its present low-density estate and high-end residential character.
Accordingly, Ladue may encourage development of such housing within the
following parameters:
\begin{itemize}
\item It should not result in a net increase in unit density from the site's
present zoning\ldots .
\end{itemize}

\readinghead{Zoning}

\ldots The City has had a carefully developed and strictly enforced zoning
ordinance since 1938 with a major emphasis on estate and high-end residential
patterns that reinforce, sustain, and further its unique residential character.
To that end, all other zoning categories are intended to complement and support
rather than compete with quality residential development, which comprises
approximately 97\% of the City's total land area.

\captionedgraphic[height=4in]{zoning-img004}{Ladue Zoning map (Kuhlmann
Design Group).}

\paragraph{``A'' Residential District} The ``A'' residential district is a
visually prominent land use form in Ladue. It is the framework for the extensive
development of estates that over the years have come to form the backbone of the
City's residential makeup.\ldots This district contains a 3-acre minimum lot
area (130,680 s.f.) with front, side and rear yard distances of 75 feet, 50
feet, and 50 feet respectively. Minimum required frontage is 150 feet. Required
minimum lot width is 200 feet. Maximum building area is 15,000 square feet,
absent a special use permit.

\paragraph{``B'' Residential District} This district requires a 1.8-acre (78,408
s.f.) minimum lot area with front, side, and rear yard distances of 50 feet
each. Frontage minimum is 135 feet, and minimum lot width is 180 feet. Maximum
building area is 15,000 square feet. The ``B'' District, coupled with the ``A''
District, together comprise the most prominent land use forms in the city.

\paragraph{``C'' Residential District} The ``C'' residential district requires
a lot area minimum of 30,000 square feet. Front, side and rear yard distances
are 50 feet, 10 feet/10\% of lot width up to 20 feet and 30 feet respectively.
Minimum lot frontage is 90 feet, with minimum required lot width of 120 feet.
Building area maximum is 15,000 square feet.

\paragraph{``D'' Residential District} This district requires lots of no less
than 15,000 square feet with front, side and rear yard distances of 40 feet, 10
feet/10\% of lot width up to 15 feet and 30 feet respectively. Minimum required
frontage is 55 feet, with minimum required lot width of 75 feet.

\paragraph{``E'' Residential District} ``E'' residential is the smallest
residential district in Ladue. It requires lots of no less than 10,000 square
feet. Required front, side and rear yard distances are 40 feet, 10 feet and 30
feet respectively. Minimum required lot frontage is 50 feet, with a required
minimum lot width of 75 feet. 

\paragraph{``E-1'' Residential District} This district requires lots of not less
than 10,000 square feet, with required front, side and rear yards of 25, 10, and
30 respectively. Minimum required frontage is 50 feet with a minimum lot width
of 70 feet\ldots .

\paragraph{``F'' Floodplain District} Ladue's regulations for the Flood Plain
district prohibit construction, reconstruction or alterations to buildings
within its boundaries, except in conformity with the City's Flood Plain
Ordinance. \ldots

\paragraph{``G'' Commercial District}\ldots Ladue's commercial district
regulations permit the following uses: Banks (drive-in facilities are not
allowed except as a Special Use), barbershops, beauty parlors, offices including
medical/dental, parks, restaurants (no drive-in facilities or outside seating
except by Special Use), and retail businesses (except automotive sales)\ldots .

\paragraph{``H'' Industrial District} Ladue's single remaining industrial
district is located at the old Rock Hill Quarry site, which has been operating
as a landfill\ldots .

Permitted uses in the Industrial district include: Any commercial use (per
above); light manufacturing not considered a nuisance because of noise, odors,
dust, gases, smoke, vibration or other factors; and enclosed storage.



\ldots.

\readinghead{Future Land Use Plan}

\ldots
The city is already completely developed, with only one large additional
underdeveloped site available (the Landfill), totaling approximately 64 acres.
Although this site is not recommended for residential development, a small
portion of land to its immediate north is already so designated and might be
appropriately considered for creative residential uses. \ldots

[T]here is a growing market for the replacement of existing homes with new
structures, driven by buyers who prefer larger rooms and additional storage
space that new homes can provide. The elevations and footprints of these infills
often dwarf not only their own lots but adjoining property as well. They can
also negatively affect a larger area when their mass is sufficient to loom over
the entire block face. In no residential area is this more potentially harmful
than in the very small-lot district (``E'') with its 10,000 square-foot minimum.
Here, the City should discourage the use of variances from historic front, side
and rear yard requirements, as well as elevations that are out-of-scale to
surrounding buildings.

[Because Ladue is presently successful, the Plan recommends only minor changes,
including tightening the standards for new construction to make sure it's
attractive and limiting the grant of variances, discussed further in the next
Part of the materials. To deal with the McMansions problem, the proposal would
focus on the size and height of a building when viewed from the curb,
``emphasizing narrower and deeper designs rather than taller and wider
configurations.'' The plan would also ``[p]romote the limited development of
downsized luxury housing with no net increase in existing densities. Downsized
Luxury Housing is defined as a single-family owner- occupied unit either with or
without common walls, 1-3 person occupancy within a reduced living area, and
with sufficient elements of architectural detail, craftsmanship, and character
to make it both elegant and uniquely personal.'']

