\reading{Euclid v. Ambler Realty Co.}

\readingcite{272 U.S. 365 (1926)}

\opinion \textsc{Mr. Justice Sutherland} delivered the opinion of the Court.

The Village of Euclid is an Ohio municipal corporation. It adjoins and
practically is a suburb of the City of Cleveland. Its estimated population is
between 5,000 and 10,000, and its area from twelve to fourteen square miles, the
greater part of which is farmlands or unimproved acreage. It lies, roughly, in
the form of a parallelogram measuring approximately three and one-half miles
each way. East and west it is traversed by three principal highways: Euclid
Avenue, through the southerly border, St. Clair Avenue, through the central
portion, and Lake Shore Boulevard, through the northerly border in close
proximity to the shore of Lake Erie. The Nickel Plate railroad lies from 1,500
to 1,800 feet north of Euclid Avenue, and the Lake Shore railroad 1,600 feet
farther to the north. The three highways and the two railroads are substantially
parallel.

Appellee is the owner of a tract of land containing 68 acres, situated in the
westerly end of the village, abutting on Euclid Avenue to the south and the
Nickel Plate railroad to the north. Adjoining this tract, both on the east and
on the west, there have been laid out restricted residential plats upon which
residences have been erected.

On November 13, 1922, an ordinance was adopted by the Village Council
establishing a comprehensive zoning plan for regulating and restricting the
location of trades, industries, apartment houses, two-family houses, single
family houses, etc., the lot area to be built upon, the size and height of
buildings, etc.

The entire area of the village is divided by the ordinance into six classes of
use districts, denominated U-1 to U-6, inclusive; three classes of height
districts, denominated H-1 to H-3, inclusive, and four classes of area
districts, denominated A-1 to A-4, inclusive. The use districts are classified
in respect of the buildings which may be erected within their respective limits,
as follows: U-1 is restricted to single family dwellings, public parks, water
towers and reservoirs, suburban and interurban electric railway passenger
stations and rights of way, and farming, noncommercial greenhouse nurseries and
truck gardening; U-2 is extended to include two-family dwellings; U-3 is further
extended to include apartment houses, hotels, churches, schools, public
libraries, museums, private clubs, community center buildings, hospitals,
sanitariums, public playgrounds and recreation buildings, and a city hall and
courthouse; U-4 is further extended to include banks, offices, studios,
telephone exchanges, fire and police stations, restaurants, theatres and moving
picture shows, retail stores and shops, sales offices, sample rooms, wholesale
stores for hardware, drugs and groceries, stations for gasoline and oil (not
exceeding 1,000 gallons storage) and for ice delivery, skating rinks and dance
halls, electric substations, job and newspaper printing, public garages for
motor vehicles, stables and wagon sheds (not exceeding five horses, wagons or
motor trucks) and distributing stations for central store and commercial
enterprises; U-5 is further extended to include billboards and advertising signs
(if permitted), warehouses, ice and ice cream manufacturing and cold storage
plants, bottling works, milk bottling and central distribution stations,
laundries, carpet cleaning, dry cleaning and dyeing establishments, blacksmith,
horseshoeing, wagon and motor vehicle repair shops, freight stations, street car
barns, stables and wagon sheds (for more than five horses, wagons or motor
trucks), and wholesale produce markets and salesrooms; U-6 is further extended
to include plants for sewage disposal and for producing gas, garbage and refuse
incineration, scrap iron, junk, scrap paper and rag storage, aviation fields,
cemeteries, crematories, penal and correctional institutions, insane and feeble
minded institutions, storage of oil and gasoline (not to exceed 25,000 gallons),
and manufacturing and industrial operations of any kind other than, and any
public utility not included in, a class U-1, U-2, U-3, U-4 or U-5 use. There is
a seventh class of uses which is prohibited altogether.

Class U-1 is the only district in which buildings are restricted to those
enumerated. In the other classes, the uses are cumulative; that is to say, uses
in class U-2 include those enumerated in the preceding class, U-1; class U-3
includes uses enumerated in the preceding classes, U-2 and U-1, and so on. In
addition to the enumerated uses, the ordinance provides for accessory uses, that
is, for uses customarily incident to the principal use, such as private garages.
Many regulations are provided in respect of such accessory uses.

The height districts are classified as follows: In class H-1, buildings are
limited to a height of two and one-half stories or thirty-five feet; in class
H-2, to four stories or fifty feet; in class H-3, to eighty feet. To all of
these, certain exceptions are made, as in the case of church spires, water
tanks, etc.

The classification of area districts is: in A-1 districts, dwellings or
apartment houses to accommodate more than one family must have at least 5,000
square feet for interior lots and at least 4,000 square feet for corner lots; in
A-2 districts, the area must be at least 2,500 square feet for interior lots,
and 2 000 square feet for corner lots; in A-3 districts, the limits are 1,250
and 1,000 square feet, respectively; in A-4 districts, the limits are 900 and
700 square feet, respectively. The ordinance contains, in great variety and
detail, provisions in respect of width of lots, front, side and rear yards, and
other matters, including restrictions and regulations as to the use of bill
boards, sign boards and advertising signs\ldots .

Appellee's tract of land comes under U-2, U-3 and U-6. The first strip of 620
feet immediately north of Euclid Avenue falls in class U-2, the next 130 feet to
the north, in U-3, and the remainder in U-6. The uses of the first 620 feet,
therefore, do not include apartment houses, hotels, churches, schools, or other
public and semi-public buildings, or other uses enumerated in respect of U-3 to
U-6, inclusive. The uses of the next 130 feet include all of these, but exclude
industries, theatres, banks, shops, and the various other uses set forth in
respect of U-4 to U-6, inclusive. 

Annexed to the ordinance, and made a part of it, is a zone map showing the
location and limits of the various use, height and area districts, from which it
appears that the three classes overlap one another; that is to say, for example,
both U-5 and U-6 use districts are in A-4 area districts, but the former is in
H-2 and the latter in H-3 height districts\ldots .

The ordinance is assailed on the grounds that it is in derogation of \S~1 of the
Fourteenth Amendment to the Federal Constitution in that it deprives appellee of
liberty and property without due process of law and denies it the equal
protection of the law, and that it offends against certain provisions of the
Constitution of the State of Ohio. The prayer of the bill is for an injunction
restraining the enforcement of the ordinance and all attempts to impose or
maintain as to appellee's property any of the restrictions, limitations or
conditions\ldots .

The bill alleges that the tract of land in question is vacant and has been held
for years for the purpose of selling and developing it for industrial uses, for
which it is especially adapted, being immediately in the path of progressive
industrial development; that, for such uses, it has a market value of about
\$10,000 per acre, but if the use be limited to residential purposes, the market
value is not in excess of \$2,500 per acre; that the first 200 feet of the
parcel back from Euclid Avenue, if unrestricted in respect of use, has a value
of \$150 per front foot, but if limited to residential uses, and ordinary
mercantile business be excluded therefrom, its value is not in excess of \$50
per front foot.

It is specifically averred that the ordinance attempts to restrict and control
the lawful uses of appellee's land so as to confiscate and destroy a great part
of its value; that it is being enforced in accordance with its terms; that
prospective buyers of land for industrial, commercial and residential uses in
the metropolitan district of Cleveland are deterred from buying any part of this
land because of the existence of the ordinance and the necessity thereby
entailed of conducting burdensome and expensive litigation in order to vindicate
the right to use the land for lawful and legitimate purposes; that the ordinance
constitutes a cloud upon the land, reduces and destroys its value, and has the
effect of diverting the normal industrial, commercial and residential
development thereof to other and less favorable locations.

The record goes no farther than to show, as the lower court found, that the
normal and reasonably to be expected use and development of that part of
appellee's land adjoining Euclid Avenue is for general trade and commercial
purposes, particularly retail stores and like establishments, and that the
normal, and reasonably to be expected use and development of the residue of the
land is for industrial and trade purposes. Whatever injury is inflicted by the
mere existence and threatened enforcement of the ordinance is due to
restrictions in respect of these and similar uses; to which perhaps should be
added---if not included in the foregoing---restrictions in respect of apartment
houses. Specifically, there is nothing in the record to suggest that any damage
results from the presence in the ordinance of those restrictions relating to
churches, schools, libraries and other public and semi-public buildings. It is
neither alleged nor proved that there is, or may be, a demand for any part of
appellee's land for any of the last named uses, and we cannot assume the
existence of facts which would justify an injunction upon this record in respect
of this class of restrictions.\ldots

Building zone laws are of modern origin. They began in this country about
twenty-five years ago. Until recent years, urban life was comparatively simple;
but with the great increase and concentration of population, problems have
developed, and constantly are developing, which require, and will continue to
require, additional restrictions in respect of the use and occupation of private
lands in urban communities. Regulations the wisdom, necessity and validity of
which, as applied to existing conditions, are so apparent that they are now
uniformly sustained a century ago, or even half a century ago, probably would
have been rejected as arbitrary and oppressive. Such regulations are sustained,
under the complex conditions of our day, for reasons analogous to those which
justify traffic regulations, which, before the advent of automobiles and rapid
transit street railways, would have been condemned as fatally arbitrary and
unreasonable. And in this there is no inconsistency, for, while the meaning of
constitutional guaranties never varies, the scope of their application must
expand or contract to meet the new and different conditions which are constantly
coming within the field of their operation. In a changing world, it is
impossible that it should be otherwise. But although a degree of elasticity is
thus imparted not to the meaning, but to the application of constitutional
principles, statutes and ordinances which, after giving due weight to the new
conditions, are found clearly not to conform to the Constitution of course must
fall.

The ordinance now under review, and all similar laws and regulations, must find
their justification in some aspect of the police power, asserted for the public
welfare. The line which in this field separates the legitimate from the
illegitimate assumption of power is not capable of precise delimitation. It
varies with circumstances and conditions. A regulatory zoning ordinance, which
would be clearly valid as applied to the great cities, might be clearly invalid
as applied to rural communities. In solving doubts, the maxim \textit{sic utere
tuo ut alienum non laedas}, which lies at the foundation of so much of the
common law of nuisances, ordinarily will furnish a fairly helpful [clue]. And
the law of nuisances likewise may be consulted not for the purpose of
controlling, but for the helpful aid of its analogies in the process of
ascertaining the scope of, the power. Thus, the question whether the power
exists to forbid the erection of a building of a particular kind or for a
particular use, like the question whether a particular thing is a nuisance, is
to be determined not by an abstract consideration of the building or of the
thing considered apart, but by considering it in connection with the
circumstances and the locality. A nuisance may be merely a right thing in the
wrong place---like a pig in the parlor instead of the barnyard. If the validity
of the legislative classification for zoning purposes be fairly debatable, the
legislative judgment must be allowed to control.

There is no serious difference of opinion in respect of the validity of laws and
regulations fixing the height of buildings within reasonable limits, the
character of materials and methods of construction, and the adjoining area which
must be left open, in order to minimize the danger of fire or collapse, the
evils of over-crowding, and the like, and excluding from residential sections
offensive trades, industries and structures likely to create nuisances. 

Here, however, the exclusion is in general terms of all industrial
establishments, and it may thereby happen that not only offensive or dangerous
industries will be excluded, but those which are neither offensive nor dangerous
will share the same fate. But this is no more than happens in respect of many
practice-forbidding laws which this Court has upheld although drawn in general
terms so as to include individual cases that may turn out to be innocuous in
themselves. The inclusion of a reasonable margin to insure effective enforcement
will not put upon a law, otherwise valid, the stamp of invalidity. Such laws may
also find their justification in the fact that, in some fields, the bad fades
into the good by such insensible degrees that the two are not capable of being
readily distinguished and separated in terms of legislation. In the light of
these considerations, we are not prepared to say that the end in view was not
sufficient to justify the general rule of the ordinance, although some
industries of an innocent character might fall within the proscribed class. It
cannot be said that the ordinance in this respect ``passes the bounds of reason
and assumes the character of a merely arbitrary fiat.'' Moreover, the
restrictive provisions of the ordinance in this particular may be sustained upon
the principles applicable to the broader exclusion from residential districts of
all business and trade structures, presently to be discussed.

It is said that the Village of Euclid is a mere suburb of the City of Cleveland;
that the industrial development of that city has now reached and in some degree
extended into the village and, in the obvious course of things, will soon absorb
the entire area for industrial enterprises; that the effect of the ordinance is
to divert this natural development elsewhere, with the consequent loss of
increased values to the owners of the lands within the village borders. But the
village, though physically a suburb of Cleveland, is politically a separate
municipality, with powers of its own and authority to govern itself as it sees
fit within the limits of the organic law of its creation and the State and
Federal Constitutions. Its governing authorities, presumably representing a
majority of its inhabitants and voicing their will, have determined not that
industrial development shall cease at its boundaries, but that the course of
such development shall proceed within definitely fixed lines. If it be a proper
exercise of the police power to relegate industrial establishments to localities
separated from residential sections, it is not easy to find a sufficient reason
for denying the power because the effect of its exercise is to divert an
industrial flow from the course which it would follow, to the injury of the
residential public if left alone, to another course where such injury will be
obviated. It is not meant by this, however, to exclude the possibility of cases
where the general public interest would so far outweigh the interest of the
municipality that the municipality would not be allowed to stand in the way.

We find no difficulty in sustaining restrictions of the kind thus far reviewed.
The serious question in the case arises over the provisions of the ordinance
excluding from residential districts, apartment houses, business houses, retail
stores and shops, and other like establishments. This question involves the
validity of what is really the crux of the more recent zoning legislation,
namely, the creation and maintenance of residential districts, from which
business and trade of every sort, including hotels and apartment houses, are
excluded. 

\ldots The matter of zoning has received much attention at the hands of
commissions and experts, and the results of their investigations have been set
forth in comprehensive reports. These reports, which bear every evidence of
painstaking consideration, concur in the view that the segregation of
residential, business, and industrial buildings will make it easier to provide
fire apparatus suitable for the character and intensity of the development in
each section; that it will increase the safety and security of home life;
greatly tend to prevent street accidents, especially to children, by reducing
the traffic and resulting confusion in residential sections; decrease noise and
other conditions which produce or intensify nervous disorders; preserve a more
favorable environment in which to rear children, etc. With particular reference
to apartment houses, it is pointed out that the development of detached house
sections is greatly retarded by the coming of apartment houses, which has
sometimes resulted in destroying the entire section for private house purposes;
that, in such sections, very often the apartment house is a mere parasite,
constructed in order to take advantage of the open spaces and attractive
surroundings created by the residential character of the district. Moreover, the
coming of one apartment house is followed by others, interfering by their height
and bulk with the free circulation of air and monopolizing the rays of the sun
which otherwise would fall upon the smaller homes, and bringing, as their
necessary accompaniments, the disturbing noises incident to increased traffic
and business, and the occupation, by means of moving and parked automobiles, of
larger portions of the streets, thus detracting from their safety and depriving
children of the privilege of quiet and open spaces for play, enjoyed by those in
more favored localities---until, finally, the residential character of the
neighborhood and its desirability as a place of detached residences are utterly
destroyed. Under these circumstances, apartment houses, which in a different
environment would be not only entirely unobjectionable but highly desirable,
come very near to being nuisances.

If these reasons, thus summarized, do not demonstrate the wisdom or sound policy
in all respects of those restrictions which we have indicated as pertinent to
the inquiry, at least the reasons are sufficiently cogent to preclude us from
saying, as it must be said before the ordinance can be declared
unconstitutional, that such provisions are clearly arbitrary and unreasonable,
having no substantial relation to the public health, safety, morals, or general
welfare. 

It is true that when, if ever, the provisions set forth in the ordinance in
tedious and minute detail come to be concretely applied to particular premises,
including those of the appellee, or to particular conditions, or to be
considered in connection with specific complaints, some of them, or even many of
them, may be found to be clearly arbitrary and unreasonable. But where the
equitable remedy of injunction is sought, as it is here, not upon the ground of
a present infringement or denial of a specific right, or of a particular injury
in process of actual execution, but upon the broad ground that the mere
existence and threatened enforcement of the ordinance, by materially and
adversely affecting values and curtailing the opportunities of the market,
constitute a present and irreparable injury, the court will not scrutinize its
provisions, sentence by sentence, to ascertain by a process of piecemeal
dissection whether there may be, here and there, provisions of a minor
character, or relating to matters of administration, or not shown to contribute
to the injury complained of, which, if attacked separately, might not withstand
the test of constitutionality. In respect of such provisions, of which specific
complaint is not made, it cannot be said that the landowner has suffered or is
threatened with an injury which entitles him to challenge their
constitutionality. 

\ldots Under these circumstances, therefore, it is enough for us to determine,
as we do, that the ordinance, in its general scope and dominant features, so far
as its provisions are here involved, is a valid exercise of authority, leaving
other provisions to be dealt with as cases arise directly involving them.

