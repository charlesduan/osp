\textit{Euclid} treated zoning as a legislative judgment deserving substantial
deference. \term[variance]{Variances} are more individualized decisions about
specific parcels,
and they raise key structural issues: How can an individualized determination
avoid arbitrariness? How should courts review these individualized
determinations---should they defer to zoning boards as much as they do with
overall zoning schemes?

Missouri law empowers city boards of adjustment, ``where there are practical
difficulties or unnecessary hardship in the way of carrying out the strict
letter of [a zoning ordinance], to vary or modify the application of\ldots such
ordinance\ldots so that the spirit of the ordinance shall be observed, public
safety and welfare secured and substantial justice done.'' \textsc{Mo. Rev.
Stat.} \S~89.090(3) (1998). This type of provision is common across the nation,
though there is some state-to-state variation. The basic requirements for a
variance in any state are (1) a showing of individualized hardship and (2) a
lack of interference with the basic goals of the zoning scheme. Both must be
shown; even substantial hardship is insufficient if granting a variance would do
significant harm to the purposes of the zoning. In such a case, only a
constitutional challenge or a federal law overriding local zoning could
potentially allow the proposed use.

Zoning authorities' basic hostility to variances is well expressed by the
\textsc{Missouri Municipal League, Planning and Zoning Procedures for Missouri
Municipalities} (Sept. 2004): 
\begin{quote}
The most common situation in which variances are sought is where a developer
divides his land into the greatest possible number of lots, barely meeting
minimum standards, and then seeks permission to create substandard lots out of
the remaining land. The subdivision regulations are intended to set forth
minimum standards for development, not maximums, and the intent of the
regulation is to use the remnants of land to increase lot sizes rather than
create substandard lots. When variances are granted allowing substandard lots,
it weakens the legal position of the city and its regulations and makes it
difficult to defend its subdivision standards.
\end{quote}

(While there is little systematic empirical evidence about actual board
practice, the litigated variance cases tend not to have this ``most common''
fact pattern.)

\paragraph{Procedure} Most jurisdictions have a formal process setting out the
deadlines and providing guidance to applicants on what they need to show to get
a variance. \textit{See, e.g.}, \textsc{St. Louis Board of Zoning Adjustment,
Citizen's Guide to the Board of Zoning Adjustment Variance Process} (n.d.). By
contrast, the city of Ladue has no formal variance procedure at all. Instead, an
applicant must seek a permit, and after the permit is denied, the City of Ladue
Building Department sends the applicant a formal denial letter with Zoning Board
of Adjustment instructions for an appeal. 



\defwebsite{quarterman-dollar-general}{
author=John S. Quarterman,
title=Dollar General (Teramore Development) @ GLPC 2013-08-26,
date=Sep 5 2013,
journal=On the LAKE Front,
url=http://www.l-a-k-e.org/blog/2013/09/dollar-general-teramore-development-glpc-2013-08-26.html,
}

\captionedgraphic{zoning-img008}{Zoning Hearing, Valdosta County, Georgia, by
John S. Quarterman, Aug. 26, 2013, CC-BY.
For a detailed recap of a zoning hearing and many more pictures, see
\protect\fullcite{quarterman-dollar-general}
\protect\clause{quarterman-dollar-general}.}

