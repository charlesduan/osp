\item Suppose a business will be in violation of the ADA if it doesn't install a
ramp, in violation of a setback requirement. Is it \textit{entitled} to a
variance under this guidance? What if the business should have known about the
problem before constructing its building? (In that case, the zoning authority is
also implicated---it shouldn't have approved any buildings that would violate
the ADA. \textit{See} U.S. Dep't of Justice, Civil Rts. Div., ADA
Standards for Accessible Design (2010),
\url{http://www.ada.gov/2010ADAstandards_index.htm}.) What considerations might
nonetheless justify denying the variance? What if the board argues that ramps
are ugly and will decrease the value of the area? What if the board has safety
concerns because the ramp will extend far enough to interfere with bicyclists?
The rule that ADA requires reasonable modifications to zoning laws may mean that
the standard requirement of exceptional and undue hardship to the property owner
isn't applicable. But another element of the test, detriment to the overall
value of the area, is relevant in determining whether a modification is
reasonable.

\item Variances usually preclude consideration of personal characteristics that
aren't inherent in the land. Where the entity seeking a variance is a business,
that question isn't particularly important---even if the business changes hands,
the next owner will need a ramp to make the store accessible. But suppose zoning
regulations require a particular elevation for residential beachfront property,
in order to address concerns about danger from flooding. A property owner uses a
wheelchair and wants a variance from the elevation requirement because otherwise
he won't be able to get into his house. Does the ADA require the variance?

