\reading{Richard H. Chused, \textit{Euclid's Historical Imagery}}

\readinghead{51 \textsc{Case W. Res. L. Rev.} 599 (2001)}

\ldots No one should be surprised that land use and urban planning emerged and
flowered in the 1920s. Chaos in America's developing urban centers,
unprecedented levels of immigration from Europe and migration from the southern
United States, burgeoning sales of automobiles, and development of new building
construction techniques generated enormous controversy during the end of the
nineteenth and beginning of the twentieth centuries. As American cities grew
like wildfire, cries of distress became common. Muckraking authors produced
``hit'' books reflecting upon widespread concern about the state of urban
America. From holding only about twenty-five percent of the nation's population
in 1870, urban areas held just over half only fifty years later. Between just
1905 and 1915, immigration increased the nation's population by more than ten
percent. With most of those arrivals settling in highly populated areas along
the coasts and industrial cities in the heartland, responding to immigration was
a major concern in urban America. The blare of urban life became a cacophony as
the number of registered automobiles passed the ten million mark in 1921.

\ldots The largely undeveloped Village of Euclid, just east of Cleveland, was
caught up in this wave of planning reforms. The Village of Euclid actually
adopted its first zoning ordinance in 1922, two years before the Commerce
Department published its final draft of the Standard Zoning Enabling Act. Euclid
followed in the footsteps of New York City, which adopted its first zoning
ordinance in 1916, two years after the New York state legislature adopted the
nation's first zoning enabling statute.\ldots

It should surprise no one that race, ethnicity, and poverty were on the minds of
those handling the dispute over Euclid's zoning scheme. The solidification of
the Jim Crow system from the end of Reconstruction through the 1920s is a
well-known story. Other startling events also brought racial and ethnic issues
to public attention on a regular basis. Race riots occurred in numerous cities
in the late nineteenth and early twentieth centuries. These were not like the
urban disturbances that began in Watts in 1965 and appeared repeatedly until
after the assassination of Reverend Martin Luther King, Jr. in 1968. In 1919
alone, for example, over twenty-five cities were faced with mobs of white people
destroying African-American neighborhoods and killing residents.\ldots Though
lynching of individuals or small groups of people peaked near the end of the
nineteenth century, urban mob killings more than made up for the decline in the
numbers of people strung up on trees individually or in small groups. The Ku
Klux Klan was a major political force at the time. Its members held elected
offices in a number of states during the first few decades of the twentieth
century. \ldots

In addition, opposition to immigration was fierce by the time Judge Westenhaver
decided Euclid. Acts restricting immigration were enacted in 1885, 1891, 1903,
1907, and 1917. The quota system, favoring those seeking admission from
northern Europe and severely limiting entry from other parts of Europe and the
rest of the world, was imposed by legislation passed in 1921 and 1924.
Immigration dropped dramatically after the last of these enactments was signed
into law. Fueled by racism and anti-semitism, and given intellectual cover by
Social Darwinism, many native-born whites saw themselves as the saviors of
culture and civilization\ldots .

When viewed in light of such a setting, the debate in \textit{Euclid} takes on
new meanings. It was not just a case about the ability of legislative bodies to
regulate property and contracts, but a debate about the sorts of social
forces---good, bad, and indifferent---that could legitimately be taken into
account by those elected to state legislatures\ldots .

By using a ``nuisance analogy''---the idea that single use zones were likely to
prevent land use conflicts---as the central feature of his argument, [Alfred
Bettman, leader of the National Conference on City Planning,] sidestepped the
intractable and circular debates\ldots about the dichotomy between the police
power, on the one hand, and takings or freedom of contract, on the other. \ldots

As Bettman himself noted in a paper he wrote while \textit{Euclid} was pending,
barring apartment buildings from residential zones was thought by many to be the
most troublesome feature of the typical planning ordinances. Responding to
claims that such zoning tactics were merely aesthetic controls and therefore
outside the police power, Bettman called upon telling imagery of middle and
upper class men protecting their children from moral risk to justify single
family residential zones:
\begin{quote}
[T]he man who seeks to place the home for his children in an orderly
neighborhood, with some open space and light and fresh air and quiet, is not
motivated so much by considerations of taste or beauty as by the assumption that
his children are likely to grow mentally, physically and morally more healthful
in such a neighborhood than in a disorderly, noisy, slovenly, blighted and
slum-like district.\ldots Disorderliness in the environment has as detrimental
an effect upon health and character as disorderliness within the house itself.
\end{quote}

In this passage, it becomes clear that use of the nuisance analogy also
permitted one other crucial step---the introduction of ``politely'' ugly
discourse. By putting the home/apartment dichotomy into the nuisance analogy,
Bettman could call forth a host of phrases well suited to convince the
conservative instincts of Supreme Court Justices that zoning was a positive
good. The moral strength of upper-class children was at risk, Bettman warned.
Keeping the kids away from a ``disorderly, noisy, slovenly, blighted and
slum-like district'' was the only protection.

\ldots Zoning rules, like many of the other moral reforms of the late nineteenth
and early twentieth centuries, were designed to significantly reduce the
likelihood that middle- and upper-class children would come into contact with
poor, immigrant, or black culture\ldots .

It was therefore possible, without ever mentioning race, immigration, or
tenement houses, to call upon other code words that had the same impact. \ldots

