\reading{The ADA and City Governments: Common Problems}
\readingcite{U.S. Department of Justice, Civil Rights Division, Disability
Rights Section, \url{https://archive.ada.gov/comprob.htm} (last updated Feb.~24,
2020)}

\readinghead{Common Problem:}

City governments may fail to consider reasonable modifications in local laws,
ordinances, and regulations that would avoid discrimination against individuals
with disabilities.

\readinghead{Result:}

Laws, ordinances, and regulations that appear to be neutral often adversely
impact individuals with disabilities. For example, where a municipal zoning
ordinance requires a set-back of 12 feet from the curb in the central business
district, installing a ramp to ensure access for people who use wheelchairs may
be impermissible without a variance from the city. People with disabilities are
therefore unable to gain access to businesses in the city.

\begin{quote}
\heregraphic{zoning-img009}
City zoning policies were changed to permit
this business to install a ramp at its entrance.
\end{quote}

\readinghead{Requirement:}

City governments are required to make reasonable modifications to policies,
practices, or procedures to prevent discrimination on the basis of disability.
Reasonable modifications can include modifications to local laws, ordinances,
and regulations that adversely impact people with disabilities. For example, it
may be a reasonable modification to grant a variance for zoning requirements and
setbacks. 

