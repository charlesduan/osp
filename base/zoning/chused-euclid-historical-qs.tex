\expected{chused-euclid-historical}

\item
Does Chused's account make you think differently about \textit{Euclid}? Suppose
the Court had ruled the other way, that zoning was an unwarranted interference
with property rights. How would our cities and suburbs look now?
\having{intro-covenants}{(Consider the role of restrictive covenants, presented
previously in this book.)}{(Consider this question again when you study
restrictive covenants, later in this book.)}{Restrictive covenants, not
discussed in this book, provide another mechanism for property owners to
negotiate over permissible uses---what role would they play?}

\defjrnart{fischel-economic-history}{
William A. Fischel, An Economic History of Zoning and a Cure for Its
Exclusionary Effects, 41 Urban Studies 317 (2004),
https://journals.sagepub.com/doi/abs/10.1080/0042098032000165271
}

\item
\Inline{fischel-economic-history}
puts a different emphasis on historical causes,
and asks why zoning became so much more restrictive over time.
\sentence{fischel-economic-history}. Fischel argues
that zoning developed, and then tightened its grip, because of homeowners' fears
that the value of their single largest asset was threatened by new
transportation technologies. The bus and truck came first, in the 1910s,
corresponding with the initial adoption of zoning. The development of the
interstate highway system in the 1960s then ``put suburban homeowners at risk
from value-reducing development in their neighborhoods and communities,''
causing them to support increasingly restrictive zoning. Zoning spread quickly
to suburbs and small towns (like Euclid itself), rather than being driven by the
well-known planners of the big cities. 

Before 1880, most people walked to work in American cities, and rich men tended
to live close to their jobs to avoid frustrating and time-consuming commutes.
Electric-powered streetcars then made it possible for urban workers to live in
residential areas, commuting to city jobs. As he notes, streetcar routes
exploded from 3,000 miles of horse-drawn routes in 1882 to 22,500 mostly
electrified miles in 1902. Developers built houses for the well-off workers who
could afford streetcar fares, and the rich began moving to the suburbs, but not
with zoning. Zoning wasn't yet needed: apartments and stores were located near
streetcar lines, but it was simple for homebuilders to avoid those areas by
building only a few blocks away from the tracks. Homebuilders and homeowners
also used political clout to keep streetcar lines from going through exclusively
residential areas.

But then, Fischel argues, trucks and buses became common, and the constraints
imposed on poorer people by streetcars diminished. It became profitable to sell
a vacant lot in a residential neighborhood to an industrial user or apartment
builder, who could expect easy access to all the resources of the city through
the new means of transportation. Restrictive covenants weren't enough to stem
the flow of intensive uses, because they usually covered only relatively small
areas of land, and restricted communities were vulnerable to development just on
the border. Instead of trying to buy up even bigger tracts of land, developers
began to support zoning, not because they trusted planners, but because they
wanted to ``induce homeowners to invest their savings in a large, undiversified
asset\ldots . As planning-historian Christine Boyer points out, zoning was seen
as a way to provide `an insurance policy that the single-family home owner's
investment would be protected in stable neighborhood communities\ldots .'\,''

The next development was political. Up to the first decade of the twentieth
century, suburban governments were routinely formed and then absorbed into the
expanding city. By the 1920s, however, suburbs became unwilling to give up their
independence, and unincorporated parts of surrounding counties became more
difficult for core cities to annex. Before zoning, Fischel contends, suburbs
regarded merger with the city, and the intrusion of city problems and costs, as
inevitable. As they grew, they needed more services, making the better-organized
city police, firefighting companies, and utilities seem more attractive. But
with zoning, suburbs determined that they could control their own growth and
fiscal destiny. Instead of merging with the city, suburbs began cooperating with
each other to provide water and other services that had previously only been
available from the central city---a pattern seen today in many St. Louis
suburbs.

People who live near where they work, Fischel posits, have to balance their
interests as homeowners with their interests as businesspeople, employers, or
employees---they are more likely to support growth than people who fear only
disruption of their living conditions from growth. Commuters, by contrast,
didn't vote where they worked, so they only voted based on the value of their
homes. Homeowners can't buy insurance against the risk that their homes will
become less valuable, and most homeowners can't diversify their assets because
they don't \textit{have} much in the way of assets other than a home. This makes
them anxious and politically active: ``They know that if things go bad in their
neighborhood, they will be stuck having paid a lot for an asset that they could
sell only at a loss. They can avoid the personal consequences of a school system
that has unexpectedly gone bad by moving, but they cannot avoid the financial
consequences. Potential buyers can see the declining test scores as well as
seller.'' As author Reihan Salan puts it, ``Renters might react to demographic
change with relative equanimity, knowing that even if it had negative
consequences, it wouldn't endanger their biggest investment. Homeowners felt
they couldn't afford not to panic.'' When demographic change nonetheless
arrived, the result, in the St. Louis suburbs and elsewhere, was ``round after
round of white flight, each one of which leaves a suburban ghetto in its wake.''
Reihan Salam, \textit{How the Suburbs Got Poor}, \textsc{Slate}, Sept. 4, 2014.

For the first fifty years of zoning, pro-development forces could win victories
in the suburbs---if one suburb resisted, another nearby might well be more
accommodating. However, Fischel argues, this changed in the 1970s, when the
suburbanization of employment and the gains of the civil rights movement changed
the political behavior of suburbs. The interstate highways of the 1960s enabled
jobs to move out to the suburbs, in ``industrial parks.'' Low-income workers
whose jobs had previously been in city centers now found that they needed to go
out to the suburbs to find work. More people, including poorer people, got
cars---up to 82\% of all households in 1970. With the ability to get to a job in
the city center less of a constraint, residential amenities such as schools
became far more important to homeowners, who became even more anxious and
insistent on keeping development away.\footnote{ Although much of the discourse
surrounding home values has to do with schools, there is no evidence that
state-level equalization of school funding, which makes property taxes less
important, has reduced exclusionary zoning. California equalized school finance
and imposed a limit on property taxes that meant that homeowners didn't need to
worry that low-income housing would increase their taxes, but exclusionary
zoning didn't diminish and even intensified.}

Meanwhile, civil rights laws barred overt discrimination, including informal
discrimination such as steering different races to different areas. While courts
were hostile to racial zoning, they accepted facially neutral economic
discrimination, which just happened to preserve racial lines. (Fischel points
out that nearly all-white states like Vermont and New Hampshire underwent the
same evolution towards increasingly restrictive zoning, suggesting that class
was independently sufficient to drive this change.) Suburban homeowners adopted
the rhetoric of environmentalism and demanded limits on growth and density,
restricting development for everyone, not just for low-income people. Forced to
choose between letting everyone in and letting no one in, they opted for no one.
Fischel concludes: ``The mottoes of no-growth, slow growth, managed growth, and
(currently) `smart growth' are all facially neutral watchwords which nonetheless
are effective substitutes for more selective means of keeping the poor out of
the suburbs.'' Changes in local government structure, such as environmental
impact statement requirements and the ``double veto'' structure in which larger
regional governments can block development but not force it, strengthened the
anti-growth forces' hand.

