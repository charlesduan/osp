\readingnote{Excerpts reprinted with permission. Prof.~Fischel, an economist,
studies zoning; he also sat on a zoning board for several years in order to
better understand its workings.}

\reading[Fischel, The Evolution of Zoning Since the 1980s]{William A. Fischel,
\textit{The Evolution of Zoning Since the 1980s: The Persistence of Localism}}
\readingcite{Draft of Sept.~2010,
\url{https://papers.ssrn.com/sol3/papers.cfm?abstract_id=1686009}, forthcoming
in \textsc{Property in Land and Other Resources} 259 (Daniel H. Cole \& Elinor
Ostrom eds., 2012)}

\ldots Two reflections about zoning boards might be useful to scholars. The
first is that all board members are put on edge by lawyers. This includes the
several lawyers who served on the board during my tenure. Having an attorney
make the presentation while the applicant sits in the back of the room (or
worse, fails to attend at all) makes board members assume that something is
fishy about the proposal. Less articulate but sincere presentation by principals
(or, for elaborate projects, their engineers and architects) are cut more slack
than their polished and practiced legal agents.

The other reflection is how much actually visiting the site in question matters.
Our board would hear applicants and then, in the week between the hearing and
the deliberation session, travel individually to the location of the proposed
project and tramp around the lot and the neighborhood. (Though its resident
population is only 10,000, Hanover is a busy employment center, and its land
area is the size of Boston, so locations were often unfamiliar.) Site visits
could change our views of the case enormously. An applicant showed charming
pictures of his antique-car hobby and sought a variance only to park some
storage trailers. A visit revealed that he actually harbored a private junkyard.
(Neighbors had not previously complained because the junkyard had been there
before their homes were built, and the owner was a nice guy.) A barn that was
proposed within a wetland setback turned out to be as high and dry as any
location in Hanover. (Wetland definitions do not actually require water to be
evident.)

I mention the importance of local knowledge because there is a literature on
zoning boards, most often by attorneys, that finds fault with their decisions.
Among the earlier and better known critiques was titled, ``The Zoning Board of
Adjustment: A Case Study in Misrule'' (Dukeminier and Stapleton 1962). A more
recent study was by an attorney who statistically examined variance decisions in
five New Hampshire towns, one of which was Hanover, during the years 1987-1992,
when I was on the zoning board. His chief finding, reported in high dudgeon, was
that variances are disproportionately granted if abutters do not object (Kent
1993, cited with similar studies in Ellickson and Been 2000, pp. 330-31). To
which most board members would say, privately and with palms up, ``Nu? Who knows
better whether the variance will have an adverse effect?'' The practice
illustrates the recurrence of an early, grass-roots approach to land use
regulation, which required nonconforming uses to obtain permission of local
property owners. The practice was struck down as unlawful delegation of the
police power in several early cases such as \emph{Eubank v. City of Richmond},
226 U.S. 137 (1912), but most local zoning boards informally operate as if it
were still in effect.

Mr. Kent, the New Hampshire critic of zoning boards (and himself a New Hampshire
lawyer), neglected to point out that four of the five towns in his sample have
administrative officers who could discourage applicants with weak cases
(Hanover's certainly did), but none of the other ``misrule-by-variance'' studies
worries much about selection bias, either. Kent also reported (accurately) that
during the period he examined, the New Hampshire Supreme Court overturned all of
the ten towns whose opponents appealed their granting of variances. This seems
to support his conclusion that local boards were prodigal in this regard.
However, a 2001 decision, \emph{Simplex v. Newington}, 145 N.H. 727, changed the
court's previous zoning variance criteria, on which Kent had relied as the
source of proper variances, to a less exacting standard that more closely
reflected actual practice.

Legal error is not practical error, much less economic harm. While the articles
critical of boards mention the possibility of variances degrading the
neighborhood, even anecdotal evidence in support of that contention is scarce.
Without visiting the site in question, it is often extremely difficult to tell
whether the variance was warranted by legal, practical, or economic criteria. An
underappreciated study by David Bryden (1977) established this more
systematically. Bryden examined scores of Minnesota lakeshore building and
septic variances (of which he had no part in granting) and concluded that what
looked like a travesty from the legal record in almost all cases made perfectly
good sense to local board members who were acquainted with the details of the
sites in question. For example, building setback variances, which by themselves
seemed to have been issued with little regard to the state's standard criteria,
were granted most often to allow septic systems to be even farther from the lake
than the state required. The local officials knew the sites and made what Bryden
inferred were appropriate tradeoffs between the serious risk of septic-tank
pollution of water bodies and the less-consequential aesthetic concerns of
building set-backs.

This is not to say that zoning boards are faultless. Some members can be, in my
experience, petty busybodies or inclined to promote a political agenda. (My
guess is that the selectboard originally suspected me of being in the latter
category.) Though I never had reason to suspect corruption, I sometimes thought
that favoritism and score-settling flavored some members' votes. But even the
least sophisticated zoning boards have an asset that is almost never available
to appellate judges or to statistical analysts: They know at least the
neighborhood and usually the specific site from personal experience. Critics
need to take that into account.


