\expected{ladue-v-horn}

\item
\textbf{Further background on the Supreme Court cases}. \textit{Village of Belle
Terre v. Boraas}, 416 U.S. 1 (1974), was primarily concerned with the Village's
attempts to exclude groups of unrelated college students from living together.
The Supreme Court cited \textit{Euclid} and similar cases in support of its
holding that the legislature can decide what kinds of uses are detrimental to
the peaceful and attractive character of the area:
\begin{quote}
The regimes of boarding houses, fraternity houses, and the like present urban
problems. More people occupy a given space; more cars rather continuously pass
by; more cars are parked; noise travels with crowds\ldots . The police power is
not confined to elimination of filth, stench, and unhealthy places. It is ample
to lay out zones where family values, youth values, and the blessings of quiet
seclusion and clean air make the area a sanctuary for people.
\end{quote}


\captionedgraphic{zoning-img012}{Juan Monroy, Belle Terre, Sept. 7, 2014, CC-BY.
Despite the gates at the entrance to the town, this is not a private gated
community, at least not in formal legal terms.}

Are college students nuisance-like? The \textit{Belle Terre} Court said the
ordinance in that case showed no animosity towards unmarried couples, as proven
by its inclusion of two unmarried people in its definition of ``family.'' But
what about an unmarried couple with children, as in \textit{Ladue}?

\item
Justice Marshall's vigorous dissent in \textit{Belle Terre} would have
distinguished between ``uses of land\ldots , for example, the number and kind of
dwellings to be constructed in a certain neighborhood or the number of persons
who can reside in those dwellings,'' which zoning authorities could validly
regulate, and ``who those persons are, what they believe, or how they choose to
live, whether they are Negro or white, Catholic or Jew, Republican or Democrat,
married or unmarried,'' which he would have found they could not. Justice
Marshall invoked both the First Amendment freedom of association and the
constitutional right to privacy:
\begin{quote}
The choice of household companions---of whether a person's ``intellectual and
emotional needs'' are best met by living with family, friends, professional
associates, or others---involves deeply personal considerations as to the kind
and quality of intimate relationships within the home. That decision surely
falls within the ambit of the right to privacy protected by the Constitution. 
\end{quote}

The family ordinance only limited the density of homes occupied by unrelated
people---thus, the crowding, noise, and other justifications offered were both
overinclusive and underinclusive, in Justice Marshall's view. ``While an
extended family of a dozen or more might live in a small bungalow, three elderly
and retired persons could not occupy the large manor house next door.'' A
neutral ordinance regulating density, noise, etc. could accomplish all the
town's goals. ``The burden of such an ordinance would fall equally upon all
segments of the community. It would surely be better tailored to the goals
asserted by the village than the ordinance before us today, for it would more
realistically restrict population density and growth and their attendant
environmental costs.'' 

\item
In \textit{Moore v. City of East Cleveland}, 431 U.S. 494 (1977), Justice
Marshall joined the plurality opinion of the Court striking down East
Cleveland's more limited definition of ``family,'' over several
dissents.\footnote{The East Cleveland ordinance stated:

\begin{quote}
``Family'' means a
number of individuals related to the nominal head of the household or to the
spouse of the nominal head of the household living as a single housekeeping unit
in a single dwelling unit, but limited to the following:
\begin{statute}
\item (a) Husband or wife
of the nominal head of the household.

\item (b) Unmarried children of the nominal
head of the household or of the spouse of the nominal head of the household,
provided, however, that such unmarried children have no children residing with
them.

\item (c) Father or mother of the nominal head of the household or of the
spouse of the nominal head of the household.

\item (d) Notwithstanding the
provisions of subsection (b) hereof, a family may include not more than one
dependent married or unmarried child of the nominal head of the household or of
the spouse of the nominal head of the household and the spouse and dependent
children of such dependent child\ldots.

\item (e) A family may consist of one
individual.
\end{statute}
\end{quote}
} \textit{Moore} involved an extended dispute among the Justices
about the nature and limits of ``substantive due process,'' which had also been
used to protect the rights to contraception, abortion, home schooling of
children, and other private choices. The question was whether the ordinary
rational basis scrutiny of zoning would apply, or a higher standard reflecting
the extent of the ordinance's intrusion into family life.\footnote{This debate
between Justices continues to the present day, notably in disputes over abortion
and the rights of same-sex couples to be free from criminal prosecution and,
more recently, to marry.}

In that case, Inez Moore lived with her son, Dale Moore, Sr., and her two
grandsons, Dale, Jr., and John Moore, Jr. The two boys were first cousins,
rather than brothers; John came to live with his grandmother and the elder and
younger Dale Moores after his mother's death. This caused the household to
violate East Cleveland's family ordinance, resulting in criminal charges against
Mrs. Moore. The Court distinguished \textit{Belle Terre} by reasoning that East
Cleveland ``has chosen to regulate the occupancy of its housing by slicing
deeply into the family itself.'' Such ``intrusive regulation of the family'' was
invalid. The City defended its goals with the same crowding and traffic
justifications as Belle Terre, and additionally argued that the ordinance
limited the burden on East Cleveland's schools. The Court found that these
legitimate goals were served ``marginally, at best,'' reiterating Justice
Marshall's points about overinclusiveness and underinclusiveness. 

The doctrine of substantive due process, which protects fundamental rights
against government intrusion, could not stop at the ``first convenient, if
arbitrary boundary---the boundary of the nuclear family.'' There was a long
tradition of ``uncles, aunts, cousins, and especially grandparents sharing a
household along with parents and children\ldots . Especially in times of
adversity, such as the death of a spouse or economic need, the broader family
has tended to come together for mutual sustenance and to maintain or rebuild a
secure home life. This is apparently what happened here.'' Justices Brennan and
Marshall, in concurrence, specifically pointed out that the ``nuclear family''
was really the pattern of ``white suburbia,'' which could not impose its
preference on others, and noted traditions among immigrants and
African-Americans of living together in multigenerational arrangements as a
matter of survival. The concurrence touted multigenerational families as
stronger and more beneficial for children than isolated nuclear families.
Ultimately, the plurality wrote, ``the Constitution prevents East Cleveland from
standardizing its children---and its adults---by forcing all to live in certain
narrowly defined family patterns.''

Justice Brennan's concurrence also discussed the possibility of seeking a
variance, and stated that ``the very existence of the `escape hatch' of the
variance procedure only heightens the irrationality of the restrictive
definition, since application of the ordinance then depends upon which family
units the zoning authorities permit to reside together and whom the prosecuting
authorities choose to prosecute.''

Justice Stewart, joined by then-Justice Rehnquist, would have upheld the
ordinance, rejecting the theory that ``that the biological fact of common
ancestry necessarily gives related persons constitutional rights of association
superior to those of unrelated person.'' The interests of a grandmother in
living with her grandchildren were simply not sufficient, in the dissenters'
view, to justify invalidating a zoning ordinance. It was acceptable for a city
to choose ``the pattern of `white suburbia,' even though that choice may reflect
`cultural myopia'\,''---Justice Stewart pointed out that East Cleveland was at
that time predominantly African-American, and that its city manager and city
commission were African-American. If the city was required to include
grandchildren, why not longtime friends? A line had to be drawn somewhere, and
this one was rational, especially since the grandmother could seek a variance if
the application of the ordinance to her wouldn't further its goals. 

\item
Given this further detail about \textit{Belle Terre} and \textit{Moore}, do you
think \textit{Ladue v. Horn} reached the right conclusion? Consider \textsc{Paul
Boudreaux, The Housing Bias: Rethinking Land Use Laws for a Diverse New America}
(2011):
\begin{quote}
[Restrictive single family] regulations provide a fascinating perspective into
the unique powers that America gives to laws governing ``land use.'' Government
cannot, of course, tell you what kind of car to drive, what to cook for dinner,
whether to watch reality TV, whether to fill the living room with ceramic gnomes
or tchotchkes, or whether to pay for your kid's college education. All these
things are considered, and rightly so, within the realm of human privacy and
basic human freedom. But under the label of land use law, governments are able
to tell you who to consider your family and who can live in your house.\ldots
Why can government be so intrusive? Because the neighbors might not like how you
live and because they have pushed the local government, through civic local
democracy, into passing a law regulating your household. It's an accepted
exercise of the police power. 
\end{quote}

\item \textbf{More recent events}. Ladue's current (as of 2023) ordinance allows
``[o]ne or more persons related by blood, marriage or legal adoption, or any
number of persons so related plus one unrelated person, or two unrelated
persons, occupying a dwelling unit as an individual housekeeping
organization.'' Is it constitutional to force an unmarried couple to leave if
they each have a child from a prior relationship, or a married couple after they
take foster children into their home? In 2006, a lesbian couple with a child was
excluded from Ladue because of its family composition ordinance, and the same
year an unmarried couple with two children was told they had to leave the home
they'd bought in Black Jack, Missouri, another St. Louis
suburb.\footnote{\textit{See, e.g.}, Nancy Larson, \textit{Gay Couples Keep
Out!}, \textsc{Advocate} 34 (Jul. 18, 2006) (discussing lesbian couple and
daughter who were warned by real estate agents that Ladue would prevent them
from living together); Eun Kyung Kim, \textit{Law Means Unwed Couple, 3 Kids May
Be\ldots Booted From Black Jack}, \textsc{St. Louis Post-Dispatch} A1 (Feb. 22,
2006); Jack W. Greer, \textit{``Family'' Crackdown Planned, Unmarried Couples
Face Citation From Attorney}, \textsc{St. Louis Post-Dispatch} A1 (July 21,
1994) (village attorney charged several unmarried couples with violating Wilbur
Park family composition ordinance); Ann Scales Cobbs, \textit{Couple Rebuffed by
Jennings, Ferguson on Occupancy Permits}, \textsc{St. Louis Post-Dispatch} 8D
(May 26, 1991) (Ferguson zoning law prevented unmarried couple from living with
two of woman's relatives); Michael Tackett, \textit{An Imperfect Family Circle
Squares Off With Zoning Law}, \textsc{Chicago Trib.} A1 (Nov. 9, 1986) (Ladue
ordinance prevented unmarried couple from living together).}  

The American Civil Liberties Union sued Black Jack. Discovery revealed that at
least four other couples had been denied occupancy permits to live in Back Jack
because they were unmarried and living with children, including a couple who
were the parents of triplets. \textit{Dispatch From Black Jack, MO},
\textsc{L.A. Times} A12 (May 21, 2006). Black Jack agreed to change its
ordinance to settle the litigation. Its ordinance now defines family as:
\begin{statute}
\item 1. An individual living as a single nonprofit housekeeping unit in a
dwelling unit; 

\item 2. Two (2) or more persons related by blood, marriage, adoption or foster
care relationship living together as a single nonprofit housekeeping unit in a
dwelling unit; 

\item 3. A group of not more than three (3) persons who need not be related by
blood, marriage, adoption or foster care relationship, living together as a
single nonprofit housekeeping unit in a dwelling unit; or 

\item 4. Two (2) unrelated individuals having a child or children related by
blood, adoption or foster care relationship to both such individuals, plus the
biological, adopted or foster children of either such individual, living
together as a single nonprofit housekeeping unit in a dwelling unit. 
\end{statute}
Now that the lesbian couple in Black Jack can legally marry, can Black Jack go
back to requiring couples to be married if they want to live in Black Jack with
their children?

\item
Compare Ferguson's definition of family:
\begin{quotation}
One or two adults and the children and/or grandchildren of such adults and not
more than two (2) other adults who are both related to either of the other two,
living together as a single housekeeping unit in a dwelling with single kitchen
facilities provided that such occupancy does not exceed the maximum occupancy
limits for such dwelling; 

or a group of not more than three (3) unrelated persons living together by joint
agreement occupying a single housekeeping unit with single kitchen facilities
provided that such occupancy does not exceed the maximum occupancy limits for
such dwelling. 

For purposes of this definition, a related person shall include any relative
within the fourth degree by consanguinity or affinity.
\end{quotation}

Under this definition, can two married couples live together with their children
if one person in the first couple is a first cousin of one person in the second
couple?

\begin{table}
\begin{center}
\readingfont
\footnotesize
\begin{tabular}{p{0.3\textwidth}p{0.3\textwidth}p{0.3\textwidth}}
\textbf{An individual or married couple and their children +} &
\textbf{No more than two people related to the individual or married couple by
blood or marriage +} &
\textbf{Not more than one additional unrelated person} \\
\hline
\multicolumn{3}{l}{Acceptable examples:} \\
\hline
Mr. \& Mrs. Jones and their children Bobby and Katie + & Mrs. Jones' parents
+ & Jennifer Doe, a friend \\
Mrs. Thomas and her three children + & Her aunt and uncle + & Her cousin
(although related, could be counted as one other unrelated person) \\
Mr. and Mrs. Rogers and their son + & Mr. Rogers' brother and his son & \\
\hline
\multicolumn{3}{l}{Examples in violation of this code:} \\
\hline
Mr. \& Mrs. Jones and their children Bobby and Katie + & Mrs. Jones'
sister and brother-in-law and their three children & \\
\multicolumn{3}{p{0.9\textwidth}}{\emph{Why is this a violation?}
The addition of Mrs. Jones sister and her family exceeds the two related people
and one additional unrelated person.} \\
\hline
Bob Campbell + & Bob's two brothers + & Two unrelated roommates \\
\multicolumn{3}{p{0.9\textwidth}}{\emph{Why is this a violation?}
The total number of occupants in this example is five. One unrelated roommate
would need to move out to be in compliance in any zoning district.} \\
\hline
John Doe + & & Three unrelated roommates \\
\multicolumn{3}{p{0.9\textwidth}}{\emph{Why is this a violation?}
It exceeds the three unrelated people allowed in R-1 zoning; it would be
allowable in all other zoning districts.} \\
\hline
Jane Roberts + & & Four unrelated roommates \\
\multicolumn{3}{p{0.9\textwidth}}{\emph{Why is this a violation?}
Four or more unrelated people are not allowed in any zoning district. If one
roommate left it would be acceptable in all zoning districts except R-1; Jane
and two roommates are acceptable in R-1.} \\
\end{tabular}
\end{center}
\caption{Table of hypothetical examples from the Columbia, Mo., guide.}
\label{t:family-def}
\end{table}

\item
Finally, consider this handy guide put out by Columbia, Missouri, \emph{What is
a Family?} (archived July 15, 2015),
\url{https://web.archive.org/web/20140715054932/https://www.gocolumbiamo.com/community_development/neighborhoods/renting/documents/WhatisaFamily.pdf}:
\begin{quotation}
What the code says: According to Chapter 29---Zoning of City Ordinance, the
definition of Family is:
\begin{statute}
\item (1) An individual or married couple and the children thereof and no more
than two (2) other persons related directly to the individual or married couple
by blood or marriage, occupying a single housekeeping unit on a nonprofit basis.
A family may include not more than one additional person, not related to the
family by blood or marriage; or

\item (2) a.\ldots In zoning districts R-1\ldots a group of not more than three
(3) persons not related by blood or marriage, living together by joint agreement
and occupying a single housekeeping unit on a nonprofit cost-sharing
basis\ldots.
\begin{statute}
\item b. In all other applicable zoning districts, a group of not more than four
(4) persons not related by blood or marriage, living together by joint agreement
and occupying a single housekeeping unit on a nonprofit cost-sharing basis.
\end{statute}
\end{statute}
Why it matters: When properties in the City of Columbia exceed our occupancy
limits, it creates additional traffic, trash and noise and can harm quality of
life for neighbors. This especially an issue when occupancy limits are exceeded
in R-1 zoning districts. The City of Columbia will investigate properties
suspected of over occupancy and may prosecute property owners and tenants in
violation.

Breaking it down---some hypothetical examples [are given in
Table~\ref{t:family-def}.]
\end{quotation}


Suppose you were asked to write a family composition ordinance. How would you
frame it?

\item
\textbf{States' varying treatment of family composition rules}. A number of
other states, either on federal or state constitutional grounds, have instead
drawn the line at ``single housekeeping units'' or ``functional families.''
\textit{See, e.g.}, \emph{Delta Charter T'ship v. Dinolfo}, 351 N.W.2d 831
(Mich. 1984) (no rational basis to preclude four childhood friends from living
together); \emph{DiStefano v. Haxton}, 1994 WL 931006 (R.I. Super. 1994)
(plaintiffs had a liberty interest in choosing their own living companions, and
city provided no evidence that unrelated groups were more likely to be
disruptive than those in related households: ``It is a strange---and
unconstitutional---ordinance indeed that would permit the Hatfields and the
McCoys to live in a residential zone while barring four scholars from the
University of Rhode Island from sharing an apartment on the same street.'');
\emph{Borough of Glassboro v. Vallorosi}, 535 A.2d 544 (N.J. Superior Ct. 1987)
(overturning ordinance aimed at keeping college students from living together;
mayor compared student residency to ``toxic waste'').

Numerous municipalities have relaxed their family definitions even without a
constitutional mandate, reflecting demographic facts. Ordinances that embrace
all functional families often survive constitutional scrutiny. \textit{See,
e.g.}, \emph{Stegman v. City of Ann Arbor}, 540 N.W.2d 724 (Mich. Ct. App. 1995)
(upholding a functional family ordinance against ``a ragtag collection of
college roommates'' who wanted to live together); \emph{Dinan v. Board of Zoning
Appeals}, 595 A.2d 864 (Conn. S.Ct. 1991) (upholding single housekeeping unit
ordinance because households with unrelated people ``are less likely to develop
the kind of friendly relationships with neighbors that abound in residential
districts occupied by traditional families\ldots they are not likely to have
children who would become playmates of other children living in the area.
Neighbors are not so likely to call upon them to borrow a cup of sugar, provide
a ride to the store, mind the family pets, water the plants or perform any of
the countless services that families, both traditional and nontraditional,
provide to each other as a result of longtime acquaintance and mutual self
interest.'').

The Court of Appeals of New York has been particularly protective of individual
choice of living arrangements. \textit{See, e.g.}, \emph{Group House of Port
Washington v. Board of Zoning and Appeals}, 380
N.E.2d 207 (N.Y. 1978) (a house consisting of two surrogate parents and seven
emotionally disturbed children was ``the functional and factual equivalent
of a natural family, and to exclude it from a residential area would be to serve
no valid purpose''); \emph{McMinn v. Town of Oyster Bay}, 488 N.E.2d 1240 (N.Y.
1985) (town could not exclude from its definition of family two unrelated people
under 62, while allowing two related people 62 or over); \emph{Baer v. Town of
Brookhaven}, 537 N.E.2d 619 (N.Y. 1989) (town could not exclude five unrelated
elderly women residing together under a definition of family providing that not
more than 4 unrelated persons living and cooking together as a single
housekeeping unit could constitute a family; state constitution precluded the
town from limiting the size of a functionally equivalent family of unrelated
persons but not the size of a traditional family); \textit{cf.} \emph{Braschi v.
Stahl Associates}, 543 N.E.2d 49 (N.Y. 1989) (two gay men living together in a
spousal-like arrangement could constitute a ``family'' within the context of the
non-eviction provisions of the New York City Rent and Eviction regulations). 

Many New York municipalities now presume that a group of individuals smaller
than four is a functional family, and presume that a larger group is not but
allow it to rebut that presumption. \textit{See, e.g.}, \emph{Unification
Theological Seminary v. City of Poughkeepsie}, 607 N.Y.S.2d 383 (N.Y. App. Div.
1994) (upholding this practice, where the ordinance provided that the zoning
administrator should consider whether the group shares the entire house; lives
and cooks together as a single housekeeping unit; shares expenses for food,
rent, utilities or other household expenses; and is permanent and stable).

However, a number of states still follow \textit{Belle Terre} when a
jurisdiction's family composition ordinance is challenged. The litigated cases
tend to be older, and even in the 1990s enforcement often drew incredulous media
coverage, but there are a few recent cases upholding restrictive definitions of
family. \textit{See, e.g.}, \emph{City of Baton Rouge/Parish of East Baton Rouge
v. Myers}, 145 So. 3d 320 (La. 2014) (upholding single-family ordinance that
allowed (1) an unlimited number of related people or (2) no more than four
unrelated people in a single housekeeping unit, if the owner occupied the
premises); \emph{State v. Champoux}, 566 N.W.2d 763 (Neb. 1997) (upholding
family composition ordinance); \emph{City of Brookings v. Winker}, 554 N.W.2d
827 (S.D. 1996) (same); \emph{Doe v. City of Butler, Pennsylvania}, 892 F.2d 315
(3d Cir. 1989) (single family zoning ordinance that prevented six victims of
domestic violence from living together in a shelter did not interfere with their
right to associate with one another because associational rights do not extend
to living with nonrelatives); \emph{Carroll v. Washington Township Zoning
Commission}, 408 N.E.2d 191 (Ohio 1980) (couple could not act as foster parents
given single family zoning); \emph{State v. Baker}, 405 A.2d 368 (N.J. 1979)
(enforcing single family ordinance against couple, their three children, adult
woman, and her three children even though they considered themselves an
``extended family''); \emph{Town of Durham v. White Enterprises, Inc.}, 348 A.2d
706 (N.H. 1975) (``The State has no particular interest in keeping together a
group of unrelated persons. The State has a clear interest, however, in
preserving the integrity of the biological or legal family.''). 

Some jurisdictions have even tightened their definitions. \textit{See, e.g.},
Stephanie McCrummen, \emph{Manassas Changes Definition of Family}, \textsc{Wash.
Post} A1 (Dec. 28, 2005) (newly enacted Manassas, VA zoning law prevented couple
from living with woman's nephew; opponents attributed enactment to
discrimination against immigrants); \textit{see generally} Rigel C. Oliveri,
\emph{Single Family Zoning, Intimate Association, and the Right To Choose
Household Companions}, \textsc{Fla. L. Rev.} (2015); Adam Lubow, \emph{``\ldots
Not Related by Blood, Marriage, or Adoption'': A History of the Definition of
``Family'' in Zoning Law}, 16 \textsc{J. Afford. Hous. \& Comm. Dev. Law} 144
(2007). In other instances, zoning authorities have focused on excluding groups
of college students, not others. \textit{See, e.g.}, \emph{Rosenberg v. City of
Boston}, 2010 WL 2090956 (Mass. Land. Ct. 2010) (upholding the constitutionality
of excluding only ``five or more persons who are enrolled as full-time
undergraduate students at a post-secondary educational institution'' from living
together in a dwelling unit).

\item
The Supreme Court, in \textit{Obergefell v. Hodges}, rejected arguments that
bans on same-sex marriage protected children, because of the numerous children
living with same-sex couples whose interests were harmed by discrimination
against their parents. Does the same rationale apply here to invalidate family
composition ordinances, at least as applied to households with children?

\item
As for college students, can measures to protect against the damage they do be
achieved through other, less stereotypical means? Oliveri, \textit{supra},
suggests that a jurisdiction's legitimate interests can be protected through
density regulations, reasonable limits on the number of cars per location,
criminal code enforcement against noise, and other code enforcement. She
concludes: ``Often, the real problem is absentee landlords, who fail to maintain
their property because they know students are unlikely to complain. In that
case, property maintenance codes should be rigorously enforced. If an
over-abundance of rentals is the problem, then owner-occupancy requirements
might be put in place that limit the percentage of houses in a particular
neighborhoods that can be rented.'' Should a jurisdiction be forced to give up
prophylactic measures in favor of case-by-case enforcement of this type? 

\item
\textbf{Occupancy permits}. Have you ever had to obtain an occupancy permit?
Many Missouri cities and towns use occupancy permits to help enforce their
family composition rules. In Ferguson, for example, an occupancy permit must be
obtained, and a fee paid, every time the composition of a dwelling unit changes.
Birth certificates for children, photo IDs for adults, and an inspection---with
a separate \$40 fee---are also required with each change. The Department of
Justice reported that this rule became a part of unfair treatment of poor
African-Americans. For example, one woman who called the police for help with
domestic violence was arrested for violating her occupancy permit because the
call revealed the presence of a boyfriend on the premises; another was given a
summons for the same reason. \textsc{U.S. Dep't of Justice, Civ. Rts. Div.,
Investigation of the Ferguson Police Department} 81 (Mar. 2015),
\url{http://www.justice.gov/sites/default/files/opa/press-releases/attachments/2015/03/04/ferguson_police_department_report_1.pdf}.\footnote{The
report also noted that ``In 2013 alone, the court issued over 9,000 warrants on
cases stemming in large part from minor violations such as parking infractions,
traffic tickets, or housing code violations. Jail time would be considered far
too harsh a penalty for the great majority of these code violations, yet
Ferguson's municipal court routinely issues warrants for people to be arrested
and incarcerated for failing to timely pay related fines and fees.'' In 2011,
the Municipal Judge in Ferguson, responding to the City's instructions to
increase revenue from the court, touted his treatment of fines for repeat
offenders, ``especially in regard to housing violations, [which] have increased
substantially and will continue to be increased upon subsequent violations.''
Ferguson requires anyone cited for a housing violation to appear in court,
whether or not they are contesting the charges; failure to appear risks
additional fines and arrest warrants.}

\item
\textbf{The Fair Housing Act and the Americans with Disabilities Act}. The FHA
and the ADA may also limit family composition rules, as applied to group homes
for people with disabilities. Disability-related zoning litigation often
involves residents of group homes, who routinely experience discrimination,
either overt or simply through indifference, usually in the form of bans on
group living arrangements. \textit{See, e.g.}, \emph{Oxford House v. Town of
Babylon}, 819 F. Supp. 1179 (E.D.N.Y. 1993) (finding that town's family
composition ordinance discriminated against individuals recovering from drug or
alcohol addition because of their handicap). While pure density regulations
capping the number of occupants per dwelling are exempt from the FHA, family
definitions are not pure density regulations and thus reasonable accommodations
to them may be required. \emph{City of Edmonds v. Oxford House}, 514 U.S. 725
(1995).

In order to deal with repeated FHA litigation around group homes, Missouri
amended its zoning authorization statute, providing comprehensive definitions
and limiting localities' power to exclude group homes: 
\begin{quote}
For the purpose of any zoning law, ordinance or code, the classification single
family dwelling or single family residence shall include any home in which eight
or fewer unrelated mentally or physically handicapped persons reside, and may
include two additional persons acting as house parents or guardians who need not
be related to each other or to any of the mentally or physically handicapped
persons residing in the home. In the case of any such residential home for
mentally or physically handicapped persons, the local zoning authority may
require that the exterior appearance of the home and property be in reasonable
conformance with the general neighborhood standards. Further, the local zoning
authority may establish reasonable standards regarding the density of such
individual homes in an specific single family dwelling neighborhood.
\end{quote}
Section 89.020.2. Does this adequately address problems of potential
discrimination, including the obligation to provide reasonable accommodation? 

