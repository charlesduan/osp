\reading{Planning and Zoning Procedures for Missouri Municipalities}
\readingcite{Missouri Municipal League (Sept. 2004)}

All cities, towns and villages in Missouri may adopt planning and zoning.
Statutory authority to enact planning and zoning is found in Chapter 89 of the
Revised Statutes of Missouri (RSMo). Chapter 89 establishes the procedural
framework in which planning and zoning is enacted and administered.\ldots Left
uncoordinated, land use patterns are unpredictable and public services are
provided in a haphazard manner, often adversely affecting the quality of life
within the community. Zoning is the set of regulations that prescribe how land
within a municipality is used\ldots .

The Missouri Revised Statutes makes provisions for a zoning commission (Section
89.070 RSMo) and a planning commission (Section 89.320 RSMo). The purpose of the
zoning commission is to write the original zoning ordinance. The planning
commission's function is to plan for the development of the municipality. \ldots

\readinghead{Planning Staff}

Many large and moderate sized cities hire a professional planning staff to
assist the planning and zoning commission in the preparation and administration
of the comprehensive development plan, zoning ordinance and subdivision
regulations. However, in most smaller cities the planning commission functions
without a professional staff. In this situation the planning commission mainly
will be concerned with the administration of the zoning ordinance and
subdivision regulations\ldots .

\readinghead{Zoning and the Comprehensive Plan}

The distinction between the zoning ordinance and the comprehensive plan is
sometimes a confusing subject for those outside the planning profession. This
confusion arises out of the fact that many cities adopt zoning ordinances before
a comprehensive plan is prepared. Therefore, it sometimes is difficult to
understand the logical connection between the two documents.

According to state law (Section 89.040 RSMo), a zoning ordinance must be based
on a comprehensive plan. A zoning ordinance that is not based on a comprehensive
plan is not legally sound.\ldots When a zoning ordinance is not based on a
comprehensive plan, there is a tendency for development to become frozen in
existing patterns or for an undesirable development pattern to occur. An
ordinance that is not developed in accordance with a plan generally requires
many amendments, which makes the ordinance very difficult to interpret and
administer.

\readinghead{What A Zoning Ordinance Does Not Do}

The zoning ordinance is not designed to regulate the types of materials used for
the construction of buildings or the manner in which buildings are constructed.
This is the function of building codes. Also, the zoning ordinance does not
establish the minimum cost of permitted structures nor control their appearance.
These matters are generally controlled by protective covenants contained in the
deed to property.

The zoning ordinance does not regulate the design of streets, the installation
of utilities or the dedication of parks, street rights-of-way and school sites
and related matters. These are controlled by the subdivision regulations and by
an official map preserving beds of proposed streets against encroachment.

Zoning ordinances deal primarily with future development and cannot be used to
correct existing conditions. These generally are addressed by the housing code,
which establishes minimum housing standards and requires the rehabilitation or
demolition of existing substandard structures\ldots .


\readinghead{Necessary Information}

Most of the information needed to develop the zoning ordinance already should
have been assembled and included in the city's comprehensive plan. Following is
the type of information that will be useful in preparing the zoning ordinance.
\begin{itemize}
\item[1)] The existing use of every piece of property within the city;

\item[2)] The terms of restrictive covenants applying to large sections of the
city;

\item[3)] The location and capacities of all utility lines and major streets;

\item[4)] The assessed valuation of properties in different sections of the
city;

\item[5)] The location and characteristics of all vacant land in the city;

\item[6)] The location of all new buildings erected during the past five years;

\item[7)] The width of streets;

\item[8)] The size of front, side and rear yards;

\item[9)] The heights of buildings;

\item[10)] The dimensions of lots; and

\item[11)] The number of families in each dwelling.
\end{itemize}
Once this information has been gathered and mapped, it should be analyzed.
Analysis of the information should focus on the amount of land used for
dwellings, businesses and industries; the predominant yard size; building
heights; population densities; availability of utilities and street types. These
studies along with the economic studies and population studies in the
comprehensive plan can aid the city in forecasting future land requirements for
each land use.


\readinghead{Elements Of A Zoning Ordinance}

Most zoning ordinances consist of two parts: a zoning map indicating the
boundaries of the various zoning districts and written regulations defining the
manner in which property may be used in each district.


\readinghead{The Zoning Map}

\ldots [I]t generally is the case, when attempting to formulate a zoning
district map, that existing land use patterns conflict with the land use plan to
some degree. When this occurs, a compromise must be made between existing land
use patterns and the city's desired land use pattern as developed in the land
use plan. The land use plan then becomes a guide for this decision process, as
well as a guide to be followed in making later amendments to the zoning
ordinance.One of the most difficult aspects of developing a zoning district map
is the drawing of exact boundary lines between districts, since all boundary
lines are somewhat arbitrary, and individual property owners are likely to raise
protests that are hard to resolve\ldots .


\readinghead{Zoning Regulations}

\ldots Each type of district will have regulations that control the height of
buildings, bulk of buildings, lot coverage, yard requirements and a special
provision dealing with off-street parking and loading\ldots .

