\reading{State ex rel. Stoyanoff v. Berkeley}

\readingcite{458 S.W.2d 305 (Mo. 1970)}

\opinion \textsc{Pritchard}, Commissioner.

Upon summary judgment the trial court issued a peremptory writ of mandamus to
compel appellant to issue a residential building permit to respondents. The
trial court's judgment is that the below-mentioned ordinances are violative of
Section 10, Article I of the Constitution of Missouri, 1945, in that
restrictions placed by the ordinances on the use of property deprive the owners
of their property without due process of law. Relators' petition pleads that
they applied to appellant Building Commissioner for a building permit to allow
them to construct a single family residence in the City of Ladue, and that plans
and specifications were submitted for the proposed residence, which was unusual
in design, ``but complied with all existing building and zoning regulations and
ordinances of the City of Ladue, Missouri.''


\captionedgraphic{zoning-img010}{Artist's rendering of proposed house.}

It is further pleaded that relators were refused a building permit for the
construction of their proposed residence upon the ground that the permit was not
approved by the Architectural Board of the City of Ladue. Ordinance 131, as
amended by Ordinance 281 of that city, purports to set up an Architectural Board
to approve plans and specifications for buildings and structures erected within
the city and in a preamble to ``conform to certain minimum architectural
standards of appearance and conformity with surrounding structures, and that
unsightly, grotesque and unsuitable structures, detrimental to the stability of
value and the welfare of surrounding property, structures and residents, and to
the general welfare and happiness of the community, be avoided, and that
appropriate standards of beauty and conformity be fostered and encouraged.'' It
is asserted in the petition that the ordinances are invalid, illegal and void,
``are unconstitutional in that they are vague and provide no standard nor
uniform rule by which to guide the architectural board,'' that the city acted in
excess of statutory powers in enacting the ordinances, which ``attempt to allow
respondent to impose aesthetic standards for buildings in the City of Ladue, and
are in excess of the powers granted the City of Ladue by said statute.''

Relators filed a motion for summary judgment and affidavits were filed in
opposition thereto. Richard D. Shelton, Mayor of the City of Ladue, deponed that
the facts in appellant's answer were true and correct, as here pertinent: that
the City of Ladue constitutes one of the finer suburban residential areas of
Metropolitan St. Louis, the homes therein are considerably more expensive than
in cities of comparable size, being homes on lots from three fourths of an acre
to three or more acres each; that a zoning ordinance was enacted by the city
regulating the height, number of stories, size of buildings, percentage of lot
occupancy, yard sizes, and the location and use of buildings and land for trade,
industry, residence and other purposes; that the zoning regulations were made in
accordance with a comprehensive plan ``designed to promote the health and
general welfare of the residents of the City of Ladue,'' which in furtherance of
said objectives duly enacted said Ordinances numbered 131 and 281. Appellant
also asserted in his answer that these ordinances were a reasonable exercise of
the city's governmental, legislative and police powers, as determined by its
legislative body, and as stated in the above-quoted preamble to the ordinances.
It is then pleaded that relators' description of their proposed residence as
``\,`unusual in design' is the understatement of the year. It is in fact a
monstrosity of grotesque design, which would seriously impair the value of
property in the neighborhood.''

The affidavit of Harold C. Simon, a developer of residential subdivisions in St.
Louis County, is that he is familiar with relators' lot upon which they seek to
build a house, and with the surrounding houses in the neighborhood; that the
houses therein existent are virtually all two-story houses of conventional
architectural design, such as Colonial, French Provincial or English; and that
the house which relators propose to construct is of ultramodern design which
would clash with and not be in conformity with any other house in the entire
neighborhood. It is Mr. Simon's opinion that the design and appearance of
relators' proposed residence would have a substantial adverse effect upon the
market values of other residential property in the neighborhood, such average
market value ranging from \$60,000 to \$85,000 each.

As a part of the affidavit of Russell H. Riley, consultant for the city planning
and engineering firm of Harland Bartholomew \& Associates, photographic exhibits
of homes surrounding relators' lot were attached. To the south is the
conventional frame residence of Mrs. T. R. Collins. To the west is the Colonial
two-story frame house of the Lewis family. To the northeast is the large brick
English Tudor home of Mrs. Elmer Hubbs. Immediately to the north are the large
Colonial homes of Mr. Alex Cornwall and Mr. L. Peter Wetzel. In substance Mr.
Riley went on to say that the City of Ladue is one of the finer residential
suburbs in the St. Louis area with a minimum of commercial or industrial usage.
The development of residences in the city has been primarily by private
subdivisions, usually with one main lane or drive leading therein (such as
Lorenzo Road Subdivision which runs north off of Ladue Road in which relators'
lot is located). The homes are considerably more expensive than average homes
found in a city of comparable size. The ordinance which has been adopted by the
City of Ladue is typical of those which have been adopted by a number of
suburban cities in St. Louis County and in similar cities throughout the United
States, the need therefor being based upon the protection of existing property
values by preventing the construction of houses that are in complete conflict
with the general type of houses in a given area. The intrusion into this
neighborhood of relators' unusual, grotesque and nonconforming structure would
have a substantial adverse effect on market values of other homes in the
immediate area. According to Mr. Riley the standards of Ordinance 131, as
amended by Ordinance 281, are usually and customarily applied in city planning
work and are: ``(1) whether the proposed house meets the customary architectural
requirements in appearance and design for a house of the particular type which
is proposed (whether it be Colonial, Tudor English, French Provincial, or
Modern), (2) whether the proposed house is in general conformity with the style
and design of surrounding structures, and (3) whether the proposed house lends
itself to the proper architectural development of the City; and that in applying
said standards the Architectural Board and its Chairman are to determine whether
the proposed house will have an adverse affect on the stability of values in the
surrounding area.''

Photographic exhibits of relators' proposed residence were also attached to Mr.
Riley's affidavit. They show the residence to be of a pyramid shape, with a flat
top, and with triangular shaped windows or doors at one or more corners\ldots .

Section 89.020 provides: ``For the purpose of promoting health, safety, morals
or the general welfare of the community, the legislative body of all cities,
towns, and villages is hereby empowered to regulate and restrict the height,
number of stories, and size of buildings and other structures, the percentage of
lot that may be occupied, the size of yards, courts, and other open spaces, the
density of population, the preservation of features of historical significance,
and the location and use of buildings, structures and land for trade, industry,
residence or other purposes.'' Section 89.040 provides: ``Such regulations shall
be made in accordance with a comprehensive plan and designed to lessen
congestion in the streets; to secure safety from fire, panic and other dangers;
to promote health and the general welfare; to provide adequate light and air; to
prevent the over-crowding of land; to avoid undue concentration of population;
to preserve features of historical significance; to facilitate the adequate
provision of transportation, water, sewerage, schools, parks, and other public
requirements. Such regulations shall be made with reasonable consideration,
among other things, to the \textit{character of the district and its peculiar
suitability for particular uses}, and with a view to conserving the values of
buildings and encouraging the most appropriate use of land throughout such
municipality.'' (italics added)

\ldots [The statutory language embraces considerations] relating to the
character of the district, its suitability for particular uses, and the
conservation of the values of buildings therein. These considerations,
sanctioned by statute, are directly related to the general welfare of the
community.\ldots ``\,`We hold that the police power of a state embraces
regulations designed to promote the public convenience or the general
prosperity, as well as regulations designed to promote the public health, the
public morals or the public safety.'\,''\ldots ``The stabilizing of property
values, and giving some assurance to the public that, if property is purchased
in a residential district, its value as such will be preserved, is probably the
most cogent reason back of zoning ordinances.'' The preamble to Ordinance 131,
quoted above in part, demonstrates that its purpose is to conform to the
dictates of \S~89.040, with reference to preserving values of property by zoning
procedure and restrictions on the use of property. This is an illustration
of\ldots a growing number of cases recognizing a change in the scope of the term
``general welfare.''\ldots ``Property use which offends sensibilities and
debases property values affects not only the adjoining property owners in that
vicinity but the general public as well because when such property values are
destroyed or seriously impaired, the tax base of the community is affected and
the public suffers economically as a result.''

Relators say further that Ordinances 131 and 281 are invalid and
unconstitutional as being an unreasonable and arbitrary exercise of the police
power. It is argued that a mere reading of these ordinances shows that they are
based entirely on aesthetic factors in that the stated purpose of the
Architectural Board is to maintain ``conformity with surrounding structures''
and to assure that structures ``conform to certain minimum architectural
standards of appearance.'' The argument ignores the further provisos in the
ordinance: ``\ldots and that unsightly, grotesque and unsuitable structures,
detrimental to the stability of value and the welfare of surrounding property,
structures, and residents, and to the general welfare and happiness of the
community, be avoided, and that appropriate standards of beauty and conformity
be fostered and encouraged.'' (Italics added.) Relators' proposed residence does
not descend to the ``\,`patently offensive character of vehicle graveyards in
close proximity to such highways'\,''\ldots. Nevertheless, the aesthetic factor
to be taken into account by the Architectural Board is not to be considered
alone. Along with that inherent factor is the effect that the proposed residence
would have upon the property values in the area. In this time of burgeoning
urban areas, congested with people and structures, it is certainly in keeping
with the ultimate ideal of general welfare that the Architectural Board, in its
function, preserve and protect existing areas in which structures of a general
conformity of architecture have been erected. The area under consideration is
clearly, from the record, a fashionable one. In State ex rel. \emph{Civello v.
City of New Orleans}, 154 La. 271, 97 So. 440, 444 (La. 1923), the court said,
``If by the term `aesthetic considerations' is meant a regard merely for outward
appearances, for good taste in the matter of the beauty of the neighborhood
itself, we do not observe any substantial reason for saying that such a
consideration is not a matter of general welfare. The beauty of a fashionable
residence neighborhood in a city is for the comfort and happiness of the
residents, and it sustains in a general way the value of property in the
neighborhood.'' [Other cases accept] the principle that aesthetics is a factor
to be considered in zoning matters.

In the matter of enacting zoning ordinances and the procedures for determining
whether any certain proposed structure or use is in compliance with or offends
the basic ordinance, it is well settled that courts will not substitute their
judgments for the city's legislative body, if the result is not oppressive,
arbitrary or unreasonable and does not infringe upon a valid preexisting
nonconforming use. The denial by appellant of a building permit for relators'
highly modernistic residence in this area where traditional Colonial, French
Provincial and English Tudor styles of architecture are erected does not appear
to be arbitrary and unreasonable when the basic purpose to be served is that of
the general welfare of persons in the entire community.

In addition to the above-stated purpose in the preamble to Ordinance 131, it
establishes an Architectural Board of three members, all of whom must be
architects. Meetings of the Board are to be open to the public, and every
application for a building permit, except those not affecting the outward
appearance of a building, shall be submitted to the Board along with plans,
elevations, detail drawings and specifications, before being approved by the
Building Commissioner. \ldots

Ordinances 131 and 281 are sufficient in their general standards calling for a
factual determination of the suitability of any proposed structure with
reference to the character of the surrounding neighborhood and to the
determination of any adverse effect on the general welfare and preservation of
property values of the community\ldots .

