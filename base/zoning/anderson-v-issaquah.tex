\reading{Anderson v. City of Issaquah}

\readingcite{851 P. 2d 744 (Wash. Ct. App. 1993)}

Appellants M. Bruce Anderson, Gary D. LaChance, and M. Bruce Anderson, Inc.
(hereinafter referred to as Anderson), challenge the denial of their application
for a land use certification, arguing, inter alia, that the building design
requirements contained in Issaquah Municipal Code (IMC) 16.16.060 are
unconstitutionally vague. The Superior Court rejected this constitutional
challenge. We reverse and direct that Anderson's land use certification be
issued.\ldots 


\readinghead{Facts}

Anderson owns property located at 145 N.W. Gilman Boulevard in the city of
Issaquah (City). In 1988, Anderson applied to the City for a land use
certification to develop the property. The property is zoned for general
commercial use. Anderson desired to build a 6,800-square-foot commercial
building for several retail tenants.

After obtaining architectural plans, Anderson submitted the project to various
City departments for the necessary approvals. The process went smoothly until
the approval of the Issaquah Development Commission (Development Commission) was
sought. This commission was created to administer and enforce the City's land
use regulations. It has the authority to approve or deny applications for land
use certification.

Section 16.16.060 of the IMC enumerates various building design objectives which
the Development Commission is required to administer and enforce. Insofar as is
relevant to this appeal, the Development Commission is to be guided by the
following criteria:
\begin{quotation}
\noindent IMC 16.16.060(B). Relationship of Building and Site to Adjoining Area.

1. Buildings and structures shall be made compatible with adjacent buildings
of conflicting architectural styles by such means as screens and site breaks, or
other suitable methods and materials.

2. Harmony in texture, lines, and masses shall be encouraged.

\ldots

\noindent IMC 16.16.060(D). Building Design.

1. Evaluation of a project shall be based on quality of its design and
relationship to the natural setting of the valley and surrounding mountains.

2. Building components, such as windows, doors, eaves and parapets, shall
have appropriate proportions and relationship to each other, expressing
themselves as a part of the overall design.

3. Colors shall be harmonious, with bright or brilliant colors used only for
minimal accent.

4. Design attention shall be given to screening from public view all
mechanical equipment, including refuse enclosures, electrical transformer pads
and vaults, communication equipment, and other utility hardware on roofs,
grounds or buildings.

5. Exterior lighting shall be part of the architectural concept. Fixtures,
standards and all exposed accessories shall be harmonious with the building
design.

6. Monotony of design in single or multiple building projects shall be
avoided. Efforts should be made to create an interesting project by use of
complimentary details, functional orientation of buildings, parking and access
provisions and relating the development to the site. In multiple building
projects, variable siting of individual buildings, heights of buildings, or
other methods shall be used to prevent a monotonous design.
\end{quotation}

As initially designed, Anderson's proposed structure was to be faced with
off-white stucco and was to have a blue metal roof. It was designed in a
``modern'' style with an unbroken ``warehouse'' appearance in the rear, and
large retail-style windows in the front. The City moved a Victorian era
residence, the ``Alexander House'', onto the neighboring property to serve as a
visitors' center. Across the street from the Anderson site is a gasoline station
that looks like a gasoline station. Located nearby and within view from the
proposed building site are two more gasoline stations, the First Mutual Bank
Building built in the ``Issaquah territorial style'', an Elks hall which is
described in the record by the Mayor of Issaquah as a ``box building'', an auto
repair shop, and a veterinary clinic with a cyclone-fenced dog run. The area is
described in the record as ``a natural transition area between old downtown
Issaquah and the new village style construction of Gilman [Boulevard].''

The Development Commission reviewed Anderson's application for the first time
at a public hearing on December 21, 1988. Commissioner Nash commented that ``the
facade did not fit with the concept of the surrounding area.'' Commissioner
McGinnis agreed. Commissioner Nash expressed concern about the building color
and stated that he did not think the building was compatible with the image of
Issaquah. Commissioner Larson said that he would like to see more depth to the
building facade. Commissioner Nash said there should be some interest created
along the blank back wall. Commissioner Garrison suggested that the rear facade
needed to be redesigned.

At the conclusion of the meeting, the Development Commission voted to continue
the hearing to give Anderson an opportunity to modify the building design. On
January 18, 1989, Anderson came back before the Development Commission with
modified plans which included changing the roofing from metal to tile, changing
the color of the structure from off-white to ``Cape Cod'' gray with
``Tahoe'' blue trim, and adding brick to the front facade. During the ensuing
discussion among the commissioners, Commissioner Larson stated that the
revisions to the front facade had not satisfied his concerns from the last
meeting. In response to Anderson's request for more specific design guidelines,
Commissioner McGinnis stated that the Development Commission had ``been giving
direction; it is the applicant's responsibility to take the
direction/suggestions and incorporate them into a revised plan that reflects the
changes.'' Commissioner Larson then suggested that ``the facade can be broken up
with sculptures, benches, fountains, etc.''

Commissioner Nash suggested that Anderson ``drive up and down Gilman and look at
both good and bad examples of what has been done with flat facades.''

As the discussion continued, Commissioner Larson stated that Anderson ``should
present a [plan] that achieves what the Commission is trying to achieve through
its comments/suggestions at these meetings'' and stated that ``architectural
screens, fountains, paving of brick, wood or other similar methods of screening
in lieu of vegetative landscaping are examples of design suggestions that can be
used to break up the front facade.'' Commissioner Davis objected to the front
facade, stating that he could not see putting an expanse of glass facing Gilman
Boulevard. ``The building is not compatible with Gilman.'' Commissioner O'Shea
agreed. Commissioner Nash stated that ``the application needs major changes to
be acceptable.'' Commissioner O'Shea agreed. Commissioner Nash stated that
``this facade does not create the same feeling as the building/environment
around this site.''

Commissioner Nash continued, stating that he ``personally liked the introduction
of brick and the use of tiles rather than metal on the roof.'' Commissioner
Larson stated that he would like to see a review of the blue to be used: ``Tahoe
blue may be too dark.'' Commissioner Steinwachs agreed. Commissioner Larson
noted that ``the front of the building could be modulated [to] have other design
techniques employed to make the front facade more interesting.''

With this, the Development Commission voted to continue the discussion to a
future hearing.

On February 15, 1989, Anderson came back before the Development Commission. In
the meantime, Anderson's architects had added a 5-foot overhang and a 7-foot
accent overhang to the plans for the front of the building. More brick had been
added to the front of the building. Wood trim and accent colors had been added
to the back of the building and trees were added to the landscaping to further
break up the rear facade.

Anderson explained the plans still called for large, floor to ceiling windows as
this was to be a retail premises: ``[A] glass front is necessary to rent the
space\ldots''. Commissioner Steinwachs stated that he had driven Gilman
Boulevard and taken notes. The following verbatim statement by Steinwachs was
placed into the minutes:
\begin{quotation}
\noindent ``My General Observation From Driving Up and Down Gilman Boulevard.''

I see certain design elements and techniques used in various combinations in
various locations to achieve a visual effect that is sensitive to the
unique character of our Signature Street. I see heavy use of brick, wood, and
tile. I see minimal use of stucco. I see colors that are mostly earthtones,
avoiding extreme contrasts. I see various methods used to provide modulation in
both horizontal and vertical lines, such as gables, bay windows, recesses in
front faces, porches, rails, many vertical columns, and breaks in roof lines. I
see long, sloping, conspicuous roofs with large overhangs. I see windows with
panels above and below windows. I see no windows that extend down to floor
level. This is the impression I have of Gilman Boulevard as it relates to
building design.
\end{quotation}

Commissioner Nash agreed stating, ``There is a certain feeling you get when you
drive along Gilman Boulevard, and this building does not give this same
feeling.'' Commissioner Steinwachs wondered if the applicant had any option but
to start ``from scratch.'' Anderson responded that he would be willing to change
from stucco to wood facing but that, after working on the project for 9 months
and experiencing total frustration, he was not willing to make additional design
changes.

At that point, the Development Commission denied Anderson's application, giving
four reasons:
\begin{quotation}
1. After four [sic] lengthy review meetings of the Development
Commission, the applicant has not been sufficiently responsive to concerns
expressed by the Commission to warrant approval or an additional continuance of
the review.

2. The primary concerns expressed relate to the building architecture as
it relates to Gilman Boulevard in general, and the immediate neighborhood in
particular. 

3. The Development Commission is charged with protecting, preserving and
enhancing the aesthetic values that have established the desirable quality and
unique character of Issaquah, reference IMC 16.16.010C.3

4. We see certain design elements and techniques used in various
combinations in various locations to achieve a visual effect that is
sensitive to the unique character of our Signature Street. On Gilman Boulevard
we see heavy use of brick, wood and tile. We see minimal use of stucco. We see
various methods used to provide both horizontal and vertical modulation,
including gables, breaks in rooflines, bay windows, recesses and protrusions in
front face. We see long, sloping, conspicuous roofs with large overhangs. We see
no windows that extend to ground level. We see brick and wood panels at
intervals between windows. We see earthtone colors avoiding extreme contrast.
\end{quotation}

Anderson, who by this time had an estimated \$250,000 into the project, timely
appealed the adverse ruling to the Issaquah City Council (City Council). After a
lengthy hearing and much debate, the City Council decided to affirm the
Development Commission's decision by a vote of 4 to 3.\ldots

Anderson filed a complaint in King County Superior Court. \ldots

Following trial, the court dismissed Anderson's complaint, rejecting the same
claims now raised in this appeal.


\readinghead{Discussion}

\ldots.

\readinghead{2. Constitutionality of IMC 16.16.060 (Building Design
Provisions).}

[A] statute which either forbids or requires the doing of an act in terms so
vague that men [and women] of common intelligence must necessarily guess at its
meaning and differ as to its application, violates the first essential of
due process of law.

\emph{Connally v. General Constr. Co.}, 269 U.S. 385, 391 (1926). In the field
of regulatory statutes governing business activities, statutes which employ
technical words which are commonly understood within an industry, or which
employ words with a well- settled common law meaning, generally will be
sustained against a charge of vagueness. The vagueness test does not require a
statute to meet impossible standards of specificity.

In the area of land use, a court looks not only at the face of the ordinance but
also at its application to the person who has sought to comply with the
ordinance and/or who is alleged to have failed to comply. The purpose of the
void for vagueness doctrine is to limit arbitrary and discretionary enforcements
of the law.

Looking first at the face of the building design sections of IMC 16.16.060, we
note that an ordinary citizen reading these sections would learn only that a
given building project should bear a good relationship with the Issaquah Valley
and surrounding mountains; its windows, doors, eaves and parapets should be of
``appropriate proportions'', its colors should be ``harmonious'' and seldom
``bright'' or ``brilliant''; its mechanical equipment should be screened from
public view; its exterior lighting should be ``harmonious'' with the building
design and ``monotony should be avoided.'' The project should also be
``interesting''. IMC 16.16.060(D)(1)- (6). If the building is not ``compatible''
with adjacent buildings, it should be ``made compatible'' by the use of screens
and site breaks ``or other suitable methods and materials.'' ``Harmony in
texture, lines, and masses [is] encouraged.'' The landscaping should provide an
``attractive\ldots transition'' to adjoining properties. IMC
16.16.060(B)(1)-(3).

As is stated in the brief of \emph{amicus curiae}, we conclude that these code
sections ``do not give effective or meaningful guidance'' to applicants, to
design professionals, or to the public officials of Issaquah who are responsible
for enforcing the code. Although it is clear from the code sections here at
issue that mechanical equipment must be screened from public view and that,
probably, earthtones or pastels located within the cool and muted ranges of the
color wheel are going to be preferred, there is nothing in the code from which
an applicant can determine whether his or her project is going to be seen by the
Development Commission as ``interesting'' versus ``monotonous'' and as
``harmonious'' with the valley and the mountains. Neither is it clear from the
code just what else, besides the valley and the mountains, a particular project
is supposed to be harmonious with, although ``harmony in texture, lines, and
masses'' is certainly encouraged. IMC 16.16.060(B)(2).

In attempting to interpret and apply this code, the commissioners charged with
that task were left with only their own individual, subjective ``feelings''
about the ``image of Issaquah'' and as to whether this project was
``compatible'' or ``interesting''. The commissioners stated that the City was
``making a statement'' on its ``signature street'' and invited Anderson to take
a drive up and down Gilman Boulevard and ``look at good and bad examples of what
has been done wit h flat facades.'' One commissioner drove up and down Gilman,
taking notes, in a no doubt sincere effort to define that which is left
undefined in the code.

The point we make here is that neither Anderson nor the commissioners may
constitutionally be required or allowed to guess at the meaning of the code's
building design requirements by driving up and down Gilman Boulevard looking at
``good and bad'' examples of what has been done with other buildings, recently
or in the past. We hold that the code sections here at issue are
unconstitutionally vague on their face. The words employed are not technical
words which are commonly understood within the professional building design
industry. Neither do these words have a settled common law meaning.

As they were applied to Anderson, it is also clear the code sections at issue
fail to pass constitutional muster. Because the commissioners themselves had no
objective guidelines to follow, they necessarily had to resort to their own
subjective ``feelings''. The ``statement'' Issaquah is apparently trying to make
on its ``signature street'' is not written in the code. In order to be
enforceable, that ``statement'' must be written down in the code, in
understandable terms. The unacceptable alternative is what happened here. The
commissioners enforced not a building design code but their own arbitrary
concept of the provisions of an unwritten ``statement'' to be made on Gilman
Boulevard. The commissioners' individual concepts were as vague and undefined as
those written in the code. This is the very epitome of discretionary, arbitrary
enforcement of the law. \ldots

As well illustrated by the appendices to the brief of \emph{amicus curiae},
aesthetic considerations are not impossible to define in a code or ordinance.
Moreover, the procedural safeguards contained in the Issaquah Municipal Code
(providing for appeal to the City Council and to the courts) do not cure the
constitutional defects here apparent.\ldots

Certainly, the IMC grants Anderson the right to appeal the adverse decision of
the Development Commission. But just as IMC 16.16.060 provides no standards by
which an applicant or the Development Commission or the City Council can
determine whether a given building design passes muster under the code, it
provides no ascertainable criteria by which a court can review a decision at
issue, regardless of whether the court applies the arbitrary and capricious
standard as the City argues is appropriate or the clearly erroneous standard as
Anderson argues is appropriate. Under either standard of review, the appellate
process is to no avail where the statute at issue contains no ascertainable
standards and where, as here, the Development Commission was not empowered to
adopt clearly ascertainable standards of its own. The procedural safeguards
provided here do not save the ordinance.\ldots

Clearly, however, aesthetic standards are an appropriate component of land use
governance. Whenever a community adopts such standards they can and must be
drafted to give clear guidance to all parties concerned. Applicants must have an
understandable statement of what is expected from new construction. Design
professionals need to know in advance what standards will be acceptable in a
given community. It is unreasonable to expect applicants to pay for repetitive
revisions of plans in an effort to comply with the unarticulated,
unpublished ``statements'' a given community may wish to make on or off
its ``signature street''. It is equally unreasonable, and a deprivation of due
process, to expect or allow a design review board such as the Issaquah
Development Commission to create standards on an ad hoc basis, during the design
review process.


\readinghead{Conclusion}

It is not disputed that Anderson's project meets all of the City's land use
requirements except for those unwritten and therefore unenforceable requirements
relating to building design which the Development Commission unsuccessfully
tried to articulate during the course of several hearings. We order that
Anderson's land use certification be issued, provided however, that those
changes which Anderson agreed to through the hearing before the City Council may
validly be imposed.

