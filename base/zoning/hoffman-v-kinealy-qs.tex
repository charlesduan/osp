\item
As the opinion notes, states are divided on whether amortization is an
acceptable technique to deal with nonconforming uses. See cases collected at
\emph{Annotation, Validity of Provisions for Amortization of Nonconforming
Uses}, 22 A.L.R. 3d (1968 \& Supp. 1990). 

Why not allow amortization? Consider the following hypothetical: Troy Barnes and
Abed Nadir each buy a parcel of unzoned land for \$100,000, each expecting to
use the land for a business. Barnes constructs a building for \$50,000, while
Nadir holds off while he develops his filmmaking career. Barnes' business opens,
making \$20,000 net each year. Five years after Barnes' business opens, the
jurisdiction converts the zoning to residential only. Each parcel, used for
residences, is worth only \$15,000. If Barnes is given an amortization period of
five more years, what is the result for Barnes, assuming the building can't be
converted to a residence? How much has Nadir lost? What justifies treating their
situations differently?

\item
Jurisdictions that reject amortization may face some pressure to limit what
counts as a nonconforming use. See, e.g., \emph{University City v. Diveley Auto
Body Co., Inc.}, 417 S.W.2d 107 (Mo. 1967) (holding that a zoning ordinance
requiring the owner of a signboard to comply with its provisions within three
years was a regulation of existing property and not a taking); \emph{St. Charles
County v. St. Charles Sign \& Elec., Inc.}, 237 S.W.3d 272 (Mo. Ct. App. 2007)
(finding that an ordinance mandating that businesses storing inventory outdoors
consisting of ``reclaimed, junked, salvaged, scrapped or otherwise previously
used inventory'' must enclose such storage with fencing was a reasonable
exercise of the police power but not a zoning ordinance, and therefore no prior
nonconforming use exception was required).

\item
\textbf{Terminating a nonconforming use}. Many situations can justify the end of
a nonconforming use exception for a particular parcel. \emph{City of Sugar Creek
v. Reese}, 969 S.W.2d 888 (Mo. Ct. App. 1998):
\begin{quote}
In determining the legislative intent, courts consider that ``the spirit of
zoning ordinances always has been and still is to diminish and decrease
nonconforming uses.'' Thus, courts have allowed municipalities to regulate and
limit nonconforming uses by various means such as prohibiting the resumption of
a nonconforming use after its abandonment or discontinuance, prohibiting the
rebuilding or alteration of nonconforming structures or structures occupied for
nonconforming uses and prohibiting or rigidly restricting a change from one
nonconforming use to another. 
\end{quote}
The Missouri Municipal League, Planning and Zoning Procedures for Missouri
Municipalities (Sept. 2004), adds that prohibiting enlargement or extension of a
nonconforming use is also common. Some zoning ordinances also requires owners of
nonconforming uses to receive permits within a certain period after the adoption
of the change that makes the use nonconforming, on pain of losing the right to
the nonconforming use if they don't get the permit. \textit{City of Sugar Creek}
held that such rules aren't prohibited amortization: the existing property right
that is protected by the no-amortization rule is the right to the specific
existing use, rather than the right to change uses at will. \textit{See also}
\emph{City of Belton v. Smoky Hill Railway \& Historical Society, Inc.}, 170
S.W.3d 429 (Mo. Ct. App. 2005) (discontinuance of use for several years meant
that prohibition on resuming nonconforming use was not an unconstitutional
taking).

\item
What about a change of ownership? Missouri holds that a transfer or change of
ownership is not an abandonment of the right to a non-conforming use, because
the use follows the land and not the person. \emph{Walker v. City of Kansas
City, Missouri}, 697 F.Supp. 1088 (W.D. Mo. 1988). Could you plausibly argue
otherwise?

\item
\textbf{Uses and rezoning close in time}. The not uncommon situation in which a
zoning change is motivated by the appearance of a new, unpopular use is
illustrated by \textit{People Tags, Inc. v. Jackson County Legislature}, 636
F.Supp. 1345 (W.D. Mo. 1986), in which People Tags opened an adult bookstore,
adult motion picture theater and adult mini motion picture theater within 1,500
feet of a church. Thereafter, the Jackson County legislature passed an ordinance
precluding adult bookstore, adult motion picture theater, or adult mini motion
picture theaters from being located within 1,500 feet of any church or school,
with 120 days allowed for noncompliant businesses to come into
compliance.\footnote{Ed. note: the law relating to First Amendment limits on
state regulation of sexually-oriented businesses is extensive. When regulations
are framed as zoning laws limiting the location of such businesses, they are
often but not always upheld as reasonable time, place, and manner restrictions.
\textit{See, e.g.}, \emph{City of Erie v. Pap's A.M.}, 529 U.S. 277 (2000);
\emph{City of Renton v. Playtime Theatres, Inc.}, 475 U.S. 41 (1986);
\emph{Young v. American Mini Theatres, Inc.}, 427 U.S. 50. In the \textit{People
Tags} case, the court found this particular regulation unconstitutional because
it operated to suppress an existing business, not just determine the location of
future businesses. \textit{See also} \emph{Larkin v. Grendel's Den}, 459 U.S.
116 (1982) (Massachusetts statute prohibiting sale of alcohol within 500 feet of
a church ``if the governing body of such church or school files written
objection thereto'' was an unconstitutional establishment of religion under the
First Amendment).} Even in a jurisdiction allowing amortization, would 120 days
be sufficient? 

In \textit{People Tags}, the court rejected the legislature's argument that the
business was not open long enough to constitute a legitimate nonconforming use.
The legislature cited \emph{Pearce v. Lorson}, 393 S.W.2d 851 (Mo. Ct. App.
1965), in which a chiropodist bought a single family home in a residential area
and placed a sign in the window which read ``Dr. R.C. Pearce, Chiropodist, Foot
Specialist.'' He had his office at another location and continued his practice
at that location throughout the time at issue, but he moved a chair and some
supplies into the new building. He also treated one patient in the new office
one hour before a new zoning ordinance banned medical offices in the area. The
\textit{Pearce} court held that Dr. Pearce hadn't established a nonconforming
use before the ordinance passed and that his efforts to do so were a sham. The
\textit{People Tags} court distinguished \textit{Pearce}: the adult bookstore
opened on September 5, 1984, and the legislature passed the first ordinance
requiring it to shut down on September 10, 1984. There was no evidence that the
bookstore wasn't open during regular business hours or didn't have a reasonable
inventory in that time. Nor did the bookstore open in response to the
anticipated passage of a new zoning ordinance. Thus, the bookstore was a
protected nonconforming use.

By contrast, \textit{Acton v. Jackson County}, 854 S.W.2d 447 (Mo. Ct. App.
1993), involved a massage parlor that was a nonconforming use. When the county
determined that the proprietor had expanded the massage parlor's activities to
the illegal activity of prostitution, that expansion ``changed the character of
the nonconforming use and, hence, discontinued it.'' Why not just require the
operator to resume non-illegal operations? Would it matter if there were
evidence that the massage parlor was also being used for prostitution since its
inception, before it became a nonconforming use? The court commented that
nonconforming uses ``are not favored in law because of their interference with
zoning plans. Policy dictates that they should not endure any longer than
necessary and should be eliminated as quickly as justice will permit.'' Thus,
zoning ordinances should be strictly construed against them, including ``rigidly
restricting a change from one nonconforming use to another.'' \textit{See also}
\emph{Huff v. Board of Adjustment of City of Independence}, 695 S.W.2d 166 (Mo.
Ct. App.1985). Relatedly, the burden of proving a nonconforming use is on the
party asserting the right. \emph{In re Coleman Highlands}, 777 S.W.2d 621 (Mo.
Ct. App.1989). Are these rules consistent with the heavily pro-property rights
rhetoric in the principal case?

\item
Despite this general distrust of nonconforming uses, not all changes or
suspension of operations will deprive the owner of the right to continue the
use. \textit{See} \emph{State ex rel. Keeven v. City of Hazelwood}, 585 S.W.2d
557 (Mo. Ct. App. 1979) (city that refused to renew liquor permit or act on
liquor store owner's application for special use permit could not claim that
nonconforming use as liquor store ended while owner was trying to comply with
licensing law). \textit{But see} \emph{Matthews v. Pernell}, 582 N.E.2d 1075
(Ohio Ct. App. 1990) (where nonconforming massage parlor was shut down for a
year because of prostitution on the premises, illegality prevented resumption of
nonconforming use).

\item
\textbf{Vested rights}. As \textit{People Tags} indicates, it can be vitally
important to determine which came first, the use or the zoning that makes it a
nonconforming use. Must the use be in full swing to trigger a property owner's
right to continue the use? Even a state that allows amortization will confront
this question, because it will determine whether an amortization period must be
allowed.

In general, a use that is in progress may be a prior nonconforming use if
sufficient commitments have been made, such as the construction of a building
(with the then-proper permits). In Missouri, as in most states, filing a permit
application under a prior zoning regime is insufficient, even if the owner
bought the land in anticipation of the use and preparing the application
required the investment of resources. \textit{See} \emph{State ex rel. Lee v.
City of Grain Valley}, 293 S.W.3d 104 (Mo. Ct. App. 2009) (``To establish a
nonconforming use, one must have at least made a substantial step, and a `mere
preliminary work which is not of a substantial nature does not constitute a
nonconforming use.'\,''). Even receiving a permit is insufficient, if the work
completed towards converting the land to the particular use isn't substantial.
\textit{See} \emph{Outcom, Inc. v. City of Lake St. Louis}, 996 S.W.2d 571 (Mo.
Ct. App. 1999); \textit{see also} \emph{Storage Masters-Chesterfield, L.L.C. v.
City of Chesterfield}, 27 S.W.3d 862 (Mo. Ct. App. 2000) (construction of sign
that was intended to be illuminated, but was not illuminated, before rezoning
did not establish prior nonconforming use; ``mere intention does not give rise
to a vested property right''). \textit{But see} \textsc{Wash. Rev. Code}
\S~58.17.033 (rights under zoning ordinance vest as of the filing of a ``valid
and fully complete building permit application'').

What should be the result when a city issues a permit in error, and the
developer relies on the permit to start building? In \textit{Parkview Associates
v. City of New York}, 519 N.E.2d 1372 (N.Y. 1988), the city and the developer
both misinterpreted a zoning map---they looked at an unlabeled version of a map
instead of the written description of the same area in the zoning
regulation---and the city gave Parkview a permit for a 31-story apartment
building where it was only zoned for 19 stories. Parkview began construction.
After ``substantial'' construction, the city discovered the error and issued a
stop work order for the top 12 stories, but Parkview kept building. New York's
highest court ruled that ``reasonable diligence by a good-faith inquirer would
have disclosed the true facts and the bureaucratic error,'' and held that
estoppel was not available against the government. The extra stories had to be
torn down at a cost of roughly \$14 million. \textit{Should} estoppel be
available against the government? \textit{Cf.} \emph{State ex rel. Casey's
General Stores, Inc. v. City of Louisiana}, 734 S.W.2d 890 (Mo. Ct. App.1987)
(applying equitable estoppel where city was consulted and gave assurances as to
a building permit). \textit{But see} \emph{Long v. Bd. of Adjustment of City of
Columbia}, 856 S.W.2d 390 (Mo. App. 1993) (estoppel does not apply to acts of
government, including acts relating to zoning); \emph{Lichte v. Heidlage}, 536
S.W.2d 898 (Mo. App. 1976). Who suffers if the government's error can't be
fixed?

The government's error, however, may justify the grant of a variance allowing
the continued use in appropriate circumstances, where that error creates
sufficient individualized hardship. \textit{See}
\having{intro-variances}{Variances, above}{Variances, below}{Variances, in
\textsc{Open Source Property}}; \emph{Taylor v. Board of Zoning Adjustment}, 738
S.W.2d 141 (Mo. Ct. App. 1987) (grant of variance held appropriate due to zoning
board's prior erroneous grant of permit resulting in \$7,000 expenditure for
oversized sign later subject to permit revocation for zoning violation).

\item
\textbf{Vested rights in easy-to-change uses?} In Missouri, the nonconforming
use itself need not be one that requires substantial investment, if there is no
doubt it precedes the enactment of the relevant regulation. In \textit{Rose v.
Board of Zoning Adjustment Platte County}, 68 S.W.3d 507 (Mo. Ct. App. 2001),
Platte County found David Rose in violation of the county's Weed Ordinance for
allowing uncultivated weeds to grow more than twelve inches high on his
residential property. Rose bought his property in 1976, before the Weed
Ordinance was enacted; he had a degree in wildlife management and ten years of
work experience as a wetlands manager with the United States Fish and Wildlife
Service. He decided to transform the cut-grass yard surrounding his home into a
natural woodlands area: He planted additional trees, shrubs and flowering plants
and allowed the natural vegetation in the yard to grow. He did not trim or mow
the yard. Over the years, the vegetation ``matured into a wooded state.'' 

Eventually, ``the uncultivated condition of Rose's yard led to an investigation
and complaints by the Platte County codes enforcement officer.'' In 1991, Rose
was criminally charged with violating the county's nuisance ordinance for
allowing noxious weeds (such as poison ivy and oak) to grow on his property,
maintaining other weeds and wooden boards conducive to breeding insects and
rodents, and having a decaying wooden deck in a dangerous condition. A jury
acquitted Rose on all charges. The codes enforcement officer complained three
more times, but the county prosecutor declined to pursue further criminal
charges, and in 1999 the county replaced the nuisance ordinance with its new
Weed Ordinance, requiring the removal of ``weeds'' from any parcel of land not
zoned for agricultural use. The county found Rose to be in violation of the new
ordinance; Rose argued that his prior nonconforming use was protected against
suppression. The court of appeals found that there was a dispute over whether
Rose had expanded his nonconforming use by allowing the vegetation to ``become
more dense and overgrown subsequent to the passage of the Weed Ordinance,'' and
held that he was entitled to a hearing on the matter.

Should the court have even allowed Rose to claim a prior nonconforming use? In a
state that allowed amortization, what sort of amortization period should Rose
have been allowed?

