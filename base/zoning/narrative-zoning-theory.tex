\paragraph{The dominance of the single-family home} Americans love their homes,
and homeownership remains a cornerstone of the ``American dream.'' Alexis de
Tocqueville noted this several hundred years ago, and also commented that
Americans would build homes and sell them as soon as the roof was complete. A
particular ideal of the home developed in the twentieth century: ``A separate
house surrounded by a yard is the ideal kind of home.'' \textsc{Mary Lockwood
Matthews, Elementary Home Economics} (1931). As a Wilmington, Delaware real
estate ad from 1905 instructed, ``Get your children into the country. The cities
murder children. The hot pavements, the dust, the noise are fatal in many cases,
and harmful always. The history of successful men is nearly always the history
of country boys.''

Results from the 2013 American Household Survey (AHS),
\url{http://www.census.gov/programs-surveys/ahs/data/2013/metropolitan-summary-tables---ahs-2013.html},
show that 64\% of all occupied housing, and 62\% of recently built units, are
detached single-family homes. Even in central cities, 79\% of owner-occupied
units are detached single-family houses. The average owner-occupied dwelling
takes up nearly a third of an acre, as does the average recently built dwelling;
bus service usually requires at least seven dwellings per acre to be
viable.\footnote{Only 55\% of housing units have sidewalks, and the percentage
is lower for over-65 homeowners.} 

Homeownership has definite benefits. Homeowners are more likely to support
school funding; even childless homeowners want their chief asset to be valuable
because of its proximity to good schools. Homeowners participate more in local
politics and community life than renters do, and their children seem to benefit
as well. On the other hand, homeownership can be an anchor---when the structure
of employment changes radically, and the best jobs are available in other
regions, homeownership, and the resulting loss on a major asset, can deter
people from moving, depressing economic growth and individual income.

\paragraph{Segregation of uses} The key principle behind Euclidean zoning is
segregation of uses, in order to protect the single-family home. One clear cost
is sprawl. Living away from density has other consequences: Wages are about
thirty-five percent higher in cities, and research shows that this is because
urban residents tend to have greater wage growth than residents in rural areas,
suggesting that growth in human capacity is enhanced by density and interacting
with closely situated others. Density allows for greater specialization and more
productive interactions with a greater variety of people. Another consequence of
use segregation is that undesirable uses tend to get concentrated in ghettoes or
red-light districts, or left to inner cities.

However, even opponents of Euclidean zoning might consider some segregation of
uses desirable. In 2013, a Texas fertilizer plant explosion leveled houses and
destroyed the middle school across the street. A former city council member said
that he couldn't recall the town discussing whether it was a good idea to build
houses and the school so close to the plant, which has been there since 1962.
``The land was available out there that way\ldots There never was any thought
about it. Maybe that was wrong.'' Theodoric Meyer, \emph{Could regulators have
prevented the Texas fertilizer plant explosion?},
\textsc{Salon}, Apr. 28, 2013,
\url{http://www.salon.com/2013/04/28/where_were_the_regulators_before_the_texas_fertilizer_plant_explosion_partner/}.

\paragraph{Churches} It might fairly be said that many homevoters' concern for
their property values amounts to religious fervor. Numerous zoning disputes have
involved the location of churches, to which neighbors often object on grounds of
weekend congestion---or, in the case of minority religions, for other reasons.
\emph{Congregation Temple Israel v. City of Creve Coeur}, 320 S.W.2d 451 (Mo.
1959), involved a religious organization (a Jewish synagogue) that wished to
construct a new building for religious purposes, including services and
religious education. Two weeks after Temple Israel bought the land, residents
petitioned to change the zoning. Before Temple Israel began construction, the
City changed the zoning to exclude churches and schools. It also established a
complex and burdensome procedure to seek an exception allowing church or school
use, and made the exception discretionary rather than mandatory. The Missouri
Supreme Court ruled that municipalities had no authority to regulate the
placement of churches or schools. Under the state's Zoning Enabling Act, Section
89.020 allowed them to regulate ``the location and use of buildings, structures
and land for trade, industry, residence and other purposes.'' Given the
constitutional interest in freedom of religion, and the history of locating
churches in residential areas, the court interpreted ``other purposes'' to
exclude control over the location and use of buildings for churches and schools,
though municipalities could regulate the buildings for health and safety
purposes.

The land use provisions of the Religious Land Use and Institutionalized Persons
Act of 2000 (RLUIPA), 42 U.S.C. \S\S~2000cc, et seq., now protect individuals,
houses of worship, and other religious institutions from discrimination in
zoning and landmarking laws. The Department of Justice has explained:
\begin{quotation}
Religious assemblies, especially, new, small, or unfamiliar ones, may be
illegally discriminated against on the face of zoning codes and also in the
highly individualized and discretionary processes of land use regulation. Zoning
codes and landmarking laws may illegally exclude religious assemblies in places
where they permit theaters, meeting halls, and other places where large groups
of people assemble for secular purposes. Or the zoning codes or landmarking laws
may permit religious assemblies only with individualized permission from the
zoning board or landmarking commission, and zoning boards or landmarking
commission may use that authority in illegally discriminatory ways.

To address these concerns, RLUIPA prohibits zoning and landmarking laws that
substantially burden the religious exercise of churches or other religious
assemblies or institutions absent the least restrictive means of furthering a
compelling governmental interest. This prohibition applies in any situation
where: (i) the state or local government entity imposing the substantial burden
receives federal funding; (ii) the substantial burden affects, or removal of the
substantial burden would affect, interstate commerce; or (iii) the substantial
burden arises from the state or local government's formal or informal procedures
for making individualized assessments of a property's uses. 
\end{quotation}
U.S. Dep't of Justice, Religious Land Use and Institutionalized Persons Act,
Aug. 6, 2015,
\url{http://www.justice.gov/crt/religious-land-use-and-institutionalized-persons-act}.

\paragraph{Longstanding critiques of suburbia} Since their inception, suburbs
have been criticized for isolating and insulating the families who lived there.
Social critic Louis Mumford wrote: ``[T]he suburb served as an asylum for the
preservation of illusion. Here domesticity could flourish, forgetful of the
exploitation on which so much of it was based. Here individuality could prosper,
oblivious of the pervasive regimentation beyond. This was not merely a
child-centered environment, it was based on a childish view of the world, in
which reality was sacrificed to the pleasure principle.'' \textsc{The City in
History: Its Origins, Its Transformations, and Its Prospects} 464 (1961).

Zoning raises distributional as well as efficiency concerns. Proponents of use
zoning defend its contribution to ``home values,'' while critics of growth
restrictions talk about ``housing prices''; the former takes the perspective of
existing owners while the latter suggests more concern for people who are priced
out of ownership. Indeed, use zoning does seem to raise the price of
single-family homes, though it's less clear that it raises overall property
values. Studies find that, in most parts of the country, home prices are roughly
at or near the costs of construction. But, where zoning limits construction,
prices can increase substantially. Thus, in heavily regulated urban areas like
New York City and many parts of California, home prices shot up in the past few
decades. 

A recent study found that land use restrictions added \$200,000 to the price of
houses in Seattle, Washington; Seattle was in the top 3\%, nationally, in
approval delays for new projects. The executive officer of the Master Builders
Association of King \& Snohomish Counties estimated that regulatory costs
comprised up to 30 percent of the total cost of building a new house (land costs
included), including transportation, school and park impact fees, stormwater
management fees, critical-areas mitigation and monitoring, pavement requirements
and rockery permits. Neighborhood-based design review committees, which use
citizen volunteers, delay the process further, sometimes requiring three or four
rounds of review. Elizabeth Rhodes, \emph{UW study: Rules add \$200,000 to
Seattle house price,} Seattle Times, Feb. 14, 2008,
\url{http://www.seattletimes.com/business/uw-study-rules-add-200000-to-seattle-house-price/}.


