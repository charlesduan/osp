\reading{Hoffmann v. Kinealy}

\readingcite{389 S.W.2d 745 (Mo. 1965)}

\opinion \textsc{A. P. Stone}, Jr., Special Judge.

This is an appeal by Carl O. Hoffmann, Jr., and Mrs. Geraldine St. Denis (herein
called relators), the owners of two adjoining lots (frequently referred to as
the lots) in the 3100 block of Pennsylvania in the City of St. Louis, from the
judgment of the Circuit Court of the City of St. Louis affirming, upon review by
certiorari, a decision of the board of adjustment sustaining a decision of the
building commissioner which denied relators' application for a certificate of
occupancy of the lots for a pre-existing lawful nonconforming use, to wit, for
the open storage of lumber, building materials and construction equipment.

\ldots Portions of the block, i.e., that portion in which the lots are located,
[and certain other parcels], are in a ``B'' two-family dwelling district, while
the remainder of the block,\ldots is in a ``J'' industrial district and is used
for the operation of a planing mill and for open storage of lumber. A small
building housing the general offices of Hoffmann Construction Company, relators'
business in connection with which the lots have been used, is located in the
``B'' two-family dwelling district\ldots just across the alley from the lots.


\captionedgraphic{zoning-img007}{Google Earth image, 2015, with contested block
in center}

The exhibits presented at the hearing before the board of adjustment, and
brought to us with the transcript on appeal, indicate that there are fourteen
buildings in the same portion of the block in which the lots are situate,
including a tavern\ldots , one three-family residence, eleven other residences,
and at the rear of one residence a building identified on a plat as used for
``tractor parts''; ten buildings in that portion of the block\ldots , including
a grocery store\ldots , eight residences (all owned by relators), and at the
rear of one residence the above-mentioned office building of Hoffmann
Construction Company; and that, on the other three corners\ldots , there are two
taverns and a cleaning and pressing shop.

Counsel for the city conceded at the hearing before the board of adjustment, and
the subsequent finding of the board (not here disputed) was, that the lots were
being used at the time of hearing for the open storage of lumber, building
materials and construction equipment and that (in the language of the board's
finding) ``these premises have been used for this same purpose continuously
since the year 1910.'' The front end of the lots is ``landscaped'' with a hedge
and shrubbery, and the area used for open storage is enclosed with a high fence.

The first comprehensive zoning ordinance of the City of St. Louis became
effective in 1926. On April 25, 1950, numerous sections of the zoning code were
amended by Ordinance 45309. Section 5~A~1 of that ordinance provided that ``No
building or land shall be used for a use other than those permitted in the
district in which such premises are located unless\ldots such use existed prior
to the effective date of this ordinance.'' Section 5~B of the same
ordinance\ldots provided that ``The use of land within any dwelling
district\ldots for purposes of open storage\ldots which do not conform to the
provisions of this ordinance shall be discontinued within six (6) years from the
effective date of this ordinance.''

About six years and three months later, to wit, on July 24, 1956, Ordinance
48007 was enacted, amending that portion of Section 5~B of Ordinance 45309, with
which we are here concerned, to read as follows: ``The use of land within any
dwelling district for the purpose of open storage is hereby prohibited.'' [The
code was subsequently revised, but not in any way that changed this provision,
and the relevant provision was renumbered as Section 903.030.]

\ldots Relators' petition in the circuit court, upon which the writ of
certiorari was issued, charged that Section 903.030 of the zoning code was
unconstitutional, null and void and was of no effect as to relators' lots
because, by prohibiting continuance of the pre-existing lawful nonconforming use
of the lots, said section would impair, restrict and deprive relators of vested
property rights and thereby would take and damage relators' private property for
public use without just compensation in violation of Article 1, Section 26,
Missouri Constitution of 1945. 

\ldots Respondants' position is that, under the statutory grant of police power
in municipal zoning and planning, the city was empowered to enact\ldots a
so-called ``amortization'' or ``toleration'' provision which required
discontinuance within six years thereafter of the nonconforming use of land
within any dwelling district for purposes of open storage, and that, such
six-year ``amortization'' or ``toleration'' period having run in April 1956, the
subsequent absolute prohibition of said nonconforming use of land\ldots was
valid.

\ldots Of course, it has long been settled that a comprehensive zoning ordinance
operating prospectively, which has a substantial relationship to the public
health, safety, morals or general welfare and is not unreasonable or
discriminatory, is valid as a proper exercise of the police power. This is so
even though, in restricting future uses, any such ordinance may impose hardship
and inflict economic loss upon some property owners, for it is recognized that
``[e]very valid exercise of the police power is apt to affect the property of
some one adversely.''

In earlier days of zoning legislation, it generally was recognized and conceded
that termination of pre-existing lawful nonconforming uses would be
unconstitutional\ldots . In \emph{Women's Christian Ass'n. of Kansas City v.
Brown}, 190 S.W.2d 900 (Mo. 1945), involving an attempted change of
nonconforming use from a riding academy to a dance hall, this court said
that:\ldots ``Within a period of another twenty years, a large number of such
`nonconforming' uses will have disappeared, either through the necessity of
enlargement and expansion which invariably is forbidden or limited by ordinance,
or by the owners realizing that it is unwise and uneconomic to be located in a
district which probably is not suitable for the nonconforming purpose, or by
obsolescence, destruction by fire or by the elements or similar inability to be
used; so that many of these nonconforming uses will `fade out,' with a resulting
substantial and definite benefit to all communities.''

\ldots Certainly, the spirit of zoning ordinances always has been and still is
to diminish and decrease nonconforming uses, and to that end municipalities have
employed various approved regulatory methods such as prohibiting the resumption
of a nonconforming use after its abandonment or discontinuance, prohibiting the
rebuilding or alteration of nonconforming structures or structures occupied for
nonconforming uses, and prohibiting or rigidly restricting a change from one
nonconforming use to another. Even so, pre-existing lawful nonconforming uses
have not faded out or eliminated themselves as quickly as had been anticipated,
so zoning zealots have been casting about for other methods or techniques to
hasten the elimination of nonconforming uses. In so doing, only infrequent use
has been made of the power of eminent domain, primarily because of the expense
of compensating damaged property owners, but increasing emphasis has been placed
upon the ``amortization'' or ``tolerance'' technique which conveniently bypasses
the troublesome element of compensation. 

Stated in its simplest terms, amortization contemplates the compulsory
termination of a non-conformity at the expiration of a specified period of time,
which period is equaled (sic) to the useful economic life of the
non-conformity. The basic idea is to determine the remaining normal useful
life of a pre-existing nonconforming use. The owner is then allowed to continue
his use for this period and at the end must either conform or eliminate it.
Courts approving the amortization technique as a valid exercise of the police
power rationalize their holdings in this fashion: ``The distinction between an
ordinance restricting future uses and one requiring the termination of present
uses within a reasonable period of time is merely one of degree, and
constitutionality depends on the relative importance to be given to the public
gain and to the private loss. Zoning as it affects every piece of property is to
some extent retroactive in that it applies to property already owned at the time
of the effective date of the ordinance. The elimination of existing uses within
a reasonable time does not amount to a taking of property nor does it
necessarily restrict the use of property so that it cannot be used for any
reasonable purpose. Use of a reasonable amortization scheme provides an
equitable means of reconciliation of the conflicting interests in satisfaction
of due process requirements. As a method of eliminating existing nonconforming
uses it allows the owner of the nonconforming use, by affording an opportunity
to make new plans, at least partially to offset any loss he might suffer\ldots .
If the amortization period is reasonable the loss to the owner may be small when
compared with the benefit to the public.'' \emph{City of Los Angeles v. Gage},
274 P.2d 34 (Cal. Ct. App. 1954).

Several cases in other jurisdictions have approved the termination of
pre-existing nonconforming uses by the amortization technique. However, there
are a number of decisions to the opposite effect, and it may be fairly said that
there is ``a decided lack of accord'' in this area. 

\ldots But, although the holdings in other jurisdictions may, in some instances,
be enlightening and persuasive, it is neither our duty nor our inclination to
rule a question of first impression in this state simply by counting foreign
cases and then falling off the judicial fence on the side on which more cases
can be found. Rather, our concern should be and is to determine the basic
constitutional right of the matter, as we see it. Property is defined as
including not only ownership and possession but also the right of use and
enjoyment for lawful purposes. In fact, ``[t]he substantial value of property
lies in its use.'' It follows that: ``[t]he constitutional guaranty of
protection for all private property extends equally to the enjoyment and the
possession of lands. An arbitrary interference by the government, or by its
authority, with the reasonable enjoyment of private lands is a taking of private
property without due process of law, which is inhibited by the Constitution.''

\ldots The amortization provision under review would terminate and take from
instant relators the right to continue a lawful nonconforming use of their lots
which has been exercised and enjoyed since 1910---a right of the character to
which the courts traditionally have referred as a ``vested right.'' To our
knowledge, no one has, as yet, been so brash as to contend that such a
pre-existing lawful nonconforming use properly might be terminated immediately.
In fact, the contrary is implicit in the amortization technique itself which
would validate a taking presently unconstitutional by the simple expedient of
postponing such taking for a ``reasonable'' time. All of this\ldots prompts us
to repeat the caveat of Mr. Justice Holmes in \emph{Pennsylvania Coal Co. v.
Mahon}, 260 U.S. 393 (1922), that ``[w]e are in danger of forgetting that a
strong public desire to improve the public condition is not enough to warrant
achieving the desire by a shorter cut than the constitutional way of paying for
the change.''\ldots

\ldots Accordingly, the judgment of the circuit court is set aside and the cause
is remanded with directions to enter judgment ordering respondents, constituting
the board of adjustment of the City of St. Louis, to issue, or cause to be
issued, to relators a certificate of occupancy for continuance of the
pre-existing lawful nonconforming use of relators' lots for the open storage of
lumber, building materials and construction equipment.

\opinion \textsc{Hyde}, Judge (dissenting).

\ldots In the leading case of \emph{Village of Euclid, Ohio v. Ambler Realty
Co.}, 272 U.S. 365, the Court said that zoning and ``all similar laws and
regulations, must find their justification in some aspect of the police power,
asserted for the public welfare.'' The court pointed out the following reasons
for this use of the police power: ``[T]he segregation of residential, business
and industrial buildings will make it easier to provide fire apparatus suitable
for the character and intensity of the development in each section; that it will
increase the safety and security of home life, greatly tend to prevent street
accident, especially to children, by reducing the traffic and resulting
confusion in residential sections, decrease noise and other conditions which
produce or intensify nervous disorders, preserve a more favorable environment in
which to rear children.''\ldots

In view of these applicable principles, it does not seem reasonable to say that
the existence of a particular use of vacant land when a zoning ordinance is
adopted gives the owner a vested right to continue it in perpetuity, especially
the right to pile material on vacant ground.\ldots High piles of stored material
are not conducive to the maintenance or development of a good residential
environment not only because they are unsightly but also because they could
provide a lurking place for thieves and other criminals and also could attract
children who might be injured playing there. While such open storage has not
been classified as a nuisance, it thus has some of the undesirable
characteristics of nuisance in a residential district. Therefore, I would hold
the ordinance in this case, for termination of open storage in residential
districts after six years, a reasonable exercise of the police power and valid.

