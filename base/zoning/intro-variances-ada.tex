Both the Americans with Disabilities Act (ADA) and the Fair Housing Act (FHA)
have provisions that can affect local zoning and variance
procedures.\footnote{The ADA had even more profound effects on local building
codes, which mandate particular building features. Along with fire and
electrical codes, building codes---which specify matters such as the minimum
width of doors and the maximum pitch of stairs---also profoundly shape the built
environment, though we will not separately consider them here. Under the ADA,
new construction of places of public accommodation must be accessible, which
includes considerations such as entrance ramps and Braille labeling.
\textit{See} U.S. Architectural and Transportation Barriers Compliance Board
(Access Board), Americans with Disabilities Act (ADA) Accessibility Guidelines
for Buildings and Facilities (2002),
\url{https://www.access-board.gov/attachments/article/1350/adaag.pdf}.}  People
with disabilities, defined as a substantial impairment to a major life activity
such as walking or seeing, as well as people who are perceived as having
disabilities, are entitled to reasonable accommodations for their disabilities,
which means that otherwise applicable laws and regulations may have to be
waived. 

