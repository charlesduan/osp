A gift, once given, is usually \textit{irrevocable}: once the gift is complete,
the donor may not change her mind and demand the gift back.  The donee may
voluntarily \textit{give} the property back, but that's a second transfer of
ownership.  Irrevocability makes it important to be able to figure out when the
gift was complete, because before it is complete, it is revocable.  

Suppose that Bobby Singer leaves his junkyard to Dean Winchester in his will. 
Dean moves in, then realizes after two days that he doesn't want to run a
junkyard, disclaims the gift, and tells Bobby's residual heirs -- Bobby's
second cousins -- that the property is theirs.  Is it?

Suppose that, on Spencer Hastings's eighteenth birthday, her parents put the
keys to a car in an envelope by her place at the table.  However, before the
birthday meal, the family gets into a screaming fight, and the parents snatch
up the keys from the table and say they no longer think Spencer deserves the
car.  Who owns the car? 

There are two notable exceptions to the irrevocability of a gift: Gifts
\textit{causa mortis} and conditional gifts.  In the materials that follow, we
will explore complications relating to both.  As you read, keep an eye on the
ways in which courts are interpreting the various elements of a gift in order
to achieve an overall result they find appropriate.

