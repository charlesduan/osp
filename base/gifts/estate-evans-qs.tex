\expected{estate-evans}

\item The majority writes, ``regardless of Mr. Evans' intention to make a gift
to appellant, he never executed that intention and we will not do it for him.''
But it also quotes approvingly the lower court's statement that ``[t]he record
clearly manifests, both by his conduct and his statements, [Mr. Evans']
donative intent.'' Has the court contradicted itself, or can these statements
be squared?

\item Why does the court note that Vivian Kellow ``received adequate
compensation'' for the services she provided to Arthur Evans? What were his
motivations for the attempted gift, and why are the appellees contesting it?
Does the family setting shed any light on the positions of the majority and
dissent?


\item The common law required manual delivery of personal property for a valid
gift unless the object was too big to move.  \emph{See, e.g.}, \emph{Newman v.
Bost}, 20 S.E.
848 (N.C. 1898) (symbolic delivery insufficient where objects were small items
that could easily have been physically delivered, even though would-be donor
was ill in bed).  If the object was too big to move, substitutes for physical
delivery were acceptable.  Keys are a classic example: handing over car keys is
``constructive'' or ``symbolic'' delivery of the car.  The keys symbolize the
car (symbolic delivery) and provide the means for exercising dominion and
control over it (constructive delivery).  Today, because all states require car
owners to register the title to their cars, many states require that a gift of
a car is not complete unless the donor also hands over the title documents. 
Why would the law require delivery of the title documents?  What happens when
someone who doesn't know this rule hands over only the keys, and then a year
later changes her mind and demands the car back?  (You should see here how a
title system can both make it easier to determine who owns property and easier
for legally unsophisticated people to make significant mistakes.)

Why isn't saying ``I give you this car'' without delivery enough to complete the
gift?  The keys could be handed over later, after all.  If there's a present
donative intent, what further purpose does a delivery requirement serve?  Most
answers focus on the evidentiary role played by delivery: possession of the
property by the putative donee is strong evidence that the putative donor
really did make a gift.  This is especially important because most gift
disputes arise after the putative donor's death.  Notice to third parties who
deal with the property and need to know who owns it is another common
rationale.  But when might a putative donee's possession not be particularly
probative of whether a gift had occurred?  Suppose a father allows his daughter
to use his second car when she moves to town, and that this continues for six
months.  If, after they have a falling out, the father sought to retrieve the
car, how would you figure out whether this was a loan or a gift? 



\item Modern courts often relax the delivery requirement to allow constructive
or symbolic delivery even of smaller, more portable items, but some delivery
requirement remains.  Suppose the would-be donor signed a document in front of
two witnesses saying ``I now give my daughter \$100,000,'' and gave the
document to his daughter.  But the donor didn't actually deliver the money. 
Should we relax the delivery requirement because we are very confident that a
gift was intended?  Or does delivery still serve an important purpose?
\emph{See} \emph{Devol v. Dye}, 24 N.E. 246 (Ind. 1890) (``The intention of a
donor in peril of
death, when clearly ascertained and fairly consummated within the meaning of
well-established rules, is not to be thwarted by a narrow and illiberal
construction of what may have been intended for and deemed by him a sufficient
delivery.''); \emph{Ferrell v. Stinson}, 11 N.W.2d 701 (Iowa 1943) (deed made
out to
intended donee was kept in box in donor's house, and recorded after grantor's
death; held: delivered given strong evidence of donor's intent and fact that
seriously ill grantor was physically inable to access box after executing
deed); \textit{cf.} \emph{Hocks v. Jeremiah}, 759 P.2d 312 (Or. App. 1988)
(bonds and
diamond ring placed over a period of years in a safe deposit box held jointly
with putative grantee were not properly delivered).  What should have happened
in the \textit{Ferrell} case if the grantor had made out the deed, put it in
the box, and then a week later, still in her sickbed, made out a deed to
another person and handed \textit{that} second deed to the intended grantee?  


\item The Restatement (Third) of Property: Wills and Other Donative Transfers
\S~6.2, Comment yy, takes the position that personal property can be validly
given without delivery ``if the donor's intent to make a gift is established by
clear and convincing evidence.''  Is this the right rule?
\having{gruen-v-gruen}{As you'll see in the
next case, some states require clear and convincing evidence of the presence of
each element for any gift.}{}{}

