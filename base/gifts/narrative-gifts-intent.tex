Although laypeople may not be aware of this distinction, there is a huge legal
difference between ``I will give you this car when you graduate'' and ``I now
give you this car.''  \having{gruen-v-gruen}{(In the next section we will investigate a third variant,
in which the gift is of a future interest.) }{}{}``I will give you this car'' is
a mere promise with no legal force.  No matter how serious the speaker's intent
is, it is not an intent to make a present gift, and it will therefore not
result in a gift.

Intent is rarely an issue in gift cases, but it can arise when it is not clear
what the donor intended to give: Suppose O says to D, ``I want you to have the
jewelry box on my dresser and the jewelry inside,'' believing that she's
storing costume jewelry in the box.  D takes the box, but inside there are no
costume jewels, only a diamond necklace.  What gift has been made?  What if O
says ``I want you to have the jewelry box and its contents,'' and the contents
are bearer bonds worth \$100,000?  If O is deceased when the issue is
litigated, how would you determine her intent?

