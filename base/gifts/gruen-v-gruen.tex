\reading{Gruen v. Gruen}

\readingcite{496 N.E.2d 869 (N.Y. 1986)}

Plaintiff commenced this action seeking a declaration that he is the rightful
owner of a painting which he alleges his father, now deceased, gave to him. He
concedes that he has never had possession of the painting but asserts that his
father made a valid gift of the title in 1963 reserving a life estate for
himself. His father retained possession of the painting until he died in 1980.
Defendant, plaintiff's stepmother, has the painting now and has refused
plaintiff's requests that she turn it over to him. She contends that the
purported gift was testamentary in nature and invalid insofar as the
formalities of a will were not met or, alternatively, that a donor may not make
a valid \textit{inter vivos} gift of a chattel and retain a life estate with a
complete right of possession. Following a seven-day nonjury trial, Special Term
found that plaintiff had failed to establish any of the elements of an
\textit{inter vivos} gift and that in any event an attempt by a donor to retain
a present possessory life estate in a chattel invalidated a purported gift of
it. The Appellate Division held that a valid gift may be made reserving a life
estate and, finding the elements of a gift established in this case \dots{}. We
now affirm.

The subject of the dispute is a work entitled ``Schloss Kammer am Attersee II''
painted by a noted Austrian modernist, Gustav Klimt. It was purchased by
plaintiff's father, Victor Gruen, in 1959 for \$8,000. On April 1, 1963 the
elder Gruen, a successful architect with offices and residences in both New
York City and Los Angeles during most of the time involved in this action,
wrote a letter to plaintiff, then an undergraduate student at Harvard, stating
that he was giving him the Klimt painting for his birthday but that he wished
to retain the possession of it for his lifetime. This letter is not in
evidence, apparently because plaintiff destroyed it on instructions from his
father. Two other letters were received, however, one dated May 22, 1963 and
the other April 1, 1963. Both had been dictated by Victor Gruen and sent
together to plaintiff on or about May 22, 1963. The letter dated May 22, 1963
reads as follows:
\begin{quotation}
\noindent Dear Michael:

I wrote you at the time of your birthday about the gift of the painting by
Klimt.

Now my lawyer tells me that because of the existing tax laws, it was wrong
to mention in that letter that I want to use the painting as long as I live.
Though I still want to use it, this should not appear in the letter. I am
enclosing, therefore, a new letter and I ask you to send the old one back to me
so that it can be destroyed.

I know this is all very silly, but the lawyer and our accountant insist that
they must have in their possession copies of a letter which will serve the
purpose of making it possible for you, once I die, to get this picture without
having to pay inheritance taxes on it.

\noindent Love,

\noindent s/Victor
\end{quotation}

Enclosed with this letter was a substitute gift letter, dated April 1, 1963,
which stated:

\begin{quotation}
\noindent Dear Michael:

The 21st birthday, being an important event in life, should be celebrated
accordingly. I therefore wish to give you as a present the oil painting by
Gustav Klimt of Schloss Kammer which now hangs in the New York living room. You
know that Lazette and I bought it some 5 or 6 years ago, and you always told us
how much you liked it.

Happy birthday again.

\noindent Love,

\noindent s/Victor
\edfootnote{As we will discuss below, Victor Gruen evaded substantial taxes on
this gift by pretending that he did not reserve a life estate.}
\end{quotation}

Plaintiff never took possession of the painting nor did he seek to do so. Except
for a brief period between 1964 and 1965 when it was on loan to art exhibits
and when restoration work was performed on it, the painting remained in his
father's possession, moving with him from New York City to Beverly Hills and
finally to Vienna, Austria, where Victor Gruen died on February 14, 1980.
Following Victor's death plaintiff requested possession of the Klimt painting
and when defendant refused, he commenced this action.

The issues framed for appeal are whether a valid \textit{inter vivos} gift of a
chattel may be made where the donor has reserved a life estate in the chattel
and the donee never has had physical possession of it before the donor's death
and, if it may, which factual findings on the elements of a valid \textit{inter
vivos} gift more nearly comport with the weight of the evidence in this case,
those of Special Term or those of the Appellate Division. The latter issue
requires application of two general rules. First, to make a valid \textit{inter
vivos} gift there must exist the intent on the part of the donor to make a
present transfer; delivery of the gift, either actual or constructive to the
donee; and acceptance by the donee. Second, the proponent of a gift has the
burden of proving each of these elements by clear and convincing evidence.

\readinghead{Donative Intent}

There is an important distinction between the intent with which an \textit{inter
vivos} gift is made and the intent to make a gift by will. An \textit{inter
vivos} gift requires that the donor intend to make an irrevocable present
transfer of ownership; if the intention is to make a testamentary disposition
effective only after death, the gift is invalid unless made by will.

Defendant contends that the trial court was correct in finding that Victor did
not intend to transfer any present interest in the painting to plaintiff in
1963 but only expressed an intention that plaintiff was to get the painting
upon his death. The evidence is all but conclusive, however, that Victor
intended to transfer ownership of the painting to plaintiff in 1963 but to
retain a life estate in it and that he did, therefore, effectively transfer a
remainder interest in the painting to plaintiff at that time. Although the
original letter was not in evidence, testimony of its contents was received
along with the substitute gift letter and its covering letter dated May 22,
1963. The three letters should be considered together as a single instrument
and when they are they unambiguously establish that Victor Gruen intended to
make a present gift of title to the painting at that time. But there was other
evidence for after 1963 Victor made several statements orally and in writing
indicating that he had previously given plaintiff the painting and that
plaintiff owned it. Victor Gruen retained possession of the property, insured
it, allowed others to exhibit it and made necessary repairs to it but those
acts are not inconsistent with his retention of a life estate. \dots{} Victor's
failure to file a gift tax return on the transaction was partially explained by
allegedly erroneous legal advice he received, and while that omission sometimes
may indicate that the donor had no intention of making a present gift, it does
not necessarily do so and it is not dispositive in this case. 

Defendant contends that even if a present gift was intended, Victor's
reservation of a lifetime interest in the painting defeated it. \dots{}
Defendant recognizes that a valid \textit{inter vivos} gift of a remainder
interest can be made not only of real property but also of such intangibles as
stocks and bonds. Indeed, several of the cases she cites so hold. That being
so, it is difficult to perceive any legal basis for the distinction she urges
which would permit gifts of remainder interests in those properties but not of
remainder interests in chattels such as the Klimt painting here. The only
reason suggested is that the gift of a chattel must include a present right to
possession. [Permitting] a gift of the remainder in this case, however, is
consistent with the distinction, well recognized in the law of gifts as well as
in real property law, between ownership and possession or enjoyment. Insofar as
some of our cases purport to require that the donor intend to transfer both
title and possession immediately to have a valid \textit{inter vivos} gift,
they state the rule too broadly and confuse the effectiveness of a gift with
the transfer of the possession of the subject of that gift. The correct test is
`` `whether the maker intended the [gift] to have no effect until after the
maker's death, or whether he intended it to transfer some present
interest.'{}'' As long as the evidence establishes an intent to make a present
and irrevocable transfer of title or the right of ownership, there is a present
transfer of some interest and the gift is effective immediately. Thus, in
Speelman v. Pascal, we held valid a gift of a percentage of the future
royalties to the play ``My Fair Lady'' before the play even existed. There, as
in this case, the donee received title or the right of ownership to some
property immediately upon the making of the gift but possession or enjoyment of
the subject of the gift was postponed to some future time.

 Defendant suggests that allowing a donor to make a present gift of a remainder 
with the reservation of a life estate will lead courts to effectuate otherwise
invalid testamentary dispositions of property. The two have entirely different
characteristics, however, which make them distinguishable. Once the gift is
made it is irrevocable and the donor is limited to the rights of a life tenant
not an owner. Moreover, with the gift of a remainder title vests immediately in
the donee and any possession is postponed until the donor's death whereas under
a will neither title nor possession vests immediately\dots. 

\readinghead{Delivery}

In order to have a valid \textit{inter vivos} gift, there must be a delivery of
the gift, either by a physical delivery of the subject of the gift or a
constructive or symbolic delivery such as by an instrument of gift, sufficient
to divest the donor of dominion and control over the property. As the statement
of the rule suggests, the requirement of delivery is not rigid or inflexible,
but is to be applied in light of its purpose to avoid mistakes by donors and
fraudulent claims by donees. Accordingly, what is sufficient to constitute
delivery ``must be tailored to suit the circumstances of the case.'' The rule
requires that `` `[t]he delivery necessary to consummate a gift must be as
perfect as the nature of the property and the circumstances and surroundings of
the parties will reasonably permit.' ''

Defendant contends that when a tangible piece of personal property such as a
painting is the subject of a gift, physical delivery of the painting itself is
the best form of delivery and should be required. Here, of course, we have only
delivery of Victor Gruen's letters which serve as instruments of gift.
Defendant's statement of the rule as applied may be generally true, but it
ignores the fact that what Victor Gruen gave plaintiff was not all rights to
the Klimt painting, but only title to it with no right of possession until his
death. Under these circumstances, it would be illogical for the law to require
the donor to part with possession of the painting when that is exactly what he
intends to retain.

Nor is there any reason to require a donor making a gift of a remainder interest
in a chattel to physically deliver the chattel into the donee's hands only to
have the donee redeliver it to the donor. As the facts of this case
demonstrate, such a requirement could impose practical burdens on the parties
to the gift while serving the delivery requirement poorly. Thus, in order to
accomplish this type of delivery the parties would have been required to travel
to New York for the symbolic transfer and redelivery of the Klimt painting
which was hanging on the wall of Victor Gruen's Manhattan apartment. Defendant
suggests that such a requirement would be stronger evidence of a completed
gift, but in the absence of witnesses to the event or any written confirmation
of the gift it would provide less protection against fraudulent claims than
have the written instruments of gift delivered in this case. 

\readinghead{Acceptance}

Acceptance by the donee is essential to the validity of an \textit{inter vivos}
gift, but when a gift is of value to the donee, as it is here, the law will
presume an acceptance on his part. Plaintiff did not rely on this presumption
alone but also presented clear and convincing proof of his acceptance of a
remainder interest in the Klimt painting by evidence that he had made several
contemporaneous statements acknowledging the gift to his friends and
associates, even showing some of them his father's gift letter, and that he had
retained both letters for over 17 years to verify the gift after his father
died. Defendant relied exclusively on affidavits filed by plaintiff in a
matrimonial action with his former wife, in which plaintiff failed to list his
interest in the painting as an asset. These affidavits were made over 10 years
after acceptance was complete and they do not even approach the evidence in
Matter of Kelly where the donee, immediately upon delivery of a diamond ring,
rejected it as ``too flashy''. We agree with the Appellate Division that
interpretation of the affidavit was too speculative to support a finding of
rejection and overcome the substantial showing of acceptance by plaintiff.

 Accordingly, the judgment appealed from and the order of the Appellate Division
brought up for review should be affirmed, with costs.

