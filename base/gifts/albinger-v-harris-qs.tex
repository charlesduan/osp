\item Would the court rule the same way if a woman had given a man an engagement
ring, or a man had given a man an engagement ring? \emph{Compare Lindh v.
Surman}, 742
A.2d 643 (Pa. 1999) (adopting the majority no-fault rule requiring return of
the ring in any broken engagement).  


\item Does the adoption of a no-fault rule require the mandatory return of the
ring in every case? Wouldn't a no-return rule also be a no-fault rule that
distributed the losses differently?  


\item If the dissent is correct, could Albinger have sought the return of his
other gifts? 


\item The dissent argues that the anti-heartbalm laws were fully consistent with
no-fault conditional gift theory, but see Tushnet, \textit{supra}:
\begin{quote}
[The early post-reform cases like \textit{Pavlicic}, holding that women had to
return engagement rings when they were at fault for breaking the engagement,]
subtly shifted the justification for antiheartbalm laws in a paternalistic
direction. The laws were no longer understood to protect wholly innocent men
who had not promised anything to mercenary women; instead, they were construed
to protect men who had foolishly, but willingly, made actual promises to marry
and who then gave gifts in reliance on a woman's deceitful promise. Blackmail
and extortion disappeared, leaving trickery and desire-induced gullibility as
the only rationales for the rule. Whereas prereform cases had presumed that
women were vulnerable to a naturally dominating male influence, the emerging
line of engagement gift cases were premised on the view that women could misuse
their power over love-blinded men. There was a related shift in the presumed
goal of a promise-breaker's deceit: Duplicitous men attempting to get sex were
replaced as the targets of premarital law by duplicitous women attempting to
get material goods.
\end{quote}
If the reason for having a unique rule for engagement rings grows out of a
history of sexist views of women, does that support the majority's conclusion?



\item Cooper v. Smith, 800 N.E.2d 372 (Ohio Ct. App. 2003), concluded that the
engagement ring was a conditional gift recoverable without fault, but that
other premarital gifts to a then-fiancee---including a car, a computer, a
tanning bed, and horses---were irrevocable \textit{inter vivos} gifts unless
they were expressly conditioned on the occurrence of the marriage, and gifts to
ex-fiancee's mother were also irrevocable: 
\begin{quotation}
The engagement ring has a special significance because it symbolizes the
couple's promise to marry. As a symbol of the promise to marry, what value does
the ring have for the donee once the engagement is ended?\edfootnote{As a symbol
of the promise to marry, what value does the
ring have for the \textit{donor} once the engagement is ended?} Moreover, we
realize that a donor proposing to his or her beloved is unlikely to expressly
condition the gift of the engagement ring on the occurrence of the marriage.
Not only do we realize how unlikely this is, we recognize how unromantic such a
requirement would be. Thus, because of the engagement ring's symbolic
significance, we are willing to imply a condition to the gift of the engagement
ring. Unless the parties have agreed otherwise, the donor is entitled to
recover the engagement ring (or its value) if the marriage does not occur,
regardless of who ended the engagement.

While we are willing to imply a condition concerning the engagement ring, we are
unwilling to do so for other gifts given during the engagement period. Unlike
the engagement ring, the other gifts have no symbolic meaning. Rather, they are
merely ``token[s] of the love and affection which [the donor] bore for the
[donee].''\dots ``Many gifts are made for reasons that sour with the passage
of time.'' Unfortunately, gift law does not allow a donor to recover/revoke an
\textit{inter vivos} gift simply because his or her reasons for giving it have
``soured.''

\dots If we were to imply a condition on gifts given during the engagement
period, then every gift the donor gave, no matter how small or insignificant,
would be recoverable.\dots We believe that the best approach is to treat
gifts exchanged during the engagement period (excluding the engagement ring) as
absolute and irrevocable \textit{inter vivos} gifts unless the donor has
expressed an intent that the gift be conditioned on the subsequent marriage.
\end{quotation}
What's the difference between a gift being ``symbolic'' and a gift being a
``token[]'' of ``love and affection''---aren't those synonyms?  Why not then
imply the same conditions on other premarital gifts?

Does it matter that willingness and ability to perform, and not actual
performance, is the usual rule for all conditional gifts except for the
engagement ring?  This would roughly equate to the older fault rule in broken
engagement cases.  \textit{See} Curtis v. Anderson, 106 S.W.3d 251 (Tex. Ct.
App. 2003) (agreement that fiancee would return ring if marriage did not occur
had to be in writing to be enforceable; applying fault rule and finding that
donor, as person responsible for calling off marriage, was not entitled to
return of engagement ring under conditional gift doctrine).  But is it actually
``fault'' to break an engagement where at least one party doesn't want to get
married?  


\item How will this rule affect the behavior of engaged couples in Montana? 
(Did you know the majority no-fault rule before reading this section?)


\item Susan and Sarah announce their engagement.  Numerous gifts arrive before
the day of the wedding, some at a ``bridal shower'' and some just before the
wedding.  On the morning of the wedding, Susan and Sarah call off their
engagement. Who owns the gifts?  (If you think the gifts don't have to be
returned, how should Susan and Sarah split them?) What if Sarah has a stroke
and dies the day before the wedding? If Susan and Sarah marry on Sunday and
divorce on Monday?


\item \textbf{Other odious conditions.} \emph{Tennessee Div. of the United
Daughters
of the Confederacy v. Vanderbilt Univ.}, 174 S.W.3d 98 (Tenn. Ct. App. 2005),
involved a 1933 donation from the UDC of \$50,000 to Peabody College. The gift
came with the express condition that a dormitory must be maintained with the
inscription ``Confederate Memorial Hall.'' In 1979, Peabody College merged into
Vanderbilt University, which in 2002 sought to change the name to repudiate the
association with slavery and racial exclusion.  The court of appeals found that
Vanderbilt could not do so unless it returned the present value of the gift. 
Is that the right result?  The court said that it wasn't within the court's
mandate ``to resolve the larger cultural and social conflicts regarding whether
and how those who fought for the Confederacy should be honored or remembered.''
 A concurring judge commented that a majority of those who fought for the
Confederacy owned no slaves, and called Vanderbilt's stance a
``misperception.''  (Query: Is the alleged misperception about what Confederate
soldiers fought to preserve?)


Was the problem merely that the court of appeals didn't find the condition
odious \textit{enough}? In fact, the agreement between the UDC and Peabody
apparently called for Peabody to house young women descended from Confederate
soldiers for free in the dormitory.  Vanderbilt stopped the free housing, with
no objection.  In the litigation, Vanderbilt argued that the conditions were
never properly put in place, but the court of appeals rejected its arguments. 
Should Vanderbilt be required either to reinstate the free housing for
descendants of Confederate soldiers or pay up?  (Laws against racial
discrimination would make the first alternative impossible.)  Ultimately,
Vanderbilt changed the name on all official documents and the university's
website, but kept ``Confederate Memorial Hall'' on the building itself.  As you
will see, this condition---apparently destined to persist forever---raises
issues of ``dead hand'' control by past owners, issues we have seen before and
will see again.

