Speaking more generally, it is possible to give other kinds of
\textit{conditional} gifts.  A conditional gift is a gift that will return to
the donor if a condition subsequent is not fulfilled.  (By contrast, a promise
to give a gift if a condition \textit{precedent} is fulfilled is an
unenforceable promise to make a gift.) For example, a student's parents might
give him a car, conditioned on his graduating law school in three years.  If
the student fails to graduate in that time, the car must be returned.  During
that period, however, the gift is otherwise irrevocable: the parents cannot
change their minds in year two and demand the car back, as long as the student
remains willing to fulfill the condition and remains on track to do so.

Since many gifts are given orally, how can we know which are conditional, and
what those conditions are?  Would you support a rule that conditions can only
be imposed on gifts if the conditions are written down?  \textit{See} State ex
rel. Pai v. Thom, 563 P.2d 982 (Haw. 1977) (delivery of a deed with an oral
condition was irrevocable even if condition was unsatisfied); \textit{but see}
Martinez v. Martinez, 678 P.2d 1163 (N.M. 1984) (allowing grantor of deed to
offer parol evidence of oral condition).  \textit{Pai} states the traditional
common law rule for transfers of land -- why do you think this is so?  What
might justify relaxing the traditional rule?

Can conditions sometimes be implied on gifts of personal property? Consider the
following case, involving an engagement ring.

