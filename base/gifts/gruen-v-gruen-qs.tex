\expected{gruen-v-gruen}

\item \textbf{Postscript.} Michael took possession and immediately sold the
painting for \$5 million.  A decade later it was resold for \$23 million.


\item \textbf{Future Interests.}  Michael's remainder interest is a type of
``future interest.''  It is a future interest because Michael can only take
\textit{possession} in the future.  However, as a legal matter, Michael's
future interest exists before his right to take possession does.  Victor needed
to have a present donative intent in order to make a valid gift, and he had
that intent: he intended to give Michael \textit{something} when he wrote. 
That something was a future interest.


Corresponding to Michael's remainder interest is Victor Gruen's ``life estate,''
which is a present possessory interest as long as Victor Gruen lives -- and
thus the court holds that Victor may keep possession of the painting, despite
the delivery requirement.  He isn't giving Michael a present possessory
interest, so requiring him to deliver the painting wouldn't serve the ordinary
purpose served by delivery of uniting the property owner with physical
possession of the property.



Does \textit{Gruen}{}'s willingness to recognize legal future interests in
personal property undermine the certainty provided by a delivery requirement? 
What if Victor had intended an even more complicated transfer -- perhaps giving
the painting to his widow for as long as she lived, then to Michael?  This
could have raised troubling issues of who owned what.



\item \textbf{Tax fraud.}  If the donor retains a life estate in property
transferred \textit{inter vivos}, that property will be included in the donor's
estate for estate tax purposes.  I.R.C. {\S}2036(a).  As a result, the estate
tax owed will be the same as if the property hadn't been transferred at all. 
This provision is designed to discourage evasion of the estate tax (which was
substantially more onerous when this case was litigated).  Victor did just
this, but his lawyer told him to ``doctor up'' the transaction so that he
wouldn't have to pay the resulting taxes.  \textit{Do not do this.}  It is
called tax fraud, and you may be disbarred, or worse.  Relatedly, backdating
the letter raises serious ethical problems, and it too could have serious
consequences for a lawyer who advised backdating in situations where the date
of the transfer matters. Should Victor's apparently successful tax evasion have
factored into the court's decision on the state law question of whether the
gift was valid?  If so, how?  


\item \textbf{Substitutes for testamentary transfers?} Consider the
outcomes of the following scenarios:
\begin{itemize}
\item Victor Gruen writes the same letter giving his son Michael a future
interest, but simply shows it to Michael rather than mailing it.
\item Victor writes a letter granting Michael total ownership of the painting,
without reserving a life estate for himself, and mails the letter to Michael,
then dies before being able to deliver the painting.
\item Victor writes a will in 1963 devising the painting to Michael.  Victor
dies in 1980.  (What interest, if any, does Michael own in the painting before
Victor dies?)
\item Victor writes a letter saying ``I intend for you to have the painting when
I die.''
\end{itemize}

As you should see, the last possibility is an attempted testamentary transfer,
but it is unlikely to meet the requirements for a transfer by will.  Rather
than being a present transfer of a future interest, it's a statement of intent
to make a transfer in the future.  Is the difference between what Victor
actually did and ``I intend for you to have the painting when I die'' big
enough to explain the different results?  If this rule allows legally savvy
people to carry out their intent more successfully than laypeople innocent of
the law, is that a good thing or a bad thing, compared to the alternatives? 
Some states now allow land transfers in this form -- a deed that expressly says
it won't take effect until the death of the grantor will be honored, but only
if it's recorded before the death of the grantor.  See Mo. Stat. Ann.
{\S}461.025(1).  Would you support such a law?



\item \textbf{Another Variant of Symbolic or Constructive Delivery: Delivery to
a Place.}  Sometimes, the would-be donor does not physically hand the object or
document to the donee, but instead puts it in a particular place, from which
she expects the donee to retrieve it.  Should this constitute delivery?  Courts
have disagreed about the details, but if the putative donee does not have any
right to control the place and other people do, then there is no delivery.  For
example, a household servant does not have dominion over the whole house, so a
piano placed in the living room would not be delivered to such a servant, even
if there was explicit donative intent.  Another rule is that, if the putative
donee has exclusive dominion over the place, there is delivery.  Thus,
furniture placed in a live-in servant's bedroom in her employer's house would
be delivered to the servant.  \textit{See }Newman v. Bost, \textit{supra}. (In
such a case, disputes might still arise over donative intent.)  


What about shared spaces?  Suppose four people are living together in a house,
and A tells B that she's giving him a book, which she leaves on the kitchen
table for him.  The kitchen is shared by all four residents.  Should this be
sufficient delivery?  Does the fact that she could easily instead have put the
book in his bedroom, which \textit{is} under his exclusive control, make any
difference?  (What should we expect laypeople to know about the law of gifts?) 
\textit{Cf.} Robinson v. Hoalton, 2 P.2d 34 (Cal. 1931) (personal property was
validly delivered when there was an oral grant and the donor and donee lived
together: ``The rule as to delivery is not so strictly applied to transactions
between members of a family living in the same house \dots{}.'').



Disclosure of the location of an item may also serve as delivery, at least when
it is otherwise hidden or inaccessible.  \textit{See} Waite v. Grubbe, 73 P.
206 (Ore. 1903) (disclosing location of buried cash sufficed for valid gift);
Teague v. Abbott, 100 N.E. 27 (Ind. Ct. App. 1912) (disclosing combination to
safe sufficed for valid gift).  



\item \textbf{Intermediated Delivery.} Can you identify a unifying principle
behind the constructive/symbolic delivery cases?  Following the logic of these
cases, suppose that, while Buffy Summers is working at the fast food restaurant
Doublemeat Palace, her friends leave a present for her at the door, before the
restaurant opens. When the manager comes to open up, he takes the present for
himself.  Who should bring the claim against him, Buffy or her friends?


Handing the property, or an appropriate symbol of the property, to a third party
for delivery to the donee will also complete delivery, as long as the donor has
no power to recall the third party.  If the donor can still control the third
party and interrupt the delivery, by contrast, then the delivery is not
complete.  These principles have led courts to diverge on the proper treatment
of checks: because a check can be stopped by the payor at any time before it's
cashed,\footnote{Be aware that the law governing checks is complicated, and
stopping a check may not always succeed or may have other consequences -- this
note just discusses how courts in gift cases have treated checks.} the majority
of courts say there's no delivery until that time.  The payor has not given up
complete control until the check is cashed.  \textit{See, e.g.}, \textit{Rosano
v. United States}, 67 F. Supp. 2d 113 (E.D.N.Y. 1999); \textit{ In re Estate of
Heyn}, 47 P.3d 724 (Colo. Ct. App. 2002); \textit{Woo v. Smart}, 442 S.E.2d 690
(Va. 1994).  Other courts say that, at least with respect to gifts made in
anticipation of death (of which more below), when the check is \textit{not}
stopped before the donor dies, the gift is complete even before the check is
cashed.  



\item \textbf{Delivering Intangibles.} Given the delivery requirement, how would
you accomplish the gift of an intangible right, such as a copyright or a share
of stock?

