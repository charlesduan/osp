Simply put, the property must in some way pass out of the grantor's control in
order for a gift to be valid; this is known as delivery.  Professor Philip
Mechem summarized the reasons for the requirement: 
\begin{enumerate}
\item Delivery has psychological significance, forcing the donor to confront the
loss of the property: the ``wrench of delivery'' protects the donor from poor
choices. 
\item Delivery signifies the gift to third party witnesses, settling doubts
about whether what was intended was a mere promise to make a gift in the future
or a present gift.  
\item Delivery lets the property itself bear mute witness to the fact of the
gift: possession itself has evidentiary weight.  
\end{enumerate}
Philip Mechem, \textit{The Requirement of Delivery in Gifts of Chattels and of
Choses in Action Evidenced by Commercial Instruments}, 21 \textsc{Ill. L. Rev.}
341, 348-49 (1926); but see Chad A. McGowan, \textit{Special Delivery: Does the
Postman Have to Ring at All -- the Current State of the Delivery Requirement
for Valid Gifts}, \textsc{31 Real Prop. Prob. \& Tr. J.} 357 (1996) (critiquing
Mechem).

Land doesn't move, at least not for these purposes, so manual delivery is
impossible, and ``symbolic'' delivery of land has always been accepted.  At
early common law, the transfer of land involved a ceremony called livery of
seisin, in which the transferor physically handed over a clod of dirt or a twig
from the land to the transferee.  Fortunately, transfer of an interest in land
is now generally accomplished by a written instrument, known as a deed.  At a
minimum, a deed must describe the land to be transferred, contain some words
indicating an intent to make a present transfer of title, and the grantor's
signature (which courts construe liberally -- almost any mark or symbol of the
grantor's approval, including the signature of the grantor's agent, will be
sufficient). These formalities will be covered in more detail in the land
conveyancing section.

What about personal property?  Most gifts, especially gifts of personal
property, are given during life.  Nonetheless, many litigated cases arise
around near-death gifts.  As you read, consider why the case law would diverge
so much from the practice of giving.

