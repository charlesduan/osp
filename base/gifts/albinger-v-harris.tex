\reading{Albinger v. Harris}

\readingcite{48 P.3d 711 (Mont. 2002)}

Who owns a ring given in anticipation of marriage after the engagement is
broken? Michelle L. Harris (Harris) appeals the disposition of an engagement
ring by the Eighth Judicial District Court, Cascade County, Montana, and
Michael A. Albinger (Albinger) cross-appeals the denial of reimbursement for
certain telephone charges incurred by Harris and the award of damages for a
prior, unlitigated assault and battery claim. We reverse the disposition of the
engagement ring, affirm the denial of reimbursement for telephone charges and
affirm the award for pain, suffering and emotional distress. 

We frame the issues on appeal as follows: 

1. Did the District Court err in determining an engagement ring is a conditional
gift that may be revoked upon termination of the engagement?...

\readinghead{Factual and Procedural Background}

Harris and Albinger met in June 1995, and began a troubled relationship that
endured for the next three years, spiked by alcohol abuse, emotional turmoil
and violence. Albinger presented Harris with a diamond ring and diamond
earrings on December 14, 1995. The ring was purchased for \$29,000. Days after
accepting the ring, Harris returned it to Albinger and traveled to Kentucky for
the holidays. Albinger immediately sent the ring back to Harris by mail. The
couple set a tentative wedding date of June 27, 1997, but plans to marry were
put on hold as Harris and Albinger separated and reconciled several times. The
ring was returned to or reclaimed by Albinger upon each separation, and was
re-presented to Harris after each reconciliation. 

Albinger and Harris lived together in Albinger's home from August 1995 until
April 1998. During this time, Albinger conferred upon Harris a new Ford Mustang
convertible, a horse and a dog, in addition to the earrings and ring. Harris
gave Albinger a Winchester hunting rifle, a necklace and a number of other
small gifts. Albinger received a substantial jury award for injuries sustained
in a 1991 railroad accident. He paid all household expenses and neither party
was gainfully employed during their cohabitation. 

On the night of February 23, 1997, during one of the couple's many separations,
Albinger broke into the house where Harris was staying. He stood over Harris'
bed, threatened her with a knife and shouted, ``I'm going to chop your finger
off, you better get that ring off.'' After severely beating Harris with a
railroad lantern, Albinger forcibly removed the ring and departed. Harris sued
for personal injuries and the county attorney charged Albinger by information
with aggravated burglary, felony assault, and partner and family member
assault. The next month, after another reconciliation, Harris requested the
county attorney drop all criminal charges in exchange for Albinger's promise to
seek anger management counseling and to pay restitution in the form of Harris'
medical expenses and repair costs for damage to her friend's back door. Harris
also directed her attorney to request the court dismiss the civil complaint
without prejudice.

The parties separated again in late April 1998. Albinger told Harris to ``take
the car, the horse, the dog, and the ring and get the hell out.'' During their
last month together, Harris ran up approximately \$1,000 in telephone charges
on Albinger's credit card. Harris had been free to use Albinger's telephone
throughout the relationship, and Albinger paid the bills. Harris moved from
Great Falls, Montana to Kentucky, where she now resides. The parties dispute
who was responsible for the end of the relationship. No reconciliation
followed, marriage plans evaporated and Harris refused to return the ring. 

Albinger filed a complaint on August 31, 1998, seeking recovery of the ring or
its monetary value and payment for \$1,000 in telephone charges. Harris
counterclaimed for damages resulting from the assault of February 23, 1997.

At the conclusion of the trial, both parties submitted briefs discussing how the
statute barring actions for breach of promise to marry, {\S} 27-1-602, MCA,
impacts an action to recover an engagement ring. The District Court found the
ring to be a gift in contemplation of marriage, and reasoned that {\S}
27-1-602, MCA, did not bar the action because the case could be decided on
common-law principles, as opposed to contract theories. The court implied the
existence of a condition attached to the gift of the engagement ring.
Disregarding allegations of fault for ``breaking'' the engagement, the court
concluded that the giver is entitled to the return of the ring upon failure of
the condition of marriage.

On September 2, 1999, the District Court awarded the engagement ring or its
reasonable value and court costs to Albinger, and denied recovery for the
telephone charges. \dots{} From this judgment, Harris appeals the disposition
of the ring and Albinger cross-appeals the denial of telephone charges \dots{}.

Did the District Court err in determining an engagement ring is a conditional
gift that may be revoked upon termination of the engagement?

Albinger and Harris gave one another numerous gifts of substantial value during
their engagement. The ring is the only item now in controversy\dots{}. 

Legal ownership of the gift of an engagement ring when marriage plans are called
off is an issue of first impression in Montana. In 1963, the Legislature barred
access to the courts for actions arising from breach of the promise to marry.
The District Court determined that this action brought to recover an
antenuptial gift is maintainable, notwithstanding {\S} 27-1-602, MCA, which
states:

\begin{quote}
All causes of action for breach of contract to marry are hereby abolished.
However, where a plaintiff has suffered actual damage due to fraud or deceit or
a defendant has been unjustly enriched, the plaintiff may maintain an action
for fraud or deceit or unjust enrichment and recover therein only the actual
damage proved or for the benefit wrongfully obtained or restitution of property
wrongfully withheld where such action otherwise is maintainable under existing
law.
\end{quote}

\dots Albinger argues that the engagement ring was a conditional gift that he
could revoke when the implied condition of marriage failed. Hence, Harris'
refusal to return the ring upon demand constituted unjust enrichment. Harris
contends she deserves the ring because Albinger repeatedly beat her, forcibly
took the ring back, and was the one who finally ended the engagement by
ordering Harris to move out of the residence where they had been living
together.

The District Court declined to undertake a determination of which party was at
fault in terminating the engagement. The court cited the following three
reasons: 1) judicial holdings that fault is an inappropriate concern in matters
of family relations; 2) pragmatic difficulties in discerning fault when the
conduct of both parties likely contributes to the failure of a relationship;
and, 3) aversion to concepts of legal ``rightness'' and ``wrongness'' regarding
the choice of a marriage partner. We agree, and affirm that judicial
fault-finding is irrelevant and immaterial in the adjudication of matters of
antenuptial gifting under existing law, absent fraud or deceit.

The District Court employed the ``conditional gift'' theory advanced by Albinger
to determine present ownership of the disputed engagement ring. The theory
holds that an implied condition of marriage attaches to the gift of a ring upon
initial delivery due to the  ring's symbolic association with the promise to
marry and, when the condition of marriage fails, the incomplete gift may be
revoked by the giver. Albinger urges this Court to affirm the District Court's
conclusion that the ownership of an engagement ring remains with the one who
gave the ring when plans to marry are called off.

Only in engagement ring cases does precedent from other jurisdictions weigh
heavily for conditional gift theory in the absence of an expressed condition.
Considering it ``unduly harsh and unnecessary'' to require a hopeful suitor to
express any condition upon which a ring might be premised, many courts stepped
in to impute the condition of marriage. In practice, courts presume the
existence of the implied condition of marriage attaching to an engagement ring
in the absence of an expressed intent to the contrary\dots.

\readinghead{Abolition of Breach of Promise Actions}

Historic breach of promise jurisprudence tended to view an engagement ring as
either a pledge of personal property given to secure a marital promise or as
consideration for the contract of marriage. When a contract to marry was
abrogated, the jilted lover could seek redress in a breach of promise action
that sounded in contract law, but availed the plaintiff of tort damages. ``The
law allows punitive or vindictive damages to be assessed by the jury; and all
the circumstances attending the breach before, at the time, and after may be
given in evidence in aggravation of damages.'' The plaintiffs were almost
invariably women seeking economic relief for themselves, compensation for
pregnancy and material support for children of the relationship. Whatever
``heart balm'' was awarded to assuage lost love, ruined reputation or
foreclosed opportunities to marry well ``rest[ed] in the sound discretion of
the jury.'' 

By the mid-1930's, several state legislatures questioned the efficacy of court
``interference with domestic relations'' and passed statutes barring actions
for breach of promise to marry, alienation of affections, criminal conversation
and other inappropriate conduct of the ``private realm.'' See Rebecca Tushnet,
Rules of Engagement (1998), 107 Yale Law Journal 2583, 2586-91. Commentators
noted all of these  actions ``afforded a fertile field for blackmail and
extortion by means of manufactured suits in which the threat of publicity is
used to force a settlement.'' ``There is good reason to believe that even
genuine actions of this type are brought more frequently than not with purely
mercenary or vindictive motives [and] that it is impossible to compensate for
such damage with what has derisively been called `heart balm.' '' 

In the wake of ``anti-heart balm'' statutes that barred breach of contract to
marry actions, courts heard a plethora of legal theories designed to involve
them in settling antenuptial property disputes while avoiding the language of
contract law. The results were mixed. Some courts allowed actions in replevin.
Others entertained claims for restitution and unjust enrichment. Out of this
legal morass, conditional gift analysis emerged as a popular way to resolve
acrimonious engagement ring disputes. While some states pursue a fault-based
determination for awarding the ring in equity, the modern wave aligns ring
disposition with no-fault divorce property disposition and follows a
bright-line rule of ring return\dots.

\readinghead{Conditional Gift Theory}

According to Montana law, ``a gift is a transfer of personal property made
voluntarily and without consideration.'' \dots When clear and convincing
evidence demonstrates the presence of the essential elements of donative
intent, voluntary delivery and acceptance, the gift is complete and this Court
will not void the transfer when the giver experiences a change of heart. 

Another essential element of a gift is that it is given without consideration. A
purported ``gift'' that is part of the inducement for ``an agreement to do or
not to do a certain thing,'' becomes the consideration essential to contract
formation. An exchange of promises creates a contract to marry, albeit an
unenforceable one. When an engagement ring is given as consideration for the
promise to marry, a contract is formed and legal action to recover the ring is
barred by the abolition of the breach of promise actions.

\dots Albinger maintains he held a reversionary interest in the gift of the
engagement ring grounded in an implied condition subsequent. Montana law
recognizes the transfer of personal property subject to an express or implied
condition which must be satisfied before title vests, as either a contract, or
as a gift in view of death. Since actions stemming from breach of the contract
to marry are barred by our ``anti-heart balm'' statute, Albinger urges the
Court to adopt a conditional gift theory patterned on the law relevant to a
gift in view of death. Under Montana law, no gift is revocable after acceptance
except a gift in view of death. While some may find marriage to be the end of
life as one knows it, we are reluctant to analogize gifts in contemplation of
marriage with a gift in contemplation of death. This Court declines the
invitation to create a new category of gifting by judicial fiat.

\readinghead{Gender Bias}

\dots The Montana Legislature made the social policy decision to relieve
courts of the duty of regulating engagements by barring actions for breach of
promise. While not explicitly denying access to the courts on the basis of
gender, the ``anti-heart balm'' statutes closed courtrooms across the nation to
female plaintiffs seeking damages for antenuptial pregnancy, ruined reputation,
lost love and economic insecurity. During the mid-20th Century, some courts
continued to entertain suits in equity for antenuptial property transfers. The
jurisprudence that rose upon the implied conditional gift theory, based upon an
engagement ring's symbolic associations with marriage, preserved a right of
action narrowly tailored for ring givers seeking ring return\dots. The
proposed no-fault adjudication of a disputed engagement ring also ignores the
particular circumstances of a couple's decision not to marry.

Conditional gift theory applied exclusively to engagement ring cases, carves an
exception in the state's gift law for the benefit of predominately male
plaintiffs.\dots While antenuptial traditions vary by class, ethnicity, age
and inclination, women often still assume the bulk of pre-wedding costs, such
as non-returnable wedding gowns, moving costs, or non-refundable deposits for
caterers, entertainment or reception halls. Consequently, the statutory
``anti-heart balm'' bar continues to have a disparate impact on women. If this
Court were to fashion a special exception for engagement ring actions under
gift law theories, we would perpetuate the gender bias attendant upon the
Legislature's decision to remove from our courts all actions for breach of
antenuptial promises. 

\readinghead{Engagement Ring Disposition}

To preserve the integrity of our gift law and to avoid additional gender bias,
we decline to adopt the theory that an engagement ring is a gift subject to an
implied condition of marriage. We hold that the engagement ring was an
unconditional, completed gift upon acceptance and remains in Harris' ownership
and control.\dots

\opinion Justice \textsc{Terry N. Trieweiler} concurring [as to the telephone
charges] and dissenting.

Gender discrimination is a bad thing. I am glad the majority is against it.
However, I regret that the majority has taken this opportunity to declare their
good intentions because gender equity has about as much to do with this case as
banking law.\dots

The simple fact is that if women are more likely to be the subject of an action
to recover a conditional gift given in anticipation of a marriage which does
not occur, it is because they are more frequently the recipient of the gift.
Should we just prohibit gifts in anticipation of marriage altogether because
men are more likely to have to pay for them?

\dots The District Court found in Finding No. 7 that each time, except for the
last time, the couple broke up (and they broke up frequently), the ring was
either returned by Michelle or taken back by Michael. These findings were fully
supported by the evidence. For example, Michael testified as follows:

\begin{quotation}
Q. Now, when you gave her the engagement ring, were you contemplating that you
were going to get married?

A. That was the whole idea.

Q. Was there any way in your mind that the engagement ring was just a ring and
she could keep it whether you were married or not?

A. No\dots{}.
\end{quotation}

It is equally clear from the record that the engagement ring was treated
differently by the couple than other gifts which had been given by Michael to
Michelle while they were engaged. He testified as follows:

\begin{quotation}
Q. Okay. Now, you heard her testify, didn't you, about the other gifts you'd
given her?

A. Yes, I did.

Q. The 1995 Mustang that cost about \$24,000, was that in contemplation of
getting married?

A. No.

Q. Have you ever made any demand that she return that to you?

A. No.

Q. How about the diamond earrings that were the Christmas gift? Have you ever
made any demands that she return those because they were given in contemplation
of marriage?

A. No.

Q. How about the horse?

A. No.

Q. Any gift you gave her beside the engagement ring, did you ever contend that
those other gifts were made in contemplation of marriage?

A. No. They were gifts. Those are gifts.
\end{quotation}

\dots [A]n older line of cases limited a donor's recovery of the gift to
situations where the engagement is dissolved by agreement or unjustifiably
broken by the donee. However, the court concluded that the notion of one party
being at blame for the termination of an engagement is archaic and outdated. 

\dots{} [E]ither gender can [be] given an engagement ring. For example, in Vigil
v. Haber (1994), 119 N.M. 9, 888 P.2d 455, the parties exchanged engagement
rings. However, their relationship deteriorated, the couple separated, and
following their separation a hearing examiner determined that the parties
should return the rings they had given each other. The plaintiff immediately
returned the ring he had along with other of the defendant's possessions.
However, the defendant objected to returning the engagement ring that had been
given to her. The New Mexico Supreme Court held that the ring was a conditional
gift depend[e]nt on the parties' marriage and should be returned. What if the
roles had been reversed and the woman had returned the engagement ring given to
her but the man had refused to do so? According to this Court, she would not be
allowed to recover the engagement ring that she had given to her fianc\'e, no
matter how substantial the value and unfair the result because requiring the
return of engagement rings is unfair to women.

The second problem with the majority's assumptions is the assumption that
conditional gift law as it relates to gifts exchanged in anticipation of
marriage only applies to wedding rings. It does not. For example, in Pavlicic,
a case which disproves the theory that jurisprudence cannot be written in
readable prose, the plaintiff was a 75-year-old man when the 26-year-old
defendant asked for his hand in marriage. While he first protested on the basis
of his age, she assured him that she was no longer interested in ``young
fellows'' and prevailed upon him to make the commitment. Over the course of the
next four years, she then prevailed upon him to pay the mortgage on her home,
buy her two new cars, an engagement ring, a diamond for her mother's ring,
remodel her house, and advance her \$5000 to purchase a saloon which they could
jointly operate. The problem was that after she received the money for the
saloon, she disappeared. She was next seen in another town operating Ruby's bar
and married to a man two years her junior. 

As noted by the Pennsylvania Supreme Court:

\begin{quote}
When George emerged from the mists and fogs of his disappointment and
disillusionment he brought an action in equity praying that the satisfaction of
the mortgage on Sara Jane's property be stricken from the record, that she be
ordered to return the gifts which had not been consumed, and pay back the
moneys [sic] which she had gotten from him under a false promise to marry.
\end{quote}

The Pennsylvania Supreme Court held that George's action was not barred by the
state's ``Heart Balm Act'' and that all gifts, not just the wedding ring, were
conditional gifts in anticipation of marriage which must be returned or repaid.
\dots

The point is that there is no reason to limit the law of ``conditional gifts''
in anticipation of marriage to engagement rings.\dots

