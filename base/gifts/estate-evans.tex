\reading{In re Estate of Evans}
\readingcite{356 A.2d 778 (Pa. 1976)}

Appellant, Vivian Kellow, objected to the inventory, proposed schedule of
distribution and final accounting of the executor of the estate of Arthur
Evans. After appellant finished the presentation of her case, the lower court
granted appellees' motion to dismiss appellant's objections.\dots The thrust
of her appeal to this Court is that certain contents of a safe deposit box were
the subject of an \textit{inter vivos} gift to her from Arthur Evans, the
deceased, and, consequently, should not have been included in his estate. 

Appellant, the niece of Arthur Evans' deceased wife, began working for the Evans
family when she was 16. For several years she took care of Mrs. Evans who for
some years prior to death was an invalid. Appellant cooked meals for the
Evanses, cleaned their house, did their laundry and generally cared for Mrs.
Evans. She received adequate compensation for performing these needed services.
When Mrs. Evans died, appellant continued to cook at least one hot meal a day
for Mr. Evans, do his laundry and make sure his house was tidy. After appellant
was married, she continued to perform these same services and visited Mr. Evans
once a day. In May of 1971, following one of his four hospitalizations, the
deceased moved into appellant's home. 

Although at times Mr. Evans was confined to his bed because of water in his
legs, he frequently took walks, had visits with his lawyers and made trips to
his bank. On October 22, 1971, appellant's husband drove Mr. Evans and a friend
of his, Mr. Turley, to town so that Mr. Evans might go to the bank. Turley
testified that Mr. Evans spent about one hour going through the contents of his
safe deposit box. Before leaving the bank, the deceased obtained both keys to
the box. 

Various witnesses presented by appellant testified to seeing the keys to the
safe deposit box beneath appellant's mattress and to statements by Mr. Evans to
the effect that the contents of the safe deposit box had been given to
appellant. Mr. Evans entered the hospital for the last time on November 5,
1971. During this last hospital stay, Reverend Cunnings visited with him and
was told that Mr. Evans was giving the Reverend's church \$10,000.00 and that
he had given the rest of his possessions and the keys to his safe deposit box
to appellant. Mr. Evans expired on November 23, 1971. 

Appellant relinquished the keys to the safe deposit box to a bank officer, but
not without protesting that the contents of the box were hers. The box revealed
a holographic will of Mr. Evans dated September 16, 1965, and approximately
\$800,000.00 in bonds, preferred and common stock and several miscellaneous
items.\readingfootnote{1}{The will was uncontested and under its terms provided
for a \$1,000.00 bequest to appellant.}

The lower court correctly noted that the requirements for a valid \textit{inter
vivos} gift were donative intent and delivery, actual or constructive. With
respect to donative intent, the court found:
\begin{quote}
Turning to the facts of this case, certainly no one can reasonably argue that
Arthur Evans lacked sufficient motive to make a gift to Vivian. The record
clearly manifests, both by his conduct and his statements, donative intent, the
first prerequisite.
\end{quote}

Nevertheless, the court ruled that no delivery had been made. This result was
predicated upon a finding that the deceased had not divested himself of
complete dominion and control over the safe deposit box. After properly noting
that constructive delivery is sufficient when manual delivery is impractical or
inconvenient, the court reasoned:
\begin{quote}
The record contains no evidence of circumstances which were such that it was
impractical or inconvenient to deliver the contents of this box into the actual
possession or control of Vivian.
\end{quote}

Arthur Evans, although suffering physical infirmities and apprehensive of death,
was nonetheless ambulatory. On October 22, 1971, he appeared at the Nanticoke
National Bank in the company of Harold Turley and Leroy Kellow and spent
approximately one hour going over the contents of his safe deposit box in a
cubicle provided in the bank for that purpose. He left the bank after
redepositing the contents and took with him only the keys which independent
testimony indicates he delivered to Vivian the next day. There was no manual
delivery of the contents. The contents of the box remained undisturbed. The
box, and its contents, were registered in the name of the decedent at the date
of his death. The objects of the gift were not placed in the hands of Vivian,
nor was there placed within her power the means of obtaining the
contents.\ldots

\begin{quote}
A claim of a gift \textit{inter vivos} against the estate of the dead must be
supported by clear and convincing evidence. In order to effectuate an
\textit{inter vivos} gift there must be evidence of an intention to make a gift
and a delivery, actual or constructive, of a nature sufficient not only to
divest the donor of all dominion over the property but also invest the donee
with complete control over the subject-matter of the gift.
\end{quote}
[\emph{Tomayko v. Carson}, 368 Pa. 379, 385 (1951).]

In the instant case, the controversy focuses on whether there was an adequate
delivery.\dots :
\begin{quote}
``If there remains something for the donor to do before the title of the
donee is complete, the donor may decline the further performance and resume his
own.''\dots ``[I]t is not possible that a chancellor would compel an executor
or administrator to complete a gift by the doing of any act which the alleged
donor if living might have refused to  do, and thereby revoked his purpose to
give.''\dots ``Though every other step be taken that is essential to the
validity of the gift, if there is no delivery, the gift must fail. Intention
cannot supply it; words cannot supply it; actions cannot supply it. It is an
indispensable requisite, without which the gift fails, regardless of
consequence.' The consequence is that no matter how often or how emphatically
the desire or intention of the donor to make the gift has been expressed, upon
his death before delivery has been completed, the promise or purpose to give is
revoked. 
\end{quote}

We have recognized that in some cases due to the form of the subject matter of
the gift or due to the immobility of the donor actual, manual delivery may be
dispensed with and constructive or symbolic delivery will suffice. In \emph{Ream
Estate}, 413 Pa. 489, 198 A.2d 556 (1964), for example, the Court found there
had been a valid constructive delivery of an automobile where the donor gave
the keys to the alleged donee and also gave him the title to the car after
executing an assignment of it leaving the designation of the assignee blank.
The assignment was executed in the presence of a justice of the peace and the
evidence was overwhelming that the name of the donee was to be inserted upon
the death of the decedent.\dots

Appellant relies heavily on \emph{Leadenham's Estate}, 289 Pa. 216, 137 A. 247 (1927),
and \emph{Leitch v. Diamond National Bank}, 234 Pa. 557, 83 A. 416 (1912). These
decisions, however, support the Court's finding that there was no delivery in
the instant case. In \emph{Leadenham's Estate}, \emph{supra}, the donor had
rented a separate
safe deposit box in the name of the intended donee, put the contents of his box
into the newly rented one and delivered the keys to it to the donee. On those
facts we held that the constructive delivery of the keys was sufficient to
sustain the \textit{inter vivos} gift because the donor had divested himself of
dominion and control and invested the donee with complete dominion and control.

In \emph{Leitch v. Diamond National Bank}, \emph{supra}, the donor and donee
were husband and
wife and had lived together harmoniously for many years. The husband had three
safe deposit boxes registered in his name and the name of his wife and he
designated one of them as his wife's. He gave her the keys to that box. The
Court found that she had complete control over that box and that he only
entered it with her permission. Since she had complete control over the access
to the box the Court found there was a valid delivery of the contents of the
box to her.

 In both of these cases, the determinative factor was that the donee had
complete dominion and control over the box and its contents. In that posture we
ruled that giving the keys to the box to the donee was a valid constructive
delivery. In the instant case, appellant did not have dominion and control over
the box even though she was given the keys to it. The box remained registered
in Mr. Evans' name and she could not have gained access to it even with the
keys. Mr. Evans never terminated his control over the box, consequently he
never made a delivery, constructive or otherwise.

 Although appellant suggests that it was impractical and inconvenient for Mr.
Evans to manually deliver the contents of his box to her because of his
physical condition and the hazards of taking such a large sum of money out of
the bank to her home, we need only note that the deceased was obviously a
shrewd investor, familiar with banking practices, and could have made delivery
in a number of simple, convenient ways. First, he was not on his deathbed. He
was ambulatory and not only went to the bank on October 22, 1971, but took
walks thereafter and did not enter the hospital until November 5, 1971. On the
day he went to the bank he could have rented a second safe deposit box in
appellant's name, delivered the contents of his box to it and then given the
keys to appellant. He could have assigned the contents of his box to appellant.
For that matter, he could have written a codicil to his will.

 The lower court noted that the deceased was an enigmatic figure. It is not for
us to guess why people perform as they do. On the record before us it is clear
that regardless of Mr. Evans' intention to make a gift to appellant, he never
executed that intention and we will not do it for him. On these facts, we are
constrained to hold that there was not an \textit{inter vivos} gift to
appellant and that the contents of the safe deposit box were properly included
in the inventory of Mr. Evans' estate\dots.

ROBERTS, Justice (dissenting).

I dissent. The central issue in this case is whether donor made an adequate
delivery of the gift to donee. The majority finds that adequate delivery was
not made because the safe deposit box was leased solely in donor's name and
supports this conclusion by pointing out that there were several alternative
means of delivering the gift which would have been adequate. I believe that the
inquiry should not be what form of delivery would have been clearly sufficient,
but rather whether the delivery made by donor was adequate. I believe that it
was.

 In \emph{Rynier Estate}, 349 Pa. 471, 32 A.2d 736 (1943), we said that delivery
is
determined on the facts of each case, with reference to the donor's intent.
\begin{quote}
As the chief factor in the determination of the question whether a legal
delivery has been effected is the intention of the donor to transfer title to
the donee, as manifested by his words and actions and by the circumstances
surrounding the transaction, it is evident that each case must depend largely
upon its own facts.
\end{quote}

The majority suggests that donor was ``obviously a shrewd investor,
familiar with banking practices.\ldots'' From this ``familiar(ity) with banking
practices,'' which is nowhere shown on the record, and the absence of a joint
lease for the box, it apparently concludes that donor did not intend a gift.
There are two reasons why this result is not correct.

 First, there is no doubt in this case that donor intended a gift. He told many
people that he had given the contents of the box to appellant. In fact, there
is competent testimony that donor directed donee to display the keys, hidden
under her mattress, to several witnesses.

Second, it is apparent from the record that donor believed undisputed and
unconditional delivery of the keys to be sufficient to complete the gift. Most
of this Court's cases dealing with \textit{inter vivos} gifts of the contents
of safe deposit boxes turn on the delivery or nondelivery of the keys to the
box to the donee. If the key was delivered, the gift was normally upheld; if
the key was not delivered, the gift was set aside, whether or not the box was
jointly leased. I have found no case which turned on the presence or absence of
a joint lease. Given this line of authority, and accepting the majority's
conclusion that donor was sophisticated in these matters, it must be concluded
that donor believed delivery of the keys to the box completed the gift. If this
were not so, why would donor cause donee to take several witnesses into her
bedroom to show them that she had the keys and why would he speak in terms that
indicated a completed gift---``I \textit{gave} to Vivian\ldots the keys and the
contents \textit{are} hers.'' Because it is donor's intention to transfer title
which is crucial to a valid delivery, and because this donor intended to
transfer title, I dissent from the majority's conclusion\dots.

