The gifts with which you are likely most familiar -- gifts to mark a special
occasion or relationship -- are generally \textit{inter vivos} gifts, that is,
gifts given by living people (the Latin literally means ``between the
living'').  A special category of gift law exists to deal with gifts that are
not given in a will, but are given because the donor fears he is soon to die. 
Again, concerns about interfering with the law of wills and estates shape
judicial treatment of this category, known as gifts \textit{causa mortis}
(literally, ``gifts on account [or `because'] of death'').

The elements of a gift \textit{causa mortis} are the same as the elements of an
\textit{inter vivos} gift: (1) intent, (2) delivery, and (3) acceptance, but
the donor must also (4) anticipate imminent death. A gift \textit{causa mortis}
is subject to a condition subsequent: if the donor survives the peril that
caused her to fear death, the gift is either revoked or revocable.  In most
states, the gift is revoked automatically, while in others the donor may choose
to revoke the gift.  In the latter states, delay may be troublesome. 
\textit{See} \textsc{Restatement (Second) of Property: Donative Transfers}
\S~31.3 (``A
failure to revoke within a reasonable time after the donor is no longer in
apprehension of imminent death eliminates the right of revocation.'').  In all
states, if the donor dies from the anticipated cause, then the gift becomes
irrevocable.  Some jurisdictions extend this to situations in which the donor
dies from something else within roughly the same time frame or in which the
cause of death is related to the anticipated peril.

Suppose D is going into the hospital for heart surgery that might end in death. 
She says to her son, ``If I die, I want you to have the contents of my safe
deposit box,'' and gives him the key.  While the surgery is a success, she dies
a week later from an infection acquired in the hospital.  Is the gift valid?
What if she dies six months later from the same infection?  \emph{See Brind v. Int'l
Trust Co.}, 179 P. 148 (Colo. 1919) (putative donor didn't die from the
operation that caused her to fear death, but six months later from the ailment
that had triggered the operation; held: no gift \textit{causa mortis}, because
putative donor was specific about the operation as the cause of the gift, and
her lawyer told her that she probably needed to take further action to reaffirm
the gift, but she didn't).  

Courts are often suspicious of gifts \textit{causa mortis}.  Courts may apply
the delivery requirement more stringently than in other gift cases.  Is this
reluctance justified? 

For example, in \textit{Foster v. Reiss}, 112 A.2d 553 (N.J. 1955), the
putative donee obtained the property at issue by taking a note written to him
from the hospital bedside of his estranged wife, who was then unconscious.  The
note disclosed the location of money and bank books (which gave access to
savings accounts) hidden in their house.  The husband found out about the note
from a friend who'd been directed to tell him about it.  He took the note, went
home, and found the cash and the bank books.  She died a few days later, never
having regained consciousness.  Her will gave \$1 to her estranged husband and
the rest of her estate to her children and grandchildren, who sued to recover
the cash and the bank books.  The court held that there had been no gift
\textit{causa mortis}, and said the following:
\begin{quotation}
[A] gift \textit{causa mortis} is essentially of a testamentary nature and as a
practical matter the doctrine, though well established, is an invasion into the
province of the statute of wills \dots. 

``These gifts \textit{causa mortis} are dangerous things. The law requires,
before Mr. Hitt can come into this court and claim \$10,000 as an ordinary
testamentary gift from Mrs. Thompson, that he should produce an instrument in
writing signed by Mrs. Thompson, and also acknowledged with peculiar solemnity
by her in the presence of two witnesses, who thereupon subscribed their names
as witnesses. That is what Mr. Hitt would have to prove if he claimed a
testamentary gift in the ordinary form of one-third of Mrs. Thompson's estate.
And yet, in cases of these gifts \textit{causa mortis}, it is possible that a
fortune of a million dollars can be taken away from the heirs, the next of kin
of a deceased person, by a stranger, who simply has possession of the fortune,
claims that he received it by way of gift, and brings parol testimony to
sustain that claim.'' \emph{Varick v. Hitt}, 55 A. 139, 153 (Ch.1903),
\emph{affirmed} 66
N.J. Eq. 442, 57 A. 406 (E. \& A. 1904).

Gifts \textit{causa mortis} are not favored in the law\dots ``for the reason
that this mode of disposition permits property without limit of value to be
transferred by mere delivery, and the proof thereof to be made when death has
closed the lips of the claimed donor.''\ldots

The first question confronting us is whether there has been ``actual,
unequivocal, and complete delivery during the lifetime of the donor, wholly
divesting him (her) of the possession, dominion, and control'' of the
property.\ldots

``\ldots The test was this: that the transfer was such that, in conjunction with
the donative intention, it completely stripped the donor of his dominion of the
thing given, whether that thing was a tangible chattel or a chose in action.''

Thus, under New Jersey law actual delivery of the property is still required
except where ``there can be no actual delivery'' or where ``the situation is
incompatible with the performance of such ceremony.'' In the case of a savings
account, where obviously there can be no actual delivery, delivery of the
passbook or other indicia of title is required. 
\end{quotation}
The court found that there had been no delivery.  Instead, the putative donee
had merely taken possession of the property, at a time when the would-be donor
was incapacitated and incapable of authorizing him to act for her.  The court
emphasized the separateness of the two elements of intent and delivery:
\begin{quotation}
As stated in \emph{Madison Trust Co. v. Allen}, \emph{supra}, 105 N.J. Eq. 230,
235, 147 A.
546, 548, ``the burden of proof is upon the alleged donee to clearly prove both
delivery \textit{and} donative intent'' (emphasis supplied). This was clearly
brought out by the court in \emph{Parker v. Copland}, 70 N.J. Eq. 685, 64 A.
129, 130 (E. \& A. 1906):

``\ldots [T]he crucial test is not the strenuousness of the language in which
the gift is couched, but in `the transfer,' which is something that is both
different from the donative intention and yet capable of acting in conjunction
with it, so that both are necessary to the creation of an enforceable
gift.\ldots [W]hen two steps are required by law to complete a transaction, the
excess of one cannot supply the lack of the other\ldots.''

Thus, an informal writing such as we have here does not satisfy the separate and
distinct requirement of delivery, but rather there must be such delivery of the
property that the donor stands absolutely deprived of his control over it.\ldots

We must not forget that since a gift \textit{causa mortis} is made in
contemplation of death and is subject to revocation by the donor up to the time
of his death it differs from a legacy only in the requirement of delivery.
Delivery is in effect the only safeguard imposed by law upon a transaction
which would ordinarily fall within the statute of wills. To eliminate delivery
from the requirements for a gift \textit{causa mortis} would be to permit any
writing to effectuate a testamentary transfer, even though it does not comply
with the requirements of the statute of wills.
\end{quotation}
The court quoted an earlier case emphasizing the risks of false testimony in
such cases: ``Around every other disposition of the property of the dead,
the legislative power has thrown safeguards against fraud and perjury; around
this mode the requirement of actual delivery is the only substantial
protection, and the courts should not weaken it by permitting the substitution
of convenient and easily proven devices.''

A strong dissent emphasized that the donor was fully competent when she wrote
her note, clearly intended to make the gift, and never revoked the gift.  The
dissent would have honored her clearly stated intent because ``justice fairly
cries out for the fulfillment of [the] wife's wishes'':
\begin{quotation}
I find neither reason nor persuasive authority anywhere which compels this
untoward result. \emph{See Gulliver and Tilson}, Classification of Gratuitous
Transfers, 51 Yale L.J. 1, 2 (1941):

``One fundamental proposition is that, under a legal system recognizing the
individualistic institution of private property and granting to the owner the
power to determine his successors in ownership, the general philosophy of the
courts should favor giving effect to an intentional exercise of that power.
This is commonplace enough but it needs constant emphasis, for it may be
obscured or neglected in inordinate preoccupation with detail or dialectic. A
court absorbed in purely doctrinal arguments may lose sight of the important
and desirable objective of sanctioning what the transferor wanted to do, even
though it is convinced that he wanted to do it.''
\end{quotation}
Concerns over fraud or uncertainty, the dissent thought, were irrelevant here,
where the donor's wishes ``were freely and clearly expressed in a written
instrument and the donee's ensuing possession was admittedly bona fide.''  The
dissent noted that, in contradiction to New Jersey's approach, other courts
have \textit{relaxed} the delivery requirement in cases of gifts \textit{causa
mortis}, rather than strengthening it.  Such courts reason that gifts
\textit{causa mortis} generally come about as the result of some emergency that
makes it impossible to write a formal will.  While delivery is still important
to avoid problems of figuring out what was really given, the requirements for
sufficient delivery ought to be liberally interpreted to protect the donor's
intent.  Here, for example, the wife's authorization of the husband to take
physical possession, and the fact that he did indeed take physical possession
before she died, ought to have sufficed.  As the dissent saw it, ``[w]hen Ethel
Reiss signed the note and arranged to have her husband receive it, she did
everything that could reasonably have been expected of her to effectuate the
gift \textit{causa mortis}; and while her husband might conceivably have
attempted to return the donated articles to her at the hospital for immediate
redelivery to him, it would have been unnatural for him to do so.''

Which position is more persuasive to you?  Should it make a difference if the
putative donee were an unrelated friend?  If the heirs named in the will were
unrelated friends?

As noted above, in many cases, a donee's control over the place in which the
gift was left is likely to suffice for delivery.  Why didn't the husband's
possession of the house in which the money was hidden in \textit{Foster}
suffice for delivery?

