Rules about transferring property are created by law.  There are only certain
ways people can rearrange property relations.  Some rearrangements happen even
if the people involved don't want them, and some don't happen even if the
people involved do want them.  Knowing the rules is a way to understand which
transfers work and why.

There are several methods of transferring property.  The key voluntary methods
are gifts, sales, and transfers at death, which can be divided  into transfers
by will (also known as transfer by devise) and transfers by operation of law
because of the decedent's intestacy (dying without a will).  Although most
litigated transfers involve sales, it is useful to study gifts in order to
appreciate the significance of possession to ownership.  Relatedly, gift law
highlights that some problems in contract law arise only out of
\textit{executory} promises: completed promises involving property will often
be valid as gifts, even if they lacked consideration.  Gift law also provides
an introduction to methods of transferring land, particularly transfers by deed
and transfers by will. 

In order for a valid \term{gift} to occur, three elements must be present: (1)
the
donor must \textit{intend} to give the property as a gift; (2) the donor must
\textit{deliver} the property to the donee; and (3) the donee must
\textit{accept} the gift.  We won't spend much time on the third element,
because when the property has some value, acceptance will generally be presumed
in the absence of an explicit rejection. 

Unlike a sale or a contract, a gift does not require consideration. This leads
to concerns that often shape judicial doctrine.  First, without tangible
consideration, we need to keep people from lying about what was given to them. 
Because gift issues often arise after the alleged donor died, courts have been
concerned to protect the donor's heirs from having the donor's estate stripped
by people who claim to be donees. 

Second and relatedly, we desire to protect the system of written wills and to
encourage its use. A standard will must be signed and witnessed.  A system that
easily allows pre-mortem gifts might undermine people's incentives to take the
time to write a will---they might think they can always just give their
property away when death approaches---and also harm the legitimate
expectations of those who are named in a will. If the person who writes a will,
known as the testator, identifies specific property in her will, but sells it
or gives it away before she dies, the devise in the will is nullified; it's no
longer her property to give away when she dies.  Although people named as
devisees in a will have no \textit{legal} rights to the property before the
testator dies, they might nonetheless have practically and morally compelling
expectations---especially if we worry about the people surrounding a dying
person exercising undue influence and extracting gifts that the dying person
wouldn't give if she were thinking more clearly.  Thus, by making it more
difficult to give gifts, we may protect the overall system of property
transfers. This concern can lead courts to find that no gift has been made even
when the would-be donor very clearly wanted to give the property away. 
Consider as you read whether this overall structural concern is justified.

