\reading[Blackstone, \emph{Commentaries on the Laws of England}]{William
Blackstone, \textit{Commentaries on the Laws of England}}
\readingcite{Vol. 2, p. 2 (1765)}

There is nothing which so generally strikes the imagination, and engages the
affections of mankind, as the right of property; or that sole and despotic
dominion which one man claims and exercises over the external things of the
world, in total exclusion of the right of any other individual in the universe.
And yet there are very few, that will give themselves the trouble to consider
the original and foundation of this right. Pleased as we are with the
possession, we seem afraid to look back to the means by which it was acquired,
as if fearful of some defect in our title; or at best we rest satisfied with the
decision of the laws in our favour, without examining the reason or authority
upon which those laws have been built. We think it enough that our title is
derived by the grant of the former proprietor, by descent from our ancestors, or
by the last will and testament of the dying owner; not caring to reflect that
(accurately and strictly speaking) there is no foundation in nature or in
natural law, why a set of words upon parchment should convey the dominion of
land; why the son should have a right to exclude his fellow creatures from a
determinate spot of ground, because his father had done so before him; or why
the occupier of a particular field or of a jewel, when lying on his death-bed
and no longer able to maintain possession, should be entitled to tell the rest
of the world which of them should enjoy it after him. These enquiries, it must
be owned, would be useless and even troublesome in common life. It is well if
the mass of mankind will obey the laws when made, without scrutinizing too
nicely into the reasons of making them.

