\begin{quote}
 [A]ll the land in the kingdom is supposed to be holden, mediately or
immediately, of the king; who is stiled the lord paramount, or above all.
\end{quote}
2 \textsc{William Blackstone, Commentaries on the Laws of
England} *59 (1765).

\begin{figure}
\begin{center}
\usegraphic[width=0.4\textwidth]{allocation-img005}
\usegraphic[width=0.4\textwidth]{allocation-img006}
\end{center}
\caption{Source: Bayeux Tapestry. Left: Harold the King Is Slain.
Right: William the Conqueror seated, center.}
\end{figure}

Unlike foxes, whales, and baseballs, \textit{real property}---that is, land and
structures and other improvements attached to land---isn't subject to the
physical control of an individual in the same way chattels are. So what might be
the legal basis for allocating private rights in real property? 

Claims to ownership of land in England trace back as much as a thousand years.
In 1066, William the Bastard, Duke of Normandy, invaded England and defeated the
Anglo-Saxon King Harold at the battle of Hastings---as immortalized in the
Bayeux Tapestry. William---now William the Conqueror---promptly set about
parceling out rights to possess land in his new kingdom. William allocated these
rights according to his political and military needs: affirming the rights of
Anglo-Saxon landholders who supported him, while expropriating the land of his
opponents and reallocating it to his loyal Norman nobles. These nobles received
their rights of \textit{tenure} (from the Latin word \textit{tenere} and Norman
French word \textit{tenir}, ``to hold'') under obligations of \textit{fealty}
(from the Norman French \textit{fedelit\'e} or {fealt\'e}, meaning fidelity or
loyalty); the land each nobleman held was referred to as his \textit{f\'e},
(variations: \textit{fief}, \textit{fee}, \textit{feud}). Hence the name
historians have applied to the resulting social system: \textit{feudalism}.
Feudal obligations typically included payment of taxes in cash or kind and
rendering of services (primarily military services) to the \textit{tenant's}
(holder's) lord and king. This system of feudal grants of possessory and
usufructary rights from the crown evolved over the centuries into the modern
system of land ownership---a historical process we will revisit later in our
chapter on Estates in Land.

Can there be any justification for the allocation of rights in land beyond the
whims of a long-dead warlord and his cronies? In early modern England this was
not merely an academic question. Huge changes in the legal regime governing
rights to land were underway: lands in England long held as ``commons'' were
being progressively ``enclosed'' (i.e., appropriated) by noble families for
their private use, the personal loyalty relationships underlying feudal land
tenure were being supplanted by a more self-consciously economic approach to
land rights, and the colonization of the Americas brought European settlers into
contact---and often conflict---with native Americans. In this period of rapid
change, Britain's leading thinkers turned to the problem of justifying private
property rights in land.

