\expected{pierson-v-post}

\item \textbf{Justifying Allocations.} \label{pierson-justify}Does awarding
ownership of a previously unowned chattel to the first possessor of that chattel
strike you as a good rule? Consider some arguments that might be raised for or
against it:
\begin{itemize}
\item Administrability: Is the rule easy to apply? Does it give clear and ready
answers? Does it make judges' and litigants' jobs easier or harder? Does it
minimize the cost and time involved in resolving disputes? Can it be applied
without resort to ambiguous or hard-to-obtain evidence?
\item Fairness: Does the rule comport with well-considered notions of fairness?
Does it treat similarly situated people similarly? Does it favor some claimants
over others based on criteria that seem irrelevant, arbitrary, or beyond the
claimants' control?
\item Morality: Does the rule reward moral behavior and punish---or at least
refrain from rewarding---immoral behavior? (This assumes of course that we have
a standard for moral and immoral behavior.)
\item Reliance: Does the rule respect the reasonable expectations of those with
an interest in contested resources? Does it result in a forfeiture of their
investment of time, money, or effort premised on such expectations? Does it
comport with tradition?
\item Pragmatism: Does the rule roughly comport with the moral intuitions of
those who are subject to it? Do we expect the rule to be obeyed?
\item Ecology: Is the rule consistent with responsible stewardship of resources?
Does it ensure that an exhaustible resource will remain available for the
benefit of future generations?
\item Incentives: Does the rule encourage or discourage the conversion of idle
resources to productive use? Does it encourage excessive, duplicative, or
wasteful efforts to exploit resources? Does it encourage or discourage disputes
or violence among rival claimants? Does it encourage would-be claimants to
expend resources on protecting themselves \textit{against other} would-be
claimants, instead of on more productive pursuits? When weighing these
incentives in the aggregate, is the rule \textit{efficient}? That is, does it
extract the greatest possible value from available resources at the lowest
possible cost?
\end{itemize}
Which of these arguments strikes you as more or less important to the
justification of a legal rule---particularly a rule of property law? Which of
them were invoked by Justices Tompkins and Livingston in \textit{Pierson}? 

Even if we agree as to which of these arguments matter in disposing of a
particular dispute, are we sure to agree whether a particular type of argument
favors a particular party? For example, is Justice Livingston correct in
claiming that the decision in Pierson's favor will provide insufficient
incentive for hunters to capture foxes? Is Justice Tompkins correct in claiming
that a decision in Post's favor would lead to increased disputes over the
trophies of the chase? Does either opinion clearly establish which outcome would
be the most fair? How could we know the answer to these questions?

\item \textbf{Alternatives to First Possession.} Is the rule of first possession
the best available rule for allocating unowned resources? Consider some possible
alternative allocation principles: 
\begin{itemize}
\item Perhaps initial allocation should go to the first \textit{claimant}---the
first to explicitly assert a right of ownership (or manifest the intent to
assert such a right, as by pursuit).
\item Perhaps initial allocation should go to the \textit{last} possessor---the
person who gains and maintains possession against the efforts of all
competitors. 
\item Perhaps possession is irrelevant: perhaps initial allocation should go to
all interested claimants in equal shares. 
\item Perhaps the resource should be owned as a \textit{commons}: it belongs to
everybody jointly; everybody has an equal right to it and nobody has a superior
right to anyone else.
\item Perhaps the government ought to own everything and simply provide rights
of possession and use by means of bureaucratic and political mechanisms. (Then
again, perhaps this is exactly what the common law of \textit{real} property
does, \having{intro-allocation-land}{as discussed above regarding allocation of
land}{as discussed below with respect to allocation of land}{under the
traditional English doctrine that ``all the land in the kingdom is supposed to
be holden, mediately or immediately, of the king'' who confers title (and
titles, like Duke or Count) on supporters. 2 \textsc{Blackstone, Commentaries}
*59}.)
\item Perhaps ownership should be determined by lot, at random. 
\end{itemize}
How would each of these rules compare to the rule of first possession in terms
of each of the justifications we have just reviewed for and against that rule?
What do you think would be the \textit{practical} result of choosing one of
these alternative allocation regimes---i.e., how would people likely shape their
behavior in response to these allocation rules?

\item Recall the first type of justification we discussed in Note
\ref{pierson-justify} above:
administrability. Do you think it will always be obvious that one claimant of a
chattel has achieved possession and another has not? Consider the following
case.

