\reading{Briggs v. Southwestern Energy Production Company}

\readingcite{224 A.3d 334 (Pa. 2020)}

\opinion Chief Justice \textsc{Saylor}.

In this appeal by allowance, we consider whether the rule of capture immunizes
an energy developer from liability in trespass, where the developer uses
hydraulic fracturing on the property it owns or leases, and such activities
allow it to obtain oil or gas that migrates from beneath the surface of another
person's land.

\readinghead{I. Background}
\readinghead{A. The Rule of Capture}

Oil and gas are minerals, and while in place they are considered part of the
land. They differ from coal and other substances with a fixed situs in that they
are fugacious in nature---meaning they tend to seep or flow across property
lines beneath the surface of the earth. Such underground movement is known as
``drainage.'' Drainage stems from a physical property of fluids in that they
naturally move across a pressure gradient from high to low pressure. Indeed, the
extraction of oil or gas by drilling is based, at least in part, on creating a
low-pressure pathway from the mineral's subterranean location to the earth's
surface.

Oil and gas have thus been described as having a ``fugitive and wandering
existence,'' \textit{Brown v. Vandergrift}, 80 Pa. 142, 147 (Pa. 1875), and have
been compared to wild animals which move about from one property to another.
\textit{See} \textit{Westmoreland \& Cambria Nat. Gas Co. v. DeWitt}, 130 Pa.
235, 249, 18 A. 724, 725 (1889) (``In common with animals, and unlike other
minerals, [oil, gas, and water] have the power and the tendency to escape
without the volition of the owner.''). Accordingly, such minerals are subject to
the rule of capture, which is
\begin{quote}
[a] fundamental principle of oil-and-gas law holding that there is no liability
for drainage of oil and gas from under the lands of another so long as there has
been no trespass \ldots.
\end{quote}
\textsc{Black's Law Dictionary} 1358 (8th ed. 2004)); \textit{accord}
\textit{Brown v. Spilman}, 155 U.S. 665, 669-70, 15 S. Ct. 245, 247, 39 L.Ed.
304 (1895).\readingfootnote{1}{The term ``capture'' is also drawn from an
analogy to wild animals. At common law, a person could acquire title to such an
animal by reducing it to possession.} A corollary to this rule is that an
aggrieved property owner's remedy for the loss, through drainage, of subsurface
oil or gas has traditionally been to offset the effects of the developer's well
by drilling his or her own well, often termed an ``offset well.'' \textit{See}
\textit{Barnard v. Monongahela Gas Co.}, 216 Pa. 362, 365, 65 A. 801, 803 (1907)
(``What then can the neighbor do? Nothing; only go and do likewise.'').

The reference to ``the lands of another'' in the above quote does not suggest a
developer may invade the subsurface area of a neighboring property by drilling
at an angle rather than vertically (referred to as slant drilling or slant
wells), or by drilling horizontally beneath the surface. This is because the
title holder of a parcel of land generally owns everything directly beneath the
surface. Rather, and as suggested by the ``no trespass'' predicate, it refers to
the potential for oil and gas to migrate from the plaintiff's property to the
developer's land when extracted from a common pool or reservoir spanning both
parcels.

\readinghead{B. Hydraulic fracturing}

One of the central questions in this matter involves how these principles apply
where hydraulic fracturing is used to extract oil or gas from subsurface
geological formations. According to the federal government, hydraulic fracturing
is used in ``unconventional'' gas production. ``Unconventional'' reservoirs can
cost-effectively produce gas only by using a special stimulation technique, like
hydraulic fracturing \ldots. This is often because the gas is highly dispersed
in the rock, rather than occurring in a concentrated underground location.
United States Environmental Protection Agency (the ``EPA''), \textit{The Process
of Unconventional Natural Gas Production},
\url{https://www.epa.gov/uog/process-unconventional-natural-gas-production}
(viewed Oct. 22, 2019). In terms of how the technique works, the EPA continues:
\begin{quote}
Fractures are created by pumping large quantities of fluids at high pressure
down a wellbore and into the target rock formation. Hydraulic fracturing fluid
commonly consists of water, proppant and chemical additives that open and
enlarge fractures within the rock formation. These fractures can extend several
hundred feet away from the wellbore. The proppants---sand, ceramic pellets or
other small incompressible particles---hold open the newly created fractures.
\end{quote}
\textit{Id.}

After injection, fluid is withdrawn from the well while leaving the proppants in
place to hold the fissures open. This enhances the drainage of oil or gas into
the wellbore where it can be captured.

\readinghead{C. Factual and Procedural History of This Case}

\readinghead{(i) Introduction}

The parties presently favor essentially the same rule of law: they both, in
substance, argue that the traditional rule of capture should apply, subject to
the common-law standard for trespass of real property based on physical
intrusion onto another's land. Each party, moreover, depicts the other as
erroneously suggesting that an exception to this framework should pertain where
hydraulic fracturing is used to obtain oil or natural gas. In particular, the
plaintiffs suggest that Southwestern wishes to convert the rule of capture into
a precept whereby energy developers may physically invade the property of others
to capture natural gas so long as they are using hydraulic fracturing. For its
part, Southwestern portrays the plaintiffs and the Superior Court decision from
which it appeals as positing that the rule of capture simply does not apply when
hydraulic fracturing is used for energy development on one's own land.

\readinghead{(ii) Undisputed Facts}

Adam, Paula, Joshua, and Sarah Briggs (``Plaintiffs'') own a parcel of real
estate consisting of approximately eleven acres in Harford Township, Susquehanna
County. During all relevant times, Plaintiffs have not leased their property to
any entity for natural gas production. Plaintiffs' property is adjacent to a
tract of land leased by Appellant Southwestern Energy Production Company for
natural gas extraction (the ``Production Parcel''). Southwestern maintains
wellbores on the Production Parcel and has used hydraulic fracturing to boost
natural gas extraction from the Marcellus Shale formation through those
wellbores.

\readinghead{(iii) Proceedings Before the Court of Common Pleas}

In November 2015, Plaintiffs commenced an action against Southwestern in which
they stated two causes of action, trespass and conversion. In Count I (the
trespass claim), Plaintiffs averred that Southwestern's actions constituted a
trespass which deprived Plaintiffs of the value of the ``natural gas extracted
from under their land[.]''In Count II (the conversion claim), Plaintiffs alleged
that, through its drilling activities, Southwestern had deprived Plaintiffs of
their possession and use of the natural gas and converted it to Southwestern's
use. Notably, Plaintiffs did not expressly allege that Southwestern's activities
had caused a physical intrusion into Plaintiffs' property.

Southwestern filed a responsive pleading denying it had extracted gas from
Plaintiffs' land and denying it had trespassed upon Plaintiffs' property or
converted their natural gas. Southwestern specifically denied it had drilled
underneath Plaintiffs' property and stated, further, that it had ``only drilled
for oil, gas or minerals from under properties for which [Southwestern] has
leases.'' 

After the parties engaged in discovery, Southwestern filed a motion for summary
judgment and a supporting brief in which it argued that it did not physically
invade Plaintiffs' property and, to the extent that it had recovered any gas
through drainage from that property to the Production Parcel, again, it was
entitled to judgment as a matter of law under the rule of capture.
Plaintiffs\ldots filed their own motion for partial summary judgment as to
liability, asserting that courts should not apply the rule of capture in
circumstances where gas has been captured through the use of hydraulic
fracturing.

By order and opinion, the common pleas court granted Southwestern's motion for
summary judgment, and denied Plaintiffs' motion for partial summary
judgment\ldots . Plaintiffs filed a notice of appeal,\ldots in which they raised
a single issue: whether the trial court erred in determining that the rule of
capture precluded liability under theories of trespass and conversion, where
Southwestern had used hydraulic fracturing to obtain natural gas which
originated under Plaintiffs' land.

\readinghead{(iv) Proceedings Before the Superior Court}

A two-judge panel of the Superior Court reversed in a published decision.\ldots
The court noted, however, that the record did not indicate whether
Southwestern's operations had resulted in a subsurface intrusion into
Plaintiffs' property, going so far as to express that ``[t]here does not appear
to be \textit{any evidence, or even an estimate}, as to how far the subsurface
fractures extend from each of the wellbore [sic] on Southwestern's
lease.''\ldots Accordingly, the panel reversed the trial court's order and
remanded for additional factual development. 

\ldots [T]he Superior Court panel's analysis can reasonably be viewed as
embodying two distinct, but interrelated, holdings: first, that whenever
``artificial means,'' such as hydraulic fracturing, are used to stimulate the
flow of underground resources, the rule of capture does not apply because
drainage does not occur through the operation of ``natural agencies,'' and
second, that in this particular case summary judgment was premature in light of
certain unspecified allegations relating to cross-boundary intrusions into
Plaintiffs' land.

\readinghead{II. Preliminary Discussion}
\readinghead{A. Trespass}

In Pennsylvania, a trespass occurs when a person who is not privileged to do so
intrudes upon land in possession of another, whether willfully or by mistake.
This conception of trespass is not disputed by the parties. Nevertheless,
meaningful appellate review at this stage is not straightforward for multiple
reasons.

\readinghead{B. Pleading Deficiencies, Decisional Irregularities, and Issue
Limitation}

\ldots Plaintiffs did not assert\ldots in their pleadings\ldots that
Southwestern had effectuated a physical intrusion onto (or into) their property.
The Superior Court panel recognized this aspect of Plaintiffs' litigation
position, but raised and resolved, \textit{sua sponte}, an issue based on the
opposite premise, that Plaintiffs \textit{had} alleged a physical intrusion.
Then, stating that there was no record evidence that such an intrusion had taken
place, and without referencing any specific aspect of the pleadings, the panel
indicated that the Complaint's allegations were alone sufficient to raise a
genuine issue of fact so as to preclude summary judgment. 


This is in some tension with the governing summary-judgment standard which
generally centers on whether the adverse party has produced enough evidence to
raise a question of material fact as to each element of the claim. 

\ldots [M]oreover, Southwestern articulated the issue for this Court's
consideration in terms of whether the rule of capture should be applied in the
same manner it has always been applied: to allow for the capture of oil and gas
which merely drains from an adjacent property after the completion of a well
using hydraulic fracturing \textit{solely within the developer's property}. This
is an issue, again, on which the parties do not presently diverge: they both
answer in the affirmative. Their disagreement is limited to whether any physical
intrusion has taken place---a question that is not fairly subsumed within the
issue framed for our review.

\readinghead{III. Analysis}

The issue as stated by Southwestern should nonetheless be resolved for purposes
of this dispute---and to provide guidance to the bench and bar---because at
least part of the Superior Court's opinion can reasonably be construed as
setting forth a \textit{per se} rule foreclosing application of the rule of
capture in hydraulic fracturing scenarios, and that rule rests on faulty
assumptions. In particular, and most saliently, the panel appears to have
indicated that one litmus for whether the rule of capture applies is whether the
defendant's gas extraction methodology relies only on the natural drainage of
oil or gas within a conventional pool or reservoir, or whether instead those
methods utilize some means of artificial stimulation to induce drainage.


The Superior Court's position in this respect logically rests on one of two
grounds: (a) the act of artificially stimulating the cross-boundary flow through
the use of hydraulic fracturing solely on the developer's property in and of
itself renders the rule of capture inapplicable; or (b) as Plaintiffs argue, any
time natural gas migrates across property lines resulting, directly or
indirectly, from hydraulic fracturing, a physical intrusion into the plaintiff's
property must necessarily have taken place.


As to the first proposition, all drilling for subsurface fugacious minerals
involves the artificial stimulation of the flow of that substance. The mere act
of drilling interferes with nature and stimulates the flow of the minerals
toward artificially-created low pressure areas, most notably, the wellbore. This
Court has held that the rule of capture applies although the driller uses
further artificial means, such as a pump, to enhance production from a source
common to it and the plaintiff---so long as no physical invasion of the
plaintiff's land occurs. \textit{See} \textit{Jones}, 194 Pa. at 384, 44 A. at
1075 (indicating that, absent physical intrusion, a developer may use ``all the
skill and invention of which a man is capable'' to appropriate resources from
under his own property). There is no reason why this precept should apply any
differently to hydraulic fracturing conducted solely within the driller's
property.


\ldots Accordingly, we reject as a matter of law the concept that the rule of
capture is inapplicable to drilling and hydraulic fracturing that occurs
entirely within the developer's property solely because drainage of natural
resources takes place as the direct or indirect result of hydraulic fracturing,
or that such drainage stems from less ``natural'' means than conventional
drainage.


The second predicate---that drainage from under a plaintiff's parcel can only
occur if the driller first physically invades that property---does not lend
itself to a purely legal resolution.\ldots By design, hydraulic fracturing
creates fissures in rock strata which store hydrocarbons within their porous
structure. On the state of the present record, this alone does not establish
that a physical intrusion into a neighboring property is necessary for such
action to result in drainage from that property. We cannot rule out, for
example, that a fissure created through the injection of hydraulic fluid
entirely within the developer's property may create a sufficient pressure
gradient to induce the drainage of hydrocarbons from the relevant stratum of
rock underneath an adjacent parcel even absent physical intrusion. Nor can we
discount the possibility that a fissure created within the developer's property
may communicate with other, pre-existing fissures that reach across property
lines. Whether these, or any other non-invasive means of drainage occasioned by
hydraulic fracturing, are physically possible in a given case is a factual
question to be established through expert evidence.


The Superior Court panel appears to have assumed, if implicitly, that such
occurrences were impossible---but, again, there is no basis in the record for
such an assumption. In all events, a plaintiff asserting a cause of action
``must be able to prove all the elements of his case by proper evidentiary
standards.'' \textit{Papieves v. Lawrence}, 437 Pa. 373, 379, 263 A.2d 118, 121
(1970). Thus, to the extent this lawsuit goes forward on Plaintiffs' new,
physical-intrusion theory, Plaintiffs will bear the burden of demonstrating that
such an intrusion took place.

We have not overlooked Southwestern's argument that trespass should not be
viewed as occurring miles beneath the surface of the earth. As Southwestern
observes, in some jurisdictions traditional concepts of physical trespass have
been relaxed where activities take place miles below the surface and the
plaintiff is not deprived of the use and enjoyment of the land. Southwestern
posits that this is analogous to the principle that trespass does not arise high
above the surface. \textit{See} \textit{Causby}, 328 U.S. at 260-61, 66 S. Ct.
at 1065. It emphasizes that other socially useful endeavors---such as carbon
sequestration projects, energy storage wells, and waste disposal sites---could
be jeopardized if the rule against trespass were to be enforced in an unduly
stringent manner where deep subsurface activities are concerned. 


Without speaking to the merit of such a claim, we note that this Court is
limited to the issue as it was framed in the petition for allowance of appeal,
and Southwestern has not articulated any reason an exception should be made in
the present dispute. Thus, to the extent Southwestern argues it should be
permitted to escape liability even if it is ultimately found to have effectuated
a physical intrusion into Plaintiff's subsurface property, its claim in this
regard has not been preserved for review by this Court.


This brings us to the question of whether the lawsuit can, indeed, progress on a
theory of trespass by physical intrusion, and by extension, to the question of
the appropriate mandate from this Court. Ordinarily, and for the reasons
explained, we would deem any such contention to be absent from the litigation,
as it does not appear to have been mentioned in Plaintiffs' pleadings or argued
as a basis to deny Southwestern's motion for summary judgment. The Superior
Court, however, evidently believed there was some legitimate basis to dispose of
the appeal on the presupposition that Southwestern was alleged to have
physically invaded Plaintiffs' subsurface property with hydraulic fracturing
liquid and proppants; and, as noted, Southwestern has not challenged the
intermediate court's action in this respect. 


That being the case,\ldots we find that the appropriate action at this juncture
is to vacate the Superior Court's order and remand for reconsideration in light
of the guidance provided in this opinion, and the certified record on
appeal\ldots .

\opinion Justice \textsc{Dougherty}[, concurring in part and dissenting in
part:]

I join the majority's holding that the rule of capture remains effective in
Pennsylvania to protect a developer from trespass liability where there has been
no physical invasion of another's property. In so holding, the majority
correctly recognizes that if there \textbf{is} such a physical invasion the rule
of capture will \textbf{not} insulate a developer engaged in hydraulic
fracturing from trespass liability. As I agree with both propositions, I also
agree the matter should be remanded for further proceedings involving a specific
inquiry into a physical invasion. I respectfully dissent, however, from the
notion that this question must be determined by the Superior Court on the
present record\ldots . Given the state of the record, which was apparently not
complete at the time the trial court erroneously entered summary judgment, I
would remand the matter to that court for further proceedings, including the
completion of discovery on the factual question of physical invasion, and trial
thereon as necessary.

