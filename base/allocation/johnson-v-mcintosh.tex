\readingnote{The symbol in McIntosh's name is not an apostrophe but a ``turned
comma,'' or a left single quotation mark. Printers often used the symbol as a
stand-in for a superscript letter c. \emph{See} Michael G. Collins,
\emph{M`Culloch and the Turned Comma}, 12 Green Bag 2d 265 (2009).}
\reading{Johnson v. M`Intosh}

\captionedgraphic{allocation-img007}{Federal Land Patent to William McIntosh}

\readingcite{21 U.S. 543 (1823)}

ERROR to the District Court of Illinois. This was an action of ejectment for
lands in the State and District of Illinois, claimed by the plaintiffs under a
purchase and conveyance from the Piankeshaw Indians, and by the defendant, under
a grant from the United States [dated July 20, 1818]. It came up on a case
stated, upon which there was a judgment below for the defendant. \ldots

\opinion Mr. Chief Justice \textsc{Marshall} delivered the opinion of the Court.

The plaintiffs in this cause claim the land, in their declaration mentioned,
under two grants, purporting to be made, the first in 1773, and the last in
1775, by the chiefs of certain Indian tribes, constituting the Illinois and the
Piankeshaw nations; and the question is, whether this title can be recognised in
the Courts of the United States? 

The facts, as stated in the case agreed, show the authority of the chiefs who
executed this conveyance, so far as it could be given by their own people; and
likewise show, that the particular tribes for whom these chiefs acted were in
rightful possession of the land they sold. The inquiry, therefore, is, in a
great measure, confined to the power of Indians to give, and of private
individuals to receive, a title which can be sustained in the Courts of this
country.

As the right of society, to prescribe those rules by which property may be
acquired and preserved is not, and cannot be drawn into question; as the title
to lands, especially, is and must be admitted to depend entirely on the law of
the nation in which they lie; it will be necessary, in pursuing this inquiry, to
examine, not singly those principles of abstract justice, which the Creator of
all things has impressed on the mind of his creature man, and which are admitted
to regulate, in a great degree, the rights of civilized nations, whose perfect
independence is acknowledged; but those principles also which our own government
has adopted in the particular case, and given us as the rule for our decision.

On the discovery of this immense continent, the great nations of Europe were
eager to appropriate to themselves so much of it as they could respectively
acquire. Its vast extent offered an ample field to the ambition and enterprise
of all; and the character and religion of its inhabitants afforded an apology
for considering them as a people over whom the superior genius of Europe might
claim an ascendency. The potentates of the old world found no difficulty in
convincing themselves that they made ample compensation to the inhabitants of
the new, by bestowing on them civilization and Christianity, in exchange for
unlimited independence. But, as they were all in pursuit of nearly the same
object, it was necessary, in order to avoid conflicting settlements, and
consequent war with each other, to establish a principle, which all should
acknowledge as the law by which the right of acquisition, which they all
asserted, should be regulated as between themselves. This principle was, that
discovery gave title to the government by whose subjects, or by whose authority,
it was made, against all other European governments, which title might be
consummated by possession.

The exclusion of all other Europeans, necessarily gave to the nation making the
discovery the sole right of acquiring the soil from the natives, and
establishing settlements upon it. It was a right with which no Europeans could
interfere. It was a right which all asserted for themselves, and to the
assertion of which, by others, all assented.

Those relations which were to exist between the discoverer and the natives, were
to be regulated by themselves. The rights thus acquired being exclusive, no
other power could interpose between them.

In the establishment of these relations, the rights of the original inhabitants
were, in no instance, entirely disregarded; but were necessarily, to a
considerable extent, impaired. They were admitted to be the rightful occupants
of the soil, with a legal as well as just claim to retain possession of it, and
to use it according to their own discretion; but their rights to complete
sovereignty, as independent nations, were necessarily diminished, and their
power to dispose of the soil at their own will, to whomsoever they pleased, was
denied by the original fundamental principle, that discovery gave exclusive
title to those who made it.

While the different nations of Europe respected the right of the natives, as
occupants, they asserted the ultimate dominion to be in themselves; and claimed
and exercised, as a consequence of this ultimate dominion, a power to grant the
soil, while yet in possession of the natives. These grants have been understood
by all, to convey a title to the grantees, subject only to the Indian right of
occupancy.

\ldots No one of the powers of Europe gave its full assent to this principle,
more unequivocally than England. The documents upon this subject are ample and
complete.\ldots Thus has our whole country been granted by the crown while in
the occupation of the Indians. These [royal] grants purport to convey the soil
as well as the right of dominion to the grantees.\ldots In all of them, the
soil, at the time the grants were made, was occupied by the Indians. Yet almost
every title within those governments is dependent on these grants.\ldots It has
never been objected to this, or to any other similar grant, that the title as
well as possession was in the Indians when it was made, and that it passed
nothing on that account.

These various patents cannot be considered as nullities; nor can they be limited
to a mere grant of the powers of government. A charter intended to convey
political power only, would never contain words expressly granting the land, the
soil, and the waters. Some of them purport to convey the soil alone; and in
those cases in which the powers of government, as well as the soil, are conveyed
to individuals, the crown has always acknowledged itself to be bound by the
grant. Though the power to dismember regal governments was asserted and
exercised, the power to dismember proprietary governments was not claimed; and,
in some instances, even after the powers of government were revested in the
crown, the title of the proprietors to the soil was respected.

\ldots Thus, all the nations of Europe, who have acquired territory on this
continent, have asserted in themselves, and have recognised in others, the
exclusive right of the discoverer to appropriate the lands occupied by the
Indians. Have the American States rejected or adopted this principle?

By the treaty which concluded the war of our revolution, Great Britain
relinquished all claim, not only to the government, but to the `propriety and
territorial rights of the United States,' whose boundaries were fixed in the
second article. By this treaty, the powers of government, and the right to soil,
which had previously been in Great Britain, passed definitively to these States.
We had before taken possession of them, by declaring independence; but neither
the declaration of independence, nor the treaty confirming it, could give us
more than that which we before possessed, or to which Great Britain was before
entitled. It has never been doubted, that either the United States, or the
several States, had a clear title to all the lands within the boundary lines
described in the treaty, subject only to the Indian right of occupancy, and that
the exclusive power to extinguish that right, was vested in that government
which might constitutionally exercise it.

Virginia, particularly, within whose chartered limits the land in controversy
lay, passed an act, in the year 1779, declaring her `exclusive right of
pre-emption from the Indians, of all the lands within the limits of her own
chartered territory, and that no person or persons whatsoever, have, or ever
had, a right to purchase any lands within the same, from any Indian nation,
except only persons duly authorized to make such purchase; formerly for the use
and benefit of the colony, and lately for the Commonwealth.' The act then
proceeds to annul all deeds made by Indians to individuals, for the private use
of the purchasers.

\ldots In pursuance of the same idea, Virginia proceeded, at the same session,
to open her land office, for the sale of that country which now constitutes
Kentucky, a country, every acre of which was then claimed and possessed by
Indians, who maintained their title with as much persevering courage as was ever
manifested by any people.

The States, having within their chartered limits different portions of territory
covered by Indians, ceded that territory, generally, to the United States, on
conditions expressed in their deeds of cession, which demonstrate the opinion,
that they ceded the soil as well as jurisdiction, and that in doing so, they
granted a productive fund to the government of the Union. The lands in
controversy lay within the chartered limits of Virginia, and were ceded with the
whole country northwest of the river Ohio. This grant contained reservations and
stipulations, which could only be made by the owners of the soil; and concluded
with a stipulation, that `all the lands in the ceded territory, not reserved,
should be considered as a common fund, for the use and benefit of such of the
United States as have become, or shall become, members of the confederation,'
\&c. `according to their usual respective proportions in the general charge and
expenditure, and shall be faithfully and \textit{bona fide} disposed of for that
purpose, and for no other use or purpose whatsoever.'

The ceded territory was occupied by numerous and warlike tribes of Indians; but
the exclusive right of the United States to extinguish their title, and to grant
the soil, has never, we believe, been doubted.

\ldots The United States, then, have unequivocally acceded to that great and
broad rule by which its civilized inhabitants now hold this country. They hold,
and assert in themselves, the title by which it was acquired. They maintain, as
all others have maintained, that discovery gave an exclusive right to extinguish
the Indian title of occupancy, either by purchase or by conquest; and gave also
a right to such a degree of sovereignty, as the circumstances of the people
would allow them to exercise.

The power now possessed by the government of the United States to grant lands,
resided, while we were colonies, in the crown, or its grantees. The validity of
the titles given by either has never been questioned in our Courts. It has been
exercised uniformly over territory in possession of the Indians. The existence
of this power must negative the existence of any right which may conflict with,
and control it. An absolute title to lands cannot exist, at the same time, in
different persons, or in different governments. An absolute, must be an
exclusive title, or at least a title which excludes all others not compatible
with it. All our institutions recognise the absolute title of the crown, subject
only to the Indian right of occupancy, and recognise the absolute title of the
crown to extinguish that right. This is incompatible with an absolute and
complete title in the Indians.

\ldots Conquest gives a title which the Courts of the conqueror cannot deny,
whatever the private and speculative opinions of individuals may be, respecting
the original justice of the claim which has been successfully asserted. The
British government, which was then our government, and whose rights have passed
to the United States, asserted a title to all the lands occupied by Indians,
within the chartered limits of the British colonies. It asserted also a limited
sovereignty over them, and the exclusive right of extinguishing the title which
occupancy gave to them. These claims have been maintained and established as far
west as the river Mississippi, by the sword. The title to a vast portion of the
lands we now hold, originates in them. It is not for the Courts of this country
to question the validity of this title, or to sustain one which is incompatible
with it.

\ldots Although we do not mean to engage in the defence of those principles
which Europeans have applied to Indian title, they may, we think, find some
excuse, if not justification, in the character and habits of the people whose
rights have been wrested from them.

The title by conquest is acquired and maintained by force. The conqueror
prescribes its limits. Humanity, however, acting on public opinion, has
established, as a general rule, that the conquered shall not be wantonly
oppressed, and that their condition shall remain as eligible as is compatible
with the objects of the conquest. Most usually, they are incorporated with the
victorious nation, and become subjects or citizens of the government with which
they are connected. The new and old members of the society mingle with each
other; the distinction between them is gradually lost, and they make one people.
Where this incorporation is practicable, humanity demands, and a wise policy
requires, that the rights of the conquered to property should remain unimpaired;
that the new subjects should be governed as equitably as the old, and that
confidence in their security should gradually banish the painful sense of being
separated from their ancient connexions, and united by force to strangers.

When the conquest is complete, and the conquered inhabitants can be blended with
the conquerors, or safely governed as a distinct people, public opinion, which
not even the conqueror can disregard, imposes these restraints upon him; and he
cannot neglect them without injury to his fame, and hazard to his power.

But the tribes of Indians inhabiting this country were fierce savages, whose
occupation was war, and whose subsistence was drawn chiefly from the forest. To
leave them in possession of their country, was to leave the country a
wilderness; to govern them as a distinct people, was impossible, because they
were as brave and as high spirited as they were fierce, and were ready to repel
by arms every attempt on their independence.

What was the inevitable consequence of this state of things? The Europeans were
under the necessity either of abandoning the country, and relinquishing their
pompous claims to it, or of enforcing those claims by the sword, and by the
adoption of principles adapted to the condition of a people with whom it was
impossible to mix, and who could not be governed as a distinct society, or of
remaining in their neighbourhood, and exposing themselves and their families to
the perpetual hazard of being massacred.

Frequent and bloody wars, in which the whites were not always the aggressors,
unavoidably ensued. European policy, numbers, and skill, prevailed. As the white
population advanced, that of the Indians necessarily receded. The country in the
immediate neighbourhood of agriculturists became unfit for them. The game fled
into thicker and more unbroken forests, and the Indians followed. The soil, to
which the crown originally claimed title, being no longer occupied by its
ancient inhabitants, was parcelled out according to the will of the sovereign
power, and taken possession of by persons who claimed immediately from the
crown, or mediately, through its grantees or deputies.

That law which regulates, and ought to regulate in general, the relations
between the conqueror and conquered, was incapable of application to a people
under such circumstances. The resort to some new and different rule, better
adapted to the actual state of things, was unavoidable. Every rule which can be
suggested will be found to be attended with great difficulty.

However extravagant the pretension of converting the discovery of an inhabited
country into conquest may appear; if the principle has been asserted in the
first instance, and afterwards sustained; if a country has been acquired and
held under it; if the property of the great mass of the community originates in
it, it becomes the law of the land, and cannot be questioned. So, too, with
respect to the concomitant principle, that the Indian inhabitants are to be
considered merely as occupants, to be protected, indeed, while in peace, in the
possession of their lands, but to be deemed incapable of transferring the
absolute title to others. However this restriction may be opposed to natural
right, and to the usages of civilized nations, yet, if it be indispensable to
that system under which the country has been settled, and be adapted to the
actual condition of the two people, it may, perhaps, be supported by reason, and
certainly cannot be rejected by Courts of justice.

\ldots After bestowing on this subject a degree of attention which was more
required by the magnitude of the interest in litigation, and the able and
elaborate arguments of the bar, than by its intrinsic difficulty, the Court is
decidedly of opinion, that the plaintiffs do not exhibit a title which can be
sustained in the Courts of the United States; and that there is no error in the
judgment which was rendered against them in the District Court of Illinois.

Judgment affirmed, with costs.

