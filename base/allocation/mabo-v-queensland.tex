\reading{Mabo v. Queensland (No. 2) [``Mabo's Case'']}

\readingcite{High Court of Australia, (1992) 175 C.L.R. 1}

\opinion \textsc{Brennan} J. 

The Murray Islands lie in the Torres Strait, at about 10 degrees S. Latitude and
144 degrees E. Longitude. They are the easternmost of the Eastern Islands of the
Strait. Their total land area is of the order of 9 square kilometres. The
biggest is Mer (known also as Murray Island), oval in shape about 2.79 kms long
and about 1.65 kms across.\ldots The people who were in occupation of these
Islands before first European contact and who have continued to occupy those
Islands to the present day are known as the Meriam people.\ldots The Meriam
people of today retain a strong sense of affiliation with their forbears and
with the society and culture of earlier times. They have a strong sense of
identity with their Islands. The plaintiffs are members of the Meriam people. In
this case, the legal rights of the members of the Meriam people to the land of
the Murray Islands are in question.

\ldots It may be assumed that on 1 August 1879 the Meriam people knew nothing of
the events in Westminster and in Brisbane that effected the annexation of the
Murray Islands and their incorporation into Queensland and that, had the Meriam
people been told of the Proclamation [of annexation] made in Brisbane on 21 July
1879, they would not have appreciated its significance. The legal consequences
of these events are in issue in this case. Oversimplified, the chief question in
this case is whether these transactions had the effect on 1 August 1879 of
vesting in the Crown absolute ownership of, legal possession of and exclusive
power to confer title to, all land in the Murray Islands. The defendant submits
that that was the legal consequence of the Letters Patent and of the events
which brought them into effect. If that submission be right, the Queen took the
land occupied by Meriam people on 1 August 1879 without their knowing of the
expropriation; they were no longer entitled without the consent of the Crown to
continue to occupy the land they had occupied for centuries past. 

\ldots In discharging its duty to declare the common law of Australia, this
Court is not free to adopt rules that accord with contemporary notions of
justice and human rights if their adoption would fracture the skeleton of
principle which gives the body of our law its shape and internal consistency.
Australian law is not only the historical successor of, but is an organic
development from, the law of England. Although our law is the prisoner of its
history, it is not now bound by decisions of courts in the hierarchy of an
Empire then concerned with the development of its colonies\ldots . It is not
possible, a priori, to distinguish between cases that express a skeletal
principle and those which do not, but no case can command unquestioning
adherence if the rule it expresses seriously offends the values of justice and
human rights (especially equality before the law) which are aspirations of the
contemporary Australian legal system. If a postulated rule of the common law
expressed in earlier cases seriously offends those contemporary values, the
question arises whether the rule should be maintained and applied. Whenever such
a question arises, it is necessary to assess whether the particular rule is an
essential doctrine of our legal system and whether, if the rule were to be
overturned, the disturbance to be apprehended would be disproportionate to the
benefit flowing from the overturning.

\ldots International law [at the time of colonization of Australia by Britain]
recognized conquest, cession, and occupation of territory that was \textit{terra
nullius} as three of the effective ways of acquiring sovereignty\ldots . Various
justifications for the acquisition of sovereignty over the territory of
``backward peoples'' were advanced. The benefits of Christianity and European
civilization had been seen as a sufficient justification from mediaeval times.
Another justification for the application of the theory of \textit{terra
nullius} to inhabited territory---a justification first advanced by Vattel at
the end of the 18th century---was that new territories could be claimed by
occupation if the land were uncultivated, for Europeans had a right to bring
lands into production if they were left uncultivated by the indigenous
inhabitants.

\ldots The fiction by which the rights and interests of indigenous inhabitants
in land were treated as non-existent was justified by a policy which has no
place in the contemporary law of this country\ldots . Whatever the justification
advanced in earlier days for refusing to recognize the rights and interests in
land of the indigenous inhabitants of settled colonies, an unjust and
discriminatory doctrine of that kind can no longer be accepted.\ldots It is
contrary both to international standards and to the fundamental values of our
common law to entrench a discriminatory rule which, because of the supposed
position on the scale of social organization of the indigenous inhabitants of a
settled colony, denies them a right to occupy their traditional lands. It was
such a rule which evoked from Deane J.[, in] Gerhardy v. Brown (1985) 159 CLR
70, at p. 149[,] the criticism that---

``the common law of this land has still not reached the stage of retreat from
injustice which the law of Illinois and Virginia had reached in 1823 when
Marshall C.J., in Johnson v. McIntosh, accepted that, subject to the assertion
of ultimate dominion (including the power to convey title by grant) by the
State, the `original inhabitants' should be recognized as having `a legal as
well as just claim' to retain the occupancy of their traditional lands''.

However, recognition by our common law of the rights and interests in land of
the indigenous inhabitants of a settled colony would be precluded if the
recognition were to fracture a skeletal principle of our legal system.\ldots

The land law of England is based on the doctrine of tenure. In English legal
theory, every parcel of land in England is held either mediately or immediately
of the King who is the Lord Paramount; the term ``tenure'' is used to signify
the relationship between tenant and lord, not the relationship between tenant
and land\ldots . When the Crown acquired territory outside England which was to
be subject to the common law, there was a natural assumption that the doctrine
of tenure should be the basis of the land law. Perhaps the assumption did not
have to be made\ldots .

By attributing to the Crown a radical title\edfootnote{``Radical title'' is a
subtle and unsettled concept; it may refer here to the common-law principle that
the government---i.e., the crown---is the ultimate source of property rights in
land within the territory subject to its
jurisdiction.} to all land within a territory over which the Crown has assumed
sovereignty, the common law enabled the Crown, in exercise of its sovereign
power, to grant an interest in land to be held of the Crown or to acquire land
for the Crown's demesne.\ldots But it is not a corollary of the Crown's
acquisition of a radical title to land in an occupied territory that the Crown
acquired absolute beneficial ownership of that land to the exclusion of the
indigenous inhabitants.\ldots Nor is it necessary to the structure of our legal
system to refuse recognition to the rights and interests in land of the
indigenous inhabitants\ldots .

Recognition of the radical title of the Crown is quite consistent with
recognition of native title to land, for the radical title, without more, is
merely a logical postulate required to support the doctrine of tenure (when the
Crown has exercised its sovereign power to grant an interest in land) and to
support the plenary title of the Crown (when the Crown has exercised its
sovereign power to appropriate to itself ownership of parcels of land within the
Crown's territory). Unless the sovereign power is exercised in one or other of
those ways, there is no reason why land within the Crown's territory should not
continue to be subject to native title. It is only the fallacy of equating
sovereignty and beneficial ownership of land that gives rise to the notion that
native title is extinguished by the acquisition of sovereignty. 

\ldots The ownership of land within a territory in the exclusive occupation of a
people must be vested in that people: land is susceptible of ownership, and
there are no other owners.\ldots Of course, since European settlement of
Australia, many clans or groups of indigenous people have been physically
separated from their traditional land and have lost their connexion with it. But
that is not the universal position. It is clearly not the position of the Meriam
people. Where a clan or group has continued to acknowledge the laws and (so far
as practicable) to observe the customs based on the traditions of that clan or
group, whereby their traditional connexion with the land has been substantially
maintained, the traditional community title of that clan or group can be said to
remain in existence. The common law can, by reference to the traditional laws
and customs of an indigenous people, identify and protect the native rights and
interests to which they give rise. However, when the tide of history has washed
away any real acknowledgment of traditional law and any real observance of
traditional customs, the foundation of native title has disappeared. A native
title which has ceased with the abandoning of laws and customs based on
tradition cannot be revived for contemporary recognition.\ldots Once traditional
native title expires, the Crown's radical title expands to a full beneficial
title, for then there is no other proprietor than the Crown.

It follows that a right or interest possessed as a native title cannot be
acquired from an indigenous people by one who, not being a member of the
indigenous people, does not acknowledge their laws and observe their customs;
nor can such a right or interest be acquired by a clan, group or member of the
indigenous people unless the acquisition is consistent with the laws and customs
of that people. Such a right or interest can be acquired outside those laws and
customs only by the Crown.

\ldots Sovereignty carries the power to create and to extinguish private rights
and interests in land within the Sovereign's territory. It follows that, on a
change of sovereignty, rights and interests in land that may have been
indefeasible under the old regime become liable to extinction by exercise of the
new sovereign power. The sovereign power may or may not be exercised with
solicitude for the welfare of indigenous inhabitants but, in the case of common
law countries, the courts cannot review the merits, as distinct from the
legality, of the exercise of sovereign power.\ldots However, the exercise of a
power to extinguish native title must reveal a clear and plain intention to do
so, whether the action be taken by the Legislature or by the Executive\ldots . A
Crown grant which vests in the grantee an interest in land which is inconsistent
with the continued right to enjoy a native title in respect of the same land
necessarily extinguishes the native title\ldots . Where the Crown grants land in
trust or reserves and dedicates land for a public purpose, the question whether
the Crown has revealed a clear and plain intention to extinguish native title
will sometimes be a question of fact, sometimes a question of law and sometimes
a mixed question of fact and law. Thus, if a reservation is made for a public
purpose other than for the benefit of the indigenous inhabitants, a right to
continued enjoyment of native title may be consistent with the specified
purpose---at least for a time---and native title will not be extinguished. But
if the
land is used and occupied for the public purpose and the manner of occupation is
inconsistent with the continued enjoyment of native title, native title will be
extinguished.\ldots [W]here the Crown has not granted interests in land or
reserved and dedicated land inconsistently with the right to continued enjoyment
of native title by the indigenous inhabitants, native title survives and is
legally enforceable.

[The Court declared that the Murray Islands are not crown lands, that the Meriam
people were entitled to ``possession, occupation, use and enjoyment'' of the
island of Mer (excluding certain parcels leased or physically used by the
Australian, provincial, or local governments), and that the Meriam people's
right to Mer is subject to the power of the Queensland government to extinguish
it by law.]

\opinion \textsc{Mason} C.J. and \textsc{McHugh} J. 

We agree with the reasons for judgment of Brennan J. and with the declaration
which he proposes.

In the result, six [out of seven] members of the Court (Dawson J. dissenting)
are in agreement that the common law of this country recognizes a form of native
title which, in the cases where it has not been extinguished, reflects the
entitlement of the indigenous inhabitants, in accordance with their laws or
customs, to their traditional lands and that, subject to the effect of some
particular Crown leases, the land entitlement of the Murray Islanders in
accordance with their laws or customs is preserved, as native title, under the
law of Queensland. The main difference between those members of the Court who
constitute the majority is that,\ldots neither of us nor Brennan J. agrees with
the conclusion to be drawn from the judgments of Deane, Toohey and Gaudron JJ.
that, at least in the absence of clear and unambiguous statutory provision to
the contrary, extinguishment of native title by the Crown by inconsistent grant
is wrongful and gives rise to a claim for compensatory damages. We note that the
judgment of Dawson J. supports the conclusion of Brennan J. and ourselves on
that aspect of the case since his Honour considers that native title, where it
exists, is a form of permissive occupancy at the will of the Crown.

We are authorized to say that the other members of the Court agree with what is
said in the preceding paragraph about the outcome of the case.

[Opinions of Toohey and Gaudron JJ. and Dawson J. omitted.]

