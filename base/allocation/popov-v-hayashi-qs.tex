\expected{popov-v-hayashi}
\expected{ghen-v-rich}
\expected{pierson-v-post}

\item \textbf{Splitting the Baby.} The cynical lawyer would call Judge
McCarthy's ruling in \textit{Popov v. Hayashi} a classic example of ``splitting
the baby.'' The implication is that ordering the division of the disputed
chattel is wishy-washy, or a cop-out. This assumes that there is a ``right''
answer that will make one party perfectly happy and utterly disappoint the
other, but for whatever reason the judge has decided to ignore that answer and
instead issue a ruling that tries to give something to everybody and therefore
satisfies nobody.\footnote{In the Old Testament parable from which the idiom is
derived, King Solomon supposedly used this device to suss out the true facts of
the case he was called on to decide---that is, to identify the true mother of a
disputed child. (He did not, in the event, actually split the baby.) (1
\textsc{Kings} 3:16-28.) Why might a judge in a modern court of law issue a
ruling that makes nobody happy? Why do you think Judge McCarthy did so in
\textit{Popov}?}

\captionedgraphic{allocation-img004}{Source: Raphael, Judgment of Solomon.
Vatican Museums.}

Is that a fair critique? Come to think of it, why don't we resolve \textit{all}
disputes over initial ownership of chattels this way? Should Pierson and Post
have split the value of the fox pelt? Should Ghen and Rich (or perhaps Ellis)
have shared the value of the whale oil? (Wouldn't they have done so under the
custom supposedly enforced by the court in that case?) Are there good reasons
\textit{not} to compel competing claimants of a resource to \textit{share}? What
would your kindergarten teacher say?

Your casebook authors would never dare contradict your kindergarten teacher, but
we might venture a few questions: How would you expect competing claimants to a
single, indivisible resource to behave under a rule that requires them to share
that resource? How do adults who share a household usually share the resources
of that household? Does it matter if the people sharing like or respect each
other? How would you expect courts to resolve their disputes under a rule
requiring sharing? What do you expect the reactions to such resolutions would
be? What would be the effect on the value and productive use of such resources? 

Finally, which of the justifications for allocation rules discussed in
Note~\ref{pierson-justify} on page \pageref{pierson-justify}
are implicated by these questions?


\item \label{popov-precedent}\textbf{Precedent.} In common-law systems,
courts rely on \textit{precedent}---earlier decided cases presenting similar
facts and legal issues---to guide their decisions. Precedent may be either
\textit{binding authority}---if it issues from a court with direct appellate
jurisdiction over the court deciding an identical issue---or \textit{persuasive
authority}---if it issues from a different court in an opinion the deciding
court finds well-reasoned and analogous. 

In \textit{Popov} Judge McCarthy cited and relied on our two earlier chattels
cases, \textit{Pierson v. Post} and \textit{Ghen v. Rich}, to justify his
ruling. Do you agree with Judge McCarthy's interpretation of these precedents?
Do you think he applied them correctly to the facts of the case before him? Do
you think he should have relied on these two decisions as persuasive authority
in the \textit{Popov} case?


\item \textbf{Escape and Return.} The common law developed particular rules to
deal with a captured wild animal that later escaped. In general, once such an
animal is free of the control of its captor, that captor loses their property
right in the animal---in becomes once again \textit{ferae naturae}, and a new
captor can become its owner by killing or capturing it, free of any claim by the
original captor. If, however, the animal in question has \textit{animus
revertendi}---a natural tendency to return to its place of captivity (like, say,
homing pigeons, hived bees, or trained hawks)---its temporary departure from the
possession of the original owner does not diminish that owner's property right.
\textit{See} 2 \textsc{William Blackstone, Commentaries} *392-93.

Might the rule of escape have any application to \textit{Popov v. Hayashi}? Or
are there other factors at work in the case that make the rule unhelpful?


\item \textbf{Postscript.} Recall Question \ref{ghen-settlement} on page
\pageref{ghen-settlement}, above. Patrick Hayashi claims that before this case
went to trial, he made a settlement offer to Alex Popov whereby the two men
would essentially do what the court ended up ordering them to do---selling the
ball and dividing the proceeds. Popov, confident in his right to sole ownership,
allegedly countered with a lowball offer of \$5,000 in exchange for return of
the ball.\footnote{Jay Posner, \textit{Possessing 73rd HR ball first made his
life a hassle, then movie}, \textsc{San Diego Union-Tribune} (June 14, 2005),
\url{http://www.utsandiego.com/uniontrib/20050614/news\_1s14bondball.html.}}
This turned out to be\ldots ill advised.

Despite speculation that Barry Bonds's record-setting home-run ball might sell
for a million dollars or more, the controversy over its ownership appears to
have negatively affected its market value. At auction, the ball sold for
\$450,000.\footnote{Ira Berkow, \textit{73rd Home Run Ball Sells for \$450,000},
\textsc{N.Y. Times} (June 26, 2003),
\url{http://www.nytimes.com/2003/06/26/sports/baseball-73rd-home-run-ball-sells-for-450000.html}.}
Split according to the court's order, that came out to \$225,000 for each
party---not a bad haul. But don't forget: this case was bitterly litigated for
over a year---including a trial that proceeded over several weeks---and that
ain't cheap. 

Patrick Hayashi's attorneys ultimately agreed to waive most of their fee
following the resolution of the case, leaving him enough from the proceeds of
the sale to cover the cost of his graduate education. He left San Francisco and
began a happy new life and career in San Diego.\footnote{Gwen Knapp,
\textit{Finally, in Bonds ball case, someone shows some class}, \textsc{S.F.
Chron.} (Dec. 30, 2003) at A1,
\url{http://www.sfgate.com/sports/article/Finally-in-Bonds-ball-case-someone-shows-some-2507738.php}.}

Alex Popov was not so lucky. The day after the ball went under the auction
hammer, Popov's attorney, Martin Triano, obtained a temporary restraining order
freezing Popov's share of the proceeds.\footnote{In re Martin Triano, Case No.
CPF 03 503194, Temporary Restraining Order, June 26, 2003 (Cal. Super. Ct. San.
Francisco Cty.).} Mr. Triano claimed that Popov still owed him attorney's fees
in the amount of \$473,500.\footnote{\textit{Id.}, Petition filed by Martin F.
Triano (June 20, 2003); \textit{see also} David Kravets, Attorney sues fan over
Bonds ball case, USA Today (July 8, 2003),
\url{http://usatoday30.usatoday.com/sports/baseball/nl/giants/2003-07-08-bonds-ball-legal-fees\_x.htm}.}
Alex Popov eventually filed for bankruptcy,\footnote{Bankruptcy Petition \#:
05-32929 (N.D. Cal. Sept. 6, 2005).} but not before suing his attorney for
malpractice and fraud.\footnote{Popov v. Triano, Case No. CGC 04 427956,
Complaint, Jan. 12, 2004 (Cal. Super. Ct. San. Francisco Cty.).} The litigation
between Messrs. Popov and Triano was last before a judge in September 2011,
nearly 10 years after Popov had his fateful brush with a piece of sports (and
legal) history. At that appearance, Mr. Popov was ordered to pay Mr. Triano an
additional \$22,241 in legal fees arising from their decade of litigation
against one another\footnote{In re Martin Triano, Case No. CPF 03 503194, Minute
Entry, Sept. 16, 2011 (Cal. Super. Ct. San. Francisco Cty.) (granting in part
Triano's motion for attorney's fees, in the amount of \$22,241).}---though
one suspects Mr. Triano may have some difficulty collecting the award. (There is
a lesson here for lawyers, not just litigants.)

To learn more about the saga of \textit{Popov v. Hayashi}, and to see video of
the infamous home run itself, we highly recommend the 2004 feature-length
documentary \textit{Up for Grabs}.


\item \textbf{Review and Application.} On September 21, 2008, Jos\'e Molina hit
what would be the last home run at the old Yankee Stadium (which was demolished
following the end of the season to make way for a new, glitzier facility). The
ball sailed into the left-field stands, and was stopped by a net hung over the
seating area specifically for the purpose of protecting fans from incoming fly
balls. Several fans attempted to reach through the net to grab the ball, and
one---Steve Harshman---managed to get his hand around it. But the net was still
between him and the ball. Harshman told reporters he had intended to rip the
ball through the net, but was interrupted by staff at the stadium, who
instructed him to release it while giving assurances that they would return it
to him. Harshman followed the staff's instructions, and the ball rolled down the
net and into an adjacent seating area, where Bronx schoolteacher Paul Russo
caught it. Yankee Stadium staff immediately confronted Russo and instructed him
to turn over the ball. Russo complied, he claimed, because he thought the staff
was offering to secure the ball on his behalf. Instead, to Mr. Russo's surprise
and chagrin, they delivered the ball to Mr. Harshman.\footnote{James Barron,
\textit{At the Stadium, Possession Is Some Tenths of the Law}, \textsc{N.Y.
Times} (Sept. 24, 2008) at B3,
\url{http://www.nytimes.com/2008/09/24/nyregion/24ball.html}.}

Imagine Mr. Russo sues Mr. Harshman for return of the last home-run ball hit at
the House that Ruth Built. What result? Would it matter if Yankee Stadium had a
long-established policy of having its staff deliver game-play balls to fans who
grasp them through protective netting on condition that the fan release the ball
when instructed? Would it matter \textit{why} the organization implemented such
a policy?


\item \textbf{First Possession? Really?} We have now examined three different
cases that purport to resolve a property dispute between an earlier pursuer and
a later captor by reference to the rule of first possession. But each of them
appears to come out a different way. \textit{Pierson} awards the chattel (or its
value) to the captor; \textit{Ghen} to the pursuer; \textit{Popov} to both in
equal shares. Are these three cases really applying the same rule? If so, what
nuances should we add to the maxims ``first in time is first in right'' or
``title goes to the first possessor'' in order to explain the outcomes of these
three cases and help us to resolve factually similar cases we may encounter in
the future? And if not, what are the \textit{multiple} rules or considerations
that govern the initial allocation of rights in chattels? Either way, how should
we justify our rule(s)?

