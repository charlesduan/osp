\reading{Coffin v. Left Hand Ditch Co.}

\readingcite{6 Colo. 443 (1882)}

\opinion \textsc{Helm, J}.

Appellee, who was plaintiff below, claimed to be the owner of certain water by
virtue of an appropriation thereof from the south fork of the St. Vrain creek.
It appears that such water, after its diversion, is carried by means of a ditch
to the James creek, and thence along the bed of the same to Left Hand creek,
where it is again diverted by lateral ditches and used to irrigate lands
adjacent to the last named stream. Appellants are the owners of lands lying on
the margin and in the neighborhood of the St. Vrain below the mouth of said
south fork thereof, and naturally irrigated therefrom.

In 1879 there was not a sufficient quantity of water in the St. Vrain to supply
the ditch of appellee and also irrigate the said lands of appellant. A portion
of appellee's dam was torn out, and its diversion of water thereby seriously
interfered with by appellants. The action is brought for damages arising from
the trespass, and for injunctive relief to prevent repetitions thereof in the
future.\ldots [T]rial was had before a jury\ldots , and verdict and judgment
given for appellee. Such recovery was confined, however, to damages for injury
to the dam alone, and did not extend to those, if any there were, resulting from
the loss of water.

\ldots It is contended by counsel for appellants that the common law principles
of riparian proprietorship prevailed in Colorado until 1876, and that the
doctrine of priority of right to water by priority of appropriation thereof was
first recognized and adopted in the constitution. But we think the latter
doctrine has existed from the date of the earliest appropriations of water
within the boundaries of the state. The climate is dry, and the soil, when
moistened only by the usual rainfall, is arid and unproductive; except in a few
favored sections, artificial irrigation for agriculture is an absolute
necessity. Water in the various streams thus acquires a value unknown in moister
climates. Instead of being a mere incident to the soil, it rises, when
appropriated, to the dignity of a distinct usufructuary estate, or right of
property. It has always been the policy of the national, as well as the
territorial and state governments, to encourage the diversion and use of water
in this country for agriculture; and vast expenditures of time and money have
been made in reclaiming and fertilizing by irrigation portions of our
unproductive territory. Houses have been built, and permanent improvements made;
the soil has been cultivated, and thousands of acres have been rendered
immensely valuable, with the understanding that appropriations of water would be
protected. Deny the doctrine of priority or superiority of right by priority of
appropriation, and a great part of the value of all this property is at once
destroyed.

\ldots We conclude, then, that the common law doctrine giving the riparian owner
a right to the flow of water in its natural channel upon and over his lands,
even though he makes no beneficial use thereof, is inapplicable to Colorado.
Imperative necessity, unknown to the countries which gave it birth, compels the
recognition of another doctrine in conflict therewith. And we hold that, in the
absence of express statutes to the contrary, the first appropriator of water
from a natural stream for a beneficial purpose has, with the qualifications
contained in the constitution, a prior right thereto, to the extent of such
appropriation.

\ldots It is urged, however, that even if the doctrine of priority or
superiority of right by priority of appropriation be conceded, appellee in this
case is not benefited thereby. Appellants claim that they have a better right to
the water because their lands lie along the margin and in the neighborhood of
the St. Vrain. They assert that, as against them, appellee's diversion of said
water to irrigate lands adjacent to Left Hand creek, though prior in time, is
unlawful.

In the absence of legislation to the contrary, we think that the right to water
acquired by priority of appropriation thereof is not in any way dependent upon
the locus of its application to the beneficial use designed. And the disastrous
consequences of our adoption of the rule contended for, forbid our giving such a
construction to the statutes as will concede the same, if they will properly
bear a more reasonable and equitable one.

The doctrine of priority of right by priority of appropriation for agriculture
is evoked, as we have seen, by the imperative necessity for artificial
irrigation of the soil. And it would be an ungenerous and inequitable rule that
would deprive one of its benefit simply because he has, by large expenditure of
time and money, carried the water from one stream over an intervening watershed
and cultivated land in the valley of another. It might be utterly impossible,
owing to the topography of the country, to get water upon his farm from the
adjacent stream; or if possible, it might be impracticable on account of the
distance from the point where the diversion must take place and the attendant
expense; or the quantity of water in such stream might be entirely insufficient
to supply his wants. It sometimes happens that the most fertile soil is found
along the margin or in the neighborhood of the small rivulet, and sandy and
barren land beside the larger stream. To apply the rule contended for would
prevent the useful and profitable cultivation of the productive soil, and
sanction the waste of water upon the more sterile lands. It would have enabled a
party to locate upon a stream in 1875, and destroy the value of thousands of
acres, and the improvements thereon, in adjoining valleys, possessed and
cultivated for the preceding decade. Under the principle contended for, a party
owning land ten miles from the stream, but in the valley thereof, might deprive
a prior appropriator of the water diverted therefrom whose lands are within a
thousand yards, but just beyond an intervening divide.

\ldots The judgment of the court below will be affirmed.

