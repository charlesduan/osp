\reading{Popov v. Hayashi}

\readingcite{2002 WL 31833731 (Cal. Sup. Ct. San Francisco Cty. Dec. 18, 2002)}

\captionedgraphic{allocation-img003}{Source: \textit{Up For Grabs} (Crooked Hook
Productions 2004)}

\opinion \textsc{Mccarthy}, J.

\readinghead{Facts}

In 1927, Babe Ruth hit sixty home runs. That record stood for thirty four years
until Roger Maris broke it in 1961 with sixty one home runs. Mark McGwire hit
seventy in 1998. On October 7, 2001, at PacBell Park in San Francisco, Barry
Bonds hit number seventy three. That accomplishment set a record which, in all
probability, will remain unbroken for years into the future.

The event was widely anticipated and received a great deal of attention.

The ball that found itself at the receiving end of Mr. Bond's bat garnered some
of that attention. Baseball fans in general, and especially people at the game,
understood the importance of the ball. It was worth a great deal of
money\readingfootnote{1}{It has been suggested that the ball might sell for
something in excess of \$1,000,000.} and whoever caught it would bask, for a
brief period of time, in the reflected fame of Mr. Bonds.

With that in mind, many people who attended the game came prepared for the
possibility that a record setting ball would be hit in their direction. Among
this group were plaintiff Alex Popov and defendant Patrick Hayashi. They were
unacquainted at the time. Both men brought baseball gloves, which they
anticipated using if the ball came within their reach.

\ldots When the seventy-third home run ball went into the arcade, it landed in
the upper portion of the webbing of a softball glove worn by Alex Popov. While
the glove stopped the trajectory of the ball, it is not at all clear that the
ball was secure. Popov had to reach for the ball and in doing so, may have lost
his balance.

Even as the ball was going into his glove, a crowd of people began to engulf Mr.
Popov. He was tackled and thrown to the ground while still in the process of
attempting to complete the catch. Some people intentionally descended on him for
the purpose of taking the ball away, while others were involuntarily forced to
the ground by the momentum of the crowd.

Eventually, Mr. Popov was buried face down on the ground under several layers of
people. At one point he had trouble breathing. Mr. Popov was grabbed, hit and
kicked. People reached underneath him in the area of his glove. [The evidence is
insufficient] to establish which individual members of the crowd were
responsible for the assaults on Mr. Popov.

Mr. Popov intended at all times to establish and maintain possession of the
ball. At some point the ball left his glove and ended up on the ground. It is
impossible to establish the exact point in time that this occurred or what
caused it to occur.

Mr. Hayashi was standing near Mr. Popov when the ball came into the stands. He,
like Mr. Popov, was involuntarily forced to the ground. He committed no wrongful
act. While on the ground he saw the loose ball. He picked it up, rose to his
feet and put it in his pocket.

\ldots It is important to point out what the evidence did not and could not
show. Neither the camera [of a local news team fortuitously recording the
incident] nor the percipient witnesses were able to establish whether Mr. Popov
retained control of the ball as he descended into the crowd. Mr. Popov's
testimony on this question is inconsistent on several important points,
ambiguous on others and, on the whole, unconvincing. We do not know when or how
Mr. Popov lost the ball.

Perhaps the most critical factual finding of all is one that cannot be made. We
will never know if Mr. Popov would have been able to retain control of the ball
had the crowd not interfered with his efforts to do so. Resolution of that
question is the work of a psychic, not a judge.

\readinghead{Legal Analysis}

Plaintiff has pled causes of actions for conversion, trespass to chattel,
injunctive relief and constructive trust.

Conversion is the wrongful exercise of dominion over the personal property of
another.\ldots If a person entitled to possession of personal property demands
its return, the unjustified refusal to give the property back is conversion. 

\ldots Conversion does not exist, however, unless the baseball rightfully
belongs to Mr. Popov. One who has neither title nor possession, nor any right to
possession, cannot sue for conversion. The deciding question in this case then,
is whether Mr. Popov achieved possession or the right to possession as he
attempted to catch and hold on to the ball.

The parties have agreed to a starting point for the legal analysis. Prior to the
time the ball was hit, it was possessed and owned by Major League Baseball. At
the time it was hit it became intentionally abandoned property. The first person
who came in possession of the ball became its new owner. 

\ldots Although the term possession appears repeatedly throughout the law, its
definition varies depending on the context in which it is used. Various courts
have condemned the term as vague and meaningless. 

This level of criticism is probably unwarranted.

While there is a degree of ambiguity built into the term possession, that
ambiguity exists for a purpose. Courts are often called upon to resolve
conflicting claims of possession in the context of commercial disputes. A stable
economic environment requires rules of conduct which are understandable and
consistent with the fundamental customs and practices of the industry they
regulate. Without that, rules will be difficult to enforce and economic
instability will result. Because each industry has different customs and
practices, a single definition of possession cannot be applied to different
industries without creating havoc.

This does not mean that there are no central principles governing the law of
possession. It is possible to identify certain fundamental concepts that are
common to every definition of possession.

\ldots We start with the observation that possession is a process which
culminates in an event. The event is the moment in time that possession is
achieved. The process includes the acts and thoughts of the would be possessor
which lead up to the moment of possession.

The focus of the analysis in this case is not on the thoughts or intent of the
actor. Mr. Popov has clearly evidenced an intent to possess the baseball and has
communicated that intent to the world.\readingfootnote{23}{Literally.} The
question is whether he did enough to reduce the ball to his exclusive dominion
and control. Were his acts sufficient to create a legally cognizable interest in
the ball?

Mr. Hayashi argues that possession does not occur until the fan has complete
control of the ball. Professor Brian Gray, suggests the following definition[:]
``A person who catches a baseball that enters the stands is its owner. A ball is
caught if the person has achieved complete control of the ball at the point in
time that the momentum of the ball and the momentum of the fan while attempting
to catch the ball ceases. A baseball, which is dislodged by incidental contact
with an inanimate object or another person, before momentum has ceased, is not
possessed. Incidental contact with another person is contact that is not
intended by the other person. The first person to pick up a loose ball and
secure it becomes its possessor.''\readingfootnote{24}{This definition is
hereinafter referred to as Gray's Rule.}

Mr. Popov argues that this definition requires that a person seeking to
establish possession must show unequivocal dominion and control, a standard
rejected by several leading cases.\readingfootnote{25}{\textit{Pierson v. Post},
3 Caines R. (N.Y.1805).} Instead, he offers the perspectives of Professor
Bernhardt and Professor Paul Finkelman who suggest that possession occurs when
an individual intends to take control of a ball and manifests that intent by
stopping the forward momentum of the ball whether or not complete control is
achieved.

Professors Finkelman and Bernhardt have correctly pointed out that some cases
recognize possession even before absolute dominion and control is achieved.
Those cases require the actor to be actively and ably engaged in efforts to
establish complete control.\readingfootnote{27}{The degree of control necessary
to establish possession varies from circumstance to circumstance. ``The
law\ldots does not always require that one who discovers lost or abandoned
property must actually have it in hand before he is vested with a legally
protected interest. The law protects not only the title acquired by one who
finds lost or abandoned property but also the right of the person who discovers
such property, and is actively and ably engaged in reducing it to possession, to
complete this process without interference from another. The courts have
recognized that in order to acquire a legally cognizable interest in lost or
abandoned property a finder need not always have manual possession of the thing.
Rather, a finder may be protected by taking such constructive possession of the
property as its nature and situation permit.'' \textit{Treasure Salvors Inc. v.
The Unidentified Wrecked and Abandoned Sailing Vessel} 640 F.2d 560, 571
(1981).} Moreover, such efforts must be significant and they must be reasonably
calculated to result in unequivocal dominion and control at some point in the
near future. 

This rule is applied in cases involving the hunting or fishing of wild
animals\readingfootnote{29}{\ldots \textit{Ghen v. Rich} 8 F. 159 (D.Mass.1881);
\textit{Pierson v. Post} 3 Caines R. (N.Y.1805)\ldots .} or the salvage of
sunken vessels. The hunting and fishing cases recognize that a mortally wounded
animal may run for a distance before falling. The hunter acquires possession
upon the act of wounding the animal not the eventual capture. Similarly, whalers
acquire possession by landing a harpoon, not by subduing the animal. 

In the salvage cases, an individual may take possession of a wreck by exerting
as much control ``as its nature and situation permit''. Inadequate efforts,
however, will not support a claim of possession. Thus, a ``sailor cannot assert
a claim merely by boarding a vessel and publishing a notice, unless such acts
are coupled with a then present intention of conducting salvage operations, and
he immediately thereafter proceeds with activity in the form of constructive
steps to aid the distressed party.''

These rules are contextual in nature. They are crafted in response to the unique
nature of the conduct they seek to regulate. Moreover, they are influenced by
the custom and practice of each industry. The reason that absolute dominion and
control is not required to establish possession in the cases cited by Mr. Popov
is that such a rule would be unworkable and unreasonable. The ``nature and
situation'' of the property at issue does not immediately lend itself to
unequivocal dominion and control. It is impossible to wrap one's arms around a
whale, a fleeing fox or a sunken ship.

The opposite is true of a baseball hit into the stands of a stadium. Not only is
it physically possible for a person to acquire unequivocal dominion and control
of an abandoned baseball, but fans generally expect a claimant to have
accomplished as much. The custom and practice of the stands creates a reasonable
expectation that a person will achieve full control of a ball before claiming
possession. There is no reason for the legal rule to be inconsistent with that
expectation. Therefore Gray's Rule is adopted as the definition of possession in
this case.

The central [tenet] of Gray's Rule is that the actor must retain control of the
ball after incidental contact with people and things. Mr. Popov has not
established by a preponderance of the evidence that he would have retained
control of the ball after all momentum ceased and after any incidental contact
with people or objects. Consequently, he did not achieve full possession.

That finding, however, does not resolve the case. The reason we do not know
whether Mr. Popov would have retained control of the ball is not because of
incidental contact. It is because he was attacked. His efforts to establish
possession were interrupted by the collective assault of a band of
wrongdoers.\readingfootnote{34}{Professor Gray has suggested that the way to
deal with this problem is to demand that Mr. Popov sue the people who assaulted
him. This suggestion is unworkable for a number of reasons. First, it was an
attack by a large group of people. It is impossible to separate out the people
who were acting unlawfully from the people who were involuntarily pulled into
the mix. Second, in order to prove damages related to the loss of the ball, Mr.
Popov would have to prove that but for the actions of the crowd he would have
achieved possession of the ball. As noted earlier, this is impossible.}

A decision which ignored that fact would endorse the actions of the crowd by not
repudiating them. Judicial rulings, particularly in cases that receive media
attention, affect the way people conduct themselves. This case demands
vindication of an important principle. We are a nation governed by law, not by
brute force. 

As a matter of fundamental fairness, Mr. Popov should have had the opportunity
to try to complete his catch unimpeded by unlawful activity. To hold otherwise
would be to allow the result in this case to be dictated by violence. That will
not happen.

\ldots The legal question presented at this point is whether an action for
conversion can proceed where the plaintiff has failed to establish possession or
title. It can[.] An action for conversion may be brought where the plaintiff has
title, possession or the right to possession. 

\ldots Consistent with this principle, the court adopts the following rule.
Where an actor undertakes significant but incomplete steps to achieve possession
of a piece of abandoned personal property and the effort is interrupted by the
unlawful acts of others, the actor has a legally cognizable pre-possessory
interest in the property. That pre-possessory interest constitutes a qualified
right to possession which can support a cause of action for conversion.

\ldots Recognition of a legally protected pre-possessory interest, vests Mr.
Popov with a qualified right to possession and enables him to advance a
legitimate claim to the baseball based on a conversion theory. Moreover it
addresses the harm done by the unlawful actions of the crowd.

It does not, however, address the interests of Mr. Hayashi. The court is
required to balance the interests of all parties.

Mr. Hayashi was not a wrongdoer. He was a victim of the same bandits that
attacked Mr. Popov.\ldots Mr. Hayashi appears on the surface to have done
everything necessary to claim full possession of the ball, [but] the ball itself
is encumbered by the qualified pre-possessory interest of Mr. Popov. At the time
Mr. Hayashi came into possession of the ball, it had, in effect, a cloud on its
title.

An award of the ball to Mr. Popov would be unfair to Mr. Hayashi. It would be
premised on the assumption that Mr. Popov would have caught the ball. That
assumption is not supported by the facts. An award of the ball to Mr. Hayashi
would unfairly penalize Mr. Popov. It would be based on the assumption that Mr.
Popov would have dropped the ball. That conclusion is also unsupported by the
facts.

Both men have a superior claim to the ball as against all the world. Each man
has a claim of equal dignity as to the other. We are, therefore, left with
something of a dilemma.

Thankfully, there is a middle ground.

\ldots The concept of equitable division has its roots in ancient Roman law. As
Helmholz points out, it is useful in that it ``provides an equitable way to
resolve competing claims which are equally strong.'' Moreover, ``[i]t comports
with what one instinctively feels to be fair''. 

\ldots The principle at work here is that where more than one party has a valid
claim to a single piece of property, the court will recognize an undivided
interest in the property in proportion to the strength of the claim.

\ldots Mr. Hayashi's claim is compromised by Mr. Popov's pre-possessory
interest. Mr. Popov cannot demonstrate full control.\ldots Their legal claims
are of equal quality and they are equally entitled to the ball.

\ldots The court therefore declares that both plaintiff and defendant have an
equal and undivided interest in the ball. Plaintiff's cause of action for
conversion is sustained only as to his equal and undivided interest. In order to
effectuate this ruling, the ball must be sold and the proceeds divided equally
between the parties\ldots .

