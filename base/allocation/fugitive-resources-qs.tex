
\expected{tyler-v-wilkinson}
\expected{coffin-v-left-hand}

\item \textbf{Different Strokes for Different Folks.} Why is the rule for
control and use of surface waters different in the Eastern United States than it
is in the West? Why is it different for water in New England than it is for wild
animals in (old) England? Is the ``priority of appropriation'' rule in Colorado
the same as the ``free taking'' rule for game in the early American frontier? If
not, how and why does it differ?

One of the important skills of lawyers (and legal scholars) is to identify
\textit{distinctions} among seemingly analogous fact patterns that could account
for courts' selection of the rules they apply to those facts. So: can we
identify some distinctions in the facts of these two cases that might account
for the difference between, say, the eastern (riparian) rule and the western
(priority of appropriation) rule for water? (Did Justice Helm identify any such
distinctions in \textit{Coffin}?)

We might examine at least three different grounds for distinguishing these types
of cases from one another. First, the characteristics of the \textit{resource
itself} may be different. That may be a relevant basis for distinguishing wild
animals from water; as we will see it may also be a basis for distinguishing
both of those resources from oil and gas. Second, the characteristics of the
\textit{society} in which the resource is being exploited may be different. As
we have already noted, the interior of the American continent in the 18th
century was a very different place than the English countryside---in terms of
its population density and in terms of the level of development and exploitation
of existing natural resources. And as the \textit{Coffin} court noted, the
quality and distribution of arable soil in the mountain west makes irrigation an
``imperative necessity'' to agriculture in a way ``unknown to'' the riparian
east. Third, the particular uses of the resource may differ from one social
context to another. For example, in New England, where surface water is
plentiful, streams were mainly used \textit{non-consumptively} to power
industrial plants in the 19th century; in Colorado, where water is scarce,
streams were used primarily for consumptive purposes---mining, farming, and
drinking. \textit{See} Carol M. Rose, \textit{Energy And Efficiency in the
Realignment of Common-Law Water Rights}, 19 \textsc{J. Leg. Stud.} 261, 290-93
(1990). Any of these types of distinctions could justify a change in legal rules
from one case to another. Which---if any---do you think best explain the
difference between \textit{Tyler} and \textit{Coffin}?


\item \textbf{Stock Resources.} \textit{Tyler} and \textit{Coffin} deal with
allocation of the right to a share of the flow of a natural watercourse. But
much water use depends not on surface waters, but on groundwater, extracted by
means of wells and pumps. Such groundwater can behave more like a stock resource
than a flow resource; excessive extraction by any one claimant \textit{today}
threatens the availability of the resource for \textit{all} claimants \textit{in
the future}. Indeed, extraction of groundwater---and even collection of
precipitation---can alter the flows of surface channels, threatening the rights
of remote riparians or prior appropriators. For this reason, some
states---particularly in the more arid Western United States---have enacted
comprehensive statutory codes and administrative regulations allocating water
rights. California's system is among the most complex, layering early common-law
riparian rights with later common-law prior appropriation rights and a
subsequent statutory code administered by a powerful administrative agency with
significant discretion to alter and limit water uses to respond to changing
conditions. The state's regulatory reach is profound; in May of 2015 the Water
Board responded to serious drought conditions by adopting emergency regulations
requiring residents to refrain from most outdoor uses of water and requiring
businesses to reduce their potable water usage by 25\%, all on pain of a fine of
\$500 per day. \textsc{State Water Resources Control Bd. Res. No. 2015-0032: To
Adopt an Emergency Regulation for Statewide Water Conservation} (May 5, 2015),
\url{http://www.waterboards.ca.gov/waterrights/water_issues/programs/drought/docs/emergency_regulations/rs2015_0032_with_adopted_regs.pdf}.

\item \textbf{Non-Renewable Fugitive Resources.} For our next category of
fugitive resource---oil and gas---stock depletion is the standard state of
affairs, exacerbated by the fact that oil stocks do not replenish themselves the
way water stocks do. As you read, consider how this characteristic of fossil
fuels affect the justifications for allocating them to one claimant or another.

\expectnext{briggs-v-southwestern}

