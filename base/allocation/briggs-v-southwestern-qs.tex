\expected{briggs-v-southwestern}

\item \textbf{Questions of Fact; Questions of Law.} Do Chief Justice Saylor and
Justice Dougherty disagree on the content of the legal rules in Pennsylvania
regarding the ownership of oil and gas? Do they disagree on the law of trespass
as it applies to mineral extraction? If the answer to both these questions is
no, what is their disagreement about?

\hyphenation{briggs-es}

In considering these questions, ask yourself what \textit{actually happened} to
the Briggses and their land in this case. Are you confident you can answer that
question? If not, it may be difficult to say whether they should prevail on
their trespass or conversion claims. This is not because the legal rule is
unclear; rather it is because it may be unclear whether the rule is satisfied
\textit{given the facts in the record}. This distinction between \textit{legal}
issues and \textit{factual} issues is central to the practice of law, and you
will surely learn more about it in your civil procedure class. How does the
court's resolution of the \textit{legal} issues in the case affect the
\textit{factual questions} that the parties must answer in litigation? How
should they go about answering those questions? What is likely to happen to the
Briggses' claim on remand, and what would have happened if Justice Dougherty's
opinion had instead carried a majority of the court? (Hint: The answer to this
last question has less to do with the law of property and more to do with the
law of civil procedure.)


\item \textbf{I Drink Your Milkshake.}\footnote{\textsc{There Will Be Blood}
(Paramount Vantage/Miramax Films 2007).} \textit{Briggs} reaffirms a principle
of long standing in oil and gas law. Imagine Alice and Bob are neighboring
landowners in an oil-rich region. Alice drills an oil well at an angle, such
that the wellhead is on Alice's land, but the bottom of the wellbore, from which
the pipe draws oil, is under Bob's land. Bob sues Alice to enjoin the continued
operation of the well and to recover the value of the oil already extracted.
Under the rule of capture and the definition of trespass as discussed in
\textit{Briggs}, what result and why? \textit{See} 1 \textsc{Summers}
\textsc{Oil and Gas} \S~2:3 (3d ed.) (``[I]f a well deviates from the vertical
and produces oil or gas from under the surface of another landowner, that is a
trespass for which the adjacent owner is entitled to damages, an accounting and
injunction.''). Why might it be acceptable to use a well on your land to draw
the oil from under your neighbor's land, but not to drill the bottom of your
well under the surface owned by your neighbor to extract the very same oil? Does
the distinction have any practical effect? Does the advent of fracking
technology change your answer?



