We have studied a fair number of cases about property rights in wild animals. By
now you may be asking yourself: who cares? This is, after all, an area of legal
doctrine that you will almost certainly never encounter in your future career as
a lawyer. Are we wasting your time?

Obviously we don't think so. We would offer two related reasons for studying
this area of law: 

First: The study of these cases has introduced you to some accessible
illustrations of how we might \textit{justify} rules for allocating control over
scarce resources among competing claimants. We have already seen several
justifications for the rules we have studied---moral reasons, economic reasons,
reasons grounded in administrability and in other pragmatic concerns. (See
note \ref{pierson-justify}, page \pageref{pierson-justify}.) These are the types
of
justifications that move courts and policymakers, and they are the kinds of
justifications that lawyers must invoke in crafting legal arguments and
explaining legal rules to their clients. 

Second: It is sometimes said that ``the law is a seamless web.'' One influential
interpretation of this principle, offered by legal philosopher Ronald Dworkin,
is that common-law judges must attempt to decide cases by reference to ``a
scheme of abstract and concrete principles that provides a coherent
justification for all common law precedent.'' Ronald Dworkin, \textit{Hard
Cases}, 88 \textsc{Harv. L. Rev.} 1057, 1094 (1975). That is to say that a legal
system's rules should not only be \textit{justified} according to discernable
principles, they should be \textit{coherent}: the principles that justify an
outcome in one area of law should apply consistently to other areas of law to
the extent possible. Thus, when deciding novel cases, common-law judges will
have to infer what principles are consistent with---or \textit{fit}---the entire
corpus of cases that have been decided before, decide which among those
principles best \textit{justifies} the cases, and use that principle as a guide
in deciding the novel case.

\expected{popov-v-hayashi}
\expected{keeble-v-hickeringill}

At a more practical level, lawyers typically reason about novel cases by
\textit{analogy} to past cases in the same general doctrinal field. The
common-law treatment of precedent, discussed above on page
\pageref{popov-precedent}, note \ref{popov-precedent}, is a special case of
this more general principle. Thus, even though we don't see many cases involving
disputes over wild animals anymore, past judicial resolutions of those disputes
will inform how we decide disputes over other resources that are similar in some
way. We have seen this type of reasoning by analogy already, in \textit{Popov v.
Hayashi}: a baseball is not a wild animal, but Judge McCarthy thought cases
about wild animals provided instruction for the dispute before him. (Query: Why
might he have thought so?) With respect to the intersection of land and
chattels, we can similarly see \textit{Keeble} and the doctrines of
\textit{ratione soli} and free taking as reflecting principles applicable to
\textit{fugitive resources}: chattels that can move of their own accord from
place to place, sometimes taking them onto owned land. There are plenty of
valuable resources that share this quality, and many of them are the subject of
heated legal disputes even today. We will focus here on two: water and oil.

Water is essential to life, but it can also be put to a variety of other
practical uses: irrigating farmland, extracting minerals from mines and oil or
gas from wells, powering dams and mills, cooling industrial equipment, and as an
input to manufacturing, for example. Fresh water from rainfall and snowmelt may
flow over the surface of land, either free-flowing (particularly during heavy
rains or spring thaws) or in defined channels as streams and lakes. Rain and
snowmelt can also seep down and be absorbed by the earth as subsurface
groundwater or deep aquifers. In either case, water has a fundamental physical
connection to land, but it also moves freely over, under, and across land.
(Sound familiar?)

Both surface and subsurface waters are renewable; they are replenished by
precipitation. But they're still scarce. This scarcity comes in two basic forms,
which map to the economic categories of \textbf{stocks} and \textbf{flows}.
Depletion of a groundwater source at a rate exceeding its natural replenishment
will eventually exhaust the \textbf{stock}---or finite total
\textit{amount}---of water at that source. A stream \textbf{flows} at a
particular (though perhaps variable) \textit{rate}, but that rate is primarily
determined by ecological rather than human processes, so adding more users or
more intense uses may not threaten \textit{future} flows but does reduce the
share of the flow available to each at any given time. Given these forms of
scarcity, competition over water resources is inevitable, and property law may
be called on to regulate that competition.

Complicating the matter, the rate of renewal of water stocks and the magnitude
of water flows vary from time to time and place to place: Hawaii gets a lot more
rain than Nevada, and California got a lot more rain in 1983 than it did in
2013. Reflecting this natural diversity, the American states have devised two
broad categories of common-law responses to the challenge of managing conflicts
over access to water, epitomized by the two cases below. The first response,
\textbf{riparian rights}, dominates in the wetter, eastern states, and was
firmly established by our first case, \textit{Tyler v. Wilkinson}. The second
response, \textbf{prior appropriation}, prevails in the more arid western
states, and is sometimes referred to as the ``Colorado Rule'' given its historic
association with our second case, \textit{Coffin v. Left Hand Ditch Co}. Both
cases deal with rights to flows, in particular the flow of a river. As you read
these cases, try to understand how the two systems differ, and what might
explain or justify the difference.

\expectnext{tyler-v-wilkinson}
\expecting{coffin-v-left-hand}
