\item \textbf{Incentives Again.} Given that any landowner can lawfully extract
all the oil and gas under not only her land, but potentially under the land of
any neighboring landowners who occupy the surface over the same geologic
formation, what incentive does each landowner over a large formation have with
respect to that underlying oil and gas? In early-20th-century California, we
found out.

\captionedgraphic{allocation-img011}{Signal Hill, California, c. 1923. Source:
U.S. Library of Congress PPOC,
\protect\url{http://www.loc.gov/pictures/resource/pan.6a17401/}.}

Figure~\ref{f:allocation-img011} is an image of Signal Hill, California, one of
the richest oil fields ever
discovered, around the peak of its productivity in 1923. Why do you think there
are so many oil derricks in such close proximity to each other? Do you think
this quantity and density of wells are necessary to extract the oil underground?
If not, isn't this duplication of investment and effort \textit{wasteful?}
Couldn't the oil be just as easily extracted with one (or at least far fewer)
wells? If so, why did the people of Signal Hill build so many? Could property
law be playing a role?


\item \textbf{The Tragedy of the Commons.} The race to drill in Signal Hill
evokes one of the key set-pieces invoked by economists to justify private
property rights: the \term{tragedy of the commons}, famously described in an
essay of the same name:
\begin{quote}
Picture a pasture open to all. It is to be expected that each herdsman will
try to keep as many cattle as possible on the commons.\ldots As a rational
being, each herdsman seeks to maximize his gain. Explicitly or implicitly, more
or less consciously, he asks, ``What is the utility to me of adding one more
animal to my herd?''\ldots [T]he herdsman receives all the proceeds from the
sale of the additional animal\ldots . Since, however, the effects of overgrazing
are shared by all the herdsmen,\ldots any particular decision-making herdsman
[bears] only a fraction of [the negative effects of his additional
animal].\ldots [T]he rational herdsman concludes that the only sensible course
for him to pursue is to add another animal to his herd. And another; and
another\ldots. But this is the conclusion reached by each and every rational
herdsman sharing a commons. Therein is the tragedy. Each man is locked into a
system that compels him to increase his herd without limit---in a world that is
limited.
\end{quote}
Garrett Hardin, \textit{The Tragedy of the Commons}, 162 \textsc{Science} 1243
(1968). 

The negative effects of each additional animal, which are suffered by all the
common owners collectively, are what economists refer to as an
\term{externality}. Some of the things we do with the resources we control can
make \textit{others} better or worse off. If I divert a stream to my mine, your
crops may wither; if I plant a rosebush in my garden, you may enjoy the smell of
my flowers on your way to work each day. The key point to keep in mind about
these externalities caused by my conduct is that \textit{I care about them less
than you do}. I am better off if the stream I diverted makes my mine more
productive; the fact that the diversion causes your crops to die doesn't affect
me directly, or perhaps at all.

Externalities can lead to the kind of misallocation of investment and effort we
see in Signal Hill or the overcrowded pasture: in deciding whether to engage in
an activity, I am unlikely to take sufficient account of the effects of my
activity on others. This, in turn, can lead to bad \textit{aggregate} outcomes:
I may impose large costs on all my neighbors by engaging in an activity that is
of only moderate benefit to me, or I may refrain from an activity that would
confer large benefits on many people at only moderate cost to myself. The
trouble is that I have no \textit{incentive} to weigh the cost of your dying
crops, your starving animals, or your dried-up well.

The economist's solution to this problem is to \textit{internalize the
externalities} that result from resource use. That is, to find some way to make
the effects of a person's actions hit that person in the pocketbook, for good or
for ill. One way to internalize the externalities that generate the tragedy of
the commons is to convert the commons to private ownership. Knowing that
pasturing too many animals today would leave nothing for his animals to eat
tomorrow, a rational \textit{owner} of the pasture would calibrate the number of
animals he keeps to maximize their number today while ensuring a stable supply
of fodder into the future. Indeed, Professor Harold Demsetz famously argued that
property rights arise precisely when the benefits of exploiting a scarce
resource have increased in value (due to increasing demand or decreasing supply)
to the point where the right to control that value would be a sufficient
incentive to undertake the costs of responsibly managing the resource (i.e.,
where an owner would be willing to internalize the externalities of using the
resource). \textit{See} Harold Demsetz, \textit{Toward a Theory of Property
Rights}, 57 \textsc{Am. Econ. Rev}. 347 (1967).

So goes the theory, at any rate. But this theory leaves open a host of practical
questions, primarily about \textit{allocation} of these theoretically attractive
private property rights. Does it make the most sense to have one owner of the
whole pasture? Should the pasture be divided into parcels, and if so, how many
and how should they be assigned? What if dividing the pasture into smaller
parcels leaves each owner with insufficient space to pasture animals? If there
is just one owner, how are we supposed to choose the lucky winner? And once the
winner is chosen, what is everyone else supposed to do? Finally, who has the
authority to decide all these questions? 

We can apply these questions to our oil and gas example. If you were trying to
avoid overexploitation of the oil field at Signal Hill in 1923, would you assign
private property rights over the entire oil field? How? To whom? Is there an
alternative to private property rights that can avoid inefficient
overexploitation? Might the experience of other societies whose territory
includes valuable fossil fuel reserves be instructive? \textit{See} Helge
Ryggvik, \textit{A Short History of the Norwegian Oil Industry: From Protected
National Champions to Internationally Competitive Multinationals}, 89
\textsc{Bus. Hist. Rev.} 3 (2015).


\item \textbf{Hardin's Problematic Legacy.} Garrett Hardin's metaphor of the
overburdened pasture was one piece of a broader worldview expressed in his
writings that strikes many today as deeply problematic. Like many
mid-20th{}-century residents of rich countries, Hardin was concerned about a
supposed ``population bomb'': a postwar trend of higher population growth in
poorer countries relative to richer countries. Some predicted that this
population growth would generate levels of consumption that would overburden the
earth's resources (particularly its capacity to produce food), leading to
exhaustion of those resources and widespread pollution, famine, and poverty.

Hardin's reaction to those predictions was to see developing nations as
adversaries in a global competition for resources, and to urge national and even
ethnic solidarity within rich countries to resist the developing world's demands
for access to those resources. Though few read the full essay today, \textit{The
Tragedy of the Commons} is ultimately an argument in favor of compulsory
restraints on procreation. Its final sections equate ``breeding'' with bank
robbery, and conclude: ``The only way we can preserve and nurture other and more
precious freedoms is by relinquishing the freedom to breed, and that very
soon.'' Hardin, \textit{supra}, at 1248. Hardin thought rich countries should
refuse to grant foreign aid, limit immigration from poor countries, impose
compulsory measures to reduce fertility rates, and harden their hearts against
any moral pangs arising from the resulting suffering of the world's
poor---policies that went hand-in-hand with his view of resource competition as
the struggle of rich societies against poor societies for survival. In his own
words:
\begin{quote}
Metaphorically each rich nation can be seen as a lifeboat full of comparatively
rich people. In the ocean outside each lifeboat swim the poor of the world, who
would like to get in, or at least to share some of the wealth. What should the
lifeboat passengers do?\ldots Suppose we decide to preserve our small safety
factor and admit no more to the lifeboat. Our survival is then possible although
we shall have to be constantly on guard against boarding parties.
\end{quote}
Garrett Hardin, \textit{Lifeboat Ethics},
\textsc{Psychology Today} (Sept. 1974), \url{https://perma.cc/MXU5-F436}. 

Today, many critics note that Hardin's arguments smack of eugenics and
imperialism. In his non-academic writings, Hardin was outspoken in his
opposition to ethnic diversity and his support of restricting non-European
immigration to the United States, and the Southern Poverty Law Center identifies
him as a white nationalist extremist. Southern Poverty Law Center,
\textit{Extremist Files: Garrett Hardin},
\url{https://perma.cc/S4J5-EL5J}. One critic rejects Hardin's argument about the
tragedy of the commons as a product of his chauvinist politics: ``[R]acist,
eugenicist, nativist and Islamophobe\ldots [h]is writings and political activism
helped inspire the anti-immigrant hatred spilling across America today\ldots .
Hardin wasn't making an informed scientific case. Instead, he was using concerns
about environmental scarcity to justify racial discrimination.'' Matto
Mildenberger, \textit{The Tragedy
of} The Tragedy of the Commons, \textsc{Scientific American: Voices} (April 23,
2019), \url{https://perma.cc/WJ49-H463?type=image}. 

Does the fact that Hardin held deplorable social and political views detract
from the force of his arguments about resource management? Your answer may
depend on whether you believe the two are related---whether his solutions to the
problem of stewarding the Earth's scarce resources were really just a means to
the particular (and contestable) ends contemplated by his political views. There
is a plausible argument that they were: that his theoretical model of
overconsumption in a commons is an abstraction of his concern that growing
resource consumption by developing Latin American, Asian, and African societies
posed a threat to the ability of rich European and North American societies to
maintain the far higher per capita levels of consumption they enjoy. In this
view, Hardin's proposed solution---giving some privileged consumers the power to
exclude others---seems conveniently designed to justify rich countries'
privileged consumption levels. The very term ``population bomb,'' popularized in
a bestselling book published in the same year as \textit{The Tragedy of the
Commons} (\textsc{Paul R. Ehrlich, The Population Bomb} (1968)), reflects a view
of the developing world as a deadly threat, and implies that the solution lies,
not in reduced consumption by rich countries, or in reallocation of resources
more generally, but in limiting the number of competitors for scarce resources. 

This view has had serious world-historical consequences. Over the second half of
the 20th century, population control was enthusiastically promoted by Western
countries, by philanthropic organizations such as the Rockefeller Foundation and
the Ford Foundation, and by the United Nations. The governments of developing
countries such as India and China---often with the support and financial
encouragement of Western-led institutions such as the World Bank---implemented
decades-long programs of incentivized or compulsory sterilization and
abortion---with mixed results, and at great cost. \textit{See generally}
\textsc{Matthew Connelly, Fatal Misconception: The Struggle to Control World
Population} (2008).

But as it turned out, Hardin and the other doomsayers were wrong in their
predictions of global famine and resource collapse. Technological advances in
food production and pollution control, as well as social and political changes
such as conservation programs, democratization, and reductions in armed
conflict, ultimately put the lie to many of their direst predictions. Food
insecurity and extreme poverty have steadily \textit{declined} worldwide since
the 1960s. Population growth rates have also steadily declined worldwide,
notably in inverse correlation with increases in income and in women's
educational attainment. But even today, similar fears and analogous political
concerns pervade debates over problems of great importance---particularly
climate change---in which resource allocation and stewardship play a crucial
role.


\item \textbf{The Comedy of the Commons.} Whether or not one finds Hardin's
arguments morally repugnant, his analyses have also been criticized as bad
social science. It turns out that the free-for-all common pasture of Hardin's
essay lacks a historical antecedent: medieval English commons were actually a
form of community resource management based on ancient rules and customs that
served to preserve the commons for future generations. \textit{See} Susan Jane
Buck Cox, \textit{No Tragedy on the Commons}, 7 \textsc{Envtl. Ethics} 49
(1985). And such community management arrangements are not unusual. 

Some of the most groundbreaking work in economics in the past
half-century---such as the Nobel Prize-winning work of Dr. Elinor Ostrom---has
demonstrated how community resource management actually works surprisingly well
in contexts as diverse as Swiss mountain farms, Filipino irrigation canals, and
Turkish fisheries. \textit{See generally} \textsc{Elinor Ostrom, Governing the
Commons} (1990). Indeed, some resources---infrastructure such as roads and
waterways, recreational facilities such as parks and beaches, and social spaces
such as public squares---may have characteristics of a ``comic'' commons in that
the more people use them, the more valuable they become (at least within a
finite community). \textit{See generally, e.g.}, Carol Rose, \textit{The Comedy
of the Commons: Commerce, Custom, and Inherently Public Property}, 53 \textsc{U.
Chi. L. Rev.} 711 (1986). \having{intro-public-use-rights}{We discussed these
ideas with respect to the public trust doctrine in our unit on Easements.}{We
will revisit these ideas when we encounter the public trust doctrine in our unit
on Easements.}{}

Given the practical problems of allocation raised by efforts to privatize
resources, and the availability of alternative management schemes for at least
some such resources, we might well question whether the absence of property
rights over scarce resources necessarily results in tragedy. In any case, we
ought to be skeptical of the argument that the tragedy of the commons must
affect all resources, in all societies, at all times.

