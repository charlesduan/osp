\item Are the doctrines we have studied regarding allocation of fugitive
resources property-based or commons-based? Take, for example, the riparian
doctrine of reasonable use: can riparian owners take as much of the waters
flowing past their land as they want, whenever they wish? Is there any middle
ground between the ``sole and despotic dominion'' of Blackstone's private
property and the tragic spiraling waste of Hardin's unregulated pasture? If so,
how does the law decide who gets what?

\expected{coffin-v-left-hand}

What about the prior appropriation rule governing water rights in western
states? Is it an instance of law stepping in to prevent a tragedy of the
commons? That is certainly one conventional interpretation of the rule. But
Professor David Schorr recently argued that early settlers in Colorado had
informally worked out relatively egalitarian water allocation arrangements,
which the \textit{Coffin} court was merely protecting against destabilizing
intrusions by new arrivals and powerful corporate interests. \textit{See
generally} \textsc{David Schorr, The Colorado Doctrine} (2012). Which makes more
sense to you: that the \textit{Coffin} court was setting economic policy to
avoid overuse of scarce water, or that it was protecting the past investments
and future expectations of the state's most established citizens? If you were a
newly arrived farmer in Colorado when \textit{Coffin} was announced, how would
you react to the opinion?


