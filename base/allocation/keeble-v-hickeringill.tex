\reading{Keeble v. Hickeringill}

\readingcite{(1707) 103 Eng. Rep. 1127, 11 East 574 (Q.B.)}

\begin{figure}
\begin{center}
\usegraphic[width=0.3\textwidth]{allocation-img008}
\usegraphic[width=0.3\textwidth]{allocation-img009}
\usegraphic[width=0.3\textwidth]{allocation-img010}
\end{center}
\caption{Source: \textsc{Ralph Payne-Gallwey, The Book of Duck Decoys} 34, 36,
166 (1886),
\protect\url{https://archive.org/details/bookofduckdecoysx00payn}.}
\end{figure}

Action upon the case. Plaintiff declares that he was, 8th November in the second
year of the Queen, lawfully possessed of a close of land called Minott's Meadow,
[where he maintained] a decoy pond, to which divers wildfowl used to resort and
come: and the plaintiff had at his own costs and charges prepared and procured
divers decoy ducks, nets, machines and other engines for the decoying and taking
of the wildfowl, and enjoyed the benefit in taking them: the defendant, knowing
which, and intending to damnify the plaintiff in his vivary, and to fright and
drive away the wildfowl used to resort thither, and deprive him of his profit,
did, on the 8th of November, resort to the head of the said pond and vivary, and
did discharge six guns laden with gunpowder, and with the noise and stink of the
gunpowder did drive away the wildfowl then being in the pond: and on the 11th
and 12th days of November the defendant, with design to damnify the plaintiff,
and fright away the wildfowl, did place himself with a gun near the vivary, and
there did discharge the said gun several times that was then charged with the
gunpowder against the said decoy pond, whereby the wildfowl were frighted away,
and did forsake the said pond. Upon not guilty pleaded, a verdict was found for
the plaintiff and 20\textit{l.} damages. 

\opinion \textsc{Holt C.J.} 

I am of opinion that this action doth lie. It seems to be new in its instance,
but is not new in the reason or principle of it. For, 1st, this using or making
a decoy is lawful. 2dly, this employment of his ground to that use is profitable
to the plaintiff, as is the skill and management of that employment. As to the
first, every man that hath a property may employ it for his pleasure and profit,
as for alluring and procuring decoy ducks to come to his pond.\ldots Then when a
man useth his art or his skill to take them, to sell and dispose of for his
profit; this is his trade; and he that hinders another in his trade or
livelihood is liable to an action for so hindering him.\ldots 

[W]here a violent or malicious act is done to a man's occupation, profession, or
way of getting a livelihood; there an action lies in all cases. But if a man
doth him damage by using the same employment; as if Mr. Hickeringill had set up
another decoy on his own ground near the plaintiff's, and that had spoiled the
custom of the plaintiff, no action would lie, because he had as much liberty to
make and use a decoy as the plaintiff. This is like the case of 11 H. 4, 47. One
schoolmaster sets up a new school to the damage of an antient school, and
thereby the scholars are allured from the old school to come to his new. (The
action was held there not to lie.) But suppose Mr. Hickeringill should lie in
the way with his guns, and fright the boys from going to school, and their
parents would not let them go thither; sure that schoolmaster might have an
action for the loss of his scholars.\ldots

There was an objection that did occur to me, though I do not remember it to be
made at the Bar; which is, that it is not mentioned in the declaration what
number or nature of wildfowl were frighted away by the defendant's
shooting.\ldots Where a man brings trespass for taking his goods, he must
declare of the quantity, because he, by having had the possession, may know what
he had, and therefore must know what he lost.\ldots The plaintiff in this case
brings his action for the apparent injury done him in the use of that employment
of his freehold, his art, and skill, that he uses thereby.\ldots And when we do
know that of long time in the kingdom these artificial contrivances of decoy
ponds and decoy ducks have been used for enticing into those ponds wildfowl, in
order to be taken for the profit of the owner of the pond, who is at the expence
of servants, engines, and other management, whereby the markets of the nation
may be furnished; there is great reason to give encouragement thereunto; that
the people who are so instrumental by their skill and industry so to furnish the
markets should reap the benefit and have their action. But, in short, that which
is the true reason is that this action is not brought to recover damage for the
loss of the fowl, but for the disturbance. 

