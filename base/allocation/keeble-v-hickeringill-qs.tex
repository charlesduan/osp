\expected{keeble-v-hickeringill}
\expected{pierson-v-post}

\item What was Keeble suing Hickeringill for, and why did he prevail? Was his
claim a property claim? (A related question: what is an ``[a]ction on the
case''? Did you look it up?) If a property claim, what was the
\textit{res}---the thing that Keeble claimed as his property? If not a property
claim, what might this case be doing in your Property casebook? 
\item Whether \textit{Keeble} is a property case or not, where did the
20{\pounds} damages measure come from?
\item You may recall that \textit{Keeble} was discussed by Justice Tompkins in
\textit{Pierson v. Post}, though not by name. (See page
\pageref{bkm:Ref302327184} note \ref{bkm:Ref302327184}, \textit{supra}.) Justice
Tompkins referred to different \textbf{reports} of the case than the one you
read. The existence of
multiple, sometimes conflicting, reports is fairly common for earlier English
cases and even for some early American cases. In earlier days, judges would
announce their opinions from the bench, and \textbf{reporters}---usually
entrepreneurial lawyers---would take notes of these opinions (often along with
the arguments of counsel), collect them, and publish them as a reference for the
bar. These days judges issue written opinions, which are collected and published
in ``official'' reporters as written. But for earlier cases, the content of a
precedential authority depended on the transcription of the reporter, and
reporters could be unreliable. The Modern King's Bench (``Mod.'') and Salkeld
(``Salk.'') reports cited by Justice Tompkins are today believed to be less
reliable than the East report you just read, which the reporter claimed to have
based on a copy of Lord Chief Justice Holt's own manuscript. Unfortunately for
Justice Tompkins, the East report of \textit{Keeble} was not published until
1815 (ten years after \textit{Pierson}). Had this report been available to the
New York Supreme Court in 1805, do you think \textit{Pierson} would have come
out differently?

