\reading{Tyler v. Wilkinson}

\readingcite{24 F.Cas. 472, 4 Mason 397 (D. R.I. 1827)}

\opinion \textsc{Story}, Circuit Justice.

[The Pawtucket River forms part of the boundary between Rhode Island and
Massachusetts. Plaintiffs owned several mills on the Massachusetts side of the
river. For over a century, mills on both sides of the river had been powered by
the flow of the Pawtucket as directed by a dam (the ``lower dam''). Defendants
owned several mills upstream of the plaintiffs on the Rhode Island side of the
river and on a man-made canal called Sergeant's Trench, which bypassed the lower
dam on the western bank. Defendants erected a new dam (the ``upper dam'') to
direct the flow of water toward their mills, interfering with the ability of
plaintiffs to rely on the flow of the Pawtucket to the lower dam to power the
plaintiffs' mills. Plaintiffs sued for a declaration that by ``ancient usage''
they had a superior claim to the waters of the Pawtucket over the defendants,
whom the plaintiffs alleged were entitled only to ``wastewater,'' or so much of
the flow as was not needed by the plaintiffs. Supreme Court Justice Joseph
Story, riding circuit, heard the dispute and rendered the following opinion.]

Before proceeding to an examination of these points, it may be proper to
ascertain the nature and extent of the right, which riparian proprietors
generally possess, to the waters of rivers flowing through their lands\ldots .

Prima facie every proprietor upon each bank of a river is entitled to the land,
covered with water, in front of his bank, to the middle thread of the stream,
or, as it is commonly expressed, \textit{usque ad filum acquae}. In virtue of
this ownership he has a right to the use of the water flowing over it in its
natural current, without diminution or obstruction. But, strictly speaking, he
has no property in the water itself; but a simple use of it, while it passes
along. The consequence of this principle is, that no proprietor has a right to
use the water to the prejudice of another. It is wholly immaterial, whether the
party be a proprietor above or below, in the course of the river; the right
being common to all the proprietors on the river, no one has a right to diminish
the quantity which will, according to the natural current, flow to a proprietor
below, or to throw it back upon a proprietor above. This is the necessary result
of the perfect equality of right among all the proprietors of that, which is
common to all. The natural stream, existing by the bounty of Providence for the
benefit of the land through which it flows, is an incident annexed, by operation
of law, to the land itself. When I speak of this common right, I do not mean to
be understood, as holding the doctrine, that there can be no diminution
whatsoever, and no obstruction or impediment whatsoever, by a riparian
proprietor, in the use of the water as it flows; for that would be to deny any
valuable use of it. There may be, and there must be allowed of that, which is
common to all, a reasonable use. The true test of the principle and extent of
the use is, whether it is to the injury of the other proprietors or not.\ldots
The maxim is applied, ``\textit{Sic utere tuo, ut non alienum laedas}.''

But of a thing, common by nature, there may be an appropriation by general
consent or grant. Mere priority of appropriation of running water, without such
consent or grant, confers no exclusive right. It is not like the case of mere
occupancy, where the first occupant takes by force of his priority of occupancy.
That supposes no ownership already existing, and no right to the use already
acquired. But our law annexes to the riparian proprietors the right to the use
in common, as an incident to the land; and whoever seeks to found an exclusive
use, must establish a rightful appropriation in some manner known and admitted
by the law. Now, this may be, either by a grant from all the proprietors, whose
interest is affected by the particular appropriation, or by a long exclusive
enjoyment, without interruption, which affords a just presumption of right. By
our law, upon principles of public convenience, the term of twenty years of
exclusive uninterrupted enjoyment has been held a conclusive presumption of a
grant or right\ldots .

With these principles in view, the general rights of the plaintiffs cannot admit
of much controversy. They are riparian proprietors, and, as such, are entitled
to the natural flow of the river without diminution to their injury. As owners
of the lower dam, and the mills connected therewith, they have no rights beyond
those of any other persons, who might have appropriated that portion of the
stream to the use of their mills. That is, their rights are to be measured by
the extent of their actual appropriation and use of the water for a period,
which the law deems a conclusive presumption in favor of rights of this nature.
In their character as mill-owners, they have no title to the flow of the stream
beyond the water actually and legally appropriated to the mills; but in their
character as riparian proprietors, they have annexed to their lands the general
flow of the river, so far as it has not been already acquired by some prior and
legally operative appropriation. No doubt, then, can exist as to the right of
the plaintiffs to the surplus of the natural flow of the stream not yet
appropriated. Their rights, as riparian proprietors, are general; and it is
incumbent on the parties, who seek to narrow these rights, to establish by
competent proofs their own title to divert and use the stream.

And this leads me to the consideration of the nature and extent of the rights of
the trench owners. There is no doubt, that in point of law or fact, there may be
a right to water of a very limited nature, and subservient to the more general
right of the riparian proprietors.\ldots But the presumption of an absolute and
controlling power over the whole flow, a continuing power of exclusive
appropriation from time to time, in the riparian proprietor, as his wants or
will may influence his choice, would require the most irresistible facts to
support it. Men who build mills, and invest valuable capital in them, cannot be
presumed, without the most conclusive evidence, to give their deliberate assent
to the acceptance of such ruinous conditions. The general presumption appears to
me to be that which is laid down by Mr. Justice Abbott in \emph{Saunders v.
Newman}, 1 Barn. \& Ald. 258: ``When a mill has been erected upon a stream for a
long period of time, it gives to the owner a right, that the water shall
continue to flow to and from the mill in the manner in which it has been
accustomed to flow during all that time. The owner is not bound to use the water
in the same precise manner, or to apply it to the same mill; if he were, that
would stop all improvements in machinery. If, indeed, the alterations made from
time to time prejudice the right of the lower mill (i.e. by requiring more
water), the case would be different.''

In this view of the matter, the proprietors of Sergeant's trench are entitled to
the use of so much of the water of the river as has been accustomed to flow
through that trench to and from their mills (whether actually used or necessary
for the same mills or not), during the twenty years last before the institution
of this suit, subject only to such qualifications and limitations, as have been
acknowledged or rightfully exercised by the plaintiffs as riparian proprietors,
or as owners of the lower mill-dam, during that period. But here their right
stops; they have no right farther to appropriate any surplus water not already
used by the riparian proprietors, upon the notion, that such water is open to
the first occupiers. That surplus is the inheritance of the riparian
proprietors, and not open to occupancy.

\ldots My opinion accordingly is, that the trench owners have an absolute right
to the quantity of water which has usually flowed therein, without any adverse
right on the plaintiffs to interrupt that flow in dry seasons, when there is a
deficiency of water. But the trench owners have no right to increase that flow;
and whatever may be the mills or uses, to which they may apply it, they are
limited to the accustomed quantity, and may not exceed it\ldots . [I]f there be
a deficiency, it must be borne by all parties, as a common loss, wherever it may
fall, according to existing rights\ldots and that the plaintiffs to this extent
are entitled to have their general right established, and an injunction granted.

It is impracticable for the court to do more, in this posture of the case, than
to refer it to a master to ascertain, as near as may be, and in conformity with
the suggestions in the opinion of the court, the quantity to which the trench
owners are entitled, and to report a suitable mode and arrangement permanently
to regulate and adjust the flow of the water, so as to preserve the rights of
all parties.

\ldots The decree of the court is to be drawn up accordingly; and all further
directions are reserved to the further hearing upon the master's report, \&c.
Decree accordingly.

