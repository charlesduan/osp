
\expected{ghen-v-rich}
\expected{pierson-v-post}

\item \textbf{Primary and Secondary Rules.} Is the rule of \textit{Ghen v. Rich}
different from the rule of \textit{Pierson v. Post}? If so, how? Are the
justifications for the rule, or for the outcome, the same in each case? If not,
how do they differ?

To answer this question, it may be helpful to distinguish between what leading
legal philosopher H.L.A. Hart called \textit{primary rules} and
\textit{secondary rules}. In Hart's account, \textit{primary rules} are those
that prescribe standards of conduct, and set forth consequences for failure to
act accordingly. Statutes defining and setting forth punishments for crimes
provide a straightforward example. \textit{Secondary rules} are basically
everything else, but in particular they include rules that give actors within
the legal system the power to create, alter, or abolish their own primary rules.
For example, contract law is largely a body of secondary rules: parties to a
contract acting within those rules have the power to create legal rights and
obligations that will bind them; the contract itself embodies the applicable
primary rules. (For more on this distinction---and more of Hart's monumental
contributions to jurisprudence---see \textsc{H.L.A. Hart, The Concept of Law}.)

Based on this admittedly limited introduction to the concept, was the
determinative legal rule in \textit{Ghen v. Rich} a primary or a secondary rule?
What about in \textit{Pierson v. Post}?

\item \textbf{Whose Custom?} In \textit{Aberdeen Arctic Co. v. Sutter}, 4 McQ.
H.L. 355 (1862), the House of Lords heard the appeal of a case involving a hired
Eskimo harpooner aboard an English whaling vessel in Cumberland Inlet, a
traditional native fishing ground in what is now Canada. The harpooner, one
Bullygar, struck a whale with a harpoon and line, at the end of which was
attached an inflated sealskin, or ``drog,'' which the native fishermen had a
custom of using to tire the harpooned animal and to make it easier to track
while it swims below the surface. The whale dove immediately, so deep that
Bullygar was forced to release his line, and it did not surface again until it
had traveled several miles. Before Bullygar and his ship could retrieve it,
another ship---the \textit{Alibi}---came upon the wounded whale, killed it, and
took it. Bullygar's captain (Sutter) sued the owners of the \textit{Alibi} for
``compensation and damages'' in the amount of {\pounds}1,200. 

The Law Lords found for the owners of the \textit{Alibi}, recognizing a custom
of English whalers in the shallower waters around Greenland. This custom was
known as ``fast and loose'' (which does not---or did not---mean what you think
it means). According to the ``fast and loose'' rule, the first ship to harpoon a
whale has a right to the animal so long as the ship holds ``fast'' to its line,
even if other ships participate in the ultimate killing and capture of the
whale. But if the whale should break free---even if mortally wounded---or if the
line should be intentionally cut or released---even for reasons of safety or
necessity---the whale becomes ``loose'' and will become the property of the
first ship to actually secure it. (\textit{See} \textsc{Herman Melville,
Moby-Dick} 372-75 (1922) [1892] (``Fast-Fish and Loose-Fish'').)

Sutter argued that Cumberland Inlet had long been governed by the custom of the
Eskimo---which conferred ownership on the first person whose harpoon struck and
remained in the animal with the drog attached---and that the English ``fast and
loose'' rule should not apply. Lord Chancellor Westbury rejected the argument.
He opined that Sutter had the burden of proving that English whaling ships
entering this new fishing ground had agreed \textit{not} to bring the ``fast and
loose'' custom with them. Indeed, he openly doubted whether the drog fishing
methods of the Eskimo---which they used primarily in seal hunting---were even
capable of capturing a whale. Moreover, he suggested that even if the case were
to be decided by the law of ``occupancy'' rather than the custom of English
whalers, the result would be the same.

Is the rule of \textit{Ghen v. Rich} the same as the rule of \textit{Aberdeen
Arctic Co. v. Sutter}? If different, which rule is better and why?


\item \label{bkm:Ref302291448}Imagine you are counsel to either Pierson or Rich,
and your adversary makes you an offer of settlement: to sell the contested
chattel and split the proceeds evenly. What would you advise your client to do?
Consider the following case.
\expectnext{popov-v-hayashi}


