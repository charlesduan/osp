\expected{johnson-v-mcintosh}
\expected{mabo-v-queensland}

\item A system of land ownership founded on violent conquest strikes us as
arbitrary and unjustifiable today. Both cases you read seem to reflect this view
in their rhetoric. But do they implement such a view in their dispositions of
the claims before them? Or do they follow Blackstone's advice to ``obey the laws
when made, without scrutinizing too nicely into the reasons of making them''?

Chief Justice Marshall seems almost embarrassed to confirm the
``extravagant\ldots pretension'' that European discovery and conquest is not
only a legitimate source of land titles in the United States, but the
\textit{only} legitimate source of such titles. But he does so anyway. Why? 

Justice Brennan is even more forceful, finding the European doctrine of
acquisition by discovery ``an unjust and discriminatory doctrine of that kind
can no longer be accepted.'' But is the rule he announces any different than the
rule of \textit{Johnson v. M`Intosh}? If so, how?

Can we think of a better justification for allocating ownership of land? What
allocation rule would result from such a better justification? If we could come
up with a better justified principle for allocating initial ownership of land
than violent conquest, could we simply implement a system based on that
principle tomorrow? If not, what has become of Judge McCarthy's defiant
assertion in \textit{Popov v. Hayashi} that ``[w]e are a nation governed by law,
not by brute force''? Is there something different about land that makes
allocation by ``brute force'' more acceptable?


\item \textbf{Wrong + Time = Right?} Perhaps the distinction between
\textit{Popov v. Hayashi} and \textit{Johnson v. M`Intosh} has to do with how
much time has passed since the violent dispossession of the aggrieved plaintiff.
Does the fact that a thousand years have passed since William the Conqueror make
his expropriation of land from the Anglo-Saxons any less unjust? What about the
five hundred years since European discovery of the Americas? The two hundred
years since the British colonization of Australia? If the United States invaded
a foreign country---say, somewhere in the Middle East---tomorrow, and purported
to sell to an American corporation legal title to land in that country that was
in possession of natives claiming ownership under the laws of the conquered
nation, would you expect the dispossessed natives to have a legal remedy? In
what court?

Note that the major split between the Justices in \textit{Mabo} was not over the
\textit{existence} of native title, but on its scope. Three (of seven) Justices
would have held that ``If common law native title is wrongfully extinguished by
the Crown,\ldots compensatory damages can be recovered provided the proceedings
for recovery are instituted within the period allowed by applicable limitations
provisions,'' and that extinguishment by inconsistent grant in the absence of an
Act of Parliament is wrongful. Opinion of Toohey and Gaudron JJ., {\P} 64-65. We
will consider how the passage of time can affect an owner's ability to assert
their rights in our units on Found and Stolen Property and on Adverse
Possession.


\item \textbf{Historical Injustices and Reparations.}
\label{note:reparations}Should injuries
to persons long dead, inflicted by persons long dead, be remediable? Are the
descendants of the wronged individuals the proper recipients of such a remedy?
Should the descendants of the inflicters of the injury be held liable?

In the United States, these are recurring issues that arise in discussions of
the dispossession and genocide of Native Americans and the enslavement of
kidnapped Africans and their descendants. \textit{See, e.g.}, Ta-Nehisi Coates,
\textit{The Case for Reparations}, \textsc{The Atlantic} (June 2014),
\url{http://www.theatlantic.com/features/archive/2014/05/the-case-for-reparations/361631/}
(citing early American examples of reparations of former slaves, cataloguing the
continued injuries inflicted on African-Americans by the discrimination they
face in American society, and laying out the case for a more comprehensive
reparations program). Reparations are also the subject of serious philosophical,
political, and legal discussion. Consider the following excerpt from Carol Rose,
\textit{The Moral Subject of Property}, 48 \textsc{Wm. \& Mary L. Rev.} 1897,
1906-07 (2006) (footnotes omitted):
\begin{quotation}
Property, as an institution, requires stability in people's expectations about
their own and other people's claims. This is why property law has several
claims-clearing devices that substitute Owner \#2 for Owner \#1 when the claims
of Owner \#1 have not been sufficiently publicized, and when most people think
that Owner \#2 is the true owner even though she is not. Adverse possession is a
classic example of this sort of claims-clearing
device.\having{intro-adverse-possession}{}{\edfootnote{We will discuss
adverse possession in \mref{intro-adverse-possession}.}}{\edfootnote{Adverse
possession is
a doctrine by which a non-owner who possesses another's property can eventually
become its owner.}} Unfortunately, Owner \#2's claims
may have arisen in dubious circumstances or even through force or fraud, and
that fact can undermine confidence in the entire institution. Contemporary
Russia is a case in point, where major capitalist figures are widely regarded as
the beneficiaries of insider favoritism and horrifically shady practices. Should
their great wealth be recognized, simply for the sake of getting on with things
and letting a modern economy unroll? Or would some kind of redistribution
actually lead to greater stability? 

Historic injustices create another source of unease: Palestinians vis-\`a-vis
Israelis, former East European landowners vis-\`a-vis the newcomers under Soviet
rule, numerous indigenous groups vis-\`a-vis the settler societies that
displaced them, descendants of slaves vis-\`a-vis the descendants of
slave-owners. Settling all those scores could be hugely disruptive, and the
passage of time itself makes proposed settlements morally ambiguous, because the
original victims and perpetrators often are no longer on the scene. Why charge A
in favor of B, when neither A nor B were personally involved in the past
injustice? Moreover, settlements could leave open the origins of the displaced
persons' own prior claims, as in the case of former aristocrats' plantations in
East Germany. Just whom did their ancestors displace, far back in the Middle
Ages? And so on back in time. 

The age-old acquisition problem is not very salient to most property regimes,
however, even though it bubbles hotly at the center in some locales. Issues of
this kind usually become peripheral because we basically follow Blackstone's
advice: we forget about the questionable origins of title.\ldots By forgetting
about origins we can keep on acquiring, investing, trading, and generally making
ourselves wealthier. The larger public good of stable claims normally outweighs
the private lapses that were entailed in some of those claims. But not
surprisingly, on occasion the situation is reversed: unjust acquisitions may
seem so gross as to eat away even the middle ground morality that makes property
regimes possible. If you think that all those who succeed are thieves, why not
be a thief yourself? That rhetorical question turns tit-for-tat practitioners
into larcenists. Under such circumstances, public morality---even in quest of
stability for property---could require some kind of restitutionary gesture, or
at least some acknowledgment of past injustice. 
\end{quotation}

For further philosophical treatments of reparations and responsibility for
ancient wrongs, see George Sher, \textit{Transgenerational Compensation}, 33
\textsc{Philos. \& Pub. Aff.} 181 (2005) (attempting to justify reparations);
Christopher W. Morris, \textit{Existential Limits to the Rectification of Past
Wrongs}, 21 \textsc{Am. Philos. Q.} 175 (1984) (casting doubt on the moral
argument for reparations); Eric A. Posner \& Adrian Vermeule,
\textit{Reparations for Slavery and Other Historical Injustices}, 103
\textsc{Colum. L. Rev.} 689 (2003) (addressing both philosophical and legal
issues in reparations programs).


\item Is the United States' dispossession of Native Americans really a
``historical'' injustice? Professor Joseph Singer has long faulted the American
legal system for its continued mistreatment of Native Americans:
\begin{quotation}
[T]itle to land in the United States rests on the forced taking of land from
first possessors---the very opposite of respect for first possession. Conquest
is a mode of original acquisition that we cannot sweep under the rug by
pretending that it accords with any recognizable principle of justice. And
conquest, unfortunately, is where American history starts---as does the title to
almost every parcel of land in the United States. This is a highly inconvenient
(not to say stunningly demoralizing) fact, not least of all to the Indian
nations that continue to inhabit the North American continent\ldots .

Many of us protect ourselves from having to think too deeply about conquest by
distancing ourselves from it.\ldots If we can relegate conquest to the distant
past, we can concentrate instead on the fact that the United States was founded
on respect for property rights. We do not acquire property by conquest today.

This comforting story is misleading at best and false at worst. We cannot
comfort ourselves with the idea that conquest became a thing of the past with
the American Revolution, independence from Great Britain, and the adoption of
the U.S. Constitution.
\end{quotation}
Joseph William Singer, \textit{Original Acquisition of Property: From Conquest
\& Possession to Democracy \& Equal Opportunity}, 86 \textsc{Ind. L.J.} 763,
766-67 (2011) (reproduced with permission of the author). As Professor Singer
explains, \textit{id.} at 767-68, most of the federal government's dispossession
of Native American land occurred during the 19th century. During the early 20th
century---while the Supreme Court was gaining a reputation for striking down
state economic legislation in the name of protecting freedom of contract and
private property (the so-called ``\textit{Lochner} era''\footnote{\emph{Lochner
v. New York}, 198 U.S. 45 (1905).})---the United States forcibly took
two-thirds of the remaining lands of the Indian nations. The Supreme Court held
in 1955 that Alaska natives possessed merely a license to live on the
land---revocable permission from whites to occupy Alaskan territory. As recently
as 2009, the Supreme Court held that the Navajo Nation had no right to sue the
federal government for damages where the Secretary of the Interior was alleged
to have colluded with a mining company to undercompensate the tribe for mining
rights on lands held under ``joint title'' between the Navajo and the United
States (by law, the Secretary must approve any leases of tribal land for mining
purposes). \textit{United States v. Navajo Nation}, 556 U.S. 287 (2009). As
Professor Singer reminds us, the conquest is not over.

