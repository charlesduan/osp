\reading{Pierson v. Post}

\readingcite{3 Cai. R. 175 (N.Y. Sup. Ct. 1805)}

\captionedgraphic{allocation-img001}{Source: \textsc{R.S. Surtees, Hawbuck
Grange} 197 (1885), British Library, \protect\url{https://flic.kr/p/i38jT2}}

THIS was an action of trespass on the case commenced in a justice's court, by
the present defendant against the now plaintiff.

The declaration stated that \textit{Post}, being in possession of certain dogs
and hounds under his command, did, ``upon a certain wild and uninhabited,
unpossessed and waste land, called the beach, find and start one of those
noxious beasts called a fox,'' and whilst there hunting, chasing and pursuing
the same with his dogs and hounds, and when in view thereof, \textit{Pierson},
well knowing the fox was so hunted and pursued, did, in the sight of
\textit{Post}, to prevent his catching the same, kill and carry it off. A
verdict having been rendered for the plaintiff below, the defendant there sued
out a \textit{certiorari}, and now assigned for error, that the declaration and
the matters therein contained were not sufficient in law to maintain an
action\ldots .

\opinion \textsc{Tompkins}, J. delivered the opinion of the court.

This cause comes before us on a return to a \textit{certiorari} directed to one
of the justices of \textit{Queens} county.

The question submitted by the counsel in this cause for our determination is,
whether \textit{Lodowick Post}, by the pursuit with his hounds in the manner
alleged in his declaration, acquired such a right to, or property in, the fox,
as will sustain an action against \textit{Pierson} for killing and taking him
away?

The cause was argued with much ability by the counsel on both sides, and
presents for our decision a novel and nice question. It is admitted that a fox
is an animal \textit{fer{\ae} natur{\ae}}, and that property in such animals is
acquired by occupancy only. These admissions narrow the discussion to the simple
question of what acts amount to occupancy, applied to acquiring right to wild
animals?

If we have recourse to the ancient writers upon general principles of law, the
judgment below is obviously erroneous. \textit{Justinian's Institutes}, lib. 2.
tit. 1. s. 13. and \textit{Fleta}, lib. 3. c. 2. p. 175. adopt the principle,
that pursuit alone vests no property or right in the huntsman; and that even
pursuit, accompanied with wounding, is equally ineffectual for that purpose,
unless the animal be actually taken. The same principle is recognised by
\textit{Bracton}, lib. 2. c. 1. p. 8.

\textit{Puffendorf}, lib. 4. c. 6. s. 2. and 10. defines occupancy of beasts
\textit{fer{\ae} natur{\ae}}, to be the actual corporal possession of them, and
\textit{Bynkershoek} is cited as coinciding in this definition. It is indeed
with hesitation that \textit{Puffendorf} affirms that a wild beast mortally
wounded, or greatly maimed, cannot be fairly intercepted by another, whilst the
pursuit of the person inflicting the wound continues. The foregoing authorities
are decisive to show that mere pursuit gave \textit{Post} no legal right to the
fox, but that he became the property of \textit{Pierson}, who intercepted and
killed him.

It therefore only remains to inquire whether there are any contrary principles,
or authorities, to be found in other books, which ought to induce a different
decision. Most of the cases which have occurred in \textit{England}, relating to
property in wild animals, have either been discussed and decided upon the
principles of their positive statute regulations, or have arisen between the
huntsman and the owner of the land upon which beasts \textit{fer{\ae}
natur{\ae}} have been apprehended; the former claiming them by title of
occupancy, and the latter \textit{ratione soli.} Little satisfactory aid can,
therefore, be derived from the \textit{English} reporters.

\textit{Barbeyrac}, in his notes on \textit{Puffendorf}, does not accede to the
definition of occupancy by the latter, but, on the contrary, affirms, that
actual bodily seizure is not, in all cases, necessary to constitute possession
of wild animals. He does not, however, \textit{describe} the acts which,
according to his ideas, will amount to an appropriation of such animals to
private use, so as to exclude the claims of all other persons, by title of
occupancy, to the same animals; and he is far from averring that pursuit alone
is sufficient for that purpose. To a certain extent, and as far as
\textit{Barbeyrac} appears to me to go, his objections to \textit{Puffendorf's}
definition of occupancy are reasonable and correct. That is to say, that actual
bodily seizure is not indispensable to acquire right to, or possession of, wild
beasts; but that, on the contrary, the mortal wounding of such beasts, by one
not abandoning his pursuit, may, with the utmost propriety, be deemed possession
of him; since, thereby, the pursuer manifests an unequivocal intention of
appropriating the animal to his individual use, has deprived him of his natural
liberty, and brought him within his certain control. So also, encompassing and
securing such animals with nets and toils, or otherwise intercepting them in
such a manner as to deprive them of their natural liberty, and render escape
impossible, may justly be deemed to give possession of them to those persons
who, by their industry and labour, have used such means of apprehending them.
\textit{Barbeyrac} seems to have adopted, and had in view in his notes, the more
accurate opinion of \textit{Grotius}, with respect to occupancy. That celebrated
author, lib. 2. c. 8. s. 3. p. 309. speaking of occupancy, proceeds thus:
``\textit{Requiritur autem corporalis qu{\ae}dam possessio ad dominium
adipiscendum; atque ideo, vulnerasse non sufficit.}{}''\edfootnote{Translation:
``Some bodily possession is required for acquiring ownership; for that reason,
wounding is not enough.''} But in the following section he
explains and qualifies this definition of occupancy: ``\textit{Sed possessio
illa potest non solis manibus, sed instrumentis, ut decipulis, retibus, laqueis
dum duo adsint: primum ut ipsa instrumenta sint in nostra potestate, deinde ut
fera, ita inclusa sit, ut exire inde nequeat.}''\edfootnote{Translation: ``But
that possession can be not only by hand, but by instruments, such as traps,
nets, and snares, where two things are present: first that this instrument
itself be in our control, and then that the wild thing, being enclosed, cannot
exit therefrom.''} This qualification embraces the full
extent of \textit{Barbeyrac's} objection to \textit{Puffendorf's} definition,
and allows as great a latitude to acquiring property by occupancy, as can
reasonably be inferred from the words or ideas expressed by \textit{Barbeyrac}
in his notes. The case now under consideration is one of mere pursuit, and
presents no circumstances or acts which can bring it within the definition of
occupancy by \textit{Puffendorf}, or \textit{Grotius}, or the ideas of
\textit{Barbeyrac} upon that subject.

The case cited from 11 \textit{Mod.}
74--130.\edfootnote{\label{pierson-keeble}This citation, and the
following citation to \textit{Salk.}, both refer to the case of \textit{Keeble
v. Hickeringill}\having{keeble-v-hickeringill}{presented earlier in this
book}{presented later in this book}{, 103 Eng. Rep. 1127, 11 East 574 (Q.B.
1707)}.}
I think clearly distinguishable from the present; inasmuch as there the action
was for maliciously hindering and disturbing the plaintiff in the exercise and
enjoyment of a private franchise; and in the report of the same case, 3
\textit{Salk.} 9. \textit{Holt}, Ch. J. states, that the ducks were in the
plaintiff's decoy pond, and \textit{so in his possession}, from which it is
obvious the court laid much stress in their opinion upon the plaintiff's
possession of the ducks, \textit{ratione soli.}\edfootnote{Translation: ``by
reason of the soil.''}

We are the more readily inclined to confine possession or occupancy of beasts
\textit{fer{\ae} natur{\ae}}, within the limits prescribed by the learned
authors above cited, for the sake of certainty, and preserving peace and order
in society. If the first seeing, starting, or pursuing such animals, without
having so wounded, circumvented or ensnared them, so as to deprive them of their
natural liberty, and subject them to the control of their pursuer, should afford
the basis of actions against others for intercepting and killing them, it would
prove a fertile source of quarrels and litigation.

However uncourteous or unkind the conduct of \textit{Pierson} towards
\textit{Post}, in this instance, may have been, yet his act was productive of no
injury or damage for which a legal remedy can be applied. We are of opinion the
judgment below was erroneous, and ought to be reversed.

\opinion \textsc{Livingston}, J. 

My opinion differs from that of the court.

Of six exceptions, taken to the proceedings below, all are abandoned except the
third, which reduces the controversy to a single question.

Whether a person who, with his own hounds, starts and hunts a fox on waste and
uninhabited ground, and is on the point of seizing his prey, acquires such an
interest in the animal, as to have a right of action against another, who in
view of the huntsman and his dogs in full pursuit, and with knowledge of the
chase, shall kill and carry him away?

This is a knotty point, and should have been submitted to the arbitration of
sportsmen, without poring over \textit{Justinian, Fleta, Bracton, Puffendorf,
Locke, Barbeyrac}, or \textit{Blackstone}, all of whom have been cited; they
would have had no difficulty in coming to a prompt and correct conclusion. In a
court thus constituted, the skin and carcass of poor
\textit{reynard}\edfootnote{Reynard was a clever (and often duplicitous) fox
character who featured in several well-known medieval European folk tales and
literary works. The character's popularity gave rise to the modern French word
for ``fox'': \textit{renard}.} would have been properly
disposed of, and a precedent set, interfering with no usage or custom which the
experience of ages has sanctioned, and which must be so well known to every
votary of \textit{Diana}. But the parties have referred the question to our
judgment, and we must dispose of it as well as we can, from the partial lights
we possess, leaving to a higher tribunal, the correction of any mistake which we
may be so unfortunate as to make. By the pleadings it is admitted that a fox is
a ``wild and noxious beast.'' Both parties have regarded him, as the law of
nations does a pirate, ``\textit{hostem humani
generis},''\edfootnote{Translation: ``enemy of the human
race.''} and although ``\textit{de mortuis nil nisi
bonum},''\edfootnote{Translation: ``Of the dead say nothing but
good.''} be a maxim of our profession, the memory of the
deceased has not been spared. His depredations on farmers and on barn yards,
have not been forgotten; and to put him to death wherever found, is allowed to
be meritorious, and of public benefit. Hence it follows, that our decision
should have in view the greatest possible encouragement to the destruction of an
animal, so cunning and ruthless in his career. But who would keep a pack of
hounds; or what gentleman, at the sound of the horn, and at peep of day, would
mount his steed, and for hours together, ``\textit{sub jove
frigido},''\edfootnote{Translation: ``Under frigid Jove'' (\textit{i.e.}, under
a cold sky).} or a vertical sun, pursue the windings of this wily
quadruped, if, just as night came on, and his stratagems and strength were
nearly exhausted, a saucy intruder, who had not shared in the honours or labours
of the chase, were permitted to come in at the death, and bear away in triumph
the object of pursuit? Whatever \textit{Justinian} may have thought of the
matter, it must be recollected that his code was compiled many hundred years
ago, and it would be very hard indeed, at the distance of so many centuries, not
to have a right to establish a rule for ourselves. In his day, we read of no
order of men who made it a business, in the language of the declaration in this
cause, ``with hounds and dogs to find, start, pursue, hunt, and chase,'' these
animals, and that, too, without any other motive than the preservation of
\textit{Roman} poultry; if this diversion had been then in fashion, the lawyers
who composed his institutes, would have taken care not to pass it by, without
suitable encouragement. If any thing, therefore, in the digests or pandects
shall appear to militate against the defendant in error, who, on this occasion,
was the foxhunter, we have only to say \textit{tempora
mutantur};\edfootnote{Translation: ``times change.'' Part of a well-known Latin
aphorism, \textit{tempora mutantur, nos et mutamur in illis}: ``times change,
and we change with them.''} and if men themselves change with
the times, why should not laws also undergo an alteration?

It may be expected, however, by the learned counsel, that more particular notice
be taken of their authorities. I have examined them all, and feel great
difficulty in determining, whether to acquire dominion over a thing, before in
common, it be sufficient that we barely see it, or know where it is, or wish for
it, or make a declaration of our will respecting it; or whether, in the case of
wild beasts, setting a trap, or lying in wait, or starting, or pursuing, be
enough; or if an actual wounding, or killing, or bodily tact and occupation be
necessary. Writers on general law, who have favoured us with their speculations
on these points, differ on them all; but, great as is the diversity of sentiment
among them, some conclusion must be adopted on the question immediately before
us. After mature deliberation, I embrace that of \textit{Barbeyrac}, as the most
rational, and least liable to objection. If at liberty, we might imitate the
courtesy of a certain emperor, who, to avoid giving offence to the advocates of
any of these different doctrines, adopted a middle course, and by ingenious
distinctions, rendered it difficult to say (as often happens after a fierce and
angry contest) to whom the palm of victory belonged. He ordained, that if a
beast be followed with \textit{large dogs and hounds}, he shall belong to the
hunter, not to the chance occupant; and in like manner, if he be killed or
wounded with a lance or sword; but if chased with \textit{beagles only}, then he
passed to the captor, not to the first pursuer. If slain with a dart, a sling,
or a bow, he fell to the hunter, if still in chase, and not to him who might
afterwards find and seize him.

Now, as we are without any municipal regulations of our own, and the pursuit
here, for aught that appears on the case, being with dogs and hounds of
\textit{imperial stature}, we are at liberty to adopt one of the provisions just
cited, which comports also with the learned conclusion of \textit{Barbeyrac},
that property in animals \textit{fer{\ae} natur{\ae}} may be acquired without
bodily touch or manucaption, provided the pursuer be within reach, or have a
\textit{reasonable} prospect (which certainly existed here) of taking, what he
has \textit{thus} discovered an intention of converting to his own use.

When we reflect also that the interest of our husbandmen, the most useful of men
in any community, will be advanced by the destruction of a beast so pernicious
and incorrigible, we cannot greatly err, in saying, that a pursuit like the
present, through waste and unoccupied lands, and which must inevitably and
speedily have terminated in corporal possession, or bodily \textit{seisin},
confers such a right to the object of it, as to make any one a wrongdoer, who
shall interfere and shoulder the spoil. The \textit{justice's} judgment ought,
therefore, in my opinion, to be affirmed.

Judgment of reversal.

