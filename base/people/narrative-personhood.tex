What are the boundaries of the person?  Can they extend past the physical body? 
Consider this account from Atul Gawande in Being Mortal, involving Keren Brown
Wilson's mother, Jessie, who suffered a devastating stroke at the age of
fifty-five:
\begin{quotation}
The stroke left her permanently paralyzed down one side of her body. She could
no longer walk or stand. She couldn't lift her arm. Her face sagged. Her speech
slurred. Although her intelligence and perception were unaffected, she couldn't
bathe herself, cook a meal, manage the toilet, or do her own laundry---let
alone any kind of paid work{\dots}. There was nowhere for Jessie but a nursing
home.  Wilson arranged for one near where she was in college. It seemed a safe
and friendly place. But Jessie never stopped asking her daughter to ``Take me
home.''

``Get me out of here,'' she said over and over again.
\end{quotation}
Wilson wrote:
\begin{quote}
She wanted a small place with a little kitchen and a bathroom. It would have her
favorite things in it, including her cat, her unfinished projects, her Vicks
VapoRub, a coffee-pot, and cigarettes. There would be people to help her with
the things she couldn't do without help. In the imaginary place, she would be
able to lock her door, control her heat, and have her own furniture. No one
would make her get up, turn off her favorite soaps, or ruin her clothes. Nor
could anyone throw out her ``collection'' of back issues and magazines and
Goodwill treasures, because they were a safety hazard. She could have privacy
whenever she wanted, and no one could make her get dressed, take her medicine,
or go to activities she did not like. She would be Jessie again, a person
living in an apartment instead of a patient in a bed.
\end{quote}
Gawande continued: ``The key word in her mind was \textit{home}. Home is the one
place where your own priorities hold sway. At home, \textit{you} decide how you
spend your time, how you share your space, and how you manage your possessions.
Away from home, you don't.''  In the ``assisted living'' concept Wilson
developed, residents would receive services similar to those provided by
nursing homes. ``But here the care providers understood they were entering
someone else's home, and that changed the power relations fundamentally. The
residents had control over the schedule, the ground rules, the risks they did
and didn't want to take.  If they wanted to stay up all night and sleep all
day, if they wanted to have a gentleman or lady friend stay over, if they
wanted not to take certain medications that made them feel groggy; if they
wanted to eat pizza and M\&M's despite swallowing problems and no teeth and a
doctor who'd said they should eat only pureed glop---well, they could.'' 
Gawande reports that residents in assisted living, rather than being at greater
risk from less supervision, had improved physical and cognitive functioning
compared to similar people in nursing homes, and were less likely to suffer
from major depression.

Also consider the following,
%\href{http://www.citylab.com/cityfixer/2014/08/cities-can-ease-homelessness-with-storage-units/379073/}{\textstyleInternetlink{from
from Kriston Capps:
\begin{quote}
For the homeless, simply being able to store belongings can be transformative.
Storage bins or storage units allow them to safeguard important documents,
especially identification and other paperwork that can be hard or expensive to
replace, as well as sentimental items and keepsakes, which can't be replaced at
all. At the First United Church facility, users tend to check in sleeping
equipment during the morning---things like blankets, sleeping bags, and
pillows---and check them out again at night. This frees people to pursue
medical check-ups, job interviews, and housing appointments during the day:
normal activities that are off limits for anyone who has to protect his or her
things around the clock.
\end{quote}
\emph{See also} Margaret Jane Radin, \emph{Property and Personhood} 34 Stan. L.
Rev. 957
(1982) (arguing that certain kinds of property are so centrally connected to
full personhood that they deserve special legal treatment). Can you identify
property that is part of your personhood in your own life?  Your childhood
home?  A piece of jewelry?  A book?  Radin also argues that some kinds of
emotional relationships with property are negative---property fetishism.  A
popular literary example would be Gollum's relationship with the One Ring in
\textit{Lord of the Rings}: his lust for an object leads him to do great harm
to himself and others.  Law, Radin suggests, should promote healthy connections
with property and not respect unhealthy connections.

Gawande criticizes what he sees as the unnecessary and extreme deprivations of
control over the external world imposed on older people by nursing homes.  What
is the relationship between privacy and property rights?  Can you have privacy
without property?

Other institutions require even more intense deprivations of property as part of
an attempt to control the residents.  Erving Goffman, \emph{Asylums: Essays on
the Social Situation of Mental Patients and Other Inmates} (1961):
\begin{quotation}
Once the inmate is stripped of his possessions, at least some replacements must
be made by the establishment, but these take the form of standard issue,
uniform in character and uniformly distributed. These substitute possessions
are clearly marked as really belonging to the institution and in some cases are
recalled at regular intervals to be, as it were, disinfected of
identifications. {\dots} Failure to provide inmates with individual lockers and
periodic searches and confiscations of accumulated personal property reinforce
property dispossession. Religious orders have appreciated the implications for
self of such separation from belongings. {\dots}

On admission to a total institution, {\dots} the individual is likely to be
stripped of his usual appearance and of the equipment and services by which he
maintains it, thus suffering a personal defacement. Clothing, combs, needle and
thread, cosmetics, towels, soap, shaving sets, bathing facilities---all
these may be taken away or denied him {\dots}.

[T]he institutional issue provided as a substitute for what has been taken away
is typically of a ``coarse'' variety, ill-suited, often old, and the same for
large categories of inmates. {\dots}
\end{quotation}

What effects will removing individual property and replacing it with
institutional property likely have on the inmates?  Involuntary dispossession
is a method the institution uses to create a different person, and a different
kind of person.  But this reshaping can also occur voluntarily, as Goffman
explains with the example of monasteries in the order of St. Benedict:

\begin{quotation}
The Benedictine Rule is explicit: 

For their bedding let a mattress, a blanket, a coverlet, and a pillow suffice.
These beds must be frequently inspected by the Abbot, because of private
property which may be found therein. If anyone be discovered to have what he
has not received from the Abbot, let him be most severely punished. And in
order that this vice of private ownership may be completely rooted out, let all
things that are necessary be supplied by the Abbot: that is, cowl, tunic,
stockings, shoes, girdle, knife, pen, needle, handkerchief, and tablets; so
that all plea of necessity may be taken away. And let the Abbot always consider
that passage in the Acts of the Apostles: ``Distribution was made to each
according as anyone had need.''
\end{quotation}
Why is removing private property important for monks?  

As Goffman points out (and as is implicit in the Benedictine Rule), it is very
difficult for institutions to fight human desires for some sort of possessory
interest.  {}``Patients who had been on a given ward for several months tended
to develop personal territories in the day room, at least to the degree that
some inmates developed favorite sitting or standing places and would make some
effort to dislodge anybody who usurped them.''  Likewise, people usually have
places to put the items they use for personal care (hairbrushes, soap, and the
like), which keep them safe and private. ``Where such private storage places
are not allowed, it is understandable that they will be illicitly developed,''
and Goffman goes on to describe this process in great detail.

When we ``personalize'' our appearance, we regularly use property to do so,
either to serve as decoration (jewelry, makeup) or as tool (for creating
hairstyles, tattoos, etc.).  How much of your own property is devoted to
maintaining your personal appearance?  How many items would you say are
important to your self-presentation?  Is this attachment to the property that
produces your self-presentation merely vanity or object-fetishism, as some
religious orders or political philosophies suggest?  Would you be a different
person without these items?

This is the Rifleman's Creed of the U.S. Marine Corps, a basic part of Marine
Corps training:

\begin{quotation}
This is my rifle. There are many like it, but this one is mine.

My rifle is my best friend. It is my life. I must master it as I must master my
life.

My rifle, without me, is useless. Without my rifle, I am useless. I must fire my
rifle true. I must shoot straighter than my enemy who is trying to kill me. I
must shoot him before he shoots me. I will{\dots}. 

My rifle is human, even as I, because it is my life. Thus, I will learn it as a
brother. I will learn its weaknesses, its strength, its parts, its accessories,
its sights and its barrel. I will keep my rifle clean and ready, even as I am
clean and ready. We will become part of each other.  We will.
\end{quotation}

Is the rifle ``property for personhood''?  If so, who owns it?
