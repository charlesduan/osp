\item James Somerset was an enslaved African man who had been transported from
colonial Virginia to England. Once in England he escaped, but was recaptured
and imprisoned on a ship docked in the Thames, soon to depart for Jamaica.
Somerset petitioned the King's Bench for a writ of \textit{habeas corpus}
challenging his confinement against his will by the ship's captain. In
\textit{Somerset v. Stewart}, 98 Eng. Rep. 499 (1772), Lord Chief Justice
Mansfield, noting that slavery was legal in both the North American colonies
and Jamaica but had never been formally recognized as legal by the English
Parliament, granted the writ, saying:
\begin{quote}
[T]he slave departed and refused to serve; whereupon he was kept, to be sold
abroad. So high an act of dominion must be recognized by the law of the country
where it is used. The power of a master over his slave is of such a nature,
that it is incapable of being introduced on any reasons, moral or political;
but only positive law, which preserves its force long after the reasons,
occasion, and time itself from whence it was created, is erased from memory:
it's so odious, that nothing can be suffered to support it, but positive law.
Whatever inconveniences, therefore, may follow from a decision, I cannot say
this case is allowed or approved by the law of England; and therefore the black
must be discharged.
\end{quote}
The result in \textit{Somerset} is, on some level, the same as in
\textit{Amistad}---both courts order captured and enslaved human beings to be
set free. But the facts that put the question and the justifications for the
result are subtly different in each case. Can you articulate the distinction(s)
between Lord Mansfield's reasoning and Justice Story's? What are the
implications of these distinctions for the law of property in England and
America, respectively, as it applies to property rights in human beings?

\captionedgraphic{amistad}{The Illustrated London News, Sept. 27, Sept. 27,
1856, p. 315. ``Slave auction at Richmond, Virginia,'' 1856. Prints and
Photographs Division, Library of Congress. Reproduction Number LC-USZ62-15398.}

\item Is your body your ``property''? The English philosopher John Locke, who
heavily influenced Blackstone and the Anglo-American legal tradition generally,
seemed to think so. In his \textit{Second Treatise on Government}, Chapter V,
Section 27, Locke wrote: 
\begin{quote}
Though the earth, and all inferior creatures, be common to all men, yet
every man has a property in his own person: this no body has any right to but
himself. The labour of his body, and the work of his hands, we may say, are
properly his.
\end{quote}
What are the implications of the view of the human body as ``property''?  If you
can own your own body, why can't someone else own it?  At the very least, could
you sell yourself into slavery?  Why don't biological mothers own their
children, who are produced from their bodies? 

It is not accidental that Locke said that every ``man'' has a property in his
own person; he didn't include women. Currently, the law insists that people are
not property, even if the relation between a person and her own body, or her
own labor, can be described in property terms.

