\reading{\emph{The Amistad}}
\readingcite{40 U.S. 518 (1841)}

STORY, Justice, delivered the opinion of the Court.

[The Amistad was a ship bound from one part of Cuba to another. On board were
three Spanish subjects: Captain Ransom Ferrer, Jose Ruiz, and Pedro Montez.
Also on board were 53 Africans, recently kidnapped from their home country and
transported to Cuba, a Spanish territory, where Ruiz and Montez had purchased
them as slaves. Slavery was legal in Cuba at the time, though Spanish law
banned the \textit{importation} of slaves from Africa to the Americas. At sea,
the Africans rose up, killed Ferrer, and took control of the Amistad,
attempting to sail it back to Africa. Instead, they ended up off the coast of
Long Island, where they and the ship were taken into custody by the U.S. Navy
and brought to port in Connecticut. Ruiz and Montez filed libels---a type of
property claim in admiralty law---seeking to recover the Africans and other
cargo they had on board. Their claim was backed by both the Spanish crown and
the Federal government, both of which cited a treaty between the two countries
(discussed by the Court below). The district court denied the Spaniards' claim
for the Africans, but granted their claim for the cargo, and the Circuit Court
summarily affirmed.]

{\dots} [T]he only parties now before the Court on one side, are the United
States, intervening for the sole purpose of procuring restitution of the
property as Spanish property, pursuant to the treaty, upon the grounds stated
by the other parties claiming the property in their respective libels. The
United States do not assert any property in themselves{\dots}. They simply
confine themselves to the right of the Spanish claimants to the restitution of
their property, upon the facts asserted in their respective allegations.

In the next place, the parties before the Court on the other side as appellees,
are {\dots} the negroes, (Cinque, and others,) asserting themselves in their
answer, not to be slaves, but free native Africans, kidnapped in their own
country, and illegally transported by force from that country; and now entitled
to maintain their freedom.

No question has been here made, as to the proprietary interests in the vessel
and cargo. It is admitted that they belong to Spanish subjects, and that they
ought to be restored. {\dots} The main controversy is, whether these negroes
are the property of Ruiz and Montez, and ought to be delivered up; and to this,
accordingly, we shall first direct our attention.

It has been argued on behalf of the United States, that the Court are bound to
deliver them up, according to the treaty of 1795, with Spain{\dots}. The ninth
article provides, `that all ships and merchandise, of what nature soever, which
shall be rescued out of the hands of any pirates or robbers, on the high seas,
shall be brought into some port of either state, and shall be delivered to the
custody of the officers of that port, in order to be taken care of and restored
entire to the true proprietor, as soon as due and sufficient proof shall be
made concerning the property thereof.' This is the article on which the main
reliance is placed on behalf of the United States, for the restitution of these
negroes. To bring the case within the article, it is essential to establish,
First, That these negroes, under all the circumstances, fall within the
description of merchandise, in the sense of the treaty. Secondly, That there
has been a rescue of them on the high seas, out of the hands of the pirates and
robbers; which, in the present case, can only be, by showing that they
themselves are pirates and robbers, and Third, That Ruiz and Montez, the
asserted proprietors, are the true proprietors, and have established their
title by competent proof.

If these negroes were, at the time, lawfully held as slaves under the laws of
Spain, and recognized by those laws as property capable of being lawfully
bought and sold; we see no reason why they may not justly be deemed within the
intent of the treaty, to be included under the denomination of merchandise,
and, as such ought to be restored to the claimants: for, upon that point, the
laws of Spain would seem to furnish the proper rule of interpretation. But,
admitting this, it is clear, in our opinion, that {\dots} these negroes never
were the lawful slaves of Ruiz or Montez, or of any other Spanish subjects.
They are natives of Africa, and were kidnapped there, and were unlawfully
transported to Cuba, in violation of the laws and treaties of Spain, and the
most solemn edicts and declarations of that government. By those laws, and
treaties, and edicts, the African slave trade is utterly abolished; the dealing
in that trade is deemed a heinous crime; and the negroes thereby introduced
into the dominions of Spain, are declared to be free. Ruiz and Montez are
proved to have made the pretended purchase of these negroes, with a full
knowledge of all the circumstances{\dots}.

If then, these negroes are not slaves, but are kidnapped Africans, who, by the
laws of Spain itself, are entitled to their freedom, and were kidnapped and
illegally carried to Cuba, and illegally detained and restrained on board the
Amistad; there is no pretence to say, that they are pirates or robbers. We may
lament the dreadful acts, by which they asserted their liberty, and took
possession of the Amistad, and endeavored to regain their native country; but
they cannot be deemed pirates or robbers in the sense of the law of nations, or
the treaty with Spain, or the laws of Spain itself; at least so far as those
laws have been brought to our knowledge. Nor do the libels of Ruiz or Montez
assert them to be such.

{\dots}It is also a most important consideration in the present case, which
ought not to be lost sight of, that, supposing these African negroes not to be
slaves, but kidnapped, and free negroes, the treaty with Spain cannot be
obligatory upon them; and the United States are bound to respect their rights
as much as those of Spanish subjects. The conflict of rights between the
parties under such circumstances, becomes positive and inevitable, and must be
decided upon the eternal principles of justice and international law. If the
contest were about any goods on board of this ship, to which American citizens
asserted a title, which was denied by the Spanish claimants, there could be no
doubt of the right to such American citizens to litigate their claims before
any competent American tribunal, notwithstanding the treaty with Spain.
\textit{A fortiori}, the doctrine must apply where human life and human liberty
are in issue; and constitute the very essence of the controversy. The treaty
with Spain never could have intended to take away the equal rights of all
foreigners, who should contest their claims before any of our Courts, to equal
justice; or to deprive such foreigners of the protection given them by other
treaties, or by the general law of nations. Upon the merits of the case, then,
there does not seem to us to be any ground for doubt, that these negroes ought
to be deemed free; and that the Spanish treaty interposes no obstacle to the
just assertion of their rights.

{\dots}Upon the whole, our opinion is, that the decree of the Circuit Court,
affirming that of the District Court, ought to be affirmed, {\dots} and that
the said negroes be declared to be free, and be dismissed from the custody of
the Court, and go without day.

BALDWIN, Justice, dissented.

