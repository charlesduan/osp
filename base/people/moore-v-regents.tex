\reading{Moore v. Regents of University of California}

\readingcite{793 P.2d 479 (Cal. 1990)}

PANELLI, Justice.

We granted review in this case to determine whether plaintiff has stated a cause
of action against his physician and other defendants for using his cells in
potentially lucrative medical research without his permission. Plaintiff
alleges that his physician failed to disclose preexisting research and economic
interests in the cells before obtaining consent to the medical procedures by
which they were extracted. The superior court sustained all defendants'
demurrers to the third amended complaint, and the Court of Appeal reversed. We
hold that the complaint states a cause of action for breach of the physician's
disclosure obligations, but not for conversion.

~

\readinghead{II. FACTS}

{\dots} The plaintiff is John Moore (Moore), who underwent treatment for
hairy-cell leukemia at the Medical Center of the University of California at
Los Angeles (UCLA Medical Center). The five defendants are: (1) Dr. David W.
Golde (Golde), a physician who attended Moore at UCLA Medical Center; (2) the
Regents of the University of California (Regents), who own and operate the
university; (3) Shirley G. Quan, a researcher employed by the Regents; (4)
Genetics Institute, Inc. (Genetics Institute); and (5) Sandoz Pharmaceuticals
Corporation and related entities (collectively Sandoz).

Moore first visited UCLA Medical Center on October 5, 1976, shortly after he
learned that he had hairy-cell leukemia. After hospitalizing Moore and
``withdr[awing] extensive amounts of blood, bone marrow aspirate, and other
bodily substances,'' Golde confirmed that diagnosis. At this time all
defendants, including Golde, were aware that ``certain blood products and blood
components were of great value in a number of commercial and scientific
efforts'' and that access to a patient whose blood contained these substances
would provide ``competitive, commercial, and scientific advantages.''

On October 8, 1976, Golde recommended that Moore's spleen be removed. Golde
informed Moore ``that he had reason to fear for his life, and that the proposed
splenectomy operation  {\dots}  was necessary to slow down the progress of his
disease.'' Based upon Golde's representations, Moore signed a written consent
form authorizing the splenectomy.

Before the operation, Golde and Quan ``formed the intent and made arrangements
to obtain portions of [Moore's] spleen following its removal'' and to take them
to a separate research unit. {\dots} [N]either Golde nor Quan informed Moore of
their plans to conduct this research or requested his permission. Surgeons at
UCLA Medical Center {\dots} removed Moore's spleen on October 20, 1976.

Moore returned to the UCLA Medical Center several times between November 1976
and September 1983. He did so at Golde's direction and based upon
representations ``that such visits were necessary and required for his health
and well-being, and based upon the trust inherent in and by virtue of the
physician-patient relationship {\dots} .'' On each of these visits Golde
withdrew additional samples of ``blood, blood serum, skin, bone marrow
aspirate, and sperm.'' On each occasion Moore travelled to the UCLA Medical
Center from his home in Seattle because he had been told that the procedures
were to be performed only there and only under Golde's direction.

{}``In fact, [however,] throughout the period of time that [Moore] was under
[Golde's] care and treatment,  {\dots}  the defendants were actively involved
in a number of activities which they concealed from [Moore] {\dots} .''
Specifically, defendants were conducting research on Moore's cells and planned
to ``benefit financially and competitively  {\dots}  [by exploiting the cells]
and [their] exclusive access to [the cells] by virtue of [Golde's] on-going
physician-patient relationship {\dots} .''

Sometime before August 1979, Golde established a cell line from Moore's
T-lymphocytes.\readingfootnote{2}{A T-lymphocyte is a type of
white blood cell. T-lymphocytes produce lymphokines, or proteins that regulate
the immune system. Some lymphokines have potential therapeutic value. If the
genetic material responsible for producing a particular lymphokine can be
identified, it can sometimes be used to manufacture large quantities of the
lymphokine through the techniques of recombinant DNA. \par While the genetic
code for lymphokines does not vary from individual to individual, it can
nevertheless be quite difficult to locate the gene responsible for a particular
lymphokine. Because T-lymphocytes produce many different lymphokines, the
relevant gene is often like a needle in a haystack. Moore's T-lymphocytes were
interesting to the defendants because they overproduced certain lymphokines,
thus making the corresponding genetic material easier to identify. {\dots}\par
Cells taken directly from the body (primary cells) are not very useful for
these purposes. Primary cells typically reproduce a few times and then die. One
can, however, sometimes continue to use cells for an extended period of time by
developing them into a ``cell line,'' a culture capable of reproducing
indefinitely. This is not, however, always an easy task. ``Long-term growth of
human cells and tissues is difficult, often an art,'' and the probability of
succeeding with any given cell sample is low, except for a few types of cells
not involved in this case.} On January 30, 1981, the Regents applied for a
patent on the cell line, listing Golde and Quan as inventors. ``[B]y virtue of
an established policy  {\dots} , [the] Regents, Golde, and Quan would share in
any royalties or profits  {\dots}  arising out of [the] patent.'' The patent
issued on March 20, 1984, naming Golde and Quan as the inventors of the cell
line and the Regents as the assignee of the patent.

The Regent's patent also covers various methods for using the cell line to
produce lymphokines. Moore admits in his complaint that ``the true clinical
potential of each of the lymphokines {\dots} [is] difficult to predict, [but]
{\dots} competing commercial firms in these relevant fields have published
reports in biotechnology industry periodicals predicting a potential market of
approximately \$3.01 Billion Dollars by the year 1990 for a whole range of
[such lymphokines] {\dots} .''

[The Regents, Golde and Quan negotiated commercial agreements that paid them for
exclusive rights to the cell line.] {\dots}.

\readinghead{III. DISCUSSION}

\readinghead{\textit{A. Breach of Fiduciary Duty and Lack of Informed Consent}}

Moore repeatedly alleges that Golde failed to disclose the extent of his
research and economic interests in Moore's cells before obtaining consent to
the medical procedures by which the cells were extracted. These allegations, in
our view, state a cause of action against Golde for invading a legally
protected interest of his patient. This cause of action can properly be
characterized either as the breach of a fiduciary duty to disclose facts
material to the patient's consent or, alternatively, as the performance of
medical procedures without first having obtained the patient's informed
consent.

Our analysis begins with three well-established principles. First, ``a person of
adult years and in sound mind has the right, in the exercise of control over
his own body, to determine whether or not to submit to lawful medical
treatment.'' Second, ``the patient's consent to treatment, to be effective,
must be an informed consent.'' Third, in soliciting the patient's consent, a
physician has a fiduciary duty to disclose all information material to the
patient's decision. {\dots}

Accordingly, we hold that a physician who is seeking a patient's consent for a
medical procedure must, in order to satisfy his fiduciary duty and to obtain
the patient's informed consent, disclose personal interests unrelated to the
patient's health, whether research or economic, that may affect his medical
judgment. {\dots} 

\readinghead{\textit{B. Conversion}}

Moore also attempts to characterize the invasion of his rights as a
conversion---a tort that protects against interference with possessory and
ownership
interests in personal property. He theorizes that he continued to own his cells
following their removal from his body, at least for the purpose of directing
their use, and that he never consented to their use in potentially lucrative
medical research. Thus, to complete Moore's argument, defendants' unauthorized
use of his cells constitutes a conversion. As a result of the alleged
conversion, Moore claims a proprietary interest in each of the products that
any of the defendants might ever create from his cells or the patented cell
line.

{\dots} In effect, what Moore is asking us to do is to impose a tort duty on
scientists to investigate the consensual pedigree of each human cell sample
used in research. To impose such a duty, which would affect medical research of
importance to all of society, implicates policy concerns far removed from the
traditional, two-party ownership disputes in which the law of conversion arose.
Invoking a tort theory originally used to determine whether the loser or the
finder of a horse had the better title, Moore claims ownership of the results
of socially important medical research, including the genetic code for
chemicals that regulate the functions of every human being's immune
system.~{\dots}

\readinghead{1. Moore's Claim Under Existing Law}

 {}``To establish a conversion, plaintiff must establish an actual interference
with his \textit{ownership} or \textit{right of possession} {\dots} . Where
plaintiff neither has title to the property alleged to have been converted, nor
possession thereof, he cannot maintain an action for conversion.'' 

Since Moore clearly did not expect to retain possession of his cells following
their removal, to sue for their conversion he must have retained an ownership
interest in them. But there are several reasons to doubt that he did retain any
such interest. First, no reported judicial decision supports Moore's claim,
either directly or by close analogy. Second, California statutory law
drastically limits any continuing interest of a patient in excised cells.
Third, the subject matters of the Regents' patent---the patented cell line and
the products derived from it---cannot be Moore's property.

Neither the Court of Appeal's opinion, the parties' briefs, nor our research
discloses a case holding that a person retains a sufficient interest in excised
cells to support a cause of action for conversion. We do not find this
surprising, since the laws governing such things as human tissues,
transplantable organs,\readingfootnote{22}{See the Uniform
Anatomical Gift Act, Health and Safety Code section 7150 et seq. The act
permits a competent adult to ``give all or part of [his] body'' for certain
designated purposes, including ``transplantation, therapy, medical or dental
education, research, or advancement of medical or dental science.'' The act
does not, however, permit the donor to receive ``valuable consideration'' for
the transfer.} blood, fetuses, pituitary glands, corneal tissue, and dead
bodies deal with human biological materials as objects sui generis, regulating
their disposition to achieve policy goals rather than abandoning them to the
general law of personal property. It is these specialized statutes, not the law
of conversion, to which courts ordinarily should and do look for guidance on
the disposition of human biological materials.

Lacking direct authority for importing the law of conversion into this context,
Moore relies, as did the Court of Appeal, primarily on decisions  addressing
privacy rights. One line of cases involves unwanted publicity. These opinions
hold that every person has a proprietary interest in his own likeness and that
unauthorized, business use of a likeness is redressible as a tort. But in
neither opinion did the authoring court expressly base its holding on property
law. Each court stated, following Prosser, that it was ``pointless'' to debate
the proper characterization of the proprietary interest in a likeness. For
purposes of determining whether the tort of conversion lies, however, the
characterization of the right in question is far from pointless. Only property
can be converted.

Not only are the wrongful-publicity cases irrelevant to the issue of conversion,
but the analogy to them seriously misconceives the nature of the genetic
materials and research involved in this case. Moore, adopting the analogy
originally advanced by the Court of Appeal, argues that ``[i]f the courts have
found a sufficient proprietary interest in one's persona, how could one not
have a right in one's own genetic material, something far more profoundly the
essence of one's human uniqueness than a name or a face?'' However, as the
defendants' patent makes clear---and the complaint, too, if read with an
understanding of the scientific terms which it has borrowed from the
patent---the goal and result of defendants' efforts has been to manufacture
lymphokines.
Lymphokines, unlike a name or a face, have the same molecular structure in
every human being and the same, important functions in every human being's
immune system. Moreover, the particular genetic material which is responsible
for the natural production of lymphokines, and which defendants use to
manufacture lymphokines in the laboratory, is also the same in every person; it
is no more unique to Moore than the number of vertebrae in the spine or the
chemical formula of hemoglobin.

{\dots} [O]ne may earnestly wish to protect privacy and dignity without
accepting the extremely problematic conclusion that interference with those
interests amounts to a conversion of personal property. Nor is it necessary to
force the round pegs of ``privacy'' and ``dignity'' into the square hole of
``property'' in order to protect the patient, since the fiduciary-duty and
informed-consent theories protect these interests directly by requiring full
disclosure.

The next consideration that makes Moore's claim of ownership problematic is
California statutory law, which drastically limits a patient's control over
excised cells. Pursuant to Health and Safety Code section 7054.4,
``[n]otwithstanding any other provision of law, recognizable anatomical parts,
human tissues, anatomical human remains, or infectious waste following
conclusion of scientific use shall be disposed of by interment, incineration,
or any other method determined by the state department [of health services] to
protect the public health and safety.''\readingfootnote{32}{\dots Surgically
removed organs, such as a spleen, are both ``recognizable
anatomical parts'' and ``human tissues.'' Virus-infected cells, such as Moore's
T-lymphocytes, fit reasonably within the statute's definition of ``infectious
waste.'' {\dots}} Clearly the Legislature did not specifically intend this
statute to resolve the question of whether a patient is entitled to
compensation for the nonconsensual use of excised cells. A primary object of
the statute is to ensure the safe handling of potentially hazardous biological
waste materials. Yet one cannot escape the conclusion that the statute's
practical effect is to limit, drastically, a patient's control over excised
cells. By restricting how excised cells may be  used and requiring their
eventual destruction, the statute eliminates so many of the rights ordinarily
attached to property that one cannot simply assume that what is left amounts to
``property'' or ``ownership'' for purposes of conversion law.

It may be that some limited right to control the use of excised cells does
survive the operation of this statute. There is, for example, no need to read
the statute to permit ``scientific use'' contrary to the patient's expressed
wish. A fully informed patient may always withhold consent to treatment by a
physician whose research plans the patient does not approve. That right,
however, as already discussed, is protected by the fiduciary-duty and
informed-consent theories.

Finally, the subject matter of the Regents' patent---the patented cell line and
the products derived from it---cannot be Moore's property. This is because the
patented cell line is both factually and legally distinct from the cells taken
from Moore's body. Federal law permits the patenting of organisms  that
represent the product of ``human ingenuity,'' but not naturally occurring
organisms. Human cell lines are patentable because ``[l]ong-term adaptation and
growth of human tissues and cells in culture is difficult---often considered
an art  {\dots} ,'' and the probability of success is low. It is this
\textit{inventive effort} that patent law rewards, not the discovery of
naturally occurring raw materials. Thus, Moore's allegations that he owns the
cell line and the products derived from it are inconsistent with the patent,
which constitutes an authoritative determination that the cell line is the
product of invention. {\dots}

{\centering
2. Should Conversion Liability Be Extended\textbf{?}
\par}

{\dots} There are three reasons why it is inappropriate to impose liability for
conversion based upon the allegations of Moore's complaint. First, a fair
balancing of the relevant policy considerations counsels against extending the
tort. Second, problems in this area are better suited to legislative
resolution. Third, the tort of conversion is not necessary to protect patients'
rights. For these reasons, we conclude that the use of excised human cells in
medical research does not amount to a conversion.

Of the relevant policy considerations, two are of overriding importance. The
first is protection of a competent patient's right to make autonomous medical
decisions. That right, as already discussed, is grounded in well-recognized and
long-standing principles of fiduciary duty and informed consent. This policy
weighs in favor of providing a remedy to patients when physicians act with
undisclosed motives that may affect their professional judgment. The second
important policy consideration is that we not threaten with disabling civil
liability innocent parties who are engaged in socially useful activities, such
as researchers who have no reason to believe that their use of a particular
cell sample is, or may be, against a donor's wishes.

To reach an appropriate balance of these policy considerations is extremely
important. In its report to Congress, the Office of Technology Assessment
emphasized that ``[u]ncertainty about how courts will resolve disputes between
specimen sources and specimen users could be detrimental to both academic
researchers and the infant biotechnology industry, particularly when the rights
are asserted long after the specimen was obtained. The assertion of rights by
sources would affect not only the researcher who obtained the original
specimen, but perhaps other researchers as well.

\begin{quote}
{}``Biological materials are routinely distributed to other researchers for
experimental purposes, and scientists who obtain cell lines or other
specimen-derived products, such as gene clones, from the original researcher
could also be sued under certain legal theories [such as conversion].
Furthermore, the uncertainty could affect product developments as well as
research. Since inventions containing human tissues and cells may be patented
and licensed for commercial use, companies are unlikely to invest heavily in
developing, manufacturing, or marketing a product when uncertainty about clear
title exists.'' 
\end{quote}

Indeed, so significant is the potential obstacle to research stemming from
uncertainty about legal title to biological materials that the Office of
Technology Assessment reached this striking conclusion: ``[R]egardless of the
merit of claims by the different interested parties, resolving the current
uncertainty may be more important to the future of biotechnology than resolving
it in any particular way.''

We need not, however, make an arbitrary choice between liability and
nonliability. Instead, an examination of the relevant policy considerations
suggests an appropriate balance: Liability based upon existing disclosure
obligations, rather than an unprecedented extension of the conversion theory,
protects patients' rights of privacy and autonomy without unnecessarily
hindering research.

To be sure, the threat of liability for conversion might help to enforce
patients' rights indirectly. This is because physicians might be able to avoid
liability by obtaining patients' consent, in the broadest possible terms, to
any conceivable subsequent research use of excised cells. Unfortunately, to
extend the conversion theory would utterly sacrifice the other goal of
protecting innocent parties. Since conversion is a strict liability tort, it
would impose liability on all those into whose hands the cells come, whether or
not the particular defendant participated in, or knew of, the inadequate
disclosures that violated the patient's right to make an informed decision. In
contrast to the conversion theory, the fiduciary-duty and informed-consent
theories protect the patient directly, without punishing innocent parties or
creating disincentives to the conduct of socially beneficial research.

Research on human cells plays a critical role in medical research. This is so
because researchers are increasingly able to isolate naturally occurring,
medically useful biological substances and to produce useful quantities of such
substances through genetic engineering. These efforts are beginning to bear
fruit. Products developed through biotechnology that have already been approved
for marketing in this country include treatments and tests for leukemia,
cancer, diabetes, dwarfism, hepatitis-B, kidney transplant rejection,
emphysema, osteoporosis, ulcers, anemia, infertility, and gynecological tumors,
to name but a few.

The extension of conversion law into this area will hinder research by
restricting access to the necessary raw materials. Thousands of human cell
lines already exist in tissue repositories, such as the American Type Culture
Collection and those operated by the National Institutes of Health and the
American Cancer Society. These repositories respond to tens of thousands of
requests for samples annually. Since the patent office requires the holders of
patents on cell lines to make samples available to anyone, many patent holders
place their cell lines in repositories to avoid the administrative burden of
responding to requests. At present, human cell lines are routinely copied and
distributed to other researchers for experimental purposes, usually free of
charge. This exchange of scientific materials, which still is relatively free
and efficient, will surely be compromised if each cell sample becomes the
potential subject matter of a lawsuit.{\dots}

In deciding whether to create new tort duties we have in the past considered the
impact that expanded liability would have on activities that are important to
society, such as research.{\dots}

[T]he theory of liability that Moore urges us to endorse threatens to destroy
the economic incentive to conduct important medical research. If the use of
cells in research is a conversion, then with every cell sample a researcher
purchases a ticket in a litigation lottery. Because liability for conversion is
predicated on a continuing ownership interest, ``companies are unlikely to
invest heavily in developing, manufacturing, or marketing a product when
uncertainty about clear title exists.''
{\dots}\readingfootnote{42}{In order to make conversion
liability seem less of a threat to research, the dissent argues that
researchers could avoid liability by using only cell lines accompanied by
documentation of the source's consent. But consent forms do not come with
guarantees of validity. As medical malpractice litigation shows, challenges to
the validity and sufficiency of consent are not uncommon. Moreover, it is sheer
fantasy to hope that waivers might be obtained for the thousands of cell lines
and tissue samples presently in cell repositories and, for that reason, already
in wide use among researchers. The cell line derived from Moore's
T-lymphocytes, for example, has been available since 1984 to any researcher
from the American Type Culture Collection. Other cell lines have been in wide
use since as early as 1951.}

{\dots} If the scientific users of human cells are to be held liable for failing
to investigate the consensual pedigree of their raw materials, we believe the
Legislature should make that decision. Complex policy choices affecting all
society are involved, and ``[l]egislatures, in making such policy decisions,
have the ability to gather empirical evidence, solicit the advice of experts,
and hold hearings at which all interested parties present evidence and express
their views {\dots} .'' Legislative competence to act in this area is
demonstrated by the existing statutes governing the use and disposition of
human biological materials{\dots}.

Finally, there is no pressing need to impose a judicially created rule of strict
liability, since enforcement of physicians' disclosure obligations will protect
patients against the very type of harm with which Moore was threatened{\dots}.

For these reasons, we hold that the allegations of Moore's third amended
complaint state a cause of action for breach of fiduciary duty or lack of
informed consent, but not conversion{\dots}. 

ARABIAN, Justice, concurring.

{\dots} I write separately to give voice to a concern that I believe informs
much of that opinion but finds little or no expression therein. I speak of the
moral issue.

Plaintiff has asked us to recognize and enforce a right to sell one's own body
tissue \textit{for profit.} He entreats us to regard the human vessel---the
single most venerated and protected subject in any civilized society---as
equal with the basest commercial commodity. He urges us to commingle the sacred
with the profane. He asks much.

My learned colleague, Justice Mosk, in an impressive if ultimately unpersuasive
dissent, recognizes the moral dimension of the matter{\dots}. He concludes,
however, that morality militates in favor of recognizing plaintiff's claim for
conversion of his body tissue. Why? Essentially, he answers, because of these
defendants' moral shortcomings, duplicity and greed. Let them be compelled, he
argues, to disgorge a portion of their ill-gotten gains to the uninformed
individual whose body was invaded and exploited and without whom such profits
would not have been possible.

I share Justice Mosk's sense of outrage, but I cannot follow its path. His
eloquent paean to the human spirit illuminates the problem, not the solution.
Does it uplift or degrade the ``unique human persona'' to treat human tissue as
a fungible article of commerce? Would it advance or impede the human condition,
spiritually or scientifically, by delivering the majestic force of the law
behind plaintiff's claim? I do not know the answers to these troubling
questions, nor am I willing---like Justice Mosk---to treat them simply as
issues of ``tort'' law, susceptible of \textit{judicial} resolution{\dots}.

Clearly the Legislature, as the majority opinion suggests, is the proper
deliberative forum. Indeed, a legislative response creating a licensing scheme,
which establishes a fixed rate of profit sharing between researcher and
subject, has already been suggested. Such an arrangement would not only avoid
the moral and philosophical objections to a free market operation in body
tissue, but would also address stated concerns by eliminating the inherently
coercive effect of a waiver system and by compensating donors regardless of
temporal circumstances{\dots}.

BROUSSARD, Justice, concurring and dissenting.

\textbf{{\dots} }Concerned that the imposition of liability for conversion will
impede medical research by innocent scientists who use the resources of
existing cell repositories---a factual setting not presented here---the
majority opinion rests its holding, that a conversion action cannot be
maintained, largely on the proposition that a patient generally possesses no
right in a body part that has already been removed from his body. Here,
however, plaintiff has alleged that defendants interfered with his legal rights
before his body part was removed. Although a patient may not retain any legal
interest in a body part after its removal when he has properly consented to its
removal and use for scientific purposes, it is clear under California law that
before a body part is removed it is the patient, rather than his doctor or
hospital, who possesses the right to determine the use to which the body part
will be put after removal. If, as alleged in this case, plaintiff's doctor
improperly interfered with plaintiff's right to control the use of a body part
by wrongfully withholding material information from him before its removal,
under traditional common law principles plaintiff may maintain a conversion
action to recover the economic value of the right to control the use of his
body part. Accordingly, I dissent from the majority opinion insofar as it
rejects plaintiff's conversion cause of action.{\dots}

As a general matter, the tort of conversion protects an individual not only
against improper interference with the right of possession of his property but
also against unauthorized use of his property or improper interference with his
right to control the use of his property. Sections 227 and 228 of the
Restatement Second of Torts specifically provide in this regard that ``[o]ne
who uses a chattel in a manner which is a serious violation of the right of
another to control its use is subject to liability to the other for
conversion'' and that ``[o]ne who is authorized to make a particular use of a
chattel, and uses it in a manner exceeding the authorization, is subject to
liability for conversion to another whose right to control the use of the
chattel is thereby seriously violated.'' California cases have also long
recognized that ``unauthorized use'' of property can give rise to a conversion
action.

{\dots} Although in this case defendants did not disregard a specific directive
from plaintiff with regard to the future use of his body part, the complaint
alleges that, before the body part was removed, defendants intentionally
withheld material information that they were under an obligation to disclose to
plaintiff and that was necessary for his exercise of control over the body
part; the complaint also alleges that defendants withheld such information in
order to appropriate the control over the future use of such body part for
their own economic benefit. If these allegations are true, defendants clearly
improperly interfered with plaintiff's right in his body part at a time when he
had the authority to determine the future use of such part, thereby
misappropriating plaintiff's right of control for their own advantage. Under
these circumstances, the complaint fully satisfies the established requirements
of a conversion cause of action{\dots}.

Although the damages which plaintiff may recover in a conversion action may not
include the value of the patent and the derivative products, the fact that
plaintiff may not be entitled to all of the damages which his complaint seeks
does not justify denying his right to maintain any conversion action at all.
{\dots}

{\centering
III
\par}

{\dots} [E]ven in the rare instance---like the present case---in which a
conversion action might be successfully pursued, the potential liability is not
likely ``to destroy the economic incentive to conduct important medical
research,'' as the majority asserts. If, as the majority suggests, the great
bulk of the value of a cell line patent and derivative products is attributable
to the efforts of medical researchers and drug companies, rather than to the
``raw materials'' taken from a patient, the patient's damages will be
correspondingly limited, and innocent medical researchers and drug
manufacturers will retain the considerable economic benefits resulting from
their own work. Under established conversion law, a ``subsequent innocent
converter'' does not forfeit the proceeds of his own creative efforts, but
rather ``is entitled to the benefit of any work or labor that he has expended
on the [property] {\dots} .'' 

Finally, the majority's analysis of the relevant policy considerations tellingly
omits a most pertinent consideration. In identifying the interests of the
patient that are implicated by the decision whether to recognize a conversion
cause of action, the opinion speaks only of the ``patient's right to make
autonomous medical decisions'' and fails even to mention the patient's interest
in obtaining the economic value, if any, that may adhere in the subsequent use
of his own body parts. Although such economic value may constitute a fortuitous
``windfall'' to the patient, the fortuitous nature of the economic value does
not justify the creation of a novel exception from conversion liability which
sanctions the intentional misappropriation of that value from the patient.

{\dots}. Far from elevating these biological materials above the marketplace,
the majority's holding simply bars \textit{plaintiff,} the source of the cells,
from obtaining the benefit of the cells' value, but permits
\textit{defendants,} who allegedly obtained the cells from plaintiff by
improper means, to retain and exploit the full economic value of their
ill-gotten gains free of their ordinary common law liability for
conversion{\dots}.

MOSK, Justice, dissenting.

{\dots}  The majority [finds] three ``reasons to doubt'' that Moore retained a
sufficient ownership interest in his cells, after their excision, to support a
conversion cause of action. In my view the majority's three reasons, taken
singly or together, are inadequate to the task.

The majority's first reason is that ``no reported judicial decision supports
Moore's claim, either directly or by close analogy.'' Neither, however, is
there any reported decision rejecting such a claim. The issue is as new as its
source---the recent explosive growth in the commercialization of
biotechnology.

The majority next cite several statutes regulating aspects of the commerce in or
disposition of certain parts of the human body, and conclude in effect that in
the present case we should also ``look for guidance'' to the Legislature rather
than to the law of conversion. Surely this argument is out of place in an
opinion of the highest court of this state. As the majority acknowledge, the
law of conversion is a creature of the common law. ``{}`The inherent capacity
of the common law for growth and change is its most significant feature. Its
development has been determined by the social needs of the community which it
serves. It is constantly expanding and developing in keeping with advancing
civilization and the new conditions and progress of society, and adapting
itself to the gradual change of trade, commerce, arts, inventions, and the
needs of the country.' In short, as the United States Supreme Court has aptly
said, `This flexibility and capacity for growth and adaptation is the peculiar
boast and excellence of the common law.'  {\dots}  Although the Legislature may
of course speak to the subject, in the common law system the primary
instruments of this evolution are the courts, adjudicating on a regular basis
the rich variety of individual cases brought before them.'' {\dots}

{\centering
2.
\par}

The majority's second reason for doubting that Moore retained an ownership
interest in his cells after their excision is that ``California statutory law 
{\dots}  drastically limits a patient's control over excised cells.'' For this
proposition the majority rely on Health and Safety Code section 7054.4, set
forth in the margin.\readingfootnote{5}{Section 7054.4
provides:\par {}``Notwithstanding any other provision of law, recognizable
anatomical parts, human tissues, anatomical human remains, or infectious waste
following conclusion of scientific use shall be disposed by interment,
incineration, or any other method determined by the state department [of health
services] to protect the public health and safety.\par {}``As used in this
section, `infectious waste' means any material or article which has been, or
may have been, exposed to contagious or infectious disease.''} The majority
concede that the statute was not meant to directly resolve the question whether
a person in Moore's position has a cause of action for conversion, but reason
that it indirectly resolves the question by limiting the patient's control over
the fate of his excised cells: ``By restricting how excised cells may be used
and requiring their eventual destruction, the statute eliminates so many of the
rights ordinarily attached to property that one cannot simply assume that what
is left amounts to `property' or `ownership' for purposes of conversion law.''
As will appear, I do not believe section 7054.4 supports the just quoted
conclusion of the majority.

First, in my view the statute does not authorize the principal use that
defendants claim the right to make of Moore's tissue, i.e., its commercial
exploitation{\dots}.

By its terms, section 7054.4 permits only ``scientific use'' of excised body
parts and tissue before they must be destroyed. {\dots} It would stretch the
English language beyond recognition, however, to say that commercial
exploitation of the kind and degree alleged here is also a usual and ordinary
meaning of the phrase ``scientific use.''

{\dots} Secondly, even if section 7054.4 does permit defendants' commercial
exploitation of Moore's tissue under the guise of ``scientific use,'' it does
not follow that---as the majority conclude---the statute ``eliminates so many
of the rights ordinarily attached to property'' that what remains does not
amount to ``property'' or ``ownership'' for purposes of the law of conversion. 

The concepts of property and ownership in our law are extremely broad. A leading
decision of this court approved the following definition: `` `The term
``property'' is sufficiently comprehensive to include every species of estate,
real and personal, and everything which one person can own and transfer to
another. It extends to every species of right and interest capable of being
enjoyed as such upon which it is practicable to place a money value.'{}''

Being broad, the concept of property is also abstract: rather than referring
directly to a material object such as a parcel of land or the tractor that
cultivates it, the concept of property is often said to refer to a ``bundle of
rights'' that may be exercised with respect to that object---principally the
rights to possess the property, to use the property, to exclude others from the
property, and to dispose of the property by sale or by gift. ``Ownership is not
a single concrete entity but a bundle of rights and privileges as well as of
obligations.'' But the same bundle of rights does not attach to all forms of
property. For a variety of policy reasons, the law limits or even forbids the
exercise of certain rights over certain forms of property. For example, both
law and contract may limit the right of an owner of real property to use his
parcel as he sees fit.\readingfootnote{6}{Zoning or nuisance
laws, or covenants running with the land or equitable servitudes, or
condominium declarations, may prohibit certain uses of the parcel or regulate
the number, size, location, etc., of buildings an owner may erect on it. Even
if rental of the property is a permitted use, rent control laws may limit the
benefits of that use. Other uses may, on the contrary, be compelled: e.g., if
the property is a lease to extract minerals, the lease may be forfeited by law
or contract if the lessee does not exploit the resource. Historic preservation
laws may prohibit an owner from demolishing a building on the property, or even
from altering its appearance. And endangered species laws may limit an owner's
right to develop the land from its natural state.} Owners of various forms of
personal property may likewise be subject to restrictions on the time, place,
and manner of their use.\readingfootnote{7}{ Public health
and safety laws restrict in various ways the manufacture, distribution,
purchase, sale, and use of such property as food, drugs, cosmetics, tobacco,
alcoholic beverages, firearms, flammable or explosive materials, and waste
products. Other laws regulate the operation of private and commercial motor
vehicles, aircraft, and vessels.} Limitations on the disposition of real
property, while less common, may also be
imposed.\readingfootnote{8}{Provisions in a condominium
declaration may give the homeowners association a right of first refusal over a
proposed sale by a member. Provisions in a commercial lease may require the
lessor's consent to an assignment of the lease.} Finally, some types of
personal property may be sold but not given
away,\readingfootnote{9}{A person contemplating bankruptcy
may sell his property at its ``reasonably equivalent value,'' but he may not
make a gift of the same property.} while others may be given away but not
sold,\readingfootnote{10}{A sportsman may give away wild fish
or game that he has caught or killed pursuant to his license, but he may not
sell it{\dots}.} and still others may neither be given away nor
sold.\readingfootnote{11}{E.g., a license to practice a
profession, or a prescription drug in the hands of the person for whom it is
prescribed.} 

In each of the foregoing instances, the limitation or prohibition diminishes the
bundle of rights that would otherwise attach to the property, yet what remains
is still deemed in law to be a protectible property interest. ``Since property
or title is a complex bundle of rights, duties, powers and immunities, the
pruning away of some or a great many of these elements does not entirely
destroy the title {\dots} .'' (People v. Walker (1939) 33 Cal.App.2d 18, 20, 90
P.2d 854 [even the possessor of contraband has certain property rights in it
against anyone other than the state].) The same rule applies to Moore's
interest in his own body tissue: even if we assume that section 7054.4 limited
the use and disposition of his excised tissue in the manner claimed by the
majority, Moore nevertheless retained valuable rights in that tissue. Above
all, at the time of its excision he at least had the right to do with his own
tissue whatever the defendants did with it: i.e., he could have contracted with
researchers and pharmaceutical companies to develop and exploit the vast
commercial potential of his tissue and its products. Defendants certainly
believe that their right to do the foregoing is not barred by section 7054.4
and is a significant property right, as they have demonstrated by their
deliberate concealment from Moore of the true value of his tissue, their
efforts to obtain a patent on the Mo cell line, their contractual agreements to
exploit this material, their exclusion of Moore from any participation in the
profits, and their vigorous defense of this lawsuit. The Court of Appeal summed
up the point by observing that ``Defendants' position that plaintiff cannot own
his tissue, but that they can, is fraught with irony.'' It is also legally
untenable. As noted above, the majority cite no case holding that an
individual's right to develop and exploit the commercial potential of his own
tissue is not a right of sufficient worth or dignity to be deemed a protectible
property interest. In the absence of such authority---or of legislation to the
same effect---the right falls within the traditionally broad concept of
property in our law.

{\centering
3.
\par}

The majority's third and last reason for their conclusion that Moore has no
cause of action for conversion under existing law is that ``the subject matter
of the Regents' patent---the patented cell line and the products derived from
it---cannot be Moore's property.'' The majority then offer a dual explanation:
``This is because the patented cell line is \textit{factually} and
\textit{legally} distinct from the cells taken from Moore's body.'' Neither
branch of the explanation withstands analysis.

First, in support of their statement that the Mo cell line is ``factually
distinct'' from Moore's cells, the majority assert that ``Cells change while
being developed into a cell line and continue to change over time,'' and in
particular may acquire an abnormal number of chromosomes. No one disputes these
assertions, but they are nonetheless irrelevant. For present purposes no
distinction can be drawn between Moore's cells and the Mo cell line. It appears
that the principal reason for establishing a cell line is not to ``improve''
the quality of the parent cells but simply to extend their life indefinitely,
in order to permit long-term study and/or exploitation of the qualities already
present in such cells. The complaint alleges that Moore's cells naturally
produced certain valuable proteins in larger than normal quantities; indeed,
that was why defendants were eager to culture them in the first place.
Defendants do not claim that the cells of the Mo cell line are in any degree
more productive of such proteins than were Moore's own cells.{\dots}

Second, the majority assert in effect that Moore cannot have an ownership
interest in the Mo cell line because defendants patented it. The majority's
point wholly fails to meet Moore's claim that he is entitled to compensation
for defendants' unauthorized use of his bodily tissues \textit{before}
defendants patented the Mo cell line: defendants undertook such use immediately
after the splenectomy on October 20, 1976, and continued to extract and use
Moore's cells and tissue at least until September 20, 1983; the patent,
however, did not issue until March 20, 1984, more than seven years after the
unauthorized use began. Whatever the legal consequences of that event, it did
not operate retroactively to immunize defendants from accountability for
conduct occurring long before the patent was granted.

Nor did the issuance of the patent in 1984 necessarily have the drastic effect
that the majority contend. {\dots} [Moore] seeks to show that he is entitled,
in fairness and equity, to some share in the profits that defendants have made
and will make from their commercial exploitation of the Mo cell line. I do not
question that the cell line is primarily the product of defendants' inventive
effort. Yet likewise no one can question Moore's crucial contribution to the
invention---an invention named, ironically, after him: but for the cells of
Moore's body taken by defendants, \textit{there would have been no Mo cell
line.} Thus the complaint alleges that Moore's ``Blood and Bodily Substances
were absolutely essential to defendants' research and commercial activities
with regard to his cells, cell lines, [and] the Mo cell-line,  {\dots}  and
that defendants could not have applied for and had issued to them the Mo
cell-line patent and other patents described herein without obtaining and
culturing specimens of plaintiff's Blood and Bodily Substances.'' Defendants
admit this allegation by their demurrers, as well they should: for all their
expertise, defendants do not claim they could have extracted the Mo cell line
out of thin air{\dots}.

{\centering
4.
\par}

Having concluded---mistakenly, in my view---that Moore has no cause of action
for conversion under existing law, the majority next consider whether to
``extend'' the conversion cause of action to this context. Again the majority
find three reasons not to do so, and again I respectfully disagree with each.

The majority's first reason is that a balancing of the ``relevant policy
considerations'' counsels against recognizing a conversion cause of action in
these circumstances.  {\dots} \readingfootnote{14}{On this
record the majority's solicitude for the protection of ``innocent parties''
seems ironic. The complaint is replete with factual allegations---which we
must accept as true on this appeal---to the effect that defendants repeatedly
lied to Moore about their commercial exploitation of his tissue. For example,
the complaint contains detailed allegations that defendants falsely told Moore
that his numerous postoperative trips from his home in Seattle to the Medical
Center of the University of California at Los Angeles between 1976 and 1983
were necessary because his blood and other bodily fluids could be extracted
only by them at the latter facility; that defendants falsely told Moore that
the purpose of such extractions was to promote his health, when in fact it was
solely to promote defendants' ongoing research and commercial activities; and
that even when Moore expressly asked if defendants had discovered anything
about his blood that might have potential commercial value, defendants falsely
told him ``they had discovered nothing of any commercial or financial value in
his Blood or Bodily Substances, and in fact actively discouraged such
inquiries.'' These are not the acts of ``innocent parties.''} 

{\dots}The majority observe that many researchers obtain their tissue samples,
routinely and at little or no cost, from cell-culture repositories. The
majority then speculate that ``This exchange of scientific materials, which
still is relatively free and efficient, will surely be compromised if each cell
sample becomes the potential subject matter of a lawsuit.'' There are two
grounds to doubt that this prophecy will be fulfilled.

To begin with, if the relevant exchange of scientific materials was ever ``free
and efficient,'' it is much less so today. Since biological products of genetic
engineering became patentable in 1980, human cell lines have been amenable to
patent protection and, as the Court of Appeal observed in its opinion below,
``The rush to patent for exclusive use has been rampant.'' Among those who have
taken advantage of this development, of course, are the defendants herein: as
we have seen, defendants Golde and Quan obtained a patent on the Mo cell line
in 1984 and assigned it to defendant Regents. With such patentability has come
a drastic reduction in the formerly free access of researchers to new cell
lines and their products {\dots}.

Secondly, to the extent that cell cultures and cell lines may still be ``freely
exchanged,'' e.g., for purely research purposes, it does not follow that the
researcher who obtains such material must necessarily remain ignorant of any
limitations on its use: by means of appropriate recordkeeping, the researcher
can be assured that the source of the material has consented to his proposed
use of it, and hence that such use is not a conversion. To achieve this end the
originator of the tissue sample first determines the extent of the source's
informed consent to its use---e.g., for research only, or for public but
academic use, or for specific or general commercial purposes; he then enters
this information in the record of the tissue sample, and the record accompanies
the sample into the hands of any researcher who thereafter undertakes to work
with it. ``Record keeping would not be overly burdensome because researchers
generally keep accurate records of tissue sources for other reasons: to trace
anomalies to the medical history of the patient, to maintain title for other
researchers and for themselves, and to insure reproducibility of the
experiment.'' As the Court of Appeal correctly observed, any claim to the
contrary ``is dubious in light of the meticulous care and planning necessary in
serious modern medical research.''{\dots}

The majority claim that a conversion cause of action threatens to ``destroy the
economic incentive'' to conduct the type of research here in issue, but it is
difficult to take this hyperbole seriously. First, the majority reason that
with every cell sample a researcher ``purchases a ticket in a litigation
lottery.'' This is a colorful image, but it does not necessarily reflect
reality: as explained above, with proper recordkeeping the researcher acquires
not a litigation-lottery ticket but the information he needs precisely in order
to avoid litigation. {\dots} Second,  {\dots}  [only one person can make a
claim]---the original source of the research material that began that process.
{\dots}

In any event, in my view whatever merit the majority's single policy
consideration may have is outweighed by two contrary considerations, i.e.,
policies that are promoted by recognizing that every individual has a legally
protectible property interest in his own body and its products. First, our
society acknowledges a profound ethical imperative to respect the human body as
the physical and temporal expression of the unique human persona. One
manifestation of that respect is our prohibition against direct abuse of the
body by torture or other forms of cruel or unusual punishment. Another is our
prohibition against indirect abuse of the body by its economic exploitation for
the sole benefit of another person. The most abhorrent form of such
exploitation, of course, was the institution of slavery. Lesser forms, such as
indentured servitude or even debtor's prison, have also disappeared. Yet their
specter haunts the laboratories and boardrooms of today's biotechnological
research-industrial complex. It arises wherever scientists or industrialists
claim, as defendants claim here, the right to appropriate and exploit a
patient's tissue for their sole economic benefit---the right, in other words,
to freely mine or harvest valuable physical properties of the patient's body:
``Research with human cells that results in significant economic gain for the
researcher and no gain for the patient offends the traditional mores of our
society in a manner impossible to quantify. Such research tends to treat the
human body as a commodity---a means to a profitable end. The dignity and
sanctity with which we regard the human whole, body as well as mind and soul,
are absent when we allow researchers to further their own interests without the
patient's participation by using a patient's cells as the basis for a
marketable product.'' 

A second policy consideration adds notions of equity to those of ethics. Our
society values fundamental fairness in dealings between its members, and
condemns the unjust enrichment of any member at the expense of another. This is
particularly true when, as here, the parties are not in equal bargaining
positions. We are repeatedly told that the commercial products of the
biotechnological revolution ``hold the promise of tremendous profit.'' In the
case at bar, for example, the complaint alleges that the market for the kinds
of proteins produced by the Mo cell line was predicted to exceed \$3 billion by
1990. These profits are currently shared exclusively between the biotechnology
industry and the universities that support that industry{\dots}.

There is, however, a third party to the biotechnology enterprise---the patient
who is the source of the blood or tissue from which all these profits are
derived. While he may be a silent partner, his contribution to the venture is
absolutely crucial: as pointed out above, but for the cells of Moore's body
taken by defendants there would have been no Mo cell line at all. Yet
defendants deny that Moore is entitled to any share whatever in the proceeds of
this cell line. This is both inequitable and immoral. {\dots}

{}``Recognizing a donor's property rights would prevent unjust enrichment by
giving monetary rewards to the donor and researcher proportionate to the value
of their respective contributions. {\dots} Failing to compensate the patient
unjustly enriches the researcher because only the researcher's contribution is
recognized.'' In short, as the Court of Appeal succinctly put it, ``If this
science has become science for profit, then we fail to see any justification
for excluding the patient from participation in those profits.''

{\centering
5.
\par}

The majority's second reason for declining to extend the conversion cause of
action to the present context is that ``the Legislature should make that
decision.'' I do not doubt that the Legislature is competent to act on this
topic. The fact that the Legislature may intervene if and when it chooses,
however, does not in the meanwhile relieve the courts of their duty of
enforcing---or if need be, fashioning---an effective judicial remedy for the
wrong here alleged. {\dots}

By selective quotation of the statutes the majority seem to suggest that human
organs and blood cannot legally be sold on the open market---thereby implying
that if the Legislature were to act here it would impose a similar ban on
monetary compensation for the use of human tissue in biotechnological research
and development. But if that is the argument, the premise is unsound: contrary
to popular misconception, it is not true that human organs and blood cannot
legally be sold.

As to organs, the majority rely on the Uniform Anatomical Gift Act (UAGA) for
the proposition that a competent adult may make a post mortem gift of any part
of his body but may not receive ``valuable consideration'' for the transfer.
But the prohibition of the UAGA against the sale of a body part is much more
limited than the majority recognize: by its terms the prohibition applies only
to sales for ``transplantation'' or ``therapy.'' Yet a different section of the
UAGA authorizes the transfer and receipt of body parts for such additional
purposes as ``medical or dental education, research, or advancement of medical
or dental science.'' No section of the UAGA prohibits anyone from selling body
parts for any of those additional purposes; by clear implication, therefore,
such sales are legal. Indeed, the fact that the UAGA prohibits \textit{no}
sales of organs other than sales for ``transportation'' or ``therapy'' raises a
further implication that it is also legal for anyone to sell human tissue to a
biotechnology company for research and development purposes.

With respect to the sale of human blood the matter is much simpler: there is in
fact no prohibition against such sales. {\dots} [I]ndeed, such sales are
commonplace, particularly in the market for plasma.

It follows that the statutes regulating the transfers of human organs and blood
do not support the majority's refusal to recognize a conversion cause of action
for commercial exploitation of human blood cells without consent. On the
contrary, because such statutes treat both organs and blood as property that
can legally be sold in a variety of circumstances, they impliedly support
Moore's contention that his blood cells are likewise property for which he can
and should receive compensation, and hence are protected by the law of
conversion. 

{\centering
6.
\par}

The majority's final reason for refusing to recognize a conversion cause of
action on these facts is that ``there is no pressing need'' to do so because
the complaint also states another cause of action that is assertedly adequate
to the task; that cause of action is ``the breach of a fiduciary duty to
disclose facts material to the patient's consent or, alternatively,  {\dots} 
the performance of medical procedures without first having obtained the
patient's informed consent.''{\dots}

The remedy is largely illusory. ``[A]n action based on the physician's failure
to disclose material information sounds in negligence. As a practical matter,
however, it may be difficult to recover on this kind of negligence theory
because the patient must prove a \textit{causal connection} between his or her
injury and the physician's failure to inform.'' There are two barriers to
recovery. First, ``the patient must show that if he or she had been informed of
all pertinent information, he or she would have declined to consent to the
procedure in question.'' {\dots} ``There must be a causal relationship between
the physician's failure to inform and the injury to the plaintiff. Such a
causal connection arises only if it is established that had revelation been
made consent to treatment would not have been given.''

The second barrier to recovery is still higher, and is erected on the first: it
is not even enough for the plaintiff to prove that he personally would have
refused consent to the proposed treatment if he had been fully informed; he
must also prove that in the same circumstances \textit{no reasonably prudent
person} would have given such consent. {\dots}

Few if any judges or juries are likely to believe that disclosure of {\dots} a
possibility of research or development would dissuade a reasonably prudent
person from consenting to the treatment. For example, in the case at bar no
trier of fact is likely to believe that if defendants had disclosed their plans
for using Moore's cells, no reasonably prudent person in Moore's
position---i.e., a leukemia patient suffering from a grossly enlarged
spleen---would have
consented to the routine operation that saved or at least prolonged his life.
{\dots}

The second reason why the nondisclosure cause of action is inadequate for the
task that the majority assign to it is that it fails to solve half the problem
before us: it gives the patient only the right to \textit{refuse} consent,
i.e., the right to prohibit the commercialization of his tissue; it does not
give him the right to \textit{grant} consent to that commercialization on the
condition that he share in its proceeds.{\dots}

Third, the nondisclosure cause of action fails to reach a major class of
potential defendants: all those who are outside the strict physician-patient
relationship with the plaintiff. {\dots}

In sum, the nondisclosure cause of action [is] not an adequate substitute, in my
view, for the conversion cause of action{\dots}.

