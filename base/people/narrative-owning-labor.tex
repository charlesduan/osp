The Thirteenth Amendment is notable, among other things, for its lack of any
state action requirement.  While the other provisions of the Constitution
control what the government may do and how it may do it, the Thirteenth
Amendment is a command to everyone: there shall be no slavery in the United
States.  Why write it this way, rather than as a constraint on government
action?  

Consider employment contracts that bar employees from competing if they leave,
or bar them from working in the same area or the same industry, or bar them
from using any information they learned while working for the employer.  These
restrictive covenants may mean that a person may be unable to work in the only
field for which she is trained if she leaves her current employer, which is
likely to give her employer substantial leverage in negotiating salary and
other terms of employment.  Do these attempted contractual restrictions raise
any Thirteenth Amendment issues?  \textit{See }Orly Lobel, \textit{The New
Cognitive Property: Human Capital Law and the Reach of Intellectual Property},
93 Texas L. Rev. 789 (2015) (discussing multiple restrictions employers have
used to restrict former employees' use of their own knowledge); Dave Jamieson,
%\href{http://www.huffingtonpost.com/2015/04/10/jimmy-johns-noncompete-agreement_n_7042112.html}{\textstyleInternetlink{
\emph{Jimmy
John's `Oppressive' Noncompete Agreement Survives Court Challenge}, Huffington
Post, Apr. 10, 2015 (discussing fast food restaurant's noncompete agreement
that precludes low-wage employees from working for any competitor).

Separately, consider the Thirteenth Amendment's exception for ``involuntary
servitude'' as punishment for crime.  Prison takes away prisoners' liberty and
their ability to use their own property, and also coerces their labor.  Does
this mean that prisoners are property?  In 1871, the Virginia Supreme Court
declared prisoners to be ``slaves of the state.'' Ruffin v. Commonwealth, 62
Va. 1024 (1871).  After the Civil War, African-Americans in the South were
routinely arrested for almost any reason, and local governments then sold their
labor to white landowners for agricultural work.  Prison labor currently
produces over \$2 billion worth of goods every year, though most production now
takes place within prison walls.  All able-bodied federal prisoners are
required to work, at a pay scale ranging from \$0.25 to \$1.15 per hour.  Texas
and Georgia require prisoners to work without any pay.

