Generally, if the government ``takes'' ``property'' for its own use, the
government has to pay the former owner the fair market value of that property,
as we will discuss in the section on takings.

The Constitution, as amended, provides: ``Neither slavery nor involuntary
servitude, except as a punishment for crime whereof the party shall have been
duly convicted, shall exist within the United States, or any place subject to
their jurisdiction.'' U.S. Const., amend. XIII. The Thirteenth Amendment, along
with its cousins the Fourteenth and Fifteenth Amendments, were designed to
embed the results of the Civil War into the Constitution.  

Before the Civil War, slavery's defenders considered enslaved people to be
property, and many of slavery's opponents conceded that enslaved people were
property according to the law of the land.  Henry Clay, speaking against
abolition, contended: ``The total value {\dots} of the slave property in the
United States, is twelve hundred millions of dollars. {\dots}  It is the
subject of mortgages, deeds of trust, and family settlements. It {\dots} is the
sole reliance, in many instances, of creditors within and without the slave
States {\dots}. \textit{That is property which the law declares to be
property}.''  Was he right?  If he was wrong, how are we to determine what is
property?

The U.S. did not compensate enslavers upon emancipation; nor did it compensate
enslaved people.  \textit{But see }Roy E. Finkenbine, \textit{Belinda's
Petition: Reparations for Slavery in Revolutionary Massachusetts}, 64 Wm. \&
Mary Q. 95 (2007) (discussing a rare example of a pension being granted to an
aged ex-slave by the Massachusetts legislature, in consideration of her long
enslavement).  By contrast, in 1833, Britain abolished slavery but also
provided for the compensation of enslavers for their lost ``property,''
representing roughly 800,000 enslaved people.  The {\pounds}20 million the
government set aside to pay enslavers off represented 40\% of the total
government expenditure for 1834, and is the equivalent of between {\pounds}16
and {\pounds}17 billion, or \$26 billion, in 2015 terms.  Until the bank
bailouts of 2009, this payout---to 46,000 enslavers---was the largest in
British history.  Moreover, enslaved people were compelled to provide 45 hours
of unpaid labor each week for their former masters for a further four years. 
Many well-known Britons can trace their ancestors---and some fraction of their
family wealth---to enslavers.  \textit{See}
%\href{https://www.ucl.ac.uk/lbs/}{\textstyleInternetlink
{Legacies of British Slave-Ownership}.

Likewise, in 1825, France, warships at the ready, demanded that its former
colony Haiti compensate France for its loss of plantations and enslaved people.
 Enslavers submitted detailed claims, which were later reduced to 90 billion
francs (roughly \$14 billion in modern terms) to be paid over thirty years. 
Haiti took until 1947 to pay off both the original claim to France and the
additional interest accrued from borrowing from French banks to meet France's
deadlines.  Haiti is currently the poorest country in the Americas.  

