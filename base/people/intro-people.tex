Is there something special about calling a right a ``property'' right?  Property
usually has a standard set of attributes, including alienability and in
particular market-alienability: the owner's ability to transfer the property
through sale.  These attributes---sometimes known as sticks in the bundle of
rights that makes up property---can often be removed and added without
changing a thing's status as property.  However, the more the bundle at issue
differs from from the standard ``property'' bundle, the more it seems like the
legislature should decide the exact contours of the right rather than calling
it property and giving it the standard set of property attributes by default.

Another way of looking at the question is to ask whether anything is inherent in
the concept of property.  That is, does property define what you can do to what
you ``own''? If the answer is yes, that may be a reason to refuse to allow
certain things to become property, like people or their parts. If the answer is
no, then what purpose is the label property serving?

