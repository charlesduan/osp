\reading{White v. Samsung Electronics America, Inc.}

\readingcite{971 F.2d 1395 (9th Cir. 1992)}

\opinion \textsc{Goodwin}, Senior Circuit Judge:

This case involves a promotional ``fame and fortune'' dispute. In running a
particular advertisement without Vanna White's permission, defendants Samsung
Electronics America, Inc. (Samsung) and David Deutsch Associates, Inc.
(Deutsch) attempted to capitalize on White's fame to enhance their fortune.
White sued, alleging infringement of various intellectual property rights, but
the district court granted summary judgment in favor of the defendants. We
affirm in part, reverse in part, and remand.

Plaintiff Vanna White is the hostess of ``Wheel of Fortune,'' one of the most
popular game shows in television history. An estimated forty million people
watch the program daily. Capitalizing on the fame which her participation in
the show has bestowed on her, White markets her identity to various
advertisers.

The dispute in this case arose out of a series of advertisements prepared for
Samsung by Deutsch. The series ran in at least half a dozen publications with
widespread, and in some cases national, circulation. Each of the advertisements
in the series followed the same theme. Each depicted a current item from
popular culture and a Samsung electronic product. Each was set in the
twenty-first century and conveyed the message that the Samsung product would
still be in use by that time. By hypothesizing outrageous future outcomes for
the cultural items, the ads created humorous effects. For example, one
lampooned current popular notions of an unhealthy diet by depicting a raw steak
with the caption: ``Revealed to be health food. 2010 A.D.'' Another depicted
irreverent ``news''-show host Morton Downey Jr. in front of an American flag
with the caption: ``Presidential candidate. 2008 A.D.''

\captionedgraphic{white-samsung-ads}{Other advertisements in Samsung's
campaign.}

The advertisement which prompted the current dispute was for Samsung
video-cassette recorders (VCRs). The ad depicted a robot, dressed in a wig,
gown, and jewelry which Deutsch consciously selected to resemble White's hair
and dress. The robot was posed next to a game board which is instantly
recognizable as the Wheel of Fortune game show set, in a stance for which White
is famous. The caption of the ad read: ``Longest-running game show. 2012 A.D.''
Defendants referred to the ad as the ``Vanna White'' ad. Unlike the other
celebrities used in the campaign, White neither consented to the ads nor was
she paid.

\captionedgraphic{white-samsung-disputed-ad}{The advertisement in dispute.}

\captionedgraphic{white-samsung-compare}{Side-by-side comparison of robot and
White.}

Following the circulation of the robot ad, White sued Samsung and Deutsch in
federal district court\dots. The district court granted summary judgment
against White on each of her claims. White now appeals.\dots

\readinghead{II. Right of Publicity}

White next argues that the district court erred in granting summary judgment to
defendants on White's common law right of publicity claim. In \emph{Eastwood v.
Superior Court}, 149 Cal.App.3d 409, 198 Cal.Rptr. 342 (1983), the California
court of appeal stated that the common law right of publicity cause of action
``may be pleaded by alleging (1) the defendant's use of the plaintiff's
identity; (2) the appropriation of plaintiff's name or likeness to defendant's
advantage, commercially or otherwise; (3) lack of consent; and (4) resulting
injury.'' The district court dismissed White's claim for failure to satisfy
\emph{Eastwood}'s second prong, reasoning that defendants had not appropriated
White's
``name or likeness'' with their robot ad. We agree that the robot ad did not
make use of White's name or likeness. However, the common law right of
publicity is not so confined.

\dots [T]he common law right of publicity reaches means of appropriation other
than name or likeness, but that the specific means of appropriation are
relevant only for determining whether the defendant has in fact appropriated
the plaintiff's identity. The right of publicity does not require that
appropriations of identity be accomplished through particular means to be
actionable.\dots

As the \emph{Carson} court explained:
\begin{quote}
[t]he right of publicity has developed to protect the commercial interest of
celebrities in their identities. The theory of the right is that a celebrity's
identity can be valuable in the promotion of products, and the celebrity has an
interest that may be protected from the unauthorized commercial exploitation of
that identity\dots. If the celebrity's identity is commercially exploited,
there has been an invasion of his right whether or not his ``name or likeness''
is used.
\end{quote}
It is not important how the defendant has appropriated the plaintiff's identity,
but whether the defendant has done so.\dots A rule which says that
the right of publicity can be infringed only through the use of nine different
methods of appropriating identity merely challenges the clever advertising
strategist to come up with the tenth.

Indeed, if we treated the means of appropriation as dispositive in our analysis
of the right of publicity, we would not only weaken the right but effectively
eviscerate it. The right would fail to protect those plaintiffs most in need of
its protection. Advertisers use celebrities to promote their products. The more
popular the celebrity, the greater the number of people who recognize her, and
the greater the visibility for the product. The identities of the most popular
celebrities are not only the most attractive for advertisers, but also the
easiest to evoke without resorting to obvious means such as name, likeness, or
voice\dots.

Viewed separately, the individual aspects of the advertisement in the present
case say little. Viewed together, they leave little doubt about the celebrity
the ad is meant to depict.\dots Indeed, defendants themselves referred to
their ad as the ``Vanna White'' ad. We are not surprised.

Television and other media create marketable celebrity identity value.
Considerable energy and ingenuity are expended by those who have achieved
celebrity value to exploit it for profit. The law protects the celebrity's sole
right to exploit this value whether the celebrity has achieved her fame out of
rare ability, dumb luck, or a combination thereof. We decline Samsung and
Deutch's invitation to permit the evisceration of the common law right of
publicity through means as facile as those in this case. Because White has
alleged facts showing that Samsung and Deutsch had appropriated her identity,
the district court erred by rejecting, on summary judgment, White's common law
right of publicity claim.

[The court rejected First Amendment claims because the Samsung ad was commercial
speech, which generally receives less constitutional protection than
noncommercial speech.  The court also allowed White's Lanham Act claim,
alleging that the ad caused confusion about whether White sponsored or was
affiliated with Samsung, to continue.]\dots

[The partial dissent of Judge Alarcon is omitted]

\readinghead{\emph{White v. Samsung Electronics America, Inc.}, 989 F.2d 1512
(9th Cir. 1993)}

\opinion Kozinski, J., dissenting from denial of rehearing en banc.

Saddam Hussein wants to keep advertisers from using his picture in unflattering
contexts.\readingfootnote{1}{\emph{See} Eben Shapiro, \emph{Rising Caution on
Using Celebrity Images}, N.Y. Times, Nov. 4, 1992, at D20 (Iraqi diplomat
objects on right of publicity grounds to ad containing Hussein's picture and
caption ``History has shown what happens when one source controls all the
information'').} Clint Eastwood doesn't want tabloids to write about
him.\readingfootnote{2}{\emph{Eastwood v. Superior Court}, 149 Cal. App. 3d 409,
198 Cal. Rptr. 342 (1983).} Rudolf Valentino's heirs want to control his film
biography.\readingfootnote{3}{\emph{Guglielmi v. Spelling-Goldberg Prods.}, 25
Cal. 3d 860, 160 Cal. Rptr. 352, 603 P.2d 454 (1979) (Rudolph Valentino);
\emph{see also} \emph{Maheu v. CBS, Inc.}, 201 Cal. App. 3d 662, 668, 247 Cal.
Rptr. 304 (1988) (aide to Howard Hughes). \emph{Cf.} Frank Gannon, \emph{Vanna
Karenina}, in \emph{Vanna Karenina and Other Reflections} (1988) (A humorous
short story with a tragic ending. ``She thought of the first day she had met
VR\_\_SKY. How foolish she had been. How could she love a man who wouldn't even
tell her all the letters in his name?'').} The Girl Scouts don't want their
image soiled by association with certain
activities.\readingfootnote{4}{\emph{Girl Scouts v. Personality Posters Mfg.},
304 F.Supp. 1228 (S.D.N.Y.1969) (poster of a pregnant girl in a Girl Scout
uniform with the caption ``Be Prepared'').} George Lucas wants to keep Strategic
Defense Initiative fans from calling it ``Star
Wars.''\readingfootnote{5}{\emph{Lucasfilm Ltd. v. High Frontier}, 622 F.Supp.
931 (D.D.C.1985).} Pepsico doesn't want singers to use the word ``Pepsi'' in
their songs.\readingfootnote{6}{Pepsico Inc. claimed the lyrics and packaging of
grunge rocker Tad Doyle's ``Jack Pepsi'' song were ``offensive to [it] and
[\dots] likely to offend [its] customers,'' in part because they ``associate
[Pepsico] and its Pepsi marks with intoxication and drunk driving.'' Deborah
Russell, \emph{Doyle Leaves Pepsi Thirsty for Compensation}, Billboard, June 15,
1991, at 43. Conversely, the Hell's Angels recently sued Marvel Comics to keep
it from publishing a comic book called ``Hell's Angel,'' starring a character of
the same name. Marvel settled by paying \$35,000 to charity and promising never
to use the name ``Hell's Angel'' again in connection with any of its
publications. Marvel, \emph{Hell's Angels Settle Trademark Suit}, L.A. Daily J.,
Feb. 2, 1993, {\S} II, at 1.\par Trademarks are often reflected in the mirror of
our popular culture. See Truman Capote, \emph{Breakfast at Tiffany's} (1958);
Kurt Vonnegut, Jr., \emph{Breakfast of Champions} (1973); Tom Wolfe, \emph{The
Electric Kool-Aid Acid Test} (1968) (which, incidentally, includes a chapter on
the Hell's Angels); Larry Niven, \emph{Man of Steel, Woman of Kleenex}, in
\emph{All the Myriad Ways} (1971); \emph{Looking for Mr. Goodbar} (1977);
\emph{The Coca-Cola Kid} (1985) (using Coca-Cola as a metaphor for American
commercialism); \emph{The Kentucky Fried Movie} (1977); \emph{Harley Davidson
and the Marlboro Man} (1991); \emph{The Wonder Years} (ABC 1988-present)
(``Wonder Years'' was a slogan of Wonder Bread); Tim Rice \& Andrew Lloyd
Webber, \emph{Joseph and the Amazing Technicolor Dream Coat} (musical).\par
\emph{Hear} Janis Joplin, \emph{Mercedes Benz}, on \emph{Pearl} (CBS 1971); Paul
Simon, \emph{Kodachrome}, on \emph{There Goes Rhymin' Simon} (Warner 1973);
Leonard Cohen, \emph{Chelsea Hotel}, on \emph{The Best of Leonard Cohen} (CBS
1975); Bruce Springsteen, \emph{Cadillac Ranch}, on \emph{The River} (CBS 1980);
Prince, \emph{Little Red Corvette}, on \emph{1999} (Warner 1982); dada,
\emph{Dizz Knee Land}, on \emph{Puzzle} (IRS 1992) (``I just robbed a grocery
store---I'm going to Disneyland / I just flipped off President George---I'm
going to Disneyland''); Monty Python, \emph{Spam}, on \emph{The Final Rip Off}
(Virgin 1988); Roy Clark, \emph{Thank God and Greyhound [You're Gone]}, on
\emph{Roy Clark's Greatest Hits Volume I} (MCA 1979); Mel Tillis,
\emph{Coca-Cola Cowboy}, on \emph{The Very Best of} (MCA 1981) (``You're just a
Coca-Cola cowboy / You've got an Eastwood smile and Robert Redford
hair\dots'').\par \emph{Dance} to Talking Heads, \emph{Popular Favorites
1976-92: Sand in the Vaseline} (Sire 1992); Talking Heads, \emph{Popsicle}, on
\emph{id.} \emph{Admire} Andy Warhol, \emph{Campbell's Soup Can}. \emph{Cf.} REO
Speedwagon, 38 Special, and Jello Biafra of the Dead Kennedys.\par The creators
of some of these works might have gotten permission from the trademark owners,
though it's unlikely Kool-Aid relished being connected with LSD, Hershey with
homicidal maniacs, Disney with armed robbers, or Coca-Cola with cultural
imperialism. Certainly no free society can \emph{demand} that artists get such
permission.} Guy Lombardo wants an exclusive property right to ads that show big
bands playing on New Year's Eve.\readingfootnote{7}{\emph{Lombardo v. Doyle,
Dane \& Bernbach, Inc.}, 58 A.D.2d 620, 396 N.Y.S.2d 661 (1977).} Uri Geller
thinks he should be paid for ads showing psychics bending metal through
telekinesis.\readingfootnote{8}{\emph{Geller v. Fallon McElligott}, No.
90-Civ-2839 (S.D.N.Y. July 22, 1991) (involving a Timex ad).} Paul Prudhomme,
that household name, thinks the same about ads featuring corpulent bearded
chefs.\readingfootnote{9}{\emph{Prudhomme v. Procter \& Gamble Co.}, 800 F.Supp.
390 (E.D.La.1992).} And scads of copyright holders see purple when their
creations are made fun of.\readingfootnote{10}{\emph{E.g.}, \emph{Acuff-Rose
Music, Inc. v. Campbell}, 972 F.2d 1429 (6th Cir.1992); \emph{Cliffs Notes v.
Bantam Doubleday Dell Publishing Group, Inc.}, 886 F.2d 490 (2d Cir.1989);
\emph{Fisher v. Dees}, 794 F.2d 432 (9th Cir.1986); \emph{MCA, Inc. v. Wilson},
677 F.2d 180 (2d Cir.1981); \emph{Elsmere Music, Inc. v. NBC}, 623 F.2d 252 (2d
Cir.1980); \emph{Walt Disney Prods. v. The Air Pirates}, 581 F.2d 751 (9th
Cir.1978); \emph{Berlin v. E.C. Publications, Inc.}, 329 F.2d 541 (2d Cir.1964);
\emph{Lowenfels v. Nathan}, 2 F.Supp. 73 (S.D.N.Y.1932).}

Something very dangerous is going on here. Private property, including
intellectual property, is essential to our way of life. It provides an
incentive for investment and innovation; it stimulates the flourishing of our
culture; it protects the moral entitlements of people to the fruits of their
labors. But reducing too much to private property can be bad medicine. Private
land, for instance, is far more useful if separated from other private land by
public streets, roads and highways. Public parks, utility rights-of-way and
sewers reduce the amount of land in private hands, but vastly enhance the value
of the property that remains.

So too it is with intellectual property. Overprotecting intellectual property is
as harmful as underprotecting it. Creativity is impossible without a rich
public domain. Nothing today, likely nothing since we tamed fire, is genuinely
new: Culture, like science and technology, grows by accretion, each new creator
building on the works of those who came before. Overprotection stifles the very
creative forces it's supposed to nurture.

The panel's opinion is a classic case of overprotection. Concerned about what it
sees as a wrong done to Vanna White, the panel majority erects a property right
of remarkable and dangerous breadth: Under the majority's opinion, it's now a
tort for advertisers to remind the public of a celebrity. Not to use a
celebrity's name, voice, signature or likeness; not to imply the celebrity
endorses a product; but simply to evoke the celebrity's image in the public's
mind. This Orwellian notion withdraws far more from the public domain than
prudence and common sense allow.\dots

\readinghead{II}

\dots Under California law, White has the exclusive right to use her name,
likeness, signature and voice for commercial purposes. But Samsung didn't use
her name, voice or signature, and it certainly didn't use her likeness. The ad
just wouldn't have been funny had it depicted White or someone who resembled
her---the whole joke was that the game show host(ess) was a robot, not a real
person. No one seeing the ad could have thought this was supposed to be White
in 2012\dots.

\readinghead{III}

\dots Intellectual property rights aren't like some constitutional rights,
absolute guarantees protected against all kinds of interference, subtle as well
as blatant. They cast no penumbras, emit no emanations: The very point of
intellectual property laws is that they protect only against certain specific
kinds of appropriation. I can't publish unauthorized copies of, say,
\textit{Presumed Innocent}; I can't make a movie out of it. But I'm perfectly
free to write a book about an idealistic young prosecutor on trial for a crime
he didn't commit. So what if I got the idea from \textit{Presumed Innocent}? So
what if it reminds readers of the original? Have I ``eviscerated'' Scott
Turow's intellectual property rights? Certainly not. All creators draw in part
on the work of those who came before, referring to it, building on it, poking
fun at it; we call this creativity, not piracy.

The majority isn't, in fact, preventing the ``evisceration'' of Vanna White's
existing rights; it's creating a new and much broader property right, a right
unknown in California law.\dots Instead of having an exclusive right in her
name, likeness, signature or voice, every famous person now has an exclusive
right to anything that reminds the viewer of her. After all, that's all Samsung
did: It used an inanimate object to remind people of White, to ``evoke [her
identity].''\readingfootnote{17}{Some viewers might have
inferred White was endorsing the product, but that's a different story. The
right of publicity isn't aimed at or limited to false endorsements; that's what
the Lanham Act is for.  \par Note also that the majority's rule applies even to
advertisements that unintentionally remind people of someone. California law is
crystal clear that the common-law right of publicity may be violated even by
unintentional appropriations.}

Consider how sweeping this new right is. What is it about the ad that makes
people think of White?\dots Remove the game board from the ad, and no one
would think of Vanna White. But once you include the game board, anybody
standing beside it---a brunette woman, a man wearing women's clothes, a monkey
in a wig and gown---would evoke White's image, precisely the way the robot
did. It's the ``Wheel of Fortune'' set, not the robot's face or dress or
jewelry that evokes White's image. The panel is giving White an exclusive right
not in what she looks like or who she is, but in what she does for a living.

This is entirely the wrong place to strike the balance. Intellectual property
rights aren't free: They're imposed at the expense of future creators and of
the public at large. Where would we be if Charles Lindbergh had an exclusive
right in the concept of a heroic solo aviator? If Arthur Conan Doyle had gotten
a copyright in the idea of the detective story, or Albert Einstein had patented
the theory of relativity? If every author and celebrity had been given the
right to keep people from mocking them or their work? Surely this would have
made the world poorer, not richer, culturally as well as economically.

This is why intellectual property law is full of careful balances between what's
set aside for the owner and what's left in the public domain for the rest of
us: The relatively short life of patents; the longer, but finite, life of
copyrights; copyright's idea-expression dichotomy; the fair use doctrine; the
prohibition on copyrighting facts; the compulsory license of television
broadcasts and musical compositions; federal preemption of overbroad state
intellectual property laws; the nominative use doctrine in trademark law; the
right to make soundalike recordings. All of these diminish an intellectual
property owner's rights. All let the public use something created by someone
else. But all are necessary to maintain a free environment in which creative
genius can flourish.

The intellectual property right created by the panel here has none of these
essential limitations: No fair use exception; no right to parody; no
idea-expression dichotomy. It impoverishes the public domain, to the detriment
of future creators and the public at large. Instead of well-defined, limited
characteristics such as name, likeness or voice, advertisers will now have to
cope with vague claims of ``appropriation of identity,'' claims often made by
people with a wholly exaggerated sense of their own fame and significance.
Future Vanna Whites might not get the chance to create their personae, because
their employers may fear some celebrity will claim the persona is too similar
to her own. The public will be robbed of parodies of celebrities, and our
culture will be deprived of the valuable safety valve that parody and mockery
create.

Moreover, consider the moral dimension, about which the panel majority seems to
have gotten so exercised. Saying Samsung ``appropriated'' something of White's
begs the question: Should White have the exclusive right to something as broad
and amorphous as her ``identity''? Samsung's ad didn't simply copy White's
schtick---like all parody, it created something new. True, Samsung did it to
make money, but White does whatever she does to make money, too; the majority
talks of ``the difference between fun and profit,'' but in the entertainment
industry fun is profit. Why is Vanna White's right to exclusive for-profit use
of her persona---a persona that might not even be her own creation, but that
of a writer, director or producer---superior to Samsung's right to profit by
creating its own inventions? Why should she have such absolute rights to
control the conduct of others, unlimited by the idea-expression dichotomy or by
the fair use doctrine?

 To paraphrase only slightly \emph{Feist Publications, Inc. v. Rural Telephone
Service Co.}, 499 U.S. 340  (1991), it may seem unfair that much of the fruit of
a creator's labor may be used by others without compensation. But this is not
some unforeseen byproduct of our intellectual property system; it is the
system's very essence. Intellectual property law assures authors the right to
their original expression, but encourages others to build freely on the ideas
that underlie it. This result is neither unfair nor unfortunate: It is the
means by which intellectual property law advances the progress of science and
art. We give authors certain exclusive rights, but in exchange we get a richer
public domain. The majority ignores this wise teaching, and all of us are the
poorer for it.\dots

