\item Notice the role played by property in the majority's argument about the
effect of a victory for Moore on research: property, the majority reasons, will
stand as a barrier to research by interfering with second-comers. But the
majority then states that patents---also property---provide an economic
incentive to conduct research.  Is this consistent?  Why wouldn't property in
body parts provide an incentive for people to offer themselves for research?
Can you think of reasons patents might be legitimate despite standing as
barriers to research by second-comers?

Contrast the majority's argument that there are already so many cell lines in
use that it's too late to impose a property right with the arguments about
ownership through conquest offered in \textit{Johnson v. M'Intosh} and similar
cases: is the majority endorsing the same view of the relationship between
scientists and human bodies as the Supreme Court did between European
colonizers and land occupied by Native Americans?

The majority argues that recordkeeping would be too difficult to provide
researchers with the necessary certainty of title.  As we will see, title
systems where ownership is established through written records are quite
important to other kinds of property, such as land.  If ownership of land,
cars, and stock can all be tracked with sufficient certainty with titling
systems, why not cell lines?  Patents, too, are intangible rights---how do you
think researchers track whether the cell lines they're using are patented?

\item A Florida firefighter took a man's severed foot from an Interstate 95
crash
scene and was charged with misdemeanor theft.  She said she took it to train
her cadaver dog.  Under \textit{Moore}, can she be prosecuted for theft?  Does
it matter whether it would have been possible to reattach the foot if she had
not taken it?  See Keyonna Summers,
%\href{http://www.firerescue1.com/fire-news/500065-fla-ex-firefighter-sentenced-for-foot-theft}{\textstyleInternetlink
\emph{Fla.
ex-firefighter sentenced for foot theft}, Jun. 1, 2009. 

\item In January 1951, a 31-year-old African-American woman named Henrietta
Lacks
was diagnosed with cervical cancer.  She died, painfully, in October 1951,
leaving five children.  Without her knowledge or consent, or that of her
family, doctors gave a sample of her tumor to Dr. George Gey, a Johns Hopkins
researcher who was trying to find cells that would live indefinitely in culture
so researchers could more easily experiment on them.  Her cells were his first
success, and the cell line developed from her body was known as HeLa (for
Henrietta Lacks).  Dr. Jonas Salk used HeLa cells to develop the first polio
vaccine, and they also helped in the development of numerous other drugs,
treating diseases as diverse as Parkinson's, leukemia and the flu. More than
60,000 articles have been written about research based on HeLa cells.  Though
Dr. Gey didn't make money from them, other researchers did.  Selling HeLa cells
has generated millions in profits, but none for the Lacks family, the members
of which suffered from poverty and lack of education.

In fact, some Lacks family members suffered serious health problems, but they
only found out about HeLa cells by accident, more than two decades later.  Mrs.
Lacks's daughter-in-law met someone who recognized her surname and said he was
working with cells from ``a woman named Henrietta Lacks.''  She then told Mrs.
Lacks's son: ``Part of your mother, it's alive!''  The family was proud their
mother's cells had saved lives, but also felt exploited.  Some members of the
family had given blood to Johns Hopkins researchers, believing they were being
tested for cancer, but in fact the researchers wanted to use their blood to
determine whether HeLa cells were contaminating other cultures.  Poverty, race,
and education clearly increased the gap between the researchers and the
Lackses, but---especially compared to Mr. Moore's story---are any of those
the key?
\begin{quotation}
Ideas about informed consent have changed in the last 60 years, and the forms
now given to people having surgery or biopsies usually spell out that tissue
removed from them may be used for research. But {\dots} patients today don't
really have any more control over removed body parts than Mrs. Lacks did. Most
people just obediently sign the forms.

Which is as it should be, many scientists say, arguing that Mrs. Lacks's
immortal cells were an accident of biology, not something she created or
invented, and were used to benefit countless others. Most of what is removed
from people is of no value anyway, and researchers say it would be too
complicated and would hinder progress if ownership of such things were assigned
to patients and royalties had to be paid.

But in an age in which people can buy songs with the click of a mouse, that
argument may become harder to defend.
\end{quotation}
Denise Grady,
%\href{http://www.nytimes.com/2010/02/02/health/02seco.html?_r=0}
\emph{A Lasting Gift to Medicine That Wasn't Really a Gift}, N.Y. Times, Feb. 1,
2010.
For more, see \textsc{Rebecca Skloot, The Immortal Life of Henrietta Lacks}
(2010).
Ultimately, the National Institutes of Health agreed with the Lacks family that
her full genome data would only be available to researchers, in order to
preserve the family's privacy; that two representatives of the Lacks family
would serve on the NIH group responsible for reviewing biomedical researchers'
applications for controlled access to HeLa cells; and that any researcher who
uses that data would be asked to include an acknowledgement to the Lacks family
in their publications.  However, no one would provide any compensation to the
Lacks family.  Art Caplan,
%\href{http://www.nbcnews.com/health/health-news/nih-finally-makes-good-henrietta-lacks-family-its-about-time-f6C10867941}{\textstyleInternetlink{
\emph{NIH
finally makes good with Henrietta Lacks' family}, NBC News, Sept. 3, 2014.
 Is this a good solution?  Can you distinguish ``property'' interests from
``privacy'' or ``dignity'' interests in this story?

\item State and federal statutes implicitly recognize some kind of property
rights
in body parts, permitting gifts from both living persons and dead donors and
even permitting sales except for sales for the purpose of transplantation. 
See, e.g., 42 U.S.C.A. {\S}274e.  Body parts are therefore alienable---title
to them can be transferred---even though they can't be sold for some purposes.
 
Should we allow organs to be fully market-alienable, so that willing sellers
could offer up a kidney for compensation?  \textit{See, e.g.}, Richard A.
Epstein, \emph{The Human and Economic Dimensions of Altruism: The Case of Organ
Transplantation}, 37 J. Legal Stud. 459, 485-497 (2008); Radhika Rao,
\emph{Property,
Privacy, and the Human Body}, 80 B.U. L. Rev. 359 (2000); Julia D. Mahoney,
\emph{The
Market for Human Tissue}, 86 Va. L. Rev. 163 (2000).

Consider the following arguments for market-alienability: There is currently a
great shortage of transplantable organs such as hearts, lungs, livers, and
kidneys, leading to tens of thousands of deaths a year. Each day, 79 people
receive transplants, but 22 people die while waiting for a transplant. 
\url{http://www.organdonor.gov/about/data.html}.  The U.S. has an opt-in system
for organ donation at death, resulting in the fourth-highest organ donor rate
(26 donors per million people in the population).  Spain has the highest rate,
with 35.3 donors per million people.  Spain, like several other European
countries, in theory has an opt-out regime in which organs will be donated at
death in the absence of an opt-out, but in practice doctors will ask relatives
for consent regardless, and that consent is often denied.

What if we allowed people to be paid during life for their agreement to be
donors at death?  What objections or obstacles do you foresee to such a scheme?

What about sales by living donors?  People can already sell semen, skin tissue,
and blood.  Poor people would likely be most of the sellers, but proponents
note that using the market to obtain a \textit{supply} of organs doesn't mean
that they need to be \textit{distributed} only to those who can pay; Medicaid
pays for dialysis, which is quite expensive, and could also pay for a kidney
for poor patients.  In Iran, which does allow payments for kidney donations to
Iranian recipients, 84\% of donors are poor, but 50\% of recipients are also
poor, and Iran eliminated its transplant list of people awaiting kidneys.  Ahad
J. Ghods \& Shekoufeh Savaj, \emph{Iranian Model of Paid and Regulated
Living-Unrelated Kidney Donation}, 1 Clinical J. Am. Soc. Nephrology 1136
(2006).

To those who say that such a system would coerce the poor to sell their organs,
proponents respond that those sellers would be better off than they are in the
present system, where they're still poor and have fewer options for earning
money, many of which are equally or more dangerous and unpleasant.  Sellers who
later suffered kidney failure could get transplants.

Opponents note that there's evidence that donated blood is higher quality than
paid-for blood, though the significance of those studies is contested. 
Donating bodily products, opponents argue, is an altruistic act that improves
the human condition and provides a better guarantee of quality.  Selling, by
contrast, leads to attempts to sell shoddy products---here, unhealthy
organs---for gain.  Proponents of organ sales respond that poor-quality organs
can be
screened out.  To this, opponents rejoin that there's evidence of ``crowding
out'' of altruistic motives by commercial motives: when money enters a system,
people who previously participated out of the goodness of their hearts may
withdraw.  They don't want to feel like suckers when they aren't getting paid
and other people are.  Payment, then, might even lead to a reduced supply of
organs compared to the present system.

Opponents also argue that organ sales are degrading, reducing a person to the
commodified sum of her parts.  Proponents respond that dying of a curable
illness is also degrading, and that Western societies used to consider surgery,
artificial insemination, and autopsies degrading.  Life insurance used to be
rejected on the ground that it wrongly commodified the value of a human life. 
It's widely accepted now---did it degrade our humanity?  Likewise, people can
sell their time and the intellectual products of their minds.

But on this argument, we should be open to selling everything---why not let a
living donor sell her heart to provide for her family?  Why not let her sell
her child?  Not reassuringly, some proponents of organ sales believe that these
options should at least be considered, with appropriate safeguards.  They
contend that proper boundaries between market and non-market activity can be
maintained even if new aspects of life enter the market.  The same society that
came to accept life insurance and artificial insemination also eventually
outlawed slavery and child labor.  In fact, it can be harder to get people to
accept markets than it perhaps should be.  

If you were a legislator, how would you decide?  Suppose we decide to allow
kidney sales by living donors.  Does that make kidneys into property?  If so,
could a bankrupt person be forced to sell her property the way she can be
forced to sell most of her other assets, to pay off her creditors?  How might a
proponent of such sales respond to these and similar concerns?

Conversely, even limiting alienability of one's body to gratuitous donation
doesn't necessarily hold market forces at bay. Many people leave directives for
their bodies to be donated for scientific uses after their deaths, supplying a
resource that is desperately needed in medical education and research. But
unlike live tissues, traffic in dead bodies and body parts after donation is
largely unregulated.  This has led to the growth of a substantial industry
around the collection, preservation, and dissection of donated human cadavers,
and their distribution (for hefty fees) to the research and educational
institutions that require them. An extensive investigation by Reuters
journalists into this industry found a number of macabre abuses and shady
operators seemingly attracted by an easy profit. As one ``body broker'' told
Reuters: ``If you can't make a business when you're getting raw materials for
free, {\dots} you're dumb as a box of rocks.'' Brian Grow \& John Shiffman, The
Body Trade, Part 1: Body Brokers, Reuters (Oct. 24, 2017),
https://www.reuters.com/investigates/special-report/usa-bodies-brokers/. 
Interestingly, these body brokers---or ``non-transplant tissue banks'' as they
prefer to be called---seem to go to great lengths not to label their
transactions as sales of body parts. Instead, they bill their customers
``service fees,'' ``preparation fees,'' and ``processing fees.'' Why do you
think that is? Does it change your view to learn that some of these same
brokers later listed their inventory of human tissues as assets in bankruptcy
filings?

\item In light of all this complexity, is the language of property helpful in
crafting rules and drawing lines regarding permissible uses of the human body?
Consider the following anti-propertization argument, applied to rape:

\expectnext{radin-market-inalienability}
