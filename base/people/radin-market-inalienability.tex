\reading{Margaret Radin, \textit{Market-Inalienability}}
\readingcite{100 \textsc{Harv. L. Rev.} 1849 (1987)}

\dots In some cases market discourse itself might be antagonistic to interests
of personhood. [Judge Richard] Posner conceives of rape in terms of a marriage
and sex market. Posner concludes that ``the prevention of rape is essential to
protect the marriage market\ldots and more generally to secure property rights
in women's persons.'' Calabresi and Melamed also use market rhetoric to discuss
rape. In keeping with their view that ``property rules'' are prima facie more
efficient than ``liability rules'' for all entitlements, they argue that people
should hold a ``property rule'' entitlement in their own bodily integrity.
Further, they explain criminal punishment by the need for an ``indefinable
kicker,'' an extra cost to the rapist ``which represents society's need to keep
all property rules from being changed at will into liability rules.'' {\dots}
[L]ike Posner's, their view conceives of rape in market rhetoric. Bodily
integrity is an owned object with a price.

What is wrong with this rhetoric? The risk-of-error argument . . . is one
answer. Unsophisticated practitioners of cost-benefit analysis might tend to
undervalue the ``costs'' of rape to the victims. But this answer does not
exhaust the problem. Rather, for all but the deepest enthusiast, market
rhetoric seems intuitively out of place here, so inappropriate that it is
either silly or somehow insulting to the value being discussed.

One basis for this intuition is that market rhetoric conceives of bodily
integrity as a fungible object. A fungible object is replaceable with money or
other objects; in fact, possessing a fungible object is the same as possessing
money. A fungible object can pass in and out of the person's possession without
effect on the person as long as its market equivalent is given in exchange. To
speak of personal attributes as fungible objects---alienable ``goods''---is
intuitively wrong. Thinking of rape in market rhetoric implicitly
conceives of as fungible something that we know to be personal, in fact
conceives of as fungible property something we know to be too personal even to
be personal property. Bodily integrity is an attribute and not an object.
{\dots}

Systematically conceiving of personal attributes as fungible objects is
threatening to personhood, because it detaches from the person that which is
integral to the person. Such a conception makes actual loss of the attribute
easier to countenance. For someone who conceives bodily integrity as
``detached,'' the same person will remain even if bodily integrity is lost; but
if bodily integrity cannot be detached, the person cannot remain the same after
loss. Moreover, if my bodily integrity is an integral personal attribute, not a
detachable object, then hypothetically valuing my bodily integrity in money is
not far removed from valuing me in money. For all but the universal
commodifier, that is inappropriate treatment of a person. . . .

