\item Kozinski's dissent is often quoted because of its eloquence (not to
mention its witty if now somewhat dated cultural references).  Is it persuasive?

Consider the following argument: Property needs boundaries.  With intangible
rights, those boundaries may be difficult to determine---though as you will
see, it may not be all that simple to determine the appropriate boundaries of
physical property either.  Kozinski argues that the difficulty of determining
where celebrity identity ends and general cultural reference or invention
begins is a reason to reject a right of publicity.  But the majority concludes
that commercial speech---here, advertising---provides an acceptable boundary.
 Why isn't that a legitimate response?  Among other things, celebrities were
not satisfied with a right of publicity that only covered advertising, and
courts proved responsive to their desires.  Subsequent cases extended
California's right of publicity to art, video games, and even a
\textit{Cheers}-themed bar featuring animatronic robots.  (As Judge Kozinski
said, ``Robots again!'')  

\item Another recurring issue raised by Kozinski's dissent is the way in which
one person's property claims can interfere with another's.  Giving Vanna White a
property right in her identity means that Samsung, which owns the copyright in
its ad, can't freely run its ad.  In the \textit{Cheers} case, two actors who
had appeared on the television show were able to prevail against the
\textit{Cheers}-themed bar even though the bar had a license from the owner of
the copyright in the television show.  Thus, granting publicity rights directly
decreased the scope of the rights conferred by the copyright in \textit{Cheers},
which otherwise would have extended to allow the creation of such ``derivative
works'' as character-imitating robots.

\item Does it matter if we call the right of publicity a ``property'' right?
Consider the following: ``[I]n addition to and independent of that right of
privacy\ldots a man has a right in the publicity value of his photograph, i.e.,
the right to grant the exclusive privilege of publishing his picture \ldots.
Whether it be labelled a `property' right is immaterial; for here, as often
elsewhere, the tag `property' simply symbolizes the fact that courts enforce a
claim which has pecuniary worth.'' \emph{Haelan Labs., Inc. v. Topps Chewing
Gum, Inc.}, 202 F.2d 866 (2d Cir. 1953).  Suppose we characterized all privacy
rights
as property rights.  Would the label ``property'' make any difference to how the
law ought to treat invasions of privacy, such as the surreptitious recording of
women trying on clothes in changing rooms?
