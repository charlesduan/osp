\expected{neponsit-v-emigrant}
\expected{el-di-v-bethany-beach}

As you have already seen, one prevalent application of restrictive covenants is
in real estate development schemes that purport to subject many disparately held
parcels within a community to a common scheme or plan. Neponsit and Bethany
Beach are both communities that were initially developed under such a common
scheme. Like zoning ordinances, the restrictive covenants that burden privately
owned land within such developments may serve to quite comprehensively regulate
the uses of land by members of the community.

Indeed, one major American city---Houston---relies largely (though not
exclusively) on restrictive covenants to do the work that most other
municipalities achieve by zoning. When zoning swept the nation in the 1920s,
Houston was a growing, libertarian city, and sometimes-overheated rhetoric led
Houstonians to reject zoning as communistic government interference with
liberty. Later attempts to introduce zoning also failed due to the persistence
of anti-zoning movements. \textit{See} Barry J. Kaplan, \textit{Urban
Development, Economic Growth, and Personal Liberty: The Rhetoric of the Houston
Anti-Zoning Movements, 1947-1962}, 84 \textsc{Sw. Hist. Q.} 133 (1980);
\textit{see also} \textsc{Joel Kotkin, Opportunity Urbanism} (Oct.~2014),
\url{https://urbanreforminstitute.org/wp-content/uploads/2016/10/Kotkin-Opportunity-Urbanism_2014.pdf}
(positing
Houston's freedom and prosperity as the result of lack of zoning). The absence
of zoning doesn't mean that land use in Houston is unregulated---the city code
imposes minimum lot size and parking restrictions that have made the city the
most sprawling American metropolis, and the most heavily dependent on
privately-owned automobiles for transportation. But more detailed restrictions
are often the work of private covenants.

Private covenants are common in Houston, replicating many of the standard
functions of zoning, particularly separation of uses. Houston encourages
covenant creation by allowing their creation by a majority vote of subdivision
residents. Houstonians separate homes from businesses through restrictive
covenants that specify the appropriate use for each lot in a subdivision, and
enable every lot owner individually to sue. This regime works most effectively
in wealthy neighborhoods. Houston's city code, unlike that of most American
cities, also allows the city attorney to sue to enforce restrictive covenants.
The city may seek civil penalties of up to \$1000 per day for a violation, and
the city prioritizes enforcement of use restrictions, rather than other
covenants such as aesthetic rules. In essence, the city has recreated ``single
use zoning'' as covenant enforcement.

\expected{neponsit-v-emigrant}

\defbook{cai-stat-review-2014}{
instauth=Community Associations Institute,
title=National and State Statistical Review for 2014,
year=2014,
url=https://foundation.caionline.org/wp-content/uploads/2017/07/CAI_2014_StatsReview_WEB.pdf,
}

Both within and outside of Houston, such uses of restrictive covenants may
allow---like the covenants in \textit{Neponsit}---for centralized
\textit{private} authority to administer and enforce the covenants through a
corporation or association constituted from among the property owners in the
community. This kind of collective governance of land uses via restrictive
covenants is what the Third Restatement refers to as a \term{common-interest
community}. There are three primary types of common-interest community in the
United States: the \term{homeowners association} (or ``HOA''), the
\term{condominium} (or ``condo''), and the \term{cooperative} (or
``co-op''). State statutes provide for the creation of these legal
entities. According to the Community Associations Institute---an international
research, education, and advocacy nonprofit organization that promotes and
supports common-interest communities---there were over 330,000 common-interest
communities in the United States in 2014, encompassing 26.7 million housing
units and 66.7 million residents. \sentence{see cai-stat-review-2014 at 1}.


