\reading{Hidden Harbour Estates, Inc. v. Norman}

\readingcite{309 So. 2d 180 (Fla. Dist. Ct. App. 1975)}

\opinion \textsc{Downey}, Judge.

The question presented on this appeal is whether the board of directors of a
condominium association may adopt a rule or regulation prohibiting the use of
alcoholic beverages in certain areas of the common elements of the condominium.

Appellant is the condominium association formed, pursuant to a Declaration of
Condominium, to operate a 202 unit condominium known as Hidden Harbour. Article
3.3(f) of appellant's articles of incorporation provides, inter alia, that the
association shall have the power `to make and amend reasonable rules and
regulations respecting the use of the condominium property.' A similar provision
is contained in the Declaration of Condominium.

Among the common elements of the condominium is a club house used for social
occasions. Pursuant to the association's rule making power the directors of the
association adopted a rule prohibiting the use of alcoholic beverages in the
club house and adjacent areas. Appellees, as the owners of one condominium unit,
objected to the rule, which incidentally had been approved by the condominium
owners voting by a margin of 2 to 1 (126 to 63). Being dissatisfied with the
association's action, appellees brought this injunction suit to prohibit the
enforcement of the rule. After a trial on the merits at which appellees showed
there had been no untoward incidents occurring in the club house during social
events when alcoholic beverages were consumed, the trial court granted a
permanent injunction against enforcement of said rule. The trial court was of
the view that rules and regulations adopted in pursuance of the management and
operation of the condominium `must have some reasonable relationship to the
protection of life, property or the general welfare of the residents of the
condominium in order for it to be valid and enforceable.' In its final judgment
the trial court further held that any resident of the condominium might engage
in any lawful action in the club house or on any common condominium property
unless such action was engaged in or carried on in such a manner as to
constitute a nuisance.

With all due respect to the veteran trial judge, we disagree. It appears to us
that inherent in the condominium concept is the principle that to promote the
health, happiness, and peace of mind of the majority of the unit owners since
they are living in such close proximity and using facilities in common, each
unit owner must give up a certain degree of freedom of choice which he might
otherwise enjoy in separate, privately owned property. Condominium unit owners
comprise a little democratic sub society of necessity more restrictive as it
pertains to use of condominium property than may be existent outside the
condominium organization. The Declaration of Condominium involved herein is
replete with examples of the curtailment of individual rights usually associated
with the private ownership of property. It provides, for example, that no sale
may be effectuated without approval; no minors may be permanent residents; no
pets are allowed.

Certainly, the association is not at liberty to adopt arbitrary or capricious
rules bearing no relationship to the health, happiness and enjoyment of life of
the various unit owners. On the contrary, we believe the test is reasonableness.
If a rule is reasonable the association can adopt it; if not, it cannot. It is
not necessary that conduct be so offensive as to constitute a nuisance in order
to justify regulation thereof. Of course, this means that each case must be
considered upon the peculiar facts and circumstances thereto appertaining.

Finally, restrictions on the use of alcoholic beverages are widespread
throughout both governmental and private sectors; there is nothing unreasonable
or unusual about a group of people electing to prohibit their use in commonly
owned areas.

Accordingly, the judgment appealed from is reversed and the cause is remanded
with directions to enter judgment for the appellant.

