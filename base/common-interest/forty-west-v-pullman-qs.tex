\expected{forty-west-v-pullman}

\item For further background on this dispute, including quotes from David
Pullman himself, see Dan Barry, \textit{Sleepless and Litigious in 7B: A Co-op
War Ends in Court}, \textsc{N.Y. Times} (June 7, 2003),
\url{https://nyti.ms/2leMd9c}.

\item What aspect of the Court of Appeals' analysis constitutes ``heightened
vigilance''?

\item The \textsc{Restatement} does not adopt the business judgment rule for
review of
board actions, instead applying a ``reasonableness'' standard. The Reporter's
comments suggest that the reasonableness of an enforcement action will depend on
any number of factors, including its proportionality to the resident's offensive
conduct (\textit{e.g.}, no \$1,000 fines for a single instance of failing to
sort an aluminum can for recycling), the logical relationship between the
offensive conduct and the remedy (\textit{e.g.}, no revocation of parking
privileges for breach of a pet restriction), and whether the resident was
provided with sufficient notice and opportunity to respond to the managers'
complaint before any enforcement action was taken. \textit{See}
\textsc{Restatement} \S~6.8 \& cmt. b. Elsewhere the \textsc{Restatement} states
that
board members and officers have duties of care, prudence, and fairness toward
members of the community. \textit{Id.} \S~6.13 \& cmt. b. Is the
\textsc{Restatement}
position consistent with \textit{Pullman}? If not, how does it differ?

\item The Court of Appeals did not consider the question whether the provision
in Pullman's proprietary lease allowing the cooperative to kick him out on
grounds that he was ``objectionable'' should be enforceable as a general matter.
If it had, what do you think would have been the result? Does it matter which
standard---reasonableness or the more permissive standard applicable to
\inline{ccrs}---applies? Which do you think ought to apply to the covenants in
the proprietary leases of a cooperative?

\item Say you live in a residential neighborhood unencumbered by any restrictive
covenants. Could you and your neighbors come together and decide to sell an
unfriendly neighbor's house over his objection? If not, what additional facts
make it possible for the residents of 40 West 67th Street (a tudor-style luxury
pre-war apartment building half a tree-lined block from Central Park) to vote
Pullman out of the apartment he bought in their building?

\expected{hidden-harbour-v-norman}

\item Common-interest communities are sometimes likened to miniature private
governments. (Recall \textit{Norman's} description of condominium owners as ``a
little democratic sub society.'') The analogy holds up somewhat: they hold
elections, the elected leaders can pass rules that all are bound to follow; they
can assess fines for breaking the rules; they can levy the equivalent of taxes
to fund common services. There are, of course, important differences---not least
failure to adhere to the principle of one-person-one-vote. But \textit{Pullman}
suggests another distinction: could any government officer or entity in the
United States do to one of its citizens what Pullman's neighbors did to him? If
not, what are the limits on government authority that would prevent such action,
and what are the justifications for those limits? Do these justifications carry
less force in the context of the enforcement of servitudes by the managers of a
common-interest-community?

