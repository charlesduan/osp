A \term{condominium} is very similar to a homeowners association, except it
typically covers either a single multi-unit structure or several structures
comprising attached residences on a single contiguous lot. Like a homeowners
association, a condominium is established by filing with the appropriate public
official a \term{condominium declaration}, which like the homeowners
association declaration will contain the \inline{ccrs} that will govern the
condominium, and will provide for a condominium association to administer the
\inline{ccrs} and otherwise act on behalf of the community. State statutes
typically impose a bit more regulation on condominiums than on subdivision HOAs,
sometimes setting forth substantive rules limiting the powers of condominium
associations or subjecting them to certain procedural requirements. But
condominium associations typically have the same types of powers as HOAs,
including the power to assess dues and special assessments from individual
owner/members. 

One important distinction between condominiums and homeowners associations has
to do with how title to property is held in each. In a condominium, each unit
owner holds title to their individual unit in fee simple, but the individual
unit owners collectively own all common areas of the condominium property
(hallways, common outdoor spaces, lobbies, recreation areas, etc.) as tenants in
common. State statutes prohibit condominium owners from seeking partition of
these commonly owned spaces. As with voting rights in the condominium
association, each owner's fractional share in this tenancy in common is
typically determined by some proxy for the value of the owner's particular unit,
such as square footage.

