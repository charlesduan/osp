What happens if a resident of a common interest community breaches a covenant?
How can the governing body of the community---the HOA managers, the condo board,
or the co-op board---enforce the rules laid down in the restrictive covenants
against breaching community members? \textit{Neponsit} provides one answer: the
breach of a covenant to pay money---such as dues and assessments---will serve as
an equitable lien on the breaching resident's property in the community. This
lien could be foreclosed, or more commonly the threat of foreclosure and the
encumbrance of the lien can be used to leverage payment if and when the resident
ever tries to sell her home. The governing body could also sue to recover unpaid
sums, but because this involves significant additional expense it is typically
an unattractive option reserved as a last resort.

But what about covenants that restrict use of property in the community---or
rules that govern the conduct of residents on the community's property? The
Restatement suggests that the governing bodies of common-interest communities
enjoy wide latitude to enforce the restrictions in governing documents. Section
6.8 provides: ``In addition to seeking court enforcement, the association may
adopt reasonable rules and procedures to encourage compliance and deter
violations, including the imposition of fines, penalties, late fees, and the
withdrawal of privileges to use common recreational and social facilities.''
Typically the governing documents will empower the association or board to levy
fines against residents for their breach of such rules of conduct or use. Those
fines, like unpaid dues or assessments, can also become an equitable lien on the
resident's property if state law and/or the declaration so provide.

How should we assess the ``reasonableness'' of any particular enforcement
action? And how searching a review should courts take of such actions if and
when they are challenged by aggrieved members of the common-interest community?

