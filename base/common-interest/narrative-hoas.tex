\expected{neponsit-v-emigrant}

\defabbrev{ccrs}{declaration of covenants, conditions, and restrictions
(CC\&Rs)}

The homeowners association is the most common type of common-interest community
in the United States---over half of all common interest communities in the
United States are HOAs. \sentence{see cai-stat-review-2014 at 1}.
In an HOA, the creation of community-wide
restrictive covenants typically happens at the planning stage: a real estate
developer plans out a subdivision of a contiguous parcel of undeveloped or
underdeveloped land, and files with the local clerk or register of deeds a
\textbf{subdivision plat} mapping out a survey of the separate lots of the
planned community and a \textbf{\inline{ccrs}} to bind each of those lots as
restrictive
covenants. When the subdivided lots are initially sold, the developer writes the
same covenants into the deed to every lot, either explicitly or incorporating
the \inline{ccrs} of the declaration by reference. The \inline{ccrs} will
typically delegate enforcement to a homeowners association---a legal entity that
is incorporated or otherwise created for the purpose of managing the
common-interest community (as with the property owners' association in
\textit{Neponsit}). The association's membership is comprised of all owners of
real property in the subdivision. These members are entitled to elect a board of
managers to act on behalf of the association, though votes are usually not
equally distributed to all residents; typically votes are allocated according to
some proxy for property value, such as lot size.

The association itself may hold title to real property in common areas of the
subdivision---such as private roads, parks and other recreational facilities,
and common utilities. It may also contract on behalf of the community for common
services, such as professional security guards. But its main function is to
administer, modify as necessary, and enforce the restrictive covenants that bind
the real property in the subdivision. This includes the collection of HOA
dues---such as the fees that were at issue in \textit{Neponsit}---that go toward
the maintenance of the subdivision and other expenses incurred by the
association (for example, professional fees for attorneys, accountants, etc.).
The association is typically also empowered to levy special assessments against
property owners in the subdivision as it deems necessary. \textit{See}
\textsc{Restatement}, \S~6.5. The authority of the association to act is
governed both by the \inline{ccrs} and by a set of bylaws---like the bylaws of
any other corporation---that set forth in detail what actions the managers may
take according to what procedures, what actions require a vote of all members of
the association, and whether there is any supermajority requirement for certain
actions. As we will see, the association may also enact regulations regarding
use and maintenance of privately owned property in the subdivision that go
beyond the \inline{ccrs}.

