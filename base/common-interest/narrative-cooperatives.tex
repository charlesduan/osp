By far the least common form of common-interest community is the
\textbf{cooperative}. In a cooperative, title to all real property in the
community (typically an apartment building) is held by a cooperative
corporation, whose shareholders are the residents of individual units. As with
the other common-interest communities, the number of shares each individual unit
owner holds is typically proportional to some proxy for the value of their
residence---such as square footage. Each resident's shares are ``appurtenant''
(i.e., connected) to a \textbf{proprietary lease} for a particular unit---a
lease whose term is tied to the resident's ownership of their shares in the
cooperative. Co-op owners therefore have a dual relationship with their
common-interest community: they are formally tenants, but at the same time they
are shareholders of the (corporate) landlord. The proprietary lease typically
plays the role that CC\&Rs serve in HOAs and condominiums: it contains the
covenants restricting residents' use of their own unit and any common spaces,
and in lieu of rent it obliges residents to pay maintenance fees---which
typically represent a fractional share of both operating expenses and carrying
costs of the entire property (such as mortgage payments and property taxes).

The board of directors of a cooperative corporation typically wields significant
power over the property and its residents. In addition to administering and
enforcing the terms of the proprietary lease and managing the property on behalf
of all the residents, co-op boards are typically empowered to create and enforce
additional rules to govern the community via their own by-laws and, sometimes,
separate and potentially quite intrusive ``house rules.'' Beyond this, the
governing documents of most co-operatives reserve to the board a right to
withhold consent to any transfer of shares in the corporation (and, thus, of the
proprietary lease to any unit in the cooperative). Absent violation of the
anti-discrimination laws, boards are generally free to arbitrarily withhold such
consent. One justification for this power is that residents of a co-operative
depend on one another for the financial stability of their homes: a shareholder
who fails to pay maintenance on time could threaten not only themselves but the
entire community with foreclosure of a mortgage or a tax lien, and the board
therefore has an interest in screening new shareholders for financial
wherewithal and reliability. But another theory justifying such power is that a
cooperative is, as its name implies, a form of collective governance of an
intimate residential community, which limits the appropriate degree of outside
legal interference. As the New York Court of Appeals put it: ``there is no
reason why the owners of the co-operative apartment house could not decide for
themselves with whom they wish to share their elevators, their common halls and
facilities, their stockholders' meetings, their management problems and
responsibilities and their homes.'' \emph{Weisner v. 791 Park Ave. Corp.}, 160
N.E.2d 720, 724 (N.Y. 1959).

Cooperatives exist almost exclusively in New York City, where they account for
the majority of owner-occupied apartments in Manhattan. Given the tremendous
power co-operative boards can exercise over admission of new shareholders, it is
perhaps unsurprising that co-ops constitute the form of ownership for many of
the city's most exclusive residential apartment buildings. Tom Wolfe famously
profiled these co-ops in the heady days of the 1980s bull market:
\begin{quote}
These so-called Good Buildings are forty-two cooperative apartment houses built
more than half a century ago. Thirty-seven of them are located in a small wedge
of Manhattan's Upper East Side known as the Triangle[,]\ldots an area defined by
Fifty-seventh Street from Sutton Place to Fifth Avenue on the south, Fifth
[Avenue] to Ninety-eighth Street on the west, and a diagonal back down to Sutton
on the east\ldots . The term Good Building was originally uttered sotto voce.
Before the First World War it was code for ``restricted to Protestants of
northern European stock''\ldots . Today Good certainly doesn't mean democratic,
but it does pertain to attributes that are at least more broadly available than
Protestant grandparents: namely, decorous demeanor, dignified behavior, business
and social connections, and sheer wealth. In short, bourgeois respectability.
The co-op boards want quiet, conservatively dressed families, although not with
too many children. Children tie up the elevators and make noise in the
lobby\ldots . The boards raise and lower their financial requirements, as well
as their social requirements, with the temperature of the market\ldots . The
first requirement is that the buyer be able to pay for the apartment in
cash\ldots . The second, in many buildings, is that he not be dependent on his
job or profession to pay for his monthly maintenance fees and keep up
appearances\ldots . The prospects and their families are also expected to drop
by the building for ``cocktails,'' which is an inspection of dress and
deportment\ldots . The stiffest known financial requirements are at a Good
Building on Park Avenue in the seventies, where the board asks that a purchaser
of an apartment demonstrate a net worth of at least \$30
million.\edfootnote{This would be over \$66 million in 2015 dollars.}
\end{quote}
Tom Wolfe, \textit{Proper Places}, \textsc{Esquire} (June 1985), at 194,
196-200.


