\expected{hidden-harbour-v-norman}

\item What is the difference between the standard applied by the trial judge and
that applied by the Court of Appeal in \textit{Norman}? Don't both merely
require rules promulgated by an association to be ``reasonable''?

\item The Hidden Harbour development was back before the Florida District Court
of Appeal six years later over a different dispute involving a resident's
private well. In \emph{Hidden Harbour Estates, Inc. v. Basso}, 393 So.2d 637
(Fla. Dist. Ct. App. 1981), the court opined:
\begin{quotation}
There are essentially two categories of cases in which a condominium association
attempts to enforce rules of restrictive uses. The first category is that
dealing with the validity of restrictions found in the declaration of
condominium itself. The second category of cases involves the validity of rules
promulgated by the association's board of directors or the refusal of the board
of directors to allow a particular use when the board is invested with the power
to grant or deny a particular use.

In the first category, the restrictions are clothed with a very strong
presumption of validity which arises from the fact that each individual unit
owner purchases his unit knowing of and accepting the restrictions to be
imposed. Such restrictions are very much in the nature of covenants running with
the land and they will not be invalidated absent a showing that they are wholly
arbitrary in their application, in violation of public policy, or that they
abrogate some fundamental constitutional right. Thus, although case law has
applied the word ``reasonable'' to determine whether such restrictions are
valid, this is not the appropriate test\ldots.

The rule to be applied in the second category of cases, however, is different.
In those cases where a use restriction is not mandated by the declaration of
condominium per se, but is instead created by the board of directors of the
condominium association, the rule of reasonableness comes into vogue. The
requirement of ``reasonableness'' in these instances is designed to somewhat
fetter the discretion of the board of directors. By imposing such a standard,
the board is required to enact rules and make decisions that are reasonably
related to the promotion of the health, happiness and peace of mind of the unit
owners. In cases like the present one where the decision to allow a particular
use is within the discretion of the board, the board must allow the use unless
the use is demonstrably antagonistic to the legitimate objectives of the
condominium association, i.e., the health, happiness and peace of mind of the
individual unit owners.
\end{quotation}
The Restatement draws the same distinction between the standard for validity of
covenants set forth in the \inline{ccrs} of a declaration and the standard for
validity of rules enacted by the governing body of a common-interest community.
Thus, restrictions in a condominium declaration are valid---even if
unreasonable---unless they are illegal, unconstitutional, or against public
policy, \textsc{Restatement} \S~3.1, while house rules and their enforcement are
subject to a reasonableness standard, \textsc{Restatement} \S~6.7 \& Reporter's
Note.

Does this distinction make sense? The court in \textit{Basso} notes that ``house
rules,'' unlike \inline{ccrs}, may be adopted \textit{after} a resident acquires
their property and thus without the notice that recording of the declaration
provides before a resident invests in the community.\footnote{Typically, either
under state law or by a declaration's own terms (or both), the \inline{ccrs} in
a declaration may only be amended by a supermajority vote of all members of the
association.} Does that distinction justify the diverging standards for
validity? Is such a justification consistent with the reasoning of
\textit{Norman}?


\item Not all jurisdictions follow the distinction drawn by \textit{Basso} and
the \textsc{Restatement}. Consider the following case. 
\expectnext{nahrstedt-v-lakeside}


