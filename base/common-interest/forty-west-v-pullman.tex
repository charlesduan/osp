\reading{40 West 67th Street v. Pullman}

\readingcite{790 N.E.2d 1174 (N.Y. 2003)}

\opinion \textsc{Rosenblatt}, J.

In \textit{Matter of Levandusky v. One Fifth Ave. Apt. Corp.}, 75 N.Y.2d 530,
554 N.Y.S.2d 807, 553 N.E.2d 1317 [1990] we held that the business judgment rule
is the proper standard of judicial review when evaluating decisions made by
residential cooperative corporations. In the case before us, defendant is a
shareholder-tenant in the plaintiff cooperative building. The relationship
between defendant and the cooperative, including the conditions under which a
shareholder's tenancy may be terminated, is governed by the shareholder's lease
agreement. The cooperative terminated defendant's tenancy in accordance with a
provision in the lease that authorized it to do so based on a tenant's
``objectionable'' conduct\ldots .

\readinghead{I.}

Plaintiff cooperative owns the building located at 40 West 67th Street in
Manhattan, which contains 38 apartments. In 1998, defendant bought into the
cooperative and acquired 80 shares of stock appurtenant to his proprietary lease
for apartment 7B.

Soon after moving in, defendant engaged in a course of behavior that, in the
view of the cooperative, began as demanding, grew increasingly disruptive and
ultimately became intolerable. After several points of friction between
defendant and the cooperative,\readingfootnote{1}{Initially, defendant sought
changes in the building services, such as the installation of video
surveillance, 24-hour door service and replacement of the lobby mailboxes. After
investigation, the Board deemed these proposed changes inadvisable or
infeasible.} defendant started complaining about his elderly upstairs neighbors,
a retired college professor and his wife who had occupied apartment 8B for over
two decades. In a stream of vituperative letters to the cooperative---16 letters
in the month of October 1999 alone---he accused the couple of playing their
television set and stereo at high volumes late into the night, and claimed they
were running a loud and illegal bookbinding business in their apartment.
Defendant further charged that the couple stored toxic chemicals in their
apartment for use in their ``dangerous and illegal'' business. Upon
investigation, the cooperative's Board determined that the couple did not
possess a television set or stereo and that there was no evidence of a
bookbinding business or any other commercial enterprise in their apartment.

Hostilities escalated, resulting in a physical altercation between defendant and
the retired professor.\readingfootnote{2}{Defendant brought charges against the
professor which resulted in the professor's arrest. Eventually, the charges were
adjourned in contemplation of dismissal.} Following the altercation, defendant
distributed flyers to the cooperative residents in which he referred to the
professor, by name, as a potential ``psychopath in our midst'' and accused him
of cutting defendant's telephone lines. In another flyer, defendant described
the professor's wife and the wife of the Board president as having close
``intimate personal relations.'' Defendant also claimed that the previous
occupants of his apartment revealed that the upstairs couple have ``historically
made excessive noise.'' The former occupants, however, submitted an affidavit
that denied making any complaints about noise from the upstairs apartment and
proclaimed that defendant's assertions to the contrary were ``completely
false.''

Furthermore, defendant made alterations to his apartment without Board approval,
had construction work performed on the weekend in violation of house rules, and
would not respond to Board requests to correct these conditions or to allow a
mutual inspection of his apartment and the upstairs apartment belonging to the
elderly couple. Finally, defendant commenced four lawsuits against the upstairs
couple, the president of the cooperative and the cooperative management, and
tried to commence three more.

In reaction to defendant's behavior, the cooperative called a special meeting
pursuant to article III (First) (f) of the lease agreement, which provides for
termination of the tenancy if the cooperative by a two-thirds vote determines
that ``because of objectionable conduct on the part of the Lessee\ldots the
tenancy of the Lessee is undesirable.''\readingfootnote{3}{The full provision
authorizes termination ``if at any time the Lessor shall determine, upon the
affirmative vote of the holders of record of at least two-thirds of that part of
its capital stock which is then owned by Lessees under proprietary leases then
in force, at a meeting of such stockholders duly called to take action on the
subject, that because of objectionable conduct on the part of the Lessee, or of
a person dwelling in or visiting the apartment, the tenancy of the Lessee is
undesirable.''} The cooperative informed the shareholders that the purpose of
the meeting was to determine whether defendant ``engaged in repeated actions
inimical to cooperative living and objectionable to the Corporation and its
stockholders that make his continued tenancy undesirable.''

Timely notice of the meeting was sent to all shareholders in the cooperative,
including defendant. At the ensuing meeting, held in June 2000, owners of more
than 75\% of the outstanding shares in the cooperative were present. Defendant
chose not attend. By a vote of 2,048 shares to 0, the shareholders in attendance
passed a resolution declaring defendant's conduct ``objectionable'' and
directing the Board to terminate his proprietary lease and cancel his shares.
The resolution contained the findings upon which the shareholders concluded that
defendant's behavior was inimical to cooperative living. Pursuant to the
resolution, the Board sent defendant a notice of termination requiring him to
vacate his apartment by August 31, 2000. Ignoring the notice, defendant remained
in the apartment, prompting the cooperative to bring this suit for possession
and ejectment, a declaratory judgment cancelling defendant's stock, and a money
judgment for use and occupancy, along with attorneys' fees and costs\ldots .


\readinghead{II. The \emph{Levandusky} Business Judgment Rule}

The heart of this dispute is the parties' disagreement over the proper standard
of review to be applied when a cooperative exercises its agreed-upon right to
terminate a tenancy based on a shareholder-tenant's objectionable conduct. In
the agreement establishing the rights and duties of the parties, the cooperative
reserved to itself the authority to determine whether a member's conduct was
objectionable and to terminate the tenancy on that basis. The cooperative argues
that its decision to do so should be reviewed in accordance with
\textit{Levandusky}'s business judgment rule. Defendant contends that the
business judgment rule has no application under these circumstances and that
RPAPL 711 requires a court to make its own evaluation of the Board's conduct
based on a judicial standard of reasonableness.

\textit{Levandusky} established a standard of review analogous to the corporate
business judgment rule for a shareholder-tenant challenge to a decision of a
residential cooperative corporation. The business judgment rule is a common-law
doctrine by which courts exercise restraint and defer to good faith decisions
made by boards of directors in business settings. The rule has been long
recognized in New York. In \textit{Levandusky}, the cooperative board issued a
stop work order for a shareholder-tenant's renovations that violated the
proprietary lease. The shareholder-tenant brought a CPLR article 78 proceeding
to set aside the stop work order. The Court upheld the Board's action, and
concluded that the business judgment rule ``best balances the individual and
collective interests at stake'' in the residential cooperative setting
(\textit{Levandusky}, 75 N.Y.2d at 537, 554 N.Y.S.2d 807, 553 N.E.2d 1317).

In the context of cooperative dwellings, the business judgment rule provides
that a court should defer to a cooperative board's determination ``[s]o long as
the board acts for the purposes of the cooperative, within the scope of its
authority and in good faith'' (\textit{id}. at 538, 554 N.Y.S.2d 807, 553 N.E.2d
1317). In adopting this rule, we recognized that a cooperative board's broad
powers could lead to abuse through arbitrary or malicious decisionmaking,
unlawful discrimination or the like. However, we also aimed to avoid impairing
``the purposes for which the residential community and its governing structure
were formed: protection of the interest of the entire community of residents in
an environment managed by the board for the common benefit'' (\textit{id.} at
537, 554 N.Y.S.2d 807, 553 N.E.2d 1317). The Court concluded that the business
judgment rule best balances these competing interests and also noted that the
limited judicial review afforded by the rule protects the cooperative's
decisions against ``undue court involvement and judicial second-guessing''
(\textit{id.} at 540, 554 N.Y.S.2d 807, 553 N.E.2d 1317).

Although we applied the business judgment rule in \textit{Levandusky}, we did
not attempt to fix its boundaries, recognizing that this corporate concept may
not necessarily comport with every situation encountered by a cooperative and
its shareholder-tenants. Defendant argues that when it comes to terminations,
the business judgment rule conflicts with RPAPL 711(1) and is therefore
inoperative.\readingfootnote{5}{RPAPL 711(1), in pertinent part, states: ``A
proceeding seeking to recover possession of real property by reason of the
termination of the term fixed in the lease pursuant to a provision contained
therein giving the landlord the right to terminate the time fixed for occupancy
under such agreement if he deem the tenant objectionable, shall not be
maintainable unless the landlord shall by competent evidence establish to the
satisfaction of the court that the tenant is objectionable.''} We see no such
conflict. In the realm of cooperative governance and in the lease provision
before us, the cooperative's determination as to the tenant's objectionable
behavior stands as competent evidence necessary to sustain the cooperative's
determination. If that were not so, the contract provision for termination of
the lease-to which defendant agreed-would be meaningless.

We reject the cooperative's argument that RPAPL 711(1) is irrelevant to these
proceedings, but conclude that the business judgment rule may be applied
consistently with the statute. Procedurally, the business judgment standard will
be applied across the cases, but the manner in which it presents itself varies
with the form of the lawsuit. \textit{Levandusky}, for example, was framed as a
CPLR article 78 proceeding, but we applied the business judgment rule as a
concurrent form of ``rationality'' and ``reasonableness'' to determine whether
the decision was ``arbitrary and capricious'' pursuant to CPLR 7803(3).

Similarly, the procedural vehicle driving this case is RPAPL 711(1), which
requires ``competent evidence'' to show that a tenant is objectionable. Thus, in
this context, the competent evidence that is the basis for the shareholder vote
will be reviewed under the business judgment rule, which means courts will
normally defer to that vote and the shareholders' stated findings as competent
evidence that the tenant is indeed objectionable under the statute. As we stated
in \textit{Levandusky}, a single standard of review for cooperatives is
preferable, and ``we see no purpose in allowing the form of the action to
dictate the substance of the standard by which the legitimacy of corporate
action is to be measured'' (\emph{id.} at 541, 554 N.Y.S.2d 807, 553 N.E.2d
1317).

Despite this deferential standard, there are instances when courts should
undertake review of board decisions. To trigger further judicial scrutiny, an
aggrieved shareholder-tenant must make a showing that the board acted (1)
outside the scope of its authority, (2) in a way that did not legitimately
further the corporate purpose or (3) in bad faith.

\readinghead{III.}

\readinghead{A. The Cooperative's Scope of Authority}

Pursuant to its bylaws, the cooperative was authorized (through its Board) to
adopt a form of proprietary lease to be used for all shareholder-tenants. Based
on this authorization, defendant and other members of the cooperative
voluntarily entered into lease agreements containing the termination provision
before us. The cooperative does not contend that it has the power to terminate
the lease absent the termination provision. Indeed, it recognizes, correctly,
that if there were no such provision, termination could proceed only pursuant to
RPAPL 711(1).

The cooperative unfailingly followed the procedures contained in the lease when
acting to terminate defendant's tenancy. In accordance with the bylaws, the
Board called a special meeting, and notified all shareholder-tenants of its
time, place and purpose. Defendant thus had notice and the opportunity to be
heard. In accordance with the agreement, the cooperative acted on a
supermajority vote after properly fashioning the issue and the question to be
addressed by resolution. The resolution specified the basis for the action,
setting forth a list of specific findings as to defendant's objectionable
behavior. By not appearing or presenting evidence personally or by counsel,
defendant failed to challenge the findings and has not otherwise satisfied us
that the Board has in any way acted ultra vires. In all, defendant has failed to
demonstrate that the cooperative acted outside the scope of its authority in
terminating the tenancy.


\readinghead{B. Furthering the Corporate Purpose}

\textit{Levandusky} also recognizes that the business judgment rule prohibits
judicial inquiry into Board actions that, presupposing good faith, are taken in
legitimate furtherance of corporate purposes. Specifically, there must be a
legitimate relationship between the Board's action and the welfare of the
cooperative. Here, by the unanimous vote of everyone present at the meeting, the
cooperative resoundingly expressed its collective will, directing the Board to
terminate defendant's tenancy after finding that his behavior was more than its
shareholders could bear. The Board was under a fiduciary duty to further the
collective interests of the cooperative. By terminating the tenancy, the Board's
action thus bore an obvious and legitimate relation to the cooperative's avowed
ends.

There is, however, an additional dimension to corporate purpose that
\textit{Levandusky} contemplates, notably, the legitimacy of purpose---a feature
closely related to good faith. Put differently, all the shareholders of a
cooperative may agree on an objective, and the Board may pursue that objective
zealously, but that does not necessarily mean the objective is lawful or
legitimate. Defendant, however, has not shown that the Board's purpose was
anything other than furthering the over-all welfare of a cooperative that found
it could no longer abide defendant's behavior. 


\readinghead{C. Good Faith, in the Exercise of Honest Judgment}

Finally, defendant has not shown the slightest indication of any bad faith,
arbitrariness, favoritism, discrimination or malice on the cooperative's part,
and the record reveals none. Though defendant contends that he raised sufficient
facts in this regard, we agree with the Appellate Division majority that
defendant has provided no factual support for his conclusory assertions that he
was evicted based upon illegal or impermissible considerations. Moreover, as the
Appellate Division noted, the cooperative emphasized that upon the sale of the
apartment it ``will `turn over [to the defendant] all proceeds after deduction
of unpaid use and occupancy, costs of sale and litigation expenses incurred in
this dispute.'\,'' Defendant does not contend otherwise.

\textit{Levandusky} cautions that the broad powers of cooperative governance
carry the potential for abuse when a board singles out a person for harmful
treatment or engages in unlawful discrimination, vendetta, arbitrary
decisionmaking or favoritism. We reaffirm that admonition and stress that those
types of abuses are incompatible with good faith and the exercise of honest
judgment. While deferential, the \textit{Levandusky} standard should not serve
as a rubber stamp for cooperative board actions, particularly those involving
tenancy terminations. We note that since \textit{Levandusky} was decided, the
lower courts have in most instances deferred to the business judgment of
cooperative boards but in a number of cases have withheld deference in the face
of evidence that the board acted illegitimately.\readingfootnote{8}{\textit{See
e.g. Abrons Found. v. 29 E. 64th St. Corp.}, 297 A.D.2d 258, 746 N.Y.S.2d 482
[1st Dept.2002] [tenant raised genuine issues of material fact as to whether
board acted in bad faith in imposing sublet fee meant solely to impact one
tenant]; \textit{Greenberg v. Board of Mgrs. of Parkridge Condominiums}, 294
A.D.2d 467, 742 N.Y.S.2d 560 [2d Dept.2002] [affirming injunction against board
because it acted outside scope of authority in prohibiting tenant from erecting
a succah on balcony]; \textit{Dinicu v. Groff Studios Corp.}, 257 A.D.2d 218,
690 N.Y.S.2d 220 [1st Dept.1999] [business judgment rule does not protect
cooperative board from its own breach of contract]; \textit{Matter of Vacca v
Board of Mgrs. of Primrose Lane Condominium}, 251 A.D.2d 674, 676 N.Y.S.2d 188
[2d Dept. 1998] [board acted in bad faith in prohibiting tenant from displaying
religious statue in yard]; \textit{Johar v. 82-04 Lefferts Tenants Corp.}, 234
A.D.2d 516, 651 N.Y.S.2d 914 [2d Dept. 1996] [board vote amending bylaws to
declare plaintiff tenant ineligible to sit on cooperative board not shielded by
business judgment rule]. While we do not undertake to address the correctness of
the rulings in all of these cases, we list them as illustrative.}

The very concept of cooperative living entails a voluntary, shared control over
rules, maintenance and the composition of the community. Indeed, as we observed
in \textit{Levandusky}, a shareholder-tenant voluntarily agrees to submit to the
authority of a cooperative board, and consequently the board ``may significantly
restrict the bundle of rights a property owner normally enjoys'' (75 N.Y.2d at
536, 554 N.Y.S.2d 807, 553 N.E.2d 1317). When dealing, however, with
termination, courts must exercise a heightened vigilance in examining whether
the board's action meets the \textit{Levandusky} test\ldots .

