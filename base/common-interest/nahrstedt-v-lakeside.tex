\reading{Nahrstedt v. Lakeside Village Condominium Ass'n, Inc.}

\readingcite{878 P.2d 1275 (Cal. 1994)}

\opinion \textsc{Kennard}, Justice.

A homeowner in a 530-unit condominium complex sued to prevent the homeowners
association from enforcing a restriction against keeping cats, dogs, and other
animals in the condominium development. The owner asserted that the restriction,
which was contained in the project's declaration recorded by the condominium
project's developer, was ``unreasonable'' as applied to her because she kept her
three cats indoors and because her cats were ``noiseless'' and ``created no
nuisance.'' Agreeing with the premise underlying the owner's complaint, the
Court of Appeal concluded that the homeowners association could enforce the
restriction only upon proof that plaintiff's cats would be likely to interfere
with the right of other homeowners ``to the peaceful and quiet enjoyment of
their property.''

Those of us who have cats or dogs can attest to their wonderful companionship
and affection. Not surprisingly, studies have confirmed this effect\ldots . But
the issue before us is not whether in the abstract pets can have a beneficial
effect on humans. Rather, the narrow issue here is whether a pet restriction
that is contained in the recorded declaration of a condominium complex is
enforceable against the challenge of a homeowner. As we shall explain, the
Legislature, in Civil Code section 1354, has required that courts enforce the
covenants, conditions and restrictions contained in the recorded declaration of
a common interest development ``unless unreasonable.''

Because a stable and predictable living environment is crucial to the success of
condominiums and other common interest residential developments, and because
recorded use restrictions are a primary means of ensuring this stability and
predictability, the Legislature in section 1354 has afforded such restrictions a
presumption of validity and has required of challengers that they demonstrate
the restriction's ``unreasonableness'' by the deferential standard applicable to
equitable servitudes. Under this standard established by the Legislature,
enforcement of a restriction does not depend upon the conduct of a particular
condominium owner. Rather, the restriction must be uniformly enforced in the
condominium development to which it was intended to apply unless the plaintiff
owner can show that the burdens it imposes on affected properties so
substantially outweigh the benefits of the restriction that it should not be
enforced against any owner. Here, the Court of Appeal did not apply this
standard in deciding that plaintiff had stated a claim for declaratory relief.
Accordingly, we reverse the judgment of the Court of Appeal and remand for
further proceedings consistent with the views expressed in this opinion.

\readinghead{I}

Lakeside Village is a large condominium development in Culver City, Los Angeles
County. It consists of 530 units spread throughout 12 separate 3-story
buildings. The residents share common lobbies and hallways, in addition to
laundry and trash facilities.

The Lakeside Village project is subject to certain covenants, conditions and
restrictions (hereafter CC \& R's) that were included in the developer's
declaration recorded with the Los Angeles County Recorder on April 17, 1978, at
the inception of the development project. Ownership of a unit includes
membership in the project's homeowners association, the Lakeside Village
Condominium Association (hereafter Association), the body that enforces the
project's CC \& R's, including the pet restriction, which provides in relevant
part: ``No animals (which shall mean dogs and cats), livestock, reptiles or
poultry shall be kept in any unit.''\readingfootnote{3}{The CC \& R's permit
residents to keep ``domestic fish and birds.''}

In January 1988, plaintiff Natore Nahrstedt purchased a Lakeside Village
condominium and moved in with her three cats. When the Association learned of
the cats' presence, it demanded their removal and assessed fines against
Nahrstedt for each successive month that she remained in violation of the
condominium project's pet restriction.

Nahrstedt then brought this lawsuit against the Association, its officers, and
two of its employees, asking the trial court to invalidate the assessments, to
enjoin future assessments, to award damages for violation of her privacy when
the Association ``peered'' into her condominium unit, to award damages for
infliction of emotional distress, and to declare the pet restriction
``unreasonable'' as applied to indoor cats (such as hers) that are not allowed
free run of the project's common areas. Nahrstedt also alleged she did not know
of the pet restriction when she bought her condominium.\ldots 

The Association demurred to the complaint. In its supporting points and
authorities, the Association argued that the pet restriction furthers the
collective ``health, happiness and peace of mind'' of persons living in close
proximity within the Lakeside Village condominium development, and therefore is
reasonable as a matter of law. The trial court sustained the demurrer as to each
cause of action and dismissed Nahrstedt's complaint. Nahrstedt appealed.

A divided Court of Appeal reversed the trial court's judgment of dismissal\ldots
. On the Association's petition, we granted review to decide when a condominium
owner can prevent enforcement of a use restriction that the project's developer
has included in the recorded declaration of CC \& R's\ldots .

\readinghead{II}

Today, condominiums, cooperatives, and planned-unit developments with homeowners
associations have become a widely accepted form of real property ownership.
These ownership arrangements are known as ``common interest''
developments.\ldots Use restrictions are an inherent part of any common interest
development and are crucial to the stable, planned environment of any shared
ownership arrangement\ldots . The restrictions on the use of property in any
common interest development may limit activities conducted in the common areas
as well as in the confines of the home itself. Commonly, use restrictions
preclude alteration of building exteriors, limit the number of persons that can
occupy each unit, and place limitations on---or prohibit altogether---the
keeping of pets.

Restrictions on property use are not the only characteristic of common interest
ownership. Ordinarily, such ownership also entails mandatory membership in an
owners association, which, through an elected board of directors, is empowered
to enforce any use restrictions contained in the project's declaration or master
deed and to enact new rules governing the use and occupancy of property within
the project. Because of its considerable power in managing and regulating a
common interest development, the governing board of an owners association must
guard against the potential for the abuse of that power. As Professor Natelson
observes, owners associations ``can be a powerful force for good or for ill'' in
their members' lives. Therefore, anyone who buys a unit in a common interest
development with knowledge of its owners association's discretionary power
accepts ``the risk that the power may be used in a way that benefits the
commonality but harms the individual.'' Generally, courts will uphold decisions
made by the governing board of an owners association so long as they represent
good faith efforts to further the purposes of the common interest development,
are consistent with the development's governing documents, and comply with
public policy.

Thus, subordination of individual property rights to the collective judgment of
the owners association together with restrictions on the use of real property
comprise the chief attributes of owning property in a common interest
development.\ldots

Notwithstanding the limitations on personal autonomy that are inherent in the
concept of shared ownership of residential property, common interest
developments have increased in popularity in recent years, in part because they
generally provide a more affordable alternative to ownership of a single-family
home\ldots .

\ldots When restrictions limiting the use of property within a common interest
development satisfy the requirements of covenants running with the land or of
equitable servitudes, what standard or test governs their enforceability? In
California, as we explained at the outset, our Legislature has made common
interest development use restrictions contained in a project's recorded
declaration ``enforceable\ldots \textit{unless unreasonable.}'' (\S~1354, subd.
(a), italics added.)\ldots In other words, such restrictions should be enforced
unless they are wholly arbitrary, violate a fundamental public policy, or impose
a burden on the use of affected land that far outweighs any benefit.

This interpretation of section 1354 is consistent with the views of legal
commentators as well as judicial decisions in other jurisdictions that have
applied a presumption of validity to the recorded land use restrictions of a
common interest development. As these authorities point out, and as we discussed
previously, recorded CC \& R's are the primary means of achieving the stability
and predictability so essential to the success of a shared ownership housing
development.\ldots When courts accord a presumption of validity to all such
recorded use restrictions and measure them against deferential standards of
equitable servitude law, it discourages lawsuits by owners of individual units
seeking personal exemptions from the restrictions. This also promotes stability
and predictability in two ways. It provides substantial assurance to prospective
condominium purchasers that they may rely with confidence on the promises
embodied in the project's recorded CC \& R's. And it protects all owners in the
planned development from unanticipated increases in association fees to fund the
defense of legal challenges to recorded restrictions.

How courts enforce recorded use restrictions affects not only those who have
made their homes in planned developments, but also the owners associations
charged with the fiduciary obligation to enforce those restrictions. When courts
treat recorded use restrictions as presumptively valid, and place on the
challenger the burden of proving the restriction ``unreasonable'' under the
deferential standards applicable to equitable servitudes, associations can
proceed to enforce reasonable restrictive covenants without fear that their
actions will embroil them in costly and prolonged legal proceedings. Of course,
when an association determines that a unit owner has violated a use restriction,
the association must do so in good faith, not in an arbitrary or capricious
manner, and its enforcement procedures must be fair and applied uniformly.

There is an additional beneficiary of legal rules that are protective of
recorded use restrictions: the judicial system. Fewer lawsuits challenging such
restrictions will be brought, and those that are filed may be disposed of more
expeditiously, if the rules courts use in evaluating such restrictions are
clear, simple, and not subject to exceptions based on the peculiar circumstances
or hardships of individual residents in condominiums and other shared-ownership
developments.

\ldots Refusing to enforce the CC \& R's contained in a recorded declaration, or
enforcing them only after protracted litigation that would require justification
of their application on a case-by-case basis, would impose great strain on the
social fabric of the common interest development. It would frustrate owners who
had purchased their units in reliance on the CC \& R's. It would put the owners
and the homeowners association in the difficult and divisive position of
deciding whether particular CC \& R's should be applied to a particular owner.
Here, for example, deciding whether a particular animal is ``confined to an
owner's unit and create[s] no noise, odor, or nuisance'' is a fact-intensive
determination that can only be made by examining in detail the behavior of the
particular animal and the behavior of the particular owner. Homeowners
associations are ill-equipped to make such investigations, and any decision they
might make in a particular case could be divisive or subject to claims of
partiality.

Enforcing the CC \& R's contained in a recorded declaration only after
protracted case-by-case litigation would impose substantial litigation costs on
the owners through their homeowners association, which would have to defend not
only against owners contesting the application of the CC \& R's to them, but
also against owners contesting any case-by-case exceptions the homeowners
association might make. In short, it is difficult to imagine what could more
disrupt the harmony of a common interest development\ldots .

Under the holding we adopt today, the reasonableness or unreasonableness of a
condominium use restriction that the Legislature has made subject to section
1354 is to be determined \textit{not} by reference to facts that are specific to
the objecting homeowner, but by reference to the common interest development as
a whole. As we have explained, when, as here, a restriction is contained in the
declaration of the common interest development and is recorded with the county
recorder, the restriction is presumed to be reasonable and will be enforced
uniformly against all residents of the common interest development
\textit{unless} the restriction is arbitrary, imposes burdens on the use of
lands it affects that substantially outweigh the restriction's benefits to the
development's residents, or violates a fundamental public policy.

Accordingly, here Nahrstedt could prevent enforcement of the Lakeside Village
pet restriction by proving that the restriction is arbitrary, that it is
substantially more burdensome than beneficial to the affected properties, or
that it violates a fundamental public policy. For the reasons set forth below,
Nahrstedt's complaint fails to adequately allege any of these three grounds of
unreasonableness.

We conclude, as a matter of law, that the recorded pet restriction of the
Lakeside Village condominium development prohibiting cats or dogs but allowing
some other pets is not arbitrary, but is rationally related to health,
sanitation and noise concerns legitimately held by residents of a high-density
condominium project such as Lakeside Village, which includes 530 units in 12
separate 3-story buildings.

Nahrstedt's complaint alleges no facts that could possibly support a finding
that the burden of the restriction on the affected property is so
disproportionate to its benefit that the restriction is unreasonable and should
not be enforced. Also, the complaint's allegations center on Nahrstedt and her
cats (that she keeps them inside her condominium unit and that they do not
bother her neighbors), without any reference to the effect on the condominium
development as a whole, thus rendering the allegations legally insufficient to
overcome section 1354's presumption of the restriction's validity\ldots .

\textsc{Lucas}, C.J., and \textsc{Mosk, Baxter, George} and \textsc{Werdegar},
JJ., concur.

\opinion \textsc{Arabian}, Justice, dissenting.

``There are two means of refuge from the misery of life: music and
cats.''\readingfootnote{1}{Albert Schweitzer.}

I respectfully dissent. While technical merit may commend the majority's
analysis, its application to the facts presented reflects a narrow, indeed
chary, view of the law that eschews the human spirit in favor of arbitrary
efficiency. In my view, the resolution of this case well illustrates the
conventional wisdom, and fundamental truth, of the Spanish proverb, ``It is
better to be a mouse in a cat's mouth than a man in a lawyer's hands.''

As explained below, I find the provision known as the ``pet restriction''
contained in the covenants, conditions, and restrictions (CC \& R's) governing
the Lakeside Village project patently arbitrary and unreasonable within the
meaning of Civil Code section 1354. Beyond dispute, human beings have long
enjoyed an abiding and cherished association with their household animals. Given
the substantial benefits derived from pet ownership, the undue burden on the use
of property imposed on condominium owners who can maintain pets within the
confines of their units without creating a nuisance or disturbing the quiet
enjoyment of others substantially outweighs whatever meager utility the
restriction may serve in the abstract. It certainly does not promote ``health,
happiness [or] peace of mind'' commensurate with its tariff on the quality of
life for those who value the companionship of animals. Worse, it contributes to
the fraying of our social fabric.

\ldots Generically stated, plaintiff challenges this restriction to the extent
it precludes not only her but anyone else living in Lakeside Village from
enjoying the substantial pleasures of pet ownership while affording no
discernible benefit to other unit owners if the animals are maintained without
any detriment to the latter's quiet enjoyment of their own space and the common
areas. In essence, she avers that when pets are kept out of sight, do not make
noise, do not generate odors, and do not otherwise create a nuisance, reasonable
expectations as to the quality of life within the condominium project are not
impaired. At the same time, taking into consideration the well-established and
long-standing historical and cultural relationship between human beings and
their pets and the value they impart[,] enforcement of the restriction
significantly and unduly burdens the use of land for those deprived of their
companionship. Considered from this perspective, I find plaintiff's complaint
states a cause of action for declaratory relief.

\ldots Our true task in this turmoil is to strike a balance
between the governing rights accorded a condominium association and the
individual freedom of its members\ldots . Pet ownership substantially enhances
the quality of life for those who desire it. When others are not only
undisturbed by, but \textit{completely unaware of}, the presence of pets being
enjoyed by their neighbors, the balance of benefit and burden is rendered
disproportionate and unreasonable, rebutting any presumption of validity\ldots .

I would affirm the judgment of the Court of Appeal.

