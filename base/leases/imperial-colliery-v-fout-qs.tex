\expected{imperial-colliery-v-fout}

\item \textbf{The basic law.} In states that recognize \term{retaliatory
eviction}, a
landlord may not punish tenants when they exercise legal rights incidental to
their tenancy.  Generally, this means that a landlord cannot raise the rent,
reduce services, refuse to renew a lease, or bring an eviction action for the
purpose of retaliating against a tenant who has complained about the condition
of the unit, filed a lawsuit concerning the fitness of the unit, contacted a
local agency, or exercised rights under the implied warranty of habitability.  


\item \textbf{Legal change.} Under the traditional English common law, a
landlord could raise the rent or refuse to renew a tenant's lease for any
reason.  How does the court in \textit{Imperial Colliery} justify changing a
long-settled rule?


\item \textbf{Rise of the doctrine.} The doctrine of retaliatory eviction came
to prominence around the same time as the implied warranty of habitability. 
What's the link between these two rules?  


\item \textbf{Retaliate for what?} West Virginia, like most states, protects
tenants from retaliatory eviction. In the case above, Fout presented evidence
that he lost his tenancy as a result of retaliation by his landlord.  Why then
did Fout lose?  Do you agree with the limitations that West Virginia has put on
the doctrine of retaliatory eviction?  Why should tenants fear losing their
homes if they exercise their First Amendment rights?

\expected{marsh-v-alabama}
\expected{state-v-shack}

\item \textbf{Property serves human values?} Recall the \textit{Marsh} case
(company owned town cannot prevent distribution of pamphlets on sidewalk) and
the \textit{Shack} case (property owners cannot bar social service workers from
meeting with migrant laborers) from earlier in the semester.  In those opinions
we saw that property rights are occasionally trumped by other values.  Why don't
Fout's rights under the First Amendment and the National Labor Relations Act
outweigh his landlord's desire to kick him out?  Can you distinguish
\textit{Imperial Colliery} from \textit{Marsh} and \textit{Shack}?  


\item \textbf{Is housing special?} Is housing a good like any other, or is it
somehow different from most things we buy and sell on the market? In
continental European countries there's a tentative national consensus that all
housing---even privately owned apartments---has a uniquely public or social
dimension.  As a result, many European nations grant citizens strong
protections against forced relocations.  For example, ``good faith'' eviction
schemes are pervasive.  In a ``good faith'' jurisdiction, a landlord can only
refuse to renew a tenancy for a good reason---generally some faulty behavior on
the part of the tenant (damaging the premise, creating a nuisance, breaching a
material term in the lease) or the landlord's desire to remodel the unit. 
Should U.S. states adopt such a rule?


\item \textbf{Remedies.} What's the appropriate remedy for a tenant who wins a
retaliatory eviction case?


\item \textbf{Establishing motive.} Peter Pan calls his local Board of Health to
complain about the conditions in The Neverland Apartments, where he rents a
two-bedroom unit. The landlord, Hook, is furious at Pan.  They get into a
heated screaming match in front of the building.  If Hook waits a year and then
dramatically raises Pan's rent, will Pan be able to win a retaliatory eviction
case?  What if Hook waits six months? Three months? Some states require the
tenant to show that the landlord would not have taken action ``but for'' the
tenant exercising a right.  Because of the difficulties in establishing motive,
other states employ a burden-shifting model in retaliatory eviction cases. In
these jurisdictions, the law presumes that the landlord has acted with a
retaliatory motive if the landlord raises the rent (or takes another
retaliatory action) within a certain amount of time after the tenant has
availed himslef of a legal entitlement. The window of time varies from three
months to a year, but many states use a six-month period.  Importantly, the
presumption against the landlord is rebuttable.


\item \textbf{How common is retaliation?} In his book, \textit{Evicted:
Poverty and Profit in the American City}, Matthew Desmond recounts an anecdote
about a landlord who would immediately begin preparing eviction papers as soon
as his tenants complained about their living conditions.

