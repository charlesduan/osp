\reading{Effel v. Rosberg}
\readingcite{360 S.W.3d 626 (Tex. App. 2012)}

\textsc{MORRIS}, Justice.

This is an appeal from the trial court's judgment awarding Robert G. Rosberg
possession of property in a forcible detainer action. Appellant Lena Effel
brings seventeen issues generally contending the trial court\ldots  erred in
concluding Rosberg was entitled to possession of the property. After examining
the record on appeal and reviewing the applicable law, we conclude appellant's
arguments are without merit. We affirm the trial court's judgment.

\readinghead{I.}

[On March 1, 2006, Robert G. Rosberg filed suit against Lena Effel's nephews,
Henry Effel and Jack Effel. The parties settled the dispute out of court and
signed a compromise settlement agreement. As part of the settlement, Rosberg
received a piece of land owned by Henry and Jack Effel.  The property contained
the home where Lena Effel lived. The settlement agreement between the Effels
and Rosberg stated that Lena Effel] ``shall continue to occupy the property for
the remainder of her natural life, or until such time as she voluntarily
chooses to vacate the premises.'' The settlement agreement further stated that
a lease agreement incorporating the terms of the settlement agreement would be
prepared before the closing date of the purchase.\ldots

The property in question was deeded to Rosberg with no reservation of a life
estate. A lease for appellant was prepared by the Effels' attorney. The term of
the lease was ``for a term equal to the remainder of the Lessee's life, or
until such time that she voluntarily vacates the premises.'' The lease also
contained various covenants relating to payment of rent and charges for
utilities as well as the use and maintenance of the grounds. The lease provided
that if there was any default in the payment of rent or in the performance of
any of the covenants, the lease could be terminated at the option of the
lessor. The lease was signed by Rosberg as lessor and by Henry Effel on behalf
of appellant under a power of attorney as lessee.

Three years later, on February 24, 2010, Rosberg, through his attorney, sent a
letter to appellant both by regular mail and certified mail stating that he was
terminating her lease effective immediately. The reason for the termination,
according to the letter, was Rosberg's discovery that appellant had installed a
wrought iron fence in the front yard of the property in violation of two
covenants of the lease. The letter stated that appellant was required to leave
and surrender the premises within ten days and, if she did not vacate the
premises, Rosberg would commence eviction proceedings. Appellant did not vacate
the property.

On April 29, 2010, Rosberg filed this forcible detainer action in the justice
court. The justice court awarded possession of the property to Rosberg, and
appellant appealed the decision to the county court at law. The county court
held a trial de novo without a jury and, again, awarded the property to
Rosberg. The court concluded the lease created a tenancy at will terminable at
any time by either party. The court further concluded that Rosberg was
authorized to terminate the lease, whether because it was terminable at will or
because appellant violated the terms of the lease, and the lease was properly
terminated on February 24, 2010. Appellant now appeals the county court's
judgment.

\readinghead{II.}

\ldots In appellant's remaining issues, she challenges the findings of fact and
conclusions of law made by the county court. In her tenth issue, appellant
challenges the county court's first conclusion of law in which it stated
``[t]he lease, which purported to be for the rest of Lena Effel's life, created
only a tenancy at will terminable at any time by either party.'' Appellant
argues that the lease must be read together with the settlement agreement and
the court must give effect to the intent of the parties. Appellant was not a
party to the settlement agreement, however. Appellant was a party only to the
lease. It is the lease, and not the settlement agreement, that forms the basis
of this forcible detainer action. Accordingly, we look solely to the lease to
determine appellant's rights in this matter.

The lease states that appellant was a lessee of the property ``for a term equal
to the remainder of Lessee's life, or until such time as she voluntarily
vacates the premises.'' It is the long-standing rule in Texas that a lease must
be for a certain period of time or it will be considered a tenancy at will.
\textit{See Holcombe v. Lorino}, 124 Tex. 446, 79 S.W.2d 307, 310 (1935).
Courts that have applied this rule to leases that state they are for the term
of the lessee's life have concluded that the uncertainty of the date of the
lessee's death rendered the lease terminable at will by either party.

Appellant argues the current trend in court decisions is away from finding a
lease such as hers to be terminable at will. Appellant relies on the 1982
decision of \textit{Philpot v. Fields}, 633 S.W.2d 546 (Tex. App. 1982). In
\textit{Philpot}, the court stated that the trend in law was away from
requiring a lease to be of a definite and certain duration. In reviewing the
law since \textit{Philpot}, however, we discern no such trend. \textit{See Kajo
Church Square, Inc. v. Walker}, 2003 WL 1848555, at *5 (Tex. App. 2003). The
rule continues to be that a lease for an indefinite and uncertain length of
time is an estate at will. \textit{See Providence Land Servs., L.L.C. v.
Jones}, 353 S.W.3d 538, 542 (Tex. App. 2011). In this case, not only was the
term of the lease stated to be for the uncertain length of appellant's life,
but her tenancy was also ``until such time that she voluntarily vacates the
premises.'' If a lease can be terminated at the will of the lessee, it may also
be terminated at the will of the lessor. Because the lease at issue was
terminable at will by either party, the trial court's first conclusion of law
was correct. We resolve appellant's tenth issue against her.

In her fourth issue, appellant contends the trial court erred in concluding that
Rosberg sent her a proper notice to vacate the premises under section 24.005 of
the Texas Property Code. Section 24.005 states that a landlord must give a
tenant at will at least three days' written notice to vacate before filing a
forcible detainer suit unless the parties contracted for a longer or shorter
notice period in a written lease or agreement. \textsc{Tex. Prop. Code Ann}.
\S~24.005(b) (West Supp. 2011). The section also states that the notice must
be delivered either in person or by mail at the premises in question. Id.
\S~24.005(f). If the notice is delivered by mail, it may be by regular mail,
registered mail, or certified mail, return receipt requested, to the premises
in question. 

The undisputed evidence in this case shows that Rosberg, through his attorney,
sent appellant a written notice to vacate the premises by both regular mail and
certified mail on February 24, 2010. The notice stated that appellant had ten
days to surrender the premises. Nothing in the lease provided for a longer
notice period. Henry Effel testified at trial that appellant received the
notice and read it. Rosberg did not bring this forcible detainer action until
April 29, 2010. The evidence conclusively shows, therefore, that Rosberg's
notice to vacate the property complied with section 24.005.\ldots

Because Rosberg had the right to terminate appellant's tenancy at any time and
properly notified her of the termination under section 24.005 of the Texas
Property Code, the trial court did not err in awarding the property at issue to
Rosberg. Consequently, it is unnecessary for us to address the remainder of
appellant's issues.  

We affirm the trial court's judgment.

