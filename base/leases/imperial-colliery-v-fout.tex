\reading{Imperial Colliery Co. v. Fout}
\readingcite{373 S.E.2d 489 (W. Va. 1988)}

Danny H. Fout, the defendant below, appeals a summary judgment dismissing his
claim of retaliatory eviction based on the provisions of W. Va. Code,
55--3A--3(g), which is our summary eviction statute. Imperial Colliery had
instituted an eviction proceeding and Fout sought to defend against it,
claiming that his eviction was in retaliation for his participation in a labor
strike.

This case presents two issues: (1) whether a residential tenant who is sued for
possession of rental property under W. Va. Code, 55--3A--1, \textit{et seq.},
may
assert retaliation by the landlord as a defense, and (2) whether the
retaliation motive must relate to the tenant's exercise of a right incidental
to the tenancy.

Fout is presently employed by Milburn Colliery Company as a coal miner. For six
years, he has leased a small house trailer lot in Burnwell, West Virginia, from
Imperial Colliery Company. It is alleged that Milburn and Imperial are
interrelated companies. A written lease was signed by Fout and an agent of
Imperial in June, 1983. This lease was for a primary period of one month, and
was terminable by either party upon one month's notice. An annual rental of
\$1.00 was payable in advance on January 1 of each year. No subsequent written
leases were signed by the parties.

On February 14, 1986, Imperial advised Fout by certified letter that his lease
would be terminated as of March 31, 1986. Fout's attorney corresponded with
Imperial before the scheduled termination date. He advised that due to various
family and monetary problems, Fout would be unable to timely vacate the
property. Imperial voluntarily agreed to a two-month extension of the lease. A
second letter from Fout's attorney, dated May 27, 1986, recited Fout's personal
problems and requested that Imperial's attempts to oust Fout be held ``in
abeyance'' until they were resolved. A check for \$1.00 was enclosed to cover
the proposed extension. Imperial did not reply.

On June 11, 1986, Imperial sued for possession of the property, pursuant to
W. Va. Code, 55--3A--1, \textit{et seq.}, in the Magistrate Court of Kanawha
County. Fout answered and removed the suit to the circuit court on June 23,
1986. He asserted as a defense that Imperial's suit was brought in retaliation
for his involvement in the United Mine Workers of America and, more
particularly, in a selective strike against Milburn. Imperial's retaliatory
motive was alleged to be in violation of the First Amendment rights of speech
and assembly, and of the National Labor Relations Act, 29 U.S.C. \S~151,
\textit{et seq}. Fout also counter-claimed, seeking an injunction against
Imperial and damages for annoyance and inconvenience.

After minimal discovery, Imperial moved for summary judgment. The circuit court
granted Imperial's motion in an amended judgment order dated October 8, 1986,
relying principally upon \textit{Criss v. Salvation Army Residences}, 173 W.Va.
634, 319 S.E.2d 403 (1984). The court concluded that the retaliation defense
``must derive from, or in some respect be related to, exercise by the tenant of
rights incident to his capacity as a `tenant'.'' Since Fout's participation in
the labor strike was admittedly unrelated to his tenancy, the defense was
dismissed and possession of the property was awarded to Imperial. It is from
this order that Fout appeals.

Our initial inquiry is whether retaliation by the landlord may be asserted by
the tenant as a defense in a suit under W. Va. Code, 55--3A--3(g). We addressed
this issue in \textit{Criss v. Salvation Army Residences}, \textit{supra}, and
stated without any extended discussion that this section ``specifically
provides for the defense of retaliation.'' 173 W. Va. at 640, 319 S.E.2d at 409.
We did not have occasion in \textit{Criss} to trace the development of the
retaliatory eviction defense.

It appears that the first case that recognized retaliatory eviction as a defense
to a landlord's eviction proceeding was \textit{Edwards v. Habib}, 397 F.2d 687
(D.C.Cir.1968), \textit{cert. denied}, 393 U.S. 1016
(1969). There, a month-to-month tenant who resided in a District of
Columbia apartment complex reported to a local health agency a number of
sanitary code violations existing in her apartment. The agency investigated and
ordered that remedial steps be taken by the landlord, who then advised Edwards
that her lease was terminated. When the landlord sued for possession of the
premises, Edwards alleged the suit was brought in retaliation for her reporting
of the violations. A verdict was directed for the landlord and Edwards
appealed.

On appeal, the court reviewed at length the goals sought to be advanced by local
sanitary and safety codes. It concluded that to allow retaliatory evictions by
landlords would seriously jeopardize the efficacy of the codes. A prohibition
against such retaliatory conduct was therefore to be implied, even though the
regulations were silent on the matter.

Many states have protected tenant rights either on the \textit{Edwards} theory
or have implied such rights from the tenant's right of habitability. Others
have utilized statutes analogous to section 5.101 of the Uniform Residential
Landlord and Tenant Act, 7B U.L.A. 503 (1985), which is now adopted in fifteen
jurisdictions. Similar landlord and tenant reform statutes in seventeen other
states also provide protection for tenancy-related activities.

Under W. Va. Code, 37--6--30, a tenant is, with respect to residential property,
entitled to certain rights to a fit and habitable dwelling. In \textit{Teller
v. McCoy}, 162 W. Va. 367, 253 S.E.2d 114 (1978), we spoke at some length of the
common law right of habitability which a number of courts had developed to
afford protection to the residential tenant. We concluded that these rights
paralleled and were spelled out in more detail in W. Va. Code, 37--6--30. In
Teller, we also fashioned remedies for the tenant where there had been a breach
of the warranty of habitability. However, we had no occasion to discuss the
retaliatory eviction issue in \textit{Teller}.

The central theme underlying the retaliatory eviction defense is that a tenant
should not be punished for claiming the benefits afforded by health and safety
statutes passed for his protection. These statutory benefits become a part of
his right of habitability. If the right to habitability is to have any meaning,
it must enable the tenant to exercise that right by complaining about unfit
conditions without fear of reprisal by his landlord. \textit{See} Annot., 40
A.L.R.3d 753 (1971).

After the seminal decision in \textit{Edwards}, other categories of tenant
activity were deemed to be protected. Such activity was protected against
retaliation where it bore a relationship to some legitimate aspect of the
tenancy. For example, some cases provided protection for attempts by tenants to
organize to protect their rights as tenants. Others recognized the right to
press complaints directly against the landlord via oral communications,
petitions, and ``repair and deduct'' remedies.\ldots

A few courts recognize that even where a tenant's activity is only indirectly
related to the tenancy relationship, it may be protected against retaliatory
conduct if such conduct would undermine the tenancy relationship. Typical of
these cases is \textit{Winward Partners v. Delos Santos}, 59 Haw. 104, 577 P.2d
326 (1978). There a group of month-to-month tenants gave testimony before a
state land use commission in opposition to a proposal to redesignate their farm
property from ``agricultural'' to ``urban'' uses. The proposal was sponsored by
the landlord, a land developer. As a result of coordinated activity by the
tenants, the proposal was defeated. Within six months, the landlord ordered the
tenants to vacate the property and brought suit for possession.

The Hawaii Supreme Court noted that statutory law provided for public hearings
on proposals to redesignate property, and specifically invited the views of the
affected tenants. The court determined that the legislative policy encouraging
such input would be jeopardized ``if\ldots [landlords] were permitted to
retaliate against\ldots tenants for opposing land use changes in a public
forum.'' 59 Haw. at 116, 577 P.2d at 333. It relied on \textit{Pohlman v.
Metropolitan Trailer Park, Inc.}, 126 N.J.Super. 114, 312 A.2d 888
(Ch.Div.1973), which involved a similar fact pattern where tenants'
intervention in zoning matters to protect their tenancy was sufficiently
germane to the landlord-tenant relationship to support the defense of
retaliatory eviction. \textit{See also S.P. Growers Ass'n v. Rodriguez}, 17
Cal.3d 719, 552 P.2d 721, 131 Cal. Rptr. 761 (1976) (retaliation for suit by
tenant charging violation of Farm Labor Contractor Registration Act, 7 U.S.C.
\S~2041, et seq.).

The Legislature, in giving approval to the retaliation defense, must have
intended to bring our State into line with the clear weight of case law and
statutory authority outlined above. We accordingly hold that retaliation may be
asserted as a defense to a summary eviction proceeding under W. Va. Code,
55--3A--1, et seq., if the landlord's conduct is in retaliation for the
tenant's exercise of a right incidental to the tenancy.

Fout seeks to bring this case within the \textit{Windward} line of authority. He
argues principally that Imperial's conduct violated a public policy which
promotes the rights of association and free speech by tenants. We do not agree,
simply because the activity that Fout points to as triggering his eviction was
unrelated to the habitability of his premises.

From the foregoing survey of law, we are led to the conclusion that the
retaliatory eviction defense must relate to activities of the tenant incidental
to the tenancy. First Amendment rights of speech and association unrelated to
the tenant's property interest are not protected under a retaliatory eviction
defense in that they do not arise from the tenancy relationship. Such rights
may, of course, be vindicated on other independent grounds.

For the reasons discussed above, the judgment of the Circuit Court of Kanawha
County is affirmed.

