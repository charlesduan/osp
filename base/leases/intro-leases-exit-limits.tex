Under the traditional common law, leaseholds were freely transferable property
interests.  Modern courts continue to recognize the alienability of tenancies
as a default position, but allow parties to contract around the basic rule.  As
a result, most leases (including yours, probably) now contain some restriction
on a tenant's ability to assign or sublease her property interests.  For
example, one oft-used lease agreement, which can be downloaded for free from
the Internet, includes the following provision: ``The tenant will not assign
this Lease, or sublet or grant any concession or license to use the Property or
any part of the Property.  Any assignment or subletting will be void and will,
at the Landlord's option, terminate the Lease.''  In most states, courts uphold
such bars on transfer as reasonable restraints on alienation. More
controversial are clauses that allow sublease or assignment but only ``with the
consent of the landlord.''

