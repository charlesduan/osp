\expected{fidelity-v-kaminsky}

\item \textbf{Evolution of the doctrine.}  As discussed above, English judges
widely recognized that tenants could terminate the lease (and sue for damages)
if the landlord physically denied them possession of the rented property. 
Eventually the basic concept was expanded to situations where the landlord
commits some act that, while it falls short of an actual eviction, so severely
affects the value of the tenancy that the tenant is forced to flee.  This is
known as \term{constructive eviction}.  


\item \textbf{Basic constructive eviction law.} To make a claim of constructive
eviction a tenant must show that some act or omission by the landlord
substantially interferes with the tenant's use and enjoyment of the property. 
The tenant also needs to notify the landlord about the problem, give the
landlord an opportunity to cure the defect, and then vacate the premises within
a reasonable amount of time.


\item \textbf{Stay or go?} Why might a tenant contemplating bringing a
constructive eviction claim worry about the requirement to vacate the premises?
 Is constructive eviction a more powerful remedy in a place like San Francisco,
which has a very tight housing market, or Houston, which has more open units?  


\item \textbf{Landlord's wrongful conduct.}  To make use of the doctrine of
quiet enjoyment, the tenant must show that the landlord committed some wrongful
act.  There's wide agreement that any affirmative step taken by the landlord
that impedes the tenant's use of the property can meet the requirement of an
``act.''  Examples would include burning toxic substances on the property,
prolonged construction activities, or a substantial alteration of an essential
feature of the leased premises.  The trickier doctrinal question is whether a
landlord's failure to act can ever qualify as the wrongful conduct. 
Traditionally, courts hesitated to impose liability on landlords for their
omissions, but the law of most states now asserts that a ``lack of action'' can
constitute the required act. For example, a landlord's failure to provide heat
in the winter months is generally found to violate the covenant of quiet
enjoyment. Some courts, nervous about unjustly expanding landlords' potential
liability, deem omissions wrongful only when the landlord fails to fulfill some
clear duty---either a duty bargained for in the lease or a statutory duty.


\item \textbf{Troublesome tenants.}  Suppose your landlord rents the floor above
your apartment to the members of a Led Zeppelin cover band.  If the band
practices every night between the hours of 3:00 am and 4:00 am, could you bring
a successful constructive eviction claim against the landlord? 


\item \textbf{Third parties.} What if the Led Zeppelin cover band
played every night at a club across the street?  If the noise from the bar kept
you awake, could you sue your landlord for constructive eviction?

