Most landlords require their tenants to pay a security deposit---a sum of money
that the landlord can raid if the tenant defaults on the rent, leaves the unit
untidy, or damages any property during the course of the tenancy.  State law
mandates that if the tenant has compiled with all terms of the lease and kept
the unit in good order, the landlord must return the security deposit
(generally within 30 or 60 days).  If the tenant causes damage, the landlord
has the right to use the security to restore the unit to its previous
condition, but must provide the tenant with a list of damages and receipts for
the repairs. 

Although the law of security deposits is generally crystal-clear, a huge number
of renters report that they have unfairly lost deposit money to their
landlords.  Why is this so? Game theorists argue that the structure of the
landlord-tenant relationship makes disputes over security deposits almost
unavoidable.  The key insight is that while the tenancy is ongoing, landlords
and tenants have incentives to get along and make compromises---the landlord
wants the tenant to make timely rent payments and the tenant wants the landlord
to respond quickly when problems arise.  However, once the landlord and tenant
decide to end their relationship, there are few checks to prevent bad behavior.
 If the landlord will never interact with the tenant again, why not fudge a
little bit with security deposit?  Additionally, the small amounts of money
involved security deposit disputes mean that it's rarely worth hiring a lawyer
or taking the time to sue the landlord in small claims court. 

