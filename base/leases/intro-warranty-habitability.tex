Although the covenant of quiet enjoyment offers tenants some protections, the
doctrine---without more---can leave renters exposed to dreadful living
conditions.  What if cockroaches invade a tenant's apartment? Or a sewer pipe
in the basement begins to leak?  What if a storm shatters the windows of the
apartment? Or a wall of a building falls down?  Unless the landlord somehow
caused any of these disasters (or had a clearly articulated duty to fix them) a
tenant cannot bring a successful case under the covenant of quiet enjoyment. 
In \textit{Hughes v. Westchester Development Corp.}, 77 F.2d 550 (D.C. Cir.
1935), for example, vermin invaded the tenant's apartment, making it
``impossible to use the kitchen and toilet facilities.'' Despite the
infestation, the court found that the tenant remained responsible for the rent
because the landlord was not to blame for the bugs' sudden appearance.  Leases,
the court ruled, contained no implied promise that the premises were fit for the
purpose it was leased.  If tenants desired more and better protection, they had
the burden to bargain for such provisions in the lease. 

All of this changed in the late 1960s and early 70s.  The most lasting
accomplishment of the tenants' rights movement was the widespread adoption of
the \term{implied warranty of habitability}. In the United States, only
Arkansas has failed to adopt the rule as of 2023.  In a nutshell, the implied
warranty of
habitability imposes a duty on landlords to provide residential tenants with a
clean, safe, and habitable living space. 

