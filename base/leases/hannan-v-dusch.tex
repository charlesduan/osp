\reading{Hannan v. Dusch}

\readingcite{153 S.E. 824 (Va. 1930)}

\textsc{Prentis}, C.J.,

The declaration filed by the plaintiff, Hannan, against the defendant, Dusch,
alleges that Dusch had on August 31, 1927, leased to the plaintiff certain real
estate in the city of Norfolk, Virginia, therein described, for fifteen years,
the term to begin January 1, 1928, at a specified rental; that it thereupon
became and was the duty of the defendant to see to it that the premises leased
by the defendant to the plaintiff should be open for entry by him on January 1,
1928, the beginning of the term, and to put said petitioner in possession of
the premises on that date; that the petitioner was willing and ready to enter
upon and take possession of the leased property, and so informed the defendant;
yet the defendant failed and refused to put the plaintiff in possession or to
keep the property open for him at that time or on any subsequent date; and that
the defendant suffered to remain on said property a certain tenant or tenants
who occupied a portion or portions thereof, and refused to take legal or other
action to oust said tenants  or to compel their removal from the property so
occupied. Plaintiff alleged damages which he had suffered by reason of this
alleged breach of the contract and deed, and sought to recover such damages in
the action. There is no express covenant as to the delivery of the
premises\ldots.

The single question of law therefore presented in this case is whether a
landlord, who without any express covenant as to delivery of possession leases
property to a tenant, is required under the law to oust trespassers and
wrongdoers so as to have it open for entry by the tenant at the beginning of
the term---that is, whether without an express covenant there is nevertheless
an implied covenant to deliver possession.\ldots

It seems to be perfectly well settled that there is an implied covenant in such
cases on the part of the landlord to assure to the tenant the legal right of
possession---that is, that at the beginning of the term there shall be no
legal obstacle to the tenant's right of possession. This is not the question
presented. Nor need we discuss in this case the rights of the parties in case a
tenant rightfully in possession under the title of his landlord is thereafter
disturbed by some wrongdoer. In such case the tenant must protect himself from
trespassers, and there is no obligation on the landlord to assure his quiet
enjoyment of his term as against wrongdoers or intruders.

Of course, the landlord assures to the tenant quiet possession as against all
who rightfully claim through or under the landlord.

The discussion then is limited to the precise legal duty of the landlord in the
absence of an express covenant, in case a former tenant, who wrongfully holds
over, illegally refuses to surrender possession to the new tenant. This is a
question about which there is a hopeless conflict of the authorities. It is
generally claimed that the weight of the authority favors the particular view
contended for. There are, however, no scales upon which we can weigh the
authorities. In numbers and respectability they may be quite equally balanced.

It is then a question about which no one should be dogmatic, but all should seek
for that rule which is supported by the better reason.\ldots

It is conceded by all that the two rules, one called the English rule, which
implies a covenant requiring the lessor to put the lessee in possession, and
that called the American rule, which recognizes the lessee's legal right to
possession, but implies no such duty upon the lessor as against wrongdoers, are
irreconcilable.

The English rule is that in the absence of stipulations to the contrary, there
is in every lease an implied covenant on the part of the landlord that the
premises shall be open to entry by the tenant at the time fixed by the lease
for the beginning of his term.\ldots

[A] case which supports the English rule is \textit{Herpolsheimer v.
Christopher}, 76 Neb. 352, 107 N.W. 382, 111 N.W. 359, 9 L.R.A. (N.S.) 1127, 14
Ann. Cas. 399, note. In that case the court gave these as its reasons for
following the English rule: 
\begin{quote}
We think\ldots that the English rule is most in consonance with good conscience,
sound principle, and fair dealing. Can it be supposed that the plaintiff in
this case would have entered into the lease if he had known at the time that he
could not obtain possession on the 1st of March, but that he would be compelled
to begin a lawsuit, await the law's delays, and follow the case through its
devious turnings to an end before he could hope to obtain possession of the
land he had leased? Most assuredly not. It is unreasonable to suppose that a
man would knowingly contract for a lawsuit, or take the chance of one. Whether
or not a tenant in possession intends to hold over or assert a right to future
term may nearly always be known to the landlord, and is certainly much more apt
to be within his knowledge than within that of the prospective tenant.
Moreover, since in an action to recover possession against a tenant holding
over, the lessee would be compelled largely to rely upon the lessor's testimony
in regard to the facts of the claim to hold over by the wrongdoer, it is more
reasonable and proper to place the burden upon the person within whose
knowledge the facts are most apt to lie. We are convinced, therefore, that the
better reason lies with the courts following the English doctrine, and we
therefore adopt it, and hold that, ordinarily, the lessor impliedly covenants
with the lessee that the premises leased shall be open to entry by him at the
time fixed in the lease as the beginning of the term.\ldots
\end{quote}
So let us not lose sight of the fact that under the English rule a covenant
which might have been but was not made is nevertheless implied by the court,
though it is manifest that each of the parties might have provided for that and
for every other possible contingency relating to possession by having express
covenants which would unquestionably have protected both.

Referring then to the American rule: Under that rule, in such cases,\ldots the
landlord is not bound to put the tenant into actual possession, but is bound
only to put him in legal possession, so that no obstacle in the form of
superior right of possession will be interposed to prevent the tenant from
obtaining actual possession of the demised premises. If the landlord gives the
tenant a right of possession he has done all that he is required to do by the
terms of an ordinary lease, and the tenant assumes the burden of enforcing such
right of possession as against all persons wrongfully in possession, whether
they be trespassers or former tenants wrongfully holding over.\ldots

So that, under the American rule, where the new tenant fails to obtain
possession of the premises only because a former tenant wrongfully holds over,
his remedy is against such wrongdoer and not against the landlord---this
because the landlord has not covenanted against the wrongful acts of another
and should not be held responsible for such a tort unless he has expressly so
contracted. This accords with the general rule as to other wrongdoers, whereas
the English rule appears to create a specific exception against lessors. It
does not occur to us now that there is any other instance in which one clearly
without fault is held responsible for the independent tort of another in which
he has neither participated nor concurred and whose misdoings he cannot
control.\ldots

For the reasons which have been so well stated by those who have enforced the
American rule, our judgment is that there is no error in the judgment
complained of.

Affirmed

