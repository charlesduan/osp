\reading{An Exercise in Advertising}

Imagine that you are a lawyer for a newspaper in a large metropolitan area.  The
local chapter of the ACLU has raised concerns that some advertisements in the
classifieds section of your paper violate the Fair Housing
Act.\footnote{Would any of these ads violate the Civil Rights Act of
1866?}  Your boss has asked you to review the ads in Figure~\ref{f:leases-04}
for any offending language. Which of the following would you feel comfortable
printing?\footnote{The government does provide some guidance to landlords
worried about triggering FHA liability through their advertisements.  There are,
for example, published lists of ``words to avoid'' and ``acceptable language.''
Although context is important, landlords can generally use these phrases: good
neighborhood, secluded setting, single family home, quality construction, near
public transportation, near places of worship, and assistance animals only.} 

\captionedgraphic[width=0.8\textwidth,
height=0.6\textwidth]{leases-04}{Hypothetical classified advertisements.}

What about this ad for a roommate on Craigslist?  Is it objectionable to you?
Does it violate the FHA?  Does it matter that the poster is looking for a
\textit{roommate}? Would your answers change if the advertisement read, ``Have
a room available for an able-bodied white man with no children?'' 

\heregraphic[height=6em]{leases-05}

