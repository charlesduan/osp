\reading{Fidelity Mutual Life Insurance Co. v. Kaminsky}

\readingcite{768 S.W.2d 818 (Tex. App. 1989)}

\opinion \textsc{Murphy}, Justice.

The issue in this landlord-tenant case is whether sufficient evidence supports
the jury's findings that the landlord and appellant, Fidelity Mutual Life
Insurance Company [``Fidelity''], constructively evicted the tenant, Robert P.
Kaminsky, M.D., P.A. [``Dr. Kaminsky''] by breaching the express covenant of
quiet enjoyment contained in the parties' lease. We affirm.

Dr. Kaminsky is a gynecologist whose practice includes performing elective
abortions. In May 1983, he executed a lease contract for the rental of
approximately 2,861 square feet in the Red Oak Atrium Building for a two year
term which began on June 1, 1983. The terms of the lease required Dr. Kaminsky
to use the rented space solely as ``an office for the practice of medicine.''
Fidelity owns the building and hires local companies to manage it. At some time
during the lease term, Shelter Commercial Properties [``Shelter''] replaced the
Horne Company as managing agents. Fidelity has not disputed either management
company's capacity to act as its agent.

The parties agree that: (1) they executed a valid lease agreement; (2) Paragraph
35 of the lease contains an express covenant of quiet enjoyment conditioned on
Dr. Kaminsky's paying rent when due, as he did through November 1984; Dr.
Kaminsky abandoned the leased premises on or about December 3, 1984 and refused
to pay additional rent; anti-abortion protestors began picketing at the
building in June of 1984 and repeated and increased their demonstrations
outside and inside the building until Dr. Kaminsky abandoned the premises.

When Fidelity sued for the balance due under the lease contract following Dr.
Kaminsky's abandonment of the premises, he claimed that Fidelity constructively
evicted him by breaching Paragraph 35 of the lease. Fidelity apparently
conceded during trial that sufficient proof of the constructive eviction of Dr.
Kaminsky would relieve him of his contractual liability for any remaining rent
payments. Accordingly, he assumed the burden of proof and the sole issue
submitted to the jury was whether Fidelity breached Paragraph 35 of the lease,
which reads as follows:

\begin{quote}

\emph{Quiet Enjoyment.}

Lessee, on paying the said Rent, and any Additional Rental, shall and may
peaceably and quietly have, hold and enjoy the Leased Premises for the said
term.
\end{quote}

A constructive eviction occurs when the tenant leaves the leased premises due to
conduct by the landlord which materially interferes with the tenant's
beneficial use of the premises. \textit{See} \textit{Downtown Realty, Inc. v.
509 Tremont Bldg.}, 748 S.W.2d 309, 313 (Tex.App.---Houston [14th Dist.] 1988,
n.w.h.). Texas law relieves the tenant of contractual liability for any
remaining rentals due under the lease if he can establish a constructive
eviction by the landlord.\ldots

In order to prevail on his claim that Fidelity constructively evicted him and
thereby relieved him of his rent obligation, Dr. Kaminsky had to show the
following: 1) Fidelity intended that he no longer enjoy the premises, which
intent the trier of fact could infer from the circumstances; 2) Fidelity, or
those acting for Fidelity or with its permission, committed a material act or
omission which substantially interfered with use and enjoyment of the premises
for their leased purpose, here an office for the practice of medicine; 3)
Fidelity's act or omission permanently deprived Dr. Kaminsky of the use and
enjoyment of the premises; and 4) Dr. Kaminsky abandoned the premises within a
reasonable period of time after the act or omission. \textit{E.g.},
\textit{Downtown Realty, Inc.}, 748 S.W.2d at 311\ldots.

[T]he jury found that Dr. Kaminsky had established each element of his
constructive eviction defense. The trial court entered judgment that Fidelity
take nothing on its suit for delinquent rent.

Fidelity raises four points of error.\ldots 

Fidelity's first point of error relies on \textit{Angelo v. Deutser}, 30 S.W.2d
707 (Tex.Civ.App.---Beaumont 1930, no writ), \textit{Thomas v. Brin}, 38
Tex.Civ.App. 180, 85 S.W. 842 (1905, no writ) and \textit{Sedberry v.
Verplanck}, 31 S.W. 242 (Tex.Civ.App.1895, no writ). These cases all state the
general proposition that a tenant cannot complain that the landlord
constructively evicted him and breached a covenant of quiet enjoyment, express
or implied, when the eviction results from the actions of third parties acting
without the landlord's authority or permission. Fidelity insists the evidence
conclusively establishes: a) that it did nothing to encourage or sponsor the
protestors and; b) that the protestors, rather than Fidelity or its agents,
caused Dr. Kaminsky to abandon the premises. Fidelity concludes that reversible
error resulted because the trial court refused to set aside the jury's answers
to the special issues and enter judgment in Fidelity's favor and because the
trial court denied its motion for a new trial. We disagree.\ldots 

The protests took place chiefly on Saturdays, the day Dr. Kaminsky generally
scheduled abortions. During the protests, the singing and chanting
demonstrators picketed in the building's parking lot and inner lobby and atrium
area. They approached patients to speak to them, distributed literature,
discouraged patients from entering the building and often accused Dr. Kaminsky
of ``killing babies.'' As the protests increased, the demonstrators often
occupied the stairs leading to Dr. Kaminsky's office and prevented patients
from entering the office by blocking the doorway. Occasionally they succeeded
in gaining access to the office waiting room area.

Dr. Kaminsky complained to Fidelity through its managing agents and asked for
help in keeping the protestors away, but became increasingly frustrated by a
lack of response to his requests. The record shows that no security personnel
were present on Saturdays to exclude protestors from the building, although the
lease required Fidelity to provide security service on Saturdays. The record
also shows that Fidelity's attorneys prepared a written statement to be handed
to the protestors soon after Fidelity hired Shelter as its managing agent. The
statement tracked \textsc{Tex. Penal Code Ann.} \S~30.05 (Vernon Supp. 1989) and
generally served to inform trespassers that they risked criminal prosecution by
failing to leave if asked to do so. Fidelity's attorneys instructed Shelter's
representative to ``have several of these letters printed up and be ready to
distribute them and verbally demand that these people move on and off the
property.'' The same representative conceded at trial that she did not
distribute these notices. Yet when Dr. Kaminsky enlisted the aid of the
Sheriff's office, officers refused to ask the protestors to leave without a
directive from Fidelity or its agent. Indeed, an attorney had instructed the
protestors to remain unless the landlord or its representative ordered them to
leave. It appears that Fidelity's only response to the demonstrators was to
state, through its agents, that it was aware of Dr. Kaminsky's problems.

Both action and lack of action can constitute ``conduct'' by the landlord which
amounts to a constructive eviction. \textit{E.g.}, \textit{Downtown Realty
Inc.}, 748 S.W.2d at 311. In \textit{Steinberg v. Medical Equip. Rental Serv.,
Inc.}, 505 S.W.2d 692 (Tex. Civ. App.---Dallas 1974, no writ) accordingly, the
court upheld a jury's determination that the landlord's failure to act amounted
to a constructive eviction and breach of the covenant of quiet enjoyment. 505
S.W.2d at 697. Like Dr. Kaminsky, the tenant in Steinberg abandoned the leased
premises and refused to pay additional rent after repeatedly complaining to the
landlord. The \textit{Steinberg} tenant complained that Steinberg placed trash
bins near the entrance to the business and allowed trucks to park and block
customer's access to the tenant's medical equipment rental business. The
tenant's repeated complaints to Steinberg yielded only a request ``to be
patient.'' Id. Fidelity responded to Dr. Kaminsky's complaints in a similar
manner: although it acknowledged his problems with the protestors, Fidelity,
like Steinberg, effectively did nothing to prevent the problems.

This case shows ample instances of Fidelity's failure to act in the fact of
repeated requests for assistance despite its having expressly covenanted Dr.
Kaminsky's quiet enjoyment of the premises. These instances provided a legally
sufficient basis for the jury to conclude that Dr. Kaminsky abandoned the
leased premises, not because of the trespassing protestors, but because of
Fidelity's lack of response to his complaints about the protestors. Under the
circumstances, while it is undisputed that Fidelity did not ``encourage'' the
demonstrators, its conduct essentially allowed them to continue to trespass.
The general rule of the \textit{Angelo}, \textit{Thomas} and \textit{Sedberry}
cases, that a landlord is not responsible for the actions of third parties,
applies only when the landlord does not permit the third party to act.
\textit{See e.g.}, \textit{Angelo}, 30 S.W.2d at 710 [``the act or omission
complained of must be that of the landlord and not merely of a third person
\textit{acting without his authority or permission}'' (emphasis added)]. We
see no distinction between Fidelity's lack of action here, which the record
shows resulted in preventing patients' access to Dr. Kaminsky's medical office,
and the \textit{Steinberg} case where the landlord's inaction resulted in
trucks' blocking customer access to the tenant's business. We overrule the
first point of error.\ldots .

In its [final] point of error, Fidelity maintains the evidence is factually
insufficient to support the jury's finding that its conduct permanently
deprived Dr. Kaminsky of use and enjoyment of the premises. Fidelity
essentially questions the permanency of Dr. Kaminsky's being deprived of the
use and enjoyment of the leased premises. To support its contentions, Fidelity
points to testimony by Dr. Kaminsky in which he concedes that none of his
patients were ever harmed and that protests and demonstrations continued
despite his leaving the Red Oak Atrium building. Fidelity also disputes whether
Dr. Kaminsky actually lost patients due to the protests.

The evidence shows that the protestors, whose entry into the building Fidelity
failed to prohibit, often succeeded in blocking Dr. Kaminsky's patients' access
to his medical office. Under the reasoning of the \textit{Steinberg} case,
omissions by a landlord which result in patients' lack of access to the office
of a practicing physician would suffice to establish a permanent deprivation of
the use and enjoyment of the premises for their leased purpose, here ``an
office for the \textit{practice} of medicine.'' \textit{Steinberg}, 505 S.W.2d
at 697; \textit{accord}, \textit{Downtown Realty, Inc.}, 748 S.W.2d at 312
(noting jury's finding that a constructive eviction resulted from the
commercial landlord's failure to repair a heating and air conditioning system
in a rooming house).

Texas law has long recited the requirement, first stated in \textit{Stillman},
266 S.W.2d at 916, that the landlord commit a ``material and permanent'' act or
omission in order for his tenant to claim a constructive eviction. However, as
the \textit{Steinberg} and \textit{Downtown Realty, Inc.} cases illustrate, the
extent to which a landlord's acts or omissions permanently and materially
deprive a tenant of the use and enjoyment of the premises often involves a
question of degree. Having reviewed all the evidence before the jury in this
case, we cannot say that its finding that Fidelity's conduct permanently
deprived Dr. Kaminsky of the use and enjoyment of his medical office space was
so against the great weight and preponderance of the evidence as to be
manifestly unjust. We overrule the fourth point of error.

We affirm the judgment of the trial court.

