\expected{sommer-v-kridel}

\item \textbf{The basic law.}  Today almost all states impose a duty to mitigate
on residential landlords.  The rule also applies to commercial tenancies in
many states.  The Restatement (Second) of Property \S~12.1(3), however,
continues to cling to the common law notion that a landlord can wait until the
end of the term and then sue the tenant for all of the unpaid rent.  The
authors of the Restatement believe the traditional rule discourages
abandonment, limits vandalism, and better protects the expectations of
landlords.  


\item \textbf{Tenants still on the hook.}  Importantly, the duty to mitigate
does not relieve an abandoning tenant of all liability.  Even if a new tenant
rents the unit, the landlord can still recover damages for all of the costs of
finding the replacement tenant and for any time that the unit remained empty.
The landlord can also recoup any unpaid rent that accrued before the
abandonment.  Finally, if the rental market in the area has softened and
landlord is forced to rent the unit at lower price, the tenant is responsible
for the difference between the new rent and the original rent.


\item \textbf{Property v. Contract.} The lingering controversy over the duty to
mitigate stems largely from the property/contract tension inherent in the
nature of the lease.  If a leasehold is primarily a property interest, then the
landlord has few responsibilities to the tenant after ceding possession and
control---the tenant is free to use the property or let it lay fallow.  If, on
the other hand, the lease is viewed through the lens of contract law, the
parties clearly have a responsibility to mitigate damages.  \textit{But see}
Edward Chase \& E. Hunter Taylor, Jr., \textit{Landlord and Tenant: A Study in
Property and Contract}, \textsc{30 Vill. L. Rev.} 571 (1985) (arguing the
distinction is overstated).


\item \textbf{What's a good faith effort?} Ken rents an apartment to Sarah for
one year.  Three months into the lease, Sarah gets a new job in a different
state and turns the apartment back over to Ken.  Ken puts an 8x11 ``for rent''
sign in the window of the unit.  Has he made a good faith effort to mitigate
damages? Does it matter how he advertises the other units? What if Tim offers
to rent Sarah's unit but Tim has bad credit: does Ken have to accept Tim?


\item \textbf{The Legend of Jim Kridel.}  The woman Jim Kridel intended to marry
came from a family with significant assets. When the engagement fell through,
Kridel---who had no income of his own---could not afford the rent at the Pierre
Apartments. The opinion mentions that Kridel notified Sommer of his predicament
in writing, but does not reflect that Kridel and Sommer also had a heated
discussion on the phone.  During the telephone conversation, Kridel offered
Sommer \$750 of the pre-paid rent as compensation for breaking the lease
(adjusted for inflation, that's roughly equivalent to \$3000 today).  Sommer,
however, knew that Kridel's stepfather was a prominent (and presumably
well-off) physician and demanded an additional \$750.  Kridel refused, and told
Sommer, ``If you don't like it, you can sue me, baby!'' Sommer did just that. 
When the litigation began, Kridel was a first year law student at Rutgers.  He
initially represented himself but gradually picked up pro bono help from
lawyers he met at summer jobs and partners in the firm where he worked after
graduating.  Kridel estimates that Sommer---a very wealthy landlord---spent
over \$500,000 on legal fees.  Kridel also recalls that the law of New Jersey
was firmly against his position that the lease should be governed by contract
principles.  On appeal, he relied primarily on a case from the state of Oregon,
which opposing counsel disparaged as a place full of bumpkin fishermen and
loggers.  When Kridel won, he wrapped the opinion around an Oregon salmon and
sent to Sommer's lawyers.  Asked why he pursued the case with such vigor, he
replied, ``Sommer was wrong.  The rule was unfair. And I was probably the only
tenant in New Jersey who could afford to pour that much time and attention into
a case like that.''  In the intervening years, Kridel has had a long and
successful legal career in New Jersey and New York.  He's currently best known
for representing \textit{Real Housewives of New Jersey} star Teresa Giudice in
her bankruptcy proceeding.  

