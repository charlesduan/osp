A lease is a transfer of the right of possession of specific property for a
limited period of time.  It's important to see that not all legal relationships
that grant the use of an owner's property qualify as leaseholds.  Take, for
example, the case of Snow White and Seven Dwarfs.  If the Dwarfs give Snow
White sole possession of their cottage for 12 months in exchange for a monthly
payment, they have almost certainly created a lease.  If, however, the Dwarfs
invite Snow White to sleep on their couch for a few nights while she evades the
Queen, they probably have created something called a \textit{license} (a
revocable permission to use the property of another, which we'll study in
greater detail later in the book) rather than a leasehold.  This determination
matters (as we'll soon see) because the law extends a number of protections to
grantees who qualify as tenants. It affects, among other things, whether the
grantee can exclude the owner from certain spaces, how the parties can
terminate the interest, whether the grantee can invite outsiders onto the
property, who has the obligation to perform maintenance, who is liable if the
grantee suffers an injury on the property, and what remedies the parties have
if a disagreement arises.

To determine whether parties have created a leasehold or some other legal
interest, courts have traditionally focused on whether a grantor has turned
over exclusive possession or a more limited set of use rights.  Possession,
however, remains a slippery concept, difficult to define. Consider the
following post from an internet message board:  
\heregraphic[width=0.8\textwidth]{leases-01}
Does the nanny have a lease?  Do we need any other information? Should courts
look beyond mere facts of possession and consider the policy considerations of
extending landlord-tenant protections to the parties in the case?  What might
those policy considerations entail?  As you read through the materials, you may
want to revisit this question.

