Defined broadly, gentrification is the movement of
wealthier people into a poor neighborhood, which results in a subsequent
increase in rents and the ultimate displacement of longtime residents.  The
stereotypic progression starts when artists and gay couples move into a
run-down but centrally located neighborhood in the urban core.  They fix up
houses, open trendy cafes, and start galleries.  The newcomers also demand
better public services and police protection from the local government.  As the
number of amenities grows, home prices and rents begin to rise.  Married
couples without children start to flow into the area, followed quickly by
bankers, lawyers, and families attracted the neighborhood's beautiful older
homes and terrific location.  As rents continue to rise, many of the original
residents---who are often poor and black---can no longer afford the
neighborhood.  They are forced to either move or pay an enormous percentage of
their income toward rent.  

\captionedgraphic{leases-11}{Photo courtesy of Flickr user Keith Hamm}

One resident of a gentrifying neighborhood in Portland gives a personal account
of the basic problem: 
\begin{quotation}
Last week I heard a shuffle at my front door and saw that my building manager
was slipping a notice under my door. I opened it only to read that my rent was
being raised by 10\%.\ldots [In the last year], my rent has gone up a total of
14\%. If it continues at this pace, I'll have to find another place to live
because I'll be priced out of my very walkable, very centrally-located
neighborhood.

[Gentrification is] an emotional tinderbox. People who are just going about
their lives are having to face eviction, displacement, or just have to spend a
lot more on housing if they want to stay where they are because of forces
completely out of their control. In other words, you could be doing everything
``right'' in your life -- being a responsible citizen, earning a viable income
and doing your best -- but it still isn't good enough. Not unlike the tragedy
of having your house destroyed by a natural phenomenon like a hurricane or a
flood, you could become a victim of the ``greed phenomenon'' where developers
look with dollar signs in their eyes at the house you live in with the
intention of razing it and building a hugely profitable and expensive condo
building there instead.
\end{quotation}

For low-income individuals pushed out of their neighborhoods, the process of
gentrification often produces traumatic effects.  In addition to the financial
costs of an unwanted move, gentrification often shatters valuable personal
networks.  People who have lived their entire lives within a small geographic
area may suddenly find themselves separated from the friends and family who
provide emotional support and economic resources that serve as a vital buffer
against the ills of poverty.  

Many activists have suggested that rent control laws are the best solution to
problems spawned by gentrification.  Rent control legislation comes in a
variety of forms but most often puts caps on the amount of rent that a landlord
can charge (first-generation controls) and/or requires that prices for rented
properties do not increase by more than a certain percent each year
(second-generation controls).  Rent controls, activists argue, allow existing
tenants to stay in their homes while continuing to devote the same percentage
of their incomes to rent has they have in the past.  

Economists have a very different perspective on fighting gentrification with
rent control mechanisms. American legal economists are typically opposed to
rent controls.  Often heatedly so.  To understand why, put yourself in the
shoes of a landlord in a city that holds the price of rent below what the
market will bear.  How would you respond if you were forced to provide a
service for less than the market price?  First and foremost, you probably
wouldn't build any new rental housing units.  Why?  Because you'd almost
certainly make more money if you used your capital to build something that's
not regulated by the government.  Ultimately, the lack of proper incentive to
build apartments lowers the supply of rental housing and thereby increases the
price (for anyone who doesn't qualify for rent controls).  Second, you might
decide to skimp on the maintenance of your rent-controlled unit in order to
recoup some of the lost profits.  After all, will a tenant in a rent-controlled
apartment really give up their unit if you don't respond to their request to
fix the sink?  

So goes the theory, at any rate---and it is a theory that has found expression
in judicial opinions, particularly among those judges of the U.S. Court of
Appeals for the Seventh Circuit who moonlight as academic legal economists of
the so-called ``Chicago School.'' \textit{See }Chicago Board of Realtors, Inc.
v. City of Chicago, 819 F.2d 732, 741-42 (7th Cir. 1987) (Opinion of Posner,
J.). In apparent agreement with these theoretical arguments, very few American
jurisdictions today maintain rent control policies---only New York, Los
Angeles, and a few places in the Bay Area have significant rent control laws. 
State and local governments are much more likely to attack problems of
affordable housing by either giving rent vouchers to the poor or building
government-owned housing projects (are these better options?).  

But perhaps the legal economists of a generation ago were mistaken---or at least
insufficiently sensitive to the potential variety of rent control measures and
the diversity of urban environments in which they can be deployed. While
first-generation rent control measures have few academic defenders in the
United States, there is some suggestion that the actual empirics of
second-generation rent controls and other tenant protections may diverge from
the dire theoretical predictions of the Chicago School. In particular, the
effects of rent control on the supply, quality, and distribution of rental
housing may depend significantly on the nature of the protective regulation
imposed, the density of existing housing stock, availability of vacant land,
the mix of other regulatory constraints on land use in general and housing in
particular, and idiosyncrasies of the local economy---particularly the degree
of competition among landlords. \textit{See generally} Richard Arnott,
\textit{Time for Revisionism on Rent Control?}, 9 \textsc{J. Econ. Perspect.}
99 (1995); Bengt Turner \& Stephen Malpezzi, \textit{A review of empirical
evidence on the costs and benefits of rent control}, 10 \textsc{Swed. Econ.
Policy Rev}. 11 (2003). Outside of the United States, moreover, economists and
politicians are less antagonistic toward rent control.  Paris, for example,
recently passed a law capping many rents.  Germany, the Netherlands, and Sweden
also have widespread limitations on how much rent landlords can charge.  

