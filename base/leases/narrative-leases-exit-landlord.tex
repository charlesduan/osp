Landlords may sell their properties to third parties at any time.  The law
categorizes a landlord's interest in rented property as a reversion and, like
most other property interests, the landlord's reversion is fully alienable. 
But what happens to a lease if a property is transferred?  As a default rule,
when a landlord sells his interest, the purchaser takes subject to any leases.
If there are tenants with unexpired term-of-years leases, for example, the new
landlord cannot evict them.  Conversely, the tenants must continue to pay the
agreed upon rent to the new owner.  If the lease is a periodic tenancy (or
tenancy at will), the new landlord may end the leasehold by providing the
tenant with the required notice.  Until then, the leases continue unabated.  

Remember that these are default rules, alterable by contract.  In fact,
landlords often insert provisions into leases that give them the option to
terminate rental agreements upon sale of the property.  

