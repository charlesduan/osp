\reading{Julian v. Christopher}

\readingcite{575 A.2d 735 (Md. 1990)}

\textsc{Chasanow}, Judge.

In 1961, this Court decided the case of \textit{Jacobs v. Klawans}, 169 A.2d 677
(1961) and held that when a lease contained a ``silent consent'' clause
prohibiting a tenant from subletting or assigning without the consent of the
landlord, landlords had a right to withhold their consent to a subletting or
assignment even though the withholding of consent was arbitrary and
unreasonable.\ldots We now have before us the issue of whether the common law
rule applied in \textit{Klawans} should be changed.

In the instant case, the tenants, Douglas Julian and William J. Gilleland, III,
purchased a tavern and restaurant business, as well as rented the business
premises from landlord, Guy D. Christopher. The lease stated in clause ten that
the premises, consisting of both the tavern and an upstairs apartment, could
not be assigned or sublet ``without the prior written consent of the
landlord.'' Sometime after taking occupancy, the tenants requested the
landlord's written permission to sublease the upstairs apartment. The landlord
made no inquiry about the proposed sublessee, but wrote to the tenants that he
would not agree to a sublease unless the tenants paid additional rent in the
amount of \$150.00 per month. When the tenants permitted the sublessee to move
in, the landlord filed an action in the District Court of Maryland in Baltimore
City requesting repossession of the building because the tenants had sublet the
premises without his permission.  

At the district court trial, the tenants testified that they specifically
inquired about clause ten, and were told by the landlord that the clause was
merely included to prevent them from subletting or assigning to ``someone who
would tear the apartment up.'' The district court judge refused to consider
this testimony. He stated in his oral opinion that he would ``remain within the
four corners of the lease, and construe the document strictly,'' at least as it
pertained to clause ten. Both the District Court and, on appeal, the Circuit
Court for Baltimore City found in favor of the landlord. The circuit judge
noted: ``If you don't have the words that consent will not be unreasonably
withheld, then the landlord can withhold his consent for a good reason, a bad
reason, or no reason at all in the context of a commercial lease, which is what
we're dealing with.'' We granted certiorari to determine whether the
\textit{Klawans} holding should be modified in light of the changes that have
occurred since that decision.

While we are concerned with the need for stability in the interpretation of
leases, we recognize that since the \textit{Klawans} case was decided in 1961,
the foundations for that holding have been substantially eroded. The
\textit{Klawans} opinion cited Restatement of Property {\S}~410 as authority
for its holding. The current Restatement (Second) of Property {\S}~15.2 rejects
the \textit{Klawans} doctrine and now takes the position that: 
\begin{quote}
A restraint on alienation without the consent of the landlord of the tenant's
interest in the leased property is valid, but the landlord's consent to an
alienation by the tenant cannot be withheld unreasonably, unless a freely
negotiated provision in the lease gives the landlord an absolute right to
withhold consent.
\end{quote}

Another authority cited in \textit{Klawans} in support of its holding was 2 R.
Powell, Powell on Real Property. The most recent edition of that text now
states:
\begin{quote}
Thus, if a lease clause prohibited the tenant from transferring his or her
interest without the landlord's consent, the landlord could withhold consent
arbitrarily. This result was allowed because it was believed that the
objectives served by allowing the restraints outweighed the social evils
implicit in them, inasmuch as the restraints gave the landlord control over
choosing the person who was to be entrusted with the landlord's property and
was obligated to perform the lease covenants. It is doubtful that this
reasoning retains full validity today. Relationships between landlord and
tenant have become more impersonal and housing space (and in many areas,
commercial space as well) has become scarce. These changes have had an impact
on courts and legislatures in varying degrees. Modern courts almost universally
adopt the view that restrictions on the tenant's right to transfer are to be
strictly construed. (Footnotes omitted.) 
\end{quote}
2 R. Powell, Powell on Real Property \S~248[1] (1988).

Finally, in support of its decision in \textit{Klawans}, this Court noted that,
``although it, apparently, has not been passed upon in a great number of
jurisdictions, the decisions of the courts that have determined the question
are in very substantial accord.'' \textit{Klawans}, 169 A.2d at 679. This is no
longer true. Since \textit{Klawans}, the trend has been in the opposite
direction. ``The modern trend is to impose a standard of reasonableness on the
landlord in withholding consent to a sublease unless the lease expressly states
otherwise.'' \textit{Campbell v. Westdahl}, 715 P.2d 288, 292 (Ariz. Ct. App.
1985).\ldots

Traditional property rules favor the free and unrestricted right to alienate
interests in property. Therefore, absent some specific restriction in the
lease, a lessee has the right to freely alienate the leasehold interest by
assignment or sublease without obtaining the permission of the lessor. R.
Schoshinski, \textit{American Law of Landlord and Tenant} \S~5:6 (1980); 1
\textit{American Law of Property} \S~3.56 (1952).

Contractual restrictions on the alienability of leasehold interests are
permitted. R. Cunningham, W. Stoebuck, and D. Whitman, \textit{The Law of
Property} \S~12.40 (1984). Consequently, landlords often insert clauses that
restrict the lessee's common law right to freely assign or sublease.
\textit{Id}. Probably the most often used clause is a ``silent consent'' clause
similar to the provision in the instant case, which provides that the premises
may not be assigned or sublet without the written consent of the lessor.

In a ``silent consent'' clause requiring a landlord's consent to assign or
sublease, there is no standard governing the landlord's decision. Courts must
insert a standard. The choice is usually between 1) requiring the landlord to
act reasonably when withholding consent, or 2) permitting the landlord to act
arbitrarily and capriciously in withholding consent.

Public policy requires that when a lease gives the landlord the right to
withhold consent to a sublease or assignment, the landlord should act
reasonably, and the courts ought not to imply a right to act arbitrarily or
capriciously. If a landlord is allowed to arbitrarily refuse consent to an
assignment or sublease, for what in effect is no reason at all, that would
virtually nullify any right to assign or sublease.

Because most people act reasonably most of the time, tenants might expect that a
landlord's consent to a sublease or assignment would be governed by standards
of reasonableness. Most tenants probably would not understand that a clause
stating ``this lease may not be assigned or sublet without the landlord's
written consent'' means the same as a clause stating ``the tenant shall have no
right to assign or sublease.'' Some landlords may have chosen the former
wording rather than the latter because it vaguely implies, but does not grant
to the tenant, the right to assign or sublet.

There are two public policy reasons why the law enunciated in \textit{Klawans}
should now be changed. The first is the public policy against restraints on
alienation. The second is the public policy which implies a covenant of good
faith and fair dealing in every contract.

Because there is a public policy against restraints on alienation, if a lease is
silent on the subject, a tenant may freely sublease or assign. Restraints on
alienation are permitted in leases, but are looked upon with disfavor and are
strictly construed. \textit{Powell on Real Property, supra.} If a clause in a
lease is susceptible of two interpretations, public policy favors the
interpretation least restrictive of the right to alienate freely. Interpreting
a ``silent consent'' clause so that it only prohibits subleases or assignments
when a landlord's refusal to consent is reasonable, would be the interpretation
imposing the least restraint on alienation and most in accord with public
policy.

Since the \textit{Klawans} decision, this Court has recognized that in a lease,
as well as in other contracts, ``there exists an implied covenant that each of
the parties thereto will act in good faith and deal fairly with the others.''
\textit{Food Fair v. Blumberg}, A.2d 166, 174 (1964). When the lease gives the
landlord the right to exercise discretion, the discretion should be exercised
in good faith, and in accordance with fair dealing; if the lease does not spell
out any standard for withholding consent, then the implied covenant of good
faith and fair dealing should imply a reasonableness standard.

We are cognizant of the value of the doctrine of \textit{stare decisis}, and of
the need for stability and certainty in the law. However, as we noted in
\textit{Harrison v. Mont. Co. Bd. of Educ.}, 456 A.2d 894, 903 (1983), a common
law rule may be modified ``where we find, in light of changed conditions or
increased knowledge, that the rule has become unsound in the circumstances of
modern life, a vestige of the past, no longer suitable to our people.'' The
\textit{Klawans} common law interpretation of the ``silent consent'' clause
represents such a ``vestige of the past,'' and should now be changed. 

\readinghead{Reasonableness of Withheld Consent}

In the instant case, we need not expound at length on what constitutes a
reasonable refusal to consent to an assignment or sublease. We should, however,
point out that obvious examples of reasonable objections could include the
financial irresponsibility or instability of the transferee, or the
unsuitability or incompatibility of the intended use of the property by the
transferee. We also need not expound at length on what would constitute an
unreasonable refusal to consent to an assignment or sublease. If the reasons
for withholding consent have nothing to do with the intended transferee or the
transferee's use of the property, the motivation may be suspect. Where, as
alleged in this case, the refusal to consent was solely for the purpose of
securing a rent increase, such refusal would be unreasonable unless the new
subtenant would necessitate additional expenditures by, or increased economic
risk to, the landlord. 

\readinghead{Prospective Effect}

The tenants ask us to retroactively overrule \textit{Klawans}, and hold that in
all leases with ``silent consent'' clauses, no matter when executed, consent to
assign or sublease may not be unreasonably withheld by a landlord. We decline
to do so. In the absence of evidence to the contrary, we should assume that
parties executing leases when \textit{Klawans} governed the interpretation of
``silent consent'' clauses were aware of \textit{Klawans} and the implications
drawn from the words they used. We should not, and do not, rewrite these
contracts.

In appropriate cases, courts may ``in the interest of justice'' give their
decisions only prospective effect. Contracts are drafted based on what the law
is; to upset such transactions even for the purpose of improving the law could
be grossly unfair.\ldots

For leases with ``silent consent'' clauses which were entered into before the
mandate in this case, \textit{Klawans} is applicable, and we assume the parties
were aware of the court decisions interpreting a ``silent consent'' clause as
giving the landlord an unrestricted right to withhold consent.

For leases entered into after the mandate in this case, if the lease contains a
``silent consent'' clause providing that the tenant must obtain the landlord's
consent in order to assign or sublease, such consent may not be unreasonably
withheld. If the parties intend to preclude any transfer by assignment or
sublease, they may do so by a freely negotiated provision in the lease. . . .
For example, the clause might provide, ``consent may be withheld in the sole
and absolute subjective discretion of the lessor.''

The final question is whether the tenants in the instant case, having argued
successfully for a change in the law, should receive the benefit of the
change.\ldots
[Even though our decision is to have only prospective effect] [t]he
tenants in the instant case should get the benefit of the interpretation of the
``silent consent'' clause that they so persuasively argued for, unless this
interpretation would be unfair to the landlord. We note that the tenants
testified they were told that the clause was only to prevent subleasing to
``someone who would tear the apartment up.'' Therefore, we will reverse the
judgment of the Circuit Court with instructions to vacate the judgment of the
District Court and remand for a new trial. At that trial, the landlord will
have the burden of establishing that it would be unfair to interpret the
``silent consent'' clause in accordance with our decision that a landlord must
act reasonably in withholding consent. He may establish that it would be unfair
to do so by establishing that when executing the lease he was aware of and
relied on the \textit{Klawans} interpretation of the ``silent consent''
clause.\ldots

