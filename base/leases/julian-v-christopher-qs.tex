\expected{julian-v-christopher}

\item \textbf{Landlords love restrictions.} Why are restrictions on transfer so
common in both commercial and residential leases?  You might want to refer back
to the Sprawl-Mart example from earlier in the chapter, which makes clear why a
landlord and tenant might disagree about who should get the benefit of the
remaining term.


\item \textbf{Status of the \textit{Julian} rule.} The
approach taken in \textit{Julian}, which reads a reasonableness requirement
into the lease, is still a minority rule.  Roughly 15 states have taken a
position similar to Maryland's highest court, including California, Illinois,
North Carolina, and Ohio.  Although the \textit{Julian}/minority approach has
gained popularity in the last two decades (and is considered the ``modern''
rule), it's important to note that in most states a landlord may still
arbitrarily refuse to consent to any sublease or assignment under a ``silent
consent'' clause.


\item \textbf{Contracting around the rule?} Imagine a lease that includes the
following provision: ``The tenant shall not sublease or assign any part of
their interest in the property without the Landlord's written permission.  The
Landlord reserves the absolute right to deny any request for any and all
reasons at his sole and absolute discretion.''  Under the holding in
\textit{Julian}, would this clause be valid?  \textit{See} Restatement (Second)
Property \S~15.2 (``A restraint on alienation with the consent of the
landlord of the tenant's interest in the leased property is valid, but the
landlord's consent to an alienation by the tenant cannot be withheld
unreasonably, unless a freely negotiated provision in the lease gives the
landlord an absolute right to withhold consent.'')


\item \textbf{Defining reasonableness.} What counts as a reasonable objection to
a sublease or assignment request?  Courts in Illinois have found that it's
proper to consider: (1) the sublessee's credit history, (2) the sublessee's
capital on hand, (3) whether the subleesee's business is compatible with
landlord's other properties, (4) whether the sublessee's business will compete
with those of the leassor or any other lessee, and, (5) the subleesee's
expertise and business plan.  See, for example, \textit{Jack Frost Sales, Inc.
v. Harris Trust \& Savings Bank}, 433 N.E. 2d 941 (Ill. App. 3d 1982). In most
jurisdictions, tenants have the burden to show the sublessee or assignee meets
the reasonable commercial standard.


\item \textbf{The Landlord's Stance.} Is the reasonableness rule fair
to landlords?  Imagine you're a landlord and your original tenant announces
that they're moving out and proffers a subleasee for your approval.  If you're
not completely satisfied with the new tenant, should you object?  If you say
``no'' and the tenant either leaves or sues you, how much will that enforcement
action cost?  


\item \textbf{Residential v. Commercial.} Courts have not imposed the
rule articulated in \textit{Julian} on residential tenants.  Why not? Aren't
commercial tenants better able to protect themselves and bargain than
residential tenants? Consider the following statute from a jurisdiction where
residential leases account for a huge proportion of extremely scarce housing
stock: New York. As you read it, consider whether and to what extent the
statute permits parties to residential leases to contract around its
provisions, and whether it is more or less restrictive than the rule of
\textit{Julian}.

