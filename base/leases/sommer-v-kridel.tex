\reading{Sommer v. Kridel}

\readingcite{378 A.2d 767 (N.J. 1977)}

\opinion \textsc{Pashman}, J.

We granted certification in these cases to consider whether a landlord seeking
damages from a defaulting tenant is under a duty to mitigate damages by making
reasonable efforts to re-let an apartment wrongfully vacated by the tenant.
Separate parts of the Appellate Division held that, in accordance with their
respective leases, the landlords in both cases could recover rents due under
the leases regardless of whether they had attempted to re-let the vacated
apartments. Although they were of different minds as to the fairness of this
result, both parts agreed that it was dictated by \textit{Joyce v. Bauman}, 174
A. 693 (1934)\ldots.  We now reverse and hold that a landlord does have an
obligation to make a reasonable effort to mitigate damages in such a situation.
We therefore overrule \textit{Joyce v. Bauman} to the extent that it is
inconsistent with our decision today.

\readinghead{I}

This case was tried on stipulated facts. On March 10, 1972 the defendant, James
Kridel, entered into a lease with the plaintiff, Abraham Sommer, owner of the
``Pierre Apartments'' in Hackensack, to rent apartment 6-L in that building.
The term of the lease was from May 1, 1972 until April 30, 1974, with a rent
concession for the first six weeks, so that the first month's rent was not due
until June 15, 1972.

\captionedgraphic{leases-10}{The Pierre Apartments today}

One week after signing the agreement, Kridel paid Sommer \$690. Half of that sum
was used to satisfy the first month's rent. The remainder was paid under the
lease provision requiring a security deposit of \$345. Although defendant had
expected to begin occupancy around May 1, his plans were changed. He wrote to
Sommer on May 19, 1972, explaining:
\begin{quotation}
I was to be married on June 3, 1972. Unhappily the engagement was broken and the
wedding plans cancelled. Both parents were to assume responsibility for the
rent after our marriage. I was discharged from the U.S. Army in October 1971
and am now a student. I have no funds of my own, and am supported by my
stepfather.

In view of the above, I cannot take possession of the apartment and am
surrendering all rights to it. Never having received a key, I cannot return
same to you.

I beg your understanding and compassion in releasing me from the lease, and will
of course, in consideration thereof, forfeit the 2 month's rent already paid.

Please notify me at your earliest convenience.
\end{quotation}
Plaintiff did not answer the letter.

Subsequently, a third party went to the apartment house and inquired about
renting apartment 6-L. Although the parties agreed that she was ready, willing
and able to rent the apartment, the person in charge told her that the
apartment was not being shown since it was already rented to Kridel. In fact,
the landlord did not re-enter the apartment or exhibit it to anyone until
August 1, 1973. At that time it was rented to a new tenant for a term beginning
on September 1, 1973. The new rental was for \$345 per month with a six week
concession similar to that granted Kridel.

Prior to re-letting the new premises, plaintiff sued Kridel in August 1972,
demanding \$7,590, the total amount due for the full two-year term of the
lease. Following a mistrial, plaintiff filed an amended complaint asking for
\$5,865, the amount due between May 1, 1972 and September 1, 1973. The amended
complaint included no reduction in the claim to reflect the six week concession
provided for in the lease or the \$690 payment made to plaintiff after signing
the agreement. Defendant filed an amended answer to the complaint, alleging
that plaintiff breached the contract, failed to mitigate damages and accepted
defendant's surrender of the premises. He also counterclaimed to demand
repayment of the \$345 paid as a security deposit.

The trial judge ruled in favor of defendant. Despite his conclusion that the
lease had been drawn to reflect ``the `settled law' of this state,'' he found
that ``justice and fair dealing'' imposed upon the landlord the duty to attempt
to re-let the premises and thereby mitigate damages. He also held that
plaintiff's failure to make any response to defendant's unequivocal offer of
surrender was tantamount to an acceptance, thereby terminating the tenancy and
any obligation to pay rent. As a result, he dismissed both the complaint and
the counterclaim. The Appellate Division reversed in a per curiam opinion, 153
N.J.Super. 1 (1976), and we granted certification.\ldots

\readinghead{II}

As the lower courts in both appeals found, the weight of authority in this State
supports the rule that a landlord is under no duty to mitigate damages caused
by a defaulting tenant. \textit{See Joyce v. Bauman}, \textit{supra}\ldots.
This rule has been followed in a majority of states\ldots and has been
tentatively adopted in the American Law Institute's Restatement of
Property.\ldots

Nevertheless, while there is still a split of authority over this question, the
trend among recent cases appears to be in favor of a mitigation
requirement.\ldots

The majority rule is based on principles of property law which equate a lease
with a transfer of a property interest in the owner's estate. Under this
rationale the lease conveys to a tenant an interest in the property which
forecloses any control by the landlord; thus, it would be anomalous to require
the landlord to concern himself with the tenant's abandonment of his own
property. \textit{Wright v. Baumann}, 398 P.2d 119, 120-21 (Or. 1965).

For instance, in \textit{Muller v. Beck}, supra, where essentially the same
issue was posed, the court clearly treated the lease as governed by property,
as opposed to contract, precepts. The court there observed that the ``tenant
had an estate for years, but it was an estate qualified by this right of the
landlord to prevent its transfer,'' 110 A. at 832, and that ``the tenant has an
estate with which the landlord may not interfere.'' \textit{Id.} at 832.
Similarly, in \textit{Heckel v. Griese}, \textit{supra}, the court noted the
absolute nature of the tenant's interest in the property while the lease was in
effect, stating that ``when the tenant vacated,\ldots no one, in the
circumstances, had any right to interfere with the defendant's possession of
the premises.'' 171 A. 148, 149. Other cases simply cite the rule announced in
\textit{Muller v. Beck}, \textit{supra}, without discussing the underlying
rationale. \textit{See Joyce v. Bauman}, \textit{supra}, 174 A. 693\ldots.

Yet the distinction between a lease for ordinary residential purposes and an
ordinary contract can no longer be considered viable. As Professor Powell
observed, evolving ``social factors have exerted increasing influence on the
law of estates for years.'' 2 \textit{Powell on Real Property} (1977 ed.),
\S~221(1) at 180-81. The result has been that:
\begin{quote}
[t]he complexities of city life, and the proliferated problems of modern society
in general, have created new problems for lessors and lessees and these have
been commonly handled by specific clauses in leases. This growth in the number
and detail of specific lease covenants has reintroduced into the law of estates
for years a predominantly contractual ingredient. 
\end{quote}
(\textit{Id.} at 181).\ldots

This Court has taken the lead in requiring that landlords provide housing
services to tenants in accordance with implied duties which are hardly
consistent with the property notions expressed in \textit{Muller v. Beck},
\textit{supra}, and \textit{Heckel v. Griese}, \textit{supra}. \textit{See}
\textit{Braitman v. Overlook Terrace Corp.}, 346 A.2d 76 (1975) (liability for
failure to repair defective apartment door lock); \textit{Berzito v. Gambino},
308 A.2d 17 (1973) (construing implied warranty of habitability and covenant to
pay rent as mutually dependent); \textit{Marini v. Ireland}, 265 A.2d 526
(1970) (implied covenant to repair); \textit{Reste Realty Corp. v. Cooper}, 251
A.2d 268 (1969) (implied warranty of fitness of premises for leased purpose).
In fact, in \textit{Reste Realty Corp. v. Cooper}, \textit{supra}, we
specifically noted that the rule which we announced there did not comport with
the historical notion of a lease as an estate for years. 251 A.2d 268. And in
\textit{Marini v. Ireland}, \textit{supra}, we found that the ``guidelines
employed to construe contracts have been modernly applied to the construction
of leases.'' 265 A.2d at 532.

Application of the contract rule requiring mitigation of damages to a
residential lease may be justified as a matter of basic
fairness. Professor
McCormick first commented upon the inequity under the majority rule when he
predicted in 1925 that eventually:
\begin{quote}
the logic, inescapable according to the standards of a `jurisprudence of
conceptions' which permits the landlord to stand idly by the vacant, abandoned
premises and treat them as the property of the tenant and recover full rent,
[will] yield to the more realistic notions of social advantage which in other
fields of the law have forbidden a recovery for damages which the plaintiff by
reasonable efforts could have avoided. (McCormick, \textit{The Rights of the
Landlord Upon Abandonment of the Premises by the Tenant}, 23 Mich. L. Rev. 211,
221-22 (1925)).
\end{quote}
Various courts have adopted this position.

The pre-existing rule cannot be predicated upon the possibility that a landlord
may lose the opportunity to rent another empty apartment because he must first
rent the apartment vacated by the defaulting tenant. Even where the breach
occurs in a multi-dwelling building, each apartment may have unique qualities
which make it attractive to certain individuals. Significantly, in
\textit{Sommer v. Kridel}, there was a specific request to rent the apartment
vacated by the defendant; there is no reason to believe that absent this
vacancy the landlord could have succeeded in renting a different apartment to
this individual.

We therefore hold that antiquated real property concepts which served as the
basis for the pre-existing rule, shall no longer be controlling where there is
a claim for damages under a residential lease. Such claims must be governed by
more modern notions of fairness and equity. A landlord has a duty to mitigate
damages where he seeks to recover rents due from a defaulting tenant.

If the landlord has other vacant apartments besides the one which the tenant has
abandoned, the landlord's duty to mitigate consists of making reasonable
efforts to re-let the apartment. In such cases he must treat the apartment in
question as if it was one of his vacant stock.

As part of his cause of action, the landlord shall be required to carry the
burden of proving that he used reasonable diligence in attempting to re-let the
premises. We note that there has been a divergence of opinion concerning the
allocation of the burden of proof on this issue. \textit{See} Annot.,
\textit{supra}, \S~12 at 577. While generally in contract actions the breaching
party has the burden of proving that damages are capable of mitigation\ldots
here the landlord will be in a better position to demonstrate whether he
exercised reasonable diligence in attempting to re-let the premises.\ldots

\readinghead{III}

The \textit{Sommer v. Kridel} case presents a classic example of the unfairness
which occurs when a landlord has no responsibility to minimize damages. Sommer
waited 15 months and allowed \$4658.50 in damages to accrue before attempting
to re-let the apartment. Despite the availability of a tenant who was ready,
willing and able to rent the apartment, the landlord needlessly increased the
damages by turning her away. While a tenant will not necessarily be excused
from his obligations under a lease simply by finding another person who is
willing to rent the vacated premises, see, e.g., \textit{Reget v.
Dempsey-Tegler \& Co.}, 216 N.E.2d 500 (Ill. App.1966) (new tenant insisted on
leasing the premises under different terms); \textit{Edmands v. Rust \&
Richardson Drug Co.}, 77 N.E. 713 (Mass. 1906) (landlord need not accept
insolvent tenant), here there has been no showing that the new tenant would not
have been suitable. We therefore find that plaintiff could have avoided the
damages which eventually accrued, and that the defendant was relieved of his
duty to continue paying rent. Ordinarily we would require the tenant to bear
the cost of any reasonable expenses incurred by a landlord in attempting to
re-let the premises\ldots but no such expenses were incurred in this case.\ldots

In assessing whether the landlord has satisfactorily carried his burden, the
trial court shall consider, among other factors, whether the landlord, either
personally or  through an agency, offered or showed the apartment to any
prospective tenants, or advertised it in local newspapers. Additionally, the
tenant may attempt to rebut such evidence by showing that he proffered suitable
tenants who were rejected. However, there is no standard formula for measuring
whether the landlord has utilized satisfactory efforts in attempting to
mitigate damages, and each case must be judged upon its own facts.

Compare\ldots \textit{Carpenter v. Wisniewski}, 215 N.E.2d 882 (Ind. App.1966)
(duty satisfied where landlord advertised the premises through a newspaper,
placed a sign in the window, and employed a realtor); \textit{Re Garment Center
Capitol, Inc.}, 93 F.2d 667, 115 A.L.R. 202 (2 Cir. 1938) (landlord's duty not
breached where higher rental was asked since it was known that this was merely
a basis for negotiations); \textit{Foggia v. Dix}, 509 P.2d 412, 414 (Or. 1973)
(in mitigating damages, landlord need not accept less than fair market value or
``substantially alter his obligations as established in the pre-existing
lease''); \textit{with} \textit{Anderson v. Andy Darling Pontiac, Inc.}, 43
N.W.2d 362 (Wis. 1950) (reasonable diligence not established where newspaper
advertisement placed in one issue of local paper by a broker);\ldots
\textit{Consolidated Sun Ray, Inc. v. Oppenstein}, 335 F.2d 801, 811 (8 Cir.
1964) (dictum) (demand for rent which is ``far greater than the provisions of
the lease called for'' negates landlord's assertion that he acted in good faith
in seeking a new tenant).

\readinghead{IV}

The judgment in \textit{Sommer v. Kridel} is reversed.

