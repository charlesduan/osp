\expected{narrative-tenant-selection}

\item William Neithamer, who is gay and HIV positive attempted to rent an
apartment from Brenneman Properties.  Neithamer did not reveal his HIV status,
but admitted to the property manager that he had dismal credit because he had
recently devoted all of his resources to taking care of a lover who had died of
AIDS.  Neithamer, however, offered to pre-pay one year's rent.  Brenneman
Properties rejected Neithamer's application and, in turn, Neithamer sued under
the FHA.  Does he have a case?  \textit{See} \textit{Neithamer v. Brenneman
Property Services, Inc.}, 81 F. Supp 2d 1 (D.D.C. 1999).  


\item Over the phone, Landlord said to Plaintiff, ``Do you have children? I
don't want any little boys because they'll mess up the house and nobody would
be here to watch them.  Really, this house isn't good for kids because it's
right next to a main road.''  Plaintiff sues. Landlord argues that her
statements only show that she is concerned about the welfare of children.  Is
that a legitimate non-discriminatory reason to refuse to rent?  


\item A local government has decided to knock down two high-rise public housing
projects within its borders.  The high-rises primarily house recent immigrants
from Guatemala.  A local advocacy group brings a lawsuit on their behalf,
claiming that the government action has a disparate impact on a protected
group.  Is this a disparate treatment or disparate impact case? Can you think
of a non-discriminatory reason why the government may have taken such an
action?


\item The FHA requires landlords to make ``reasonable accommodations'' for
individuals with handicaps.  Which of the following requests by a tenant would
qualify as a reasonable accommodation? (a) Asking a landlord with a
first-come/first-served parking policy to create a reserved parking space for a
tenant who has difficulty walking; (b) Requesting that a landlord waive parking
fees for a disabled tenant's home health care aide; (c) Asking the landlord to
make an exception to the building's ``no pets'' rule for a tenant with a
service animal; (d) Requesting landlord to pay for a sign language interpreter
for a deaf individual during the application process; (e) Asking the landlord
to provide oral reminders to pay the rent for a tenant with documented
short-term memory loss. 

