\expected{hilder-v-st-peter}

\item \textbf{Residential v. commercial.} Unlike the covenant of quiet
enjoyment, the implied warranty of habitability only applies to residential
leases.  Commercial tenants still largely operate under common-law legal rules.
 Commonly, commercial landlords and tenants do not rely on the default rules,
but rather assign the duty of upkeep and repair with an express provision in
the lease. 


\item \textbf{What is habitability?} Do all defects in an apartment
amount to violations?  What is the standard of habitability as laid out in
\textit{Hilder}?  


\item \textbf{Paternalism?} Is the implied warranty of habitability too
paternalistic?  Some economists argue that the poorest Americans should have
more freedom over how they spend their limited dollars.  Isn't it possible that
some individuals might want to occupy a really cheap (if slightly dangerous)
dwelling so that they have more money to spend on healthy foods,
transportation, and clothes?  Would it matter if the evidence showed that such
apartments were in fact cheaper than ``habitable'' apartments?


\item \textbf{Necessary?} Do you agree with the arguments made by the court in
\textit{Hilder} about the necessity of the implied warranty of habitability? 
Don't landlords already have excellent incentives to maintain their buildings?

\defwebsite{vice-arkansas-worst}{
title=Arkansas: The Worst Place to Rent in America,
journal=Vice News,
date=jun 24 2014,
url=https://www.youtube.com/watch?v=9G2Pk2JZP-E,
}

\item \textbf{Arkansas and beyond.}  As mentioned above, Arkansas is the one
state that has not adopted the implied warranty of habitability---either by
statute or judicial fiat.  Is Arkansas a Mad Max-style hellscape for renters? 
Are tenants there worse (or worse off) than the tenants in other states?  Some
people think so.  \textit{Vice} magazine recently dubbed Arkansas, ``The Worst
Place to Rent in America.'' \sentence{see vice-arkansas-worst}.
But does the implied
warranty of habitability provide much practical protection?  Do poor tenants
know about it?  Do they have the resources to push back against aggressive
landlords who threaten lawsuits and other forms of retaliation?  Professor
David Super has suggested that the decision of tenants' rights movement to
focus on habitability over affordability and overcrowding was a strategic
mistake.  \textit{See} David A. Super, \textit{The Rise and Fall of the Implied
Warranty of Habitability}, 99 \textsc{Cal. L. Rev}. 389-463 (2011).  Is there a
nirvana for renters anywhere?  


\item \textbf{Procedure \& remedies.} If a tenant believes his apartment does
not meet the standard of habitability, he must first must notify the landlord
of the defects and give the landlord a reasonable amount of time to cure the
problems.  If the landlord either cannot or will not make repairs, the implied
warranty of habitability offers the renter a menu of options.  Each option
presents a different combination of costs and risks to the tenant. If the
landlord breaches, the tenant may:
\begin{enumerate}
\item \textit{Leave, terminate contract}.  The tenant may consider the lease
terminated and move out.  
\item \textit{Stay and sue for damages}. As with the covenant of quiet
enjoyment, a tenant may stay in the unit and pay rent, while suing the landlord
for damages.  There is significant disagreement among jurisdictions about how
to calculate damages.  In \textit{Hilder}, the court uses the difference
between the rental price of the dwelling if it met the standard of habitability
and the value of the dwelling as it exists; the rent charged is not evidence of
actual value, but rather evidence of the appropriate price if it met the
standard of habitability.  [Note that given the court's calculation, the value
was apparently zero?] Other courts look at the difference between the amount of
rent stated in the lease and the fair market value of the premises.  What is
the better approach?  Should the rent charged be considered evidence of fair
market value?  If not, why not?
\item \textit{Stay and charge the cost of repair}.  A tenant has the option to
fix the defect and then deduct the cost of repair from the rent.  
\item \textit{Stay and withhold rent}.  In most jurisdictions, a tenant can
withhold the entire rent for violations of the implied warranty of habitability
(although, a cautious tenant should pay the rent into an escrow account).  This
is a very powerful remedy.  First, it gives the landlord strong incentive to
respond to valid complaints from tenants.  Second, it puts the burden on the
landlord (rather than the tenant) to initiate a lawsuit when contested issues
arise.  Finally, if the landlord does move to evict the tenant for non-payment,
violations of the implied warranty of habitability can serve as a defense.
\item \textit{Extreme violations}.  Tenants have won punitive damages in cases
where the landlord committed repeated or gruesome violations of the implied
warranty.  
\end{enumerate}

