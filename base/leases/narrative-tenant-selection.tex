
As we saw earlier in the textbook, the right to exclude remains a cornerstone of
property ownership.  Owners have expansive power to keep others off of their
land and out of their homes.  Generally speaking, this right extends to
landlords, who have broad discretion to select tenants as they see fit. 
Landlords, for example, remain free to exclude smokers from their properties. 
They can also refuse to rent to a tenant who acts erratically, possesses a
criminal record, or has a low credit score. Landlords, however, cannot violate
state or federal anti-discrimination laws when they go through the leasing
process.  

\paragraph{The Civil Rights Act of 1866}

One of the oldest laws that protects tenants against discrimination in the
housing market is the Civil Rights Act of 1866.  Passed in the aftermath of the
Civil War, the Civil Rights Act of 1866 prohibits all discrimination based on
race in the purchase or rental of real or personal property.  \textit{See Jones
v. Alfred H. Mayer Co.}, 392 U.S. 409 (1968).  Thus, landlords cannot deny an
apartment unit to a potential tenant based on tenant's heritage or the color of
their skin.  There are no exceptions.

\paragraph{The Fair Housing Act of 1968}

The Fair Housing Act of 1968 (and its many amendments) greatly expanded the
number of individuals covered by anti-discrimination law.  Broadly speaking,
the Fair Housing Act (FHA) prohibits discrimination in the renting, selling,
advertising, or financing of real estate on the basis of race, color, national
origin, religion, sex, familial status, and disability.  It is worth looking
closely at some of its provisions. The Act begins with a statement of policy
and a few (counter-intuitive) definitions:

\begin{quotation}
\S~3601.  \textsc{Declaration of Policy}.---It is the policy of the United
States to provide, within constitutional limitations, for fair housing
throughout the United States.

\S~3602.  \textsc{Definitions}.---As used in this subchapter\ldots
\begin{statute}
\item (c) ``Family'' includes a single individual.\ldots

\item (h) ``Handicap'' means, with respect to a person---

\begin{statute}
\item (1) a physical or mental impairment which substantially limits one or
more of such person's major life activities,

\item (2) a record of having such an impairment, or

\item (3) being regarded as having such an impairment, but such term does not
include current, illegal use of or addiction to a controlled substance (as
defined in section 802 of title 21).\ldots
\end{statute}
\item (k) ``Familial status'' means one or more individuals (who
have not attained the age of 18 years) being domiciled with---
\begin{statute}
\item (1) a parent or another person having legal custody of such individual or
individuals; or  

\item (2) the designee of such parent or other person having such custody, with
the written permission of such parent or other person. 
\end{statute}
The protections afforded against discrimination on the basis of familial status
shall apply to any person who is pregnant or is in the process of securing
legal custody of any individual who has not attained the age of 18 years.
\end{statute}
\end{quotation}

The definition of ``familial status'' surprises many students. Whom, exactly,
does it protect? Unmarried people? Single mothers? Although more intuitive, the
definition of handicap has generated a number of legal disputes.  Alcohol, for
example, is not a controlled substance under section 802 of title 21. Does that
mean that a landlord cannot refuse to rent to a person who drinks heavily or
sounds very drunk (and belligerent) over the phone?  

The real meat of the Fair Housing act comes in \S~3604.  The first subsection
makes it unlawful to ``refuse to sell or rent\ldots or otherwise make
unavailable'' a ``dwelling'' to any person because of race, color, religion,
sex, familial status, or national origin. \textit{See} 42 U.S.C. \S~3604(a). 
Later sections provide similar protections for the handicapped.  The Act then
takes a number of additional steps designed to eliminate discrimination from
the housing market.  Under the terms of the law it is illegal to:
\begin{enumerate}
\item discriminate in the terms or conditions of a sale or rental [\S~3604(b)];

\item create or publish an advertisement or statement that express a preference
or hostility toward individuals in any of the protected categories
[\S~3604(c)];
\item  lie about or misrepresent the availability of housing [\S~3604(d)];
\item refuse to permit handicapped tenants from making reasonable modifications
of the existing premise at their own expense [\S~3604(f)(3)(A)];
\item refuse to make reasonable accommodations in rules and policies to
accommodate individuals with handicaps [\S~3604(f)(3)(B)];
\item Harass or intimidate persons in their enjoyment of a dwelling [\S~3617]. 
\end{enumerate}
Unlike the Civil Rights Act of 1866, the Fair Housing Act does contain a number
of important exemptions. Section 3607(b), for example, allows housing
designated for older persons to bar families with young children.  Similarly,
section 3607(a) allows religious organizations and private clubs to give
preferences to their own members.  The most controversial exemption, reproduced
below, is the so-called Mrs. Murphy exemption:
\begin{quotation}
(b) Nothing in section 3604 of this title (other than subsection (c)) shall
apply to---
\begin{statute}
\item (1) any single-family house sold or rented by an owner:
\begin{statute}
\item Provided, That such private individual owner does not own more than three
such single-family houses at any one time:
\item Provided further, That in the case of the sale of any such single-family
house by a private individual owner not residing in such house at the time of
such sale or who was not the most recent resident of such house prior to such
sale, the exemption granted by this subsection shall apply only with respect to
one such sale within any twenty-four month period:
\item Provided further, That such bona fide private individual owner does not
own any interest in, nor is there owned or reserved on his behalf, under any
express or voluntary agreement, title to or any right to all or a portion of the
proceeds from the sale or rental of, more than three such single-family houses
at any one time:
\item Provided further, That after December 31, 1969, the sale or rental of any
such single-family house shall be excepted from the application of this
subchapter only if such house is sold or rented
\begin{statute}
\item (A) without the use in any manner of the sales or rental facilities or the
sales or rental services of any real estate broker, agent, or salesman, or of
such facilities or services of any person in the business of selling or renting
dwellings, or of any employee or agent of any such broker, agent, salesman, or
person and
\item (B) without the publication, posting or mailing, after notice, of any
advertisement or written notice in violation of section 3604(c) of this title;
\end{statute}
but nothing in this proviso shall prohibit the use of attorneys, escrow agents,
abstractors, title companies, and other such professional assistance as
necessary to perfect or transfer the title, or
\end{statute}

\item (2) rooms or units in dwellings containing living quarters occupied or
intended to be occupied by no more than four families living independently of
each other, if the owner actually maintains and occupies one of such living
quarters as his residence.
\end{statute}
\end{quotation}
42 U.S.C. \S~3603(b) (paragraph breaks added).
What does this exemption allow?  If the act is intended to root out pernicious
discrimination, why include this provision?  

It is crucial to note that the plain text of the Mrs. Murphy exemption states
that it does not apply to 3604(c)---the subsection that prohibits
discriminatory advertising.  Thus, although certain categories of landlords are
exempted from the statute's basic framework, they are still not allowed to post
discriminatory advertisements.

\paragraph{State Anti-Discrimination Efforts}

Some state legislatures have passed laws that afford far more protection from
discrimination than the federal statutes provide.  Minnesota, for example,
protects against housing discrimination on the basis of sexual orientation,
gender identity, marital status, and source of income.  Other states in the
Northeast and West Coast provide similar coverage, but these positions are in
no way a majority.  As the map in Figure~\ref{f:leases-02} indicates, in most
states nothing prevents a landlord from denying an apartment to an engaged
heterosexual couple, based on the belief that cohabitation before marriage is
sinful.

\defbook{nfha-modernizing-fair}{
instauth=National Fair Housing Alliance,
title=Modernizing the Fair Housing Act for the 21st Century: 2013 Fair Housing
Trends Report,
date=apr 11 2013,
url=https://nationalfairhousing.org/wp-content/uploads/2017/04/2013_trends_report.pdf,
}


\captionedgraphic{leases-02}{States (in blue) that lack laws protecting against
discrimination based on marital status.
\protect\fullcite{nfha-modernizing-fair}\protect\sentence{nfha-modernizing-fair
at 13}.}

Despite the small number of states that extend protection against housing
discrimination on the basis of sexual orientation or transgender status, recent
changes in the interpretation of federal antidiscrimination law may make state
protections a moot point. In June of 2020, the Supreme Court held that an
employer who fires an employee for their sexual orientation or transgender
status unlawfully discriminates on the basis of sex for purposes of Title VII
of the Civil Rights Act of 1964. \textit{See} 42 U.S.C. {\S} 2000e--2(a)(1)
(prohibiting employment discrimination ``because of\dots sex.''). As the
Court explained:
\begin{quote}
[I]t is impossible to discriminate against a person for being homosexual or
transgender without discriminating against that individual based on sex.
Consider, for example, an employer with two employees, both of whom are
attracted to men. The two individuals are, to the employer's mind, materially
identical in all respects, except that one is a man and the other a woman. If
the employer fires the male employee for no reason other than the fact he is
attracted to men, the employer discriminates against him for traits or actions
it tolerates in his female colleague. Put differently, the employer
intentionally singles out an employee to fire based in part on the employee's
sex, and the affected employee's sex is a but-for cause of his discharge. Or
take an employer who fires a transgender person who was identified as a male at
birth but who now identifies as a female. If the employer retains an otherwise
identical employee who was identified as female at birth, the employer
intentionally penalizes a person identified as male at birth for traits or
actions that it tolerates in an employee identified as female at birth. Again,
the individual employee's sex plays an unmistakable and impermissible role in
the discharge decision.
\end{quote}
\emph{Bostock v. Clayton Cty., Ga.}, 140 S.Ct. 1731, 1741-42 (2020). Given that
the Fair Housing Act uses the same ``because of\dots sex'' language as Title
VII, it seems probable that the federal courts will conclude that sexual
orientation and transgender status are protected categories under the Fair
Housing Act as well. 

\paragraph{Proving Discrimination}

Two broad categories of cases may be brought under the FHA: disparate treatment
claims and disparate impact claims.  

\captionedgraphic{leases-03}{A sign erected by white homeowners trying to
prevent black tenants from moving into their Detroit neighborhood (1942).}

\textit{Disparate treatment} claims target intentional forms of discrimination,
including the refusal to rent based on one of the protected categories. A
plaintiff can show intent to discriminate with ``smoking gun'' style evidence,
such as statements by the landlord that he ``would never rent to an Irishman.''
Of course, modern landlords rarely make such forthright admissions.  As a
result, courts in the United States have established a burden-shifting approach
that allows plaintiffs to prove intentional discrimination with indirect
circumstantial evidence.  The initial burden is on the plaintiff to make a
\textit{prima facie} case of discrimination.  In a refusal to rent case, the
plaintiff must show that (1) that she is a member of a class protected by the
FHA; (2) that she applied for and was qualified to rent the unit; (3) that she
was rejected; and (4) the unit remained unrented.  Once the plaintiff has
established sufficient evidence to state a \textit{prima facie} case, the
burden shifts to the defendant landlord to proffer a legitimate
nondiscriminatory reason for the refusal to rent.  If the defendant meets this
requirement, the burden then shifts back to the tenant to prove that the reason
offered is a pretext.

Discrimination is often ferreted out through the use of ``testers.'' Advocacy
groups, many of which are funded by the federal government, will send
comparable white and black individuals to inquire about renting a vacant unit. 
If the landlord treats the testers differently (e.g., provides different levels
of assistance, shows different units, provides different information about unit
availability) this provides persuasive evidence of illegal discrimination.  

\textit{Disparate impact} claims allege that some seemingly neutral policy has a
disproportionately harmful effect on members of a group protected by the FHA. 
These cases rely heavily on statistical evidence and employ a very similar
burden-shifting methodology as the disparate treatment claims.  Using
statistics, plaintiffs need to show that a defendant's policy has actually
caused some disparity.  The defendant then has the opportunity to escape
liability if it can show show that its actions are necessary to achieve a valid
goal. See \textit{Texas Department of Housing \& Community Affairs v. Inclusive
Communities Project, Inc}., 135 S. Ct. 2507 (2015).

