\reading{Berg v. Wiley}

\readingcite{264 N.W.2d 145 (Minn. 1978)}

\opinion \textsc{Rogosheske}, Justice.

Defendant landlord, Wiley Enterprises, Inc., and
defendant Rodney A. Wiley (hereafter collectively
referred to as Wiley) appeal from a judgment upon a
jury verdict awarding plaintiff tenant, A Family Affair Restaurant, Inc.,
damages for wrongful eviction from its leased premises. The issues for review
are whether the evidence was sufficient to support the jury's finding that the
tenant did not abandon or surrender the premises and whether the trial court
erred in finding Wiley's reentry forcible and wrongful
as a matter of law. We hold that the jury's verdict is supported by sufficient
evidence and that the trial court's determination of unlawful entry was correct
as a matter of law, and affirm the judgment.

On November 11, 1970, Wiley, as lessor\ldots executed a
written lease agreement letting land and a building in Osseo, Minnesota, for
use as a restaurant. The lease provided a 5-year term beginning December 1,
1970, and specified that the tenant agreed to bear all costs of repairs and
remodeling, to ``make no changes in the building structure'' without prior
written authorization from Wiley, and to ``operate the
restaurant in a lawful and prudent manner.'' Wiley also
reserved the right ``at (his) option (to) retake possession'' of the premises
``(s)hould the Lessee fail to meet the conditions of this
Lease.'' In early 1971, plaintiff Kathleen
Berg took assignment of the lease from the prior
lessee, and on May 1, 1971, she opened ``A Family Affair Restaurant'' on the
premises. In January 1973, Berg incorporated the
restaurant and assigned her interest in the lease to ``A Family Affair
Restaurant, Inc.'' As sole shareholder of the corporation, she alone continued
to act for the tenant.

The present dispute has arisen out of Wiley's objection
to Berg's continued remodeling of the restaurant
without procuring written permission and her consequent operation of the
restaurant in a state of disrepair with alleged health code violations.
Strained relations between the parties came to a head in June and July 1973. In
a letter dated June 29, 1973, Wiley's attorney charged
Berg with having breached lease items 5 and 6 by making
changes in the building structure without written authorization and by
operating an unclean kitchen in violation of health regulations. The letter
demanded that a list of eight remodeling items be completed within 2 weeks from
the date of the letter, by Friday, July 13, 1973, or
Wiley would retake possession of the premises under
lease item 7. Also, a June 13 inspection of the restaurant by the Minnesota
Department of Health had produced an order that certain listed changes be
completed within specified time limits in order to comply with the health code.
The major items on the inspector's list, similar to those listed by
Wiley's attorney, were to be completed by July 15,
1973.

During the 2-week deadline set by both Wiley and the
health department, Berg continued to operate the
restaurant without closing to complete the required items of remodeling. The
evidence is in dispute as to whether she intended to permanently close the
restaurant and vacate the premises at the end of the 2 weeks or simply close
for about 1 month in order to remodel to comply with the health code. At the
close of business on Friday, July 13, 1973, the last day of the 2-week period,
Berg dismissed her employees, closed the restaurant,
and placed a sign in the window saying ``Closed for Remodeling.'' Earlier that
day, Berg testified, Wiley came
to the premises in her absence and attempted to change the locks. When she
returned and asserted her right to continue in possession, he complied with her
request to leave the locks unchanged. Berg also
testified that at about 9:30 p.m.\@ that evening, while she and four of her
friends were in the restaurant, she observed Wiley
hanging from the awning peering into the window. Shortly thereafter, she heard
Wiley pounding on the back door demanding admittance.
Berg called the county sheriff to come and preserve
order. Wiley testified that he observed
Berg and a group of her friends in the restaurant
removing paneling from a wall. Allegedly fearing destruction of his property,
Wiley called the city police, who, with the sheriff,
mediated an agreement between the parties to preserve the status quo until each
could consult with legal counsel on Monday, July 16, 1973.

Wiley testified that his then attorney advised him to
take possession of the premises and lock the tenant out. Accompanied by a
police officer and a locksmith, Wiley entered the
premises in Berg's absence and without her knowledge on
Monday, July 16, 1973, and changed the locks. Later in the day,
Berg found herself locked out. The lease term was not
due to expire until December 1, 1975. The premises were re-let to another
tenant on or about August 1, 1973. Berg brought this
damage action against Wiley\ldots [for] intentional
infliction of emotional distress\ldots and other tort damages based upon claims
in wrongful eviction.\ldots Wiley answered with an
affirmative defense of abandonment and surrender and counterclaimed for damage
to the premises.\ldots With respect to the wrongful eviction claim, the trial
court found as a matter of law that Wiley did in fact
lock the tenant out, and that the lockout was wrongful.

The jury, by answers to the questions submitted, found no liability on
Berg's claim for intentional infliction of emotional
distress and no liability on Wiley's counterclaim for
damages to the premises, but awarded Berg \$31,000 for
lost profits and \$3,540 for loss of chattels resulting from the wrongful
lockout. The jury also specifically found that Berg
neither abandoned nor surrendered the premises.\ldots 

On this appeal, Wiley seeks an outright reversal of the
damages award for wrongful eviction, claiming insufficient evidence to support
the jury's finding of no abandonment or surrender and claiming error in the
trial court's finding of wrongful eviction as a matter of law.

The first issue before us concerns the sufficiency of evidence to support the
jury's finding that Berg had not abandoned or
surrendered the leasehold before being locked out by
Wiley. Viewing the evidence to support the jury's
special verdict in the light most favorable to Berg, as
we must, we hold it amply supports the jury's finding of no
abandonment or surrender of the premises. While the evidence bearing upon
Berg's intent was strongly contradictory, the jury
could reasonably have concluded, based on Berg's
testimony and supporting circumstantial evidence, that she intended to retain
possession, closing temporarily to remodel. Thus, the lockout cannot be excused
on ground that Berg abandoned or surrendered the
leasehold.

The second and more difficult issue is whether Wiley's
self-help repossession of the premises by locking out
Berg was correctly held wrongful as a matter of law.

Minnesota has historically followed the common-law rule that a landlord may
rightfully use self-help to retake leased premises from a tenant in possession
without incurring liability for wrongful eviction provided two conditions are
met: (1) The landlord is legally entitled to possession, such as where a tenant
holds over after the lease term or where a tenant breaches a lease containing a
reentry clause; and (2) the landlord's means of reentry are peaceable.
\textit{Mercil v. Broulette}, 69 N.W. 218 (1896). Under the common-law rule, a
tenant who is evicted by his landlord may recover damages for wrongful eviction
where the landlord either had no right to possession or where the means used to
remove the tenant were forcible, or both. \textit{See, e. g.}, \textit{Poppen
v. Wadleigh}, 51 N.W.2d 75 (1952)\ldots. 

Wiley contends that Berg had
breached the provisions of the lease, thereby entitling
Wiley, under the terms of the lease, to retake
possession, and that his repossession by changing the locks in
Berg's absence was accomplished in a peaceful manner.
In a memorandum accompanying the post-trial order, the trial court stated two
grounds for finding the lockout wrongful as a matter of law: (1) It was not
accomplished in a peaceable manner and therefore could not be justified under
the common-law rule, and (2) any self-help reentry against a tenant in
possession is wrongful under the growing modern doctrine that a landlord must
always resort to the judicial process to enforce his statutory remedy against a
tenant wrongfully in possession. Whether Berg had in
fact breached the lease and whether Wiley was hence
entitled to possession was not judicially determined.\ldots 

In applying the common-law rule, we have not before had occasion to decide what
means of self-help used to dispossess a tenant in his absence will constitute a
nonpeaceable entry, giving a right to damages without regard to who holds the
legal right to possession. Wiley argues that only
actual or threatened violence used against a tenant should give rise to damages
where the landlord had the right to possession. We cannot agree.

It has long been the policy of our law to discourage landlords from taking the
law into their own hands, and our decisions and statutory law have looked with
disfavor upon any use of self-help to dispossess a tenant in circumstances
which are likely to result in breaches of the peace. We gave early recognition
to this policy in \textit{Lobdell v. Keene}, 88 N.W. 426, 430 (1901), where we
said:
\begin{quote}
The object and purpose of the legislature in the enactment of the forcible
entry and unlawful detainer statute was to prevent those claiming a right of
entry or possession of lands from redressing their own wrongs by entering into
possession in a violent and forcible manner. All such acts tend to a breach of
the peace, and encourage high-handed oppression. The law does not permit the
owner of land, be his title ever so good, to be the judge of his own rights
with respect to a possession adversely held, but puts him to his remedy under
the statutes.
\end{quote}

To facilitate a resort to judicial process, the legislature has provided a
summary procedure in Minn. St. 566.02 to 566.17 whereby a landlord may recover
possession of leased premises upon proper notice and showing in court in as
little as 3 to 10 days. As we recognized in \textit{Mutual Trust Life Ins. Co.
v. Berg}, 246 N.W. 9, 10 (1932), ``(t)he forcible entry and unlawful detainer
statutes were intended to prevent parties from taking the law into their own
hands when going into possession of lands and tenements\ldots.'' To further
discourage self-help, our legislature has provided treble damages for forcible
evictions, \S\S~557.08 and 557.09, and has provided additional criminal
penalties
for intentional and unlawful exclusion of a tenant. \S~504.25.
In \textit{Sweeney v. Meyers}, \textit{supra}, we allowed a business tenant not
only damages for lost profits but also punitive damages against a landlord who,
like Wiley, entered in the tenant's absence and locked the tenant out.

In the present case, as in Sweeney, the tenant was in possession, claiming a
right to continue in possession adverse to the landlord's claim of breach of
the lease, and had neither abandoned nor surrendered the premises. Wiley, well
aware that Berg was asserting her right to possession, retook possession in her
absence by picking the locks and locking her out. The record shows a history of
vigorous dispute and keen animosity between the parties. Upon this record, we
can only conclude that the singular reason why actual violence did not erupt at
the moment of Wiley's changing of the locks was Berg's absence and her
subsequent self-restraint and resort to judicial process. Upon these facts, we
cannot find Wiley's means of reentry peaceable under the common-law rule. Our
long-standing policy to discourage self-help which tends to cause a breach of
the peace compels us to disapprove the means used to dispossess Berg. To
approve this lockout, as urged by Wiley, merely because in Berg's absence no
actual violence erupted while the locks were being changed, would be to
encourage all future tenants, in order to protect their possession, to be
vigilant and thereby set the stage for the very kind of public disturbance
which it must be our policy to discourage.\ldots 

We recognize that the growing modern trend departs completely from the
common-law rule to hold that self-help is never available to dispossess a
tenant who is in possession and has not abandoned or voluntarily surrendered
the premises. Annotation, 6 A.L.R.3d 177, 186; 76 Dickinson L. Rev. 215, 227.
This growing rule is founded on the recognition that the potential for violent
breach of peace inheres in any situation where a landlord attempts by his own
means to remove a tenant who is claiming possession adversely to the landlord.
Courts adopting the rule reason that there is no cause to sanction such
potentially disruptive self-help where adequate and speedy means are provided
for removing a tenant peacefully through judicial process. At least 16 states
have adopted this modern rule, holding that judicial proceedings, including the
summary procedures provided in those states' unlawful detainer statutes, are
the exclusive remedy by which a landlord may remove a tenant claiming
possession.\ldots 

While we would be compelled to disapprove the lockout of Berg in her absence
under the common-law rule as stated, we approve the trial court's reasoning and
adopt as preferable the modern view represented by the cited cases. To make
clear our departure from the common-law rule for the benefit of future
landlords and tenants, we hold that, subsequent to our decision in this case,
the only lawful means to dispossess a tenant who has not abandoned nor
voluntarily surrendered but who claims possession adversely to a landlord's
claim of breach of a written lease is by resort to judicial process. We find
that Minn. St. 566.02 to 566.17 provide the landlord with an adequate remedy for
regaining possession in every such case. Where speedier action than provided in
\S\S~566.02 to 566.17 seems necessary because of threatened
destruction of the property or other exigent circumstances, a temporary
restraining order under Rule 65, Rules of Civil Procedure, and law enforcement
protection are available to the landlord. Considered together, these statutory
and judicial remedies provide a complete answer to the landlord. In our modern
society, with the availability of prompt and sufficient legal remedies as
described, there is no place and no need for self-help against a tenant in
claimed lawful possession of leased premises.

Applying our holding to the facts of this case, we conclude, as did the trial
court, that because Wiley failed to resort to judicial remedies against Berg's
holding possession adversely to Wiley's claim of breach of the lease, his
lockout of Berg was wrongful as a matter of law. The rule we adopt in this
decision is fairly applied against Wiley, for it is clear that, applying the
older common-law rule to the facts and circumstances peculiar to this case, we
would be compelled to find the lockout nonpeaceable for the reasons previously
stated. The jury found that the lockout caused Berg damage and, as between Berg
and Wiley, equity dictates that Wiley, who himself performed the act causing
the damage, must bear the loss. 

Affirmed.

