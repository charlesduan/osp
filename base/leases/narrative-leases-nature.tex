In its simplest form, the \term{lease} is a transfer in which the owner of real
property conveys exclusive possession to a tenant (generally in exchange for
rent).  Most law students know through personal experience that the process of
renting generally entails signing a lease contract.  Like other contracts, a
lease's terms can be negotiated and they explicitly govern many of the rights
and responsibilities of the parties involved.  So why then are leases discussed
in the property course rather than contracts?

The short response is that a lease is a property--contract hybrid. While it is
surely a contract, it's a contract for a very particular kind of property
interest. The fuller answer, like so much in property, lies in the history of
feudal land law.  Under the traditional common law, a leasehold was understood
primarily as a property interest, similar in nature to the estates covered in
our chapter on Estates and Future Interests.  A lord (often a baron) conveyed a
possessory right to a tenant (usually a peasant) and retained for himself a
future interest (typically a reversion). Importantly, once the landlord
transferred the right to possession, he had few other obligations to the
tenant.

This basic model survived until the 1960s, when many jurisdictions began to
introduce general contract law principles (e.g. the implied duty of good faith
and fair dealing) into the law of landlord-tenant.  Importing contract theories
into the lease has had two practical effects.  First, parties to a lease now
have the option to terminate in the case of \textit{any} material breach; in
the past tenants could only terminate if the landlord interfered with their
possession.  Second, modern tenants have far more protections from indifferent
and unscrupulous landlords than their counterparts 50 years ago.  Courts and
legislatures have proven particularly eager to help residential tenants---whom
they view as vulnerable---from predations of the free market. 

