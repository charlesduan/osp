If a tenant fails to pay rent or otherwise commits a material breach of the
lease, the landlord can elect to terminate the leasehold and
\term[eviction]{evict} the tenant
from the property.  It is undoubtedly true that the eviction process and the
subsequent scramble for a new place to live can be a traumatic, humiliating,
and disruptive occurrence.  Eviction displaces children from their schools,
rends the social networks of the poor, and forces many families into shelters
or onto the streets.  Matthew Desmond, a sociologist at Harvard, has found that
forced relocations are also shockingly common.  In Milwaukee, the location of
Desmond's research, 17 percent of the moves undertaken by renters over a
two-year period were forced relocations.  \textit{See} Matthew Desmond et al.,
\textit{Forced Relocation and Residential Instability Among Urban Renters}, 89
\textsc{Soc. Sci. Rev}. 227 (2015).  In response to the social cost of
eviction, some American cities and many countries around the world make it
difficult for landlords to remove tenants.  Should more U.S. jurisdictions
follow suit? Consider the following story:
\begin{quote}
A patient political scientist\dots might be able to place American cities on
a left-to-right spectrum according to how long tenants whose eviction has
become a cause manage to stay where they are.  It may be, for instance that
some city like Houston is on the far right of the spectrum.\ldots Houston's
most powerful citizens are known for a devotion to private property so intense
that they see routine planning and zoning as acts of naked confiscation.\ldots
San Francisco might qualify for the left end of the spectrum.  [I]ts best-known
evictees [are] the tenants of the run-down three-story building called the
International Hotel\ldots.  In the fall of 1968, about a hundred and fifty
people who were living in the hotel\ldots were told to be out of the building
by January 1, 1969.  The building was finally cleared---in what amounted to a
military operation requiring several hundred policemen---on August 4, 1977.  
\end{quote}
Calvin Trillin, \textit{Some Thoughts on the International Hotel Controversy},
New Yorker, Dec. 19, 1977, at 116.

\begin{questions}
\item Would you rather be a tenant in a place like Houston---where evictions
happen quickly---or in San Francisco---where they do not?  


\item Imagine you're a landlord in a jurisdiction where it takes a long time to
remove a tenant for non-payment of rent.  How would that change your business
strategy? Would you ever take a chance on a tenant with bad credit or a history
of being evicted? 
\end{questions}

We turn now to the procedure of eviction.  When a landlord believes that a
tenant has committed a material breach of the lease, how exactly does she go
about removing a lessee from the property? 
