A landlord's responsibility for injuries sustained on the leased premise has
dramatically expanded in the last 50 years.  As discussed in the previous
subsection, under the traditional common law rule, the tenant had the duty to
undertake all repairs and maintenance on the rented property.  As a result, the
law absolved landlords from liability for injuries sustained because of
dangerous conditions in the unit.  The costs of damage (to both property and
persons) sustained from rotted decks, falling plaster, and collapsing walls all
fell squarely on tenants.

Almost every jurisdiction now imposes greater duties on landlords.  At the very
least, landlords must exercise reasonable care in keeping common areas safe,
use reasonable care when making repairs, and warn tenants about latent
defects---unsafe conditions that would not be obvious upon an inspection. 
Other jurisdictions, following the logic of the implied warranty of
habitability, have gone farther.  They reason that since the landlord now has a
duty to provide tenants (and their guests) with safe and clean premises, a
failure to comply with this obligation may amount to negligence.  The basic
rule in these states is that a landlord must take reasonable steps to repair
defects of which the landlord becomes aware.  Failure to comply exposes
landlords to liability for injuries that result from the defective conditions.

Landlords sometimes attempt to avoid the obligation to repair by inserting into
the lease a clause stating that the lessor is not responsible for personal
injury or property damage that occurs on the premise.  While such exculpatory
clauses are typically upheld in commercial settings, courts increasingly strike
them from residential leases as violations of public policy. 

