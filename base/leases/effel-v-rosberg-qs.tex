\expected{effel-v-rosberg}

\item \textbf{The parties' intent?} When Henry and Jack Effel drafted the
settlement agreement transferring their property to Robert Rosberg, what where
they trying to accomplish?  Did the court carry out the intentions of the
parties?  Why?


\item \textbf{Other approaches.} In \textit{Garner v. Gerrish}, 473 N.E.2d 223
(N.Y. 1984), the New York Court of Appeals faced a case with very similar
facts.  The tenant, Lou Gerrish, had a lease stating, ``Lou Gerrish [sic] has
the privilege of termination [sic] this agreement at a date of his own
choice.''  The New York court found that the document created a new kind of
leasehold---a lease for life. The \textit{Garner} opinion attacked the argument
in favor of the tenancy at will as being grounded in the ``antiquated
notion[s]'' of medieval property law.  Is there any good reason for the law to
only recognize three leasehold tenancies?  What if, instead, the lease gave
only the \textit{landlord} the power to terminate, and required the tenant to
stay and pay as long as the landlord desired?


\item \textbf{Working within the system.} Could the lease have been drafted in a
way that would have let Lena Effel stay on the property for the duration of her
life or until she chose to move, as long as she kept paying the rent?


\item \textbf{Institutional competence.} Are courts or legislatures better
positioned to create new property forms?  


\item \textbf{The background story.} Lena Effel lived in the house owned by her
nephews for over 20 years.  Before that, her twin brother (Henry and Jack's
father) had lived in the home for many years.  At the time the compromise
settlement agreement was signed, Lena was 93 years old.  At the time Rosberg
sought to evict her, Lena was 97.  Should any of those facts have influenced
the judges in the case?

