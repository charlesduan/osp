\reading{New York Real Property Law \S~226-B}

1. Unless a greater right to assign is conferred by the lease, a tenant renting
a residence may not assign his lease without the written consent of the owner,
which consent may be unconditionally withheld without cause provided that the
owner shall release the tenant from the lease upon request of the tenant upon
thirty days notice if the owner unreasonably withholds consent which release
shall be the sole remedy of the tenant. If the owner reasonably withholds
consent, there shall be no assignment and the tenant shall not be released from
the lease.

2.
\begin{statute}
\item (a) A tenant renting a residence pursuant to an existing lease in a
dwelling having four or more residential units shall have the right to sublease
his premises subject to the written consent of the landlord in advance of the
subletting. Such consent shall not be unreasonably withheld.

\item (b) The tenant shall inform the landlord of his intent to sublease by
mailing a notice of such intent by certified mail, return receipt requested.
Such request shall be accompanied by the following information: (i) the term of
the sublease, (ii) the name of the proposed sublessee, (iii) the business and
permanent home address of the proposed sublessee, (iv) the tenant's reason for
subletting, (v) the tenant's address for the term of the sublease, (vi) the
written consent of any cotenant or guarantor of the lease, and (vii) a copy of
the proposed sublease, to which a copy of the tenant's lease shall be attached
if available, acknowledged by the tenant and proposed subtenant as being a true
copy of such sublease.

\item (c) Within ten days after the mailing of such request, the landlord may
ask the tenant for additional information as will enable the landlord to
determine if rejection of such request shall be unreasonable. Any such request
for additional information shall not be unduly burdensome. Within thirty days
after the mailing of the request for consent, or of the additional information
reasonably asked for by the landlord, whichever is later, the landlord shall
send a notice to the tenant of his consent or, if he does not consent, his
reasons therefor. Landlord's failure to send such a notice shall be deemed to be
a consent to the proposed subletting. If the landlord consents, the premises may
be sublet in accordance with the request, but the tenant thereunder, shall
nevertheless remain liable for the performance of tenant's obligations under
said lease. If the landlord reasonably withholds consent, there shall be no
subletting and the tenant shall not be released from the lease. If the landlord
unreasonably withholds consent, the tenant may sublet in accordance with the
request and may recover the costs of the proceeding and attorneys fees if it is
found that the owner acted in bad faith by withholding consent.
\end{statute}
\dots.

5. Any sublet or assignment which does not comply with the provisions of this
section shall constitute a substantial breach of lease or tenancy.

6. Any provision of a lease or rental agreement purporting to waive a provision
of this section is null and void. 

