The law of concurrent ownership, discussed in the previous chapter, generally
regulates relationships between intimates.  Arrangements like the joint tenancy
generally arise between individuals who know each other and remain locked in
ongoing relationships.  As a result, there's not much arms-length bargaining
and relatively few disputes work their way into the court system.

The law of landlord--tenant is very different.  It is the law of
strangers---strangers who often have little in common and may never interact
after the lease terminates.  How the law responds to this difference is one of
the central theoretical questions you will wrestle with in this chapter.  More
practically, in this section of the course you will learn about the types of
leaseholds, tenant selection, transferring leases, ending leases, and the
various rights and responsibilities of tenants and landlords during the course
of the lease.

