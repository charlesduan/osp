\expected{berg-v-wiley}

\item \textbf{Who did what wrong?} Kathleen Berg, the tenant, never missed a
rent payment.  Why, exactly, did Wiley think he was entitled to enter the
property and exclude the tenant?  Is Rodney Wiley at fault for this dispute? 
If you were his lawyer at the time, would you have given him different advice? 
If he was entitled to possession, how did he end up owing \$34,500 to Berg?


\item \textbf{Tending to Cause a Breach of the Peace.} In case you aren't
convinced that repossession carries an inherent risk of a breach of the peace,
consider the story of Erskine G. Bryce. In the summer of 2001, Mr. Bryce---a
66-year-old city marshal in Brooklyn, New York---arrived at the second-story
apartment of 53-year-old JoAnne Jones to remove her from possession pursuant to
a duly issued court order for her eviction. At the time, Ms. Jones owed about
\$14,000 in back rent. She violently attacked the marshal, knocking him over a
stairwell railing down to the ground floor below. Mr. Bryce's head hit a
refrigerator on the way down. Ms. Jones grabbed an aluminum rod, ran down the
stairs, and began beating Mr. Bryce with the rod. She then doused his body with
paint thinner and set him on fire with a cigarette lighter. Almost as quickly
as it had arisen, Ms. Jones's rage subsided, and she attempted to put out the
flames she had ignited by running back and forth to her apartment to fetch
basins of water---but it was too late. The medical examiner concluded that Mr.
Bryce died from a combination of blunt force injuries and the flames that
quickly consumed his upper body---in other words, that he had been beaten to
within an inch of his life and then burned alive. C.J. Chivers, \textit{Tenant
Held in Murder of Marshal}, \textsc{N.Y. Times} (Aug. 23, 2001).

Mr. Bryce had two decades of experience as a marshal and a reputation for
dealing calmly and compassionately with those he evicted. He was a stranger to
Ms. Jones until he arrived to evict her. But in the moment, the situation still
exploded into horrific, deadly violence. How much more likely do we think such
violence would be where a landlord---who has a personal stake in recovering
possession, no particular professional experience in managing or defusing tense
situations, no imprimatur of government authority, and a bitter history with
the tenant---attempts to repossess?  


\item \textbf{Do landlords love violence?} If the court here is correct that all
self-help remedies contain the inherent potential for violence, why do
landlords seem so eager to employ them? Why would a landlord ever resist going
through the court process, which the Justice Rogosheske describes as ``adequate
and speedy''?


\item \textbf{Can landlords stand their ground?} Many states have
so-called ``stand your ground'' laws.  Stand your ground laws authorize
individuals to use deadly force in self-defense when faced with a reasonable
threat.  There is no duty to retreat first.  Why are legislatures concerned
about violence in the landlord/tenant context but not in the self-defense
setting? 


\item \textbf{Costs.} Who does the demise of self-help hurt?


\item \textbf{Basic eviction procedure.} Every state has now enacted
statutes---often referred to as forcible entry and detainer laws---that help
landlords to promptly regain possession when a tenant holds over or commits a
material breach of the lease.  In most jurisdictions, statutes mandate that
landlords pursue relief through the court system and refrain from self-help
remedies.  While these eviction procedures vary between jurisdictions, there
are some significant commonalities between most states' forced entry and
detainer laws.  In all jurisdictions, for example, a landlord who wishes to
evict a tenant must first send the tenant proper written notice.  The notice
requirement generally obliges the landlord to accurately state the tenant's
name and address, and reveal the nature of the alleged breach.  Most states
also require the landlord to give the tenant an opportunity (often 3 days, but
sometimes as long as 14) to either cure the default or move out.  These are
often referred to as ``Cure or Quit'' notices.  If the tenant corrects the
problem, they must be allowed to stay.  However, if the tenant stays in the
unit and does not cure the default, the landlord can file a petition for
eviction with the local housing court.  Upon the landlord's request, the court
will quickly set a trial date and a process server will deliver a summons and
complaint to each tenant.  Most tenants do not contest their evictions.  If the
tenant does not respond to the summons, the court will enter a judgment in
favor of the landlord and the landlord will then hire a local sheriff to remove
the tenant from the property.  The entire process generally takes from 20 to 60
days. 


\item \textbf{Defending against eviction.} Occasionally a tenant will mount a
vigorous defense to an eviction notice.  The most commonly raised defenses are
(1) notice was faulty, (2) the tenant cured the default, (3) the landlord
illegally retaliated against the tenant, and, (4) the tenant had a right to
withhold rent because the unit failed to meet certain minimum standards
required by law. 

