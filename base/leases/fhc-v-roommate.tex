\reading{Fair Housing Council of San Fernando Valley v. Roommate.com, LLC}

\readingcite{666 F.3d 1216 (9th Cir. 2012)}

\textsc{Kozinski}, Chief Judge:

There's no place like home. In the privacy of your own home, you can take off
your coat, kick off your shoes, let your guard down and be completely yourself.
While we usually share our homes only with friends and family, sometimes we
need to take in a stranger to help pay the rent. When that happens, can the
government limit whom we choose? Specifically, do the anti-discrimination
provisions of the Fair Housing Act (``FHA'') extend to the selection of
roommates?

Roommate.com, LLC (``Roommate'') operates an internet-based business that helps
roommates find each other. Roommate's website receives over 40,000 visits a day
and roughly a million new postings for roommates are created each year. When
users sign up, they must create a profile by answering a series of questions
about their sex, sexual orientation and whether children will be living with
them. An open-ended ``Additional Comments'' section lets users include
information not prompted by the questionnaire. Users are asked to list their
preferences for roommate characteristics, including sex, sexual orientation and
familial status. Based on the profiles and preferences, Roommate matches users
and provides them a list of housing-seekers or available rooms meeting their
criteria. Users can also search available listings based on roommate
characteristics, including sex, sexual orientation and familial status.  The
Fair Housing Councils of San Fernando Valley and San Diego (``FHCs'') sued
Roommate in federal court, alleging that the website's questions requiring
disclosure of sex, sexual orientation and familial status, and its sorting,
steering and matching of users based on those characteristics, violate the Fair
Housing Act (``FHA''), 42 U.S.C. {\S} 3601 et seq.\ldots

\readinghead{Analysis}

If the FHA extends to shared living situations, it's quite clear that what
Roommate does amounts to a violation. The pivotal question is whether the FHA
applies to roommates. 

\readinghead{I}

The FHA prohibits discrimination on the basis of ``race, color, religion, sex,
familial status, or national origin'' in the ``sale or rental \textit{of a
dwelling}.'' 42 U.S.C. {\S} 3604(b) (emphasis added). The FHA also makes it
illegal to: 
\begin{quote}
make, print, or publish, or cause to be made, printed, or published any notice,
statement, or advertisement, with respect to the sale or rental \textit{of a
dwelling} that indicates any preference, limitation, or discrimination based on
race, color, religion, sex, handicap, familial status, or national origin, or
an intention to make any such preference, limitation, or discrimination.  
\end{quote}
Id. \S~3604(c) (emphasis added). The reach of the statute turns on the meaning
of ``dwelling.''

The FHA defines ``dwelling'' as ``any building, structure, or portion thereof
which is occupied as, or designed or intended for occupancy as, a residence by
one or more families.'' Id. \S~3602(b). A dwelling is thus a living unit
designed or intended for occupancy by a family, meaning that it ordinarily has
the elements generally associated with a family residence: sleeping spaces,
bathroom and kitchen facilities, and common areas, such as living rooms, dens
and hallways.  

It would be difficult, though not impossible, to divide a single-family house or
apartment into separate ``dwellings'' for purposes of the statute. Is a
``dwelling'' a bedroom plus a right to access common areas? What if roommates
share a bedroom? Could a ``dwelling'' be a bottom bunk and half an armoire? It
makes practical sense to interpret ``dwelling'' as an independent living unit
and stop the FHA at the front door.

 There's no indication that Congress intended to interfere with personal
relationships inside the home. Congress wanted to address the problem of
landlords discriminating in the sale and rental of housing, which deprived
protected classes of housing opportunities. But a business transaction between
a tenant and landlord is quite different from an arrangement between two people
sharing the same living space. We seriously doubt Congress meant the FHA to
apply to the latter. Consider, for example, the FHA's prohibition against sex
discrimination. Could Congress, in the 1960s, really have meant that women must
accept men as roommates? Telling women they may not lawfully exclude men from
the list of acceptable roommates would be controversial today; it would have
been scandalous in the 1960s.

While it's possible to read dwelling to mean sub-parts of a home or an
apartment, doing so leads to awkward results.\ldots Nonetheless, this
interpretation is not wholly implausible and we would normally consider
adopting it, given that the FHA is a remedial statute that we construe broadly.
Therefore, we turn to constitutional concerns, which provide strong
countervailing considerations. 

\readinghead{II}

The Supreme Court has recognized that ``the freedom to enter into and carry on
certain intimate or private relationships is a fundamental element of liberty
protected by the Bill of Rights.'' \textit{Bd. of Dirs. of Rotary Int'l v.
Rotary Club of Duarte}, 481 U.S. 537 (1987). ``[C]hoices to enter into and
maintain certain intimate human relationships must be secured against undue
intrusion by the State because of the role of such relationships in
safeguarding the individual freedom that is central to our constitutional
scheme.'' \textit{Roberts v. U.S. Jaycees}, 468 U.S. 609, 617-18 (1984). Courts
have extended the right of intimate association to marriage, child bearing,
child rearing and cohabitation with relatives. \textit{Id.} While the right
protects only ``highly personal relationships,'' \textit{IDK, Inc. v. Clark
Cnty.}, 836 F.2d 1185, 1193 (9th Cir. 1988) (quoting \textit{Roberts}, 468 U.S.
at 618), the right isn't restricted exclusively to family, \textit{Bd. of Dirs.
of Rotary Int'l}, 481 U.S. at 545. The right to association also implies a
right not to associate. \textit{Roberts}, 468 U.S. at 623.  

To determine whether a particular relationship is protected by the right to
intimate association we look to ``size, purpose, selectivity, and whether
others are excluded from critical aspects of the relationship.'' \textit{Bd. of
Dirs. of Rotary Int'l}, 481 U.S. at 546. The roommate relationship easily
qualifies: People generally have very few roommates; they are selective in
choosing roommates; and non-roommates are excluded from the critical aspects of
the relationship, such as using the living spaces. Aside from immediate family
or a romantic partner, it's hard to imagine a relationship more intimate than
that between roommates, who share living rooms, dining rooms, kitchens,
bathrooms, even bedrooms.

Because of a roommate's unfettered access to the home, choosing a roommate
implicates significant privacy and safety considerations. The home is the
center of our private lives. Roommates note our comings and goings, observe
whom we bring back at night, hear what songs we sing in the shower, see us in
various stages of undress and learn intimate details most of us prefer to keep
private.\ldots

Equally important, we are fully exposed to a roommate's belongings, activities,
habits, proclivities and way of life. This could include matter we find
offensive (pornography, religious materials, political propaganda); dangerous
(tobacco, drugs, firearms); annoying (jazz, perfume, frequent overnight
visitors, furry pets); habits that are incompatible with our lifestyle (early
risers, messy cooks, bathroom hogs, clothing borrowers). When you invite others
to share your living quarters, you risk becoming a suspect in whatever illegal
activities they engage in.

Government regulation of an individual's ability to pick a roommate thus
intrudes into the home, which ``is entitled to special protection as the center
of the private lives of our people.'' Minnesota v. Carter, 525 U.S. 83, 99
(1998) (Kennedy, J., concurring).\ldots Holding that the FHA applies inside a
home or apartment would allow the government to restrict our ability to choose
roommates compatible with our lifestyles. This would be a serious invasion of
privacy, autonomy and security.

For example, women will often look for female roommates because of modesty or
security concerns. As roommates often share bathrooms and common areas, a girl
may not want to walk around in her towel in front of a boy. She might also
worry about unwanted sexual advances or becoming romantically involved with
someone she must count on to pay the rent.

An orthodox Jew may want a roommate with similar beliefs and dietary
restrictions, so he won't have to worry about finding honey-baked ham in the
refrigerator next to the potato latkes. Non-Jewish roommates may not understand
or faithfully follow all of the culinary rules, like the use of different
silverware for dairy and meat products, or the prohibition against warming
non-kosher food in a kosher microwave.\ldots

It's a ``well-established principle that statutes will be interpreted to avoid
constitutional difficulties.'' Frisby v. Schultz, 487 U.S. 474, 483 (1988).
``[W]here an otherwise acceptable construction of a statute would raise serious
constitutional problems, the Court will construe the statute to avoid such
problems unless such construction is plainly contrary to the intent of
Congress.'' \textit{Pub. Citizen v. U.S. Dep't of Justice}, 491 U.S. 440, 466
(1989). Because the FHA can reasonably be read either to include or exclude
shared living arrangements, we can and must choose the construction that avoids
raising constitutional concerns.\ldots Reading ``dwelling'' to mean an
independent housing unit is a fair interpretation of the text and consistent
with congressional intent. Because the construction of ``dwelling'' to include
shared living units raises substantial constitutional concerns, we adopt the
narrower construction that excludes roommate selection from the reach of the
FHA.\ldots

As the underlying conduct is not unlawful, Roommate's facilitation of
discriminatory roommate searches does not violate the FHA.

