\reading{Hilder v. St. Peter}
\readingcite{478 A.2d 202 (Vt. 1984)}

\opinion \textsc{Billings}, Chief Justice.

Defendants appeal from a judgment rendered by the Rutland Superior Court.  The
court ordered defendants to pay plaintiff damages in the amount of \$4,945.00,
which represented ``reimbursement of all rent paid and additional compensatory
damages'' for the rental of a residential apartment over a fourteen month
period in defendants' Rutland apartment building. Defendants filed a motion for
reconsideration on the issue of the amount of damages awarded to the plaintiff,
and plaintiff filed a cross-motion for reconsideration of the court's denial of
an award of punitive damages. The court denied both motions. On appeal,
defendants raise [two] issues for our consideration: first, whether the court
correctly calculated the amount of damages awarded the plaintiff; secondly,
whether the court's award to plaintiff of the entire amount of rent paid to
defendants was proper since the plaintiff remained in possession of the
apartment for the entire fourteen month period.\ldots

The facts are uncontested. In October, 1974, plaintiff began occupying an
apartment at defendants' 10--12 Church Street apartment building in Rutland
with her three children and new-born grandson. Plaintiff orally agreed to pay
defendant Stuart St. Peter \$140 a month and a damage deposit of \$50;
plaintiff paid defendant the first month's rent and the damage deposit prior to
moving in. Plaintiff has paid all rent due under her tenancy. Because the
previous tenants had left behind garbage and items of personal belongings,
defendant offered to refund plaintiff's damage deposit if she would clean the
apartment herself prior to taking possession. Plaintiff did clean the
apartment, but never received her deposit back because the defendant denied
ever receiving it. Upon moving into the apartment, plaintiff discovered a
broken kitchen window. Defendant promised to repair it, but after waiting a
week and fearing that her two year old child might cut herself on the shards of
glass, plaintiff repaired the window at her own expense. Although defendant
promised to provide a front door key, he never did. For a period of time,
whenever plaintiff left the apartment, a member of her family would remain
behind for security reasons. Eventually, plaintiff purchased and installed a
padlock, again at her own expense. After moving in, plaintiff discovered that
the bathroom toilet was clogged with paper and feces and would flush only by
dumping pails of water into it. Although plaintiff repeatedly complained about
the toilet, and defendant promised to have it repaired, the toilet remained
clogged and mechanically inoperable throughout the period of plaintiff's
tenancy. In addition, the bathroom light and wall outlet were inoperable.
Again, the defendant agreed to repair the fixtures, but never did. In order to
have light in the bathroom, plaintiff attached a fixture to the wall and
connected it to an extension cord that was plugged into an adjoining room.
Plaintiff also discovered that water leaked from the water pipes of the
upstairs apartment down the ceilings and walls of both her kitchen and back
bedroom. Again, defendant promised to fix the leakage, but never did. As a
result of this leakage, a large section of plaster fell from the back bedroom
ceiling onto her bed and her grandson's crib. Other sections of plaster
remained dangling from the ceiling. This condition was brought to the attention
of the defendant, but he never corrected it. Fearing that the remaining plaster
might fall when the room was occupied, plaintiff moved her and her grandson's
bedroom furniture into the living room and ceased using the back bedroom.
During the summer months an odor of raw sewage permeated plaintiff's apartment.
The odor was so strong that the plaintiff was ashamed to have company in her
apartment. Responding to plaintiff's complaints, Rutland City workers unearthed
a broken sewage pipe in the basement of defendants' building. Raw sewage
littered the floor of the basement, but defendant failed to clean it up.
Plaintiff also discovered that the electric service for her furnace was
attached to her breaker box, although defendant had agreed, at the commencement
of plaintiff's tenancy, to furnish heat.

In its conclusions of law, the court held that the state of disrepair of
plaintiff's apartment, which was known to the defendants, substantially reduced
the value of the leasehold from the agreed rental value, thus constituting a
breach of the implied warranty of habitability. The court based its award of
damages on the breach of this warranty and on breach of an express contract.
Defendant argues that the court misapplied the law of Vermont relating to
habitability because the plaintiff never abandoned the demised premises and,
therefore, it was error to award her the full amount of rent paid. Plaintiff
counters that, while never expressly recognized by this Court, the trial court
was correct in applying an implied warranty of habitability and that under this
warranty, abandonment of the premises is not required. Plaintiff urges this
Court to affirmatively adopt the implied warranty of habitability.

Historically, relations between landlords and tenants have been defined by the
law of property. Under these traditional common law property concepts, a lease
was viewed as a conveyance of real property. \textit{See} Note, \emph{Judicial
Expansion of Tenants' Private Law Rights: Implied Warranties of Habitability
and Safety in Residential Urban Leases,} 56 Cornell L.Q. 489, 489--90 (1971)
(hereinafter cited as \emph{Expansion of Tenants' Rights}). The relationship
between landlord and tenant was controlled by the doctrine of caveat lessee;
that is, the tenant took possession of the demised premises irrespective of
their state of disrepair. Love, \emph{Landlord's Liability for Defective
Premises: Caveat Lessee, Negligence, or Strict Liability?}, 1975 Wis. L. Rev.
19, 27--28. The landlord's only covenant was to deliver possession to the
tenant. The tenant's obligation to pay rent existed independently of the
landlord's duty to deliver possession, so that as long as possession remained
in the tenant, the tenant remained liable for payment of rent. The landlord was
under no duty to render the premises habitable unless there was an express
covenant to repair in the written lease. \emph{Expansion of Tenants' Rights,
supra}, at 490. The land, not the dwelling, was regarded as the essence of the
conveyance.

An exception to the rule of caveat lessee was the doctrine of constructive
eviction. \textit{Lemle v. Breeden}, 462 P.2d 470, 473 (Haw. 1969). Here, if
the landlord wrongfully interfered with the tenant's enjoyment of the demised
premises, or failed to render a duty to the tenant as expressly required under
the terms of the lease, the tenant could abandon the premises and cease paying
rent. \textit{Legier v. Deveneau}, 126 A. 392, 393 (Vt. 1924).

Beginning in the 1960's, American courts began recognizing that this approach to
landlord and tenant relations, which had originated during the Middle Ages, had
become an anachronism in twentieth century, urban society. Today's tenant
enters into lease agreements, not to obtain arable land, but to obtain safe,
sanitary and comfortable housing.  
\begin{quote}
[T]hey seek a well known package of goods and services---a package which
includes not merely walls and ceilings, but also adequate heat, light and
ventilation, serviceable plumbing facilities, secure windows and doors, proper
sanitation, and proper maintenance.
\end{quote}
\textit{Javins v. First National Realty Corp.}, 428 F.2d 1071, 1074 (D.C.Cir.),
cert. denied, 400 U.S. 925, 91 S.Ct. 186, 27 L.Ed.2d 185 (1970).

Not only has the subject matter of today's lease changed, but the
characteristics of today's tenant have similarly evolved. The tenant of the
Middle Ages was a farmer, capable of making whatever repairs were necessary to
his primitive dwelling. \textit{Green v. Superior Court}, 517 P.2d 1168, 1172
(Cal. 1974). Additionally, ``the common law courts assumed that an equal
bargaining position existed between landlord and tenant.\ldots'' Note,
\textit{The Implied Warranty of Habitability: A Dream Deferred}, 48
\textsc{UMKC L.Rev}. 237, 238 (1980) (hereinafter cited as \textit{A Dream
Deferred}).

In sharp contrast, today's residential tenant, most commonly a city dweller, is
not experienced in performing maintenance work on urban, complex living units.
\textit{Green v. Superior Court}, \textit{supra}, 517 P.2d at 1173. The
landlord is more familiar with the dwelling unit and mechanical equipment
attached to that unit, and is more financially able to ``discover and cure''
any faults and break-downs. \textit{Id}. Confronted with a recognized shortage
of safe, decent housing, see 24 V.S.A. \S~4001(1), today's tenant is in an
inferior bargaining position compared to that of the landlord. \textit{Park
West Management Corp. v. Mitchell}, 391 N.E.2d 1288, 1292 (N.Y. 1979). Tenants
vying for this limited housing are ``virtually powerless to compel the
performance of essential services.'' \textit{Id}.

In light of these changes in the relationship between tenants and landlords, it
would be wrong for the law to continue to impose the doctrine of caveat lessee
on residential leases.  
\begin{quote}
The modern view favors a new approach which recognizes that a lease is
essentially a contract between the landlord and the tenant wherein the landlord
promises to deliver and maintain the demised premises in habitable condition
and the tenant promises to pay rent for such habitable premises. These promises
constitute interdependent and mutual considerations. Thus, the tenant's
obligation to pay rent is predicated on the landlord's obligation to deliver
and maintain the premises in habitable condition.  
\end{quote}
\textit{Boston Housing Authority v. Hemingway}, 293 N.E.2d 831, 842 (Mass.
1973).

Recognition of residential leases as contracts embodying the mutual covenants of
habitability and payment of rent does not represent an abrupt change in Vermont
law. Our case law has previously recognized that contract remedies are
available for breaches of lease agreements. \textit{Clarendon Mobile Home
Sales, Inc. v. Fitzgerald}, 381 A.2d 1063, 1065 (Vt. 1977).\ldots More
significantly, our legislature, in establishing local housing authorities, 24
V.S.A. \S~4003, has officially recognized the need for assuring the existence
of adequate housing.  
\begin{quote}
[S]ubstandard and decadent areas exist in certain portions of the state of
Vermont and\ldots there is not\ldots an adequate supply of decent, safe and
sanitary housing for persons of low income and/or elderly persons of low
income, available for rents which such persons can afford to pay\ldots this
situation tends to cause an increase and spread of communicable and chronic
disease\ldots [and] constitutes a menace to the health, safety, welfare and
comfort of the inhabitants of the state and is detrimental to property values
in the localities in which it exists\ldots .  
\end{quote}
24 V.S.A. \S~4001(4). In addition, this Court has assumed the existence of an
implied warranty of habitability in residential leases. \textit{Birkenhead v.
Coombs}, 465 A.2d 244, 246 (Vt. 1983).

Therefore, we now hold expressly that in the rental of any residential dwelling
unit an implied warranty exists in the lease, whether oral or written, that the
landlord will deliver over and maintain, throughout the period of the tenancy,
premises that are safe, clean and fit for human habitation. This warranty of
habitability is implied in tenancies for a specific period or at will.
\textit{Boston Housing Authority v. Hemingway}, \textit{supra}, 293 N.E.2d at
843. Additionally, the implied warranty of habitability covers all latent and
patent defects in the essential facilities of the residential unit.
\textit{Id}. Essential facilities are ``facilities vital to the use of the
premises for residential purposes.\ldots'' \textit{Kline v. Burns}, 276 A.2d
248, 252 (N.H. 1971). This means that a tenant who enters into a lease
agreement with knowledge of any defect in the essential facilities cannot be
said to have assumed the risk, thereby losing the protection of the warranty.
Nor can this implied warranty of habitability be waived by any written
provision in the lease or by oral agreement.

In determining whether there has been a breach of the implied warranty of
habitability, the courts may first look to any relevant local or municipal
housing code; they may also make reference to the minimum housing code
standards enunciated in 24 V.S.A. \S~5003(c)(1)--5003(c)(5). A substantial
violation of an applicable housing code shall constitute prima facie evidence
that there has been a breach of the warranty of habitability. ``[O]ne or two
minor violations standing alone which do not affect'' the health or safety of
the tenant, shall be considered \textit{de minimus} and not a breach of the
warranty. \textit{Javins v. First National Realty Corp.}, \textit{supra}, 428
F.2d at 1082 n. 63.\ldots In addition, the landlord will not be liable for
defects caused by the tenant. \textit{Javins v. First National Realty Corp.},
\textit{supra}, 428 F.2d at 1082 n. 62.

However, these codes and standards merely provide a starting point in
determining whether there has been a breach. Not all towns and municipalities
have housing codes; where there are codes, the particular problem complained of
may not be addressed. \textit{Park West Management Corp. v. Mitchell},
\textit{supra}, 391 N.E.2d at 1294. In determining whether there has been a
breach of the implied warranty of habitability, courts should inquire whether
the claimed defect has an impact on the safety or health of the tenant.
\textit{Id}.

In order to bring a cause of action for breach of the implied warranty of
habitability, the tenant must first show that he or she notified the landlord
``of the deficiency or defect not known to the landlord and [allowed] a
reasonable time for its correction.'' \textit{King v. Moorehead},
\textit{supra}, 495 S.W.2d at 76.

Because we hold that the lease of a residential dwelling creates a contractual
relationship between the landlord and tenant, the standard contract remedies of
rescission, reformation and damages are available to the tenant when suing for
breach of the implied warranty of habitability. \textit{Lemle v. Breeden},
\textit{supra}, 462 P.2d at 475. The measure of damages shall be the difference
between the value of the dwelling as warranted and the value of the dwelling as
it exists in its defective condition. \textit{Birkenhead v. Coombs},
\textit{supra}, 465 A.2d at 246. In determining the fair rental value of the
dwelling as warranted, the court may look to the agreed upon rent as evidence
on this issue. \textit{Id}. ``[I]n residential lease disputes involving a
breach of the implied warranty of habitability, public policy militates against
requiring expert testimony'' concerning the value of the defect. \textit{Id}.
at 247. The tenant will be liable only for ``the reasonable rental value [if
any] of the property in its imperfect condition during his period of
occupancy.'' \textit{Berzito v. Gambino}, 308 A.2d 17, 22 (N.J. 1973).

We also find persuasive the reasoning of some commentators that damages should
be allowed for a tenant's discomfort and annoyance arising from the landlord's
breach of the implied warranty of habitability. \textit{See} Moskovitz,
\textit{The Implied Warranty of Habitability: A New Doctrine Raising New
Issues}, 62 \textsc{Cal. L. Rev}. 1444, 1470--73 (1974) (hereinafter cited as
\textit{A New Doctrine}); \textit{A Dream Deferred}, \textit{supra}, at
250--51. Damages for annoyance and discomfort are reasonable in light of the
fact that:  
\begin{quote}
the residential tenant who has suffered a breach of the warranty\ldots cannot
bathe as frequently as he would like or at all if there is inadequate hot
water; he must worry about rodents harassing his children or spreading disease
if the premises are infested; or he must avoid certain rooms or worry about
catching a cold if there is inadequate weather protection or heat. Thus,
discomfort and annoyance are the common injuries caused by each breach and
hence the true nature of the general damages the tenant is claiming.  
\end{quote}
Moskovitz, \textit{A New Doctrine}, \textit{supra}, at 1470--71. Damages for
discomfort and annoyance may be difficult to compute; however, ``[t]he trier
[of fact] is not to be deterred from this duty by the fact that the damages are
not susceptible of reduction to an exact money standard.'' \textit{Vermont
Electric Supply Co. v. Andrus}, 315 A.2d 456, 459 (Vt. 1974).

Another remedy available to the tenant when there has been a breach of the
implied warranty of habitability is to withhold the payment of future rent.
\textit{King v. Moorehead}, \textit{supra}, 495 S.W.2d at 77. The burden and
expense of bringing suit will then be on the landlord who can better afford to
bring the action. In an action for ejectment for nonpayment of rent, 12 V.S.A.
\S~4773, ``[t]he trier of fact, upon evaluating the seriousness of the breach
and the ramification of the defect upon the health and safety of the tenant,
will abate the rent at the landlord's expense in accordance with its
findings.'' \textit{A Dream Deferred}, \textit{supra}, at 248. The tenant must
show that: (1) the landlord had notice of the previously unknown defect and
failed, within a reasonable time, to repair it; and (2) the defect, affecting
habitability, existed during the time for which rent was withheld. \textit{See}
\textit{A Dream Deferred}, \textit{supra}, at 248--50. Whether a portion, all
or none of the rent will be awarded to the landlord will depend on the findings
relative to the extent and duration of the breach. \textit{Javins v. First
National Realty Corp.}, \textit{supra}, 428 F.2d at 1082--83. Of course, once
the landlord corrects the defect, the tenant's obligation to pay rent becomes
due again. \textit{Id}. at 1083 n. 64.

Additionally, we hold that when the landlord is notified of the defect but fails
to repair it within a reasonable amount of time, and the tenant subsequently
repairs the defect, the tenant may deduct the expense of the repair from future
rent. 11 Williston on Contracts \S~1404 (3d ed. W. Jaeger 1968);
\textit{Marini v. Ireland}, 265 A.2d 526, 535 (N.J. 1970).

In addition to general damages, we hold that punitive damages may be available
to a tenant in the appropriate case. Although punitive damages are generally
not recoverable in actions for breach of contract, there are cases in which the
breach is of such a willful and wanton or fraudulent nature as to make
appropriate the award of exemplary damages. \textit{Clarendon Mobile Home
Sales, Inc. v. Fitzgerald}, \textit{supra}, 381 A.2d at 1065. A willful and
wanton or fraudulent breach may be shown ``by conduct manifesting personal ill
will, or carried out under circumstances of insult or oppression, or even by
conduct manifesting\ldots a reckless or wanton disregard of [one's]
rights\ldots.'' \textit{Sparrow v. Vermont Savings Bank}, 112 A. 205, 207 (Vt.
1921).
When a landlord, after receiving notice of a defect, fails to repair the
facility that is essential to the health and safety of his or her tenant, an
award of punitive damages is proper. \textit{111 East 88th Partners v. Simon},
434 N.Y.S.2d 886, 889 (N.Y. Civ. Ct. 1980).
\begin{quote}
The purpose of punitive damages\ldots is to punish conduct which is morally
culpable.\ldots  Such an award serves to deter a wrongdoer\ldots from
repetitions of the same or similar actions. And it tends to encourage
prosecution of a claim by a victim who might not otherwise incur the expense or
inconvenience of private action.\ldots The public benefit and a display of
ethical indignation are among the ends of the policy to grant punitive damages.
\end{quote}
\textit{Davis v. Williams}, 402 N.Y.S.2d 92, 94 (N.Y.Civ.Ct.1977).

In the instant case, the trial court's award of damages, based in part on a
breach of the implied warranty of habitability, was not a misapplication of the
law relative to habitability. Because of our holding in this case, the doctrine
of constructive eviction, wherein the tenant must abandon in order to escape
liability for rent, is no longer viable. When, as in the instant case, the
tenant seeks, not to escape rent liability, but to receive compensatory damages
in the amount of rent already paid, abandonment is similarly unnecessary.
\textit{Northern Terminals, Inc. v. Smith Grocery \& Variety, Inc.},
\textit{supra}, 418 A.2d at 26--27. Under our holding, when a landlord breaches
the implied warranty of habitability, the tenant may withhold future rent, and
may also seek damages in the amount of rent previously paid.

In its conclusions of law the trial court stated that the defendants' failure to
make repairs was compensable by damages to the extent of reimbursement of all
rent paid and additional compensatory damages. The court awarded plaintiff a
total of \$4,945.00; \$3,445.00 represents the entire amount of rent plaintiff
paid, plus the \$50.00 deposit.\ldots

Additionally, the court denied an award to plaintiff of punitive damages on the
ground that the evidence failed to support a finding of willful and wanton or
fraudulent conduct. \textit{See} \textit{Clarendon Mobile Home Sales, Inc. v.
Fitzgerald}, \textit{supra}, 381 A.2d at 1065. The facts in this case, which
defendants do not contest, evince a pattern of intentional conduct on the part
of defendants for which the term ``slumlord'' surely was coined. Defendants'
conduct was culpable and demeaning to plaintiff and clearly expressive of a
wanton disregard of plaintiff's rights. The trial court found that defendants
were aware of defects in the essential facilities of plaintiff's apartment,
promised plaintiff that repairs would be made, but never fulfilled those
promises. The court also found that plaintiff continued, throughout her
tenancy, to pay her rent, often in the face of verbal threats made by defendant
Stuart St. Peter. These findings point to the ``bad spirit and wrong
intention'' of the defendants, \textit{Glidden v. Skinner}, 458 A.2d 1142, 1144
(Vt. 1983), and would support a finding of willful and wanton or fraudulent
conduct, contrary to the conclusions of law and judgment of the trial judge.
However, the plaintiff did not appeal the court's denial of punitive damages,
and issues not appealed and briefed are waived. \textit{R. Brown \& Sons, Inc.
v. International Harvester Corp.}, 453 A.2d 83, 84 (Vt. 1982).

