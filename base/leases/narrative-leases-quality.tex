In feudal England, policy makers and government officials expressed little
concern over the housing conditions of renters.  The law was well-settled: Once
a landlord turned over the right of possession, the tenant became responsible
for maintenance of the leased property.  If a tenant decided to live in squalor
rather than complete basic repairs, that was the tenant's problem, not the
landlord's worry. Although it may seem counterintuitive to modern readers (who
rely on landlords to fix nearly everything), putting the burden on the tenant
to maintain the property actually produced efficient results in the medieval
world: landlords often lived long distances from their lessees, communication
was slow, houses were simply constructed, and most tenants had the knowledge
and skills to complete basic repairs.  

The basic principle that tenants are responsible for their own living conditions
remained unchallenged until the 1960s, when both academics and politicians
expressed growing concern about the rental housing stock in central cities. 
Many worried that exploitative landlords were flouting safety regulations and
taking advantage of tenants who had few housing choices as a result of their
poverty and the rampant discrimination in the housing market.  The problems in
the poorest neighborhoods also had spillover effects in surrounding
communities---disease, vermin, and fires do not respect municipal borders.  In
response to these problems, the law began to vest tenants with a new series of
rights against their landlords.  This subsection traces the evolution of these
rights and explores the rise of legal tools to ensure minimum housing standards
for all renters. 

