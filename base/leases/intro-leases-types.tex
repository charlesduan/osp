As we have seen throughout this course, property interests come in a limited
number of forms, many of which we have inherited directly from feudal England. 
This theme holds in landlord-tenant.  The common law developed three types of
leaseholds that our modern property system still recognizes: the term of years,
the periodic tenancy, and the tenancy at will.

\paragraph{The Term of Years}

The \textit{term of years} is a leasehold measured by any fixed period of time. 
The most familiar term of years lease is the residential one-year lease. The
actual term, however, may vary greatly.  In 2001, the U.S. government signed a
99-year lease for an embassy in Singapore.  Leases of hundreds or even
thousands of years are not unheard of, either.  \textit{See} \emph{Monbar v.
Monaghan}, 18 Del. Ch. 395 (1932) (two thousand year lease).  At the other end
of the spectrum, vacation properties like beach condos and lake houses commonly
rent for one-week periods. 

Whatever the duration, a term of years automatically ends when the stated term
expires.  For example, imagine L leases Blackacre to T ``from September 1, 2015
to August 31, 2016.''  Neither party is required to give the other notice of
termination.  The tenant must simply surrender possession to the landlord by
midnight on August 31.  The death of either contracting party does not affect a
term of years lease, unless the landlord and tenant have agreed otherwise.  If
the tenant dies, the law requires her estate to carry out the lease. 

\paragraph{The Periodic Tenancy}

The \textit{periodic tenancy} is a lease for some fixed duration that
automatically renews for succeeding periods until either the landlord or tenant
gives notice of termination.  This automatic renewal is the chief practical
difference between the periodic tenancy and the term of years.  The most common
type of periodic tenancy is the month-to-month lease.  As the name suggests, a
month-to-month lease lasts for a month and then continues for subsequent
months, until either the landlord or tenant ends the lease.  Periodic tenancies
have no certain end date; some residential tenants with month-to-month leases
stay in their apartments for decades.  

Termination requires one party to give advance notice to the other.  These
notice requirements are now heavily regulated by statute in most jurisdictions.
 Under the common law (which is still the basis for many state regulations),
for year-to-year periodic leases (or any periodic lease with a longer initial
duration), parties must give notice at least six months before the period ends.
 For leases less than a year, the minimum notice equals the length of the lease
period.  Additionally, unless the parties make an agreement to the contrary,
the lease must terminate on the final day of a period.  Assume, for example,
that T signs a month-to-month lease that begins May 1.  On August 20, T gives
notice of termination to her landlord.  When will the lease end?  T must give
the landlord a minimum of one month notice.  That pushes T's obligations under
the lease to September 19.  A periodic tenancy, however, must end on the last
day of a period.  Thus, T's lease will terminate on September 30 at midnight. 

The death of either the landlord or tenant does not end a periodic tenancy.  If,
for example, the tenant dies before the lease terminates, the law vests the
tenant's estate with the responsibility to fulfill the remaining obligations
under the lease.  

\paragraph{The Tenancy at Will}

The \textit{tenancy at will} has no fixed duration and endures so long as both
of the parties desire.  For example, if the landlord and tenant sign a document
that reads, ``Tenant will pay the Landlord \$500 on the first of the month and
the lease will endure as long as both of us wish'' they have created a tenancy
at will.  Under the common law, either party could end such a lease at any
moment.  Today, most states have enacted statutes that establish minimum notice
periods---30 days is common.  Tenancies at will also terminate if the landlord
sells the property, the tenant abandons the unit, or either party
dies.\footnote{In jurisdictions that require 30-day notice periods before the
termination of a tenancy at will, this is one of the key remaining differences
between the month-to-month periodic lease and the tenancy at will.}

Tenancies at will can arise as a result of the clear intention of the
parties---the ease of termination is a valued feature in some negotiations. 
But note, the tenancy at will is also the catchall lease category.  If a
leasehold doesn't qualify as either a term of years or periodic tenancy, the
law crams it into the tenancy at will box---even if that clearly violates the
goals of the parties.  This occasionally creates real hardship for individuals
with sloppily drafted leases.  

