\paragraph{The Tenancy at Sufferance}

Imagine that you own a small apartment building in a college town.  At the
end of the school year, one of your tenants refuses to move out.  The law
refers to such tenants as \textit{holdovers}.  As a landlord, what are your
options in this situation?  How does the legal system treat individuals who
stay past the end of their leases?  Can you kick them out? Are they obligated
to pay you rent?

When a tenant stays in possession after the lease has expired, the law allows
the landlord to make a one-time election.  The landlord has the option to treat
the holdover as a trespasser, bring an eviction proceeding, and sue for
damages.  Alternatively, the landlord may renew the holdover's lease for
another term.  This second option is typically referred to as a \textit{tenancy
at sufferance}.  Some hornbooks list the tenancy at sufferance as a fourth type
of common law leasehold.  The tenancy at sufferance, however, is not based on
any affirmative agreement between parties and is probably better understood as
a remedy for wrongful occupancy.  Also note that disputes sometimes pop-up over
what election the landlord has made.  For example, what if the landlord does
nothing for two months but then initiates eviction?  

In most jurisdictions, when a landlord chooses to hold the tenant to a new
lease, it creates a periodic tenancy.  States differ, however, on how to
compute the length of the period and, thus, the amount of the damages.  Some
simply copy over the length of the original lease (with a maximum of one year).
Others divine the repeating period by looking at how the rent was paid. 
Imagine, for example, your tenant had originally signed a lease reading, ``This
lease will run from January 1, 2014 to December 31, 2014.  Rent is due on the
first of each month.''  The tenancy created by the holdover would either be a
year-to-year lease or a month-to-month lease depending on the jurisdiction.

Still other states take other approaches.  Some, for example, specify that a
holdover must pay double (or triple) rent for the holdover period.  

