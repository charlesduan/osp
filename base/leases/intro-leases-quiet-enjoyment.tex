Traditional common law principles do not leave renters completely defenseless
against unprincipled landlords.  Every lease, whether residential or
commercial, contains a \textit{covenant of quiet enjoyment}.  Often this
promise is explicitly stated in the lease contract.  Where it's not
specifically mentioned, all courts will imply it into the agreement.  The basic
idea is that the landlord cannot interfere with the tenant's use of the
property.  Most courts state the legal test this way: A breach of the covenant
of quiet enjoyment occurs when the landlord substantially interferes with the
tenant's use or enjoyment of the premises.  

Consider the following hypothetical: 
\begin{quote}\edfont
Little Bo Peep Detective Services rents the second floor of a four-floor
building.  A year into the five-year lease, the landlord suddenly begins a
construction project designed to update the suites on the first floor.  These
renovations create loud noise and regular interruptions of electric service.
The construction work has also made the parking lot inaccessible.  Employees
and customers need to walk a quarter-mile to access the building from a nearby
parking garage.  
\end{quote}
Do these problems amount to a violation of the covenant of quiet enjoyment?  To
determine whether the interference is ``substantial'' courts generally consider
the purpose the premises are leased for, the foreseeability of the problem, the
potential duration, and the degree of harm. In this example, if the
construction project lasts for more than a few days, then Little Bo Peep can
most likely bring a successful claim against its landlord under the covenant of
quiet enjoyment.  The problems here are not mere trifles---the noise, lack of
electricity, and inadequate parking fundamentally affect the company's ability
to use the property as they intended.  

The difficult conceptual issue with the covenant of quiet enjoyment concerns the
remedy.  If the landlord breaks the covenant, what are the tenant's options? 
After a breach, the tenant can always choose to stay in the leased property,
continue to pay rent, and sue the landlord for damages.  

Additionally, certain violations of the covenant of quiet enjoyment allow the
tenant to consider the lease terminated, leave, and stop paying rent.  Recall
from earlier in the chapter that the landlord's fundamental responsibility is
to provide the tenant with possession (or, in some jurisdictions, the right to
possession).  From that principle, courts developed a rule that in cases where
the landlord wrongfully evicts the tenant, all the tenant's obligations under
the lease cease.  Imagine:
\begin{quote}\edfont
Landlord and tenant both sign a lease that reads, ``Landlord agrees to provide
Tenant with possession of 123 Meadowlark Lane for a period of 12 months
beginning April 1.  Tenant agrees to pay \$100 per month.''  After 4 months,
however, the Landlord retakes possession of the property by forcing the tenant
out and changing the locks. 
\end{quote}
Assuming the tenant hasn't committed a material breach, the landlord's actions
constitute an obvious violation of the covenant of quiet enjoyment---the tenant
can no longer use the property for any purpose. Thus, any eviction where the
tenant is physically denied access to the unit ends the tenant's obligation to
pay rent and allows the tenant to sue for damages incurred from being removed
from possession (A tenant could also sue to regain the unit).  The law is very
clear on this point.  Relatedly, if the landlord denies the tenant access to
some portion of the rented space (say, an allotted parking space) that, too,
constitutes a breach of the covenant of quiet enjoyment.  The tenant subject to
such a partial eviction has the option to terminate the lease and sue for
damages.  

But what if the landlord doesn't physically interfere with her tenant's
occupancy? What if the landlord creates an environment that's so miserable that
the tenant is forced to flee?  Is this an ``eviction'' that would allow the
tenant to consider the lease terminated or must the tenant stay and continue
paying rent while he brings a damages lawsuit?

