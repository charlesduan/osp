\expected{hannan-v-dusch}

\item \textbf{The basic law.} U.S. jurisdictions remain split over the
landlord's duty to provide possession.  A majority of jurisdictions (and the
Uniform Residential Landlord and Tenant Act) now follow the English rule, but
the American rule remains alive and well.  As should be obvious, the biggest
difference between the two approaches is the remedy available to the
dispossessed tenant.  Under the English view, the tenant may terminate the
lease and sue the landlord for damages.  The tenant can also choose to withhold
payment from the landlord until the tenant is able to take possession.  In
contrast, under the American rule, the tenant must bring an eviction action
directly against the holdover.


\item \textbf{Justifying the rules.} What policies support the English view? 
What polices support the American view?  Would you find the remedies available
under the American rule helpful?  


\item \textbf{Conceptual Arguments.} The \textit{Hannan} case does an excellent
job discussing the policy rationales for and against the two rules.  But what
about the more conceptual arguments?  If we view the lease as a conveyance of
the legal right to possession, isn't the American rule ``correct?''  Once a
landlord turns over possessory rights, aren't her obligations fulfilled?  


\item \textbf{What do tenants know?} Do you think that tenants in American rule
jurisdictions know that their landlord has no obligation provide them with
actual possession?  Should that matter?


\item \textbf{What rules are mandatory?} Imagine that you sit in a state
legislature that wants to adopt the English Rule by statute.  Should you make
the new law a mandatory rule or a default position that parties can negotiate
around?


\item \textbf{Your Lease?} Does the lease for the apartment you're currently
renting make any provision for this problem?  

