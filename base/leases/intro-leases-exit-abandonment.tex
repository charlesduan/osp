A tenant who needs to exit a lease early and cannot find another party to sublet
must seek out other alternatives.  For example, a tenant can always ask her
landlord to terminate the lease before the term ends.  The tenant generally
agrees to turn over the property and pay a small fee and, in return, the
landlord releases the tenant from all further obligations. This is called a
\term{surrender}.  

Alternatively, a tenant may \term[abandonment]{abandon} the lease: simply pack
her things, vacate the premises, and
stop making rent payments.  This often happens if a tenant cannot work out a
surrender agreement or finds herself in desperate financial circumstances. 
What are the rights and obligations of the parties in this scenario? What
happens if a tenant breaks a lease and leaves?  

