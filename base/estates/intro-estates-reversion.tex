We encountered the reversion once before, when discussing leases as an
introduction to the concept of a future interest. But reversions often arise in
non-leasehold contexts too. Consider what happens when A, owning a life estate
in Blackacre, dies. A's life estate terminates by operation of law; it simply
ceases to exist and disappears. Who ``owns'' Blackacre now? It seems obvious
that \textit{somebody} must have a right to possession of the land, but it
seems equally obvious that whoever that somebody is, they had \textit{no right
to possession} before A died. Whoever they are, during the term of A's life
estate they must have held an interest that would entitle them to take
possession at \textit{some point in the future} (that is, a \textit{future
interest}).

There are two candidates for such an interest. We will begin with the most
basic: the \term{reversion}. Suppose that O, owning a fee simple absolute in
Blackacre, conveys Blackacre ``to A for life,'' and says nothing more? What is
the legal effect of this grant?

Based on the formula we just learned, it should be clear that A receives a life
estate in Blackacre. But what other effects does the grant have on the legal
rights of the parties? Think about the interest O held prior to the conveyance:
the fee simple absolute. Remember that a fee simple absolute is an interest of
\textit{infinite duration}---it never ends. So when O starts with a
possessory interest of infinite duration, and then gives away a life
estate---whose duration is limited by a human lifespan---to A,
\textit{something was left over}. Specifically, O never gave away the right to
possession of Blackacre from the day of A's death to the end of time. Whether
meaning to or not, O gave away less of an interest in Blackacre than what he
owned, meaning \textit{he still holds some interest}. We call this type of
interest---the residual interest left over when a grantor gives away less than
they have---a \textit{retained} interest.

This retained interest can't entitle O to possession during A's life---A has the
exclusive right to possession as the life tenant. So O's interest must be a
\textit{future interest} during the term of A's life estate: an interest that
will entitle O to possession \textit{after the natural termination of the life
estate}. As we discussed in the example of the lease, we call this kind of
future interest a \textbf{reversion}. It is a \textit{retained interest in the
grantor}---created when a grantor conveys less than his entire
interest---that will become possessory by operation of law upon the
\textit{natural termination} of the preceding estate. Colloquially, we say that
Blackacre ``reverts'' to O. In some opinions, you will see the holder of a
reversion referred to as a ``reversioner.''

A reversion can of course also be created explicitly, for example, if O conveys
Blackacre ``to A for life, then to O.'' In this case, O has explicitly created
a life estate in A followed by a reversion in O.

