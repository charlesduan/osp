\expected{estate-craigen}

\item \textbf{Holographic Wills.} A holographic will---a will handwritten by the
testator---often presents a particular challenge for courts attempting to
interpret it. Indeed, they are thought to be so problematic that about half of
American jurisdictions refuse to recognize them as valid wills at all.
\textit{See} Stephen Clowney, \textit{In Their Own Hand: An Analysis of
Holographic Wills and Homemade Willmaking}, \textsc{Real Property, Trust and
Estate Law Journal 27} (2008) (arguing that the defects of holographic wills,
though real, are overstated). Lay testators attempting to settle their affairs
without assistance of counsel often make legal or technical errors of various
kinds, including errors of ambiguity such as the one that generated the
litigation in \textit{Craigen}.


\item \textbf{Presumptions and Rules of Construction.} The court reviews a
number of rules of construction applied by courts in construing ambiguous
grants. Most jurisdictions have similar rules of construction---sometimes
promulgated by statute, other times judge-made. In \textit{Craigen}, two rules
in particular do considerable work: the presumption against intestacy and the
clear-statement rule for creation of a life estate. The latter rule is
sometimes expressed in other jurisdictions as a presumption in favor of the
largest estate the grantor could convey. \textit{See, e.g.}, \textit{White v.
Brown}, 559 S.W.2d 938, 939 (Tenn. 1977) (quoting Tenn. C. Ann. {\S} 32-301)
(``Every grant or devise of real estate, or any interest therein, shall pass
all the estate or interest of the grantor or devisor, unless the intent to pass
a less estate or interest shall appear by express terms, or be necessarily
implied in the terms of the instrument.'').


What justification is there for presuming that an ambiguous grant conveys a fee
simple absolute rather than a life estate? Is it any different for the
justification underlying the presumption against intestacy? Was
\textit{Craigen} an appropriate case for the application of these presumptions?



\item \textbf{Finding Ambiguity.} Are you convinced by the court's arguments
that the language ``till she dies'' does not ``clearly express the testator's
will to create a life estate''? What do you think Dalton Craigen meant by this
phrase?


\item \textbf{Dueling Presumptions.} The court mentions another rule of
construction---the presumption against disinheritance---but declines to apply
it. Why? Is its reason for following the presumption against intestacy but
declining to follow the presumption against disinheritance persuasive? How is a
court to decide when a presumption or other rule of construction applies and
when it doesn't?

