\reading{\emph{In the Estate of Dalton Edward Craigen}}
\readingcite{305 S.W.3d 825 (Ct. App. Tex. 2010)}

\opinion \textsc{Hollis Horton}, Justice.

We are asked to determine whether the trial court properly interpreted the
dispository language in a holographic will. If the will is ambiguous, the
applicable rules of will construction yield one result. If the will is
unambiguous, the trial court was required to give effect to the express
language of the will, and arguably should have reached a different result. 

The trial court, in construing the testator's intentions under the will, found
``[t]hat it was the intent of the [t]estator to leave his entire estate to his
surviving wife in full.'' The trial court further found ``[t]hat there was no
intention to leave a life estate to her.'' In a single issue on appeal, the
testator's adult children contend the testator intended to leave a life estate
to his wife, and they argue that the remainder of the estate passed to them
through the laws of descent and distribution. We find the will is ambiguous and
hold that under the appropriate rules of will construction, the trial court
properly construed the will. Accordingly, we affirm the judgment.

\readinghead{The Will}

Dalton Edward Craigen left a holographic will that in its entirety stated:
\begin{quote}
Last Will \& testament

Debbie gets everything till

she dies.

Being of sound mind \& this

is my w.\ last will \& testament.

I leave to my Wife Daphne

Craigen all p.\ real \& personal property.

12--17--99 Dalton Craigen
\end{quote}

\readinghead{Contentions of the Parties}

The parties stipulated ``[t]hat Debbie and Daphne named in Dalton Craigen's will
are one and the same person.'' Brian Craigen and Sabrina Brumley, Craigen's
adult children, argue that the testator's intent under the will is ``crystal
clear---the testator left everything (all of his real and personal property,
his definition of `everything') to his wife for as long as she lived.''
According to Brian and Sabrina, the dominant provision of the will (the first
sentence) creates a life estate, and the will's third sentence can be
harmonized with the will's first sentence by construing the third sentence to
define the property that Craigen intended to include in his wife's life estate.
Brian and Sabrina ask that we render a judgment in their favor by holding that
Daphne received only a life estate under Craigen's will.

Daphne died on January 17, 2009. Yvonne Christian, the independent
administratrix of Daphne's estate, argues we should affirm the trial court's
judgment. According to Christian, the will is not ambiguous as it reflects
Craigen's intent to leave his entire estate to Daphne.

\readinghead{Rules of Construction}

The rules involved in construing wills are well settled. ``The primary object of
inquiry in interpreting a will is determining the intent of the testator.''
\textit{Gee v. Read}, 606 S.W.2d 677, 680 (Tex.1980). ``The [testator's] intent
must be drawn from the will, not the will from the intent.'' \textit{Id.} We
ascertain intent from the language found within the four corners of the will.
``In construing the will, all its provisions should be looked to, for the
purpose of ascertaining what the real intention of the [testator] was; and, if
this can be ascertained from the language of the instrument, then any
particular paragraph of the will which, considered alone, would indicate a
contrary intent, must yield to the intention manifested by the whole
instrument.'' \textit{McMurray v. Stanley}, 69 Tex. 227, 6 S.W. 412, 413
(1887). 

When a will has been drafted by a layperson who is not shown to be familiar with
the technical meanings of certain words, courts do not place ``{}`too great
emphasis on the precise meaning of the language used where the will is the
product of one not familiar with legal terms, or not trained in their use.'{}''
\textit{Gilkey v. Chambers}, 146 Tex. 355, 207 S.W.2d 70, 71 (1947) (quoting 69
C.J. Wills {\S} 1120 (1934)). Instead, in arriving at the meaning intended by
the layman-testator, courts refer to the popular meaning of the words the
testator chose to use. In summary, the testator's intent, as gathered from the
will as a whole, prevails against a technical meaning that might be given to
certain words or phrases, unless the testator intended to use the word or
phrase in the technical sense.

With respect to the creation of a life estate, no particular words are needed to
create a life estate, but the words used must clearly express the testator's
intent to create a life estate. A very strong presumption arises that when a
person makes a will, the testator intended a complete disposition of his
property. ``[T]he very purpose of a will is to make such provisions that the
testator will not die intestate.'' \textit{Gilkey}, 207 S.W.2d at 73. When
faced with ambiguity, and in applying that presumption, courts generally
interpret wills to avoid creating an intestacy. 

\dots In reconciling different parts of a will, the Texas Supreme Court has
explained:
\begin{quote}
Where, however, the language of one part of a will is not easily reconciled with
that used in another, the principal and subordinate provisions should be
construed in their due relation to each other, and the intent which is
disclosed in the express clause ought to prevail over the language used in
subsidiary provisions, unless modified or controlled by the latter. And a
clearly expressed intention in one portion of the will will not yield to a
doubtful construction in any other portion of the instrument.
\end{quote}
\textit{Heller v. Heller}, 114 Tex. 401, 269 S.W. 771, 774 (1925).

\readinghead{Analysis}

A will is ambiguous if it is capable of more than one meaning. Because Debbie
and Daphne are in fact the same person, the ambiguity in Craigen's will becomes
apparent. Why would Craigen in the first sentence grant his wife a life estate,
but then in the concluding sentences bestow upon her all of his property? The
resolution of that question by Craigen's children seems reasonable, as the last
sentence could be construed to merely describe the property that Craigen
intended to include in Daphne's life estate.

On the other hand, Craigen did not mention his children in his will and he made
no provisions to expressly benefit them. Moreover, Brian and Sabrina's
construction of Craigen's will would, if adopted, allow all of Craigen's
property to pass under the laws of intestacy at Daphne's death. Brian and
Sabrina's construction assumes that Craigen, when writing his will, did not
intend to completely dispose of his estate. The rule that Craigen did not
likely intend to create an intestacy favors the construction of the will that
the trial court adopted.

Brian and Sabrina contend that the will gave Daphne a life estate, but Craigen
did not utilize those exact words in his will. Although no particular words are
needed to create a life estate, the words used must clearly express the
testator's intent to create one. In the absence of a remainderman clause, we
are skeptical that Craigen used the phrase ``till she dies'' in a technical
sense to create a life estate. Instead, Craigen likely intended to limit
Daphne's use of his property; nevertheless, the will manifests an intent that
she have his property in fee simple absolute. Consequently, although the first
sentence in the will is susceptible to the interpretation that Craigen created
a life estate, the will becomes ambiguous when, in the will's third sentence,
Craigen expressly names Daphne as the beneficiary of all of his property and he
makes no further provision for his estate upon her death.

We conclude that the will is reasonably capable of more than one meaning;
therefore, we resort to the rules of construction that apply to ambiguous
wills\dots. Craigen's will can be interpreted to avoid the intestacy certain
to result under Brian and Sabrina's construction of the will. The potential
intestacy is avoided if the phrase ``till she dies'' is interpreted as a
conditional bequest. The third sentence then functions as intended to give
Daphne all of Craigen's property in fee simple. The immediate vesting
construction favors Daphne, the sole beneficiary named in Craigen's will. It
also affords the phrase ``till she dies'' a nontechnical meaning.

We decline to apply the presumption that Craigen did not intend to disinherit
his children when the will expressly states that Craigen gave all of his real
and personal property to Daphne and when Brian and Sabrina offered no evidence
regarding Craigen's situation and the circumstances surrounding the execution
of the will. Taking the will as a whole, the dominant gift is all of Craigen's
real and personal property, and he made that gift to his wife. As this is the
dominant clause, Craigen's expressed intention prevails.

We hold that under the appropriate rules of will construction, the trial court
correctly construed the will. We overrule the issue and affirm the judgment.

\textsc{Affirmed}.

