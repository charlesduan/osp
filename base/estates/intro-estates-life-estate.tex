The \textit{life estate} is just what it sounds like: an estate that confers a
right to possession for the life of its owner. The owner of a life estate is
referred to as a \textit{life tenant.} The life estate terminates by operation
of law upon the owner's death (i.e., it ceases to exist). It is created by the
formula: ``to A for life.'' Because it must by definition end---we all have to
die sometime---any land held by a life tenant must also be subject to a
\textit{future interest} in some other person. We'll explore what those future
interests might be shortly.

\having{kotis-v-nowlin-qs}{Recall}{Consider}{Consider}
the legal principle of \textit{nemo dat quod non habet} (or \textit{nemo dat}
for short)\having{kotis-v-nowlin-qs}{, which we encountered in our discussion of
good faith purchasers}{}{}:
a grantor cannot convey title to something she doesn't herself own.
Following this principle, life estates are alienable \textit{inter vivos}
during the life of the life tenant, but obviously not devisable or
descendible by the life tenant: they cease to exist upon the death of their
owner, so the life
tenant's estate has nothing to convey.\footnote{A life estate can theoretically
be devised or inherited in the (perhaps contrived) situation where the life
tenant conveys to a third party, who dies before the life tenant; the third
party's heirs or devisees would receive the estate insofar as the original life
tenant is still alive.}
\textit{Nemo dat} also implies that the
owner of an interest in real property cannot convey \textit{more} than their
interest; a life tenant cannot convey a fee simple absolute, for example. More
to the point, if a life tenant A transfers their life estate to a grantee B, B
cannot receive anything more than what A owns: a possessory estate that will
terminate by operation of law \textit{when A dies}. Because such an interest is
measured by the life of someone other than its owner, it is called a
\textit{life estate pur autre vie} (literally, in Law French, ``for another
life''). A life estate \textit{pur autre vie} can also be created explicitly,
as by a grant ``to A for the life of B.''

We'll hold off on any further illustrative problems at this point, because we
still need some exposition of what happens \textit{after} a life tenant dies.
The answer, as we've already noted, involves \textit{future interests}.

