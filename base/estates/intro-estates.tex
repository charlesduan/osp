\captionedgraphic{homage-ceremony}{Homage Ceremony. From \textsc{James Henry
Breasted \& James Harvey Robinson}, 1 \textsc{Outlines of European History}
399 (1914).}

All land under the dominion of the English crown is held ``mediately or
immediately, of the king''---that is, the crown has ``radical title'' to all
land under its political dominion. William the Conqueror declared that all land
in England was literally the king's property; everyone else had to settle for
the privilege of holding it for him---the privilege of \textit{tenure} (from
the Norman French word ``tenir''---to hold). Tenurial rights were intensely
personal in early feudal society: the right to hold land was a privilege
granted by the crown in exchange for an oath of allegiance and a promise of
military service by the tenant---the oath of homage. The word homage derives
from the French word \textit{homme}---literally ``man''---precisely because
the ceremony surrounding the oath created not only the right of tenure, but a
political and military relationship between ``lord and man.''\footnote{The
ceremony of homage, recorded by the 13th-century jurist and
ecclesiastic Henry de Bracton, required the tenant to come to the lord in a
public place, and there
\begin{quote}
to place both his hands between the two hands of his
lord, by which there is symbolized protection, defense and warranty on the part
of the lord and subjection and reverence on that of the tenant, and say these
words: ``I become your man with respect to the tenement which I hold of you\dots
and I will bear you fealty in life and limb and earthly honour \dots
and I will bear you fealty against all men \dots saving the faith owed the
lord king and his heirs.'' And immediately after this [to] swear an oath of
fealty to his lord in these words: ``Hear this, lord N., that I will bear you
fealty in life and limb, in body, goods, and earthly honour, so help me God and
these sacred relics.''
\end{quote}
2 \textsc{Henry Bracton}, \textsc{On the Laws and Customs of England} 232
(Samuel E. Thorne trans., 1968) (c. 1230). The Anglo-Saxon Chronicle contains a
remarkable and much-debated passage in which William the Conqueror is said to
have held court at Salisbury twenty years into his reign, and there summoned and
taken direct oaths of homage and fealty from every landowner ``of any account''
in the whole of England. \textit{See} H. A. Cronne, \textit{The Salisbury Oath},
19 \textsc{History} 248 (1934); J.C. Holt, \textit{1086}, in \textsc{Colonial
England, 1066-1215, }at 31 (1997).} In exchange for the tenant's loyal support,
or \textit{fealty}, the lord warranted the tenant's right to hold a plot of
land, called a fief, or \textit{fee}.

Acceptance of this form of military tenure obligated the tenant to provide a
certain number of knights when called on by the king, and the land held by the
tenant was supposed to provide sufficient material support to enable him to
meet this military obligation. Sometimes, by the process of
\textit{subinfeudation}, the King's direct tenants (or ``tenants-in-chief'')
could spread this burden around by in turn accepting homage from other, lesser
nobles and freemen, each of whom would be responsible to the tenant-in-chief
for a portion of the tenant-in-chief's obligation to provide knight-service.
The tenants-in-chief thereby became ``mesne lords'' in their own right
(``mesne'' being Norman French for ``middle'' or ``intermediate''). There could
be several layers of mesne lords (i.e., ``land lords'') in the feudal
hierarchy, at the bottom of which were ``tenants in demesne'' (``demesne''
being Norman French for ``domain'' or ``dominion'')---who actually held the
land rather than subinfeudating it further. Of course, holding land did not
mean one actually worked it; a tenant in demesne often left the cultivation and
productive use of land to those of lower social status. These could be
``villeins''---serfs legally bound to the land by birth---or ``leasehold''
tenants---a leasehold being a right to hold land for a term of years in
exchange for payment of rent in cash or (more often) kind, and of lesser status
than the ``freehold'' estate held by feudal tenants tracing their rights up the
feudal pyramid to the crown.

Because a feudal tenant's land rights were intimately connected to this web of
personal, political, and military relationships, there was no logical reason
why the tenant ought to be free to transfer those rights to anyone else---and
good reason for the lords to resist such alienation of the fee by their
tenants. Indeed, fees could be forfeited to the lord for the tenant's breach of
the homage relationship or commission of some other ``felony,'' and on the
tenant's death it was not clear that his family members had the right to
inherit the fee. The king was assumed to have the right to retake the fee and
re-grant it to a preferable new tenant upon his displeasure with or the death
of the old tenant (it was his land, after all). Within a century, however, the
dynastic ambitions of the baronage compelled King Henry I to concede (in his
Coronation Charter of 1100) that a recently deceased baron's heir could redeem
his fee upon payment of ``a just and lawful relief''---i.e., a payment of money
to the crown, as a kind of inheritance tax. Under the principle of
primogeniture that took hold in England around this time, the lord's heir was
his eldest son; landowners were not free to choose who would take over their
tenancy after their death. Thus, subject to the payment of a relief, the fee
became \textit{descendible}---capable of being inherited from one generation
to the next---and the grant of a descendible tenancy by the crown was now made
not ``to Lord Hobnob,'' but ``to Lord Hobnob \textit{and his heirs}.'' To this
day, the latter phrase remains the classic common-law formula for creating the
broadest interest in land that the law will recognize: the \textit{fee simple
absolute}.

Descendibility of the fee simple having been settled early in the history of
English land law, the broader question of full alienability took several more
centuries to work out. The history of medieval English land law is a history of
tenants trying to secure their families' wealth and power by expanding
alienability and evading tenurial obligations to their lords and the crown,
while the crown and higher nobility tried to adapt the law to preserve their
status and prevent such evasions. There is a dialectical quality to this
history. For example: for complicated reasons subinfeudation quickly came to
present a greater threat to the economic interests of the higher ranks of the
feudal hierarchy than simple substitution of one tenant for another. Thus, in
1290 the Statute of Quia Emptores banned subinfeudation. But in doing so it
validated substitution, and with it the practice of selling an entire fee in
exchange for money during the life of the tenant. Similarly, in 1536, at the
insistence of King Henry VIII, the Statute of Uses abolished many clever
schemes adopted by tenants to use intermediaries to direct the disposition of
real property interests after death and to put those interests outside the
reach of the law courts (and of the crown's feudal authority). But in doing so,
the statute validated one type of flexible property arrangement we have come to
know as a \textit{trust}. Moreover, the removal of the primary mechanism
lawyers had developed to meet tenants' demand for intergenerational planning
was sufficiently unpopular that Henry felt compelled to consent to the
enactment of the Statute of Wills in 1540---finally permitting tenants to pass
their legal estates in land by will rather than being at the whim of the rule
of primogeniture. Finally, since the 16th century, primogeniture has given way
to a more complex system of default inheritance rights for various relatives of
the deceased who leaves no will; these rights are designed to try to
approximate what legislatures think the \textit{decedent} would have wanted,
not necessarily what is best for the government. This set of default rights
comprises the law of \textit{intestate succession}, which we will discuss in a
separate unit (or which you may study in a separate course on trust and estates
law).

Various other statutes and common-law developments over the centuries culminated
in the system of possessory estates and future interests that were imported
into the North American English colonies, and thus into the independent
American states (excluding Louisiana). Underlying them all is a fundamental
distinction that traces back to the ``radical title'' asserted by William the
Conqueror in 1066: \textbf{there is a conceptual difference between the
ownership \textit{of land} and the ownership of
\textit{a legal interest in that land}}. This distinction
remains important to modern property law, and this unit will introduce you to
the types of legal interests in land that American law will recognize. In
particular, it examines how the common law divides up legal interests in land
among successive owners over time.

Before delving into this material, we should warn you that the estates system
has limited relevance even for the practicing real estate lawyer of today. The
study of estates and future interests remains in property courses for three
primary reasons: (1) the estates are still legally valid property interests,
and their complexity can therefore can be a danger to lawyers who encounter
them and are unfamiliar with them; (2) some of the legal estates and future
interests in real property can be usefully extended to \textit{equitable}
interests in property held in trust; and (3)~the bar examiners are fond of
testing aspirant attorneys on future interests---perhaps simply because they
are fairly mechanical and therefore highly testable. To be sure, mastering the
system of estates and future interests requires considerable exercise of the
lawyerly skills of close reading, logical reasoning, and breaking down a big
problem into lots of smaller problems. But there are other ways of learning
those things, and a contemporary lawyer whose client wanted to divide up
interests in property would be courting malpractice by relying on legal estates
and future interests in land (which makes the bar examiners' continued
affection for them even more baffling). Instead, the modern lawyer should look
to the much more flexible law of trusts and to the various forms of business
associations---such as corporations---that can own property in their capacity
as fictional legal ``persons.''
\having{trusts-corporations}{We discussed these strategies in the chapter on
trusts and corporate property.}{We discuss these strategies in a separate
chapter on trusts and corporate property.}{}

