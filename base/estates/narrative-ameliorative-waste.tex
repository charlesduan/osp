\reading{Note on Ameliorative Waste}

What if, instead of doing something that \textit{decreases} the value of the
future interest, the holder of the possessory estate does something that
\textit{increases} the market value of the land, but in doing so changes the
premises in ways the future interest holder doesn't like? Such
alterations---known as \term{ameliorative waste}---have generated two types
of approaches in the courts.

The first approach, adopted in \textit{Melms v. Pabst Brewing Co.}, 79 N.W. 738
(Wisc. 1899), looks to the effect of the life tenant's actions on the market
value of the parcel and whether those actions were necessitated by a change in
conditions surrounding the parcel. In \textit{Melms}, the Pabst Brewing Company
had torn down an old mansion abutting a brewery it owned, mistakenly believing
it owned the lot in fee simple when in fact it owned only the life estate of
the widow Melms (the remainder being owned by her children). At the time of the
demolition, the neighborhood around the house had become heavily
industrialized, and had been re-graded such that the house stood 20-30 feet
above street level and was worthless as a residential property. In these
circumstances, the court held, whether the act of destroying the mansion and
re-grading the lot on which it stood to street level constitutes waste is a
question of fact for the jury. The court suggested that such actions will not
constitute waste ``when it clearly appears that the change will be, in effect,
a meliorating change, which rather improves the inheritance than injures it.''
\textit{Id.} at 739.

The second approach---more consistent with the common-law roots of waste
doctrine---holds that \textit{any} material change to real property caused by a
lawful possessor without the consent of the holder of the future interest is
waste, full stop. This approach informed the decision of the New York Supreme
Court in \textit{Brokaw v. Fairchild}, 237 N.Y.S. 6 (Sup. Ct. N.Y. Cty. 1929).
In that case, the court refused to allow the life tenant of a stately mansion
on New York's Fifth Avenue at 79th Street to tear the mansion
down over the objections of the holders of future interests in the lot, even
though living in the mansion had become cost-prohibitive and the neighborhood
had become a prime location for luxury apartment buildings, which could be
built and operated on the site for a substantial profit. The theory underlying
this result is that a life tenant has merely the rights of use, not full rights
of ownership, and that the holder of the future interest is entitled to take
possession of the parcel in substantially the same condition as it existed at
the time the future interest was created: ``The act of the tenant in changing
the estate, and whether or not such act is lawful or unlawful, i.e., whether
the estate is so changed as to be an injury to the inheritance, is the sole
question involved.'' \textit{Id.} at 15.

The opinion in \textit{Brokaw} generated a backlash in New York's reform-minded
legislature, which enacted a statute redefining waste law along the lines set
forth in \textit{Melms}; that statute remains in force today. \textit{See}
\textsc{N.Y. Real Prop. Acts. \& Procs. L.} {\S}~803. But interestingly, the
opinion in \textit{Melms} itself seems to have arisen from a number of
questionable factual and legal pronouncements from the Wisconsin courts. The
full, fascinating story is recounted in Thomas W. Merrill, \emph{\emph{Melms v.
Pabst Brewing Co.} and the Doctrine of Waste in American Property Law},
94
\textsc{Marq. L. Rev}. 1055 (2011). As of 2009, the rule of \textit{Melms} was
followed in most U.S. jurisdictions, while a small number continued to follow
the rule of \textit{Brokaw}. \textit{Id.} at 1083 (citing Gina Cora,
\textit{Want Not, Waste Not: Contracting Around the Law of Ameliorative Waste}
(Apr. 1, 2009) (Yale Law School Student Prize Papers: Paper 47), 
\url{http://digitalcommons.law.yale.edu/ylsspps_papers/47}).

Which of these two rules do you think is most consistent with the pluralist
justifications for waste doctrine described by Professor Purdy? Which do you
think is most consistent with the law-and-economics approach? Do either of the
rules require some other form of justification, and if so, what might that
justification be?

