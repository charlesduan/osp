\expected{jackson-v-brownson}

\item What exactly is the dispute between the majority and the dissent? Do they
agree on the existence of a remedy for waste under New York law? On the
definition of waste? On the applicability of waste doctrine to the lease before
the court? On the remedy for waste?


\item Although this case deals with a lease for life---a peculiar hybrid estate
that is not recognized in many jurisdictions---the doctrine of waste applies
between freehold possessory estate holders and future interest holders just as
it applies between leasehold tenants and landlords. Thus, even in the absence
of a lease contract, Brownson could have been held liable for damages, or
enjoined from felling any further timber, in an action for waste by the
reversioners (if the jury concluded that it would indeed be waste for a
possessor in Brownson's position to fell such timber).


\item \textbf{Forms of Waste.} Waste can be either \textit{voluntary} or
\textit{permissive}. Volutnary waste (sometimes called \textit{affirmative}
waste) refers to \textit{acts} of the holder of the possessory estate, such as
erecting or demolishing a structure, or extracting non-replenishing natural
resources. Permissive waste refers to \textit{omisssions} of the holder of the
possessory estate, such as failing to pay property taxes, or failure to make
needed repairs. Either can support a claim for waste by the owner of a future
interest whose rights are permanently devalued as a result. Which form of waste
was at issue in \textit{Jackson}?


\item \textbf{Theories of Waste.} One commentator argues that \textit{Jackson}
was the starting point for a peculiarly American departure from the English
doctrine of waste deplored by the dissenters. In this view, ``courts created
the American law of waste for several reasons: to promote efficient use of
resources that the English rule would have inhibited; to advance an idea of
American landholding as a republican enterprise, free of feudal hierarchy; and
perhaps to advance a belief that a natural duty to cultivate wild land underlay
the Anglo-American claim to North America.'' Jedediah Purdy, \textit{The
American Transformation of Waste Doctrine: A Pluralist Interpretation}, 91
\textsc{Cornell L. Rev}. 653, 661 (2006). And indeed, the sensitivity of both
opinions in \textit{Jackson} to local conditions, the desirability of
converting wild lands to agricultural use, and the sustainability of yeoman
farming tend to support this pluralist view.


\item Law-and-economics theorists, in contrast, identify waste doctrine solely
with the criterion of efficiency, and particularly the internalization of
externalities and mitigation of holdout problems. As Judge Posner puts it:
``The incentive of a life tenant is to maximize not the value of the
property---that is, the present value of the entire stream of future earnings
obtainable from it---but only the present value of the earnings stream
obtainable during his expected lifetime. So he will, for example, want to cut
timber before it has attained its mature growth even though the present value
of the timber would be greater if the cutting of some or all of it were
postponed; for the added value from waiting would inure to the
remainderman\dots. [Moreover,] since tenant and remainderman would have only
each other to contract with, the situation would be one of bilateral monopoly
and transaction costs might be high.'' To avoid these problems, ``[t]he law of
waste forbids the tenant to reduce the value of the property as a whole by
considering only his own interest in it.'' Richard A. Posner, \textit{Comment
on Merrill on the Law of Waste}, 94 \textsc{Marq. L. Rev}. 1095-96 (2011).

