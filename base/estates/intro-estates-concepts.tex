To begin understanding how the law divides up interests in land over time, we
begin with the fundamental distinction between \textbf{possessory
estates} and \textbf{future interests}. A \textbf{possessory estate}
is a legal interest that confers on its owner \textit{the right to
present possession} of some thing. A \textbf{future interest} is a legal
interest \textit{that exists in the present}, but does not entitle the
owner to possession until some point \textit{in the future}.

This may sound confusing, but you are probably already familiar with an
arrangement that follows this pattern: a lease. A lease is a transaction in
which the landlord gives the tenant a possessory estate (a leasehold estate),
and \textit{retains} a future interest---the right to retake possession after
the lease term ends. This retained future interest---an unqualified right to
future possession retained by the party who created the possessory interest
that precedes it---is called a \textbf{reversion}. (Landlord-tenant
relationships are obviously more complicated than this---they entail a number
of contractual rights and obligations and are heavily regulated by statutory
and decisional law and, in many cases, administrative codes.\having{leases}{}{We
cover these relationships more thoroughly in our unit on leases.}{})

The idea that both landlord and tenant can have legal interests in the same
parcel of land at the same time, even though only one of them has the right to
\textit{possess} the land at any given time, is a good introduction to the
concept of future interests. If you think about it, you will probably recognize
that the basic idea of a lease implies certain rights and powers of a landlord
in the leased premises even \textit{during} the term of the lease. The most
important one is the reversionary right itself: the right to take possession at
some point in the future. That's a right the tenant can't take away, even while
the tenant has the right to possession. The landlord might be interested in
selling (or mortgaging) this reversionary right, even before the lease ends.
And if she does sell or mortgage her interest (which she may, subject to the
tenant's interest), the thing sold is not ``the property''; it is \textit{the
landlord's reversion}: a legal interest in real property \textit{that exists in
the present} but will not entitle its holder to \textit{possession} of that
real property until some point \textit{in the} \textit{future}.

When learning about estates and future interests, we will follow some
conventions that will simplify our discussion as much as possible. Most of our
problems will involve an owner of land transferring some interest in that land
to one or more other parties. Following longstanding tradition in the study of
Anglo-American property law, we will refer to the parcel of land in question as
``Blackacre'' (or ``Whiteacre,'' ``Greenacre,'' ``Ochreacre,'' etc.\ if more
than one parcel is at issue). We will refer to the original owner as O, and the
other parties as A, B, C, etc.

In addition, there are a variety of technical terms that arise, a few of which
you should be familiar with:
\begin{itemize}
\item A \textit{grant} or \textit{conveyance} is a transfer of an interest in
property. The person making the grant is the \textit{grantor} (or
\textit{transferor}); the person receiving the grant is the \textit{grantee} (or
\textit{transferee}). If the grant is made during the life of the grantor,
it is said to be an \textit{inter vivos} conveyance (literally, ``between the
living''). If in a will, it is said to be a \textit{testamentary} conveyance. A
testamentary conveyance of real property is called a \textit{devise}. A
testamentary conveyance of personal property is called a \textit{bequest} (or
sometimes a \textit{legacy}).

\item When a person dies, they will either have left a valid will or not. A
person who dies with a valid will dies \textit{testate}; one who dies without a
valid will dies \textit{intestate}. Either way, the dead person can be
referred to as a \textit{decedent}. If the decedent did leave a valid will,
they may also be referred to as a \textit{testator} if male, or a
\textit{testatrix} if female. 

\item The assets that a decedent owned at her death are collectively referred to
as the decedent's \textit{estate}. An estate can sometimes take on the
qualities of a legal person---it is not uncommon to say that a certain asset is
owned by ``the estate of O.'' The property rights of this fictional legal
person are managed by an actual person whose title depends on whether the
decedent left a will. The instructions in a will are carried out by an
\textit{executor} (if male) or \textit{executrix} (if female), designated as
such in the will itself. An intestate estate is disposed of by a
court-appointed \textit{administrator} (if male) or \textit{administratrix} (if
female). 
\item The authority of an administrator or executor to dispose of the estate's
assets is conferred by a \textit{probate court}. When a valid will is filed
with the probate court and deemed valid, the court will \textit{admit the will
to probate} (or \textit{probate the will}), and will issue \textit{letters
testamentary} to the executor authorizing him to take possession of the
estate's assets and dispose of them according to the will's instructions. If
the decedent died intestate, the court will issue \textit{letters of
administration} to an administrator authorizing him to take possession of the
estate's assets and dispose of them according to the laws of intestate
succession.

\item If the decedent did leave a valid will, it will typically contain
instructions for transferring assets to various identified people or entities.
The parties receiving the bequests are referred to as the will's
\textit{beneficiaries}, \textit{devisees} (for real property),
or \textit{legatees} (for personal property). When a decedent passes property
by will he or she is said to have \textit{devised} that property. A property
interest that the decedent has the power to transfer by will is said to be
\textit{devisable}. 

\item Sometimes a will fails to provide instructions for all the assets owned by
the testator at death; in this case the unallocated assets are said to create a
\textit{partial intestacy}. When this happens, assets designated in the will
are distributed according to the will's terms, while the estate's remaining
assets are distributed according to the laws of intestate succession. In order
to avoid partial intestacy, it is good practice to include a \textit{residuary
clause} in a will, disposing of all the assets of the decedent not devised
through specific bequests. Such unenumerated assets are referred to as the
\textit{residuary estate}.

\item If the decedent did not leave a valid will, her property will pass to her
\textit{heirs} (sometimes referred to as \textit{heirs at law}). Heirs are
those who are designated by law as successors to property that passes by
intestate succession rather than by will. When heirs take such property, they
are said to \textit{inherit} it. A property interest that can pass by intestate
succession is said to be \textit{descendible}.

\item Note that until the decedent actually dies, we don't know who her heirs
are; rights of inheritance are allocated only to relatives of the decedent who
\textit{survive} her---who are still alive when the decedent dies. Thus, until
a property owner dies, her relatives have no legally enforceable rights in her
property under the laws of intestate succession. It is sometimes said that such
relatives have a mere \textit{expectancy}, and they are sometimes referred to
as \textit{heirs apparent}.

\item Heirs under intestacy laws are drawn from various categories of relatives.
In addition to spouses, there are \textit{issue}: the direct descendants of the
decedent (children, grandchildren, great-grandchildren, etc.);
\textit{ancestors} (parents, grandparents, great-grandparents, etc.); and
\textit{collaterals}: relatives who are not direct ancestors or descendants
(siblings, aunts, uncles, nieces, nephews, cousins).

\item If a person dies without a will and without any heirs at law, any property
in their estate \textit{escheats} to the state, which becomes its owner.
\end{itemize}



