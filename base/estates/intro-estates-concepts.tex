To begin understanding how the law divides up interests in land over time, we
begin with the fundamental distinction between possessory
estates and future interests. A \term{possessory estate}
is a legal interest that confers on its owner \textit{the right to
present possession} of some thing. A \term{future interest} is a legal
interest \textit{that exists in the present}, but does not entitle the
owner to possession until some point \textit{in the future}.

This may sound confusing, but you are probably already familiar with an
arrangement that follows this pattern: a lease. A lease is a transaction in
which the landlord gives the tenant a possessory estate (a leasehold estate),
and \textit{retains} a future interest---the right to retake possession after
the lease term ends. This retained future interest---an unqualified right to
future possession retained by the party who created the possessory interest
that precedes it---is called a \term{reversion}. (Landlord-tenant
relationships are obviously more complicated than this---they entail a number
of contractual rights and obligations and are heavily regulated by statutory
and decisional law and, in many cases, administrative codes.\having{leases}{}{We
cover these relationships more thoroughly in our unit on leases.}{})

The idea that both landlord and tenant can have legal interests in the same
parcel of land at the same time, even though only one of them has the right to
\textit{possess} the land at any given time, is a good introduction to the
concept of future interests. If you think about it, you will probably recognize
that the basic idea of a lease implies certain rights and powers of a landlord
in the leased premises even \textit{during} the term of the lease. The most
important one is the reversionary right itself: the right to take possession at
some point in the future. That's a right the tenant can't take away, even while
the tenant has the right to possession. The landlord might be interested in
selling (or mortgaging) this reversionary right, even before the lease ends.
And if she does sell or mortgage her interest (which she may, subject to the
tenant's interest), the thing sold is not ``the property''; it is \textit{the
landlord's reversion}: a legal interest in real property \textit{that exists in
the present} but will not entitle its holder to \textit{possession} of that
real property until some point \textit{in the future}.

When learning about estates and future interests, we will follow some
conventions that will simplify our discussion as much as possible. Most of our
problems will involve an owner of land transferring some interest in that land
to one or more other parties. Following longstanding tradition in the study of
Anglo-American property law, we will refer to the parcel of land in question as
``Blackacre'' (or ``Whiteacre,'' ``Greenacre,'' ``Ochreacre,'' etc.\ if more
than one parcel is at issue). We will refer to the original owner as O, and the
other parties as A, B, C, etc.




