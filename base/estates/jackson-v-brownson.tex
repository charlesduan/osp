\reading{Jackson v. Brownson}

\readingcite{7 Johns. 227 (N.Y. Sup. Ct. 1810)}

\dots This was an action of ejectment for a farm in Whitestown. The cause was
tried at the Oneida circuit, the 5th June, 1809, before Mr. Justice Yates.

At the trial, the plaintiff gave in evidence the counterpart of a lease, dated
the 3d September, 1790, from Philip Schuyler,\edfootnote{Yes, \emph{that} Philip
Schuyler. \emph{See} \emph{The Schuyler Sisters}, \emph{in} Lin Manuel-Miranda,
\emph{Hamilton} (2015).} of Albany, to the defendant, for the premises in
question, for the lives of the defendant, his wife, and Samuel Shaw,
respectively. The farm contained 133 acres and a half. The lease contained
various covenants, reservations and conditions, among which was the
following:\dots``And it is further conditioned on the part of the said lessee,
that
neither the said lessee, his executors, \&c.,\dots shall, at any time
hereafter, commit any waste.''

``And in case the said lessee, his, \&c., shall not perform, fulfil, abide by,
and keep all and every of the covenants and conditions herein covenanted and
conditioned, \&c., then in each of the said cases, it shall thenceforth be
lawful for the lessor, his, \&c., into the whole of the said premises, or into
any part thereof, in the name of the whole, to re\"enter, and the same to have
again, repossess and enjoy, as his or their former estate,'' \&c.

The lessors were the heirs of Philip Schuyler; this action was brought to
recover the possession of the south half of the premises, on the ground of
forfeiture by a breach of the covenant; the lessee or his assigns having
committed waste thereon by clearing and draining off the land more than a
reasonable and due proportion of the wood. It was admitted that, at the date of
the lease, the premises were wild and uncultivated, and covered throughout with
a forest of heavy timber.

The plaintiff proved that the defendant occupied the south half of the premises,
which were entirely cleared of wood, before the commencement of the suit; and
that on the north half occupied by Shaw, the whole was cleared except about six
or eight acres, on which more than half the wood and timber had been cut down
and removed, before the commencement of the suit.

It was also proved, that a permanent supply of fuel, timber for buildings, and
wood for fences, for the use of the demised premises, would require that, at
least, thirty acres should have been preserved in wood.

\dots It was also proved, that about 12 years since, there were 35 acres of
land covered with wood and timber on the premises, and about 12 acres of
woodland, on that part in the possession of the defendant, only half of which
was good for timber,\dots that the defendant had cut no wood or timber on
the part in his possession, except for fuel, fences, and building for the use
of the farm, and which had been gradually cut,\dots [that] the defendant had
built a house on the premises, which was completed about four years since; and
had used the farm in a husbandlike manner, and had carried on more materials
for fences than he had taken off; that\dots cleared land was of much greater
value than land covered with wood and timber; and that good farms in the
vicinity of the premises had not reserved more than 12 acres of woodland out of
100 acres\dots.

The judge was of opinion,\dots that the gradual clearing of that part in
possession of the defendant,\dots did not, in law, amount to waste; and he
directed the jury to find a verdict for the defendant; and the jury found
accordingly.

A motion was made to set aside the verdict and for a new trial, for the
misdirection of the judge.\dots

\opinion \textsc{Van Ness}, J.

\dots It is a general principle, that the law considers every thing to be
waste which does a permanent injury to the inheritance. Now, to say that
cutting down the wood on almost every acre of the demised premises is not
waste, within the spirit and meaning of the covenant in the case, is to say
that no waste, by the destruction of wood, can be committed at all. We are
bound to give effect to this covenant if we can, but to decide that the facts
stated in the case do not constitute waste, would be destroying it almost
altogether. That the destruction of the timber is a lasting injury to the
reversion cannot be disputed. For this injury the lessors of the plaintiff may,
at their election, bring covenant, or enter as for condition broken. 

\dots It is true, that what would in England be waste, is not always so here.
The covenant must be construed with reference to the state of the property at
the time of the demise. The lessee undoubtedly had a right to fell part of the
timber, so as to fit the land for cultivation; but it does not follow that he
may, with impunity, destroy all the timber, and thereby essentially and
permanently diminish the value of the inheritance. Good sense and sound policy,
as well as the rules of good husbandry, require that the lessee should preserve
so much of the timber as is indispensably necessary to keep the fences and
other erections upon the farm in proper repair. The counsel for the defendant
is mistaken when he says that lessees in England are prohibited from cutting
wood upon the demised premises altogether; the prohibition, in principle,
extends no further, in this respect, there than it does here. In England, that
species of wood which is denominated timber shall not be cut down, because
felling it is considered as an injury done to the inheritance, and therefore
waste. Here, from the different state of many parts of our country, timber may,
and must be cut down to a certain extent, but not so as to cause an irreparable
injury to the reversioner. To what extent wood may be cut before the tenant is
guilty of waste, must be left to the sound discretion of a jury, under the
direction of the court, as in other cases.\dots The principle upon which all
these cases were decided is that which I have before stated, namely, that
whenever wood has been cut in such a manner as materially to prejudice the
inheritance, it is waste; and that is the principle upon which I place the
decision of this cause.

\dots My opinion, therefore, is, that the motion for setting aside the nonsuit,
and granting a new trial, ought to be granted.

KENT, Ch. J., and THOMPSON, J., were of the same opinion.

SPENCER, J.

\dots The land was covered with heavy timber; and, for the use of it, the lessee
was to pay a rent. The parties must, therefore, have intended that the lessee
should be at liberty to fell the timber to a certain extent, at least, for
agricultural purposes.

If the restriction to commit waste would operate to restrain the lessee from the
use of the premises, it would be void, as repugnant to the grant. I shall have
no difficulty in maintaining that, according to the common law of England, the
lessee could not enjoy the land, nor derive any benefit from it, without the
commission of waste; and should that point be established, this covenant must
be rejected. The general definition of waste is, that it is a destruction in
houses, gardens, trees, or other corporeal hereditaments, to the disherison of
him in remainder or reversion. It is not every injury to lands that the law
considers as waste, nor every act which injures the remainder-man, or the
reversioner. To test this supposed waste, by considering the reversioner
injured by the acts done, is not warranted by law; and, in point of fact, when
the premises were cleared of the timber, cleared land was more valuable than
wood land.\dots I insist that, according to the common law of England, no
tenant can cut down timber, \&c., or clear land for agricultural purposes; and
that the quantity of timber cut down never enters into the consideration
whether waste has or has not been committed; but that it is always tested by
the fact of cutting timber, without the justifiable excuse of having done
it\dots. A single tree cut down, without such justifiable cause, is waste as
effectually as if a thousand had been cut down; and the reason is this, that
such trees belong to the owner of the inheritance, and the tenant has only a
qualified property in them for shade and shelter.

The doctrine of waste, as understood in England, is inapplicable to a new,
unsettled country.\dots The rule furnished by the common law is fixed and
certain; and the lessor knows what wood he may cut, and for what purposes; but
if a covenant not to commit waste is hereafter to be considered as a covenant
to leave a sufficient quantity of land in wood, no lessee is safe. If the act
of cutting timber on the premises, without the justifiable excuse already
stated, was not waste, cutting more or less was immaterial. Under the covenant
not to commit waste, we have no right to say some waste might be committed, and
other waste might not; the covenant is inapt to the case, and if any remedy
exists, it must lie in covenant. I am, therefore, against granting a new trial.

\opinion \textsc{Yates}, J., was of the same opinion.

Rule granted.

