A \textbf{remainder} is a type of future interest created in someone
\textit{other than} the grantor. The distinguishing characteristic of the
remainder is that---like a reversion---it \textit{cannot cut short or divest
any possessory estate}. (We will later encounter other future
interests that can.) A remainder simply ``remains,'' sitting around and waiting
for the natural termination of the preceding possessory estate (be it a life
estate or a lease), at which point the remainder will become possessory by
operation of law. Suppose that O, owning a fee simple absolute in Blackacre,
conveys Blackacre ``to A for life, \textit{and then to B.}'' Again, A would
have a life estate, but now O has also affirmatively created a future interest
in B. Because the future interest is created in someone \textit{other than} the
grantor, it isn't a reversion. And because it cannot cut short A's life estate
(note the ``and then'' language), it must therefore be a
\textbf{remainder}. Due to the persistence of dated gendered terms in
legal discourse, you will often see the holder of a remainder referred to as a
``remainderman,'' even today, regardless of that person's gender.

Future interests get a lot more complicated than this, but you now have enough
to begin examining some problems that can arise from even this limited set of
interests.

