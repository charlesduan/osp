Even if we are very clear on the nature and allocation of possessory and future
interests in a parcel of land, we soon run into a practical problem: it can be
difficult to protect the value of a future interest while someone else is in
possession of the land, acting for most purposes as its owner. What if a life
tenant burns down the structures on the parcel? Or decides to undertake a
remodeling project that would make the parcel less desirable to future renters?
Or fails to do anything about a leaky pipe, leading to a costly mold
infestation? What if the possessor uses the property in such a way as to
maximize its current value at the expense of its future value---depleting
natural resources, wearing out buildings and fixtures without repairing or
maintaining them---in ways that can't be recovered? Can it be wrongful---as a
matter of property law---for a lawful possessor to use the possessed premises
however they wish, for good or for ill?

The common law recognized that it \textit{could} be wrongful for a present
lawful possessor to take (or fail to take) certain acts with respect to land in
their possession---if those acts affected the ability of a \textit{future}
possessor to enjoy their interest when their turn came around. To vindicate the
rights of these future interest holders, the common law gave them a private
right of action to enjoin, and obtain damages for, the acts and omissions of
possessors that permanently decrease the value of the future interest. This was
the action for \term{waste}.

