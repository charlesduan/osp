\reading{Uniform Commercial Code}

\readinghead{\S~2-312. Warranty of title}

(1) Subject to subsection (2) there is in a contract for sale a warranty by the
seller that

\begin{statute}
\item (a) The title conveyed shall be good, and its transfer rightful; and

\item (b) The goods shall be delivered free from any security interest or other
lien or encumbrance of which the buyer at the time of contracting has no
knowledge.
\end{statute}

(2) A warranty under subsection (1) will be excluded or modified only by
specific language or by circumstances which give the buyer reason to know that
the person selling does not claim title in himself or that he is purporting to
sell only such right or title as he or a third person may have. \ldots

\readinghead{{\S} 2-403. Power to transfer; good faith purchase of goods;
``entrusting''}

(1) A purchaser of goods acquires all title which his transferor had or had
power to transfer except that a purchaser of a limited interest acquires rights
only to the extent of the interest purchased. A person with voidable title has
power to transfer a good title to a good faith purchaser for value. When goods
have been delivered under a transaction of purchase the purchaser has such
power even though

\begin{statute}
\item (a) The transferor was deceived as to the identity of the purchaser, or

\item (b) The delivery was in exchange for a check which is later dishonored, or

\item (c) It was agreed that the transaction was to be a ``cash sale,'' or

\item (d) The delivery was procured through fraud punishable as larcenous under
the criminal law.
\end{statute}

(2) Any entrusting of possession of goods to a merchant who deals in goods of
that kind gives him power to transfer all rights of the entruster to a buyer in
ordinary course of business.

(3) ``Entrusting'' includes any delivery and any acquiescence in retention of
possession regardless of any condition expressed between the parties to the
delivery or acquiescence and regardless of whether the procurement of the
entrusting or the possessor's disposition of the goods have been such as to be
larcenous under the criminal law.

