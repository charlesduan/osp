A chimney-sweep finds a jewel. He gives it to his friend, a jeweler, who designs
and crafts a gold ring around the jewel's unique shape. Frederica van Snoot
sees the ring, buys it for \$10,000, and wears it around town. One day,
Jeremiah Hobnob recognizes the jewel he lost last month and demands it back. As
against the chimney-sweep, this is an easy case; nowhere near enough time has
passed to satisfy the statute of limitations, even in a jurisdiction that
imposes a stringent duty of diligent search on owners. But the jeweler and van
Snoot are harder cases, because both of them have made investments. The jeweler
invested gold and labor to turn the jewel into a ring. For her part, van Snoot
paid out \$10,000. If Hobnob is entitled to the jewel, the jeweler or van Snoot
or both will end up poorer than when they started.

The common law mitigated the harshness of this result with two doctrines. One,
the rule of \textit{accession}, provided that someone who sufficiently improves
another person's property is allowed to keep it. Importantly, the hornbook rule
is that accession only operates in favor of \textit{good-faith} improvers;
someone who knows the property is not hers acts at her own peril when she
combines it with her own property or labor. The jeweler is potentially
protected by accession. The other doctrine protected \textit{good faith
purchasers for value} from the unknown claims of third parties. It too only
protects only parties who act in good faith, i.e., those who do not know or
have reason to know they are buying property with clouded title. Frederica van
Snoot may be just such a purchaser. 

This section presents both doctrines. As you read, pay close attention to the
preconditions that allow them to be invoked; despite their similarities, they
have important differences. For example, it is hornbook law that ``a thief
takes no title and can give none'': good-faith purchase can never cut off the
claims of an owner from whom the property was stolen. But accession can, as you
will see. Also, observe that while ownership of the property may be the primary
question in these cases, it is often not the only issue. Once ownership is
allocated, courts often require restitutionary payments to shift losses from
more innocent to more culpable parties. 

