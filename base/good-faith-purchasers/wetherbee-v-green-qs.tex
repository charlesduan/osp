\expected{wetherbee-v-green}

\item What factors matter most to the court's holding? Is this a case about the
relative value contributed by the plaintiff and defendant, about the difficulty
of identifying the plaintiff's original property, about the difficulty of
separating it, or about the degree to which it has been physically altered?
Consider \textit{Atlas Assurance Co. v. Gibbs}, 183 A. 690 (Conn.
1936), which involved the engine from a damaged car (the Hibben car)
that had been properly sold and the body of a car (the Sherline car) that had
been stolen. The defendant's predecessor in title combined the two to make one
working car. In an action for replevin by the assignee of title to the Sherline
car, who should get what?


\item How important is Wetherbee's good faith? What if he had been told by Green
that Sumner lacked authority, but had examined Sumner's title in some detail
and concluded that Green was wrong? What if Wetherbee steals a set of paints
and uses it to create a portrait that sells for \$500,000?


\item Note that Green retains ``an action to recover damages for the
unintentional trespass.'' What is the measure of those damages? Given that
Wetherbee owns the hoops via accession, why does he need to pay? Or, to look
at it another way, why \textit{doesn't} an adverse possessor need to pay for
the value of the property he retains after the statute of limitations has run?


\item Sometimes property transforms itself. A cow from Farmer Jones's herd
wanders onto Farmer Smith's land, where it is impregnated by Farmer Smith's
bull. Who owns the calf? Does it matter where the cow gives birth? Felix Cohen,
in \textit{Dialogue on Private Property}, 9 \textsc{Rutgers
L. Rev.} 357 (1954), claimed that every legal system in human history appears
to have resolved these cases in the same way. Compare the case in which Farmer
Smith's bull kicks Farmer Jones's cow and badly injures it. What result then?


\item \label{wetherbee-v-green-q-fungible}
Another theme in confusion cases involves the distinction between unique
and fungible property. If I mistakenly pour your 55-gallon drum of water into
my storage tank, you are entitled to draw 55 gallons of water from the tank,
even though it is astoundingly improbable that you will get back the same water
molecules you started with. Water is water. If I mistakenly mix your your
bottle of 1967 Chateau de Snoot wine with my bottle of 2015 Rotgut Red, I can't
give you a bottle of the resulting mixture and call it even. (What \textit{are}
you entitled to?) 

But note that uniqueness is something courts create as well as discover. At the
start of the 19th century, wheat and other grains were stored and sold as
though they were unique goods; each farmer's and merchant's sacks of grain were
treated as distinct from each other's. Today, grain has been standardized and
is sold as a commodity: a merchant could order 100 bushels of U.S. No. 1 Hard
Red Spring Wheat without needing to specify or worry about what particular
farms it came from. A key to this shift was courts' willingness to treat grain
(and many other agricultural commodities) as fungible. A merchant whose sacks
of wheat were dumped into a grain elevator without his consent would be
entitled to the same quantity of wheat of the same standard class, not to his
specific sacks or even to wheat with the same more specific characteristics.
What was gained and what was lost in this shift?

