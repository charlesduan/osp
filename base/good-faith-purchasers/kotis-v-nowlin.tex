\reading{Kotis v. Nowlin Jewelry, Inc.}
\readingcite{844 S.W.2d 920 (Tex. Ct. App. 1992)}

\opinion \textsc{Draughn}, Justice.

Eddie Kotis appeals from a judgment declaring appellee, Nowlin Jewelry, Inc.,
the sole owner of a Rolex watch, and awarding appellee attorney's fees. Kotis
raises fourteen points of error. We affirm.

On June 11, 1990, Steve Sitton acquired a gold ladies Rolex watch, President
model, with a diamond bezel from Nowlin Jewelry by forging a check belonging to
his brother and misrepresenting to Nowlin that he had his brother's
authorization for the purchase. The purchase price of the watch, and the amount
of the forged check, was \$9,438.50. The next day, Sitton telephoned Eddie
Kotis, the owner of a used car dealership, and asked Kotis if he was interested
in buying a Rolex watch. Kotis indicated interest and Sitton came to the car
lot[.] Kotis purchased the watch for \$3,550.00. Kotis also called Nowlin's
Jewelry that same day and spoke with Cherie Nowlin.

Ms. Nowlin told Kotis that Sitton had purchased the watch the day before. Ms.
Nowlin testified that Kotis would not immediately identify himself. Because she
did not have the payment information available, Ms. Nowlin asked if she could
call him back. Kotis then gave his name and number. Ms. Nowlin testified that
she called Kotis and told him the amount of the check and that it had not yet
cleared. Kotis told Ms. Nowlin that he did not have the watch and that he did
not want the watch. Ms. Nowlin also testified that Kotis would not tell her how
much Sitton was asking for the watch.

John Nowlin, the president of Nowlin's Jewelry, testified that, after this call
from Kotis, Nowlin's bookkeeper began attempting to confirm whether the check
had cleared. When they learned the check would not be honored by the bank,
Nowlin called Kotis, but Kotis refused to talk to Nowlin. Kotis referred Nowlin
to his attorney. On June 25, 1990, Kotis' attorney called Nowlin and suggested
that Nowlin hire an attorney and allegedly indicated that Nowlin could buy the
watch back from Kotis. Nowlin refused to repurchase the watch.

After Sitton was indicted for forgery and theft, the district court ordered
Nowlin's Jewelry to hold the watch until there was an adjudication of the
ownership of the watch. Nowlin then filed suit seeking a declaratory judgment
that Nowlin was the sole owner of the watch. Kotis filed a counterclaim for a
declaration that Kotis was a good faith purchaser of the watch and was entitled
to possession and title of the watch. After a bench trial, the trial court
rendered judgment declaring Nowlin the sole owner of the watch. The trial court
also filed Findings of Fact and Conclusions of Law.

In point of error one, Kotis claims the trial court erred in concluding that
Sitton did not receive the watch through a transaction of purchase with Nowlin,
within the meaning of Tex.Bus. \& Com.Code Ann. {\S} 2.403(a). Where a party
challenges a trial court's conclusions of law, we may sustain the judgment on
any legal theory supported by the evidence. Incorrect conclusions of law will
not require reversal if the controlling findings of facts will support a
correct legal theory.

Kotis contends there is evidence that the watch is a ``good'' under the UCC,
there was a voluntary transfer of the watch, and there was physical delivery of
the watch. Thus, Kotis maintains that the transaction between Sitton and Nowlin
was a transaction of purchase such that Sitton acquired the ability to transfer
good title to a good faith purchaser under {\S} 2.403 [which was identical in
relevant part to the UCC excerpt quoted above]. \ldots

Neither the code nor case law defines the phrase ``transaction of purchase.''
``Purchase'' is defined by the code as a ``taking by sale, discount,
negotiation, mortgage, pledge, lien, issue or reissue, gift or any other
voluntary transaction creating an interest in property.'' Tex. Bus. \& Com.
Code Ann. {\S} 1.201(32) (Vernon 1968). Thus, only voluntary transactions can
constitute transactions of purchase.

Having found no Texas case law concerning what constitutes a transaction of
purchase under {\S} 2.403(a), we have looked to case law from other states.
Based on the code definition of a purchase as a voluntary transaction, these
cases reason that a thief who wrongfully takes the goods against the will of
the owner is not a purchaser. \textit{See} \textit{Suburban Motors, Inc. v.
State Farm Mut. Automobile Ins. Co.}, 268 Cal. Rptr. 16, 18 (Cal. Ct. App.
1990); \textit{Charles Evans BMW, Inc. v. Williams}, 395 S.E.2d 650, 651-52 (Ga.
Ct. App. 1990); \textit{Inmi-Etti v. Aluisi}, 492 A.2d 917 (Md. Ct. App. 1985).
On the other hand, a swindler who fraudulently induces the victim to deliver
the goods voluntarily is a purchaser under the code.

In this case, Nowlin's Jewelry voluntarily delivered the watch to Sitton in
return for payment by check that was later discovered to be forged. Sitton did
not obtain the watch against the will of the owner. Rather, Sitton fraudulently
induced Nowlin's Jewelry to deliver the watch voluntarily. Thus, we agree with
appellant that the trial court erred in concluding that Sitton did not receive
the watch through a transaction of purchase under {\S} 2.403(a). We sustain
point of error one.

In point of error two, Kotis contends the trial court erred in concluding that,
at the time Sitton sold the watch to Kotis, Sitton did not have at least
voidable title to the watch. In point of error nine, Kotis challenges the trial
court's conclusion that Nowlin's Jewelry had legal and equitable title at all
times relevant to the lawsuit. The lack of Texas case law addressing such
issues under the code again requires us to look to case law from other states
to assist in our analysis.

In \textit{Suburban Motors, Inc. v. State Farm Mut. Automobile Ins. Co.}, the
California court noted that {\S} 2.403 provides for the creation of voidable
title where there is a voluntary transfer of goods. Section 2.403(a)(1)-(4) set
forth the types of voluntary transactions that can give the purchaser voidable
title. Where goods are stolen such that there is no voluntary transfer, only
void title results. Subsection (4) provides that a purchaser can obtain
voidable title to the goods even if ``delivery was procured through fraud
punishable as larcenous under the criminal law.'' This subsection applies to
cases involving acts fraudulent to the seller such as where the seller delivers
the goods in return for a forged check. Although Sitton paid Nowlin's Jewelry
with a forged check, he obtained possession of the watch through a voluntary
transaction of purchase and received voidable, rather than void, title to the
watch. Thus, the trial court erred in concluding that Sitton received no title
to the watch and in concluding that Nowlin's retained title at all relevant
times. We sustain points of error two and nine.

In point of error three, Kotis claims the trial court erred in concluding that
Kotis did not give sufficient value for the watch to receive protection under
{\S} 2.403, that Kotis did not take good title to the watch as a good faith
purchaser, that Kotis did not receive good title to the watch, and that Kotis
is not entitled to the watch under {\S} 2.403. In points of error four through
eight, Kotis challenges the trial court's findings regarding his good faith,
his honesty in fact, and his actual belief, and the reasonableness of the
belief, that the watch had been received unlawfully.

Under {\S} 2.403(a), a transferor with voidable title can transfer good title to
a good faith purchaser. Good faith means ``honesty in fact in the conduct or
transaction concerned.'' Tex.Bus. \& Com. Code Ann. {\S} 1.201(19) (Vernon
1968). The test for good faith is the actual belief of the party and not the
reasonableness of that belief. \textit{La Sara Grain v. First Nat'l Bank}, 673
S.W.2d 558, 563 (Tex.1984).

Kotis was a dealer in used cars and testified that he had bought several cars
from Sitton in the past and had no reason not to trust Sitton. He also
testified that on June 12, 1990, Sitton called and asked Kotis if he was
interested in buying a Ladies Rolex. Once Kotis indicated his interest in the
watch, Sitton came to Kotis's place of business. According to Kotis, Sitton
said that he had received \$18,000.00 upon the sale of his house and that he
had used this to purchase the watch for his girlfriend several months before.
Kotis paid \$3,550.00 for the watch. Kotis further testified that he then spoke
to a friend, Gary Neal Martin, who also knew Sitton. Martin sagely advised
Kotis to contact Nowlin's to check whether Sitton had financed the watch. Kotis
testified that he called Nowlin's after buying the watch.

Cherie Nowlin testified that she received a phone call from Kotis on June 12,
1990, although Kotis did not immediately identify himself. Kotis asked if
Nowlin's had sold a gold President model Rolex watch with a diamond bezel about
a month before. When asked, Kotis told Ms. Nowlin that Sitton had come to
Kotis' car lot and was trying to sell the watch. Ms. Nowlin testified that
Kotis told her he did not want the watch because he already owned a Rolex. Ms.
Nowlin told Kotis that Sitton had purchased the watch the day before. Kotis
asked about the method of payment. Because Ms. Nowlin did not know, she agreed
to check and call Kotis back. She called Kotis back and advised him that Sitton
had paid for the watch with a check that had not yet cleared. When Ms. Nowlin
asked if Kotis had the watch, Kotis said no and would not tell her how much
Sitton was asking for the watch. Ms. Nowlin did advise Kotis of the amount of
the check.

After these calls, the owner of Nowlin's asked his bookkeeper to call the bank
regarding Sitton's check. They learned on June 15, 1990 that the check would be
dishonored. John Nowlin called Kotis the next day and advised him about the
dishonored check. Kotis refused to talk to Nowlin and told Nowlin to contact
his attorney. Nowlin also testified that a reasonable amount to pay for a
Ladies President Rolex watch with a diamond bezel in mint condition was
\$7,000.00--\$8,000.00. Nowlin maintained that \$3,500.00 was an exorbitantly
low price for a watch like this.

The trier of fact is the sole judge of the credibility of the witnesses and the
weight to be given their testimony. Kotis testified that he lied when he spoke
with Cherie Nowlin and that he had already purchased the watch before he
learned that Sitton's story was false. The judge, as the trier of fact, may not
have believed Kotis when he said that he had already purchased the watch. If
the judge disbelieved this part of Kotis' testimony, other facts tend to show
that Kotis did not believe the transaction was lawful. For example, when Kotis
spoke with Nowlin's, he initially refused to identify himself, he said that he
did not have the watch and that he did not want the watch, he refused to
divulge Sitton's asking price, and he later refused to talk with Nowlin and
advised Nowlin to contact Kotis' attorney. Thus, there is evidence supporting
the trial court's finding that Kotis did not act in good faith.

There are sufficient facts to uphold the trial court's findings even if the
judge had accepted as true Kotis' testimony that, despite his statements to
Nowlin's, he had already purchased the watch when he called Nowlin's. The
testimony indicated that Kotis was familiar with the price of Rolex watches and
that \$3,550.00 was an extremely low price for a mint condition watch of this
type. An unreasonably low price is evidence the buyer knows the goods are
stolen. Although the test is what Kotis actually believed, we agree with
appellee that we need not let this standard sanction willful disregard of
suspicious facts that would lead a reasonable person to believe the transaction
was unlawful. Thus, we find sufficient evidence to uphold the trial court's
findings regarding Kotis' lack of status as a good faith purchaser. We overrule
points of error three through eight.\ldots

We affirm the trial court's judgment.

