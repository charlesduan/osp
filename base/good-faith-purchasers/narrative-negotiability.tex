\reading{Note on Negotiability}

\expected{wetherbee-v-green-qs}


Another version of the good-faith purchaser doctrine developed in the law of
intangible property called \term[negotiable instrument]{negotiable instruments}.
In the centuries before
the development of good national and international banking systems, merchants
commonly did business by passing around various promises or instructions to
pay. So, for example, Abel might buy a cartload of barrels of wine from Baker
on March 1 by giving Baker a signed promise to pay {\pounds}200 on June 1.
Baker could in theory sit on this \term{promissory note} until June 1 and
then demand payment from Abel. But instead, Baker was more likely to use the
note to pay his own debts: he might, for example, give it to Crumleigh on April
1 to buy a gold chain. Baker would sign, or \term{indorse}, the note, making
Crumleigh an assignee of Baker's right to collect from Abel, so that come June
1, Crumleigh could present the note to Abel and demand payment. Of course,
there was no need to stop there: Crumleigh could indorse the note over to
Daniels, and so on. In such a way, credit became a kind of currency, with the
note (collecting indorsements as it went) functioning as a token to indicate
who currently held the right to collect when the debt came due on June 1.

Another kind of signed promise, the \term{bill of exchange}, functioned
similarly. The difference was where Abel's note was a promise by Abel to pay, a
bill of exchange would be an instruction from Abel to a third party to pay.
Perhaps the bill would be ``drawn on'' Abel's business partner Absalom, or
perhaps more usefully it would be drawn on another merchant who had agreed to
extend Abel credit or make payments against amounts Abel had deposited with
him. If this sounds a bit like banking, it is not a coincidence; the modern
check is a direct descendent of the medieval bill of exchange.

Now back to our story. Suppose that Abel discovers that the wine Baker sold him
was rotten, good only as vinegar. Abel chases down Baker, only to learn that
Baker has already indorsed the note over to Crumleigh, who has already done the
same to Daniels. Come June 1, Daniels demands payment, but Abel refuses,
pointing to the worthless vinegar. Baker didn't hold up his side of the deal;
why should Abel have to do the same? In \textit{Miller v. Race}, (1758) 97 Eng.
Rep. 398 (K.B.), the great commercial jurist Lord Mansfield gave an answer:
\begin{quote}
After stating the case at large, he declared that at the trial, he had no sort
of doubt, but this action was well brought, and would lie against the defendant
in the present case; upon the general course of business, and from the
consequences to trade and commerce: which would be much incommoded by a
contrary determination. \ldots A bank-note is constantly and universally, both
at home and abroad, treated as money, as cash; and paid and received, as cash;
and it is necessary, for the purposes of commerce, that their currency should
be established and secured.
\end{quote}
The point is that if Daniels needs to check the details of the Abel-Baker
transaction---including inspecting the wine---to determine whether he will be
paid on Abel's note, he will refuse. He doesn't know Abel; he doesn't even know
Baker. The doctrine of \textit{Miller v. Race} is a good-faith-purchaser
doctrine for negotiable instruments; it lets Daniels rely on the note itself,
rather than inspecting the details over the underlying transaction. That in
turn lets the note circulate as money, enabling other transactions that
otherwise would have frozen up for lack of financing.

The doctrine of \term{negotiability}---``negotiation'' being the act of
assigning the
promise to pay from one recipient to another, typically by indorsing the note
and/or physically handing it over---took root in the United States. Indeed,
\textit{Swift v. Tyson}---famous for being the case overruled in \textit{Erie
Railroad Co. v. Tompkins}, 304 U.S. 64, 78 (1938) (``There is no federal
general common law.'')---was a case about negotiability. Norton and Keith
convinced Tyson to sign a bill of exchange for \$540.30, made payable to
Norton, who then negotiated it to Swift to pay off a preexisting debt. But when
Swift demanded payment from Tyson, Tyson replied that he had given it to Norton
and Keith ``as part consideration for the purchase of certain lands in the
state of Maine, which Norton and Keith represented themselves to be the owners
of'' but were not. The case turned on whether Swift was a ``bona fide holder
for a valuable consideration, without notice,'' in which case he was entitled
to collect from Tyson regardless of the land fraud Norton and Keith had
perpetrated on Tyson. The only issue there was whether cancellation of the
preexisting debt to Swift meant that Swift had given ``valuable consideration''
for the note, and again Justice Story's reasoning was pragmatic:
\begin{quote}
And we have no hesitation in saying, that a pre-existing debt does constitute a
valuable consideration in the sense of the general rule already stated, as
applicable to negotiable instruments. \ldots It is for the benefit and
convenience of the commercial world to give as wide an extent as practicable to
the credit and circulation of negotiable paper, that it may pass not only as
security for new purchases and advances, made upon the transfer thereof, but
also in payment of and as security for pre-existing debts. \ldots But
establish the opposite conclusion, that negotiable paper cannot be applied in
payment of or as security for pre-existing debts, without letting in all the
equities between the original and antecedent parties, and the value and
circulation of such securities must be essentially diminished, and the debtor
driven to the embarrassment of making a sale thereof, often at a ruinous
discount, to some third person, and then by circuity to apply the proceeds to
the payment of his debts. \ldots Probably more than one-half of all bank
transactions in our country, as well as those of other countries, are of this
nature. The doctrine would strike a fatal blow at all discounts of negotiable
securities for pre-existing debts.
\end{quote}
Today, negotiability shows up in many areas of commercial law. One good
illustration comes from Article 3 of the Uniform Commercial Code. A person is a
``holder in due course'' of a negotiable instrument (and here, think ``check''
or ``promissory note'') if 
\begin{quotation}
(1) The instrument when issued or negotiated to the holder does not bear such
apparent evidence of forgery or alteration or is not otherwise so irregular or
incomplete as to call into question its authenticity; and 

(2) The holder took the instrument (i) for value, (ii) in good faith, (iii)
without notice that the instrument is overdue or has been dishonored or that
there is an uncured default with respect to payment of another instrument
issued as part of the same series, (iv) without notice that the instrument
contains an unauthorized signature or has been altered, (v) without notice of
any claim to the instrument [either to recover the instrument after a theft or
to rescind the transaction in which it was transferred], and (vi) without
notice that any party has a defense or claim in recoupment\ldots
\end{quotation}
UCC {\S} 3-302(a). That's a long list of circumstances, but they're what you'd
expect. In addition to the usual requirement that the holder in due course give
value (and hence have a reliance interest in being paid), these are all issues
that either affect the authenticity of the instrument itself (paragraph (1)) or
go to the the holder's notice that something sketchy is afoot. But if a person
qualifies as a holder in due course, she receives extensive protections:
\begin{quotation}
(a) Except as stated in subsection (b), the right to enforce the obligation of a
party to pay an instrument is subject to the following: 

\begin{statute}
\item (1) A defense of the obligor based on (i) infancy of the obligor to the
extent
it is a defense to a simple contract, (ii) duress, lack of legal capacity, or
illegality of the transaction which, under other law, nullifies the obligation
of the obligor, (iii) fraud that induced the obligor to sign the instrument
with neither knowledge nor reasonable opportunity to learn of its character or
its essential terms, or (iv) discharge of the obligor in insolvency
proceedings; 

\item (2) A defense of the obligor stated in another section of this title or a
defense of the obligor that would be available if the person entitled to
enforce the instrument were enforcing a right to payment under a simple
contract; and 

\item (3) A claim in recoupment of the obligor against the original payee of the
instrument\ldots.
\end{statute}

(b) The right of a holder in due course to enforce the obligation of a party to
pay the instrument is subject to defenses of the obligor stated in subsection
(a)(1), but is not subject to defenses of the obligor stated in subsection
(a)(2) or claims in recoupment stated in subsection (a)(3) against a person
other than the holder. 
\end{quotation}
UCC {\S} 3-305. Notice the difference between the unconditional defenses in
(a)(1) and the ``personal'' defenses in (a)(2) and (a)(3). Only a few of the
contract defenses---infancy, incapacity, fraud in the factum, and bankruptcy
-- are available against a holder in due course. Something like Baker's
delivery of spoiled wine, even though it would give Abel a right to refuse
payment against Baker, will not be effective against a holder in due course
like Daniels. The effect is to turn a promise to pay into something stronger.
It is freely transferrable and it is no longer subject to the individual
defenses of the original promisor. In other words, assignability plus
negotiability turn an \textit{in personam} contract right into something that
looks much more like an \textit{in rem} property right. (Does this remind you
at all of how the courts turned unique things into fungible commodities in note
\ref{wetherbee-v-green-q-fungible} after \textit{Wetherbee}? It should.)

Negotiability is a powerful doctrine, and it can be a dangerous one. It can be
hard on promisors, particularly when they are the victims of fraud that doesn't
appear on the face of the negotiable instrument itself. In particular,
negotiability can be highly dangerous for consumers. If the promissory notes
for their debts have been sold by the initial creditor to another financial
institution, that institution may be able to collect on the debt even if the
initial transaction was fraudulent, unconscionable, or even criminal. For that
reason, the Federal Trade Commission's Holder in Due Course Rule, 16 C.F.R. pt.
433, requires consumer credit contracts to include language specifically
disclaiming negotiability. But the rule does not apply to mortgage loans,
\textit{see, e.g.}, \emph{Johnson v. Long Beach Mortg. Loan Trust 2001-4}, 451
F.
Supp. 2d 16, 54-55 (D.D.C. 2006). We will see some of the mischief and misdeeds
that resulted from the serial negotiation of residential mortgages in the
section on the mortgage crisis. For a sustained argument that the doctrine of
negotiability has long outlived its original purpose and does more harm than
good in an age of robust financial infrastructure, \emph{see}
\textsc{James Steven Rogers: The End of Negotiable
Instruments: Bringing Payment Systems Law Out of the Past} (2011).

