\expected{kotis-v-nowlin}

\item The common-law baseline is \textit{nemo dat quod non habet}: no man can
give what he does not have. If I ``give'' you a car I don't own, you don't own
it either. If I sell you a tract of land encumbered by a mortgage and an
easement, you receive only as much as I owned, so you take the land subject to
the mortgage and the easement. This \textit{nemo dat} baseline is the source of
the maxim that a thief cannot give good title. So if Sitton had held up
Nowlin's at gunpoint, how would the case have come out, and on what reasoning?

{\S} 2-403(1), as applied in \textit{Kotis}, distinguishes the thief's ``void''
title from merely ``voidable'' title: the quality of title obtained by the
buyer in a transaction that is for some reason defective. If the seller in that
defective transaction discovers the problem, she has a right to unwind the
transaction (and get her stuff back). But until she does, the buyer has the
power to convey not just his own, voidable title, but something even better. A
good-faith purchaser for value receives good title, \textit{even as against the
original seller}. Her right to unwind the transaction has been cut off. This is
a harsh way to treat an innocent victim of fraud or mistake. Why would property
law do something like that?


\item How did the parties get into this mess? Obviously Sitton is most to blame,
but is there anything Kotis or Nowlin could have done? Who is left holding the
bag and why? Is there anything Kotis can do to recover his \$3,550.00?


\item {\S} 2-403 provides for two tests that the buyer must meet to be protected
(in addition to the threshold question of whether his seller had voidable
title): he must act in good faith and he must give ``value.'' Which of these
tripped up Kotis? And what is the reason for not protecting donees along with
buyers? 

