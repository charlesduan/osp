\reading{Wetherbee v. Green}
\readingcite{22 Mich. 311 (1871)}

This was an action of replevin, brought by George Green, Charles H. Camp and
George Brooks, in the circuit court for the county of Bay, against George
Wetherbee, for one hundred and fifty-eight thousand black ash barrel-hoops,
alleged to be of the value of eight hundred dollars. \ldots

COOLEY J.:

The defendants in error replevied of Wetherbee a quantity of hoops, which he had
made from timber cut upon their land. Wetherbee defended the replevin suit on
two grounds. First, he claimed to have cut the timber under a license from one
Sumner, who was formerly tenant in common of the land with Green, and had been
authorized by Green to give such license. [This defense failed; Sumner was not
authorized to let Wetherbee cut timber on the land.]

But if the court should be against him on this branch of the case, Wetherbee
claimed further that replevin could not be maintained for the hoops, because he
had cut the timber in good faith, relying upon a permission which he supposed
proceeded from the parties having lawful right to give it, and had, by the
expenditure of his labor and money, converted the trees into chattels immensely
more valuable than they were as they stood in the forest, and thereby he had
made such chattels his own. And he offered to show that the standing timber was
worth twenty-five dollars only, while the hoops replevied were shown by the
evidence to be worth near seven hundred dollars; also [facts tending to show
Wetherbee's lack of knowledge of Sumner's duplicity]. The evidence offered to
establish these facts was rejected by the court, and the plaintiffs obtained
judgment.

The principal question which, from this statement, appears to be presented by
the record, may be stated thus: Has a party who has taken the property of
another in good faith, and in reliance upon a supposed right, without intention
to commit wrong, and by the expenditure of his money or labor, worked upon it
so great a transformation as that which this timber underwent in being
transformed from standing trees into hoops, acquired such a property therein
that it cannot be followed into his hands and reclaimed by the owner of the
trees in its improved condition?

The objections to allowing the owner of the trees to reclaim the property under
such circumstances are, that it visits the involuntary wrong-doer too severely
for his unintentional trespass, and at the same time compensates the owner
beyond all reason for the injury he has sustained. In the redress of private
injuries the law aims not so much to punish the wrongdoer as to compensate the
sufferer for his injuries; and the cases in which it goes farther and inflicts
punitory or vindictive penalties are those in which the wrong-doer has
committed the wrong recklessly, willfully, or maliciously, and under
circumstances presenting elements of aggravation. Where vicious motive or
reckless disregard of right are not involved, to inflict upon a person who has
taken the property of another, a penalty equal to twenty or thirty times its
value, and to compensate the owner in a proportion equally enormous, is so
opposed to all legal idea of justice and right and to the rules which regulate
the recovery of damages generally, that if permitted by the law at all, it must
stand out as an anomaly and must rest upon peculiar reasons.

As a general rule, one whose property has been appropriated by another without
authority has a right to follow it and recover the possession from any one who
may have received it; and if, in the mean time, it has been increased in value
by the addition of labor or money, the owner may, nevertheless, reclaim it,
provided there has been no destruction of substantial identity. So far the
authorities are agreed. A man cannot generally be deprived of his property
except by his own voluntary act or by operation of law; and if unauthorized
parties have bestowed expense or labor upon it, that fact cannot constitute a
bar to his reclaiming it, so long as identification is not impracticable. But
there must, nevertheless, in reason be some limit to the right to follow and
reclaim materials which have undergone a process of manufacture. Mr. Justice
Blackstone lays down the rule very broadly, that if a thing is changed into a
different species, as by making wine out of another's grapes, oil from his
olives, or bread from his wheat, the product belongs to the new operator, who
is only to make satisfaction to the former proprietor for the materials
converted: 2 Bl. Com., 404. We do not understand this to be disputed as a
general proposition, though there are some authorities which hold that, in the
case of a willful appropriation, no extent of conversion can give to the
willful trespasser a title to the property so long as the original materials
can be traced in the improved article. The distinction thus made between the
case of an appropriation in good faith and one based on intentional wrong,
appears to have come from the civil law, which would not suffer a party to
acquire a title by accession, founded on his own act, unless he had taken the
materials in ignorance of the true owner, and given them a form which precluded
their being restored to their original condition: 2 Kent, 363. While many cases
have followed the rule as broadly stated by Blackstone, others have adopted the
severe rule of the civil law where the conversion was in willful disregard of
right. The New York cases of \textit{Betts v. Lee}, 5 Johns., 348;
\textit{Curtis v. Groat}, 6 Johns., 168, and \textit{Chandler v. Edson}, 9
Johns., 362, were all cases where the willful trespasser was held to have
acquired no property by a very radical conversion, and in \textit{Silsbury v.
McCoon}, 3 N. Y., 378, 385, the whole subject is very fully examined \ldots .
[In \textit{Silsbury}, a thief who turned the plaintiff's corn into whiskey did
not thereby acquire ownership of it.] But we are not called upon in this case
to express any opinion regarding the rule applicable in the case of a willful
trespasser, since the authorities agree in holding that, when the wrong had
been involuntary, the owner of the original materials is precluded, by the
civil law and common law alike, from following and reclaiming the property
after it has undergone a transformation which converts it into an article
substantially different.

The cases of confusion of goods are closely analogous. It has always been held
that he who, without fraud, intentional wrong, or reckless disregard of the
rights of others, mingled his goods with those of another person, in such
manner that they could not be distinguished, should, nevertheless, be protected
in his ownership so far as the circumstances would permit. The question of
motive here becomes of the highest importance; for, as Chancellor Kent says, if
the commingling of property ``was willfully made without mutual consent, * *
the common law gave the entire property, without any account, to him whose
property was originally invaded, and its distinct character destroyed: Popham's
Rep. 38, Pl. 2. If A will willfully intermix his corn or hay with that of B, or
casts his gold into another's crucible, so that it becomes impossible to
distinguish what belonged to A from what belonged to B, the whole belongs to B.
But this rule only applies to wrongful or fraudulent intermixtures. There may
be an intentional intermingling, and yet no wrong intended, as where a man
mixes two parcels together, supposing both to be his own; or, that he was about
to mingle his with his neighbor's, by agreement, and mistakes the parcel. In
such cases, which may be deemed accidental intermixtures, it would be
unreasonable and unjust that he should lose his own or be obliged to take and
pay for his neighbor's, as he would have been under the civil law: In many
cases there will be difficulty in determining precisely how he can be protected
with due regard to the rights of the other party; but it is clear that the law
will not forfeit his property in consequence of the accident or inadvertence,
unless a just measure of redress to the other party renders it inevitable. 

The important question on this branch of the case appears to us to be, whether
standing trees, when cut and manufactured into hoops, are to be regarded as so
far changed in character that their identity can be said to be destroyed within
the meaning of the authorities. And as we enter upon a discussion of this
question, it is evident at once that it is difficult, if not impossible, to
discover any invariable and satisfactory test which can be applied to all the
cases which arise in such infinite variety. ``If grain be taken and made into
malt, or money taken and made into a cup, or timber taken and made into a
house, it is held in the old English law that the property is so altered as to
change the title:'' 2 Kent, 363. But cloth made into garments, leather into
shoes, trees hewn or sawed into timber, and iron made into bars, it is said may
be reclaimed by the owner in their new and original shape: Some of the cases
place the right of the former owner to take the thing in its altered condition
upon the question whether its identity could be made out by the senses. But
this is obviously a very unsatisfactory test, and in many cases would wholly
defeat the purpose which the law has in view in recognizing a change of title
in any of these cases. That purpose is not to establish any arbitrary
distinctions, based upon mere physical reasons, but to adjust the redress
afforded to the one party and the penalty inflicted upon the other, as near as
circumstances will permit, to the rules of substantial justice[.]

It may often happen that no difficulty will be experienced in determining the
identity of a piece of timber which has been taken and built into a house; but
no one disputes that the right of the original owner is gone in such a case. A
particular piece of wood might, perhaps, be traced without trouble into a
church organ, or other equally valuable article; but no one would defend a rule
of law which, because the identity could be determined by the senses, would
permit the owner of the wood to appropriate a musical instrument, a hundred or
a thousand times the value of his original materials, when the party who, under
like circumstances, has doubled the value of another man's corn by converting
it into malt, is permitted to retain it, and held liable for the original value
only. Such distinctions in the law would be without reason, and could not be
tolerated. When the right to the improved article is the point in issue, the
question, how much the property or labor of each has contributed to make it
what it is, must always be one of first importance. The owner of a beam built
into the house of another loses his property in it, because the beam is
insignificant in value or importance as compared to that to which it has become
attached, and the musical instrument belongs to the maker rather than to the
man whose timber was used in making it---not because the timber cannot be
identified, but because, in bringing it to its present condition the value of
the labor has swallowed up and rendered insignificant the value of the original
materials. The labor, in the case of the musical instrument, is just as much
the principal thing as the house is in the other case instanced; the timber
appropriated is in each case comparatively unimportant.

No test which satisfies the reason of the law can be applied in the adjustment
of questions of title to chattels by accession, unless it keeps in view the
circumstance of relative values. When we bear in mind the fact that what the
law aims at is the accomplishment of substantial equity, we shall readily
perceive that the fact of the value of the materials having been increased a
hundred-fold, is of more importance in the adjustment than any chemical change
or mechanical transformation, which, however radical, neither is expensive to
the party making it, nor adds materially to the value. There may be complete
changes with so little improvement in value, that there could be no hardship in
giving the owner of the original materials the improved article; but in the
present case, where the defendant's labor---if he shall succeed in sustaining
his offer of testimony---will appear to have given the timber in its present
condition nearly all its value, all the grounds of equity exist which influence
the courts in recognizing a change of title under any circumstances.

We are of opinion that the court erred in rejecting the testimony offered. The
defendant, we think, had a right to show that he had manufactured the hoops in
good faith, and in the belief that he had the proper authority to do so; and if
he should succeed in making that showing, he was entitled to have the jury
instructed that the title to the timber was changed by a substantial change of
identity, and that the remedy of the plaintiff was an action to recover damages
for the unintentional trespass. \ldots

