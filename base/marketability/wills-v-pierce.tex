\reading{Wills v. Pierce}
\readingcite{208 Ga. 417 (1951)}

\dots Mrs. Walter Tilley Pierce and others filed in Terrell Superior Court,
against Mrs. J. C. Wills and others, a petition, which alleged substantially
the following: On December 1, 1923, J. W. Tilley by warranty deed conveyed
described realty known as the Aven Home to J. C. Wills. The deed contained the
clause: ``The above property is conveyed to J. C. Wills [the grantee] to be used
as a home by himself, his family and his heirs, upon condition that the same be
used by him or them as a home and a residence, and further that upon the
failure of the said condition and the abandonment of said property as a
residence by [the grantee],\dots his family or heirs, the same shall revert
to [the grantor's]\dots estate and go as directed by [the grantor's]\dots
will.'' The grantor died testate in 1924, and under the terms of his will the
petitioners are the owners of the reversionary interest in the realty. The
grantee died intestate in 1945, leaving as his sole surviving heirs his widow,
Mrs. J. C. Wills, and two named children, who are the defendants. The condition
under which the realty was conveyed has been violated, in that the defendants
have abandoned the property as a home and residence, and are now residing
elsewhere.\dots The petitioners prayed \dots that the interest of the
defendants in the realty be declared forfeited, and the fee-simple title
thereof be decreed to be in the petitioners; and that the petitioners have
general equitable relief.

The defendants demurred to the petition on the ground that it failed to set
forth any cause of action against them. The trial court overruled the demurrer,
and the defendants excepted\dots.

\textsc{Atkinson}, Presiding Justice (after stating the foregoing facts).

The granting clause in the deed under consideration was: ``In consideration of
the sum of one dollar to me paid, I\dots do hereby sell and convey to [the
grantee and]\dots his heirs, a tract or parcel of land and appurtenances in
fee simple.'' Then followed a description of the land, after which the grantor
inserted the provision that the property was to be used as a home by the
grantee, his family, and his heirs, and that upon the abandonment of the
property as a residence by the grantee, his family, or his heirs, the same
should revert to the grantor's estate and go as provided in his will.

Standing alone, the first clause in the deed would have conveyed an
unconditional fee-simple estate, and the sole question for determination is
whether or not the condition subsequent under which the forfeiture is claimed
is valid and enforceable.

A provision in a deed or will that a fee-simple estate may not be sold is void
as being repugnant to the estate granted.

While no express language is used in the present deed inhibiting alienation of
the property, nevertheless---the condition being that the property was to be
used as a home by the grantee, his family, and his heirs---the requirement to
use as a home and the right to sell are mutually exclusive, and whether or not
the case falls within the rule against perpetuities, the conclusion is
inescapable that since the grantee and his heirs must use the premises as a
home they cannot sell it.

\dots A different question would have been presented if the condition subsequent
had been that the premises should be used ``as a home'' or ``for residential
purposes'' generally. See, in this connection, \emph{City of Barnesville v.
Stafford}, 161 Ga. 588(1), 131 S.E. 487, 43 A.L.R. 1045; \emph{Taylor v. Bird},
150 Ga. 626, 104 S.E. 502; \emph{Rustin v. Butler}, 195 Ga. 389, 24 S.E.2d 318;
\emph{Williams v. Ramey}, 201 Ga. 737(1), 41 S.E.2d 159; \emph{Tabor v. Gilmer
County}, 205 Ga. 439(1), 53 S.E.2d 915; and similar cases, where conditions
subsequent requiring use of property generally for park, school, religious, and
courthouse purposes were held valid and enforceable.

Accordingly, the present petition, seeking to enforce a forfeiture for breach of
a void condition subsequent, failed to set forth a cause of action, and the
trial court erred in overruling the defendants' general demurrer.

Judgment reversed.

All the Justices concur.

