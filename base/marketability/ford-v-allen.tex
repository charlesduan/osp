\reading{Ford v. Allen}

\readingcite{526 S.W.2d 643 (Ct. Civ. App. Tex. 1975)}

\textsc{O'Quinn}, Justice.

Chester Melvin Ford and Lola Mae Ford, the deceased persons whose wills are
under review, were married in August of 1943, and Clyde M. Ford, appellant here
and plaintiff below, was the only child born to their marriage. Mr. [Chester]
Ford had been married twice prior to his marriage to Mrs. Ford, but had no
children from those marriages. Mrs. Ford also had been married earlier, and
from that marriage she had a son, Otis Martin Allen, who died in April of 1958,
leaving three sons, resulting from two marriages. The three surviving sons were
defendants below and are appellees in this appeal.

The undisputed evidence supports the finding of the trial court that Chester
Melvin Ford and his wife, Lola Mae Ford, each executed a holographic will on
the same day in April of 1960, and each of them devised ``all my property to my
beloved'' spouse, followed by certain additional identical language which is
under dispute. Mr. Ford died November 25, 1972, and less than a month later
Mrs. Ford died, on December 18, 1972.

It also appears undisputed that, as the court found, Mr. Ford at the time of his
death owned approximately 450 acres of land in Bell County\dots.

The language of the wills giving rise to this suit, as contained in the will of
Mr. Ford, follows:
\begin{quote}
After the Payments of my Just Debts I devise all my property to my beloved
wife Lola Mae Ford to do with as she See fit except that she is not to Sell,
Morage (sic), or Lease any of our real Estate for more than Three (3) years
without the written agreement of our son Clyde Melvin Ford.
\end{quote}
Appellant contends that the language is ambiguous and requires construction, and
that under a proper construction the language ``created a life estate in real
property in Lola Mae Ford with remainder to Clyde M. Ford in fee simple, or
alternatively created a testamentary trust expressly or by implication for the
use and benefit of Clyde M. Ford.''

\dots The trial court concluded (1) that the language in the wills, providing
that the devisee was not to sell, mortgage, or lease any of the realty for
three years without written agreement of Clyde M. Ford, was ``void as being a
restraint on alienation and repugnant to the devise in fee;'' and (2) the
language of the wills\dots devised fee simple title to all property, since
the wills contained no ``language clearly showing a lesser estate than the fee
was intended to be devised.'' We approve these conclusions as correct
applications of the law to the language of the wills.

Appellant contends that by extrinsic evidence it may be demonstrated that the
true intent of Mr. and Mrs. Ford was to devise their real property to their
only son, Clyde M. Ford, and that because of the ambiguity of the language in
the wills, such evidence should have been considered\dots.

In brief, the evidence was that the real estate was the separate property of Mr.
Ford, and that the three grandsons of Mrs. Ford were not kin to Mr. Ford; that
Clyde M. Ford had helped to work the lands contained in the 450 acres, whereas
the defendants had never worked any part of the land; that the grandsons were
not close to their grandmother or to Mr. Ford, and none of them attended either
the funeral of Mr. Ford or their grandmother; that Mrs. Ford set up a savings
account for the grandsons and this alone was intended to take care of them;
that Clyde M. Ford was close to his parents and was the natural object of the
deceaseds' bounty, and the defendants were not; that during their life both Mr.
and Mrs. Ford indicated orally that they wanted Clyde to have the land.

It is the established rule that an ambiguity arises only when the meaning which
emanates from language used in the will admits of more than one interpretation.
We find no ambiguity in the language of the Ford wills which in each writing
clearly and plainly devises all property to the other spouse to do with as the
other may see fit. The attempt, in language that follows, to place a restraint
on alienation could not change or nullify the devise. It is not a function of
the courts, nor is it a role the courts may assume, to revise or to make over
the writing in a will to achieve results different from results which flow from
the plain language used by the maker of the will. The courts may not speculate,
from extrinsic evidence or otherwise, that some other result may have been
intended.

\dots Appellant also urges that the trial court erred in refusing to make a
determination of heirship, and under these points insists that if Mrs. Ford
died intestate, appellant is entitled to one-half of her estate and defendants
are entitled only to the remaining one-half. The trial court correctly declined
to decide the matter of heirship since administration of the estates is still
pending in Bell County, where the County Court has acquired jurisdiction to
determine heirs of the deceased.

All of appellant's point of error have been carefully examined and considered,
and all points are overruled.

The trial court in its judgment denied the request of Clyde M. Ford that
attorney's fees, court costs, and other expenses incurred by this suit to
construe the wills be paid out of the two estates as costs and expenses of
administration, and ordered all such costs and expenses to be paid by Ford
individually.

The judgment of the trial court is in all things affirmed. It is ordered that
costs of this appeal be taxed against appellant, Clyde M. Ford, individually.

