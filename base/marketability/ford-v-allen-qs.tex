\expected{ford-v-allen}

\item Do you think Clyde is right that his parents wanted him to have the farm
after both of them died? Or at least that they would rather Clyde have it than
Lola Mae's estranged grandchildren from another marriage? If so, why do you
think both Chester and Lola Mae executed wills without any explicit devise to
Clyde? If not, why do you think the Fords' wills included a restriction on
alienating the farm without Clyde's consent? \withterm{holographic
will}{Consider the previous discussion of \checkterm{holographic
will}holographic wills here.}{(Incidentally, what is a ``holographic'' will?)}


\item Why is the court unwilling to consider Clyde's evidence that his parents
wanted him to have the farm? What's wrong with looking outside the four corners
of the will itself to understand what the testator \textit{really} wanted?
Would we take a similar view of extrinsic evidence if the document being
interpreted were, say, a contract for the sale of goods?


\item Justice O'Quinn says that the language of the Fords' wills ``clearly and
plainly devises all property to the other spouse to do with as the other may
see fit.'' But this is at best disingenuous and at worst deliberately false:
the wills also, \textit{in the very next clause}, ``clearly and plainly''
purport to limit what the other spouse can do with the property in the absence
of Clyde's consent. Why does the court enforce the former clause and render the
latter clause void? 


The reasoning of the trial court in this case may help explain things. Note that
the trial court is said to have given two somewhat different reasons for
invalidating Clyde's power to block any effort by his surviving parent to
alienate the farm (and with it any future interest he might have claimed by
implication from this right). We are told that an attempt to convey such a
power to Clyde must be void, both ``as a restraint on alienation,'' and as
``repugnant to the\dots fee.'' These reasons invoke two long-standing
common-law principles: a policy against \term{restraints on alienation}, and the
doctrine of \term{numerus clausus}.



Courts have generally strongly disfavored overt restraints on a grantee's right
to alienate their interest. Such restraints can make it quite difficult to move
resources from lower-valued to higher-valued uses. A current owner of a
resource might well be willing to sell it to a willing buyer who wants it more
and can make more valuable use of it, but if we enforce a restraint on
alienation imposed on the current owner by a past grantor, such a beneficial
transaction cannot happen. The result would be serious misallocation of
resources, and the rule that restraints on alienation are void demonstrates the
common law's willingness to defeat even the clearly expressed intent of a
grantor where necessary to avoid such misallocation.



\textit{Numerus clausus} (literally, ``the number is closed'') is a legal
principle derived from civil law systems but invoked in Anglo-American property
law to refuse recognition of any interest in land other than the traditional
common-law estates. Under this principle property owners may not create any new
``bundle of rights'' other than those that are already represented by the
common-law estates themselves. So, because a possessory estate subject to a
veto on the right of alienation by someone other than the possessory estate's
owner is not a ``bundle of rights'' that we can identify among our common-law
estates, it must be outside the power of the Fords to create it. Courts have
similarly rejected efforts by testators to, for example, give their surviving
spouses unfettered control over devised property while also giving any property
left over at the surviving spouse's death to another beneficiary. Such hybrid
bequests are, like the devise in \textit{Ford}, typically treated as a fee
simple (rendering the putative future interest void). \textit{See, e.g.},
\emph{Sumner v. Borders}, 98 S.W.2d 918 (Ky. 1936).



Is the rule of \textit{numerus clausus} motivated by the same rationales that
give rise to the rule against restraints on alienation? Imagine if, rather than
selecting from the fixed menu of common-law estates, property owners were free
to build their own tailored bundles of property interests for grantees, with
their own ad hoc collections of limitations and restrictions on the rights of
those grantees, and that these idiosyncratic collections of rights and
limitations became commonplace across society. Suppose you now want to buy a
parcel of land in that society. Can you be sure what you're buying? How? How
well would we expect a real estate market built on a potentially infinite
variety of interests in real property to function? \textit{See generally}
Thomas W. Merrill \& Henry E. Smith, \textit{Optimal Standardization in the Law
of Property: The Numerus Clausus Principle}, 110 \textsc{Yale L.J.} 1 (2000).


\item Are there other principles underlying the rule against restraints on
alienation or the \textit{numerus clausus} principle other than ensuring a
well-working market for property rights? Consider that the law of
\textit{intellectual property} has long included a so-called ``first
sale'' doctrine, which provides that the first authorized purchaser of a good
embodying an intellectual property right (for example, a book embodying a
copyrighted work, or a machine embodying a patented invention) has the power to
alienate \textit{that particular article} free of any claim by the
intellectual property right owner. \textit{See, e.g.}, 17 U.S.C. {\S} 109(a)
(copyright); \emph{Adams v. Burke}, 84 U.S. 453, 456 (1873) (``[W]hen the
patentee, or
the person having his rights, sells a machine or instrument whose sole value is
in its use, he receives the consideration for its use and he parts with the
right to restrict that use.''). At least where the owners of the relevant
intellectual property rights can be clearly identified, can this rule be
justified by the same principle as the rule against restraints on alienation of
\textit{land}? If not, what is the rationale for the first-sale doctrine?

\defwebsite{kindle-tou}{
name=Kindle Store Terms of Use,
date=sept 6 2012,
journal=Amazon,
url={https://www.amazon.com/gp/help/customer/display.html?nodeId=201014950},
}

\item Consider the following excerpts from the September 6, 2012 Amazon Kindle
Store Terms of Use Agreement,\note{kindle-tou}
which governs the downloading of electronic copies of copyrighted literary
works from Amazon for viewing on electronic devices. 
\begin{quotation}
``Kindle Content'' means digitized electronic content obtained through the
Kindle Store, such as books, newspapers, magazines, journals, blogs, RSS feeds,
games, and other static and interactive electronic content. 

\dots.

\textbf{Use of Kindle Content.} Upon your download of Kindle Content
and payment of any applicable fees (including applicable taxes), the Content
Provider grants you a non-exclusive right to view, use, and display such Kindle
Content an unlimited number of times, solely on the Kindle or a Reading
Application or as otherwise permitted as part of the Service, solely on the
number of Kindles or Supported Devices specified in the Kindle Store, and
solely for your personal, non-commercial use. Kindle Content is licensed, not
sold, to you by the Content Provider\dots.

\textbf{Limitations.} Unless specifically indicated otherwise, you may
not sell, rent, lease, distribute, broadcast, sublicense, or otherwise assign
any rights to the Kindle Content or any portion of it to any third party, and
you may not remove or modify any proprietary notices or labels on the Kindle
Content. In addition, you may not bypass, modify, defeat, or circumvent
security features that protect the Kindle Content. 

\dots.

\textbf{Termination.} Your rights under this Agreement will
automatically terminate if you fail to comply with any term of this Agreement.
In case of such termination, you must cease all use of the Kindle Store and the
Kindle Content, and Amazon may immediately revoke your access to the Kindle
Store and the Kindle Content without refund of any fees. Amazon's failure to
insist upon or enforce your strict compliance with this Agreement will not
constitute a waiver of any of its rights.
\end{quotation}
Is this agreement consistent with the rules you've just learned? If not, is it
enforceable? \textit{See} \emph{Vernor v. Autodesk, Inc.}, 621 F.3d 1102, 1111
(9th Cir. 2010).


\item Might a grantor impose restrictions on a grantee \textit{other than}
explicit limitations on the power to alienate that would raise the same
concerns as those that motivate the rule against restraints on alienation?
Consider the following cases:

\expectnext{wills-v-pierce}
\expecting{smedley-v-waldron}

