\having{advanced-estates}{
\item We learned in our unit on estates and future interests that grantors may
use the defeasible fees to impose conditions subsequent on the continued
enjoyment of a possessory estate. Are we still confident that grantors have
such a power? If so, what are its limits?
}{
\item We will learn in our unit on estates and future interests that grantors
may use the defeasible fees to impose conditions subsequent on the continued
enjoyment of a possessory estate. Does \emph{Wills} suggest that grantors have
such a power? If so, what are its limits?
}{
\item A ``fee simple subject to a condition subsequent'' is another type of
estate, in which the grantee's property rights may terminate if a condition that
the grantor specified becomes true. Do grantors appear to have the power to set
such conditions subsequent after \emph{Wills}? If so, what are its limits?
}

\item Why does the court consider the enforcement of the condition that the
property at issue ``be used as a home by [the grantee], his family and his
heirs'' to present a ``different question'' than the enforcement of a condition
``that the premises should be used `as a home' or `for residential purposes'
generally''? What makes these questions different?


\item Is there any relationship between the holding of \textit{Wills} and our
previously discussed rule against restraints on alienation or the principle of
\textit{numerus clausus}? If so, what's the connection?

