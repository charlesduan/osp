\reading{Smedley v. City of Waldron}

\readingcite{739 F.2d 399 (8th Cir. 1984)}

PER CURIAM.

In 1940, the City of Waldron, lacking funds to acquire a reservoir site, asked
Hannah Smedley to donate land for that purpose. The governing agreement
provided in part that:
\begin{quote}
5. The City of Waldron shall never sell, transfer, convey, lease, rent or
otherwise dispose of the lands herein above described to other persons, firms,
groups and/or corporations, except successors and/or assigns of itself, and if
it attempts to do so, the lands immediately revert to Hannah Smedley and her
heirs[.]
\end{quote}

In 1977, Harry Smedley (Hannah Smedley's sole heir and devisee) sued
unsuccessfully for reconveyance, arguing that the city had abandoned the land.
In dismissing the complaint, the district court found that the city had not
abandoned the reservoir; rather, it continuously maintained and used it as a
reserve water supply.

In 1981, the city leased the oil and gas rights of the deeded land to Texas Oil
and Gas Corporation. As a result, Harry Smedley brought this case, alleging
that the city's lease of the mineral rights subjacent to the land violated
paragraph 5 of the 1940 agreement. For relief, he demanded immediate
reconveyance of the land and payment of all monies the city received under the
lease. Both parties moved for summary judgment. The district court found that
the agreement was an impermissible restraint on alienation and granted the
city's motion for summary judgment. We reverse and remand for further
proceedings.

Some Arkansas courts have disapproved restraints on alienation. \textit{See,
e.g., First National Bank of Fort Smith v. Graham}, 195 Ark. 586, 593, 113
S.W.2d 497 (1938); \textit{Letzkus v. Nothwang}, 170 Ark. 403, 408, 279 S.W.
1006 (1926).\dots [But w]hen the grant is to a governmental unit for a
public purpose, Arkansas courts have been reluctant to void the grant as
impermissibly restraining alienation if doing so would flout the grantor's
intent. One line of Arkansas cases, for example, approved disabling language in
grants to localities where the land was to be used for school purposes.
\textit{McCrory School Dist. of Woodruff v. Brogden}, 231 Ark. 664, 333 S.W.2d
246, 249--50 (1960); \textit{Vanndale Special School Dist. No. 6 v. Feltner},
215 Ark. 252, 220 S.W.2d 131, 133 (1949); \textit{Taylor v. School Dist. No. 45
of Searcy County}, 214 Ark. 434, 216 S.W.2d 789 (1949); \textit{Coffelt v.
Decatur School Dist. No. 17}, 212 Ark. 743, 208 S.W.2d 1, 2 (1948);
\textit{Milner v. New Edinburg School Dist.}, 211 Ark. 337, 200 S.W.2d 319, 322
(1947); \textit{Williams v. Kirby School Dist.}, 207 Ark. 458, 181 S.W.2d 488,
490 (1944); \textit{Steel v. Rural Special School Dist. No. 15}, 180 Ark. 36,
20 S.W.2d 316, 317 (1929). Because summary judgment in favor of the city
ignores the public purpose of the grant and defeats the donor's intent, we
reverse the district court's judgment.

Having decided that the restraint on alienation here is not impermissible, we
remand the case to the district court to resolve the important remaining
factual questions. The district court shall determine whether the mineral lease
is a violation of the parties' agreement. Because Arkansas courts hold that if
the restraint is valid the intent of the donor controls, \textit{Gibson v.
Pickett}, 256 Ark. 1035, 512 S.W.2d 532, 535 (1974), the district court shall
determine whether the donor intended that the city would lose the land only if
the land was not used for a reservoir. Finally, the district court should
determine the best means of fulfilling the donor's intent: will her intentions
be satisfied merely by awarding her heirs the revenues from the lease, or will
the extreme remedy of forfeiture of the reservoir to the heirs be necessary?

