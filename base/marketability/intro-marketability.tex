As we saw in our discussion of estates and future interests, the common law gave
property owners a fairly diverse and subtle array of tools to effectuate their
intent regarding the use and disposition of their property. But this level of
control raises serious potential for conflicts between the plans and wishes of
the property owners of yesterday and the needs and desires of (actual and
aspiring) property owners of today.

Consider that about 80 years before the Empire State Building was constructed,
the land on which it now stands was a farm situated a mile beyond the northern
edge of the urban quarters of New York City. \textit{See} \textsc{James
Remington Mccarthy \& John Rutherford, Peacock Alley: The Romance Of The
Waldorf-Astoria} 4-10 (1931). What if the first private owner of that
farm---John Thompson, who purchased it in 1799 out of the common lands held by
the city government for \$2,400 (\textit{id.})---had executed a conveyance of
the land that included a future interest in ``the eldest of my
great-great-great-great-great grandchildren''? What if he had devised the land
to his eldest child ``on condition that the family farm may never be sold''? Or
``on condition that the land may be used for farming purposes only''? Could the
Empire State Building ever have been built? If not, is that a result we would
be happy with?

The common law recognized that some property owners might try to dictate the
disposition of property much farther into the future than could be justified by
any legitimate interest or expertise they might have. As one commentator put
it, writing in 1967: ``[I]t would have been utterly impossible for any testator
dying in 1866 to foresee the events that have taken place in the succeeding
century, and\ldots any prediction as to what may occur in the century following
1966 would be even more unlikely to conform to reality.'' \textsc{W. Barton
Leach, Property Law Indicted!} 71 (1967). As years pass, new generations
undertake stewardship of resources, and the economic, social, and cultural
demands on those resources change with the times. Allowing long-dead property
owners to dictate the disposition of those resources to the fourth, fifth, or
sixth generation after they're gone significantly limits the ability of the
possessors of today to flexibly direct resources to uses appropriate to the
age. 

The common law developed various doctrines designed to balance respect for
property owners' wishes to provide for their families as they see fit with
vigilance against the dangers of dead-hand control. One powerful tool for
striking this balance is the infamous \term{Rule Against Perpetuities}.
We will not be
studying the Rule at any length here, but its classic formulation---that an
interest in property is void unless it necessarily will vest within 21 years of
the end of a life in being at the time the interest is created---essentially
operates to limit a property owner's control to one generation beyond the end
of his own life. For example, a grant to John Thompson's
great-great-great-great-great grandchild would be clearly invalid under the
Rule Against Perpetuities, but a grant by John Thompson to his living
daughter's yet-unborn child would almost certainly be valid.\footnote{We say
``almost'' only because if Thompson for some reason made the future interest in
his unborn grandchild subject to the condition precedent of that grandchild
attaining an age of more than 21 years, the interest would be void under the
common-law Rule Against Perpetuities.}

Beyond limiting the \textit{duration} of property owners' control, the common
law developed additional rules regarding the \textit{types} of restrictions
grantors could place on otherwise valid interests in property that they
conveyed. The following cases provide some examples. As you read them, consider
how the principles they rely on relate to the aforementioned balance between
respecting property owners' wishes and guarding against dead-hand control.

