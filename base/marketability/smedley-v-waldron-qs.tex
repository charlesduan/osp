\item \textbf{Wait{\dots}what?} How can Hannah Smedley's clearly expressed
intent to absolutely forbid the City of Waldron from alienating the reservoir
get around the common-law rule against restraints on alienation? And why can't
Chester Ford's far more modest but no less clearly expressed intent to restrain
Lola Mae's right to alienate do the same thing?


\item \textbf{Restraints on Alienation vs. Restrictions on Use.} In
\textit{Wills} the court seemed to be concerned that the condition subsequent
restricting the grantee's use of the land conveyed was a sort of restraint on
alienation in disguise. Could a naked restraint on alienation---such as the one
in \textit{Smedley}{}---really be a restriction on \textit{use }in disguise? If
so, would it be any less offensive to the principles underlying the rule
against restraints on alienation?


\item Does \textit{Smedley} reach the opposite result from \textit{Ford} and
\textit{Wills} because the grantor's \textit{motivation} is different in
\textit{Smedley} than in the other cases? (Is it?) Because the grantee is a
public entity rather than a private individual? Because the restraint on the
grantee is less onerous? (Is it?)

