Concurrent owners can generally contract among themselves to allocate the
various benefits and burdens of ownership as they see fit. But in the absence
of such agreement, there are several default rules regarding the rights and
obligations that arise between cotenants of property.

This system of default rules begins with the premise that each cotenant is
entitled to all the rights of ownership in the entire co-owned parcel. Thus,
for example, cotenants do not necessarily have the right to compromise other
cotenants' right to exclude.  If one cotenant objects to a police search and
the other would allow it, the objecting cotenant prevails.  A warrantless
search is not allowed unless an exception to the warrant requirement applies.
\emph{Georgia v. Randolph}, 547 U.S. 103 (2006).

The implications of multiple equal and undivided interests in a co-owned parcel
become far more complicated with respect to other rights of
ownership---particularly the rights of possession and use. If all co-owners are
equally entitled to possession and use of the whole parcel, what happens when
more than one cotenant decides to assert those rights at the same time? Is it
physically possible to put co-equal rights of all concurrent owners into
practice? And if not, what if any obligation does a cotenant in possession owe
to cotenants out of possession? Consider the following case:

\expectnext{martin-v-martin}
