\reading{United States v. Craft}

\readingcite{535 U.S. 274 (2002)}

Justice O'CONNOR delivered the opinion of the Court.

\dots English common law provided three legal structures for the concurrent
ownership of property that have survived into modern times: tenancy in common,
joint tenancy, and tenancy by the entirety. The tenancy in common is now the
most common form of concurrent ownership. The common law characterized tenants
in common as each owning a separate fractional share in undivided property.
Tenants in common may each unilaterally alienate their shares through sale or
gift or place encumbrances upon these shares. They also have the power to pass
these shares to their heirs upon death. Tenants in common have many other
rights in the property, including the right to use the property, to exclude
third parties from it, and to receive a portion of any income produced from it.


Joint tenancies were the predominant form of concurrent ownership at common law,
and still persist in some States today. The common law characterized each joint
tenant as possessing the entire estate, rather than a fractional share:
``[J]oint-tenants have one and the same interest\dots held by one and the
same undivided possession.'' Joint tenants possess many of the rights enjoyed
by tenants in common: the right to use, to exclude, and to enjoy a share of the
property's income. The main difference between a joint tenancy and a tenancy in
common is that a joint tenant also has a right of automatic inheritance known
as ``survivorship.'' Upon the death of one joint tenant, that tenant's share in
the property does not pass through will or the rules of intestate succession;
rather, the remaining tenant or tenants automatically inherit it. Joint
tenants' right to alienate their individual shares is also somewhat different.
In order for one tenant to alienate his or her individual interest in the
tenancy, the estate must first be severed---that is, converted to a tenancy in
common with each tenant possessing an equal fractional share. Most States
allowing joint tenancies facilitate alienation, however, by allowing severance
to automatically accompany a conveyance of that interest or any other overt act
indicating an intent to sever.

A tenancy by the entirety is a unique sort of concurrent ownership that can only
exist between married persons. Because of the common-law fiction that the
husband and wife were one person at law (that person, practically speaking, was
the husband), Blackstone did not characterize the tenancy by the entirety as a
form of concurrent ownership at all. Instead, he thought that entireties
property was a form of single ownership by the marital unity. Neither spouse
was considered to own any individual interest in the estate; rather, it
belonged to the couple.

Like joint tenants, tenants by the entirety enjoy the right of survivorship.
Also like a joint tenancy, unilateral alienation of a spouse's interest in
entireties property is typically not possible without severance. Unlike joint
tenancies, however, tenancies by the entirety cannot easily be severed
unilaterally. Typically, severance requires the consent of both spouses, or the
ending of the marriage in divorce. At common law, all of the other rights
associated with the entireties property belonged to the husband: as the head of
the household, he could control the use of the property and the exclusion of
others from it and enjoy all of the income produced from it. The husband's
control of the property was so extensive that, despite the rules on alienation,
the common law eventually provided that he could unilaterally alienate
entireties property without severance subject only to the wife's survivorship
interest.

With the passage of the Married Women's Property Acts in the late 19th century
granting women distinct rights with respect to marital property, most States
either abolished the tenancy by the entirety or altered it significantly.
Michigan's version of the estate is typical of the modern tenancy by the
entirety. Following Blackstone, Michigan characterizes its tenancy by the
entirety as creating no individual rights whatsoever: ``It is well settled
under the law of this State that one tenant by the entirety has no interest
separable from that of the other \dots. Each is vested with an entire
title.'' And yet, in Michigan, each tenant by the entirety possesses the right
of survivorship. Each spouse---the wife as well as the husband---may also use
the property, exclude third parties from it, and receive an equal share of the
income produced by it. Neither spouse may unilaterally alienate or encumber the
property, although this may be accomplished with mutual consent. Divorce ends
the tenancy by the entirety, generally giving each spouse an equal interest in
the property as a tenant in common, unless the divorce decree specifies
otherwise.\dots

