\reading{Martin v. Martin}

\readingcite{878 S.W. 2d 30 (Ky. Ct. App. 1994)}

\dots Garis and Peggy own an undivided one-eighth interest in a tract of land
in Pike County. This interest was conveyed to Garis by his father, Charles
Martin, in 1971. Appellees, Charles and Mary Martin, own a life estate in the
undivided seven-eighths of the property for their joint lives, with remainder
to appellants.

In 1982, Charles Martin improved a portion of the property and developed a four
lot mobile home park which he and Mary rented. In July of 1990, Garis and Peggy
moved their mobile home onto one of the lots. It is undisputed that Garis and
Peggy expended no funds for the improvement or maintenance of the mobile home
park, nor did they pay rent for the lot that they occupied.

In 1990, Garis and Peggy filed an action which sought an accounting of their
claimed one-eighth portion of the net rent received by Charles and Mary from
the lots. The accounting was granted, however, the judgment of the trial court
required appellants to pay ``reasonable rent'' for their occupied lot. It is
that portion of the judgment from which this appeal arises.

The sole issue presented is whether one cotenant is required to pay rent to
another cotenant. Appellants argue that absent an agreement between cotenants,
one cotenant occupying premises is not liable to pay rent to a co-owner.
Appellees respond that a cotenant is obligated to pay rent when that cotenant
occupies the jointly owned property to the exclusion of his co-owner.

Appellants and appellees own the subject property as tenants in common. The
primary characteristic of a tenancy in common is unity of possession by two or
more owners. Each cotenant, regardless of the size of his fractional share of
the property, has a right to possess the whole.

The prevailing view is that an occupying cotenant must account for outside
rental income received for use of the land, offset by credits for maintenance
and other appropriate expenses. The trial judge correctly ordered an accounting
and recovery of rent in the case sub judice.

However, the majority rule on the issue of whether one cotenant owes rent to
another is that a cotenant is not liable to pay rent, or to account to other
cotenants respecting the reasonable value of the occupancy, absent an ouster or
agreement to pay.

The trial court relied erroneously on \emph{Smither v. Betts, Ky.}, 264 S.W.2d
255 (1954), for its conclusion that appellants were ``obligated to pay
seven-eighths of the reasonable rental for the use of the lot they occupy.'' In
\textit{Smither}, one cotenant had exclusive possession of jointly owned
property by virtue of a lease with a court-appointed receiver and there was an
agreement to pay rent. That clearly is not the case before us. There was no
lease or any other agreement between the parties.

The appellees reason that the award of rent was proper upon the premise that
Garis and [Peggy] ousted their cotenants. While the proposition that a cotenant
who has been ousted or excluded from property held jointly is entitled to rent
is a valid one, we are convinced that such ouster must amount to exclusive
possession of the entire jointly held property. We find support for this
holding in \emph{Taylor}, supra, in which the Court stated at 807-08:
\begin{quote}
But, however this may be, running throughout all the books will be found two
essential elements which must exist before the tenant sought to be charged is
liable. These are: (a) That the tenant sought to be charged and who is claimed
to be guilty of an ouster must assert exclusive claim to the property in
himself, thereby necessarily including a denial of any interest or any right or
title in the supposed ousted tenant; (b) he must give notice to this effect to
the ousted tenant, or his acts must be so open and notorious, positive and
assertive, as to place it beyond doubt that he is claiming the entire interest
in the property.
\end{quote}
We conclude that appellants' occupancy of one of the four lots did not amount to
an ouster. To hold otherwise is to repudiate the basic characteristic of a
tenancy in common that each cotenant shares a single right to possession of the
entire property and each has a separate claim to a fractional share.

Accordingly, the judgment of the Pike Circuit Court is reversed as to the award
of rent to the appellees.

