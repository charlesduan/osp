Property issues arise again and again in family law practice. Property rights
within marriage are typically not that significant when the spouses function as
an economic unit. But the details matter (a) when there are creditors (i.e. the
\textit{Cross}/\textit{Sawada} situation, considered above), (b) on divorce,
and (c) at death. The law of wills and estates deals with the last situation.
Here, we will focus on divorce in states that have not adopted a community
property regime.

\captionedgraphic{concurrent-2}{Source: John C. Bullas, CC-NC-ND, Nov. 6, 2009,
\protect\url{https://www.flickr.com/photos/johnbullas/4081360430/}.}

Modern American divorce law is no-fault: one or both spouses can get a divorce
simply because they want one, without having to show any misbehavior by the
other spouse. But should fault matter in the distribution of property at
divorce? If so, what kind of fault should matter---infidelity? Physical
abuse? Neglect and indifference? Gambling away most of the family's money?
Does it matter if the wealthier spouse was the one at fault?

\paragraph{Alimony} In the past, courts in common law states divided property
based on who held title, which often favored men over women. However, courts
would award alimony on divorce when it was deemed necessary to support an
ex-spouse. Alimony required periodic ongoing payments by one ex-spouse to
another. Only a fraction of women ever received alimony. See \textsc{Lenore J.
Weitzman, The Divorce Revolution} 144 (1985). It is even less likely to be
awarded now. The Uniform Marriage and Divorce Act, \S~308(a), allows alimony
only if the spouse seeking it lacks sufficient property to provide for his
reasonable needs and is also unable to support himself through employment, or
is the custodian of a child ``whose conditions or circumstances make it
appropriate that the custodian not be required to seek employment outside the
home.'' Under 20\% of women receive alimony in modern divorce cases, and only
for a short period, after which the recipient is supposed to find a job and
become self-sufficient. See Mary E. O'Connell, \textit{Alimony After No-Fault:
A Practice in Search of a Theory}, 23 \textsc{New Eng. L. Rev.} 437 (1988).

The New Jersey Supreme Court explained the pressures that led to a change in the
old common law rules in \emph{Rothman v. Rothman}, 320 A.2d 496 (1974):
\begin{quote}
Hitherto future financial support for a divorced wife has been available only by
grant of alimony. Such support has always been inherently precarious. It ceases
upon the death of the former husband and will cease or falter upon his
experiencing financial misfortune disabling him from continuing his regular
payments. This may result in serious misfortune to the wife and in some cases
will compel her to become a public charge. An allocation of property to the
wife at the time of the divorce is at least some protection against such an
eventuality. [The new regime] seeks to right what many have felt to be a grave
wrong. It gives recognition to the essential supportive role played by the wife
in the home, acknowledging that as homemaker, wife and mother she should
clearly be entitled to a share of family assets accumulated during the
marriage. Thus the division of property upon divorce is responsive to the
concept that marriage is a shared enterprise, a joint undertaking, that in many
ways it is akin to a partnership. Only if it is understood that far more than
economic factors are involved, will the resulting distribution be equitable
within the true intent and meaning of the statute.
\end{quote}
The \textit{Rothman} court refused to presume that an even split of marital
property was the appropriate starting point, however. Instead, courts are
supposed to conduct an equitable analysis in each case.

The common law states now use a model very similar to community property in
treating most property acquired during marriage as marital property to be
divided between the spouses at divorce. Such ``equitable distribution'' is an
attempt to be fair to both parties, recognizing that family relations are
complicated and that the fact that one partner earned most of the money during
a marriage doesn't necessarily mean that he should take most of the assets out.
Nor does equitable distribution require an equal split. James R. Ratner,
\emph{Distribution of Marital Assets in Community Property Jurisdictions:
Equitable Doesn't Equal Equal}, 72 \textsc{La. L. Rev.} 21 (2011) (participation
in asset generation and perceived need are primary factors that cause courts to
depart from strictly equal division).

\paragraph{The family home} The most significant asset in most divorce cases is
a family house. If the house is to be sold, then a fair means of dividing the
proceeds must be found. The departing spouse's interest in the house must be
addressed in some way, for example by giving her more of the other marital
assets. \textit{See generally} Melissa J. Avery, \emph{The Marital Residence and
Other Black Holes: Dealing With Real Estate in Divorce}, 53 \textsc{Res Gestae}
30 (Oct. 2009) (discussing partition sales, buy-outs, and auction sales); David
W. Griffin, \emph{It's Nearly Always About the House: Grasping the Givens of
Real Property Interests, Considerations, and Concerns}, 32 \textsc{Fam. Advoc.}
8 (Spring 2010) (addressing titling issues, forms of ownership, and ways to
secure future payments of equitable distribution amounts); Marcia Soto, \emph{A
House Divided}, 31
\textsc{Fam. Advoc.} 10 (Summer 2008) (considering, among other issues, the tax
consequences of selling a home as a single person versus as a married couple).

The facts in individual cases may be complicated, and the issues around the
family home are often especially emotional. As a general rule, however, a
custodial parent generally retains the family home. \emph{In re Marriage of
King}, 700 P.2d 591 (Mont. 1985) (awarding family home to wife over husband's
objection when remaining in the home was in the best interests of the minor
children, and the husband's income from professional gambling wasn't steady
enough to ensure regular child support payments). \textit{But see} \emph{Ramsey
v. Ramsey}, 546 N.E.2d 1280 (Ind. Ct. App. 1989) (upholding the trial judge's
order that the family house should be sold over the objections of both parties,
even though the couple had been married for twenty years and wanted the wife to
stay in the home with their five children, allowing the father daily visits);
\emph{Stolow v. Stolow}, 540 N.Y.S2d 484 (App. Div. 1989) (ordering the sale of
the family's ``mini-mansion'' because of its extravagance, allowing the husband
to get his share of its value, even though the husband was wealthy enough to
afford the payments and even though exclusive possession of a marital residence
is generally awarded to a custodial spouse with minor children); \emph{Behrens
v. Behrens}, 532 N.Y.S.2d 893 (App. Div. 1988) (ordering the family house sold
because neither party had sufficient individual resources to afford the
mortgage, despite the wife's objection that the sale would force her and her
children to leave the community where they'd established strong ties); \emph{In
re Marriage of Stallworth}, 237 Cal. Rptr. 829 (Ct. App. 1987) (economic,
emotional, and social impact on a child from being forced to move out of the
family home would be minimal, even though the child was under psychiatric care
and in a special education program; harm to the child was outweighed by the
husband's economic interest in having the home sold). As you should be able to
see from a review of these summaries, divorce provides an opportunity for
courts to opine on the moral merits or demerits of the spouses, and many courts
take it, despite the formal abolition of fault-based divorce.

\paragraph{Religious and cultural issues} Are there instances in which cultural
differences should determine distribution of property at divorce? Some
cultures systematically disadvantage women on divorce; one concern for allowing
religious control of divorce is for the resultant inequality in property
division. Secular courts may attempt to use property law to induce better
behavior by spouses. Most notably, Orthodox Jews hold that a woman is not
divorced unless her husband gives her a document known as a ``get.'' Divorcing
husbands may withhold a get out of spite or in order to induce the wife to
agree to a favorable property distribution. \textit{See} Jessica Davidson
Miller, \emph{The History of the Agunah in America: A Clash of Religious Law and
Social Progress}, 19 \textsc{Women's Rights L. Rptr.} 1 (1997).

The New York legislature responded by changing the law. To get a divorce, each
party must file an affidavit stating ``(i) that he or she has, to the best of
his or her knowledge, taken all steps solely within his or her power to remove
all barriers to the other party's remarriage following the annulment or divorce;
or (ii) that the other party has waived in writing the requirements of this
subdivision.'' \textsc{N.Y. Dom. Rel. L.} \S~253. Further, divorce courts must
take a refusal to ``remove all barriers to the other party's remarriage'' into
account in dividing marital property. \textsc{N.Y. Dom. Rel. L.} \S~256. Some
recalcitrant husbands have received zero in marital property as a result, and
have even been held in contempt for withholding a get. \emph{Fischer v.
Fischer}, 237 A.2d 559 (N.Y. App. Div. 1997). Conservative\footnote{One major
branch of Judaism. Other branches include Reconstruction, Reform, and Orthodox;
in general, Reform and Reconstruction Jewish women would not consider themselves
bound to get a get before remarriage. The standard Orthodox ketubah says
nothing about the husband's obligation to provide a get in case of divorce.}
Jews now use a ketubah (a religious marriage contract) providing that a man who
refuses to provide a get must appear before a Bet Din, a Jewish court, which
will strongly encourage him to give his wife the get. At least one civil court
has enforced the ketubah, rejecting First Amendment arguments against
enforcement. \emph{Avitzur v. Avitzur}, 446 N.E.2d 136 (N.Y. 1983). What should
happen if the husband claims that he has converted to Catholicism, and that to
give his wife a get would violate his religious principles?

In \emph{Estate of Bir}, 83 Cal. App. 2d 256 (1948), the decedent and his two
wives were married in Punjab. He died in California. The trial court refused the
wives' petition for an equal division of the property on public policy grounds,
but the appellate court reversed, since he hadn't attempted to cohabit with
them in California and they were the only interested parties. If this case
occurred today, when nonmarital and polyamorous cohabitation is not illegal,
should the court treat both women as widows, or does the continuing prohibition
on bigamy matter?

Are you aware of other distinctive cultural traditions around marriage that
should be provided for in law? \textit{Cf.} Vickie Enis, Comment, \emph{Yours,
Mine, Ours? Renovating the Antiquated Apartheid in the Law of Property Division
in Native American Divorce}, 35 \textsc{Am. Indian L. Rev.} 661 (2011)
(discussing special considerations when Native Americans who own property
distributed to tribal members divorce non-Natives).

\paragraph{General principles for dividing property} Most states leave the
determination of what is equitable or just division of marital property to the
family court's discretion, but there is usually a statutory list of relevant
factors for the court to consider, as well as some attempt to define what
marital property is. Here are portions of Missouri's code:
\begin{quotation}
1. In a proceeding for dissolution of the marriage or
legal separation, or in a proceeding for disposition of property following
dissolution of the marriage by a court which lacked personal jurisdiction over
the absent spouse or lacked jurisdiction to dispose of the property, the court
shall set apart to each spouse such spouse's nonmarital property and shall
divide the marital property and marital debts in such proportions as the court
deems just after considering all relevant factors including:
\begin{statute}
\item (1) The economic circumstances of each spouse at the time the division of
property is to become effective, including the desirability of awarding the
family home or the right to live therein for reasonable periods to the spouse
having custody of any children;

\item (2) The contribution of each spouse to the acquisition of the marital property,
including the contribution of a spouse as homemaker;

\item (3) The value of the nonmarital property set apart to each spouse;

\item (4) The conduct of the parties during the marriage; and

\item (5) Custodial arrangements for minor children.
\end{statute}

2. For purposes of sections 452.300 to 452.415 only, ``marital
property'' means all property acquired by either spouse subsequent
to the marriage except:
\begin{statute}
\item (1) Property acquired by gift, bequest, devise, or descent;

\item (2) Property acquired in exchange for property acquired prior to the marriage or
in exchange for property acquired by gift, bequest, devise, or descent;

\item (3) Property acquired by a spouse after a decree of legal separation;

\item (4) Property excluded by valid written agreement of the parties; and

\item (5) The increase in value of property acquired prior to the marriage or
pursuant to subdivisions (1) to (4) of this subsection, unless marital assets
including labor, have contributed to such increases and then only to the extent
of such contributions.
\end{statute}

3. All property acquired by either spouse subsequent to the marriage and prior
to a decree of legal separation or dissolution of marriage is presumed to be
marital property regardless of whether title is held individually or by the
spouses in some form of co-ownership such as joint tenancy, tenancy in common,
tenancy by the entirety, and community property. The presumption of marital
property is overcome by a showing that the property was acquired by a method
listed in subsection 2 of this section.

4. Property which would otherwise be nonmarital property shall not become
marital property solely because it may have become commingled with marital
property.
\end{quotation}
Mo. Stat. Ann. 452.330.1--.4.
Under this regime, can courts take one spouse's infidelity or abuse into account
in dividing property? Other states use a rebuttable presumption that an equal
division is equitable, especially for a marriage of long duration.

\paragraph{Special kinds of property} How should courts deal with property whose
greatest value is sentimental? In \textit{M.R. v. E.R.}, the parties divorced
after more than 20 years of marriage. They agreed on how to divide the marital
home and their retirement accounts, and they agreed on child custody, but
couldn't agree on more than 7000 photos. At one point, they agreed that the
husband would keep the albums and split the cost of scanning them for the wife,
but the quality of the reproductions became a sticking point. Accepting the
husband's argument that he was the one responsible for creating a meticulous
photo catalog and that the wife was generally apathetic about the photographic
process during the marriage, the judge awarded him 75\% of the photos. How
should the photos the wife receives be selected? See \emph{M.R. v. E.R.}, 27
Misc.3d 1206 (N.Y. Sup. Ct. 2010).

What marital assets count as property?

In \emph{O'Brien v O'Brien}, 489 N.E.2d 712 (N.Y. 1985), the parties' ``only
asset of
any consequence'' was the husband's newly acquired license to practice
medicine. The wife had supported him throughout his medical education,
contributing 76\% of their income and giving up her own opportunity to obtain
certification as a teacher, only to have him file for divorce just after he
obtained his medical license. The New York Court of appeals held that his
medical license was ``marital property'' subject to equitable distribution.
Expert testimony was used to establish the present value of a medical degree.
The majority commented:
\begin{quote}
As this case demonstrates, few undertakings during a marriage better qualify as
the type of joint effort that the statute's economic partnership theory is
intended to address than contributions toward one spouse's acquisition of a
professional license. Working spouses are often required to contribute
substantial income as wage earners, sacrifice their own educational or career
goals and opportunities for child rearing, perform the bulk of household duties
and responsibilities and forego the acquisition of marital assets that could
have been accumulated if the professional spouse had been employed rather than
occupied with the study and training necessary to acquire a professional
license. In this case, nearly all of the parties' nine-year marriage was
devoted to the acquisition of plaintiff's medical license and defendant played
a major role in that project. She worked continuously during the marriage and
contributed all of her earnings to their joint effort, she sacrificed her own
educational and career opportunities, and she traveled with plaintiff to Mexico
for three and one-half years while he attended medical school there. The
Legislature has decided, by its explicit reference in the statute to the
contributions of one spouse to the other's profession or career, that these
contributions represent investments in the economic partnership of the marriage
and that the product of the parties' joint efforts, the professional license,
should be considered marital property.
\end{quote}
The majority rejected the argument that ``a professional license is not marital
property because it does not fit within the traditional view of property as
something which has an exchange value on the open market and is capable of
sale, assignment or transfer.'' Statutes can create new species of property,
and ``[a]professional license is a valuable property right, reflected in the
money, effort and lost opportunity for employment expended in its acquisition,
and also in the enhanced earning capacity it affords its holder, which may not
be revoked without due process of law. That a professional license has no
market value is irrelevant.'' The limits on its alienation merely meant that
the other spouse was entitled to an award sharing in its value.

A concurrence cautioned against ``the potential for unfairness involved in
distributive awards based upon a license of a professional still in training,''
arguing that a professional in training could be ``locked into a particular
kind of practice simply because the monetary obligations imposed by the
distributive award made on the basis of the trial judge's conclusion (prophecy
may be a better word) as to what the career choice will be leaves him or her no
alternative.''

