\reading{Posik v. Layton}

\readingcite{695 So. 2d 759 (Fla. Dist. Ct. App. 1997).}

Nancy Layton was a doctor practicing at the Halifax Hospital in Volusia County
and Emma Posik was a nurse working at the same facility when Dr. Layton decided
to remove her practice to Brevard County. In order to induce Ms. Posik to give
up her job and sell her home in Volusia County, to accompany her to Brevard
County, and to reside with her ``for the remainder of Emma Posik's life to
maintain and care for the home,'' Dr. Layton agreed that she would provide
essentially all of the support for the two, would make a will leaving her
entire estate to Ms. Posik, and would ``maintain bank accounts and other
investments which constitute non-probatable assets in Emma Posik's name to the
extent of 100\% of her entire non-probatable assets.'' Also, as part of the
agreement, Ms. Posik agreed to loan Dr. Layton \$20,000 which was evidenced by
a note. The agreement provided that Ms. Posik could cease residing with Dr.
Layton if Layton failed to provide adequate support, if she requested in
writing that Ms. Posik leave for any reason, if she brought a third person into
the home for a period greater than four weeks without Ms. Posik's consent, or
if her abuse, harassment or abnormal behavior made Ms. Posik's continued
residence intolerable. In any such event, Dr. Layton agreed to pay as
liquidated damages the sum of \$2,500 per month for the remainder of Ms.
Posik's life.

It is apparent that Ms. Posik required this agreement as a condition of
accompanying Dr. Layton to Brevard. The agreement was drawn by a lawyer and
properly witnessed. Ms. Posik, fifty-five years old at the time of the
agreement, testified that she required the agreement because she feared that
Dr. Layton might become interested in a younger companion. Her fears were well
founded. Some four years after the parties moved to Brevard County and without
Ms. Posik's consent, Dr. Layton announced that she wished to move another woman
into the house. When Ms. Posik expressed strong displeasure with this idea, Dr.
Layton moved out and took up residence with the other woman.

Dr. Layton served a three-day eviction notice on Ms. Posik. Ms. Posik later
moved from the home and sued to enforce the terms of the agreement and to
collect on the note evidencing the loan made in conjunction with the
agreement.\dots Dr. Layton counterclaimed for a declaratory judgment as to
whether the liquidated damages portion of the agreement was enforceable.

The trial judge found that because Ms. Posik's economic losses were reasonably
ascertainable as to her employment and relocation costs, the \$2,500 a month
payment upon breach amounted to a penalty and was therefore unenforceable. The
court further found that although Dr. Layton had materially breached the
contract within a year or so of its creation, Ms. Posik waived the breach by
acquiescence. Finally, the court found that Ms. Posik breached the agreement by
refusing to continue to perform the house work, yard work and cooking for the
parties and by her hostile attitude which required Dr. Layton to move from the
house. Although the trial court determined that Ms. Posik was entitled to
quantum meruit, it also determined that those damages were off-set by the
benefits Ms. Posik received by being permitted to live with Dr. Layton. The
court did award Ms. Posik a judgment on the note executed by Dr. Layton.

Although neither party urged that this agreement was void as against public
policy, Dr. Layton's counsel on more than one occasion reminded us that the
parties had a sexual relationship. Certainly, even though the agreement was
couched in terms of a personal services contract, it was intended to be much
more. It was a nuptial agreement entered into by two parties that the state
prohibits from marrying. But even though the state has prohibited same-sex
marriages and same-sex adoptions, it has not prohibited this type of
agreement\dots. But the State has not denied these individuals their right to
either will their property as they see fit nor to privately commit by contract
to spend their money as they choose. The State is not thusly condoning the
lifestyles of homosexuals or unmarried live-ins; it is merely
recognizing their constitutional private property and contract rights.

Even though no legal rights or obligations flow as a matter of law from a
non-marital relationship, we see no impediment to the parties to such a
relationship agreeing between themselves to provide certain rights and
obligations. Other states have approved such individual agreements. In
\emph{Marvin v. Marvin}, 557 P.2d 106 (1976), the California Supreme Court held:
\begin{quote}
[W]e base our opinion on the principle that adults who voluntarily live together
and engage in sexual relations are nonetheless as competent as any other
persons to contract respecting their earnings and property rights. So long as
the agreement does not rest upon illicit meretricious consideration, the
parties may order their economic affairs as they choose.
\end{quote}
\dots In a case involving unmarried heterosexuals, a Florida appellate court
has passed on the legality of a non-marital support agreement. In \emph{Crossen
v. Feldman}, 673 So.2d 903 (Fla. 2d DCA 1996), the court held:
\begin{quote}
Without attempting to define what may or may not be ``palimony,''\edfootnote{
``Palimony'' is a lay term with no real legal meaning. It is a
portmanteau of ``pal'' and ``alimony,'' representing the concept that unmarried
partners may have interests sufficient to treat their breakup as similar to a
divorce in terms of one partner's right to some form of compensation from the
other, although not necessarily in the form of periodic alimony payments. Is
enforcement of a contract, either express or implied, really similar to
alimony, a mandated division of property that occurs because of the legal
relationship of the parties even if they never agreed to it?} this case simply
involves whether these parties entered into a contract for support, which is
something that they are legally capable of doing.
\end{quote}

Addressing the invited issue, we find that an agreement for support between
unmarried adults is valid unless the agreement is inseparably based upon
illicit consideration of sexual services. Certainly prostitution, heterosexual
or homosexual, cannot be condoned merely because it is performed within the
confines of a written agreement. The parties, represented by counsel, were well
aware of this prohibition and took pains to assure that sexual services were
not even mentioned in the agreement. That factor would not be decisive,
however, if it could be determined from the contract or from the conduct of the
parties that the primary reason for the agreement was to deliver and be paid
for sexual services. This contract and the parties' testimony show that such
was not the case here. Because of the potential abuse in marital-type
relationships, we find that such agreements must be in writing.\dots

The obligations imposed on Ms. Posik by the agreement include the obligation
``to immediately commence residing with Nancy L.R. Layton at her said residence
for the remainder of Emma Posik's life.'' This is very similar to a
``until death do us part'' commitment. And although the parties undoubtedly
expected a sexual relationship, this record shows that they contemplated much
more. They contracted for a permanent sharing of, and participating in, one
another's lives. We find the contract enforceable.

We disagree with the trial court that waiver was proved in this case. Ms. Posik
consistently urged Dr. Layton to make the will as required by the agreement and
her failure to do so was sufficient grounds to declare default. And even more
important to Ms. Posik was the implied agreement that her lifetime commitment
would be reciprocated by a lifetime commitment by Dr. Layton-and that this
mutual commitment would be monogamous. When Dr. Layton introduced a third
person into the relationship, although it was not an express breach of the
written agreement, it explains why Ms. Posik took that opportunity to hold Dr.
Layton to her express obligations and to consider the agreement in default.

We also disagree with the trial court that Ms. Posik breached the agreement by
refusing to perform housework, yard work, provisioning the house, and cooking
for the parties. This conduct did not occur until after Dr. Layton had first
breached the agreement. One need not continue to perform a contract when the
other party has first breached. Therefore, this conduct did not authorize Dr.
Layton to send the three-day notice of eviction which constituted a separate
default under the agreement.

We also disagree that the commitment to pay \$2,500 per month upon termination
of the agreement is unenforceable as a penalty. We agree with Ms. Posik that
her damages, which would include more than mere lost wages and moving expenses,
were not readily ascertainable at the time the contract was created. Further,
the agreed sum is reasonable under the circumstances of this case. It is less
than Ms. Posik was earning some four years earlier when she entered into this
arrangement. It is also less than Ms. Posik would have received had the
long-term provisions of the contract been performed. She is now in her sixties
and her working opportunities are greatly reduced.

We recognize that this contract, insisted on by Ms. Posik before she would
relocate with Dr. Layton, is extremely favorable to her. But there is no
allegation of fraud or overreaching on Ms. Posik's part. This court faced an
extremely generous agreement in \emph{Carnell v. Carnell}, 398 So.2d 503 (Fla.
5th DCA 1981). In \emph{Carnell}, a lawyer, in order to induce a woman to become
his wife, agreed that upon divorce the wife would receive his home owned by him
prior to marriage, one-half of his disposable income and one-half of his
retirement as alimony until she remarried. Two years after the marriage, she
tested his commitment. We held:
\begin{quote}
The husband also contends that the agreement is so unfair and unreasonable that
it must be set aside. ``The freedom to contract includes the right to make a
bad bargain.'' The controlling question here is whether there was
overreaching and not whether the bargain was good or bad.
\end{quote}

Contracts can be dangerous to one's well-being. That is why they are kept away
from children. Perhaps warning labels should be attached. In any event,
contracts should be taken seriously. Dr. Layton's comment that she considered
the agreement a sham and never intended to be bound by it shows that she did
not take it seriously. That is regrettable.

We affirm that portion of the judgment below which addresses the promissory note
and attorney's fees and costs associated therewith. We reverse that portion of
the judgment that fails to enforce the parties' agreement.

\textsc{Affirmed} in part; \textsc{reversed} in part and \textsc{remanded} for
further action
consistent with this opinion.

\opinion \textsc{Peterson}, C.J., concurs specially, with opinion.

\dots Each and every term of this agreement could have been included in one
between a single invalid or an elderly married couple who seek the
companionship and household services of a housekeeper, cook or a practical or
professional nurse, in which no sexual relationship was involved\dots.

