\expected{delfino-v-vealencis}

\item \textbf{Owelty.}\footnote{This charming term is followed in
\textit{Black's Legal Dictionary} by another winner. To quote Blackstone,
``Owling, so called from its being usually carried on in the night,\dots is
the offense of transporting sheep or wool out of this kingdom.''} Courts have
the equitable power to order \term{owelty} payments when it is impractical to
partition in kind according to exact shares, but when monetary payments can
adjust for the variance in the value of the parcels from the interests held by
the respective cotenants. \textit{See} \emph{Dewrell v. Lawrence}, 58 P.3d 223,
227
(Okla. Civ. App. 2002); \textsc{Code of Ala.} \S~35-6-24 (2010); \textsc{Cal.
Civ. Proc. Code} \S~873.250 (West 2009).


\item \textbf{Denouement}. Thomas Merrill and Henry Smith did some digging for
their property casebook, \textit{Property: Principles and Policies}.
Apparently, Vealencis was a difficult client and antagonized the trial judge,
which meant that her victory on the law did not translate to victory in the
real world. In \textit{Delfino}, Vealencis was awarded three lots, including
her homestead, a total of about one acre worth \$72,000. (See lot 135-1 on far
left of image.) She was also required to pay \$26,000 in owelty to the
Delfinos to compensate them for the harm her garbage operation imposed on the
proposed subdivision.


While Vealencis had a 5/16 interest in the land, her net benefit was only
\$46,000, or less than one-fourth of what she was due. Three years later, the
Delfinos sold their roughly 19 acres to a developer for \$725,000. The
developer separated Vealencis' lot from the rest of the subdivision by a
two-foot-wide strip of land (see lots 39 and 40). This deprived her of access
to Dino Road and its sewer and water connections, as well as preventing her
trucks from entering the subdivision (even though she'd already paid for
diminishing the value of the homes in the subdivision). Vealencis' only access
to the land was a 16.5 foot easement over lot 9C. She was required to use an
artesian well and a septic tank. \textit{See} Manel Baucells \& Steven A.
Lippman, \textit{Justice Delayed Is Justice Denied: A Cooperative Game
Theoretic Analysis to Hold-up in Coownership}, 22 \textsc{Cardozo L. Rev.} 1191
(2001). Vealencis died in 1990, still running the garbage business.



Why was she required to pay owelty up-front rather than waiting to see if the
harm materialized and allowing the Delfinos to recover in an action for
nuisance later? Is there anything the court could have done in its division to
avoid the unfairness to Vealencis? And what does this result suggest about the
appropriate choice of remedies---injunction or damages---in nuisance cases?



\item \textbf{Implementing partition in kind}. In a partition in kind, how
should the court determine who gets what land? \textit{See} \emph{Anderson v.
Anderson}, 560 N.W.2d 729 (Minn. Ct. App. 1997) (cotenants awarded parcel on
which they had a residence); \emph{Barth v. Barth}, 901 P.2d 232 (Okla. Ct. App.
1995) (considering cotenant's ownership of adjacent land). In Louisiana,
partition in kind is not allowed unless parcels of equal value can be created,
and parcels must be drawn by lot. \emph{See} \emph{McNeal v. McNeal}, 732 So. 2d
663 (La. Ct. App. 1999). Is this a good idea? What about ``I cut, you choose''
as a way of implementing partition in kind? There's a large literature in game
theory, mathematics, and computer science on these problems, dealing with more
than two parties, heterogenous resources, etc. Very little of this seems to have
made its way into law. \textit{But see} Note, \textit{Cutting the Baby in Half},
77 \textsc{Brook. L. Rev.} 263 (2011) (surveying some of the literature).


Some state laws also provide for allotment, in which the court allocates part of
the property to a cotenant---which can include an owelty payment if the
allocated portion is more than the cotenant's share---and then sells the
remainder. \emph{E.g.}, 25 \textsc{Del. Code} \S~730; \textsc{S.C. Code Ann.}
\S~15-61-50; \textsc{Va.
Code Ann.} \S~8.01-83. Sometimes a cotenant must show an equitable claim to
allotment in order to get it. \textsc{Haw. Rev. Stat.} \S\S~668-7(5)-(6).



\item \textbf{Partition by sale as the default?} Consider the court's
claims about the preference for partition in kind. Partition in kind will
essentially always diminish the market value of the land compared to partition
by sale. Do other, intangible interests nonetheless adequately justify a
preference for partition in kind?


\textit{Ark Land Co. v. Harper}, 599 S.E.2d 754 (W. Va. 2004), suggests that a
rule favoring maximization of market value ``would permit commercial entities
to always `evict' pre-existing co-owners, because a commercial entity's
interest in property will invariably increase its value.''

\defbook{uphpa}{
name=Uniform Partition of Heirs Property Act,
date=2010,
publisher=National Conference of Commissioners on Uniform State Laws,
url={https://www.uniformlaws.org/viewdocument/final-act-97?CommunityKey=50724584-e808-4255-bc5d-8ea4e588371d&tab=librarydocuments},
}

Though partition in kind is supposedly favored by the law on the books,
governing legal practice is different, as the Uniform Law Commission has
written:
\begin{quote}
Despite the overwhelming statutory preference for partition in kind, courts in a
large number of states typically resolve partition actions by ordering
partition by sale which usually results in forcing property owners off their
land without their consent. This occurs even in cases in which the property
could easily have been divided in kind or an overwhelming majority of the
cotenants had opposed partition by sale or even in some cases when the only
remedy any cotenant petitioned the court to order was partition in kind and not
partition by sale.
\end{quote}
\sentence{uphpa at 1}.
``Heirs'
property,'' that is, property whose ownership is divided by intestate
succession over several generations, has resulted in highly fractionated
ownership of land in many African-American families. The ULC explains that
``many of these owners [in possession] believe that their property ownership is
secure because they pay property taxes, they live on the land, and they make
productive use of the land. They also believe that their property may only be
sold against their will if a majority or more of their cotenants agree, which
gives some of these families with a large number of members with an interest in
the property false confidence that their ownership is extremely secure.'' But
their rights are, in fact, highly insecure. ``Unfortunately, the first time
that many of these owners are informed about the actual legal rules governing
partition is after a partition action has been filed, and often after critical,
early court rulings have been made against them.''

When heirs' property became valuable for development, third parties would often
acquire the interest of a distant relative who has a fractional share and
petition for partition. Given the often hundreds of people who own interests
in a piece of heirs' property, courts generally hold that partition in kind is
impossible. The resulting sale can dispossess people who have lived on or used
the land for decades; family members who would like to keep the land are rarely
able to outbid developers, who nonetheless usually pay substantially
below-market prices because of the forced nature of the sale. Ironically, once
sale is ordered, courts will not overturn a sale price unless it ``shocks the
conscience,'' even though the rationale for ordering the sale was that it would
provide the cotenants with more benefit than partition in kind. Sales have
been confirmed even though the property sold for twenty percent or less of its
market value. In many states, family members who oppose partition by sale can
even be required to pay the petitioner's attorneys' fees. Thomas Mitchell, a
law professor at the University of Wisconsin-Madison, says, ``It would be like
if you owned incredibly small shares of Microsoft, and you were given the right
to go to your local state court and file a motion to liquidate Microsoft at a
fire sale.''

\defwebsite{persky-cross-heirs}{
author=Anna Stolley Persky,
journal=ABA Journal,
title=In the Cross-Heirs,
date=may 2 2009,
url={http://www.abajournal.com/magazine/article/in_the_cross-heirs/},
}

The problem is substantial:
\begin{quote}
According to the Land Loss Prevention Project, a Durham, N.C.-based organization
that provides legal support to financially distressed farmers and landowners in
the state, of the 15 million acres of land acquired by African-Americans after
Emancipation, about 2 million remain owned by their descendants. Nationally,
it's estimated that African-American land ownership has decreased from as much
as 19 million acres in 1910 to 1.5 million acres in 1997, according to the
Southern Coalition for Social Justice.
\end{quote}
\sentence{see persky-cross-heirs}.

The problem also occurs in urban areas, where a family home may have been passed
down through several generations. Barriers to transfers by will include
poverty, lack of knowledge, or an unwillingness to cause family conflict by
picking specific heirs. Heirs property created significant problems in New
Orleans after Hurricane Katrina, when many people who thought they were owners
were unable to show title to their homes.

The common law operated under a presumption that grants to multiple grantees
created a joint tenancy---precisely the opposite of the modern presumption in
favor of a tenancy in common. Should we return to a presumption in favor of
joint tenancy, at least for family homes where children are the heirs by
intestacy?

Or should small fractional interests disappear over time? Recall that
traditionally, one cotenant's possession is not adverse to any other cotenant's
possession, unless there is an ouster. Although cotenants are due their share
of rents or other income arising from use of the property, mere failure to pay
them does not start the adverse possession clock running. Would it make sense
to change these rules? What are the risks from doing so? (There would be due
process and takings issues if legislatures tried to extinguish fractional
interests outright.\footnote{Due to previous legislation attempting to
assimilate members of Indian tribes into (white) American society, combined
with generations of inherited interests, reservation land has often become
highly fractionated. Many allotments have several hundred owners. Fractional
shares have been denominated in millions, billions, and even 54 trillion. For
example, one tract of forty acres produced \$1080 in annual income. It had 439
owners, one-third of whom received less than five cents in annual rent and
two-thirds of whom received less than a dollar. The largest interest holder
received \$82.85 a year, while the smallest was entitled to one penny every 177
years. The administrative costs to the Bureau of Indian Affairs of managing
this distribution were over \$17,000 per year. \emph{Hodel v. Irving}, 481 U.S. 704
(1987). Fractionation makes productive use of land almost impossible. Indian
Land Consolidation Act Amendment, \textsc{S. Rep. No.} 98-632, at 82-83 (1984),
reprinted in 1984 U.S.C.C.A.N. 5470. Allotment lands can only be leased or
partitioned with the unanimous consent of all interest holders. The Indian Land
Consolidation Act of 1983 attempted to solve these problems by mandating that
extremely fractionated interests would escheat to the relevant tribe, without
compensation to the fractional owners. The Supreme Court invalidated this law
as an unconstitutional taking, \textit{Hodel}, and likewise invalidated the
attempted replacement, \textit{Babbitt v. Youpee}, 519 U.S. 234 (1997). \par
The American Indian Probate Reform Act of 2004 tried again; the Department of
the Interior runs a land consolidation program under which it buys back
fractionated shares. Under the AIPRA, if Indian land would pass by intestate
succession, the Department of the Interior can buy any interests in the land
that are under 5\%. This purchase can occur even if the heir objects, unless
the heir is living on the land. Other heirs, co-owners, and the tribe on whose
reservation the land is located can also buy the land, as long as they pay fair
market value and have the consent of anyone holding more than a 5\% interest.})


The Uniform Partition of Heirs' Property Act, enacted in six states as of 2015,
provides co-owners with a right of first refusal to buy the petitioning
co-owner's share, and, if they do not exercise that right, attempts to create a
more competitive bidding process. The expectation is that even co-owners who
can't raise enough money to buy the entire parcel at fair market value, as at a
traditional partition sale, are more likely to be able to buy another
cotenant's fractional share. Under the Act, courts can also consider the
historical and cultural value of the land to the people living on it, not just
the economic value of the land, in deciding whether to reject partition by
sale. \emph{See, e.g.}, \emph{Chuck v. Gomes}, 532 P.2d 657 (Haw. 1975)
(Richardson, C.J., dissenting):
\begin{quote}
[T]here are interests other than financial expediency which I recognize as
essential to our Hawaiian way of life. Foremost is the individual's right to
retain ancestral land in order to perpetuate the concept of the family
homestead. Such right is derived from our proud cultural heritage.\ldots [W]e
must not lose sight of the cultural traditions which attach fundamental
importance to keeping ancestral land in a particular family line.
\end{quote}

\item \textbf{Contracting around partition rights}. Should cotenants be able to
waive their right to partition? \textit{See} \emph{Gore v. Beren}, 867 P.2d 330
(Kan. 1994) (cotenants agreed to a right of first refusal if any cotenant wished
to sell; this agreement impliedly waived the right to partition and didn't
violate the Rule Against Perpetuities because it was personal to the parties and
would necessarily end during the lifetime of one of the parties); \textit{see
also} \emph{Michalski v. Michalski}, 142 A.2d 645 (N.J. Super. 1958) (otherwise
valid restriction on right to partition may be unenforceable when circumstances
have changed so much that enforcement would be unduly harsh); \emph{Reilly v.
Sageser}, 467 P.2d 358 (Wash. Ct. App. 1970) (option to purchase from cotenant
at cost of cotenant's investment in land and improvements was valid unless both
parties agreed or one party substantially breached other elements of agreement);
\textit{cf.} \emph{Low v. Spellman}, 629 A.2d 57 (Me. 1993) (invalidating right
of first refusal given to grantors, heirs, and assigns as in violation of the
Rule Against Perpetuities; fixed price of \$6500 also created unreasonable
restraint on alienation).


\item \textbf{Partitioning a future interest}. Can owners who own only a future
interest seek partition of that interest? At common law, the answer was no
because they lacked a present possessory interest, and some states still adhere
to this rule. \emph{See, e.g.}, \emph{Trieber v. Citizens State Bank}, 598 N.W.2d 96 (N.D.
1999). Many states, however, allow co-owners of vested future interests to seek
partition of that interest. \emph{See, e.g.}, \textsc{Ark. Code} \S~18-60-401.

\item \textbf{Partitioning personal property}. Are there circumstances in which
a physical partition of personal property would make sense? How would you
divide up a photo album with hundreds of photographs? \textit{Cf.} \emph{In re
Estate of McDowell}, 345 N.Y.S.2d 828 (Sur. Ct. 1973) (custody of rocking chair
desired by both heirs should be divided in six-month increments, remainder to
the survivor); Ronen Perry \& Tal Zarsky, \textit{Taking Turns}, 43
\textsc{Fla. St. U. L. Rev.} (2015). This solution raises a more general
question: why don't we see more co-ownership of real property on the time-share
model?

