More than one person can ``own'' a thing at any given time.  Their rights will
be exclusive as against the world, but not exclusive as against each other. 
When conflicts between them develop, or when the outside world seeks to
regulate their behavior, we need to understand the nature and limits of their
rights.

In this section, we will not address the form of concurrent ownership known as
partnership, which we cover separately, though you will see some comparative
references to it in the case that follows.
\having{intro-corporations}{Nor will we address corporations
(in which ownership can be nearly infinitely divided and is separated from
control; see Corporations section)}{Nor will we address corporations
(in which ownership can be nearly infinitely divided and is separated from
control; see Corporations section)}{Nor will we address corporations
(in which ownership can be nearly infinitely divided and is separated from
control)}.  These topics are dealt with in detail in
business associations and similar courses.  We will also not consider forms of
concurrent ownership that are of purely historical interest, such as
coparceny.\footnote{A form of ownership only available to female heirs, when
there were no male heirs.}  The main types of co-ownership we will consider are
(1) tenancy in common, (2) joint tenancy, and (3) tenancy by the entireties,
along with a brief look at (4) community property, a particular kind of
co-ownership available in some states.

In the late 1980s, a sample of real estate records showed that about two-thirds
of residential properties were held in some form of co-ownership.  Evelyn Alicia
Lewis, \emph{Struggling with Quicksand: The Ins and Outs of Cotenant Possession
Value Liability and a Call for Default Rule Reform}, 1994 \textsc{Wis. L. Rev.}
331; see also \textsc{Carole Shammas et al.}, \textsc{Inheritance in America
from Colonial Times to the Present} 171-72 (1987) (showing percentage of land
held in joint tenancies rising from under 1\% in 1890 to nearly 80\% in 1960,
then dropping to 63\% in 1980); N. William Hines, \emph{Real Property Joint
Tenancies: Law, Fact, and Fancy}, 51 \textsc{Iowa L. Rev.} 582 (1966) (finding
that joint tenancies in Iowa rose from under 1\% of acquisitions in 1933 to over
a third of farm acquisitions and over half of urban acquisitions in 1964, almost
exclusively by married couples); Yale B. Griffith, \emph{Community Property in
Joint Tenancy Form}, 14 \textsc{Stan. L. Rev.} 87 (1961) (study of California
counties in 1959 and 1960 finding that married couples held over two-thirds of
property as cotenants, 85\% of which was as joint tenants).  

Given that many justifications for the institution of private property rely on
the idea that competing interests in property lead to inefficiency, waste, and
conflict, it is perhaps surprising that so much private property is, in
practice, owned by more than one person.  If communal ownership is so
inefficient, why do we recognize so many kinds of co-ownership?  

