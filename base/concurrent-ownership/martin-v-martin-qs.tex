\expected{martin-v-martin}

\item \textbf{Recurring conflicts between cotenants}. Rules for cotenant
liability are incoherent and unsatisfactory despite centuries of litigated
cases. Evelyn Lewis speculates that ``cotenant conflicts receive little
attention from property law reformers'' because they involve
``\,`one-shotters'---parties who rarely litigate, who are predominantly members
of the obedient middle-class and who suffer quietly the rules of law they were
too unsophisticated to know or consider in advance of the conflict.'' Evelyn
Lewis, \textit{Struggling with Quicksand: The Ins and Outs of Cotenant
Possession Value Liability}, 1994 \textsc{Wis. L. Rev}. 331.


Management conflicts can arise easily because, unlike in a trust or a
corporation (both forms of joint ownership) there is no one with the legal
right to manage the property on behalf of the other owners, and a cotenant who
takes on the burden of management is not entitled to be paid for her services
to the others. \emph{See} \emph{Combs v. Ritter}, 223 P.2d 505 (Cal. Ct. App.
1950). Although each cotenant has the right to possess and benefit from the
property, and the duty to pay her share of necessary expenses such as taxes,
there is no mechanism for group decision-making. If co-owners can't agree, they
may simply have to split---by divorce followed by a transfer to one party or
sale in the case of tenancy by the entirety and community property; by severance
and partition for joint tenants; and by partition for tenants in common. Short
of partition, which involves selling or physically dividing the property, the
only assistance the courts offer cotenants is a claim for accounting for rents
or profits received by another cotenant, or a claim for contribution for
payments of another cotenant's share of taxes, mortgage payments, and necessary
maintenance expenses.



\item \textbf{Ouster}. Denial of a right to possession constitutes ouster, and
the damages are the non-possessing cotenant's share of the rental value of the
property. \emph{Harlan v. Harlan}, 168 P.2d 985 (Cal. Ct. App. 1946) (damages
for ouster are rental value).


Evelyn Lewis concludes that, as with adverse possession, the standard for what
constitutes an ouster is so manipulable that courts can reach almost any result
on any given set of facts. \emph{See, e.g.}, \emph{Cox v. Cox}, 71 P.3d 1028
(Idaho 2003) (tenant in common was ousted and was entitled to {\textonehalf} of
the fair rental value of the house occupied by her brother when he told her he
was selling the house and that she ``had better find a place to live'');
\emph{Mauch v. Mauch}, 418 P.2d 941 (Okla. 1966) (cotenants in possession of
family farm ousted widowed sister-in-law by telling her they ``didn't want to
have her on the place'' and that she ``was not to come back''); \emph{but see}
\emph{Fitzgerald v. Fitzgerald}, 558 So.2d 122 (Fla. Dist. Ct. App. 1990)
(ex-wife didn't oust ex-husband by telling him to leave the family home and that
otherwise ``she'd call the law'').


What if one cotenant denies that the other has any title to the property?
\emph{Estate of Duran}, 66 P.3d 326 (N.M. 2003) (cotenant lived on the property
kept silent or gave evasive answers to questions about his use of the property;
this was not ouster where he ``never expressly told [the other cotenants] that
he claimed to own their portions of the property''). Purporting to convey full
title to the property is an ouster, since it sets up a claim for adverse
possession by the grantee. \emph{Whittington v. Cameron}, 52 N.E.2d 134 (Ill.
1943).




What if one cotenant seeks to use a portion of the land, and the other prevents
her from doing so, perhaps by building a structure on it?



\item \textbf{Constructive Ouster.} What if the property is a single-family home
and the co-tenants are recently divorced or separated? \emph{Hertz v. Hertz},
657 P. 2d 1169 (N.M. 1983) (applying theory of ``constructive ouster'' to
require payment of half of fair rental value); \emph{Stylianopoulos v.
Stylianopoulos}, 455 N.E.2d 477 (Mass. Ct. App. 1983) (divorce constituted
ouster, so ex-wife had to pay fair rental value to ex-husband); \emph{In re
Marriage of Watts}, 217 Cal. Rptr. 301 (Ct. App. 1985) (separated spouse must
reimburse community for exclusive use of house); \emph{Palmer v. Protrka}, 476
P.2d 185 (Or. 1970) (if, as a practical matter, the couple can't live together,
requiring the cotenant in possession to pay half of fair rental value most
closely matches parties' intentions).


Suppose a woman moves out of her family home after being physically assaulted by
her husband. The husband begs her to come back, but she refuses. After two
years, when their divorce becomes final, the ex-wife sues for half the fair
rental value of the house during the two-year period she was out of possession.
 Should she win? What if, instead of the wife leaving, she ejects the husband
and tells him not to come back, and two years later, after she's awarded the
house in the divorce, he sues for half the fair rental value of the house
during the two-year period he was out of possession? \emph{See} \emph{Cohen v.
Cohen}, 746 N.Y.S2d 22 (App. Div. 2002) (no right to rent for period during
which a court protective order barred cotenant from the property due to his
assaultive conduct).



The majority rule is against constructive ouster, in the absence of physical
exclusion. \emph{See, e.g.}, \emph{Reitmeier v. Kalinoski}, 631 F. Supp. 565
(D.N.J.
1986) (``[T]he mere fact that defendant does not wish to live with plaintiff on
the premises is of no import. What counts is that she could physically live on
the premises.'').



Which rule is better? If you were advising a client in a divorce, how would you
deal with co-owned property?



What if the property is so small that physical occupation by all cotenants is
impracticable? Some courts will also call this a ``constructive ouster'' of
the cotenants out of possession. \emph{Capital Fin. Co. Delaware Valley, Inc. v.
Asterbadi}, 942 A.2d 21 (N.J. Super. Ct. App. Div. 2008) (finding that a bank
that was a cotenant through foreclosure with the wife of the defaulting
mortgagor was constructively ousted from a single-family home).



\item \textbf{Contribution: sharing the costs}. ``[T]he protection of the
interest of each cotenant from extinction by a tax or foreclosure sale imposes
on each the duty to contribute to the extent of his proportionate share the
money required to make such payments.'' 2 \textsc{American Law of Property}
\S~6.17. Because failure to pay carrying costs increases the risk that the
asset will be lost to all cotenants, every concurrent owner has an obligation
to pay her share. \textit{See also} \emph{Beshear v. Ahrens}, 709 S.W.2d 60 (Ark.
1986) (allowing contribution for mortgage payments and property taxes as
``expenditures necessarily made for the protection of the common property'').


The majority rule is that cotenants out of possession need not share in the
costs of repairs in the absence of an agreement to do so. The idea is that
questions ``of how much should be expended on repairs, their character and
extent, and whether as a matter of business judgment such expenditures are
justified,'' are too uncertain for judicial resolution. 2 \textsc{American Law
of Property} \S~6.18. But then, in a partition action, cotenants who pay for
repairs will get credit for them---does that make sense? Further, some courts
will allow contribution for ``necessary'' repairs. \emph{Palanza v. Lufkin}, 804
A.2d 1141 (Me. 2002) (finding contribution towards necessary repairs justified,
even though some of the repairs had cosmetic effects). Some jurisdictions
require a cotenant to provide her fellow cotenants with notice and opportunity
to object to the repairs in order to be entitled to contribution. \emph{Anderson
v. Joseph}, 26 A.3d 1050 (Md. Ct. Spec. App. 2011) (denying contribution for
repairs that resulted from ``massive flooding'' for failure to provide notice).



\item \textbf{Accounting: the right to share in profits}. Cotenants who allow
others to use the land, whether to exploit resources or to rent, must give
other cotenants their shares of any consideration received from the third-party
users.


Recall that in at least some contexts one cotenant cannot unilaterally exercise
the right to exclude of the other cotenants. But that isn't always true with
respect to productive uses of land by third parties with permission of one
cotenant. To be sure, in some states, a lease from only one co-owner is void
and the lessee can be ejected. But in other states, one cotenant can lease his
interest, subject only to a duty to account to the non-leasing cotenants for
net profits. \emph{Swartzbaugh v. Sampson}, 54 P.2d 73 (Cal. Ct. App. 1936).
Where there is such a duty, to whom does the lessee owe rent? The answer is that
she only owes rent to the leasing cotenant, unless she ousts the other
cotenants. Those other cotenants must look to a contribution action against the
leasing cotenant.



The usual rule is that cotenants must account for the raw value of resources
they extract themselves, but particularly bad misbehavior by a cotenant may lead
to an award of the processed value. \emph{Kirby Lumber Co. v. Temple Lumber
Co.}, 83 S.W.2d 638 (Tex. 1935) (raw value of timber where timber was taken in
good faith); \emph{cf.} \emph{White v. Smyth}, 214 S.W.2d 967 (Tex. 1948)
(cotenant who mined asphalt without consent from other cotenants had to account
for net profits, although he took no more than his one-ninth interest---resource
could not be partitioned in kind because the quality and quantity of asphalt
varied sharply across the parcel in ways that could not be easily determined;
cotenant couldn't take the most easily mined resources for himself and make his
own partition).

Absent an ouster, an accounting usually just requires the cotenant to share the
actual value received, not the fair market value. Suppose a lease claims to be
nonexclusive and to only lease one cotenant's share, and is for half of the fair
market rental value of the property. What should happen when the other cotenant
seeks an accounting? \emph{See} Annot., 51 A.L.R.2d 388 (1957). Suppose the
lease is made by one cotenant to spite or harm another? \textit{Cf.}
\emph{George v. George}, 591 S.W.2d 655 (Ark. Ct. App. 1979) (where 99-year
lease carried nominal rent and the court found an intent to defraud the
cotenant, the lease was set aside).

\item \textbf{Tenants in possession; tenants out of possession}. \textit{Martin}
applies the majority rule that---absent ouster---a cotenant in possession need
not pay anything to cotenants out of possession if she lives on and farms the
land, absent an ouster. \emph{DesRoches v. McCrary}, 24 N.W.2d 511 (Mich. 1946)
(no duty of cotenant in possession to pay rent to other cotenants).
Reciprocally, there is generally no ouster if one cotenant requests her share of
the fair rental value of the land from the occupying cotenant, and the occupying
cotenant denies the request. \emph{Von Drake v. Rogers}, 996 So. 2d 608 (La. Ct.
App. 2008) (``A co-owner in exclusive possession may be liable for rent, but
only beginning on the date another co-owner has demanded \textit{occupancy} and
been refused.'') (emphasis added). But a few cases hold that denying a request
for rent constitutes an ouster. \emph{Eldridge v. Wolfe}, 221 N.Y.S. 508 (1927).

Why might courts have developed a practice of requiring cotenants to account for
profits from mining and cutting lumber, but not for profits from their own
farming or residential uses of co-owned property? Logically, the cotenant in
possession should have to pay---she is receiving a benefit from using the
land, the fair market rental value of the property, and the other cotenants are
not. As \textit{Martin} itself proves, if she did rent the land to a third
party, she would be required to share that benefit with the other co-owners.
This rule creates an incentive for the cotenant to stay in possession rather
than renting the land out, even if renting to a third party would be more
efficient overall.


\item \textbf{The relationship between contribution and accounting}. If one
cotenant occupies the property, with no ouster, and seeks contribution from the
non-occupant for his share of the taxes and insurance, can the non-occupant
offset the amounts due by the value of living on the property to the occupant?
Many courts say yes. \emph{See, e.g.}, \emph{Barrow v. Barrow}, 527 So. 2d 1373
(Fla. 1988) (occupant can only recover contribution if non-occupant's
proportionate share of expenses is greater than the value of occupying the
property); \emph{Esteves v. Esteves}, 775 A.2d 163 (N.J. Super. Ct. App. Div.
2001) (parents who occupied house for 18 years were entitled to be reimbursed by
their son for half of the expenses of mortgage and maintenance, but the son was
allowed to set off the amount equal to the reasonable value of the parents' sole
occupancy). This view is not strictly consistent with the majority rule that
non-ousting tenants are not liable to non-possessing cotenants for rent, because
it means that the occupant is essentially paying the non-occupant for being able
to live on the land. Is this rule, which will often keep much actual cash from
changing hands nonetheless fair?

The minority view is that no defensive offset is available against a cotenant in
possession, absent ouster. \emph{Yakavonis v. Tilton}, 968 P.2d 908 (Wash. Ct.
App. 1998); \emph{Baird v. Moore}, 141 A.2d 324 (N.J. App. Div. 1958) (cotenant
out of possession may not offset value of occupation if cotenant's possession is
not adverse). Which rule makes more logical sense? More practical sense?


Basically, courts often have enough flexibility to rule in the direction the
equities point---finding that contribution is or isn't available. The need to
balance the harms from imposition of unexpected costs on cotenants out of
possession with the harms to the property's value from negligent co-owners also
gives courts flexibility. Ultimately, because partition is always available to
cotenants who truly can't agree, it makes sense for courts to point them
towards partition if they're fighting over maintenance and repairs.


In \textit{Martin}, when calculating Garis and Peggy's 1/8
share of the ``net rent,'' what expenses should be deducted? Can they be
required to pay a share of the costs of developing the mobile home park, such
as putting in sewage lines and electrical connections? Note that a cotenant is
generally not entitled to contribution from other cotenants for the costs of
improving the property (see note \ref{martin-improvements} below). But, on
partition, the improver is entitled to the part of the property that's been
improved, or in case of sale to the lesser of (1) the increase in value due to
the improvement or (2) the cost of the improvement. Should that rule be
applied in an accounting as well?

Lewis suggests that courts use ouster to enagage in the ``equitable
second-guessing that so often blurs crystalline rules.'' \emph{Compare}
\emph{Spiller v. Mackereth}, 334 So. 2d 859 (Ala. 1976) (lock change wasn't
ouster), \emph{with} \emph{Morga v. Friedlander}, 680 P.2d 1267 (Ariz. Ct. App.
1984) (lock change was ouster). In effect, courts use ouster, plus the majority
rule allowing offset of the value of an occupying cotenant's possession in an
action for contribution, to nullify the formal rule that any cotenant can occupy
the land rent-free, regardless of the size of his or her share, and still seek
contribution for necessary expenses.


\item \textbf{Quasi-fiduciary duties of good faith}. Cotenants are fiduciaries
for each other, at least if they receive their interests in the same will or
grant, or through the same inheritance. \emph{Poka v. Holi}, 357 P.2d 100 (Haw.
1960) (cotenants have fiduciary obligation to give other cotenants adequate
notice of adverse claims to the property); \emph{but see} \emph{Wilson v. S.L.
Rey, Inc.}, 21 Cal. Rptr. 2d 552 (Ct. App. 1993) (cotenants who acquire
interests at different times by different instruments have no fiduciary
relationship).


If one co-tenant buys the property at a tax sale or a foreclosure sale, the
title is shared with the other co-tenants: for these purposes, the co-tenant is
a fiduciary for the other co-tenants. \emph{Johnson v. Johnson}, 465 S.W.2d 309
(Ark. 1971); \textit{but cf.} \emph{Stevenson v. Boyd}, 96 P. 284 (Cal. 1908)
(finding assertion of cotenant's claim barred by laches after four-year delay).
However, the purchasing co-tenant can seek contribution from the others, so that
they bear their fair share of the cost of removing the lien or mortgage. Why
would the courts create such a fiduciary duty? What is the abusive practice that
they fear?


\item \label{martin-improvements}\textbf{Improvements}. Any cotenant has the
right
to make improvements to the property, but other cotenants are not required to
contribute. \emph{See} \emph{Knight v. Mitchell}, 240 N.E.2d 16 (Ill. Ct. App.
1968) (cotenant couldn't seek contribution for developing and running oil wells,
though he could set off necessary operating expenses in other cotentant's action
for accounting of his profits); Johnie L. Price, \textit{The Right of a
Coteanant to Reimbursement for Improvements to the Common Property}, 18
\textsc{Baylor L. Rev}. 111 (1966).

In most states, the interests of the improver will be protected if that won't
harm the interests of the other cotenants. This usually allows the improver to
recoup the added value, if any, resulting from his improvements on partition, or
in accounting for rents and profits. \emph{Graham v. Inlow}, 790 S.W.2d 428
(Ark. 1990). But if improvements fail to pay off, the improver is not
compensated---he bears all the risk. A few cases limit recovery to the smaller
of the amount of value added by an improvement or its costs. The risk is borne
by the improver, but the rewards are shared. Which rule makes more sense?

\item \textbf{Waste}. If one cotenant damages the property or harms
its value, other cotenants may have claims for waste. While the ordinary
remedy for waste is treble damages, courts will normally just hold the tenant
in possession accountable for net profits from exploiting the property, as
explained above in the discussion of removing timber and similar resources.
\textsc{Casner, American Law of Property}, \S~6.15. What effects does that
rule have on the use of land?


Waste claims are correspondingly difficult to win. \emph{Davis v. Byrd}, 185
S.W.2d 866 (Mo. 1945) (mining by one cotenant isn't waste as long as the other
cotenants aren't excluded and the miner doesn't willfully or negligently injure
the land); \emph{Hihn v. Peck}, 18 Cal. 640 (1861) (cotenant may remove valuable
timber ``to an extent corresponding to [his] share of the estate'' without
commiting waste); \emph{Prairie Oil \& Gas Co. v. Allen}, 2 F.2d 566 (9th Cir.
1924) (cotenant can produce oil without other cotenants' consent, but cannot
exclude other cotenants from exercising the same right). Consider whether time
matters: should the standard for what constitutes waste vary depending on
whether the other interest-holders have present interests (and could act now to
reap their own benefits, albeit at greater cost than waiting) or future
interests (and thus can only wait for their ownership interests to attach)?

\item \textbf{Adverse possession by cotenants against other cotenants}. Because
each cotenant has the right to possession, it can be difficult for one cotenant
to possess adversely to another. Under New York law, a cotenant must have
exclusive possession for ten years before the statutory adverse possession can
even \textit{begin} to run against other cotenants. \emph{Myers v. Bartholomew},
697 N.E.2d 160 (N.Y. 1998). After all, the fact that someone else is living on
and using the land lacks its ordinary significance to cotenants. \emph{Ex parte
Walker}, 739 So. 2d 3 (Ala. 1999) (cotenant's redemption of property at tax sale
in 1934, payment of all property taxes, exclusive possession for over 50 years,
demolition of old buildings, and harvesting of timber did not show adverse
possession against other cotenants); \emph{Tremayne v. Taylor}, 621 P.2d 408
(Idaho 1980) (``A cotenant who claims to have adversely possessed the interest
of his cotenants must prove that the fact of adverse possession was `brought
home' to the cotenants.''); \emph{Hare v. Chisman}, 101 N.E.2d 268 (Ind. 1951)
(husband's sole possession of house after wife died was not adverse to his
cotenants, her heirs, since it ``was not an unnatrual act of them to permit
their father to occupy this property, collect the income, pay the expense, and
enjoy the surplus''); \emph{West v. Evans}, 175 P.2d 219 (Cal. 1946) (cotenant
out of possession must have either actual or constructive notice of hostility;
recordation of a deed isn't sufficient notice); Official Code Ga. Ann.
\S~44-6-123 (allowing cotenant to gain title by adverse possession if she
``effects an actual ouster, retains exclusive possession after demand, or gives
[her] cotenant express notice of adverse possession'').

Adverse possession is, however, not entirely impossible in these circumstances.
\textit{See} \emph{Johnson v. James}, 377 S.W.2d 44 (Ark. 1964) (presumption
against adversity is even stronger when cotenants are related, though
presumption was overcome through sole possession for 36 years, where cotenants
knew of a will purportedly granting occupant sole possession and said nothing);
\emph{McCree v. Jones}, 430 N.E.2d 676 (Ill. Ct. App. 1981) (finding in favor of
claimant who'd been in possession for thirty years under a quitclaim deed
purporting to give title to the entire property).


\item \textbf{Intangible assets}. In the U.S., ``joint authorship'' occurs when
two or more authors contribute to the creation of a unitary work of authorship,
such as a song with music written by one author and lyrics written by another.
(Here, ``joint'' doesn't mean what it means in real property. There is no
right of survivorship, so the ownership rights behave more like what you know
as tenancy in common.) Courts have interpreted copyright law to impose a
default rule, absent explicit agreement, that each joint author owns an equal
share of the resulting work, even if her contribution was substantially less
than that of other authors. This rule, which is not mandated by the statute,
has led courts to be extremely reluctant to find joint authorship when there is
one clear ``dominant'' author and someone else seeks to be recognized as a
co-author. Because copyrights are intangible, they cannot be partitioned, nor
can there be an ouster of one co-author by another. Instead, each co-owner can
grant a nonexclusive license to other people to use the work---whether that
means putting a song on a record, using a sample of it in a new song, or using
it in a television show. This right to license is subject only to a duty to
account to the other co-owners for their shares of the resulting profits. An
exclusive license requires the agreement of all the co-owners acting together.



Suppose one co-author, angry at her co-author, grants Quentin Tarantino a
nonexclusive license to turn their book into a movie for \$1, and duly gives
her co-author fifty cents. Because of this license, no other moviemaker will
buy the rights, fearing competition from Tarantino's movie. Has the licensor
committed waste? Would it matter if, instead of acting out of malice, the
co-author granted the \$1 license because she believed in Tarantino's vision
for the film and only a low price would induce him to take on the book as his
next project? Do tenancy in common rules work for property that can't be
exclusively possessed?



\item \textbf{Concluding thoughts: crystals and mud}. Transaction
costs---the costs of managing the property and getting cotenants to agree---can
be very high among cotenants, as compared to the costs of having a manager
with authority to make decisions for the group. (For example, consider the
issue of approving a particular tenant who wishes to rent the property and have
exclusive possession.) The actively engaged cotenant who rents to a third
party gets only some of the gain, but takes most of the risk. After all, if
the renter turns into a nightmare who trashes the place, the cotenant who
rented the property will be liable for any harm; but the other cotenants might
sue to share in any gains that materialize. Professor Carol Rose argues that
courts sometimes impose equitable duties---muddy rules---on parties in order
to replicate the results that would have occurred had they trusted each other
and behaved fairly and decently towards one another. Thus, our rules about
co-ownership are not just rules about economic efficiency, but about how people
should behave. \textit{See generally} Carol Rose, \textit{Crystals and Mud in
Property Law}, 40 \textsc{Stan. L. Rev.} 577 (1988). Does this help you make
any sense of the co-ownership rules?

