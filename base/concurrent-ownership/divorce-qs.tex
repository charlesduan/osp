\expected{narrative-divorce}

\item As of 2014, Dr. O'Brien was apparently still practicing emergency
medicine.

In 2016, an amendment to New York law became effective, prohibiting treating
professional licenses and enhanced earnings as a marital asset. However, the
new law specified that, ``in arriving at an equitable division of marital
property, the court shall consider the direct or indirect contributions to the
development during the marriage of the enhanced earning capacity of the other
spouse.'' Of course, if the only asset is the license or earning potential,
that won't help, and that was the exact situation addressed in
\textit{O'Brien}. Why do you think the New York legislature made this change?

\item
Are lump-sum payments or periodic payments better means of handling the property
division issues here?

\item
How would the \textit{O'Brien} rule influence the husband's choices
post-divorce? What \textit{should} happen if he switches to a less lucrative
specialty? Would it matter if the switch were made out of spite, versus if
there were no demand for his specialty and he switched as a matter of economic
rationality? What should happen if he switches to a more lucrative specialty?
What if he leaves the field entirely---could the court order him to work?
What should happen if he wins \$50 million in the lottery and quits working?

Even without all these possibilities, the present value of a lengthy career can
be hard to predict. Valuation of things like pensions (if there ever are any
again) or other non-vested rights (such as potential stock options) may
likewise be very complicated, but nonetheless they may form a significant part
of a couple's assets.

\item
Should the wife's post-divorce choices matter? What if she founds a web startup
that makes her three times as much money as he has? What if, during the
pendency of the divorce proceedings, the startup is doing wonderfully, but five
years later her business partner embezzles the cash and leaves her responsible
for a \$250,000 debt? What if she dies before the support period ends --
should her estate receive the remaining money due?

Suppose, instead of earning a degree, the husband had simply lain around all
day, allowing his wife to support him. Would she be entitled to support to
compensate her for her lost years? Would \textit{he} be entitled to support
because his skills had deteriorated over time? Would it matter if the divorce
occurred before the husband received a degree and a license to practice?

Do your answers give you any insight into whether the label ``property'' is
helpful in this case?

\item
The majority view is that a degree is not ``property.'' \textit{See In re
Marriage of Graham}, 574 P.2d 75 (Colo. 1978):
\begin{quote}
An educational degree, such as an M.B.A., is simply not encompassed even by the
broad views of the concept of `property.' It does not have an exchange value or
any objective transferable value on an open market. It is personal to the
holder. It terminates on death of the holder and is not inheritable. It cannot
be assigned, sold, transferred, conveyed, or pledged. An advanced degree is a
cumulative product of many years of previous education, combined with diligence
and hard work. It may not be acquired by the mere expenditure of money. It is
simply an intellectual achievement that may potentially assist in the future
acquisition of property. In our view, it has none of the attributes of property
in the usual sense of that term.
\end{quote}

However, even in majority-rule states, principles of equitable division include
considerations such as each party's contribution to the acquisition of the
property, including contributions that assisted one spouse in developing the
other's earning power. \textit{See} \emph{Ferguson v. Ferguson}, 639 So. 2d 921
(Miss. 1994); \emph{Schaefer v. Schaefer}, 642 N.W.2d 792 (Neb. 2002) (graduate
degree isn't property, but one spouse's support of the other's education is a
factor to be considered in dividing the marital assets, as well as in
determining whether to award alimony). As the New Jersey Supreme Court stated in
\emph{Mahoney v. Mahoney}, 453 A.2d 527 (N.J. 1982):
\begin{quotation}
[E]very joint undertaking has its bounds of fairness. Where a partner to
marriage takes the benefits of his spouse's support in obtaining a professional
degree or license with the understanding that future benefits will accrue and
inure to both of them, and the marriage is then terminated without the
supported spouse giving anything in return, an unfairness has occurred that
calls for a remedy.

\dots. In effect, through her contributions, the supporting spouse has
consented to live at a lower material level while her husband has prepared for
another career. She has postponed, as it were, present consumption and a higher
standard of living, for the future prospect of greater support and material
benefits. The supporting spouse's sacrifices would have been rewarded had the
marriage endured and mutual expectations of both of them been fulfilled\dots.
In this sense, an award that is referable to the spouse's monetary contribution
to her partner's education significantly implicates basic considerations of
marital support and standard of living-factors that are clearly relevant in the
determination and award of conventional alimony.
\end{quotation}
Under \textit{Mahoney}, courts can't make a permanent distribution of the value
of professional degrees and licenses, because of the ``potential for inequality
to the failed professional or one who changes careers'' and the difficulty of
valuation. However, New Jersey courts may award reimbursement alimony, based
on the contributions received from the supporting spouse, because marriage
shouldn't be a ``free ticket'' to education and training. \textit{See} \emph{Guy
v. Guy}, 736 So.2d 1042 (Miss. 1999) (professional degrees are not marital
property, but former husband would be entitled to some reimbursement if he paid
for former wife's education).

Similarly, California law presumes that reimbursement is appropriate for
contributions to a spouse's education that substantially enhance her earning
potential. This presumption can be overcome, and reimbursement reduced or
eliminated, if the couple has already substantially benefited from the
education; California further presumes this substantial benefit has occurred
after ten years of marriage; if the supporting spouse received similar support
for his own education; or if the education allows the supported spouse to get
employment that reduces support to which she would otherwise be entitled.
\textsc{Cal. Fam. Code} \S~2641.\footnote{Contribution to education that
increases a spouse's earning potential is also relevant to whether a court
should award alimony, along with other factors such as the extent to which each
person's earning capacity is sufficient to maintain the marital standard of
living, the length of the marriage, and the needs of the parties. \textsc{Cal.
Fam. Code} \S\S~4320, 4330.}

Is a reimbursement theory sufficient? Should housekeeping and childcare
services, if provided by the supporting spouse, be factored into the necessary
reimbursement? What about emotional support, such as a counselor or ``life
coach'' might provide?

\item
If most gifts between people who are engaged, except for the engagement ring,
are unrecoverable donative transfers, why should the result be any different
after marriage? Consider the Pennsylvania Supreme Court's reasoning in
\emph{Bold v. Bold}, 524 Pa. 487, 574 A.2d 552 (Pa. 1990):
\begin{quote}
While we agree \dots that marriage is not a business enterprise in which
strict accountings are to be had for moneys spent by one spouse for the benefit
of the other, it appears to us that this case does not involve strict
accountings, but gross accountings. Supporting spouses in these cases feel
entitled to reimbursement, we believe, not because they have sacrificed to
support the other spouse, but because they are, to use a strong word,
``jettisoned'' as soon as the need for their sacrifice, albeit in part a legal
obligation, comes to an end. In retrospect, perhaps unintentionally, the
supporting spouse in such a case can be said to have been ``used.'' At least
this is the perception of the supporting spouse, and we believe that this
perception is not totally without foundation in all cases \dots [T]he supporting
spouse in a case such as this should be awarded equitable reimbursement to the
extent that his or her contribution to the education, training or increased
earning capacity of the other spouse exceeds the bare minimum legally obligated
support\dots.
\end{quote}

What about the earning power of a celebrity---should that be ``property''
divisible at divorce? After all, no one needs a license to be a celebrity
(even if they should). \textit{See} \emph{Elkus v. Elkus}, 572 N.Y.S.2d 901
(N.Y. App. Div. 1991) (finding that a performing career and celebrity status are
marital property subject to equitable distribution ``to the extent the
defendant's contributions and efforts led to an increase in the value of the
plaintiff's career''); \emph{Piscopo v. Piscopo}, 555 A.2d 1190 (N.J. Super.),
\emph{aff'd},
557 A.2d 1040 (celebrity goodwill enhanced during marriage was marital property
subject to equitable distribution even if it came from innate talent);
\textit{see also} Paloma Peracchio, Comment, \emph{The Value of Creative
Professionals in the Entertainment Capital of the World: Why ``Celebrity
Goodwill'' Should Be a Divisible Community Property Interest in California
Divorces}, 28 \textsc{Loy. L.A. Ent. L. Rev.} 129 (2008).

\item
Relatedly, goodwill is a concept designed to capture the ongoing value of a
business due to its reputation and other intangibles, over and above the value
of its equipment, cash in the bank, accounts receivable, etc. Earning capacity
itself is not goodwill, but a reputation that makes future business more likely
is, and many states treat goodwill earned during the marriage as property
subject to equitable distribution. \emph{Dugan v. Dugan}, 457 A.2d 1 (N.J. 1983)
(goodwill of a solo law practice was subject to equitable distribution, even
though the goodwill could not be sold separate from the ex-spouse's own
services); \emph{Mace v. Mace}, 818 So. 2d 1130 (Miss. 2002) (husband's medical
practice, as income-producing enterprise made possible by his professional
degree, was subject to equitable division). \textit{But see} \emph{Prahinski v.
Prahinski}, 582 A.2d 784 (Md. 1990) (goodwill of a solo law practice is not
subject to equitable distribution).

Goodwill intrinsic to a spouse's business is usually deemed to be marital
property. \emph{Finch v. Finch}, 825 S.W.2d 218 (Tex. Ct. App. 1992);
\emph{Nicholson v. Nicholson}, 669 S.W.2d 514 (Ark. Ct. App. 1984). But goodwill
attributable to the spouse's continued involvement in the business is usually
considered to be separate property. \emph{Thompson v. Thompson}, 576 So. 2d 267
(Fla. 1991) (``If goodwill depends on the continued presence of a particular
individual, such goodwill, by definition, is not a marketable asset distinct
from the individual. Any value which attaches to the entity solely as a result
of personal goodwill represents nothing more than probable future earning
capacity, which \dots is not a proper consideration in dividing the personal
property.''). How would you tell the difference? Suppose one spouse runs the
beloved local bakery Mother's Macaroons. She greets her customers by name and
uses recipes handed down from her grandmother. However, it would be perfectly
legal for her to transfer the premises, the name, and the recipes to someone
else. Is the goodwill of Mother's Macaroons personal to her or intrinsic to the
business? If the latter, does the rule discriminate in favor of professionals
like doctors (and, not surprisingly, lawyers)?

\item
The American Law Institute (ALI), an often-influential organization that seeks
to create coherent bodies of law, follows the majority position that earning
capacity, skills, education, and the like are not marital property.
\textsc{Principles of the Law of Family Dissolution: Analysis and
Recommendations} \S~4.07 (2002). But it also holds that business and
professional goodwill earned during marriage are marital property ``to the
extent that they have value apart from the value of spousal earning capacity,
spousal skills, or post-dissolution labor.'' The ALI also states that spouses
should be compensated for lower earning capacity due to taking on a
disproportionate share of childcare; for loss of a standard of living due to
divorce when one spouse makes significantly more than another; and for
reimbursement for financial contributions to the ex-spouse's education or
training. \S\S~5.05-.04, 5.12.

Results in the New York courts with other types of careers have been mixed.
\emph{Hougie v. Hougie}, 689 N.Y.S.2d 490 (App. Div. 1999) (applying
\textit{O'Brien} to investment banker's earning capacity); \emph{but see
Bystricky v. Bystricky}, 677 N.Y.S.2d 443 (Sup. Ct. 1998) (refusing to apply
\textit{O'Brien} to career police officer who passed the civil service exam and
was promoted to sergeant).

Which is better, the New York approach or the alternate approaches? How, if at
all, should courts account for educational attainment and career success in a
community property state, which mandates equal or equitable division of
community property and bars distribution of one spouse's separate property to
the other on divorce?

\item
\textbf{Other kinds of marital ``property.''} In one actual case, the husband
donated a kidney to his wife. When they later divorced, he argued that he
should be allowed to count the value of the kidney as part of her share of the
marital estate. Is he right?

What about human sperm, eggs, or embryos created by assisted reproduction
techniques? \textit{See} Steven H. Snyder, \emph{``I'm a Divorce Lawyer! So Why
Should I Read About ART?,''} 34 \textsc{Fam. Advoc.} 6 (Fall 2011) (``[o]ne in
five couples seeking a divorce has an assisted reproduction issue''; some courts
have treated embryos and stored sperm as property). As with professional
degrees, courts generally decline to treate gametes and the like as property,
but still account for them in some way in a divorce. \emph{Davis v. Davis}, 842
S.W.2d 588 (Tenn. 1992) (fertilized preembryos were not property, but
ex-husband's desire to avoid parenthood outweighed ex-wife's desire to donate
them to another couple, so he was allowed to control their disposition);
\emph{Kass v. Kass}, 696 N.E.2d 174 (N.Y. 1998) (upholding couple's agreement
that, if they no longer wanted to begin a pregnancy or couldn't agree on the
disposition of stored frozen pre-zygotes, the IVF program should dispose of them
by allowing them to be used in research); \emph{Hecht v. Superior Court}, 20
Cal. Rptr. 2d 275 (Ct. App. 1993) (decedent had right to leave his frozen sperm
to his female companion; his children and former spouse were not entitled to
have the sperm destroyed). Does the label ``property'' actually matter here?

\item
\textbf{Enforcement of divisions}. Once a relationship has broken down, things
can almost always get worse. Divorcing spouses may refuse to follow the rules
laid down by a court. As a result, an extensive legal framework has developed
to enforce these rules, with varying amounts of success. Courts may enter
support orders requiring an ex-spouse to provide a certain amount of monetary
support; if the ex-spouse doesn't pay voluntarily, a court may order
garnishment of his wages or attachment of her bank accounts. Civil and
criminal contempt, in which the contemnor is jailed for nonpayment, have even
been used to coerce payment. Wealthier ex-spouses often have an easier time
avoiding payment than poorer ones, because they may have an easier time
concealing bank accounts or arranging their affairs so there are no wages to
garnish.

\item
\textbf{Prenuptial agreements}. Lawyers sometimes get involved before things go
wrong, rather than after. Although they were initially resisted---at this point,
you should be able to make the basic arguments about the incompatibility of love
and hardhearted economic rationality---most states now accept them, at least
where there are no children involved and no undue influence. Still, many people
with significant assets enter marriage without a prenuptial agreement. Given the
benefits of prenuptial agreements for simply and cheaply dividing property at
divorce, why do so many people resist them? \textit{See} John B. Burns,
\emph{The Prenup as Estate Planning Tool or Trap?}, 33 \textsc{Fam. Advoc.} 7
(Winter 2011); Cheryl I. Foster, \emph{When a Prenup \& Religious Principles
Collide}, 33 \textsc{Fam. Advoc.} 7 (Winter 2011); Melvyn B. Frumkes, \emph{Why
a Prenuptial Agreement?}, 33 \textsc{Fam. Advoc.} 7 (Winter 2011).

Under the Uniform Premarital Agreement Act \S~6(a) (1983), adopted in more than
20 states, a premarital agreement isn't enforceable if the person against whom
enforcement is sought proves (1) that her agreement wasn't voluntary; or (2)
that the agreement was unconscionable when it was executed \textit{and} that
she (a) wasn't provided fair and reasonable disclosure of the other person's
property or financial obligations; (b) didn't voluntarily and expressly waive
her right to disclosure in writing; and (c) didn't have, and couldn't
reasonably have had, adequate knowledge of the other person's property or
financial obligations. \textit{See also Mamot v. Mamot}, 813 N.W.2d 440 (Neb.
2012) (finding agreement involuntary when fiance demanded it a few days before
the wedding, and fiancee could not reasonably have consulted a lawyer);
\textsc{Cal. Fam. Code} \S~1612(c) (restrictions on spousal support are allowed
only if the party waiving rights consulted with independent counsel).

Why shouldn't \textit{either} nondisclosure or unconscionability be sufficient
to invalidate a prenuptial agreement? A premarital agreement will also not be
enforced to the extent it would leave one spouse eligible for public
assistance, and child support rights may not be ``adversely affected'' by any
premarital agreement. After marriage, the agreement may be ended or changed
only by a written agreement signed by both parties, but no consideration is
required to end or change it.

Some states evaluate the agreements for fairness. \textit{See, e.g.}, \emph{Ansin v.
Craven-Ansin}, 929 N.E.2d 955 (Mass. 2010) (courts will uphold agreements that
are ``fair and reasonable''). The majority rule assesses unconscionability or
fairness as of the date the agreement was signed. \textit{See, e.g.}, \textsc{Va. Code}
\S~20-151; N.J. Stat. \S~37:2-38(c). But a not inconsiderable minority may
refuse to enforce an agreement that is unconscionable when enforcement is
sought, especially if the parties' circumstances have changed substantially.
\textit{See, e.g.}, \textsc{Conn. Gen. Stat.} \S~46b-36g(a)(2).

\item
What if the parties don't have a premarital agreement, but make an agreement
after marriage? In \textit{Borelli v. Brusseau}, 16 Cal. Rptr. 2d 16 (Ct. App.
1993), a stroke victim promised his wife that he'd leave her a significant
amount of his separate property, and pay for the education of her daughter by a
prior marriage, if she took care of him at home instead of putting him in
institutional care. After he died, his widow discovered that he had not
fulfilled his promise, and sued. The court refused to enforce the parties' oral
agreement because consideration was absent. The court reasoned that the marital
relationship already included a ``duty of support,'' which ``includes caring
for a spouse who is ill.'' The fact that the wife personally performed the
caretaking didn't constitute new consideration. Do you agree?

More generally, the trend of American law has been to allow people in intimate
relationships more freedom in selecting the property rules applied to them. No
state requires married people to hold by the entireties, or jointly.
Prenuptial agreements are regularly enforced. Legal forms developed to protect
the interests of same-sex couples, though now supplemented by the option of
marriage, remain available to people who choose not to marry. Careful planners
have choices even on death. \textit{See} Katherine D. Black et al.,
\emph{Community Property for Non-Community Property States}, 24
\textsc{Quinnipiac Prob. L.J.} 260 (2011) (describing ways in which a testator
may choose what state's property laws will control and how to enter into
community property agreements in non-community property jurisdictions).

