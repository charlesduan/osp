\reading{United States v. Craft}
\readingcite{535 U.S. 274 (2002)}


\opinion Justice \textsc{O'Connor} delivered the opinion of the Court.

This case raises the question whether a tenant by the entirety possesses
``property'' or ``rights to property'' to which a federal tax lien may attach.
Relying on the state law fiction that a tenant by the entirety has no separate
interest in entireties property, the United States Court of Appeals for the
Sixth Circuit held that such property is exempt from the tax lien. We conclude
that, despite the fiction, each tenant possesses individual rights in the
estate sufficient to constitute ``property'' or ``rights to property'' for the
purposes of the lien, and reverse the judgment of the Court of Appeals.

\readinghead{I}

In 1988, the Internal Revenue Service (IRS) assessed \$482,446 in unpaid income
tax liabilities against Don Craft, the husband of respondent Sandra L. Craft,
for failure to file federal income tax returns for the years 1979 through 1986.
When he failed to pay, a federal tax lien attached to ``all property and rights
to property, whether real or personal, belonging to'' him. 26 U.S.C. \S~6321.

At the time the lien attached, respondent and her husband owned a piece of real
property in Grand Rapids, Michigan, as tenants by the entirety. After notice of
the lien was filed, they jointly executed a quitclaim deed purporting to
transfer the husband's interest in the property to respondent for one dollar.
When respondent attempted to sell the property a few years later, a title
search revealed the lien. The IRS agreed to release the lien and allow the sale
with the stipulation that half of the net proceeds be held in escrow pending
determination of the Government's interest in the property.

Respondent brought this action to quiet title to the escrowed proceeds. The
Government claimed that its lien had attached to the husband's interest in the
tenancy by the entirety. It further asserted that the transfer of the property
to respondent was invalid as a fraud on creditors. The District Court granted
the Government's motion for summary judgment, holding that the federal tax lien
attached at the moment of the transfer to respondent, which terminated the
tenancy by the entirety and entitled the Government to one-half of the value of
the property.

Both parties appealed. The Sixth Circuit held that the tax lien did not attach
to the property because under Michigan state law, the husband had no separate
interest in property held as a tenant by the entirety. It remanded to the
District Court to consider the Government's alternative claim that the
conveyance should be set aside as fraudulent.

On remand, the District Court concluded that where, as here, state law makes
property exempt from the claims of creditors, no fraudulent conveyance can
occur. It found, however, that respondent's husband's use of nonexempt funds to
pay the mortgage on the entireties property, which placed them beyond the reach
of creditors, constituted a fraudulent act under state law, and the court
awarded the IRS a share of the proceeds of the sale of the property equal to
that amount.\dots

We granted certiorari to consider the Government's claim that respondent's
husband had a separate interest in the entireties property to which the federal
tax lien attached.

\readinghead{II}

Whether the interests of respondent's husband in the property he held as a
tenant by the entirety constitutes ``property and rights to property'' for the
purposes of the federal tax lien statute, is ultimately a question of federal
law. The answer to this federal question, however, largely depends upon state
law. The federal tax lien statute itself ``creates no property rights but
merely attaches consequences, federally defined, to rights created under state
law.'' Accordingly, ``[w]e look initially to state law to determine what rights
the taxpayer has in the property the Government seeks to reach, then to federal
law to determine whether the taxpayer's state-delineated rights qualify as
`property' or `rights to property' within the compass of the federal tax lien
legislation.''

A common idiom describes property as a ``bundle of sticks''---a collection of
individual rights which, in certain combinations, constitute property. State
law determines only which sticks are in a person's bundle. Whether those sticks
qualify as ``property'' for purposes of the federal tax lien statute is a
question of federal law.

In looking to state law, we must be careful to consider the substance of the
rights state law provides, not merely the labels the State gives these rights
or the conclusions it draws from them. Such state law labels are irrelevant to
the federal question of which bundles of rights constitute property that may be
attached by a federal tax lien. In \emph{Drye v. United States}, 528 U.S. 49
(1999), we considered a situation where state law allowed an heir subject to a
federal tax lien to disclaim his interest in the estate. The state law also
provided that such a disclaimer would ``creat[e] the legal fiction'' that the
heir had predeceased the decedent and would correspondingly be deemed to have
had no property interest in the estate. We unanimously held that this state law
fiction did not control the federal question and looked instead to the realities
of the heir's interest. We concluded that, despite the State's characterization,
the heir possessed a ``right to property'' in the estate---the right to accept
the inheritance or pass it along to another---to which the federal lien could
attach.

\readinghead{III}

We turn first to the question of what rights respondent's husband had in the
entireties property by virtue of state law.\dots

In determining whether respondent's husband possessed ``property'' or ``rights
to property'' within the meaning of 26 U.S.C. \S~6321, we look to the
individual rights created by these state law rules. According to Michigan law,
respondent's husband had, among other rights, the following rights with respect
to the entireties property: the right to use the property, the right to exclude
third parties from it, the right to a share of income produced from it, the
right of survivorship, the right to become a tenant in common with equal shares
upon divorce, the right to sell the property with the respondent's consent and
to receive half the proceeds from such a sale, the right to place an
encumbrance on the property with the respondent's consent, and the right to
block respondent from selling or encumbering the property unilaterally.

\readinghead{IV}

We turn now to the federal question of whether the rights Michigan law granted
to respondent's husband as a tenant by the entirety qualify as ``property'' or
``rights to property'' under \S~6321. The statutory language authorizing the
tax lien ``is broad and reveals on its face that Congress meant to reach every
interest in property that a taxpayer might have.'' ``Stronger language could
hardly have been selected to reveal a purpose to assure the collection of
taxes.'' We conclude that the husband's rights in the entireties property fall
within this broad statutory language.

Michigan law grants a tenant by the entirety some of the most essential property
rights: the right to use the property, to receive income produced by it, and to
exclude others from it. \emph{See Dolan v. City of Tigard}, 512 U.S. 374, 384
(1994) (``[T]he right to exclude others'' is ``\,`one of the most essential
sticks in the bundle of rights that are commonly characterized as
property'\,''). These rights alone may be sufficient to subject the husband's
interest in the entireties property to the federal tax lien. They gave him a
substantial degree of control over the entireties property, and, as we noted in
Drye, ``in determining whether a federal taxpayer's state-law rights constitute
`property' or `rights to property,' [t]he important consideration is the breadth
of the control the [taxpayer] could exercise over the property.''

The husband's rights in the estate, however, went beyond use, exclusion, and
income. He also possessed the right to alienate (or otherwise encumber) the
property with the consent of respondent, his wife. It is true, as respondent
notes, that he lacked the right to unilaterally alienate the property, a right
that is often in the bundle of property rights. There is no reason to believe,
however, that this one stick---the right of unilateral alienation---is
essential to the category of ``property.''\dots

Excluding property from a federal tax lien simply because the taxpayer does not
have the power to unilaterally alienate it would, moreover, exempt a rather
large amount of what is commonly thought of as property.\dots Community
property States often provide that real community property cannot be alienated
without the consent of both spouses. Accordingly, the fact that respondent's
husband could not unilaterally alienate the property does not preclude him from
possessing ``property and rights to property'' for the purposes of \S~6321.

Respondent's husband also possessed the right of survivorship---the right to
automatically inherit the whole of the estate should his wife predecease him.
Respondent argues that this interest was merely an expectancy, which we
suggested in \emph{Drye} would not constitute ``property'' for the purposes of a
federal tax lien. 528 U.S., at 60, n. 7 (``[We do not mean to suggest] that an
expectancy that has pecuniary value\dots would fall within \S~6321 prior
to the time it ripens into a present estate''). \emph{Drye} did not decide this
question, however, nor do we need to do so here. As we have discussed above, a
number of the sticks in respondent's husband's bundle were presently existing.
It is therefore not necessary to decide whether the right to survivorship alone
would qualify as ``property'' or ``rights to property'' under \S~6321.

That the rights of respondent's husband in the entireties property constitute
``property'' or ``rights to property'' ``belonging to'' him is further
underscored by the fact that, if the conclusion were otherwise, the entireties
property would belong to no one for the purposes of \S~6321. Respondent had
no more interest in the property than her husband; if neither of them had a
property interest in the entireties property, who did? This result not only
seems absurd, but would also allow spouses to shield their property from
federal taxation by classifying it as entireties property, facilitating abuse
of the federal tax system.

Justice \textsc{Scalia}'s and Justice \textsc{Thomas}' dissents claim that the
conclusion that the
husband possessed an interest in the entireties property to which the federal
tax lien could attach is in conflict with the rules for tax liens relating to
partnership property. This is not so. As the authorities cited by Justice
\textsc{Thomas} reflect, the federal tax lien does attach to an individual
partner's interest in the partnership, that is, to the fair market value of his
or her share in the partnership assets. As a holder of this lien, the Federal
Government is entitled to ``receive \dots the profits to which the assigning
partner would otherwise be entitled,'' including predissolution distributions
and the proceeds from dissolution.\ldots

There is, however, a difference between the treatment of entireties property and
partnership assets. The Federal Government may not compel the sale of
partnership assets (although it may foreclose on the partner's interest). It is
this difference that is reflected in Justice \textsc{Scalia}'s assertion that
partnership property cannot be encumbered by an individual partner's debts. This
disparity in treatment between the two forms of ownership, however, arises from
our decision in \emph{United States v. Rodgers}, [416 U.S. 677 (1983)] (holding
that the
Government may foreclose on property even where the co-owners lack the right of
unilateral alienation), and not our holding today. In this case, it is instead
the dissenters' theory that departs from partnership law, as it would hold that
the Federal Government's lien does not attach to the husband's interest in the
entireties property at all, whereas the lien may attach to an individual's
interest in partnership property\dots.

We therefore conclude that respondent's husband's interest in the entireties
property constituted ``property'' or ``rights to property'' for the purposes of
the federal tax lien statute. We recognize that Michigan makes a different
choice with respect to state law creditors: ``[L]and held by husband and wife
as tenants by entirety is not subject to levy under execution on judgment
rendered against either husband or wife alone.'' But that by no means dictates
our choice. The interpretation of 26 U.S.C. \S~6321 is a federal question,
and in answering that question we are in no way bound by state courts' answers
to similar questions involving state law. As we elsewhere have held,
``\,`exempt status under state law does not bind the federal
collector.'\,''\dots

\opinion Justice \textsc{Scalia}, with whom Justice \textsc{Thomas} joins,
dissenting.

\dots I write separately to observe that the Court nullifies (insofar as
federal taxes are concerned, at least) a form of property ownership that was of
particular benefit to the stay-at-home spouse or mother. She is overwhelmingly
likely to be the survivor that obtains title to the unencumbered property; and
she (as opposed to her business-world husband) is overwhelmingly unlikely to be
the source of the individual indebtedness against which a tenancy by the
entirety protects. It is regrettable that the Court has eliminated a large part
of this traditional protection retained by many States.

\opinion Justice \textsc{Thomas}, with whom Justice \textsc{Stevens} and Justice
\textsc{Scalia} join, dissenting.

\dots The Court does not contest that the tax liability the IRS seeks to
satisfy is Mr. Craft's alone, and does not claim that, under Michigan law, real
property held as a tenancy by the entirety belongs to either spouse
individually. Nor does the Court suggest that the federal tax lien attaches to
particular ``rights to property'' held individually by Mr. Craft. Rather,
borrowing the metaphor of ``property as a `bundle of sticks'---a collection of
individual rights which, in certain combinations constitute property,'' the
Court proposes that so long as sufficient ``sticks'' in the bundle of ``rights
to property'' ``belong to'' a delinquent taxpayer, the lien can attach as if
the property itself belonged to the taxpayer.

This amorphous construct ignores the primacy of state law in defining property
interests \dots.

\readinghead{I}

Title 26 U.S.C. \S~6321 provides that a federal tax lien attaches to ``all
property and rights to property, whether real or personal, belonging to'' a
delinquent taxpayer. It is uncontested that a federal tax lien itself ``creates
no property rights but merely attaches consequences, federally defined, to
rights created under state law.'' Consequently, the Government's lien under
\S~6321 ``cannot extend beyond the property interests held by the delinquent
taxpayer,'' under state law\dots.

\readinghead{A}

\dots As the Court recognizes, pursuant to Michigan law, as under English
common law, property held as a tenancy by the entirety does not belong to
either spouse, but to a single entity composed of the married persons. Neither
spouse has ``any separate interest in such an estate.'' An entireties estate
constitutes an indivisible ``sole tenancy.'' Because Michigan does not
recognize a separate spousal interest in the Grand Rapids property, it did not
``belong'' to either respondent or her husband individually when the IRS
asserted its lien for Mr. Craft's individual tax liability. Thus, the property
was not property to which the federal tax lien could attach for Mr. Craft's tax
liability.

\textit{Drye}\dots was concerned not with whether state law recognized
``property'' as belonging to the taxpayer in the first place, but rather with
whether state laws could disclaim or exempt such property from federal tax
liability after the property interest was created. \textit{Drye} held only
that a state-law disclaimer could not retroactively undo a vested right in an
estate that the taxpayer already held, and that a federal lien therefore
attached to the taxpayer's interest in the estate. 528 U.S., at 61 (recognizing
that a disclaimer does not restore the status quo ante because the heir
``determines who will receive the property---himself if he does not disclaim,
a known other if he does'').\dots

\readinghead{B}

\dots Rather than adopt the majority's approach, I would ask specifically, as
the statute does, whether Mr. Craft had any particular ``rights to property''
to which the federal tax lien could attach. He did
not.\readingfootnote{5}{Even such
rights as Mr. Craft arguably had in the Grand Rapids property bear no
resemblance to those to which a federal tax lien has ever attached. See W.
Elliott, \emph{Federal Tax Collections, Liens, and Levies} \P\P~9.09[3][a]--[f]
(2d ed. 1995 and 2000 Cum. Supp.) (listing examples of rights to property to
which a federal tax lien attaches, such as the right to compel payment; the
right to withdraw money from a bank account, or to receive money from accounts
receivable; wages earned but not paid; installment payments under a contract of
sale of real estate; annuity payments; a beneficiary's rights to payment under
a spendthrift trust; a liquor license; an easement; the taxpayer's interest in
a timeshare; options; the taxpayer's interest in an employee benefit plan or
individual retirement account).}\dots With such rights subject to lien, the
taxpayer's interest has ``ripen[ed] into a present estate'' of some form and is
more than a mere expectancy, and thus the taxpayer has an apparent right ``to
channel that value to [another].''

In contrast, a tenant in a tenancy by the entirety not only lacks a present
divisible vested interest in the property and control with respect to the sale,
encumbrance, and transfer of the property, but also does not possess the
ability to devise any portion of the property because it is subject to the
other's indestructible right of survivorship. This latter fact makes the
property significantly different from community property, where each spouse has
a present one-half vested interest in the whole, which may be devised by will
or otherwise to a person other than the spouse. See 4 G. Thompson, \emph{Real
Property} \S~37.14(a) (D. Thomas ed. 1994) (noting that a married person's
power to devise one-half of the community property is ``consistent with the
fundamental characteristic of community property'': ``community ownership means
that each spouse owns 50\% of each community asset'').

It is clear that some of the individual rights of a tenant in entireties
property are primarily personal, dependent upon the taxpayer's status as a
spouse, and similarly not susceptible to a tax lien. For example, the right to
use the property in conjunction with one's spouse and to exclude all others
appears particularly ill suited to being transferred to another, and to lack
``exchangeable value.''

Nor do other identified rights rise to the level of ``rights to property'' to
which a \S~6321 lien can attach, because they represent, at most, a
contingent future interest, or an ``expectancy'' that has not ``ripen[ed] into
a present estate.'' By way of example, the survivorship right wholly depends
upon one spouse outliving the other, at which time the survivor gains
``substantial rights, in respect of the property, theretofore never enjoyed by
[the] survivor.''\dots

Similarly, while one spouse might escape the absolute limitations on individual
action with respect to tenancy by the entirety property by obtaining the right
to one-half of the property upon divorce, or by agreeing with the other spouse
to sever the tenancy by the entirety, neither instance is an event of
sufficient certainty to constitute a ``right to property'' for purposes of
\S~6321. Finally, while the federal tax lien could arguably have attached to a
tenant's right to any ``rents, products, income, or profits'' of real property
held as tenants by the entirety, the Grand Rapids property created no rents,
products, income, or profits for the tax lien to attach to\dots.

Ownership by ``the marriage'' is admittedly a fiction of sorts, but so is a
partnership or corporation. There is no basis for ignoring this fiction so long
as federal law does not define property, particularly since the tenancy by the
entirety property remains subject to lien for the tax liability of both
tenants\dots.

