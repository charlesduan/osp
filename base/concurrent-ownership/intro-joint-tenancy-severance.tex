\term[severance]{Severance} is any act that destroys one or more of the four
unities required to
maintain a joint tenancy. The legal consequence of severance is that the joint
tenancy is converted to a tenancy in common. (For those rare joint tenancies
involving three or more joint tenants, one joint tenant may sever the joint
tenancy as to his interest, but the others remain joint tenants with each
other.) The traditional rule for severance required either that all the tenants
expressly agree to hold as tenants in common, or that one of the tenants convey
to a third person in order to destroy the unities (particularly the unities of
time and title), to turn a joint tenancy into a tenancy in common. In modern
times, a conveyance from oneself as joint tenant to oneself as tenant in common
is likely to succeed just as well as a conveyance by one tenant to a straw owner
plus a reconveyance from the straw. \textit{See} \emph{Hendrickson v.
Minneapolis Fed. Sav. \& Loan Ass'n}, 161 N.W.2d 688 (Minn. 1968); \emph{Riddle
v. Harmon}, 162 Cal. Rptr. 530 (Cal. Ct. App. 1980); \textit{see also}
\emph{Countrywide Funding Corp. v. Palmer}, 589 So. 2d 994 (Fla. Dist. Ct. App.
1991) (one joint tenant forged the other's signature in purported conveyance to
himself; court held that his act severed the tenancy). \textit{But see}
\emph{Krause v. Crossley}, 277 N.W.2d 242 (Neb. 1979) (rejecting this modern
trend and requiring conveyance to a third party for an effective severance);
L.B. 694, \S~11, 1980 \textsc{Neb. Laws} 577 (codified as \textsc{Neb. Rev.
Stat.} \S~76-118(4) (Reissue 1996)) (reversing result in \textit{Krause} and
allowing self-conveyance to sever).

The largest problem in severance is one of surprise, which can occur whether or
not a third party straw is required to partipate in the severance. As Helmholz
explains:
\begin{quote}
Since one joint tenant has always been able to sever the tenancy without the
concurrence or even the knowledge of the other, the possibility of a severance
that is unfair to the other has long existed. It can take several forms, as
where the joint tenant who has contributed nothing to the purchase of the
assets then severs unilaterally, thereby upsetting the normal expectations of
the other joint tenant. Its most extreme form is the secret severance. If the
tenant who severs secretly is the first to die, the heirs or successors produce
the severing document and take half of the property. It accrues to them under
the tenancy in common that was the result of the severance. If the severing
tenant survives, however, the severing document is suppressed and the survivor
takes the whole. The heirs or successors of the first to die get nothing. It is
what the economists call ``strategic behavior.''
\end{quote}
Helmholz, \textit{supra}, at 25-26.

Why not impose a notice requirement for a deliberate severance? What about
imposing a requirement that a severing instrument be timely recorded in the
public land records? \emph{See} \textsc{Cal. Civ. Code} \S~683.2 (West 1998) (if
a joint tenancy is recorded, severance is only effective against the
non-severing tenant if the severance is recorded either before the severing
tenant's death or, in limited circumstances, recorded within seven days after
death; the severing tenant's right of survivorship is cut off even without
recording); \textsc{Minn. Stat. Ann.} \S~500.19--5 (West 1997) (requiring
recording to make unilateral severance valid); \textsc{N.Y. Real Prop. Law}
\S~240-c(2) (similar). Does a recording requirement solve the problem of
surprise?

Joint tenants may also take acts that are more ambiguous with respect to their
rights. Courts then have to decide what kinds of acts are sufficient to work a
severance.

