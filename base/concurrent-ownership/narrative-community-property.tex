Nine states, representing roughly 30\% of the population of the U.S., recognize
community property for married people: Arizona, California, Idaho, Louisiana,
Nevada, New Mexico, Texas, Washington, and Wisconsin. Under community property
regimes, marital property belongs to each spouse equally. Each spouse has a
right to pass on his or her share to anyone by will, making community property
different from joint tenancy; however, it is also possible to hold community
property with a right of survivorship, highly similar to joint tenancy. In the
absence of a right of survivorship, a surviving spouse is typically entitled to
some of the community property when the other spouse dies intestate; his or her
share generally depends on whether there are surviving issue (children and
other descendants), and how many there are.

The basic idea of community property is that a marriage is a cooperative
endeavor, and each spouse contributes to gains, whether directly or indirectly.
Except for Alaska, which requires an explicit agreement, \textsc{Alaska Stat.} \S~34.77.090 (2002), the default rule under a community property regime is that
property earned by a spouse during marriage belongs to the marital community,
and each spouse owns half of the community property as an equal undivided
interest. This includes property purchased with income earned during the
marriage. This contrasts to common law states, in which property belongs by
default to the spouse who acquires it during the marriage.

Property owned before marriage, as well as property acquired by inheritance or
gift during the marriage, remains separate property in most states. States are
divided about whether and when income from separate property, such as interest,
royalties, and rent, becomes part of the community property. Idaho, Louisiana,
Texas and Wisconsin treat the income from all property as community property,
while the other states allow such income to remain separate property.
Classification may prove complicated: for example, is an award of damages from
a bike accident involving one spouse community property? The answer may depend
on whether the award represents economic harm such as lost earnings (community
property) or pain and suffering (separate property). What if the award is for
loss of a limb, which has both earnings-related and quality of life-related
aspects? What if the award is for loss of consortium---the caretaking and
intimate relations shared between spouses?

In general, spouses are free to take property as separate property by agreement,
and to convert property from one regime to the other by agreement. If
community and separate property are commingled, tracing the shares may prove
very difficult, and the party with the burden of showing that the property is
separate may have a hard time prevailing. Carefully kept records may allow a
tracing spouse to overcome the presumption that assets held during marriage are
community property. Under the ``family expense presumption,'' family expenses
are presumed to have come from community assets in a commingled account. If
such expenses exceeded deposits of community funds, the balance will be
separate property. \emph{See v. See}, 415 P.2d 776 (Cal. 1966). As for
outstanding debt paid off in part with community property, California apportions
community and separate property according to the contributions made. Thus, a
person who has a house subject to a mortgage before she marries, and then pays
the remainder of the mortgage with money earned during marriage, will own the
house partly as separate property and partly as community property. Other states
use an ``inception'' theory and consider the house entirely separate property
because the purchase was made before the marriage. And other states use a
``vesting'' theory and consider the house entirely community property because
title didn't vest until the mortgage was paid off.

In most cases, either spouse may manage community property. However, if title
is in only one spouse's name, that spouse may be the only one who can manage
the property. In addition, a spouse who runs a business that is community
property may have exclusive control. The controlling spouse has a kind of
fiduciary duty: she must act in good faith towards her spouse, but she is not
required to act with good judgment. Transferring or mortgaging community
property, unlike day-to-day management, requires the consent of both spouses in
a number of community property states, though not all. \textit{See} J. Thomas
Oldham, \textit{Management of the Community Estate During an Intact Marriage},
56 \textsc{L. \& Contemp. Probs.} 99 (1993). The fact that a deed says that
property is separate property is not controlling, because the law prevents a
spouse from converting community property to separate property unilaterally.
In some states, such as Texas, the controlling spouse can make reasonable gifts
of community property, while California and Washington allow any gift by the
managing spouse to be set aside by the other spouse. In most states, a bona
fide purchaser from any managing spouse is protected against invalidation of
the sale.

In some states, creditors can reach whatever property a spouse is entitled to
manage. If the spouses share the family car, for example, then a creditor of
either spouse could seize the car to satisfy one spouse's debt (after following
the appropriate procedures). Others only allow creditors to reach community
property if both spouses consented to the relevant debt, and others limit the
amount of community property creditors of only one spouse can reach.

A spouse may dispose of half of the community property at his or her death.
There is no right of survivorship, but the other half belongs to the survivor.
The decedent can allocate the property however she wants in a will; if there is
no will, then some community property states make the other spouse the heir,
while others give the decedent's issue priority.

There is no such thing as a tenancy by the entireties in a community property
state; there can be joint tenancy or tenancy in common, but property held in
those forms is separate property. Like a tenancy by the entireties, community
property can only exist between married people. Moreover, neither spouse alone
can convey his or her undivided share to another person, except to the other
spouse. Community property is not subject to partition. Without agreement,
the spouse's only option to separate the couple's undivided interests is
divorce, which will result in an equal or ``equitable'' division of community
property, depending on the state. California, New Mexico, and Louisiana divide
community property and debts equally,\footnote{In the absence of agreement to
the contrary or deliberate misappropriation of community property by one
spouse.} while courts use the more flexible equitable division in the other
community property states. In California, absent a written agreement to the
contrary, a spouse who contributes separate property to acquiring community
property must be reimbursed for the contribution at divorce, though the spouse
can't get interest or an adjustment for a change in the value of the property,
and the reimbursement can't exceed the net value of the property at the time
the property was acquired. \textsc{Cal. Family Code} \S~2640(b). Can you see why
the legislature felt it necessary to impose the net value cap? What kind of
unsavory activities might result if the rule were different?

If a married couple moves to a non-community property state, community property
retains its character, which can lead to some complicated situations.

A family law course will cover the significant differences between community
property and joint tenancy in more detail, including tax implications. The
regimes reward careful planning, especially for people with substantial assets.
\textit{See} Andrea B. Carroll, \textit{Incentivizing Divorce}, 30
\textsc{Cardozo L. Rev.} 1925 (2009) (arguing that marital property rules,
particularly in community property states, create perverse incentives toward
divorce).

