\expected{posik-v-layton}

\item
The court's reasoning at some points suggests that, because the parties could
not legally marry even if they wanted to, they should in fairness be allowed to
recreate by agreement as many of the rights and obligations of married couples
as possible. Now that lesbian couples can marry at will, is the basis of the
decision undermined? If it's possible to recreate the legal incidents of
marriage without marriage, won't that undermine the state's preference for
marriage when couples live together, which was an important aspect of
\emph{Obergefell v. Hodges}, 2015 WL 213646 (2015), the Supreme Court decision
holding that same-sex marriage is a constitutional right?

\item What if there is no explicit agreement? As it became more common for
unmarried couples to live together without holding themselves out as married,
lawyers searched for theories that could replace common law marriage when a
long-standing nonmarital relationship ended. The Supreme Court of California,
relying on a theory of unjust enrichment, held that a contract for property
division or support could be implied from the conduct of the parties.
\emph{Marvin v. Marvin}, 557 P.2d 106 (Cal. 1976). A number of states have
followed \textit{Marvin}, at least in part. \emph{Glasgo v. Glasgo}, 410 N.E.2d
1325 (Ind. Ct. App. 1980) (court may divide property based on contract and
equitable principles); \emph{Carroll v. Lee}, 712 P.2d 923 (Ariz. 1986) (implied
contract); \emph{In re Marriage of Lindsey}, 678 P.2d 328 (Wash. 1984)
(equitable principles); \emph{Kozlowski v. Kozlowski}, 403 A.2d 902 (N.J. 1979);
\emph{Goode v. Goode}, 396 S.E.2d 430 (W. Va. 1990) (express or implied
contract, or constructive trust); \emph{Watts v. Watts}, 405 N.W.2d 303 (Wis.
1987) (allowing courts to divide property between unmarried people who lived
together under theories such as breach of contract, constructive trust, and
quantum meruit).

\item
New York requires a written or oral express contract to share earnings and
assets between unmarried partners, but will enforce express contracts.
\emph{Morone v. Morone}, 413 N.E.2d 1154 (N.Y. 1980). Likewise, \emph{In re
Estate of Roccamonte},
808 A.2d 838 (N.J. 2002), enforced a man's promise to provide for a woman
``financially for the rest of her life,'' even though he was married to another
woman when he made that promise. The court directed the trial court to award
her a lump sum payment based on her life expectancy, diminishing the amount
inherited by his wife and children. Roccamonte's estate was valued at \$1.4
million, and the claimant was entitled to receive \$450,000 from the estate
based on her life expectancy. The New Jersey legislature overturned the
result, providing that all such ``palimony'' agreements must be in writing to
be enforceable.

\item
Was \textit{In re Estate of Roccamonte} a step too far, given the indefinite
language of the promise? If we construed the decedent's words as a promise to
make a will, that promise would be unenforceable, at least absent estoppel.
\emph{Williams v. Mason}, 556 So. 2d 1045 (Miss. 1990) (refusing to enforce an oral
promise to devise property to his cohabitant in return for her promise to live
in his home and ``do his bidding''). The \textit{Williams} court wrote:
\begin{quotation}
Though a party may satisfy the court of the existence of an unwritten agreement
to devise, the statute precludes specific performance as a remedy our courts
may decree. This is so even though the promisee has done all he was expected to
do under the agreement. Holmes put the point well a century ago in Bourke v.
Callahan, 160 Mass. 195, 35 N.E. 460 (1893):
\begin{quote}
\dots [T]he statute of frauds may be made an instrument of fraud. But this is
always true, whenever the law prescribes a form for an obligation. The very
meaning of such a requirement is that a man relies at his peril on what
purports to be such an obligation without that form\dots.
\end{quote}
Notwithstanding these well settled principles [that contracts or promises to
make a will are unenforceable], experience has taught that gross unfairness may
result where one acts in good faith and lives up to an oral agreement to
provide services for another under circumstances such as today's. Our law has
seen in such situations a potential for unjust enrichment, if not fraud. In
recognition of these practical realities, the positive law of this state
directs that a person, who provides services to another in good faith and in
consequence of an oral agreement to devise property in exchange for the
services, is not without enforceable rights. These rights arise not out of the
agreement but the conduct of the parties\dots.

When the parties have so acted with respect one to the other, that is, when one
has provided services for the other in reasonable reliance upon a promise to
give consideration therefor, our cases are legion that, upon the death of the
promisor, the promisee may recover of and from the estate on a quantum meruit
basis. In such cases the amount of recovery is limited to the monetary
equivalent of the reasonable value of the services rendered to the decedent for
which payment has not been received\dots.

Our law recognizes an additional basis upon which---assuming proper proof---a
person such as Mason may recover. Where parties live together without
benefit of marriage and where, through their joint efforts, accumulate real
property or personal property, or both, a party having no legal title
nevertheless acquires rights to an equitable share enforceable at law\dots.

In so holding, we well realize that we hold enforceable rights predicated upon
the conduct of the parties but unattended by any writing. Although neither the
statute of frauds nor the statute of wills per se preclude quantum meruit
recovery in such circumstances, we are not unaware that the policy
considerations supporting the existence and enforcement of those statutes may
be present nevertheless. Because the decedent is not available to provide his
version of the matter, courts must view with a touch of skepticism claims for
services rendered asserted only at death. We have in the past suggested that
the party alleging such an agreement must prove its existence by something more
than the ordinary preponderance of the evidence.\dots
\end{quotation}

Is this a fair compromise of the relevant interests? Can Carol Rose's theory of
crystals and mud in property law, and the courts' solicitude for dupes and sad
sacks, help explain the results in this area of the law?

\item
The ALI's Principles of the Law of Family Dissolution also deals with domestic
partners, hodling that obligations may arise from the parties' conduct, even
without a formal agreement. \S~6.02. Under the ALI approach, if domestic
partners share a primary residence and a life together as a couple for a
significant period of time, the couple's property should be divided as if they
were married. If one partner dies, however, the survivor's rights depend on the
decedent's will or, if there is no will, the state's law of intestate
succession.

\item
A minority of states still refuses to recognize legal obligations growing from
extended cohabitation. \emph{Hewitt v. Hewitt}, 394 N.E.2d 1204 (Ill. 1979)
(holding that to recognize mutual property rights in unmarried cohabitants under
a contract theory would contravene the state's policy of strengthening and
preserving the integrity of marriage, because cohabitation was unlawful and
state refused to recognize common-law marriage); \textit{see also Long v.
Marino}, 441 S.E.2d 475 (Ga. Ct. App. 1994) (rejecting a breach of implied
contract claim against a Catholic archbishop; stating that ``[m]eretricious
sexual relationships are by nature repugnant to social stability, and our courts
have on sound public policy declined to reward them by allowing a money recovery
therefor''); \emph{Tapley v. Tapley}, 449 A.2d 1218 (N.H. 1982) (rejecting claim
because personal services are frequently provided by two people living together
because they value each other, not because of an agreement to pay); Marsha
Garrison, \emph{Nonmarital Cohabitation: Social Revolution and Legal
Regulation}, 42 \textsc{Fam. L.Q.} 309 (2008). Even in states that recognize
some form of ``palimony,'' an agreement that includes sexual services may be
unenforceable. \textit{See} \emph{Kastil v. Carro}, 536 N.Y.S.2d 63 (App. Div.
1988) (breach of oral contract claim failed because illicit sexual relations
can't provide consideration for a contract).

\item
Given that making cohabitation illegal is clearly unconstitutional under current
law, \textit{see} \emph{Lawrence v. Texas}, 539 U.S. 558 (2003), should the
result in \textit{Hewitt} remain the same on the theory that it is still
acceptable for law to \textit{prefer} marriage to cohabitation and thus to give
legal incentives to cohabitants to marry? \textit{Hewitt} has been strongly
criticized. Among other things, other courts have understood that their refusal
to enforce promises that seem credible to the parties may cause great
injustice, and won't in fact discourage other people from living together.
Also, at least one person in most couples would prefer the \textit{Hewitt} rule
and thus experience it as a \textit{disincentive} to marry.

\item Ultimately, should marriage matter to property law, and if so, in what
ways? There is an extremely large literature on these subjects. \textit{See,
e.g.}, \textsc{Martha Ertman, Love's Promises: How Formal and Informal Contracts
Shape All Kinds of Families} (2015); Charlotte K. Goldberg, \emph{Opting In,
Opting Out: Autonomy in the Community Property States}, 72 \textsc{La. L. Rev.}
1 (2011) (discussing formal and informal mechanisms to avoid community property
rules, such as cohabiting without marriage or creating a prenuptial agreement);
Shahar Lifshitz, \emph{Married Against Their Will? Toward a Pluralist Regulation
of Spousal Relationships}, 66 \textsc{Wash. \& Lee L. Rev.} 1565 (2009) (arguing
against the equalization of obligations between married and cohabiting couples);
Candace Saari Kovacic-Fleischer, \emph{Cohabitation and the Restatement (Third)
of Restitution \& Unjust Enrichment}, 68 \textsc{Wash. \& Lee L. Rev.} 1407
(2011) (addressing claims of cohabitants to property that is acquired jointly
and exploring the equitable bases on which courts apportion such property).

\item
\textbf{Is there anything special about romantic couples?} Suppose an adult
child supports a parent for several decades after the parent loses her job and
decides not to seek another, based on his belief that the parent will leave her
estate to the child. If the parent disinherits the child, would he have any
claim for reimbursement against the parent's estate? What if the parent
promised, orally or in writing, to leave her estate to the child? \textit{See}
\textsc{Hendrik Hartog, Someday All This Will Be Yours: A History of Inheritance
and Old Age} (2012) (discussing numerous lawsuits throughout American history
following this pattern and finding that young men who gave up potential careers
to help their parents often succeeded in their claims, while young women who
gave up potential marriages to do the same failed).

