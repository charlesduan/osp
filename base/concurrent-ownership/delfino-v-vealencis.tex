\reading{Delfino v. Vealencis}
\readingcite{436 A.2d 27 (Conn. 1980)}

\opinion \textsc{Arthur H. Healey}, Associate Justice.

The central issue in this appeal is whether the Superior Court properly ordered
the sale, pursuant to General Statutes \S~52-500,\readingfootnote{1}{General
Statutes \S~52-500 states: ``Sale of Real or Personal Property Owned by Two or
More. Any court of equitable jurisdiction may, upon the complaint of any person
interested, order the sale of any estate, real or personal, owned by two or more
persons, when, in the opinion of the court, a sale will better promote the
interests of the owners.\dots A conveyance made in pursuance of a decree
ordering a sale of such land shall vest the title in the purchaser thereof, and
shall bind the person entitled to the life estate and his legal heirs and any
other person having a remainder interest in the lands; but the court passing
such decree shall make such order in relation to the investment of the avails of
such sale as it deems necessary for the security of all persons having any
interest in such land.''} of property owned by the plaintiffs and the defendant
as tenants in common.

The plaintiffs, Angelo and William Delfino, and the defendant, Helen C.
Vealencis, own, as tenants in common, real property located in Bristol,
Connecticut. The property consists of an approximately 20.5 acre parcel of land
and the dwelling of the defendant thereon. The plaintiffs own an undivided
99/144 interest in the property, and the defendant owns a 45/144 interest. The
defendant occupies the dwelling and a portion of the land, from which she
operates a rubbish and garbage removal
business.\readingfootnote{3}{The defendant's business
functions on the property consist of the overnight parking, repair and storage
of trucks, including refuse trucks, the repair, storage and cleaning of
dumpsters, the storage of tools, and general office work. No refuse is actually
deposited on the property.} Apparently, none of the parties is in actual
possession of the remainder of the property. The plaintiffs, one of whom is a
residential developer, propose to develop the property, upon partition, into
forty-five residential building lots.

\captionedgraphic[height=0.4\textheight]{concurrent-1}{Subdivision plot plan for
the 20.5 acre parcel}

In 1978, the plaintiffs brought an action in the trial court seeking a partition
of the property by sale with a division of the proceeds according to the
parties' respective interests. The defendant moved for a judgment of in-kind
partition and the appointment of a committee to conduct said partition. The
trial court, after a hearing, concluded that a partition in kind could not be
had without ``material injury'' to the respective rights of the parties, and
therefore ordered that the property be sold at auction by a committee and that
the proceeds be paid into the court for distribution to the parties.

On appeal, the defendant claims essentially that the trial court's conclusion
that the parties' interests would best be served by a partition by sale is not
supported by the findings of subordinate facts, and that the court improperly
considered certain factors in arriving at that conclusion. In addition, the
defendant directs a claim of error to the court's failure to include in its
findings of fact a paragraph of her draft findings.

General Statutes \S~52-495 authorizes courts of equitable jurisdiction to order,
upon the complaint of any interested person, the physical partition of any real
estate held by tenants in common, and to appoint a committee for that
purpose.\readingfootnote{7}{If the physical partition results
in unequal shares, a money award can be made from one tenant to another to
equalize the shares.} When, however, in the opinion of the court a sale of the
jointly owned property ``will better promote the interests of the owners,'' the
court may order such a sale under \S~52-500.

It has long been the policy of this court, as well as other courts, to favor a
partition in kind over a partition by sale.\dots Due to the possible
impracticality of actual division, this state, like others, expanded the right
to partition to allow a partition by sale under certain circumstances. The
early decisions of this court that considered the partition-by-sale statute
emphasized that ``(t)he statute giving the power of sale introduces\dots no
new principles; it provides only for an emergency, when a division cannot be
well made, in any other way. The court later expressed its reason for
preferring partition in kind when it stated: ``(A) sale of one's property
without his consent is an extreme exercise of power warranted only in clear
cases.'' \emph{Ford v. Kirk}, 41 Conn. 9, 12 (1874). Although under General
Statutes \S~52-500 a court is no longer required to order a partition in kind
even in cases
of extreme difficulty or hardship; it is clear that a partition by sale should
be ordered only when two conditions are satisfied: (1) the physical attributes
of the land are such that a partition in kind is impracticable or inequitable;
and (2) the interests of the owners would better be promoted by a partition by
sale. Since our law has for many years presumed that a partition in kind would
be in the best interests of the owners, the burden is on the party requesting a
partition by sale to demonstrate that such a sale would better promote the
owners' interests.

The defendant claims in effect that the trial court's conclusion that the rights
of the parties would best be promoted by a judicial sale is not supported by
the findings of subordinate facts. We agree.

Under the test set out above, the court must first consider the practicability
of physically partitioning the property in question. The trial court concluded
that due to the situation and location of the parcel of land, the size and area
of the property, the physical structure and appurtenances on the property, and
other factors, a physical partition of the property would not be feasible. An
examination of the subordinate findings of facts and the exhibits, however,
demonstrates that the court erred in this respect.

It is undisputed that the property in question consists of one 20.5 acre parcel,
basically rectangular in shape, and one dwelling, located at the extreme
western end of the property. Two roads, Dino Road and Lucien Court, abut the
property and another, Birch Street, provides access through use of a
right-of-way. Unlike cases where there are numerous fractional owners of the
property to be partitioned, and the practicability of a physical division is
therefore drastically reduced; in this case there are only two competing
ownership interests: the plaintiffs' undivided 99/144 interest and the
defendant's 45/144 interest. These facts, taken together, do not support the
trial court's conclusion that a physical partition of the property would not be
``feasible'' in this case. Instead, the above facts demonstrate that the
opposite is true: a partition in kind clearly would be practicable under the
circumstances of this case.

Although a partition in kind is physically practicable, it remains to be
considered whether a partition in kind would also promote the best interests of
the parties. In order to resolve this issue, the consequences of a partition in
kind must be compared with those of a partition by sale.

The trial court concluded that a partition in kind could not be had without
great prejudice to the parties since the continuation of the defendant's
business would hinder or preclude the development of the plaintiffs' parcel for
residential purposes, which the trial court concluded was the highest and best
use of the property. The court's concern over the possible adverse economic
effect upon the plaintiffs' interest in the event of a partition in kind was
based essentially on four findings: (1) approval by the city planning
commission for subdivision of the parcel would be difficult to obtain if the
defendant continued her garbage hauling business; (2) lots in a residential
subdivision might not sell, or might sell at a lower price, if the defendant's
business continued; (3) if the defendant were granted the one-acre parcel, on
which her residence is situated and on which her business now operates, three
of the lots proposed in the plaintiffs' plan to subdivide the property would
have to be consolidated and would be lost; and (4) the proposed extension of
one of the neighboring roads would have to be rerouted through one of the
proposed building lots if a partition in kind were ordered. The trial court
also found that the defendant's use of the portion of the property that she
occupies is in violation of existing zoning regulations. The court presumably
inferred from this finding that it is not likely that the defendant will be
able to continue her rubbish hauling operations from this property in the
future. The court also premised its forecast that the planning commission would
reject the plaintiffs' subdivision plan for the remainder of the property on
the finding that the defendant's use was invalid. These factors basically led
the trial court to conclude that the interests of the parties would best be
protected if the land were sold as a unified unit for residential subdivision
development and the proceeds of such a sale were distributed to the parties.

\dots The defendant claims that the trial court erred in finding that the
defendant's use of a portion of the property is in violation of the existing
zoning regulations, and in refusing to find that such use is a valid
nonconforming use.\dots [T]he court's finding in this regard must be
stricken as unsupported by sufficient competent evidence. We are left, then,
with an unassailed finding that the defendant's family has operated a ``garbage
business'' on the premises since the 1920s and that the city of Bristol has
granted the defendant the appropriate permits and licenses each year to operate
her business. There is no indication that this practice will not continue in
the future.

Our resolution of this issue makes it clear that any inference that the
defendant would probably be unable to continue her rubbish hauling activity on
the property in the future is unfounded. We also conclude that the court erred
in concluding that the city's planning commission would probably not approve a
subdivision plan relating to the remainder of the property. Any such forecast
must be carefully scrutinized as it is difficult to project what a public body
will decide in any given matter.\dots The court's finding indicates that
only garbage trucks and dumpsters are stored on the property; that no garbage
is brought there; and that the defendant's business operations involve ``mostly
containerized\dots dumpsters, a contemporary development in technology which
has substantially reduced the odors previously associated with the rubbish and
garbage hauling industry.'' These facts do not support the court's speculation
that the city's planning commission would not approve a subdivision permit for
the undeveloped portion of the parties' property.

The court's remaining observations relating to the effect of the defendant's
business on the probable fair market value of the proposed residential lots,
the possible loss of building lots to accommodate the defendant's
business\readingfootnote{13}{It should be noted in this
regard that a partition in kind would result in a physical division of the land
according to the parties' respective interests. The defendant would, therefore,
not obtain any property in excess of her beneficial share of the parties'
concurrent estates.} and the rerouting of a
proposed subdivision road, which may have some validity, are not dispositive of
the issue. It is the interests of all of the tenants in common that the court
must consider; and not merely the economic gain of one tenant, or a group of
tenants. The trial court failed to give due consideration to the fact that one
of the tenants in common has been in actual and exclusive possession of a
portion of the property for a substantial period of time; that the tenant has
made her home on the property; and that she derives her livelihood from the
operation of a business on this portion of the property, as her family before
her has for many years. A partition by sale would force the defendant to
surrender her home and, perhaps, would jeopardize her livelihood. It is under
just such circumstances, which include the demonstrated practicability of a
physical division of the property, that the wisdom of the law's preference for
partition in kind is evident.

\dots Since the property in this case may practicably be physically divided,
and since the interests of all owners will better be promoted if a partition in
kind is ordered, we conclude that the trial court erred in ordering a partition
by sale, and that, under the facts as found, the defendant is entitled to a
partition of the property in kind.

