Joint tenancy (in some jurisdictions called a ``joint tenancy with right of
survivorship'' and abbreviated ``JTWROS'') is a form of ownership that can be
unilaterally severed and turned into a tenancy in common. Its distinctive
feature is the right of survivorship: If a joint tenancy is not severed before
a joint tenant's death, that joint tenant's interest disappears and the
remaining tenant continues to own an undivided interest, allowing the survivor
to avoid probate. Thus, joint tenancy is most widely used today as a
substitute for a will.\footnote{Note that the federal government does not
follow the fiction that nothing passes at death to the surviving joint tenant;
the decedent's interest will be taxed as if it were transferred to the
survivor, though if the joint tenants are married no tax will be due.}

In modern times, tenancy in common is preferred to other kinds of co-ownership.
A conveyance ``to Alice and Beth'' therefore creates a tenancy in common by
default, though it's relatively standard to include ``as tenants in common'' to
avoid all uncertainty. The creation and continuation of a joint tenancy is
beset with traps, even though it may well be most co-owners' preferred form of
ownership for residential property. Some states have statutes that appear to
abolish the joint tenancy, but they will often find joint tenancies with a right
of survivorship if the intent to create them is clear enough. \emph{See, e.g.},
\emph{McLeroy v. McLeroy}, 40 S.W.2d 1027 (Tenn. 1931).

