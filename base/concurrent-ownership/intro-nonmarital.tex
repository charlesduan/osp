Marriage is deeply embedded in American culture. Currently, 86\% of young men
and 89\% of young women are projected to marry at some point in their lives.
However, it is increasingly common for people to spend significant amounts of
time in nonmarital arrangements.\footnote{In 2011, there were 56 million
households with married couples at the head, out of a total of about 76 million
family households, and 115 million households total. Five million households
were headed by single men, and 15 million by single women. According to U.S.
Census data, 66\% of households in 2012 were family households, down from 81\%
in 1970. During that period, the share of households comprised of married
couples with children under 18 halved from 40\% to 20\%. Of family groups
living together, 71\% were married couples, down from 74\% in 2003. Of family
groups with children under 18, 63\% had married couples at the head, down from
67\% in 2003. Five percent of family groups were unmarried couples with
children. Unmarried people in opposite-sex relationships who were living
together were just as likely as married opposite-sex couples to have children
under 18 in their households (40-41\% of both groups), while 16\% of same-sex
couples had children under 18 present. Among opposite-sex married couples who
had children, however, almost 90\% had children who were the biological
offspring of both spouses, whereas that percentage dropped to 51\% of unmarried
couples living together with children. \emph{See} \textsc{Jonathan Vespa, Jamie
M. Lewis, \& Rose M. Kreider, America's Families and Living Arrangements: 2012}
(Aug. 2013), \url{https://www.census.gov/prod/2013pubs/p20-570.pdf}.} This means
that property law regularly faces disputes related to such arrangements. When,
if at all, should courts shift property rights around as a result of a
nonmarital relationship?

