\reading{Harms v. Sprague}

\readingcite{473 N.E.2d 930 (1984)}

\opinion Thomas J. \textsc{Moran}, Justice.

Plaintiff, William H. Harms, filed a complaint to quiet title and for
declaratory judgment in the circuit court of Greene County. Plaintiff had taken
title to certain real estate with his brother John R. Harms, as a joint tenant,
with full right of survivorship. The plaintiff named, as a defendant, Charles
D. Sprague, the executor of the estate of John Harms and the devisee of all the
real and personal property of John Harms. Also named as defendants were Carl T.
and Mary E. Simmons, alleged mortgagees of the property in question. Defendant
Sprague filed a counterclaim against plaintiff, challenging plaintiff's claim
of ownership of the entire tract of property and asking the court to recognize
his (Sprague's) interest as a tenant in common, subject to a mortgage lien. At
issue was the effect the granting of a mortgage by John Harms had on the joint
tenancy. Also at issue was whether the mortgage survived the death of John
Harms as a lien against the property.

The trial court held that the mortgage given by John Harms to defendants Carl
and Mary Simmons severed the joint tenancy. Further, the court found that the
mortgage survived the death of John Harms as a lien against the undivided
one-half interest in the property which passed to Sprague by and through the
will of the deceased. The appellate court reversed, finding that the mortgage
given by one joint tenant of his interest in the property does not sever the
joint tenancy. Accordingly, the appellate court held that plaintiff, as the
surviving joint tenant, owned the property in its entirety, unencumbered by the
mortgage lien.\dots

Two issues are raised on appeal: (1) Is a joint tenancy severed when less than
all of the joint tenants mortgage their interest in the property? and (2) Does
such a mortgage survive the death of the mortgagor as a lien on the property?

 A review of the stipulation of facts reveals the following. Plaintiff, William
Harms, and his brother John Harms, took title to real estate located in
Roodhouse, on June 26, 1973, as joint tenants. The warranty deed memorializing
this transaction was recorded on June 29, 1973, in the office of the Greene
County recorder of deeds.

Carl and Mary Simmons owned a lot and home in Roodhouse. Charles Sprague entered
into an agreement with the Simmons whereby Sprague was to purchase their
property for \$25,000. Sprague tendered \$18,000 in cash and signed a
promissory note for the balance of \$7,000. Because Sprague had no security for
the \$7,000, he asked his friend, John Harms, to co-sign the note and give a
mortgage on his interest in the joint tenancy property. Harms agreed, and on
June 12, 1981, John Harms and Charles Sprague, jointly and severally, executed
a promissory note for \$7,000 payable to Carl and Mary Simmons. The note states
that the principal sum of \$7,000 was to be paid from the proceeds of the sale
of John Harms' interest in the joint tenancy property, but in any event no
later than six months from the date the note was signed. The note reflects that
five monthly interest payments had been made, with the last payment recorded
November 6, 1981. In addition, John Harms executed a mortgage, in favor of the
Simmonses, on his undivided one-half interest in the joint tenancy property, to
secure payment of the note. William Harms was unaware of the mortgage given by
his brother.

John Harms moved from his joint tenancy property to the Simmons property which
had been purchased by Charles Sprague. On December 10, 1981, John Harms died.
By the terms of John Harms' will, Charles Sprague was the devisee of his entire
estate. The mortgage given by John Harms to the Simmonses was recorded on
December 29, 1981.

Prior to the appellate court decision in the instant case no court of this State
had directly addressed the principal question we are confronted with herein-the
effect of a mortgage, executed by less than all of the joint tenants, on the
joint tenancy. Nevertheless, there are numerous cases which have considered the
severance issue in relation to other circumstances surrounding a joint tenancy.
All have necessarily focused on the four unities which are fundamental to both
the creation and the perpetuation of the joint tenancy. These are the unities
of interest, title, time, and possession. The voluntary or involuntary
destruction of any of the unities by one of the joint tenants will sever the
joint tenancy.

In a series of cases, this court has considered the effect that judgment liens
upon the interest of one joint tenant have on the stability of the joint
tenancy. In \emph{Peoples Trust \& Savings Bank v. Haas} (1927), 328 Ill. 468,
160 N.E. 85, the court found that a judgment lien secured against one joint
tenant did not serve to extinguish the joint tenancy. As such, the surviving
joint tenant ``succeeded to the title in fee to the whole of the land by
operation of law.''

 \dots Clearly, this court adheres to the rule that a lien on a joint tenant's
interest in property will not effectuate a severance of the joint tenancy,
absent the conveyance by a deed following the expiration of a redemption
period. It follows, therefore, that if Illinois perceives a mortgage as merely
a lien on the mortgagor's interest in property rather than a conveyance of
title from mortgagor to mortgagee, the execution of a mortgage by a joint
tenant, on his interest in the property, would not destroy the unity of title
and sever the joint tenancy.

Early cases in Illinois, however, followed the title theory of mortgages. In
1900, this court recognized the common law precept that a mortgage was a
conveyance of a legal estate vesting title to the property in the mortgagee.
Consistent with this title theory of mortgages, therefore, there are many cases
which state, in dicta, that a joint tenancy is severed by one of the joint
tenants mortgaging his interest to a stranger. Yet even the early case of
\emph{Lightcap v. Bradley}, cited above, recognized that the title held by the
mortgagee was for the limited purpose of protecting his interests. The court
went on to say that ``the mortgagor is the owner for every other purpose and
against every other person. The title of the mortgagee is anomalous, and exists
only between him and the mortgagor\ldots.'' \emph{Lightcap v. Bradley} (1900),
186 Ill. 510, 522-23, 58 N.E. 221.

Because our cases had early recognized the unique and narrow character of the
title that passed to a mortgagee under the common law title theory, it was not
a drastic departure when this court expressly characterized the execution of a
mortgage as a mere lien\dots

[A] joint tenancy is not severed when one joint tenant executes a mortgage on
his interest in the property, since the unity of title has been preserved. As
the appellate court in the instant case correctly observed: ``If giving a
mortgage creates only a lien, then a mortgage should have the same effect on a
joint tenancy as a lien created in other ways.'' Other jurisdictions following
the lien theory of mortgages have reached the same result.

\dots An inherent feature of the estate of joint tenancy is the right of
survivorship, which is the right of the last survivor to take the whole of the
estate. Because we find that a mortgage given by one joint tenant of his
interest in the property does not sever the joint tenancy, we hold that the
plaintiff's right of survivorship became operative upon the death of his
brother. As such plaintiff is now the sole owner of the estate, in its
entirety.

Further, we find that the mortgage executed by John Harms does not survive as a
lien on plaintiff's property. A surviving joint tenant succeeds to the share of
the deceased joint tenant by virtue of the conveyance which created the joint
tenancy, not as the successor of the deceased. The property right of the
mortgaging joint tenant is extinguished at the moment of his death. While John
Harms was alive, the mortgage existed as a lien on his interest in the joint
tenancy. Upon his death, his interest ceased to exist and along with it the
lien of the mortgage. Under the circumstances of this case, we would note that
the mortgage given by John Harms to the Simmonses was only valid as between the
original parties during the lifetime of John Harms since it was unrecorded. In
addition, recording the mortgage subsequent to the death of John Harms was a
nullity. As we stated above, John Harms' property rights in the joint tenancy
were extinguished when he died. Thus, he no longer had a property interest upon
which the mortgage lien could attach\dots.

