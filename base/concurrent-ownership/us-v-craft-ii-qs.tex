\expected{us-v-craft-ii}

\item \textit{Sawada v. Endo}, 561 P.2d 1291 (Haw. 1977), reached a different
result under state law. \textit{Sawada} allowed a transfer of entireties
property (the family home) by a husband and wife to their children, in order to
avoid the risk that the home would be vulnerable to claims by Masako and Helen
Sawada, who'd been injured when they were struck by a car operated by the
husband, and who eventually became judgment creditors as a result of the
lawsuits they filed against the husband, Kokichi Endo. Given that any lien
against the house could only attach to the husband's interest and that the
house couldn't be sold without the wife's consent, what exactly was the risk to
the Endos' ownership of the house?


The Endos conveyed the house to their children, for no valuable consideration,
after the accident and after the first complaint was filed. The parents
continued to live in the house, though they had no legal interest in it. After
trial, both Sawadas were awarded a total of roughly \$25,000. The wife, Ume
Endo, died shortly thereafter, survived by Kokichi. The Sawadas, unable to
recover against Kokichi Endo's personal property, sought to invalidate the
transfer of the family home to the children as fraudulent.



The Hawaii Supreme Court found that a spouse's interest in property held by the
entireties was not subject to levy and execution by that spouse's individual
creditors, even though some states do allow seizure and sale by creditors,
subject to the other spouse's contingent right of survivorship. The Hawaii
Supreme Court reasoned that the Married Women's Property Acts equalized husband
and wife, creating a unity of equals who both had the right to use and enjoy
the whole estate. This insulated the wife's interest in the estate from the
separate debts of her husband, and vice versa. ``A joint tenancy may be
destroyed by voluntary alienation, or by levy and execution, or by compulsory
partition, but a tenancy by the entirety may not. The indivisibility of the
estate, except by joint action of the spouses, is an indispensable feature of
the tenancy by the entirety.'' Creditors of one spouse could not even attach
that spouse's right of survivorship, because that would make a conveyance by
both spouses too uncertain, harming the other spouse's interest.



The Hawaii Supreme Court continued, ``there is obviously nothing to prevent [a]
creditor from insisting upon the subjection of property held in tenancy by the
entirety as a condition precedent to the extension of credit. Further, the
creation of a tenancy by the entirety may not be used as a device to defraud
existing creditors.'' That's all well and good for voluntary creditors, but
what about involuntary creditors like the Sawadas? They weren't offered any
options before they extended ``credit'' to Kokichi Endo in the form of the
injuries he inflicted on them. Is this rule fair to them? (Is the proper
comparison a world in which Kokichi Endo didn't own a house at all when he hit
them, or a world in which he owned a house jointly or in common when he hit
them? Does it matter that the law is less directly involved in whether Endo
owned a house than in the rules of co-ownership?)


The Hawaii Supreme Court concluded that public policy supported its holding,
because tenancy by the entirety protected an interest in family solidarity:
\begin{quote}
When a family can afford to own real property, it becomes their single most
important asset. Encumbered as it usually is by a first mortgage, the fact
remains that so long as it remains whole during the joint lives of the spouses,
it is always available in its entirety for the benefit and use of the entire
family. Loans for education and other emergency expenses, for example, may be
obtained on the security of the marital estate. This would not be possible
where a third party has become a tenant in common or a joint tenant with one of
the spouses, or where the ownership of the contingent right of survivorship of
one of the spouses in a third party has cast a cloud upon the title of the
marital estate, making it virtually impossible to utilize the estate for these
purposes.
\end{quote}
561 P.2d at 1297. A dissent pointed out that, under the Married Women's Property
Acts, what was required was equality as between spouses, not any particular
rule about creditors. At common law, ``the interest of the husband in an
estate by the entireties could be taken by his separate creditors on execution
against him, subject only to the wife's right of survivorship.'' Thus, the
dissent reasoned, equal treatment merely required that both spouses be
subjected to this rule.



One way of looking at the matter: entireties property is specifically designed,
at least in its modern incarnation, to protect the interest of one spouse
against the other's independent acts. If that's the case, then aren't the
\textit{Craft} dissents correct? If a state may choose this objective in its
property law, why shouldn't this choice be respected? Or are there special
concerns relating to federal tax that justify overriding this choice? If so,
should the government be able to force the sale of entireties property, or
should it be forced to wait to see which spouse survives the other?



\item \textbf{Forfeiture}. What about criminal forfeiture of property involved
in a crime, such as a house in which a drug transaction occurred? Some
forfeiture statutes exempt property used without the consent or knowledge of
its owner. Under those statutes, some courts allow the innocent spouse to
retain use and possession of entirety property during her lifetime, as well as
her right of survivorship. \textit{Compare} \emph{United States v. 1500 Lincoln
Ave.}, 949 F.2d 73 (3d Cir. 1991) (guilty spouse's interest is forfeited,
subject to innocent spouse's possession and survivorship rights), \textit{with}
\emph{United States v. 15621 S.W. 209th Ave.}, 894 F.2d 1511 (11th Cir. 1990)
(not allowing current forfeiture, but allowing government to file lis pendens
preserving its right to guilty spouse's interest upon death of innocent spouse
or severance of estate). What if a forfeiture statute doesn't protect innocent
owners? In that case, the government can seize the entire property, including
the innocent spouse's interest. \emph{Bennis v. Michigan}, 516 U.S. 442 (1996)
(rejecting takings and due process claims).


\item \textbf{Homestead acts as an alternative?} Many states have so-called
``homestead'' acts, protecting the family home (up to a certain value or size)
from many creditors' claims, though not against foreclosure of a mortgage on
that home. California provides for \$50,000 for a single person, \$75,000 for a
``family unit,'' and \$150,000 for people 65 or older, disabled, or 55 or older
with an annual income under \$15,000. \textsc{Cal. Code Civ. Proc.} \S~704.730
(2003). Washington provides for protections for \$40,000 real property or
\$15,000 personal property. \textsc{Wash. Rev. Code} \S~6.13.030 (1999). Should
the tenancy by the entirety be abolished in favor of homestead exemptions?
Compare the protections for mortgagors\having{intro-mortgages}{, discussed in
the unit on Mortgages}{, which we will discuss in the unit on Mortgages}{}.


\item \textbf{Creating a tenancy by the entirety.} Traditionally, a tenancy by
the entirety was created by granting property ``to X and Y, husband and wife,
as tenants by the entirety.'' Today, X and Y can be any spouses, and states
that recognize tenancies by the entirety often presume that a transfer ``to A
and B, [spouses],'' creates that estate. \emph{See, e.g.}, \emph{Constitution
Bank v. Olson}, 620 A.2d 1146 (Pa. Super. Ct. 1993). Other states always presume
a tenancy in common even when the co-owners are married, so a clear expression
of the requisite intent is required. \emph{See} \textsc{Miss. Code Ann.}
\S~89-11-7. As a rule, the magic words ``tenants by the entirety'' should be
used.


If the cotenants are not married, the magic words will not work. In
\emph{Riccelli v. Forcinito}, 595 A.2d 1322 (Pa. Super. Ct. 1991), Sam Riccelli
and Carmen Pirozek bought property in 1962 ``as tenants by the entireties with
the right of survivorship.'' However, they weren't married at the time of the
purchase, and so they couldn't have a tenancy by the entirety. What kind of
tenancy did they have? The court reasoned: ``The appropriate form of tenancy is
to be determined by the intention of the parties, `the ultimate guide by which
all deeds must be interpreted.'\dots [J]oint tenancy with the right of
survivorship best effectuates their intention to the extent legally
permissible, that being the form of tenancy for unmarried persons most nearly
resembling the tenancy by the entireties enjoyed by husband and wife, since in
both instances the survivor takes the whole.'' The modern presumption in favor
of tenancy in common yielded to a clearly expressed contrary intent.
\textit{See also Funches v. Funches}, 413 S.E.2d 44 (Va. 1992) (``tenancy by
the entirety'' with express survivorship language that was given to unmarried
parties created a joint tenancy because of the survivorship language).
\textit{But see Smith v. Stewart}, 596 S.W.2d 346 (Ark. Ct. App. 1980) (deed
``to A and B, his wife,'' when A and B were unmarried, failed to create a joint
tenancy; the relevant state statute required an express declaration of joint
tenancy with right of survivorship), \emph{aff'd}, 601 S.W.2d 837 (Ark. 1980).



\item \textbf{Divorce.} Because marriage is a requirement for a tenancy by the
entirety, divorce ends that form of ownership. What should replace it? The
modern preference is for tenancy in common as a general rule, and many states
follow that rule with tenancies by the entireties that end by divorce.
\emph{See, e.g.}, \textsc{Mich. Comp. Laws Ann.} \S~552.102. A few states
presume that a tenancy by the entirety is converted to a joint tenancy unless
the parties otherwise agree. \emph{See, e.g.}, \emph{Estate of Childress v.
Long}, 588 So. 2d 192 (Miss. 1991).


\item \textbf{Common law marriage.} Common law marriage was widely recognized
when access to formal marriage was sometimes difficult, particularly in rural
areas. However, it is now recognized only in 11 states and the District of
Columbia. Where it is recognized, the parties must manifest an intent to be
married and hold themselves out as husband and wife. If they do so, they have
exactly the same rights as any other married couple. Is this a kind of
``adverse possession'' of the benefits of marriage?


Many states abolished common law marriage on the theory that it was no longer
required, given the ease of accessing a marriage license, and that it
encouraged people to lie about whether they'd held themselves out as husband
and wife. Moreover, a marriage license makes it easy to understand who is
entitled to pensions and other benefits, which became more important as those
types of assets became more significant throughout the twentieth century.

