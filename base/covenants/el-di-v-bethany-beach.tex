\reading{El Di, Inc. v. Town of Bethany Beach}

\readingcite{477 A.2d 1066 (Del. 1984)}

\opinion \textsc{Herrmann}, Chief Justice for the majority:

This is an appeal from a permanent injunction granted by the Court of Chancery
upon the petition of the plaintiffs, The Town of Bethany Beach, et al.,
prohibiting the defendant, El Di, Inc. (``El Di'') from selling alcoholic
beverages at Holiday House, a restaurant in Bethany Beach owned and operated by
El Di.

\readinghead{I.}

The pertinent facts are as follows:

El Di purchased the Holiday House in 1969. In December 1981, El Di filed an
application with the State Alcoholic Beverage Control Commission (the
``Commission'') for a license to sell alcoholic beverages at the Holiday House.
On April 15, 1982, finding ``public need and convenience,'' the Commission
granted the Holiday House an on-premises license. The sale of alcoholic
beverages at Holiday House began within 10 days of the Commission's approval.
Plaintiffs subsequently filed suit to permanently enjoin the sale of alcoholic
beverages under the license.

On appeal it is undisputed that the chain of title for the Holiday House lot
included restrictive covenants prohibiting both the sale of alcoholic beverages
on the property and nonresidential construction.\readingfootnote{1}{The restrictive
covenant stated:

``This covenant is made expressly subject to and upon the
following conditions: viz; That no intoxicating liquors shall ever be sold on
the said lot, that no other than dwelling or cottage shall be erected thereon
and but one to each lot, which must be of full size according to the said
plan\ldots a breach of which said conditions, or any of them, shall cause said
lot to revert to and become again the property of the grantor, his heirs and
assigns; and upon such breach of said conditions or restrictions, the same may
be restrained or enjoined in equity by the grantor, his heirs or assigns, or by
any co-lot owner in said plan or other party injured by such breach.''} The same
restriction was placed on property in Bethany Beach as early as 1900 and 1901
when the area was first under development.

As originally conceived, Bethany Beach was to be a quiet beach community. The
site was selected at the end of the nineteenth-century by the Christian
Missionary Society of Washington, D.C. In 1900, the Bethany Beach Improvement
Company (``BBIC'') was formed. The BBIC purchased lands, laid out a development
and began selling lots. To insure the quiet character of the community, the BBIC
placed restrictive covenants on many plots, prohibiting the sale of alcohol and
restricting construction to residential cottages. Of the original 180 acre
development, however, approximately 1/3 was unrestricted.

The Town of Bethany Beach was officially incorporated in 1909. The municipal
limits consisted of 750 acres including the original BBIC land (hereafter the
original or ``old-Town''), but expanded far beyond the 180 acre BBIC
development. The expanded acreage of the newly incorporated Town, combined with
the unrestricted plots in the original Town, left only 15 percent of the new
Town subject to the restrictive covenants.

Despite the restriction prohibiting commercial building (``no other than a
dwelling or cottage shall be erected\ldots''), commercial development began in
the 1920's on property subject to the covenants. This development included
numerous inns, restaurants, drug stores, a bank, motels, a town hall, shops
selling various items including food, clothing, gifts and novelties and other
commercial businesses. Of the 34 commercial buildings presently within the Town
limits, 29 are located in the old-Town originally developed by BBIC. Today,
Bethany Beach has a permanent population of some 330 residents. In the summer
months the population increases to approximately 10,000 people within the
corporate limits and to some 48,000 people within a 4 mile radius. In 1952, the
Town enacted a zoning ordinance which established a central commercial district
designated C-1 located in the old-Town section. Holiday House is located in this
district.

Since El Di purchased Holiday House in 1969, patrons have been permitted to
carry their own alcoholic beverages with them into the restaurant to consume
with their meals. This ``brown-bagging'' practice occurred at Holiday House
prior to El Di's ownership and at other restaurants in the Town. El Di applied
for a license to sell liquor at Holiday House in response to the increased
number of customers who were engaging in ``brown-bagging'' and in the belief
that the license would permit restaurant management to control excessive use of
alcohol and use by minors. Prior to the time El Di sought a license, alcoholic
beverages had been and continue to be readily available for sale at nearby
licensed establishments including: one restaurant {\textonehalf} mile outside
the Town limits, 3 restaurants within a 4 mile radius of the Town, and a package
store some 200-300 yards from the Holiday House.

The Trial Court granted a stay pending the outcome of this appeal.

\readinghead{II.}

In granting plaintiffs' motion for a permanent injunction, the Court of Chancery
rejected defendant's argument that changed conditions in Bethany Beach rendered
the restrictive covenants unreasonable and therefore unenforceable. The Chancery
Court found that although the evidence showed a considerable growth since 1900
in both population and the number of buildings in Bethany Beach, ``the basic
nature of Bethany Beach as a quiet, family oriented resort has not changed.''
The Court also found that there had been development of commercial activity
since 1900, but that this ``activity is limited to a small area of Bethany Beach
and consists mainly of activities for the convenience and patronage of the
residents of Bethany Beach.''

The Trial Court also rejected defendant's contention that plaintiffs'
acquiescence and abandonment rendered the covenants unenforceable. In this
connection, the Court concluded that the practice of ``brown-bagging'' was not a
sale of alcoholic beverages and that, therefore, any failure to enforce the
restriction as against the practice did not constitute abandonment or waiver of
the restriction.

\readinghead{III.}

We find that the Trial Court erred in holding that the change of conditions was
insufficient to negate the restrictive covenant.

A court will not enforce a restrictive covenant where a fundamental change has
occurred in the intended character of the neighborhood that renders the benefits
underlying imposition of the restrictions incapable of enjoyment. Review of all
the facts and circumstances convinces us that the change, since 1901, in the
character of that area of the old-Town section now zoned C-1 is so substantial
as to justify modification of the deed restriction. We need not determine a
change in character of the entire restricted area in order to assess the
continued applicability of the covenant to a portion thereof.

It is uncontradicted that one of the purposes underlying the covenant
prohibiting the sale of intoxicating liquors was to maintain a quiet,
residential atmosphere in the restricted area. Each of the additional covenants
reinforces this objective, including the covenant restricting construction to
residential dwellings. The covenants read as a whole evince an intention on the
part of the grantor to maintain the residential, seaside character of the
community.

But time has not left Bethany Beach the same community its grantors envisioned
in 1901. The Town has changed from a church-affiliated residential community to
a summer resort visited annually by thousands of tourists. Nowhere is the
resultant change in character more evident than in the C-1 section of the
old-Town. Plaintiffs argue that this is a relative change only and that there is
sufficient evidence to support the Trial Court's findings that the residential
character of the community has been maintained and that the covenants continue
to benefit the other lot owners. We cannot agree.

In 1909, the 180 acre restricted old-Town section became part of a 750 acre
incorporated municipality. Even prior to the Town's incorporation, the BBIC
deeded out lots free of the restrictive covenants. After incorporation and
partly due to the unrestricted lots deeded out by the BBIC, 85 percent of the
land area within the Town was not subject to the restrictions. Significantly,
nonresidential uses quickly appeared in the restricted area and today the
old-Town section contains almost all of the commercial businesses within the
entire Town.

The change in conditions is also reflected in the Town's decision in 1952 to
zone restricted property, including the lot on which the Holiday House is
located, specifically for commercial use. Although a change in zoning is not
dispositive as against a private covenant, it is additional evidence of changed
community conditions. 

Time has relaxed not only the strictly residential character of the area, but
the pattern of alcohol use and consumption as well. The practice of
``brown-bagging'' has continued unchallenged for at least twenty years at
commercial establishments located on restricted property in the Town. On appeal,
plaintiffs rely on the Trial Court finding that the ``brown-bagging'' practice
is irrelevant as evidence of waiver inasmuch as the practice does not involve
the sale of intoxicating liquors prohibited by the covenant. We find the
``brown-bagging'' practice evidence of a significant change in conditions in the
community since its inception at the turn of the century. Such consumption of
alcohol in public places is now generally tolerated by owners of similarly
restricted lots. The license issued to the Holiday House establishment permits
the El Di management to better control the availability and consumption of
intoxicating liquors on its premises. In view of both the ready availability of
alcoholic beverages in the area surrounding the Holiday House and the
long-tolerated and increasing use of ``brown-bagging'' enforcement of the
restrictive covenant at this time would only serve to subvert the public
interest in the control of the availability and consumption of alcoholic
liquors.

\ldots In view of the change in conditions in the C-1 district of Bethany Beach,
we find it unreasonable and inequitable now to enforce the restrictive covenant.
To permit unlimited ``brown-bagging'' but to prohibit licensed sales of
alcoholic liquor, under the circumstances of this case, is inconsistent with any
reasonable application of the restriction and contrary to public policy.

We emphasize that our judgment is confined to the area of the old-Town section
zoned C-1. The restrictions in the neighboring residential area are unaffected
by the conclusion we reach herein.

Reversed.

\opinion \textsc{Christie}, Justice, with whom \textsc{Moore}, Justice, joins,
dissenting:

I respectfully disagree with the majority.

I think the evidence supports the conclusion of the Chancellor, as finder of
fact, that the basic nature of the community of Bethany Beach has not changed in
such a way as to invalidate those restrictions which have continued to protect
this community through the years as it has grown. Although some of the
restrictions have been ignored and a portion of the community is now used for
limited commercial purposes, the evidence shows that Bethany Beach remains a
quiet, family-oriented resort where no liquor is sold. I think the conditions of
the community are still consistent with the enforcement of a restrictive
covenant forbidding the sale of intoxicating beverages.

In my opinion, the toleration of the practice of ``brown bagging'' does not
constitute the abandonment of a longstanding restriction against the sale of
alcoholic beverages. The restriction against sales has, in fact, remained intact
for more than eighty years and any violations thereof have been short-lived. The
fact that alcoholic beverages may be purchased right outside the town is not
inconsistent with my view that the quiet-town atmosphere in this small area has
not broken down, and that it can and should be preserved. Those who choose to
buy land subject to the restrictions should be required to continue to abide by
the restrictions.

I think the only real beneficiaries of the failure of the courts to enforce the
restrictions would be those who plan to benefit commercially.

I also question the propriety of the issuance of a liquor license for the sale
of liquor on property which is subject to a specific restrictive covenant
against such sales.

I think that restrictive covenants play a vital part in the preservation of
neighborhood schemes all over the State, and that a much more complete breakdown
of the neighborhood scheme should be required before a court declares that a
restriction has become unenforceable.

I would affirm the Chancellor.

