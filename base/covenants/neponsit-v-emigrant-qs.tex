\expected{neponsit-v-emigrant}

\item Does the touch-and-concern requirement lessen the potential for conflict
between the law of restrictive covenants and the common-law doctrines designed
to preserve marketability of land, such as \textit{numerus clausus} and the rule
against restraints on alienation?

\item Is the court's resolution of the privity-of-estate issue consistent with
what you've learned about corporate property? With the later New York case of
\textit{Walkovszky v. Carlton?}

\item As with easements, restrictive covenants may be implied in particular
circumstances, and they may arise by estoppel. The most common context for such
a covenant by implication is a common-scheme development, where purchasers
acquire an interest in a parcel that is part of a community that appears to have
commonly planned features---such as residential uses of particular size and
density. Such purchasers may be charged with notice of an implied reciprocal
covenant restricting their parcels to uses consistent with the common scheme or
plan. \textit{See} \emph{Sanborn v. McLean}, 206 N.W. 496 (Mich. 1925);
\textsc{Restatement} \S\S~2.11 \& illus. 7; \S~2.14. Conversely, where the
seller touts
the benefits of such features to purchasers who buy in reliance on the seller's
representations, the seller and his successors may be estopped from using the
seller's retained land in a manner inconsistent with those uses. Indeed, such an
estoppel may even serve as an acceptable substitute for the writing required
under the Statute of Frauds. \textsc{Restatement} \S\S~2.9-2.10.

\item A historical note in the \textsc{Third Restatement} explains: 
\begin{quote}
At the beginning of the 20th century, four doctrines peculiar to servitudes law
constrained landowners in the creation of servitudes: the horizontal-privity
doctrine, the prohibition on creating benefits in gross, the prohibition on
imposing affirmative burdens on fee owners, and the touch-or-concern doctrine.
At the end of the century, little remains of those doctrines, which have
gradually been displaced by doctrines that more specifically target the harms
that may be caused by servitudes. 
\end{quote}
\textsc{Restatement} \S~3.1, cmt. a. The touch-and-concern doctrine comes in for
particular criticism in the \textsc{Restatement}, which attacks the doctrine's
``vagueness, its obscurity, its intent-defeating character, and its growing
redundancy.'' \textit{Id.} \S~3.2 cmt. b. Accordingly, the \textsc{Restatement}
adopts a
very different approach to the question of enforceability of restrictive
covenants:
\expectnext{restatement-servitudes}

