\reading{Shelley v. Kraemer}
\readingcite{334 U.S. 1 (1948)}

\opinion Mr. Chief Justice \textsc{Vinson} delivered the opinion of the Court.

These cases present for our consideration questions relating to the validity of
court enforcement of private agreements, generally described as restrictive
covenants, which have as their purpose the exclusion of persons of designated
race or color from the ownership or occupancy of real property. Basic
constitutional issues of obvious importance have been raised.

The first of these cases comes to this Court on certiorari to the Supreme Court
of Missouri. On February 16, 1911, thirty out of a total of thirty-nine owners
of property fronting both sides of Labadie Avenue between Taylor Avenue and Cora
Avenue in the city of St. Louis, signed an agreement, which was subsequently
recorded, providing in part:
\begin{quote}
\ldots the said property is hereby restricted to the use and occupancy for the
term of Fifty (50) years from this date, so that it shall be a condition all the
time and whether recited and referred to as (sic) not in subsequent conveyances
and shall attach to the land, as a condition precedent to the sale of the same,
that hereafter no part of said property or any portion thereof shall be, for
said term of Fifty-years, occupied by any person not of the Caucasian race, it
being intended hereby to restrict the use of said property for said period of
time against the occupancy as owners or tenants of any portion of said property
for resident or other purpose by people of the Negro or Mongolian Race.
\end{quote}

\ldots On August 11, 1945, pursuant to a contract of sale, petitioners Shelley,
who are Negroes, for valuable consideration received from one Fitzgerald a
warranty deed to the parcel in question. The trial court found that petitioners
had no actual knowledge of the restrictive agreement at the time of the
purchase.

On October 9, 1945, respondents, as owners of other property subject to the
terms of the restrictive covenant, brought suit in Circuit Court of the city of
St. Louis praying that petitioners Shelley be restrained from taking possession
of the property and that judgment be entered divesting title out of petitioners
Shelley and revesting title in the immediate grantor or in such other person as
the court should direct. The trial court denied the requested relief on the
ground that the restrictive agreement, upon which respondents based their
action, had never become final and complete because it was the intention of the
parties to that agreement that it was not to become effective until signed by
all property owners in the district, and signatures of all the owners had never
been obtained.

The Supreme Court of Missouri sitting en banc reversed and directed the trial
court to grant the relief for which respondents had prayed. That court held the
agreement effective and concluded that enforcement of its provisions violated no
rights guaranteed to petitioners by the Federal Constitution. At the time the
court rendered its decision, petitioners were occupying the property in
question.

\ldots Petitioners have placed primary reliance on their contentions, first
raised in the state courts, that judicial enforcement of the restrictive
agreements in these cases has violated rights guaranteed to petitioners by the
Fourteenth Amendment of the Federal Constitution and Acts of Congress passed
pursuant to that Amendment. Specifically, petitioners urge that they have been
denied the equal protection of the laws, deprived of property without due
process of law, and have been denied privileges and immunities of citizens of
the United States. We pass to a consideration of those issues.

\readinghead{I.}

Whether the equal protection clause of the Fourteenth Amendment inhibits
judicial enforcement by state courts of restrictive covenants based on race or
color is a question which this Court has not heretofore been called upon to
consider.

\ldots It should be observed that these covenants do not seek to proscribe any
particular use of the affected properties. Use of the properties for residential
occupancy, as such, is not forbidden. The restrictions of these agreements,
rather, are directed toward a designated class of persons and seek to determine
who may and who may not own or make use of the properties for residential
purposes. The excluded class is defined wholly in terms of race or color;
`simply that and nothing more.'

It cannot be doubted that among the civil rights intended to be protected from
discriminatory state action by the Fourteenth Amendment are the rights to
acquire, enjoy, own and dispose of property. Equality in the enjoyment of
property rights was regarded by the framers of that Amendment as an essential
pre-condition to the realization of other basic civil rights and liberties which
the Amendment was intended to guarantee.7 Thus, \S~1978 of the Revised Statutes,
derived from \S~1 of the Civil Rights Act of 1866 which was enacted by Congress
while the Fourteenth Amendment was also under consideration, provides:
\begin{quote}
All citizens of the United States shall have the same right, in every State
and Territory, as is enjoyed by white citizens thereof to inherit, purchase,
lease, sell, hold, and convey real and personal property.
\end{quote}

This Court has given specific recognition to the same principle.

It is likewise clear that restrictions on the right of occupancy of the sort
sought to be created by the private agreements in these cases could not be
squared with the requirements of the Fourteenth Amendment if imposed by state
statute or local ordinance. We do not understand respondents to urge the
contrary.

\ldots But the present cases\ldots do not involve action by state legislatures
or city councils. Here the particular patterns of discrimination and the areas
in which the restrictions are to operate, are determined, in the first instance,
by the terms of agreements among private individuals. Participation of the State
consists in the enforcement of the restrictions so defined. The crucial issue
with which we are here confronted is whether this distinction removes these
cases from the operation of the prohibitory provisions of the Fourteenth
Amendment.

Since the decision of this Court in the \emph{Civil Rights Cases}, 1883, 109
U.S. 3, 3 S.Ct. 18, 27 L.Ed. 835, the principle has become firmly embedded in
our constitutional law that the action inhibited by the first section of the
Fourteenth Amendment is only such action as may fairly be said to be that of the
States. That Amendment erects no shield against merely private conduct, however
discriminatory or wrongful.

We conclude, therefore, that the restrictive agreements standing alone cannot be
regarded as a violation of any rights guaranteed to petitioners by the
Fourteenth Amendment. So long as the purposes of those agreements are
effectuated by voluntary adherence to their terms, it would appear clear that
there has been no action by the State and the provisions of the Amendment have
not been violated.

But here there was more. These are cases in which the purposes of the agreements
were secured only by judicial enforcement by state courts of the restrictive
terms of the agreements. The respondents urge that judicial enforcement of
private agreements does not amount to state action; or, in any event, the
participation of the State is so attenuated in character as not to amount to
state action within the meaning of the Fourteenth Amendment. Finally, it is
suggested, even if the States in these cases may be deemed to have acted in the
constitutional sense, their action did not deprive petitioners of rights
guaranteed by the Fourteenth Amendment. We move to a consideration of these
matters\ldots .


\readinghead{III.}

\ldots We have no doubt that there has been state action in these cases
in the full and complete sense of the phrase. The undisputed facts disclose that
petitioners were willing purchasers of properties upon which they desired to
establish homes. The owners of the properties were willing sellers; and
contracts of sale were accordingly consummated. It is clear that but for the
active intervention of the state courts, supported by the full panoply of state
power, petitioners would have been free to occupy the properties in question
without restraint.

These are not cases, as has been suggested, in which the States have merely
abstained from action, leaving private individuals free to impose such
discriminations as they see fit. Rather, these are cases in which the States
have made available to such individuals the full coercive power of government to
deny to petitioners, on the grounds of race or color, the enjoyment of property
rights in premises which petitioners are willing and financially able to acquire
and which the grantors are willing to sell. The difference between judicial
enforcement and nonenforcement of the restrictive covenants is the difference to
petitioners between being denied rights of property available to other members
of the community and being accorded full enjoyment of those rights on an equal
footing.

The enforcement of the restrictive agreements by the state courts in these cases
was directed pursuant to the common-law policy of the States as formulated by
those courts in earlier decisions. In the Missouri case, enforcement of the
covenant was directed in the first instance by the highest court of the
State\ldots . The judicial action in each case bears the clear and unmistakable
imprimatur of the State. We have noted that previous decisions of this Court
have established the proposition that judicial action is not immunized from the
operation of the Fourteenth Amendment simply because it is taken pursuant to the
state's common-law policy. Nor is the Amendment ineffective simply because the
particular pattern of discrimination, which the State has enforced, was defined
initially by the terms of a private agreement. State action, as that phrase is
understood for the purposes of the Fourteenth Amendment, refers to exertions of
state power in all forms. And when the effect of that action is to deny rights
subject to the protection of the Fourteenth Amendment, it is the obligation of
this Court to enforce the constitutional commands.

We hold that in granting judicial enforcement of the restrictive agreements in
these cases, the States have denied petitioners the equal protection of the laws
and that, therefore, the action of the state courts cannot stand. We have noted
that freedom from discrimination by the States in the enjoyment of property
rights was among the basic objectives sought to be effectuated by the framers of
the Fourteenth Amendment. That such discrimination has occurred in these cases
is clear. Because of the race or color of these petitioners they have been
denied rights of ownership or occupancy enjoyed as a matter of course by other
citizens of different race or color.\ldots

The historical context in which the Fourteenth Amendment became a part of the
Constitution should not be forgotten. Whatever else the framers sought to
achieve, it is clear that the matter of primary concern was the establishment of
equality in the enjoyment of basic civil and political rights and the
preservation of those rights from discriminatory action on the part of the
States based on considerations of race or color. Seventy-five years ago this
Court announced that the provisions of the Amendment are to be construed with
this fundamental purpose in mind. Upon full consideration, we have concluded
that in these cases the States have acted to deny petitioners the equal
protection of the laws guaranteed by the Fourteenth Amendment. Having so
decided, we find it unnecessary to consider whether petitioners have also been
deprived of property without due process of law or denied privileges and
immunities of citizens of the United States.

For the reasons stated, the judgment of the Supreme Court of Missouri and the
judgment of the Supreme Court of Michigan must be reversed.

Reversed.

Mr. Justice \textsc{Reed}, Mr. Justice \textsc{Jackson}, and Mr. Justice
\textsc{Rutledge} took no part in the consideration or decision of these cases.

