As courts became more amenable to the enforcement of restrictive covenants by
and against successors to the property interests of the original covenanting
parties, they developed a set of requirements for such covenants to run with the
land. As one court described these requirements:
\begin{quote}
The prerequisites for a covenant to ``run with the land'' are these: (1) the
covenants must have been enforceable between the original parties, such
enforceability being a question of contract law except insofar as the covenant
must satisfy the statute of frauds; (2) the covenant must ``touch and concern''
both the land to be benefitted and the land to be burdened; (3) the covenanting
parties must have intended to bind their successors-in-interest; (4) there must
be vertical privity of estate, i.e., privity between the original parties to the
covenant and the present disputants; and (5) there must be horizontal privity of
estate, or privity between the original parties.
\end{quote}
\emph{Leighton v. Leonard}, 589 P.2d 279, 281 (Ct. App. Wash. Div. 1 1978). A
further requirement is that a restrictive covenant is enforceable only against
parties who are on actual or constructive notice of it. \textit{See id.} at
281-282; \textit{accord} \emph{Inwood N. Homeowners' Ass'n, Inc. v. Harris}, 736
S.W.2d 632, 635 (Tex. 1987).

The \textsc{Third Restatement}, following general trends in the caselaw,
significantly relaxes this approach. Section 2.1 of the \textsc{Restatement}
provides in relevant part:
\begin{quotation}
A servitude is created

(1) if the owner of the property to be burdened
\begin{statute}
\item (a) enters into a contract or makes a conveyance intended to create a
servitude that complies with\ldots [the] Statute of Frauds\ldots or\ldots [a
recognized e]xception to the Statute of Frauds\ldots ; or

\item (b) conveys a lot or unit in a general-plan development or common-interest
community subject to a recorded declaration of servitudes for the development or
community; or
\end{statute}
(2) if the requirements for creation of a servitude by estoppel, implication,
necessity, or prescription\ldots are met\ldots .
\end{quotation}
A few features of the \textsc{Restatement} approach are worth noting. The first
is that
the common law's requirement of ``horizontal privity of estate''---that the
covenant be created in an instrument that conveys some interest in real property
between the original covenantor and the original covenantee\footnote{Thus, at
common law, if B promised to use her land only for residential purposes
\textit{in a deed from A to B}, A and B would be in horizontal privity
of estate with one another. However, if A and B simply entered into a
\textit{contract} whereby A paid B a sum of money in exchange for B's promise to
use her land only for residential purposes, they would not be in horizontal
privity of estate---because no interest in real property passed under the
contract.}---is eliminated. Under the \textsc{Restatement} view, a contract
containing
the covenant is sufficient to bind successors, even if it passes no
\textit{other} property interest, so long as the parties intended the covenant
to run with the land. (Under this view, a covenant intended to bind successors
is \textit{itself} a sufficient interest in land.) Second, there is a deep
connection between covenants that run with the land and ``common-interest
communities''---a property law institution that we will investigate further in a
later chapter. Third, the Restatement elsewhere treats the common law
requirement of notice as essentially a matter for the recording system, making
the unenforceability of covenants for want of notice subject to the same rules
as any other property interest. \emph{See} \textsc{Restatement} \S~7.14.

Finally, the \textsc{Restatement} rejects, with heavy criticism, the common law
requirement that a restrictive covenant ``touch or concern'' land.
\textsc{Restatement} \S~3.1 cmt. a. Nevertheless, many jurisdictions continue
to apply touch-and-concern doctrine, sometimes explicitly declining to follow
the Restatement approach. \textit{See} Note, \emph{Touch and Concern, the
Restatement
(Third) of Property: Servitudes, and a Proposal}, 122 \textsc{Harv. L. Rev.}
938, 942--45 (2009). It is worth comparing the two approaches.

