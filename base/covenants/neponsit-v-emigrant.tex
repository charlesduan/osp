\reading{Neponsit Property Owners' Ass'n v. Emigrant Industrial Savings Bank}

\readingcite{15 N.E.2d 793 (N.Y. 1938)}

\opinion \textsc{Lehman}, Judge.

The plaintiff, as assignee of Neponsit Realty Company, has brought this action
to foreclose a lien upon land which the defendant owns. The lien, it is alleged,
arises from a covenant, condition or charge contained in a deed of conveyance of
the land from Neponsit Realty Company to a predecessor in title of the
defendant. The defendant purchased the land at a judicial sale. The referee's
deed to the defendant and every deed in the defendant's chain of title since the
conveyance of the land by Neponsit Realty Company purports to convey the
property subject to the covenant, condition or charge contained in the original
deed\ldots .

Upon this appeal the defendant contends that the land which it owns is not
subject to any lien or charge which the plaintiff may enforce. Its arguments are
confined to serious questions of law.\ldots On this appeal we may confine our
consideration to the merits of these questions, and, in our statement of facts,
we drew indiscriminately from the allegations of the complaint and the
allegations of the answer.

It appears that in January, 1911, Neponsit Realty Company, as owner of a tract
of land in Queens county, caused to be filed in the office of the clerk of the
county a map of the land. The tract was developed for a strictly residential
community, and Neponsit Realty Company conveyed lots in the tract to purchasers,
describing such lots by reference to the filed map and to roads and streets
shown thereon. In 1917, Neponsit Realty Company conveyed the land now owned by
the defendant to Robert Oldner Deyer and his wife by deed which contained the
covenant upon which the plaintiff's cause of action is based.

That covenant provides:
\begin{quotation}
And the party of the second part for the party of the second part and the
heirs, successors and assigns of the party of the second part further covenants
that the property conveyed by this deed shall be subject to an annual charge in
such an amount as will be fixed by the party of the first part, its successors
and assigns, not, however exceeding in any year the sum of four (\$4.00) Dollars
per lot 20x100 feet. The assigns of the party of the first part may include a
Property Owners' Association which may hereafter be organized for the purposes
referred to in this paragraph, and in case such association is organized the
sums in this paragraph provided for shall be payable to such association. The
party of the second part for the party of the second part and the heirs,
successors and assigns of the party of the second part covenants that they will
pay this charge to the party of the first part, its successors and assigns on
the first day of May in each and every year, and further covenants that said
charge shall on said date in each year become a lien on the land and shall
continue to be such lien until fully paid. Such charge shall be payable to the
party of the first part or its successors or assigns, and shall be devoted to
the maintenance of the roads, paths, parks, beach, sewers and such other public
purposes as shall from time to time be determined by the party of the first
part, its successors or assigns. And the party of the second part by the
acceptance of this deed hereby expressly vests in the party of the first part,
its successors and assigns, the right and power to bring all actions against the
owner of the premises hereby conveyed or any part thereof for the collection of
such charge and to enforce the aforesaid lien therefor.

These covenants shall run with the land and shall be construed as real
covenants running with the land until January 31st, 1940, when they shall cease
and determine.
\end{quotation}

Every subsequent deed of conveyance of the property in the defendant's chain of
title, including the deed from the referee to the defendant, contained, as we
have said, a provision that they were made subject to covenants and restrictions
of former deeds of record.

There can be no doubt that Neponsit Realty Company intended that the covenant
should run with the land and should be enforceable by a property owners
association against every owner of property in the residential tract which the
realty company was then developing. The language of the covenant admits of no
other construction. Regardless of the intention of the parties, a covenant will
run with the land and will be enforceable against a subsequent purchaser of the
land at the suit of one who claims the benefit of the covenant, only if the
covenant complies with certain legal requirements. These requirements rest upon
ancient rules and precedents. The age-old essentials of a real covenant, aside
from the form of the covenant, may be summarily formulated as follows: (1) It
must appear that grantor and grantee intended that the covenant should run with
the land; (2) it must appear that the covenant is one ``touching'' or
``concerning'' the land with which it runs; (3) it must appear that there is
``privity of estate'' between the promisee or party claiming the benefit of the
covenant and the right to enforce it, and the promisor or party who rests under
the burden of the covenant\ldots .

The covenant in this case is intended to create a charge or obligation to pay a
fixed sum of money to be ``devoted to the maintenance of the roads, paths, parks,
beach, sewers and such other public purposes as shall from time to time be
determined by the party of the first part [the grantor], its successors or
assigns.'' It is an affirmative covenant to pay money for use in connection with,
but not upon, the land which it is said is subject to the burden of the
covenant. Does such a covenant ``touch'' or ``concern'' the land?\ldots In truth
such a description or test so formulated is too vague to be of much assistance
and judges and academic scholars alike have struggled, not with entire success,
to formulate a test at once more satisfactory and more accurate. ``It has been
found impossible to state any absolute tests to determine what covenants touch
and concern land and what do not. The question is one for the court to determine
in the exercise of its best judgment upon the facts of each case.'' Clark, op.
cit. p. 76.

Even though that be true, a determination by a court in one case upon particular
facts will often serve to point the way to correct decision in other cases upon
analogous facts. Such guideposts may not be disregarded. It has been often said
that a covenant to pay a sum of money is a personal affirmative covenant which
usually does not concern or touch the land. Such statements are based upon
English decisions which hold in effect that only covenants, which compel the
covenanter to submit to some \textit{restriction on the use} of his property,
touch or concern the land, and that the burden of a covenant which requires the
covenanter to do an affirmative act, even on his own land, for the benefit of
the owner of a ``dominant'' estate, does not run with his land.\ldots
[Nevertheless s]ome promises to pay money have been enforced, as covenants
running with the land, against subsequent holders of the land who took with
notice of the covenant.\ldots [T]hough it may be inexpedient and perhaps
impossible to formulate a rigid test or definition which will be entirely
satisfactory or which can be applied mechanically in all cases, we should at
least be able to state the problem and find a reasonable method of approach to
it. It has been suggested that a covenant which runs with the land must affect
the legal relations---the advantages and the burdens---of the parties to the
covenant, as owners of particular parcels of land and not merely as members of
the community in general, such as taxpayers or owners of other land. That method
of approach has the merit of realism. The test is based on the effect of the
covenant rather than on technical distinctions. Does the covenant impose, on the
one hand, a burden upon an interest in land, which on the other hand increases
the value of a different interest in the same or related land?

Even though we accept that approach and test, it still remains true that whether
a particular covenant is sufficiently connected with the use of land to run with
the land, must be in many cases a question of degree. A promise to pay for
something to be done in connection with the promisor's land does not differ
essentially from a promise by the promisor to do the thing himself, and both
promises constitute, in a substantial sense, a restriction upon the owner's
right to use the land, and a burden upon the legal interest of the owner. On the
other hand, a covenant to perform or pay for the performance of an affirmative
act disconnected with the use of the land cannot ordinarily touch or concern the
land in any substantial degree. Thus, unless we exalt technical form over
substance, the distinction between covenants which run with land and covenants
which are personal, must depend upon the effect of the covenant on the legal
rights which otherwise would flow from ownership of land and which are connected
with the land. The problem then is: Does the covenant in purpose and effect
substantially alter these rights?

\ldots Looking at the problem presented in this case\ldots and stressing the
intent and substantial effect of the covenant rather than its form, it seems
clear that the covenant may properly be said to touch and concern the land of
the defendant and its burden should run with the land. True, it calls for
payment of a sum of money to be expended for ``public purposes'' upon land other
than the land conveyed by Neponsit Realty Company to plaintiff's predecessor in
title. By that conveyance the grantee, however, obtained not only title to
particular lots, but an easement or right of common enjoyment with other
property owners in roads, beaches, public parks or spaces and improvements in
the same tract. For full enjoyment in common by the defendant and other property
owners of these easements or rights, the roads and public places must be
maintained. In order that the burden of maintaining public improvements should
rest upon the land benefited by the improvements, the grantor exacted from the
grantee of the land with its appurtenant easement or right of enjoyment a
covenant that the burden of paying the cost should be inseparably attached to
the land which enjoys the benefit. It is plain that any distinction or
definition which would exclude such a covenant from the classification of
covenants which ``touch'' or ``concern'' the land would be based on form and not
on substance\ldots .

\ldots Another difficulty remains. Though between the grantor and the grantee
there was privity of estate, the covenant provides that its benefit shall run to
the assigns of the grantor who ``may include a Property Owners' Association
which may hereafter be organized for the purposes referred to in this
paragraph.'' The plaintiff has been organized to receive the sums payable by the
property owners and to expend them for the benefit of such owners. Various
definitions have been formulated of ``privity of estate'' in connection with
covenants that run with the land, but none of such definitions seems to cover
the relationship between the plaintiff and the defendant in this case. The
plaintiff has not succeeded to the ownership of any property of the grantor. It
does not appear that it ever had title to the streets or public places upon
which charges which are payable to it must be expended. It does not appear that
it owns any other property in the residential tract to which any easement or
right of enjoyment in such property is appurtenant. It is created solely to act
as the assignee of the benefit of the covenant, and it has no interest of its
own in the enforcement of the covenant.

The arguments that under such circumstances the plaintiff has no right of action
to enforce a covenant running with the land are all based upon a distinction
between the corporate property owners association and the property owners for
whose benefit the association has been formed. If that distinction may be
ignored, then the basis of the arguments is destroyed. How far privity of estate
in technical form is necessary to enforce in equity a restrictive covenant upon
the use of land, presents an interesting question. Enforcement of such covenants
rests upon equitable principles, and at times, at least, the violation ``of the
restrictive covenant may be restrained at the suit of one who owns property or
for whose benefit the restriction was established, irrespective of whether there
were privity either of estate or of contract between the parties, or whether an
action at law were maintainable.'' \emph{Chesebro v. Moers}, 233 N.Y. 75, 80,
134 N.E. 842, 843, 21 A.L.R. 1270.\ldots We do not attempt\ldots to formulate a
definite rule as to when, or even whether, covenants in a deed will be enforced,
upon equitable principles, against subsequent purchasers with notice, at the
suit of a party without privity of contract or estate. There is no need to
resort to such a rule if the courts may look behind the corporate form of the
plaintiff.

The corporate plaintiff has been formed as a convenient instrument by which the
property owners may advance their common interests. We do not ignore the
corporate form when we recognize that the Neponsit Property Owners' Association,
Inc., is acting as the agent or representative of the Neponsit property owners.
As we have said in another case: when Neponsit Property Owners' Association,
Inc., ``was formed, the property owners were expected to, and have looked to
that organization as the medium through which enjoyment of their common right
might be preserved equally for all.'' \emph{Matter of City of New York, Public
Beach, Borough of Queens}, 269 N.Y. 64, 75, 199 N.E. 5, 9. Under the conditions
thus presented we said: ``It may be difficult, or even impossible to classify
into recognized categories the nature of the interest of the membership
corporation and its members in the land. The corporate entity cannot be
disregarded, nor can the separate interests of the members of the corporation''
(page 73, 199 N.E. page 8). Only blind adherence to an ancient formula devised
to meet entirely different conditions could constrain the court to hold that a
corporation formed as a medium for the enjoyment of common rights of property
owners owns no property which would benefit by enforcement of common rights and
has no cause of action in equity to enforce the covenant upon which such common
rights depend. Every reason which in other circumstances may justify the ancient
formula may be urged in support of the conclusion that the formula should not be
applied in this case. In substance if not in form the covenant is a restrictive
covenant which touches and concerns the defendant's land, and in substance, if
not in form, there is privity of estate between the plaintiff and the
defendant\ldots .

