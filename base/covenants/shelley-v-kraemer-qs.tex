\expected{shelley-v-kraemer}
\expected{narrative-redlining-history}

\item Racially restrictive covenants were widespread in the United States in the
first half of the twentieth century. \textit{See generally} Michael
Jones-Correa, \textit{The Origins and Diffusion of Racial Restrictive
Covenants}, 115 \textsc{Pol. Sci. Q.} 541 (2001). Indeed, just two decades prior
to its decision in \textit{Shelley}, in the case of \textit{Corrigan v.
Buckley}, 271 U.S. 323 (1926), the Supreme Court had affirmed the enforcement of
such a covenant (against the original covenantor) in the District of Columbia
(on grounds that the Equal Protection Clause of the 14th Amendment was
inapplicable to the federal government---a proposition the Court retreated from
in \textit{Bolling v. Sharpe}, 347 U.S. 497 (1954)). And, as discussed in our
unit on Redlining, in the years leading up to
\textit{Shelley} it was federal government policy to encourage mortgage lenders
to insist on the inclusion of racially restrictive covenants in the deeds to
homes that were to serve as collateral for federally-insured loans.


Note that three justices recused themselves from consideration of
\textit{Shelley}. Justice John Paul Stevens, in his memoir, surmises that they
had to do so because they owned homes burdened (and, in the view of many white
Americans of the day, benefited) by racially restrictive covenants.
\textsc{Justice John Paul Stevens, Five Chiefs: A Supreme Court Memoir} 69
(2011).



\item Does \textit{Shelley} provide useful guidance on what types of privately
agreed restrictions will be enforced and what types will go unenforced on
constitutional or public policy grounds? Does the Restatement do any better?

\item Like racism, racially restrictive covenants have not gone away. Though
unenforceable in court, they remain in the chain of title of much residential
real estate today, and linger in historical title records. Several universities
and public interest organizations have undertaken the work of identifying these
lingering covenants in the hopes of removing them from title records. Examples
include the University of Minnesota's Mapping Prejudice project
(\url{https://www.mappingprejudice.org/}), the University of Washington's
Seattle Civil Rights \& Labor History Project
(\url{https://depts.washington.edu/civilr/covenants.htm}), and Prologue DC's
Mapping Segregation project (\url{https://www.mappingsegregationdc.org/}). 

In the wake of white supremacist violence in Charlottesville, Virginia, in
August of 2017, Charlottesville resident and legal commentator Dahlia Lithwick
recounted: 
\begin{quote}
Our lawyer once told us, when we purchased our home in Charlottesville, that
the house to this day carries a racially restrictive covenant. No blacks, no
Jews. That covenant is illegal and unenforceable. And so I have a house in
Charlottesville that could once have been taken from me by the force of law.
\end{quote}
Dahlia Lithwick, \textit{They Will Not Replace Us}, \textsc{Slate} (Aug. 13,
2017),
\url{http://www.slate.com/articles/news_and_politics/politics/2017/08/dahlia_lithwick_on_the_nazis_in_charlottesville.html}.
The white supremacists had descended on Charlottesville as a show of force
centered on an equestrian statute of Confederate general Robert E. Lee, which
the city had voted to remove. Jacey Fortin, \textit{The Statue at the Center of
Charlottesville's Storm}, \textsc{N.Y. Times} (Aug. 13, 2017),
\url{https://www.nytimes.com/2017/08/13/us/charlottesville-rally-protest-statue.html}.
In recent years, the law's treatment of racially restrictive covenants has come
to take on some of the features of the culture war over the removal of
Jim-Crow-era monuments to the Confederacy.

As we will see in the next section, removing a restrictive covenant from a chain
of title can be quite difficult. In recent years, as attention has been drawn to
the perpetuation of unenforceable racially restrictive covenants in title
records, a number of states have enacted laws to make it easier---and in some
cases mandatory---to file replacement deeds and other title documents with such
covenants removed or stricken. \textit{See, e.g.}, \textsc{Cal. Gov't Code}
\S~12956.2;
\textsc{Cal. Civ. Code} \S~4225; \textsc{Md. Code Ann., Real Prop.} \S\S~3-112,
11B-113.3; \textsc{Minn.
Stat.} \S~507.18; \textsc{Nev. Rev. Stat.} \S~111.237(3); \textsc{Va. Code Ann.}
\S~36-96.6.
Note that under such statutes, the original instrument containing the covenant
is not \textit{removed} from the title records; the new document is simply
\textit{added} to the record with a reference to the location of the original,
while the index is amended to point to the modified instrument rather than (or
in addition to) the original instrument.

In the absence of such statutory intervention, however, it can be a challenge to
remove racially restrictive language from title documents, even though---and
perhaps even \textit{because}---such language is unenforceable. In \textit{Mason
v. Adams Cty. Recorder}, 901 F.3d 753 (6th Cir. 2018), a suit seeking to compel
county recorders in Ohio to ``stop printing and publishing historical documents
that contain racially restrictive covenants, to remove all such records from
public view, and to permit the inspection and redaction of such documents'' had
been dismissed for lack of standing. In an opinion by Judge Boggs, the Court of
Appeals affirmed the dismissal, explaining:
\begin{quote}
In ancient Rome, the practice of \textit{damnatio memoriae}, or the condemnation
of memory, could be imposed on felons whose very existence, including
destruction of their human remains, would literally be erased from history for
the crimes they had committed. Land title documents with racially restrictive
covenants that we now find offensive, morally reprehensible, and repugnant
cannot be subject to \textit{damnatio memoriae}, as those documents are part of
our living history and witness to the evolution of our cultural norms. Mason's
feeling of being unwelcomed may be real. A feeling cannot be unfelt. But Mason's
discomfort at the expression of historical language does not create
particularized injury. The language in question is purely historical and is
unenforceable and irrelevant in present-day land transactions.
\end{quote}
901 F.3d at 757 (footnote omitted). In a concurrence, Judge Clay agreed that the
plaintiff had not adequately pleaded a particularized injury, but held open the
possibility that he \textit{could} do so: 
\begin{quote}
Justice may require us to repudiate or revise elements of our ``living history''
if those elements---whether they be public records, flags, or statutes---are
shown to encourage or perpetuate discrimination or the badges and incidents of
slavery; indeed, racial epithets that were once accepted as commonplace have not
been preserved, and they have sometimes been stricken from our modern
vernacular. We apply an even stricter standard where, as here, the government is
the source of, or has ratified, language that has the purpose or effect of
encouraging racial animus. We need not erase our history in order to disarm its
harmful legacy, but victims of invidious discrimination who have suffered
particularized injury as a result of the application of historical language
should be able to seek redress, consistent with the context and the factual
circumstances of their cases.
\end{quote}
\textit{Id.} at 758 (Clay, J. concurring). The debate between these two
opinions---over the nature and gravity of the harms caused by the persistence of
racist symbols, the appropriate response to those harms, and the nature of our
obligations to preserve historical memory---is strikingly (and probably
intentionally) similar to the debate over the removal of Confederate monuments.
Is this debate helpful in determining what to do about title records? Are the
issues presented by title records the same as those presented by statues of
Confederate generals? If not, how do they differ?


\item Precisely because they remain in the chain of title for many parcels of
real property, these types of discriminatory covenants still occasionally lead
to disputes, particularly where residents continue to believe they are a good
idea. For example, the Long Island, NY village of Yaphank, founded in the 1930s
as ``Camp Siegfried,'' owes its origin to the expression of Nazi sympathies.
This German-American community started as a summer camp for would-be Hitler
Youth, and was financed by the German-American Bund party (a pro-Nazi
organization). \textit{See} Nicholas Casey, \textit{Buyers' Rule in L.I. Town Is
Relic of Its Nazi Past}, \textsc{N.Y. Times}, Oct. 20, 2015, at A1,
\url{https://nyti.ms/2ktUqEW}. The land comprising village's residential
subdivision of about 50 homes is actually owned by the ``German American
Settlement League, Inc.'' whose bylaws restricted residency to League members,
and restricted League membership ``primarily'' to people ``of German
extraction.'' Yaphank residents owned, and could sell, the \textit{structures}
on their lots, but in order to take possession the buyer needed a lease to the
underlying land, which the League controlled.

In 2015, the nonprofit advocacy organization Long Island Housing Services sued
the German American Settlement League on behalf of two homeowners (both of
German extraction) who were having difficulty selling their home subject to the
restrictions. In 2016, the plaintiffs secured a settlement in which the League
agreed to remove the racial restrictions from its bylaws and to comply with fair
housing laws. LIHS's complaint is available here,
\url{https://storage.courtlistener.com/recap/gov.uscourts.nyed.376778.1.0.pdf}
and the settlement is available here,
\url{https://storage.courtlistener.com/recap/gov.uscourts.nyed.376778/gov.uscourts.nyed.376778.12.0.pdf}.

