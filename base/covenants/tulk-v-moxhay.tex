\reading{Tulk v. Moxhay}

\readingcite{[1845] 47 Eng. Rep. 1345}

\captionedgraphic{covenants-img001}{Leicester Square in the 18th Century.
Source: \textsc{John Hollingshead, The Story of Leicester
Square} 19 (1892), British Library Online,
\protect\url{http://access.bl.uk/item/pdf/lsidyv3c48bb3a}.}

This was a motion by way of appeal from the Master of the Rolls to dissolve an
injunction.

In the month of July 1808, the Plaintiff was seised in fee-simple not only of
the piece of ground which formed the open space or garden in Leicester Square,
but also of several houses situated in that square.

By an indenture of release, dated the 15th of July 1808, and made between the
Plaintiff, of the one part, and Charles Elms, of the other part, after reciting
that the Plaintiff was seised of that piece of land in fee-simple, and had
contracted to sell it to Elms, but not reciting that that contract was made
subject to any condition, in consideration of {\pounds}210, the Plaintiff
conveyed to Elms, in fee-simple, ``all that piece or parcel of land, commonly
called Leicester Square Garden or pleasure-ground, with the equestrian statue
then standing in the centre thereof, and the iron railings and stone- work round
the garden, and all easements or ways, \&c., to hold the same to Elms, his heirs
and assigns for ever.'' And in that indenture there was contained a covenant by
Elms, in the words following:--- ``And the said Charles Elms, for himself, his
heirs, executors, administrators, and assigns, doth covenant, promise, and agree
to and with the said Charles Augustus Tulk, his heirs, executors, and
administrators, in manner following---that is to say, that he, the said Charles
Elms, his heirs and assigns, shall and will, from time to time, and at all times
for ever hereafter, at his and their own proper costs and charges, keep and
maintain the said piece or parcel of ground and square garden, and the iron
railing round the same, in its present form, and in sufficient and proper repair
as a square garden and pleasure-ground, in an open state, uncovered with any
buildings, in a neat and ornamental order; and shall not nor will take down, nor
permit or suffer to be taken down or defaced, at any time or times hereafter,
the equestrian statue now standing or being in the centre of the said square
garden, but shall and will continue and keep the same in its present situation,
as it now is; and also, that it shall be lawful to and for the inhabitants of
Leicester Square aforesaid, tenants of the said Charles Augustus Tulk, and of
John Augustus Tulk, Esq., his father, their heirs and assigns, as well as the
said Charles Augustus Tulk and John Augustus Tulk, their heirs and assigns, on
payment of a reasonable rent for the same, to have keys (at their own expense),
and the privilege of admission therewith annually, at any time or times, into
the said square garden and pleasure-ground.''

The bill then stated, that\ldots the Defendant had become the owner of that
piece of ground by Virtue of a title derived from Elms [through several
successive conveyances]; and that he had formed a plan, or scheme for erecting
certain lines of shops and buildings thereon; but that the Plaintiff objected to
such scheme, as being contrary to the aforesaid covenant, and injurious to the
Plaintiff's houses in the square; that the Defendant had, nevertheless,
proceeded to cut down several of the trees and shrubs, and had pulled down part
of the iron railing, and had erected a hoarding or boards across the said piece
of ground.

The bill charged, that, at the time when the Defendant purchased the piece of
ground, and also when he took possession thereof, and also when he committed the
acts complained of, he had notice of the covenant.

The bill prayed, that the Defendant, and his agents and workmen, might be
restrained from\ldots doing or committing, or permitting or suffering to be done
or committed, any waste, spoil, destruction, or nuisance to be in or upon the
said piece of garden ground.

An \textit{ex parte} injunction was obtained from the Master of the Rolls, and
the Defendant\ldots by his answer, stated, that the inhabitants of Leicester
Square and of the Plaintiff's houses had entirely ceased to use this piece of
ground as a garden and pleasure-ground, or to pay any sum for the privilege of
admission; and that, for many years before the Defendant purchased it, it had
been in a ruinous condition, and not in an ornamental state, but altogether out
of repair; that Tulk never took any steps to enforce the covenant, or to have
the site of the ground improved; that the square was no longer a quiet place of
residence, but that a thoroughfare had lately been made through it from Long
Acre to Piccadilly; that he proposed to open two footpaths diagonally across the
square, putting up gates and fences; that he had not yet fixed on any plan for
building on it; or as to the ultimate use he should make of it; but he reserved
by his answer the right to make all such use of the land as he might thereafter
think fit, and lawfully could do; and he also submitted to the Court, that the
covenant did not run with the land, and did not bind him as assignee.

The Defendant applied to the Master of the Rolls to dissolve the injunction,
which his Lordship refused to do\ldots . The effect of the injunction, as
varied, was to restrain the Defendant, his workmen, \&c., from converting or
using the piece of ground and square garden in the bill mentioned, and the iron
railing round the same, to or for any other purpose than as a square garden and
pleasure-ground, in an open state, uncovered with buildings, until the hearing
of this cause, or the further order of this Court.

The motion to dissolve the injunction was now renewed before the Lord
Chancellor.\ldots

\opinion The Lord Chancellor [Cottenham].

\ldots It is not disputed that a party selling land may, by some means or other,
provide that the party to whom he sells it shall conform to certain rules, which
the parties may think proper to lay down as between themselves. They may so
contract as to bind the party purchasing to deal with the land according to the
stipulation between him and the vendor\ldots . Here, then, upon the face of the
instrument, and in a manner free from doubt\ldots the owner of the houses sells
and disposes of land adjoining to those houses with an express covenant on the
part of the purchaser, his heirs and assigns, that there shall be no buildings
erected upon that land. It is now contended, not that Elms, the vendee, could
violate that contract---not that he could build immediately after he had
covenanted not to build, or that this Court could have had any difficulty, if he
had made that attempt, to prevent him from building---but that he might sell
that piece of land as if it were not incumbered with that covenant; and that the
person to whom he sold it might at once, without the risk of the interference of
this Court, violate the covenant of the party from whom he purchased it.

Now, I do not apprehend that the jurisdiction of this Court is fettered by the
question, whether the covenant runs with the land or not. The question is,
whether a party taking property with a stipulation to use it in a particular
manner---that stipulation being imposed on him by the vendor in such a manner as
to be binding by the law and principles of this Court---will be permitted by
this Court to use it in a way diametrically opposite to that which the party has
stipulated for.\ldots Of course, the party purchasing the property, which is
under such restriction, gives less for it than he would have given if he had
bought it unincumbered. Can there, then, be anything much more inequitable or
contrary to good conscience, than that a party, who takes property at a less
price because it is subject to a restriction, should receive the full value from
a third party, and that such third party should then hold it unfettered by the
restriction under which it was granted? That would be most inequitable, most
unjust, and most unconscientious; and, as far as I am informed, this Court never
would sanction any such course of proceeding; but, on the contrary, it has
always acted upon this principle, that you, who have the property, are bound by
the principles and law of this Court to submit to the contract you have entered
into; and you will not be permitted to hand over that property, and give to your
assignee or your vendee a higher title, with regard to interest as between
yourself and your vendor, than you yourself possess.

That is quite unconnected with the doctrine of a covenant running with the
land.\ldots There is no question about the legal liability, which is best proved
by this: that if there be a merely legal agreement, and no covenant---no
question about the covenant running with the land---the party who takes the land
takes it subject to the equity which the owner of the property has created: and
if he takes it, subject to that equity, created by those through whom he has
derived a title to it, is it not the rule of this Court, that the party, who has
taken the property with knowledge of the equity, is liable to the equity? Is not
this an equity attached to the property, by the party who is competent to bind
the property? If a party enters into an agreement for a lease, and then sells
the property which was to be demised, the purchaser of that property, with
knowledge of the agreement, cannot set up his title against the party claiming
the benefit of that contract; because, if there had been an equity attaching to
the property in the owner, the owner is not permitted to give a better title to
the purchaser with notice than he himself possesses. The other party is entitled
to the benefit of the contract, and to have it exercised and carried into effect
against the person who is in possession, unless that person can shew he
purchased it without notice. Here there is a clear, distinct, and admitted
equity in the vendor, as against Mr. Elms; and as to the party now sought to be
affected by it, it is not in dispute that he took the land with notice of the
covenant: indeed, it appears on the face of the instrument which is the
foundation of his title. It seems to me to be the simplest case that a Court of
Equity ever acted upon, that a purchaser cannot have a better title than the
party under whom he claims.

Without adverting to any question about a covenant running with land or not, I
consider that this piece of land is purchased subject to an equity created by a
party competent to create it; that the present Defendant took it with distinct
knowledge of such equity existing; and that such equity ought to be enforced
against him, as it would have been against the party who originally took the
land from Mr. Tulk.

\ldots I think, therefore, that the Master of the Rolls is quite right\ldots and
that this motion must be refused, with costs.


