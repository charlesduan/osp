\expected{tulk-v-moxhay}

\item Is the result in \textit{Tulk} attributable to a difference in the
willingness of courts of equity (as compared to courts of law) to find a
covenant will ``run with the land''? To the principle of
\textit{\useterm{nemo dat}}? To
the rules regarding good faith purchasers? To something else?

\expected{intro-marketability}

\item Is the result in \textit{Tulk} consistent with the principle of
\textit{\useterm{numerus clausus}}? With the common-law policy against
\useterm{restraints on alienation}?

\item \textit{Tulk v. Moxhay} represented a new opening for private ordering
regarding competing land uses, which hinged on the distinction between law and
equity. In the end, the equitable exception swallowed the legal rule against
restrictive covenants running with land. As one court explained:
\begin{quotation}
In the past, some courts\ldots have distinguished between a ``real covenant''
that runs with the land and an ``equitable covenant'' (sometimes called an
``equitable servitude'' or ``equitable restriction'') that runs with the land.
Today however, the \textit{Restatement} [\textit{(Third), Property
(Servitudes)}] sensibly explains:

[T]he differences between covenants that historically could be enforced at law
and those enforceable in equity\ldots have all but disappeared in modern law.
Continuing use of the dual terminology of real covenant and equitable servitude
is confusing because it suggests the continued existence of two separate
servitude categories with important differences. In fact, however, in modern law
there are no significant differences. Valid covenants, like other contracts and
property interests, can be enforced and protected by both legal and equitable
remedies as appropriate, without regard to the form of the transaction that
created the servitude.
\end{quotation}
\emph{Lake Limerick Country Club v. Hunt Mfg. Homes, Inc.}, 120 Wash. App. 246,
253-54, 84 P.3d 295, 298-99 (2004) (footnotes omitted).

\item It is worth noting again that the \textsc{Third Restatement}, quoted in
\textit{Lake Limerick Country Club}, is somewhat unique in not simply restating
the law but also pushing it in a particular direction. Many jurisdictions have
yet to adopt its more modern approach on merging the various servitudes, or on
other important issues. As always in property law, it is important to consult
the relevant authorities in your jurisdiction in order to determine whether
courts there still follow more traditional rules regarding the creation,
enforcement, modification, and termination of restrictive covenants.

\expected{narrative-coase-theorem}

\item \textbf{Coase Revisited}. Which way do the equities really cut in
\textit{Tulk}? Lord Chancellor Cottenham concluded that it was unfair for Moxhay
to deprive Tulk of the benefit of his bargain with Elms. Couldn't we just as
easily say it is unfair for Tulk to interfere with Moxhay's use of the land he
purchased? Indeed, given that English law courts of the time typically refused
to hold that restrictive covenants would run with the land, doesn't Moxhay have
the stronger equitable case? Wasn't it unreasonable for Tulk to expect he could
obtain an enforceable covenant \textit{from Elms alone} on behalf of Elms's
``heirs, executors, administrators, and assigns''? 

\item Put another way, isn't the problem here \textit{reciprocal} in that the
parties simply have incompatible land use preferences? Thus, when Lord Cottenham
rhetorically asks, ``Is not this an equity attached to the property, by the
party who is competent to bind the property?'' is he merely assuming the initial
allocation of the relevant entitlement to the party that was there first? If so,
is the application of a restrictive covenant to successors a circumstance in
which the parties could effectively bargain to reach the efficient result?

\item Recall the dispute between Abigail, Beatrice, and Clara on page
\pageref{abigail-beatrice-clara}. Does the
principle of ``first in time is first in right'' provide any reason to privilege
Abigail's preferred use of Clara's land over Clara's preferred use? Does the
fact that Abigail and Beatrice reached their agreement \textit{before} Clara
became involved suggest that, as a matter of general property law principles,
later comers will have to either abide by that agreement or obtain \textit{both
parties'} consent to abrogate it? Is such a rule necessary to protect Abigail's
legitimate expectations with respect to the use and enjoyment of her own
property? 

\item More generally, are the arguments supporting the principle of priority in
time persuasive when applied to land \textit{use} conflicts (as opposed to
disputes over \textit{title} or \textit{possession})? Conversely, if we
\textit{do} allow agreements like the one Abigail and Beatrice to run with the
land, are we giving past owners too much control over the ability of present and
future owners to adapt their land uses to changing circumstances?

