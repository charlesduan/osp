Restrictive covenants, like easements, can be modified or terminated in many
ways. The Restatement mostly does not draw a distinction between these two types
of servitudes with respect to modification or
termination\having{intro-terminating-easements}{, meaning that the grounds for
termination discussed in \mref{intro-terminating-easements}---merger, agreement,
abandonment, etc.---apply with equal force to restrictive covenants}{, meaning
that the grounds for termination discussed in
\mref{intro-terminating-easements}---merger, agreement, abandonment,
etc.---apply with equal force to restrictive covenants}{}. Note,
however, that where a covenant benefits and burdens multiple lots simultaneously
(as in \textit{Neponsit}), these grounds for termination will be inordinately
more difficult to satisfy, simply because more parties must give their consent
or acquiescence and thus any one of them could effectively veto the covenant's
termination.

One basis for modification or termination that is perhaps more likely to arise
with respect to restrictive covenants than it is for easements is that
conditions of the land have changed to such an extent that continued enforcement
is inappropriate. This is particularly so where the restrictive covenants are
part of a common scheme or plan for a community---precisely the circumstance in
which other means of termination are likely to be difficult. In such a
community, what types of changes to ``facts on the ground'' should justify
terminating the covenants shaping the community's land uses?

