\expected{restatement-servitudes}
\expected{neponsit-v-emigrant}

\item Is the rationale of the touch-and-concern requirement discussed in
\textit{Neponsit} reflected in Section 3.1 of the \textsc{Restatement}? If not,
are there
other features of Section 3.1 that serve the common-law rules designed to ensure
marketability of real property?

\item The \textsc{Restatement}'s invalidation of servitudes that impose ``an
unreasonable
restraint on alienation'' draws further distinctions between ``direct'' and
``indirect'' restraints. ``Direct'' restraints---including overt prohibitions on
lease or transfer, rights to withhold consent, options to purchase, and rights
of first refusal---are valid if ``reasonable,'' with reasonableness being
determined ``by weighing the utility of the restraint against the injurious
consequences of enforcing the restraint.'' \textsc{Restatement} \S~3.4. An
``indirect'' restraint is any other restriction on use that might incidentally
``limit[] the numbers of potential buyers or\ldots reduc[e] the amount the owner
might otherwise realize on a sale of the property,'' and such a covenant is
valid unless it ``lacks a rational justification.'' \textit{Id.} \S~3.5 \& cmt.
a.

\item In the late 2000s, as the financial crisis and the collapse of the housing
market dealt crippling blows to the construction industry, one firm came up with
what it thought was a clever solution that built on the same securitization
model that powered the mortgage market in the run-up to the collapse. The firm,
Freehold Capital Partners, advised real estate developers to insert a covenant
in all the deeds to lots in their new housing subdivisions that would require
the purchaser and their successors to pay a portion of the resale price
\textit{to the developer} on every subsequent transfer of the property.
\textit{See} Robbie Whelan, \textit{Home-Resale Fees Under Attack}, \textsc{Wall
St. J}. (July 30, 2010),
\url{http://www.wsj.com/articles/SB1000 14240 52748 70331 49045 75399 29051
18023 82}.
The plan was to securitize these ``private transfer fee'' payments: sell off
slices of the right to the income stream from the transfer fees, and use the
sale price of the securities to finance the construction of the homes that would
be encumbered by the private transfer fee covenants. The scheme as conceived
would not necessarily require the developer to retain title to any real property
in the developments bound by these covenants.


Realtors, title search agencies, legislators, and eventually the federal
government mobilized against this business model. Many states passed statutes
prohibiting or seriously restricting these private fee transfer covenants.
\textit{See, e.g.}, \textsc{Tex. Prop. Code} \S~5.202 (effective June 17, 2011).
As of March 16, 2012, the Federal agencies that repurchase or otherwise backstop
many American residential mortgages will not deal in mortgages on properties
encumbered by such covenants.

Was all this legislative and regulatory action necessary? Would Freehold Capital
Partners' private transfer fee covenants be enforceable under the common law of
restrictive covenants as set forth in \textit{Neponsit}? Under the
\textsc{Restatement}?


\item What other types of covenants might offend public policy? And how far will
public policy intrude on private ordering of property rights? Consider the
following case.
\expectnext{shelley-v-kraemer}

