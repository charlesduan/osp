\expected{el-di-v-bethany-beach}

\item Several types of events may constitute ``changed conditions'' sufficient
to at least trigger an inquiry whether a covenant ought still to be enforceable.
Typical examples include condemnation of the burdened parcel through the power
of eminent domain (typically bringing with it dedication to some purpose outside
the scope of the covenant); zoning or rezoning (which may make the land
incapable of legal use within the scope of the covenant); and nearby
redevelopment that otherwise frustrates the purpose of the covenant.

\item The rule of \textit{El Di} would hold covenants unenforceable for changed
conditions if those conditions ``render[] the benefits underlying imposition of
the restrictions incapable of enjoyment.'' Do residents really derive
\textit{no} benefit from a limit on the available venues for the sale of
alcoholic beverages in their family vacation town? Does anyone else derive a
benefit from such limits? If so, are they the kind of benefits that are
enforceable as a matter of the law of servitudes?

\item There are subtle differences in the framing of the test courts apply under
the doctrine of changed conditions, particularly in the context of the covenants
governing a common-interest community. As the \textsc{Third Restatement} puts
it:
\begin{quote}
The test for finding changed conditions sufficient to warrant termination of
reciprocal-subdivision servitudes is often said to be whether there has been
such a radical change in conditions since creation of the servitudes that
perpetuation of the servitude would be of no substantial benefit to the dominant
estate. However, the test is not whether the servitude retains value, but
whether it can continue to serve the purposes for which it was created.
\end{quote}
\textsc{Restatement} \S~7.10, cmt. c. Do you think the difference between these
two tests
is likely to make a difference in the resolution of disputes? Which (if either)
did the court apply in \textit{El Di?} If \textit{El Di} had applied the other
test, would the outcome have been any different?

\expected{tulk-v-moxhay}

\item Does the mere fact of the disagreement between the majority and the
dissent in \textit{El Di} have any implications for the soundness of the
doctrine of changed conditions? If reasonable minds can differ as to whether a
covenant can still serve its purpose or still provides some benefit to the
dominant owner, might that in itself be a reason to continue enforcing the
parties' private agreement? How does the answer to this question relate to the
public policy limits on enforceability of restrictive covenants? On the danger
of dead-hand control discussed in the notes following \textit{Tulk v. Moxhay?}

