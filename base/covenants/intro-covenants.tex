\expected{intro-negative-easements}

\label{abigail-beatrice-clara}

The historical antipathy of English law toward \textit{negative} easements---the
right of a landowner to \textit{prevent} particular uses of \textit{someone
else's} land---made private ordering over conflicting land uses somewhat
difficult. The basic problem is relatively easy to understand. Suppose Abigail
pays her neighbor Beatrice \$1000 in exchange for a promise that Beatrice will
use her land only for residential purposes, because Abigail does not want to
live next door to a busy commercial or industrial facility. Suppose that
Beatrice then begins to construct a factory on her land. Abigail could sue for
breach of contract and obtain appropriate remedies---perhaps including
injunctive relief barring Beatrice from building the factory.

But now suppose that instead of building a factory herself, Beatrice sells her
land to Clara, who intends to build a factory on the land. Clara didn't promise
Abigail anything, and Abigail gave Clara no consideration---they are not in
privity of contract. We might therefore conclude that Abigail is out of luck:
she cannot enforce a contract against someone who didn't agree to be bound by
it. But if that is our conclusion, there is now a huge obstacle to Abigail and
Beatrice ever reaching their agreement in the first place: how could Abigail
ever trust that her consideration is worth paying if Beatrice can deprive
Abigail of the benefit of the bargain by selling her (Beatrice's) land? More
generally, if a promise to \textit{refrain} from certain uses will not ``run
with the land,'' can private parties ever effectively resolve their disputes
over competing land uses by agreement?

\expected{intro-negative-easements}

Notwithstanding this concern, English courts were historically quite resistant
to enforcing such restrictions against successors to the promisor's property
interest. As you've already learned, only a very small number of negative
easements were recognized. Furthermore, actions at law---seeking the remedy of
money damages---for breach of a covenant restricting the use of land were
available only in quite limited circumstances, in cases involving
landlord-tenant relationships. Early American courts were more willing to
enforce such covenants outside of the landlord-tenant context, but still
required quite strict chains of privity of estate---voluntary transfers of title
by written instruments---before they would enforce such covenants by an action
for money damages. Of course, where the dispute is over competing uses of
neighboring land, perhaps money damages are not the appropriate---or even the
desired---remedy. And herein was the key to substantial liberalization of the
enforcement of restrictive covenants. Eventually, landowners with an interest in
enforcing such covenants found a workaround.

