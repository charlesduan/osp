For much of the 20th century in the United States, many
private lenders refused to extend mortgage credit---and federal agencies
refused to extend or insure mortgage loans---on the basis of the racial
composition of the neighborhood in which the mortgaged property was located.
Community activists in the Austin neighborhood of Chicago coined the term
``redlining'' in the 1960s to describe the phenomenon:
\begin{quote}
Savings and loan associations, at the time the primary source of residential
mortgages, drew red lines around neighborhoods they thought were susceptible to
racial change and refused to make mortgages in those neighborhoods. Using the
U.S. Department of Housing and Urban Development (HUD) appraisal methodology
developed by Homer Hoyt from the University of Chicago, these lending
institutions considered racially changing neighborhoods a bad credit risk
because they assumed property values would decline. By extension, neighborhoods
that were not racially changing but in close proximity to racially changing
neighborhoods were labeled unstable and redlined. The resulting limitations on
the availability of residential credit became a self-fulfilling prophecy as
residents found it difficult to get a fair market price for their homes.
\end{quote}
Jean Pogge, \textit{Reinvestment in Chicago Neighborhoods: A Twenty-Year
Struggle}, \textit{in} \textsc{From Redlining to Reinvestment: Community
Responses to Urban Disinvestment} 133, 134 (Gregory Squires ed., Temple
University Press 2011). 

The term ``redlining'' refers to the practice of lenders, insurers, and
government agencies of drawing literal red lines on city maps around
neighborhoods that were \textit{collectively} deemed an unacceptable credit
risk. This ``location-based'' discrimination is distinct from discrimination
against particular \textit{individuals} on the basis of race, though the two
forms of discrimination often go hand in hand.

In the 1980s, historian Kenneth Jackson showed that the practice of redlining
could be traced back at least as far as the Home Owners Loan Corporation
(HOLC), a federal agency created in 1933 to help stem the tide of foreclosures
generated by the Great Depression. Kenneth T. Jackson, \textit{Race, Ethnicity,
and Real Estate Appraisal: The Home Owners Loan Corporation and the Federal
Housing Administration}, 6 \textsc{J. Urb. Hist.} 419 (1980). HOLC was largely
responsible for introducing and popularizing the type of federally-backed,
fully-amortized, long-term residential mortgage loan that is now the norm in
the United States. But it also helped to systematize and institutionalize the
appraisal theories and methods that gave rise to redlining, even creating what
can now be recognized as the earliest extant redlined maps: HOLC's
``Residential Security Maps,'' such as the maps of Chicago shown in
Figure~\ref{f:chicago-maps}.

\begin{figure}
\begin{centering}
\heregraphic{chicago-map-1}

\heregraphic{chicago-map-2}
\end{centering}
\caption{HOLC maps of Chicago.}
\label{f:chicago-maps}
\end{figure}

\defwebsite{nelson-mapping-inequality}{
author=Robert K. Nelson,
author=LaDale Winling,
author=Richard Marciano,
author=Nathan Connolly et al.,
title=Mapping Inequality,
journal=American Panorama,
editor=Robert K. Nelson,
editor=Edward L. Ayers,
date=dec 11 2023,
edition=version 3.0,
url=https://dsl.richmond.edu/panorama/redlining/,
noetal,
}


The University of Richmond Digital Scholarship Lab's ``Mapping Inequality''
project has recently digitized HOLC's maps and reports, converted them into
standardized geographical data formats, and made the results available on their
website. The project's
authors explain how HOLC's appraisal methods worked in practice:
\begin{quote}
Neighborhoods receiving the highest grade of ``A''---colored green on the
maps---were deemed minimal risks for banks and other mortgage lenders when they
were determining who should receive loans and which areas in the city were safe
investments. Those receiving the lowest grade of ``D,'' colored red, were
considered ``hazardous.'' Conservative, responsible lenders, in HOLC judgment,
would ``refuse to make loans in these areas [or] only on a conservative
basis.'' HOLC created area descriptions to help to organize the data they used
to assign the grades. Among that information was the neighborhood's quality of
housing, the recent history of sale and rent values, and, crucially, the racial
and ethnic identity and class of residents that served as the basis of the
neighborhood's grade. These maps and their accompanying documentation helped
set the rules for nearly a century of real estate practice.
\end{quote}
\sentence{nelson-mapping-inequality}.

Historians and sociologists have long argued over whether the HOLC maps were a
\textit{cause} of redlining, or were instead a \textit{symptom} of the
prevailing theories of real estate value that were widespread at the time that
HOLC was created. The answer is probably a bit of both. HOLC certainly did not
\textit{invent} the practice of valuing real estate on the basis of the racial
makeup of the neighborhood in which such real estate is located. For example, a
leading real estate appraisal textbook from the early 1930s was merely
restating conventional wisdom among real estate professionals when it said that
``racial heritage and tendencies seem to be of paramount importance'' in
influencing land values, and that ``there is one difference in people, namely
race, which can result in a very rapid decline [in real estate values, which]
can be partially avoided by segregation.'' \textsc{Frederick M. Babcock, The
Valuation of Real Estate} 86, 91 (1932). And indeed, HOLC itself did not even
engage in redlining, insofar as it actually did most of its lending in areas it
designated ``declining'' or ``hazardous.'' However, as Jackson argued in his
influential book, \textsc{Crabgrass Frontier: The Suburbanization of the United
States}\textit{ }199 (1985), HOLC and its parent agency, the Federal Home Loan
Bank Board (FHLBB), ``applied these notions of ethnic and racial worth to
real-estate appraising on an unprecedented scale,'' hiring and training armies
of appraisers and analysts to use these racial theories of neighborhood value
in mapping out and categorizing nearly every residential parcel in over 200
American cities. Those professionals, the procedures they developed, and the
maps they prepared, influenced private mortgage lenders as well as another New
Deal agency of more lasting importance: the Federal Housing Administration
(FHA), created in 1934.

FHA---which still exists today as a division of the department of Housing and
Urban Development---plays an outsize role in the American residential mortgage
market. FHA does not itself extend mortgage credit, but instead insures private
mortgage loans that meet certain criteria. This mitigates the risk to lenders
by allowing them to purchase insurance---backed by the full weight of the
federal treasury---against losses arising from non-repayment of residential
mortgage loans they originate. Such risk reduction, in turn, encourages more
lending at lower prices, expanding access to mortgage credit. In order to
qualify for FHA insurance, a mortgage loan must meet certain underwriting
criteria designed by the agency to keep the risk of loss on any given insured
loan within acceptable ranges. The strong demand for FHA insurance among
mortgage lenders means that the agency's underwriting criteria exert tremendous
influence on the availability (and cost) of residential mortgage loans in the
private market. 

\defbook{fha-underwriting-manual}{
url=https://www.google.com/books/edition/_/HTrVAAAAMAAJ,
instauth=Federal Housing Administration,
title=Underwriting and Valuation Procedure Under Title II of the National
Housing Act,
year=1936,
}

\defjrnart{light-discriminating-appraisals}{
Jennifer Light,
Discriminating Appraisals: Cartography, Computation, and Access to
Federal Mortgage Insurance in the 1930s, 52 Technology and Culture 485
(2011)
}
\defgovdoc{hoyt-instructions-dividing}{
author=Homer Hoyt,
agency=Federal Housing Administration,
title=Instructions for Dividing the City into Neighborhoods,
date=nd,
}


FHA's underwriting criteria are collected in a series of manuals for appraisers,
the earliest of which were prepared in large part by Frederick Babcock (who
wrote the 1932 textbook quoted above, and was an early head of FHA's
underwriting division). From its inception, FHA included neighborhood racial
characteristics among its criteria for insurability. The agency's manual for
1936 allocated 20\% of its location risk rating points to ``Protection from
Adverse Influences.'' Among those adverse influences were ``infiltration of
business and industrial uses, lower-class occupancy, and inharmonious racial
groups.''
\sentence{fha-underwriting-manual at part II, paragraphs 226-233}.
With respect to the latter, the agency instructed as follows:
\begin{quote}
The Valuator should investigate areas surrounding the location to determine
whether or not incompatible racial and social groups are present, to the end
that an intelligent prediction may be made regarding the possibility or
probability of the location being invaded by such groups. If a neighborhood is
to retain stability it is necessary that properties shall continue to be
occupied by the same social and racial classes. A change in social or racial
occupancy generally leads to instability and a reduction in values{\dots}. Once
the character of a neighborhood has been established it is usually impossible
to induce a higher social class than those already in the neighborhood to
purchase and occupy properties in its various locations.
\end{quote}
\sentence{fha-underwriting-manual at part II, paragraphs 233}.
The head of FHA's Division of Economics and Statistics instructed staff
members applying these criteria ``to outline blocks with `a considerable
number' of populations commonly associated with low real estate values, such as
`Italians or Jews in the lower income group,' as well as those with 10 percent
or more `negroes or race other than white.'\,''
\sentence{light-discriminating-appraisals at 499 (quoting:
hoyt-instructions-dividing)}. Staff members were similarly encouraged to reflect
local prejudices in downgrading neighborhoods populated by immigrants of other
disfavored national origins and their descendants. \textit{See generally}
Jennifer S. Light, \textit{Nationality and Neighborhood Risk at the Origins of
FHA Underwriting}, 36 \textsc{J. Urb. Hist.} 634 (2010). 

These FHA evaluations looked not only to current conditions but also to likely
future developments, on the theory that once a nonwhite resident entered a
White neighborhood the White residents would relocate to more homogeneously
White neighborhoods in an accelerating cascade---a phenomenon that came to be
known as ``White Flight.''\footnote{In the mid-20th century,
real estate speculators developed a practice of stoking fears of nonwhite
neighbors and falling property values among residents of homogeneously White
neighborhoods in order to profit from changing racial demographics. These
speculators often spread rumors that Black families were moving in to a
neighborhood, hoping to induce White homeowners to sell their homes at a
discount out of fear that if they waited until the racial balance of their
neighborhood had changed, they would have to accept an even lower price. The
speculators often snapped up the homes of these panicky White homeowners for
cash and then resold the homes to Black families at a substantial
markup---often under onerous financing terms such as installment contracts that
Black families were compelled to accept because of their inability to access
mortgage credit through mainstream channels. This practice, known as
``blockbusting,'' accelerated so-called ``white flight'' from urban
neighborhoods to the suburbs from the late 1950s through the 1980s, and is
unlawful under the Fair Housing Act. \textit{See} 42 U.S.C. {\S} 3204(e);
24~C.F.R. \S~100.85.} Jackson notes that ``In a March 1939 map of Brooklyn,
for example, the presence of a single, non-white family on any block was
sufficient to mark that entire block black.'' \textsc{Jackson, Crabgrass
Frontier}, \textit{supra}, at 208\nobreakdash-09. FHA's theory of race and home
values led it to recommend that deeds to parcels of residential real estate
include covenants ``[p]rohibit[ing] of the occupancy of properties except by
the race for which they are intended.''
\sentence{fha-underwriting-manual at part II, paragraph
284/3g}.\footnote{\having{shelley-v-kraemer}{We discussed these racially
restrictive
deed covenants, when the Supreme Court held them unconstitutional in
\emph{Shelley}.}{We will discuss these racially
restrictive deed covenants, held unenforceable as unconstitutional by the
Supreme Court in \textit{Shelley v. Kraemer}, 334 U.S. 1 (1948), in our unit on
Restrictive Covenants.}{Racially restrictive deed covenants were held
unconstitutional in \textit{Shelley v. Kraemer}, 334 U.S. 1 (1948).}
In general, they forbade any owner of property subject
to such a covenant from selling or leasing their property to anybody who was
not White and Christian. Some forbade anyone who was not White and Christian
from using the property as a residence, sometimes excepting domestic servants.
Such covenants were---until \textit{Shelley} was decided---enforceable in an
action by other property owners within the same community for injunctive
relief.}

``There are striking similarities between the survey methods and
questionnaires used by FHA and FHLBB [HOLC's parent agency], as well as their
maps and grading system, and it is hard to imagine that the two agencies were
not working together.'' Amy Hillier, \textit{Redlining and the Homeowners' Loan
Corporation}, 29 \textsc{J. Urban Hist.} 394, 403 (2003). But even though FHA
clearly had access to HOLC's maps, it also clearly conducted its own
independent analyses and even made maps of its own. Indeed, many private
lenders of the period were also using similar methods to generate similar
maps---the predecessors of the maps that became the focus of neighborhood
activists in Chicago in the 1960s. Thus, while the HOLC maps are a vivid
illustration of the redlining era, and HOLC's institutionalization of
race-based neighborhood appraisals did have consequences for other players in
the mortgage market, it seems that ``HOLC was as much a follower as a leader
when it came to neighborhood appraisals.'' Hillier, \textit{supra}, at 412;
\textit{see also} Louis Lee Woods, \textit{The Federal Home Loan Bank Board,
Redlining, and the National Proliferation of Racial Lending Discrimination,
1921--1950}, 38 \textsc{J. Urb. Hist.} 1036, 1038 (2012) (``While the HOLC did
not create racial and socioeconomic lending bias, it certainly helped
nationalize the practice.'').

\defworkingpaper{aaronson-effects-holc}{
author=Daniel Aaronson et al.,
title=The Effects of the 1930s HOLC
``Redlining'' Maps,
publisher=Federal Reserve Bank of Chicago Working Paper Series,
number=No. WP~2017-12,
date=Feb. 2019,
url=http://hdl.handle.net/10419/200568,
}

Whether HOLC was a leader or a follower in tying real estate values and mortgage
availability to the race of neighborhood residents, there is no dispute that
\textit{de jure} redlining was a real phenomenon, nor that the redlining era
had a profound and lasting effect on the housing stock of the United States,
the density of its residential neighborhoods, and the segregation of those
neighborhoods on the basis of race. FHA removed neighborhood racial
characteristics from its underwriting criteria in 1966, and the use of racial
criteria by any public or private residential mortgage lending institution was
rendered unlawful by the Fair Housing Act of 1968, but three decades of
\textit{de jure} redlining had already reshaped the face of American
residential neighborhoods. One recent study found that the Black share of
population in areas graded ``D'' on HOLC maps grew from 1930 to 1970, as it did
in areas graded ``C'' bordering areas graded ``B,'' despite the fact that both
``B'' and ``C'' areas had virtually no Black residents prior to the advent of
HOLC and FHA (the authors refer to this phenomenon as ``yellow-lining'' in
reference to the yellow color-coding of ``C''-graded areas on HOLC
maps). \sentence{aaronson-effects-holc}.
That same study showed that less favorable HOLC rankings were correlated with
falling homeownership rates, falling home values, and rising vacancy rates
during the same period, which the authors argue reflects housing disinvestment
in redlined and even ``yellow-lined'' neighborhoods.
\sentence{aaronson-effects-holc}.

