During the redlining era, neighborhoods where substantial numbers of nonwhite
people lived were often deemed categorically ineligible for FHA-insured loans.
And because federal insurance allowed lenders to offer credit more cheaply,
this meant that residents of predominantly White neighborhoods were able to
borrow more easily and cheaply to buy or improve their homes, while residents
of racially mixed or majority-minority neighborhoods either paid more for
mortgage credit or were not able to access the mortgage market at all. In the
latter case, those who wanted to buy homes in redlined neighborhoods had to
purchase under (often usurious) installment contracts that carried high risks
of default and forfeiture. This systematic deprivation of access to mortgage
credit and housing wealth over the course of decades has had long-term
consequences, both for individuals and, importantly, for neighborhoods.


\defbook{mitchell-franco-holc-redlining}{
author=Bruce Mitchell,
author=Juan Franco,
instauth=National Community Reinvestment Coalition,
title=HOLC ``Redlining'' Maps: The Persistent Structure of Segregation and
Economic Inequality,
date=March 20 2018,
url=https://perma.cc/Z2V5-2R39,
}

Half a century after the FHA disavowed redlining, American residential
communities are still marked by substantial racial segregation. The
\textit{degree} of segregation across HOLC boundaries does seem to have
diminished somewhat in some areas since the advent of the Fair Housing Act.
\sentence{see aaronson-effects-holc}. But that does not mean there has
been broad racial residential integration. In many locations the neighborhood
lines set down by HOLC in the late 1930s bear striking resemblances to
boundaries of residential racial segregation in the 21st
century. One recent study found that in the aggregate, over 85\% of
neighborhoods graded ``A'' in HOLC maps from the late 1930s were majority-White
neighborhoods in 2016, while approximately 64\% of neighborhoods graded ``D''
in those maps were majority-minority in 2016. \sentence{see
mitchell-franco-holc-redlining}.

To visualize this persistence of residential racial segregation, consider the
following comparison of HOLC maps from the 1930s with Racial Dot Maps---maps
that represent individual residents with dots color-coded by census racial
classification---of the same areas in 2010:

\def\redliningset#1#2#3{%
    \readinghead{#1}

    \noindent\hbox{}\hfill
    \usegraphic[width=0.45\textwidth]{redlining-#2}
    \hfill
    \usegraphic[width=0.45\textwidth]{redlining-#3}
    \hfill\hbox{}%

}

\redliningset{Legends}{03}{04}
\redliningset{St.~Louis, Mo.}{05}{06}
\redliningset{Detroit, Mich.}{07}{08}
\redliningset{Brooklyn, N.Y.}{09}{10}
\redliningset{Dallas, Tex.}{11}{12}
\redliningset{Denver, Colo.}{13}{14}
\redliningset{Oakland, Berkeley, and Alameda, Cal.}{15}{16}
\redliningset{Minneapolis, Minn.}{17}{18}

\defwebsite{cooper-racial-dot}{
title=Racial Dot Maps,
journal=Weldon Cooper Center for Public Service,
url=http://racialdotmap.demographics.coopercenter.org/,
date=2010,
}

Sources:
\begin{itemize}
\item \sentence{nelson-mapping-inequality}. CC-BY-NC 2.5.
\item \sentence{cooper-racial-dot}. Images Copyright, 2013, Weldon Cooper Center
for Public Service, Rector and Visitors of the University of Virginia (Dustin A.
Cable, creator). Map data by OpenStreetMap, under CC-BY-SA.
\end{itemize}

If the era of \textit{de jure} redlining ostensibly ended no later than the
passage of the federal Fair Housing Act of 1968, why are residential
communities largely segregated even now, half a century later? Scholars have
tended to coalesce around three overlapping explanations. \textit{See generally}
Camille Zubrinsky Charles, \textit{The Dynamics of Racial Residential
Segregation}, 2 \textsc{Ann. Rev. Soc.} 167 (2003).

\begin{questions}
\item \textbf{Racial Wealth Gaps and ``Lock-In.''} One explanation may arise
from the observation that decades of \textit{de jure} discrimination and
segregation rendered Black home buyers \textit{as a group} less wealthy than
White home buyers \textit{as a group}, and resulted in systematic disparities
in the value of housing stock in neighborhoods where mainly nonwhite people
lived relative to housing stock in predominantly White neighborhoods. Since
housing is a durable but depreciating asset, over time the unequal allocation
of credit and capital investment generated a disparity of home values that
correlated with neighborhood racial demographics: predominantly white
neighborhoods came to be comprised of more valuable homes and wealthier
homeowners than racially mixed or majority-minority neighborhoods. \textit{See
generally} \textsc{Douglas Massey \& Nancy A. Denton, American Apartheid:
Segregation and the Making of the Underclass} (1993).

Even when \textit{de jure} discrimination ostensibly ended, these consequences
of its history remained, and they would be difficult to change even if one were
to assume the absence of overt racial bias (by no means a safe assumption).
Because (1)~wealthier people can afford and tend to prefer more expensive
homes, (2)~wealth correlates with race, and (3)~the value of homes correlates
with the racial demographics of the neighborhood in which those homes are
located, segregation may be, in a sense, ``locked in'': Black homebuyers cannot
afford to buy into high-value predominantly White neighborhoods, and White
homebuyers can afford \textit{not} to live in lower-value racially diverse or
majority-minority neighborhoods, so segregation persists. \textit{See}
\textsc{Daria Roithmayr, Reproducing Racism} 93-120 (2014).

This explanation, while theoretically sound and grounded in historical
experience, does not seem to be the whole story. While racial wealth and income
gaps are real, and can be traced in part to policy decisions of the redlining
era, empirical analysis suggests that these gaps are only a modest (though
real) contributor to the persistence of residential racial segregation. In
particular, wealth gaps alone do not explain observed levels of exclusion of
Black homeowners from predominantly White neighborhoods. \textit{See generally}
Kyle Crowder et al., \textit{Wealth, Race, and Inter-Neighborhood Migration},
71 \textsc{Am. Soc. Rev.} 72 (2006). Other, overlapping factors must also be
playing a role. 


\item \textbf{Segregation as an Emergent Phenomenon.} In the 1970s, the
prominent game theorist Thomas C. Schelling (whose insights about cooperation
and conflict had helped defuse the Cuban Missile Crisis and would later lead to
his being awarded the Nobel Prize in economics) developed a still-influential
mathematical model of segregation. \textit{See} Thomas C. Schelling,
\textit{Dynamic Models of Segregation}, 1 \textsc{J. Mathematical Soc.} 143
(1971). This model demonstrated how large systematic discriminatory effects can
emerge from the interaction of uncoordinated, individual private choices.
Moreover, such systematic effects could arise and persist even where individual
decisions are motivated by preferences far milder than what we might consider
overt racial animus or hatred. The example Schelling gave was a preference not
to live in a neighborhood where one is part of a small racial minority:
\begin{quote}
Whites and blacks may not mind each other's presence, may even prefer
integration, but may nevertheless wish to avoid minority status. Except for a
mixture at exactly 50:50, no mixture will then be self-sustaining because there
is none without a minority, and if the minority evacuates, complete segregation
occurs. If both blacks and whites can tolerate minority status but there is a
limit to how small a minority the members of either color are willing to
be---for example, a 25\% minority---initial mixtures ranging from 25\% to 75\%
will survive but initial mixtures more extreme than that will lose their
minority members and become all of one color. And if those who leave move to
where they constitute a majority, they will increase the majority there and may
cause the other color to evacuate. Evidently if there are lower limits to the
minority status that either color can tolerate, and if complete segregation
obtains initially, no individual will move to an area dominated by the other
color. Complete segregation is then a stable equilibrium.
\end{quote}

\defwebsite{hart-case-parable-polygons}{
author=Vi Hart,
author=Nicky Case,
title=Parable of the Polygons,
url=https://ncase.me/polygons/,
date=last updated nov 7 2022,
}


Simulations based on Schelling's model demonstrate how segregation can indeed
emerge from such uncoordinated choices and entrench itself once established. In
particular, they demonstrate that if an environment is already segregated,
simply \textit{removing} bias is not enough to desegregate it. You can try such
a simulation for yourself at: \sentence{hart-case-parable-polygons}. (This
simulation also suggests that cultivating widespread homeowner preferences
\textit{against} racial homogeneity may be one tool to promote desegregation). 


\item \textbf{Private Discrimination.} Just as widespread invidious racial
discrimination in the allocation of housing and mortgage credit existed long
before HOLC was created, private discrimination did not magically disappear
when the Fair Housing Act was enacted. When the activists of Austin coined the
term ``redlining,'' they were referring to lines drawn on maps at their local,
privately owned savings and loan offices, not maps drawn by the federal
government (though there was a historical and causal relationship between the
two).

Sadly, even though the federal Fair Housing Act, analogous state and local laws,
and associated regulations have long forbidden various discriminatory behaviors
with respect to housing, such behaviors persist even today. For example, an
ambitious three-year investigation conducted by the newspaper \textit{Long
Island Newsday} from 2016 to 2019 demonstrated that in the New York City
suburbs of Long Island, real estate agents regularly discriminated against
nonwhite house hunters compared to White house hunters. Notably, about a
quarter of the agents investigated steered White house hunters toward more
homogeneously White neighborhoods, while steering comparable nonwhite house
hunters toward more mixed or majority-minority neighborhoods. \textit{See }Ann
Choi, Bill Dedman, Keith Herbert, \& Olivia Winslow, \textit{Long Island
Divided}, \textsc{Newsday} (Arthur Browne ed. Nov. 17, 2019), available at 
\url{https://perma.cc/KG54-DLF3}. Such steering is unlawful under the Fair
Housing Act and related regulations. \textit{See, e.g.}, 24 C.F.R. {\S}
100.70(c). Nevertheless, it clearly persists and---along with other forms of
private discrimination---likely contributes to the continued \textit{de facto}
racial segregation of residential neighborhoods.
\end{questions}
