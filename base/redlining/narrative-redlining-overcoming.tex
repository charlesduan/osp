If we believe that racially segregated residential neighborhoods are a problem,
what is the solution? Consider the following proposals. Do you think they are
likely to be effective? Might they be counterproductive?

\begin{questions}[]
\item \textbf{Racial Wealth Gaps and Reparations.} If today's racial wealth and
income gaps are indeed the product of \textit{past} discrimination, can we
count on these gaps to close over time as people in formerly redlined
neighborhoods take advantage of economic opportunities now available to them?
Is the problem simply that not enough time has passed since the redlining era?
Is ``lock-in'' a phenomenon that might fade away if only we are patient enough?

If private housing discrimination were absent, and intergenerational social
mobility (defined as upward changes in income and wealth from one generation to
the next) were high, we might expect the passage of time to ameliorate racial
wealth gaps and thereby reduce residential racial segregation. But as we have
seen, private housing discrimination persists even half a century after being
outlawed. And the latest research demonstrates that social mobility in the
United States is not only low, it is \textit{negatively correlated with
segregation}{}---people who grow up in segregated neighborhoods are less likely
to see generational increases in income or wealth. \textit{See} Raj Chetty et
al., \textit{Where is the Land of Opportunity? The Geography of
Intergenerational Mobility in the United States}, 129 \textsc{Q.J. Econ.}
1553, 1607-09 (2014) (``[B]oth blacks and whites living in areas with large
African American populations have lower rates of upward income mobility\dots.
[and m]ore racially segregated areas have less upward mobility.'') If this
research is right, and lock-in will not simply fade away, what more should be
done?

\having{intro-allocation}{As we discussed in our unit on Allocation,
historical}{As we
will discuss in our unit on Allocation, historical}{Historical} injustices whose
effects
are felt in the present can be difficult to resolve through litigation. For
this reason, many have proposed broader programs of redistribution to address
such injustices, and the legacy of redlining is among the targets of such
programs.
\having{mabos-case-qs}{Indeed, the Ta-Nehisi Coates article cited after our
discussion of \textit{Johnson v. M'Intosh} and \textit{Mabo's Case}}{An article
by Ta-Nehisi Coates (discussed later in this book)}{An article by Ta-Nehisi
Coates} cites redlining as its primary example of systematic racial injustice in
the United States that deprived Blacks of opportunities afforded to Whites to
build intergenerational wealth, calling out for reparations as a remedy.
Ta-Nehisi Coates, \textit{The Case for Reparations}, \textsc{The Atlantic} (June
2014), \textit{available at}
\url{http://www.theatlantic.com/features/archive/2014/05/the-case-for-reparations/361631/}.


\item \textbf{Tackling Demand-Side Discrimination.} If Schelling's model of
segregation is correct, real estate professionals like those investigated by
\textit{Newsday} might believe they have reasons to perpetuate continued
residential segregation even if those professionals privately deplore (or
believe they deplore) segregation and racial bias. For example, real estate
agents might believe they are faithfully catering to the odious preferences of
their house hunting clients when they preferentially direct White clients to
homogeneously White neighborhoods, or direct nonwhite clients to racially mixed
or majority-minority neighborhoods. That is, they may think that homebuyers
generally prefer to live among people who look like them. 

This belief may not be wrong! Consistent with the Schelling model, surveys find
that White people tend to express preferences to live in an area where Whites
are clearly in the majority, while nonwhites are more likely to express
preferences to live in areas where races are evenly mixed.  \textit{See, e.g.},
W. A. V. Clark, \textit{Residential Preferences and Neighborhood Racial
Segregation: A Test of the Schelling Segregation Model}, 28 \textsc{Demography}
1 (1991). Some house hunters might even explicitly instruct their brokers about
the racial demographics they are seeking in a new neighborhood. But if real
estate agents tailor housing searches to such client preferences---even if the
agents themselves abhor racial bias---they may be contributing to continued
residential racial segregation. Should professional acquiescence to such
discriminatory preferences on the part of house hunters be unlawful?

Professor Lee Ann Fennell thinks so. \textit{See} Lee Anne Fennell,
\textit{Searching for Fair Housing}, 97 \textsc{B.U. L. Rev}. 349 (2017). She
argues that the law ought to treat such home seekers' preferences for
segregated housing the same way it treats the proverbial Mrs. Murphy's desire
to discriminate in her selection of boarders\having{narrative-tenant-selection}{
(you'll remember the ``Mrs. Murphy
exception'' to the Fair Housing Act's prohibitions from our unit on Leasing
Real Property)}{ (we'll see the ``Mrs.~Murphy exception'' to the Fair Housing
Act's prohibitions in our unit on Leasing Real Property)}{ (the Fair Housing
Act's exception permitting certain single-family homeowners to discriminate when
renting out their house)}. That is, she would allow home seekers to ultimately
decide to
purchase a home based on the racial demographics of the neighborhood in which
the home is located, but she would forbid them from publishing those
preferences or communicating them to real estate professionals, and would
prohibit those professionals from soliciting or complying with such client
instructions. \textit{Id.} at 395-403. 

Professor Fennell's proposal is---by her own admission---in tension with current
understandings of fair housing law. In particular, the Court of Appeals for the
Seventh Circuit has opined that real estate brokers who follow their
house-hunting customers' instructions to limit their search based on the racial
composition of neighborhoods would not violate the Fair Housing Act, even if
the result is residential racial segregation, because such brokers would not
subject \textit{those customers} to disparate treatment on the basis of race.
\textit{Village of Bellwood v. Dwivedi}, 895 F.2d 1521, 1530-31 (7th Cir.
1990). In the court's view, the Fair Housing Act is designed to protect home
buyers from discrimination, not to construct a less segregated society. So long
as home buyers are not subjected to disparate treatment, the court said,
brokers ought not be held liable. However, the same opinion affirmed that ``a
person who serves as a conduit for another person's discrimination can, it is
true, be guilty of intentional discrimination, or, what is the same thing, of
disparate treatment.\dots And it is actionable discrimination, regardless of
its effects and notwithstanding the merchant's own freedom from racial
animus.'' \textit{Id.} On this understanding of the law, a \textit{seller} who
instructed a broker not to show their home to potential buyers on the basis of
race could still be held liable, as could a broker who complied with such
instructions, on grounds that the broker would be subjecting \textit{potential
buyers} to disparate treatment on the basis of race.

Is the distinction drawn by the \textit{Village of Bellwood} court consistent
with the justifications for the Mrs. Murphy exception? Is Professor Fennell's
proposal? Do the justifications for the Mrs. Murphy exception apply to a
homebuyer's selection of a neighborhood to live in?


\item \textbf{Segregation vs. Gentrification.} How might residential
neighborhoods change if more homeowners began valuing racial diversity among
their neighbors as a more important factor in deciding where to live? Given the
persistent racial gaps in wealth and income, and the correlation of housing
values with the race of neighborhood residents, how might such an increased
preference for diversity manifest itself in homeowner behaviors? Who will be in
the best position to choose to live in a more integrated neighborhood as demand
for such neighborhoods increases?

\having{narrative-leasing-gentrification}{Recall our discussion of the problem
of gentrification in our unit on Leasing Real Property.}{In our unit on Leasing
Real Property, we will discuss the problem of gentrification, in which wealthier
people move into a neighborhood and displace longtime existing
residents.}{Gentrification, generally defined, is the situation in which
wealthier people move into a neighborhood and displace existing longtime
residents.} The same dynamics that arise when wealthy tenants suddenly begin
leasing residences in a poorer neighborhood can arise when racially diverse or
majority-minority neighborhoods get an influx of new investment in
owner-occupied housing from White home buyers. So long as wealth and race are
correlated, and housing values tend to drive local costs of living,
integration-by-gentrification may perversely have the effect of \textit{driving
out} nonwhite residents of diverse neighborhoods who can no longer afford to
live there, ultimately simply rearranging the boundary lines of racially
segregated neighborhoods.

At least, integration-by-gentrification \textit{could} have that effect. So far,
empirical evidence is somewhat mixed. Perhaps because displacement is likely to
lag gentrification, and because it is difficult to distinguish displacement
(poorer, nonwhite residents being forced out by richer, White newcomers) from
succession (poorer, nonwhite residents leaving at a normal rate and being
replaced by richer, White newcomers), researchers disagree on whether
displacement is happening at all in gentrifying neighborhoods, let alone the
extent to which it is a problem. For a literature review, \textit{see} Miriam
Zuk et al., \textit{Gentrification, Displacement, and the Role of Public
Investment}, 33 \textsc{J. Planning Lit.} 31, 36-39 \& tbl. 2 (2018). 

Setting aside the potential problem of displacement, are segregation and
gentrification our only options? Can you imagine a mechanism for residential
racial integration that \textit{doesn't} involve gentrification? If the problem
is that race, wealth, and neighborhood home values are correlated, is there any
way to create opportunities for poorer homeowners to move into more expensive
neighborhoods?


\item \textbf{Creditworthiness and Community Investment.} Like real estate
agents who believe they are satisfying their clients' preferences, finance
professionals looking out for their institutions' balance sheets might make
decisions that have the effect of perpetuating residential racial segregation,
even if the professionals themselves privately deplore (or believe they
deplore) segregation and racial bias. Because, as discussed above, the value of
residential real estate in racially diverse or majority-minority neighborhoods
tends to be lower than the value of real estate in predominantly White
neighborhoods, lending officers might well reason that loans secured by homes
in racially diverse or majority-minority neighborhoods are likelier to result
in losses for the lender than loans secured by real estate in predominantly
White neighborhoods. A lending officer who sincerely believed this could in
good faith tell himself that it would be irresponsible to extend credit secured
by property in such neighborhoods. But writing off the credit needs of entire
neighborhoods on the basis of such generalizations can be expected to leave
many otherwise creditworthy borrowers without access to mortgage financing,
based solely on the racial demographics of the neighborhood where they
live---exactly the harm caused by \textit{de jure} redlining. 

To counteract this tendency, the Community Reinvestment Act of 1977 (``CRA''),
12 U.S.C. {\S} 2901 \textit{et seq.}, was enacted to require financial
institutions to extend credit in all communities where they accept
deposits---including low- and moderate-income neighborhoods---consistent with
the ``safe and sound'' operation of the institution. 12 U.S.C. {\S} 2903(a)(1).
The CRA and its implementing regulations require federally regulated financial
institutions to report their lending activities to the federal agency that
regulates them. The agencies, in turn, must evaluate whether the reporting
institutions are adequately meeting the credit needs of the communities they
serve, and publish their evaluation accompanied by a rating of the
institutions' performance. In addition to subjecting financial institutions to
public scrutiny, these evaluations are also used by federal regulators to
determine whether the regulated institution should be granted permission to
expand by acquisition, merger, or the opening of new branches and offices. 
\end{questions}

