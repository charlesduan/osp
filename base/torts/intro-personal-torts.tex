One of the early variants of writs for forcible trespasses, trespass \textit{de
bonis asportatis} (literally, ``of taking away goods,'' and often abbreviated
to ``trespass d.b.a.'') was available when the defendant carried away the
plaintiff's property, and its remedy was damages. But beyond this simple core,
the personal property actions were a confused mess that defies easy description
and took many centuries to clean up. The hard part was to determine just what
kinds of facts ought to entitle a plaintiff to recover when he could not allege
a taking from his possession, perhaps because he had voluntarily parted with
possession (e.g. in a bailment) or perhaps because the defendant had not taken
them (e.g. for found property). 

One approach was the older writ of detinue, which was available against a bailee
who ``detained'' the goods from the plaintiff. The courts extended detinue so
that it ran against other parties (at first the executor of the estate of a
deceased bailee, and then anyone) as long as there had been an initial
bailment. But since a defendant could defeat detinue by disproving the
allegations in the writ, detinue was really only safe when the plaintiff could
trace with confidence the chain from his hands to the defendant's. As a result,
detinue ``on a bailment'' was gradually supplanted by detinue ``\textit{sur
trover}'' (literally, ``upon finding''): the plaintiff alleged that he had
lost the property and the defendant had found it but refused to return it. The
defendant could show that he had the property rightfully (e.g. through a sale
tracing back to the plaintiff), but otherwise ``lost and found'' was a
conveniently broad formula that could cover actual cases of missing property,
bailments gone wrong, and even cases of suspected theft. All the plaintiff
needed to show was that the property was his and that the defendant now had it.
Even so, detinue in its trover variation still was frequently unsatisfactory:
\begin{quote}
A \textit{praecipe} action [the general name of a category of writs including
detinue] was barred by performance, even imperfect performance, and so in
detinue damages could not be awarded if the goods were restored. The bailee who
starved a horse to death, or who rode it further than agreed, or who returned
other goods in a damaged state, was not liable in detinue. The plaintiff in
detinue could not count on a bailment or loss of the thing demanded if it was
no longer the same thing as he had bailed or lost, as where it had been made
part of something else or fashioned into something new. And on the same
principle, it was arguable that he could not allege a detaining of something
which no longer existed at all.
\end{quote}
\textsc{J.H. Baker, An Introduction to English Legal History} 394 (4th ed.
2007). 

The solution lay, yet again, in trespass. The road to reform is paved with legal
fictions. The royal courts had no difficulty treating outright theft as
satisfying the requirement of trespass d.b.a that the taking be forcible. But
plaintiffs soon started pleading claims of trespass d.b.a for injuries to
horses against defendants named Smith, and claims for the forcible chopping up
of lumber against defendants described as carpenters. These were garden-variety
contract actions (for defectively shoeing a horse or for botching a
construction job)---or would have been, if the common law had had an effective
form of action for breach of contract. It didn't, and so plaintiffs who could
stretched the facts to fit within trespass d.b.a. The royal courts solved this
particular problem around 1350 by abandoning the need to plead \textit{vi et
armis} in trespass, as long as the plaintiff could set forth in more detail the
special facts entitling him to recover. This was the origin of actions on the
case, mentioned above; it had the effect of kickstarting a burst of creative
experimentation with new variation of trespass.

One approach, reflecting bailments' place on the border between property and
contract, was to plead that the defendant had negligently or deceitfully
violated a promise to keep the goods safe. Another was to plead that a bailee
had intentionally converted goods to his own use---as with a bailee who drinks
a bottle of wine or spends the silver coins in a strongbox. This latter idea
had staying power; by the 16th century, trespass on the case for conversion was
regularly used against bailees. Then history repeated itself: just as detinue
was extended from bailees to third parties by alleging the fictitious finding
called trover, so was conversion. A plaintiff could even plead that he had
``lost'' his ship and that the defendant had ``found'' it in London. The final
stage in conversion's triumph was to treat a wrongful withholding itself---the
old ``detinue''---as a form of conversion to the defendant's own use. And with
that, the modern tort of \textbf{conversion} or \textbf{trover} took shape: the
plaintiff claimed that the property was his and that the defendant had treated
it as his own. The defendant might still have the property, or might not; the
property might still exist, or it might have been destroyed; what mattered was
the defendant's use in a manner inconsistent with the plaintiff's ownership
resulting in the plaintiff's dispossession. As the Restatement puts it,
``Conversion is an intentional exercise of dominion or control over a chattel
which so seriously interferes with the right of another to control it that the
actor may justly be required to pay the other the full value of the chattel.''
\textsc{Restatement (Second) of Torts} \S~222A (1965).

\expected{jacque-v-steenberg}

What if the defendant merely damages the plaintiff's property, or interferes
with its use, but stops short of converting it---as by breaking the headlights
on the plaintiff's car, or taking it for a forty-eight-hour joyride? Conversion
traditionally did not quite work here; instead the plaintiff's remedy lay in
\textbf{trespass to chattels}, which evolved from the original action for
trespass d.b.a. Its use in a case of forcible misuse (like smashing headlights
or temporary taking) was straightforward enough. Over time, courts extended its
use to other cases involving indirect or non-forcible harms. But unlike with
trespass to land---which as \textit{Jacque} shows is actionable even without
harm to the property---the Restatement says that trespass to chattels requires
that the defendant deprive the plaintiff of possession, impair the value of the
property, or deprive the plaintiff of its use. \textsc{Restatement (Second) of
Torts} {\S} 218. \textit{See also} \textit{Intel v. Hamidi}, 71 P.3d 296 (Cal.
2003) (no trespass to chattels for sending emails addressed to Intel employees
to Intel computers over Intel's objections).

A final member of the property torts family is \textbf{replevin}. Initially, it
was a purely feudal form of action. If a tenant failed to perform the feudal
services due to his lord, the lord could ``distrain'' the tenant's personal
property by taking possession of it. The tenant's remedy for a wrongful
distraint was replevin: by posting a bond of twice the value of the property,
the tenant was entitled to possession immediately while the suit over the
underlying dispute proceeded. As the feudal character dropped out of the
landlord-tenant relationship, replevin became a general-purpose action to
recover possession of property wrongfully withheld. Its immediate-recovery
remedy made it attractive to plaintiffs who just wanted their stuff back,
particularly in the United States. (``Mattie Ross: The saddle is not
for sale. I will keep it. Lawyer Dagget will prove ownership of the gray horse.
He will come after you with a writ of replevin.'' \textsc{True Grit} (Paramount
Pictures 2010)). Today in some states it remains at least the name of the
action to recover possession, although it has often been superseded by
procedures to recover possession in state civil procedure codes.

Criminal law also protects personal property ownership and possession. The menu
of common-law personal property crimes bore the same confused stamp of history
as the menu of personal property torts. Larceny required a felonious carrying
away from possession; over time, both the carrying away and the possession
became thin shadows of their former selves, but not quite fictional. Larceny by
trick, at least in theory, plugged the gap for owners who parted with
possession voluntarily under the influence of fraudsters' lies; embezzlement
covered faithless bailees and employees who abused their positions of trust to
steal from the cash register, literally or metaphorically. Robbery was theft
achieved by a threat of violence. Looking back on the fine distinctions courts
contrived to distinguish these various crimes (e.g., in \textit{Bazely's Case},
(1799) 168 Eng. Rep. 517 (Cent. Cr. Ct.), the court held it was embezzlement
for a teller to put money in a bank drawer and then put it in his pocket, but
not embezzlement for the teller to put the money in his pocket directly), it is
hard not to concur with historian S.F.C. Milsom's assessment: ``The miserable
history of crime in England can shortly be told. Nothing worth-while was
created.''  \textsc{S.F.C. Milsom, Historical Foundations of the
Common Law 353} (1969). Many states, influenced by the Model Penal Code, have
tried to reform their theft statutes to create a single, integrated law of
theft. \textit{See generally} \textsc{Stuart P. Green, 13 Ways to Steal a
Bicycle: Theft Law in the Information Age} 4 (2012) (arguing that ``theft law
reformers threw out the baby with the bathwater''). But hard problems remain,
such as defining the kinds of property that can be ``stolen'' at all---e.g.,
is it theft to sneak into a movie without paying or to download that movie on
BitTorrent, or is ``theft'' simply the wrong word to describe conduct that
deprives no one else of their possession and enjoyment?

