The name of the most familiar tort protecting real property, \textbf{trespass},
was originally the name of an entire family of actions that first emerged in
the 12th and 13th centuries. A plaintiff would commence his case by going to
the royal Chancery and purchasing a writ commanding the defendant to come
before the courts and explain why he had done such-and-such a thing against the
plaintiff's rights. The Latin phrases used by the Chancery clerks who filled
out the writs---and which the royal courts insisted on when hearing a
case---came to define individual forms of action.

One of the earliest such formulaic phrases, and one with one of the longest
careers in the common law, was trespass \textit{quare clausum fregit}
(literally, ``why he broke the close,'' and often abbreviated to ``trespass
q.c.f.''). The gist of the action was that the defendant, wrongfully, with
force and arms (in Latin, \textit{vi et armis}) and against the King's peace,
had broken into the plaintiff's enclosed lands and caused injury. As in a
trespass action for intentional battery, a plaintiff bringing an action for
trespass q.c.f.\ could obtain money damages to the extent of his injuries.
Trespass q.c.f.\ was the natural cause of action for damaging the plaintiff's
crops or destroying his buildings.

Another early formula, trespass \textit{de ejectione firmae} (literally, ``of
ejection from his term,'' and often simply ``\textbf{ejectment}''), protected
a lessee against being wrongfully evicted from his lands by an intruder. To the
extent that the medieval legal mind made such a distinction, ejectment
protected not against injury as such but against disposession; by the sixteenth
century, the common-law courts would put a victorious plaintiff back in
posession. This development made ejectment a potentially attractive way to
litigate competing claims to land---in modern terms, to ``try title.'' Among
other things, ejectment (like the other trespass writs) led to a trial before a
jury; a defendant sued under an older ``writ of right'' could elect trial by
battle. There was only one problem: ejectment was only available to lessees.
The result was one of the great legal fictions of the common law: the
fictitious lessee.

\begin{quote}
When two parties wished to try the title to a piece of land, one of them leased
it to an imaginary person (John Doe) and the other similarly leased to another
(William Styles). One lessee ejects the other (this will be all fiction), and
in order to try the rights of the lessees the court has to enter into the
question of the rights of the lessors.
\end{quote}
\textsc{Theodore F.T. Plucknett, A Concise History of the Common Law} 374 (5th
ed. 1956). This fictional use of ejectment crossed the Atlantic and survived in
the captions of famous cases like \textit{Johnson \& Graham's Lessee v.
M'Intosh}, 21 U.S. 543 (1823) and \textit{Martin v. Hunter's Lessee}, 14 U.S.
304 (1816). There were no actual lessees in these cases; they were simply
fictitious parties required by the formula of ejectment.

Today, the distinctions between trespass (q.c.f.) and ejectment are far less
significant but not gone entirely. Courts can generally reach any legal issues
necessary to resolve a case, regardless of the plaintiff's initial choice of
cause of action, and they have far more freedom to select appropriate legal and
equitable remedies, such as money damages for injuries to land or lost income
from being out of possession, injunctions to order a defendant to cease
trespassing or execute a conveyance to the plaintiff, or declaratory judgments
about the state of title.

One remaining hole in the common-law system was that both trespass and ejectment
required some interference with possession, but there are many cases of
disputed title in which the parties are civilized enough not to be constantly
elbowing each other off the land. The action to \textbf{quiet title} provides a
remedy here; it is brought by a plaintiff objecting that another's claims
amount to a ``cloud'' on her title. Other claimants must either defend and
prove their competing title or be estopped from asserting them. Quiet title,
for example, is typically the appropriate cause of action to establish that one
has acquired title to land through adverse possession, or that an easement has
been abandoned through non-use, or that a deed sitting in the land records is
void as a forgery. Although frequently quiet title actions are brought
\textit{in personam} against specific claimants, state statutes can authorize
\textit{in rem} quiet title actions that extinguish the rights of all parties,
known and unknown, unless they appear to defend their claims. \textit{See}
\textit{Arndt v. Griggs}, 134 U.S. 316, 327 (1890) (``[A] State has power by
statute to provide for the adjudication of titles to real estate within its
limits as against non-residents who are brought into court only by publication
\ldots.''). Particularly in view of the long-standing ``situs rule'' giving
state courts exclusive jurisdiction over land located within their states, the
\textit{in rem} quiet title action probably survives the Supreme Court's
20th-century Due Process revolution.

Originally, the assize of \textbf{nuisance} protected plaintiffs' rights to use
land they did not themselves own (such as a right to pasture cows on another's
land, much like a modern easement) or to be free from some specific harms
caused by a neighbor (such as straying cows). In the fourteenth century,
plaintiffs began to be able to use writs of trespass to allege a nuisance
without needing to plead that the defendant had acted \textit{vi et armis}, and
this new formula developed into a general action for what we would today
recognize as nuisances: unreasonable interferences with the use and enjoyment
of land. (Nuisance was thus an ``action on the case''; it belonged to the same
branch of non-forcible trespasses as the one from which the modern tort of
negligence developed.) In keeping with its origins in actions ``on the case,''
nuisance has become an extremely versatile cause of action, encompassing a
variety of injuries to interests in real property and a variety of potential
remedies for those injuries.

Trespass is also a crime, but it is a surprisingly mild one. Vermont's basic
trespass offense is typical:
\begin{quote}
A person shall be imprisoned for not more than three months or fined not more
than \$500.00, or both, if, without legal authority or the consent of the
person in lawful possession, he or she enters or remains on any land or in any
place as to which notice against trespass is given by:
\begin{itemize}
\item[(A)] actual communication by the person in lawful possession or his or her
agent or by a law enforcement officer acting on behalf of such person or his or
her agent;

\item[(B)] signs or placards so designed and situated as to give reasonable
notice \ldots.
\end{itemize}
\end{quote}
\textsc{Vt. Stat.} tit. 13, {\S} 3705. Many states' laws contain exceptions
relaxing the notice requirement in specified cases where the lack of permission
ought to be obvious in context. \textit{See, e.g.,} \textsc{Md. Code Ann.,
Crim. Law} {\S} 6-408 (making trespass a crime even without specific notice not
to enter if the trespass is committed ``for the purpose of invading the privacy
of an occupant of a building or enclosure located on the property by looking
into a window, door, or other opening.'').

Given the harshness of civil trespass remedies,
\having{jacque-v-steenberg}{as in \textit{Jacque}, }{as will be seen in
\emph{Jacque v.~Steenberg}, }{}what
explains the leniency of criminal trespass law? In many states, this mild
baseline is supplemented with more severe penalties for certain sorts of
trespasses. New York, for example, treats criminal trespass (ordinarily a
violation) as a class B misdemeanor when it involves entry onto fenced land, a
school or children's overnight camp, a public housing project, or a railroad
yard. \textsc{N.Y. Penal L. {\S} 140.10}. Are these principled attempts to
distinguish among trespasses, or special favors for particular landowners?

