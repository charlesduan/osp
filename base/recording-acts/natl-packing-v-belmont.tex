\reading{National Packing Corp. v. Belmont}
\readingcite{547 N.E.2d 373 (Ohio Ct. App. 1988)}

DOAN, J.

In the instant appeal, we must determine whether to apply the venerable doctrine
of \textit{idem sonans} to the facts and circumstances set forth in the record.
\dots{}

The record reveals that the plaintiff, National Packaging Corporation (``NPC''),
sued Michael Bolan, d.b.a. Trade Packaging, in the Franklin County Court of
Common Pleas. On November 25, 1983, NPC obtained a judgment for \$3,331.76 plus
interest; and, at a later time, it certified the judgment in Hamilton County,
with Bolan's name incorrectly spelled ``B-O-L-E-N'' in the docket book. At the
time the judgment was certified, Michael Bolan owned property in Hamilton
County at 8107 Camargo Road and 815 Indian Hill Road.

Bolan's ex-wife, Elaine (now Elaine Belmont), brought a foreclosure action
against the property located at 815 Indian Hill Road to collect overdue
child-support payments. The property was sold in a sheriff's sale to L. Michael
and Elaine Belmont (Bolan's ex-wife and her new husband). Because NPC's
judgment was filed under an incorrect spelling of Bolan's name, NPC did not
receive notice of the sheriff's sale and was unable to protect its interest in
the property.

The Belmonts subsequently sold the Indian Hill Road property to Richard E. and
Vera DeCamp. It was only after this second conveyance that NPC brought its own
foreclosure action, asserting the certified judgment from Franklin County
against both the Indian Hill Road property and the Camargo Road property.

The Belmonts moved to dismiss NPC's complaint in a filing that the trial court
treated as a motion for summary judgment. The DeCamps then moved for summary
judgment against NPC and the Belmonts. NPC responded with its own motion for
summary judgment against the Belmonts and the DeCamps. On April 30, 1987, the
trial court overruled NPC's motion for summary judgment, entered summary
judgment for the DeCamps and the Belmonts against NPC and held that the
DeCamps' motion for summary judgment against the Belmonts was moot. The court
dismissed all other claims existing among these three parties, adding a Civ. R.
54(B) certification in a \textit{nunc pro }[\textit{tunc}] entry dated July 6,
1987, and this appeal followed.

In its single assignment of error, NPC asserts that the trial court erred to its
prejudice by overruling its motion for summary judgment. Relying upon the
doctrine of \textit{idem sonans},\readingfootnote{1}{The doctrine of idem sonans
is
defined in 70 \textsc{Ohio Jurisprudence} 3d (1986) 21-22, \textit{Names},
Section 18, as follows:\par {}``The arbitrary orthography and pronunciation
given to proper names, and the variant spelling resulting from ignorance have
led the courts to formulate the doctrine of `idem sonans,'
which means `sounding the same.' Under this doctrine a mistake
in spelling the name of a party is immaterial if both modes of spelling have
the same sound. The grounds for applying the doctrine to slight variations in
spelling is that of de minimis non curat lex -- the principle that the law is
not concerned with trifles. The general rule in Ohio seems to be that a change
in the spelling of a word which does not alter its meaning, or in the spelling
of a name where the idem sonans is preserved, is not a material variance. Thus,
it is not every mistake in names which will invalidate an instrument or
proceeding. To have this effect, the mistake must be such that a person cannot
be identified, or that the error describes another. Since words are intended to
be spoken, bad spelling will not vitiate their intended effect where the sound
is substantially preserved.''} it argues that the certified judgment filed
under a similar sounding but incorrect spelling of the debtor Bolan's name,
retaining the same initial letters as the correctly spelled name, should have
been held to give rise to a valid lien for the benefit of NPC and to provide
the appropriate constructive notice to title searchers.

The doctrine of \textit{idem sonans} was adopted by the Ohio Supreme Court in
\textit{Lessee of Pillsbury} v. \textit{Dugan's Administrator} (1839), 9 Ohio
117. There the court held that Mrs. Pillsbury was on sufficient notice that her
one-eighth interest in certain real estate was being adjudicated, even though
the petition for partition listed her as ``Pillsby.'' Its reasoning was
expressed in these terms:

\begin{quote}
{}``In adjudicating upon transactions occurring in the early settlement of our
state, we must never forget the absence of precedents and system, the different
usages introduced by people emigrating from every part of the country, the want
of knowledge or neglect of technical learning, and the risk of loss of evidence
from the lapse of time. Hence errors of form have always been overlooked, where
the acts of a court are manifest, and its jurisdiction established. \dots{}
\end{quote}

\begin{quote}
 {}``It is not every mistake in names which will invalidate an instrument or
proceeding. \textit{This effect will follow where the person can not be
identified,} or where the error is such as to describe another. But words are
intended to be spoken; and where the sound is substantially preserved, bad
spelling will not vitiate. * * *'' (Emphasis added.) \textit{Id.} at 119-120.
\end{quote}

The petition in \textit{Pillsbury} otherwise identified Mrs. Pillsbury by
reference to her father, who died seized of the subject real estate, and to her
apparent siblings and in-laws. Further, the petition correctly identified the
real estate. Thus, she could be identified in spite of her misspelled name.
\dots{}

The Ohio Supreme Court next mentioned the doctrine of \textit{idem sonans} in
\textit{Buchanan} v. \textit{Roy's Lessee} (1853), 2 Ohio St. 251, a
quiet-title action initiated by Nicholas Longworth. The court, in dicta, mused
that the misspelling of Sarah Roy's name as Sarah Ray, in a notice by
publication, standing alone might be fatal to the action. Without squarely
deciding the question, however, the court went on to note that Sarah Roy could
otherwise be identified in the published notice. Thus the ``otherwise
identified'' standard was carried forward from \textit{Pillsbury} to
\textit{Buchanan.}

In 1869, the Ohio Supreme Court again dealt with \textit{idem sonans,} this time
in a criminal matter. In \textit{Turpin} v. \textit{State} (1869), 19 Ohio St.
540, a case involving an allegedly forged signature, there was a variance
between the spelling of the name as it appeared on the state's exhibit
(``R-e-n-n-i-c-k'' or ``R-u-n-i-c-k''), and the spelling of the name as it
appeared in an indictment (``R-e-n-i-c-k''). The court, however, found no fatal
flaw in the variance \dots{} .

NPC cites \textit{Rauch} v. \textit{Immel} (1936), 55 Ohio App. 71, 23 Ohio Law
Abs. 629, 8 O.O. 354, 8 N.E.2d 569, and \textit{Horton} v. \textit{Matheny}
(1943), 72 Ohio App. 187, 27 O.O. 69, 51 N.E.2d 41, as the basis for validation
of its claimed error. We note, however, that in \textit{Rauch} the doctrine was
applied to a misnomer in a notice of a lawsuit, and in \textit{Horton} it was
applied to a misspelling in a deed description. In these two cases, as in
\textit{Pillsbury, Buchanan,} and \textit{Turpin,} the error did not involve
misspellings in a name index.

The case of \textit{Gleich} v. \textit{Earnest} (1930), 36 Ohio App. 326, 173
N.E. 212, is NPC's best authority because it is factually analogous to the
instant matter. In \textit{Gleich,} the court held that \textit{idem sonans}
validated the assertion of a mechanic's lien against the purchaser of the
property at a foreclosure sale, even though the foreclosure suit did not name
the lienholder. The lienholder had filed its lien against ``C. C. Ernest,''
when in fact the property was held in the name of ``Chester C. Earnest.'' The
court's decision upholding the lienholder's position rested upon the testimony
of two abstractors who testified that they would have searched the records
under both ``Ernest'' and ``Earnest.''

In the matter \textit{sub judice,} three experts have given affidavits stating
that the doctrine of \textit{idem sonans} should not be applied today as a
standard for determining the marketable title of real estate on the basis of
irregularities in last names or surnames, and that by custom it is not applied
by abstractors in southwestern Ohio.

We hold that the doctrine of \textit{idem sonans} is inapplicable to names that
are misspelled in judgment-lien name indexes. We are not a frontier society of
pioneers with little education or an absence of precedent and system. Since the
Supreme Court issued its opinion in \textit{Pillsbury} in 1839, we have
experienced a tremendous growth in the population and the economy, and those
developments have spawned countless real estate sales and a volume of
litigation resulting in an abundance of indexed judgment liens. In modern
society we cannot overlook matters of form by continuing to indulge the
outmoded premises of our societal infancy. To impose rigidly the doctrine of
\textit{idem sonans} to name indexes now maintained for judgment liens would
tax all land abstractors beyond reasonable limits and require them to be poets,
phonetic linguists, or multilingual specialists. The additional time necessary
to examine name indexes under such a stringent doctrine would make the
examinations financially prohibitive.

The appellees, in their brief, demonstrate the difficulty in applying the
doctrine of \textit{idem sonans} to the range of spellings implicated in the
instant case: Bolan, Bolen, Bolin, Bowlin, Bowlan, Bowlen, Bolun; the addition
of double ``l,'' ``ein,'' and ``ien'' spellings does not even exhaust all
conceivable spelling possibilities. The impossibility of the task created by
the doctrine of \textit{idem sonans} is further illustrated by the fact that
we, as a society and state, are no longer a small homogeneous population
primarily of European abstraction. Since our infancy, we have added Asian,
African, South American, Oriental and Arabic surnames. The spelling, sound, and
pronunciation of our population's surnames create an insurmountable burden for
an abstractor to face in appreciating all the possible variations. Under all
the circumstances, a strict application of the doctrine today would leave a
real estate purchaser with a lingering fear that misspelled lienholders, either
negligently or deliberately, might be lurking under the \textit{idem sonans}
doctrine in the judgment-debtor indexes.

We further conclude that the misspelling of Bolan as ``B-o-l-e-n,'' does not
rise to being ``otherwise identifiable.'' Unlike many states that statutorily
require land descriptions with lien filings, Ohio's indexes merely require a
name.

Finally, with the exception of \textit{Gleich,} the courts have not strictly
applied \textit{idem sonans.} We find instead a conditional application, which
includes as a factor whether the individual is otherwise identified, and in
only one case has the doctrine been applied to listings in a judgment-lien name
index. We cannot, in sum, find any authority mandating strict application of
the \textit{idem sonans} doctrine. \dots{}

