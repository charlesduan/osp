\expected{hartig-v-stratman}
\expected{natl-packing-v-belmont}

\item Would it have been easier or harder for Hartig to have discovered the
Connell-Stratman easement agreement than it would have been for the DeCamps to
have discovered NPC's judgment in \textit{Belmont}? 

\item \textit{Morse v. Curtis}, 2 N.E. 929 (Mass. 1885), explained,
\begin{quote}
If [a title examiner] can start with an owner who is known to have a good title
\dots{} he is obliged to run through the index of grantors until he finds a
conveyance by the owner of the land in question. After such conveyance the
former owner becomes a stranger to the title, and the examiner must follow down
the name of the new owner to see if he has conveyed the land, and so on. It
would be a hardship to require an examiner to follow in the index of grantors
the name of every person who at any time, through, perhaps, a long chain of
title, was the owner of the estate.
\end{quote}
Does this policy persuasively justify \textit{Hartig}?



\item Suppose the Stratmans reason as follows: Hartig bought without notice, so
we lost to him, but we can fix the notice problem going forward. We'll put up a
big sign on our front yard at 2208 E. Walnut St.\ reading, ``We own a driveway
easement across 2210 E. Walnut St.'' True or false: if the Stratmans do this,
anyone who buys the house from Hartig takes with notice of the easement and is
therefore bound by it. Would it make a difference if Hartig v. Stratman had
never been litigated and reduced to judgment?


\item In Hartig, the Connell-Stratford deed was outside Hartig's chain of title
because it was recorded too late: after Connell was no longer the record owner.
Deeds can also be recorded too early: before the grantor is the record owner.
How could that happen? Suppose that the Bluth Corporation is building a
subdivision on land in Sudden Valley that it is purchasing from the federal
government. Although the contract of sale with the government has been signed,
the delivery of the actual deed to Bluth has been delayed for complex
bureaucratic reasons. You represent the Devon Bank, which is financing the
development with a loan to Bluth, which will be secured with a mortgage on the
land. The bank's manager proposes that the bank go through with the loan and
protect its interest by recording the mortgage immediately and ensuring that
Bluth records the deed from the government as soon as it issues. Is this a good
idea?


\item Who is a ``purchaser \dots{} for a valuable consideration'' entitled to
the protection of a recording act? In Hood v. Webster, 2 N.E.2d 43 (N.Y. 1936),
Florence Hood executed a deed to a farm to her brother-in-law William Hood in
1913; the deed was escrowed and was to be delivered to William at Florence's
death. In 1928, Florence executed and delivered another deed to the same farm,
this time to her nephew, Howard Webster. Howard's deed recited that it was
given for ``One Dollar and other good and valuable consideration.'' Florence
died in 1933. Held: the ``recital was not enough to put [Howard] into the
position of purchaser[] for a valuable consideration.'' Is this right? What if
the deed had recited that it was given for ``Ten Billion Dollars and other good
and valuable consideration?'' What if Howard had testified that the farm was
worth \$20,000 and he had given Florence \$1,000 in cash? The dissent argued
that the record showed that William had reneged on a promise to pay Florence
\$200 a month for life, and that Howard had ``come to live with [Florence] and
help her on the farm'' when she executed the deed to him. Do these facts affect
your view of the case?

