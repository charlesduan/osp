\expected{natl-packing-v-belmont}

\item True or false: the result in \textit{Belmont} follows from the principle
of \textit{\useterm{nemo dat}}? True or false: the result in \textit{Belmont}
follows
from the principle that a good faith purchaser for value takes free of
conflicting transfers of which she had no notice? What work, if any, are these
principles doing in the case?


\item What result if a deed as delivered to the clerk reads ``Bolan'' but the
clerk enters it in the indexes under ``Bolen?'' Under ``Nolan?'' What if the
clerk neglects to index it at all? What if the clerk drops the deed behind the
radiator and forgets about it?


\item Elaine Bolan changed her name to Elaine Belmont when she remarried. If
during her first marriage, she co-signed a mortgage with her now ex-husband
Michael Bolan, will a title searcher find it in a search under ``Belmont?''
Whose problem is that? What if her new name is Elaine Bolan-Belmont?


\item Problems also arise in determining what a recorded deed actually gives
notice of. In \textit{Luthi v. Evans}, 576 P.2d 1064 (Kan. 1978), Grace Owens
executed a deed in 1971 granting International Tours {}``all interest of
whatsoever nature in all working interests and overriding royalty interest in
all Oil and Gas Leases in Coffey County, Kansas, owned by [Owens].'' The deed,
which was properly recorded, listed seven leases in Coffey County, but Owens
also owned an eighth, the Kufahl lease. Who owns the Kufahl lease? Four years
later, Owens gave a deed to the Kufahl lease to J.R. Burris, who had no actual
knowledge of the deed to International Tours. Who owns the Kufahl lease now?
Would the result be the same if the deed to International Tours had referred
instead to ``all Oil and Gas Leases described in the attached Schedule A,'' and
Schedule A had listed all eight leases, but the deed had been recorded without
Schedule A?


\item Land records are increasingly maintained using computer databases rather
than paper records. Does this cut for or against applying the doctrine of
\textit{idem sonans}? Are there ways to improve the quality of the recording
system by using computers, so that mistakes like this (and others) are less
likely?


\item In 2008, the \textit{New York Daily News} ``stole'' the Empire State
Building by recoding a phony deed to it in New York City's recording system.
The deed purported to transfer ownership of the building to ``Nelotis
Properties L.L.C.''; it bore a fake notary stamp and purported to be
``witnessed'' by Fay Wray, the actress who starred in \textit{King Kong}.
Clerks made no attempt to verify any of the information on the deed, which was
duly recorded. Should they have?


\item Tax protesters -- people who challenge the authority of government to
collect taxes -- sometimes record multi-million-dollar ``liens'' against the
houses of local tax officials and government lawyers. Unlike the lien in
\textit{Belmont}, which resulted from a judgment of an Ohio state court, these
``liens'' arise from the actions of ``common law courts'' established by the
protesters themselves. What effect does recording one of these ``liens'' have?
What can the victims do about it? How might a state government deal with the
general problem of fraudulent recording?

