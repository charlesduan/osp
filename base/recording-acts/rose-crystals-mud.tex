
\readingnote{Excerpts reprinted by permission.}
\reading[Rose, \emph{Crystals and Mud in Property Law}]{Carol M. Rose,
\textit{Crystals And Mud In Property Law}}
\readingcite{40 \textsc{Stan. L. Rev.} 577 (1988)}

\ldots In establishing recording systems, legislatures have lent
support to private parties' efforts to sharpen the definition of their
entitlements. The \textit{raison d'etre} of such systems is to clarify and
perfectly specify landed property rights for the sake of easy and smooth
transfers of land.

But the Anglo-American recording system in fact has been a saga of frustrated
efforts to make clear who has what in land transfers. Common law transfers of
land required a certain set of formalities between the parties, but thereafter,
conflicting claims were settled by the age-old principle, `first in time, first
in right.' Thus, on Tuesday I might sell my farm to you, and on Wednesday I
might wrongfully purport to sell it once again to innocent Farmer Brown. Poor
Farmer Brown remains landless even though he knew nothing about the prior sale
to you and indeed had no way of knowing about it. This outcome was hardly
satisfactory from a property rights perspective. `First in time, first in
right' may work well enough in a community where everyone knows all about
everyone else's transactions, but outside that context, the doctrine does
little to put people on notice of who owns what, and the opportunities for
conflicting claims are endless.

But the efforts to remedy this flaw have gone through new cycles of certainty
and uncertainty. Henry VIII attempted -- without great success -- to establish
public registration of land claims through the Statute of Enrollments in 1536.
Versions of the Statute resurfaced in Massachusetts' 1640 recording act and in
other seventeenth and eighteenth century colonial recording acts, all of which
were much more widely (though still somewhat irregularly) applied than their
Henrician model had been.

~Henry's Statute and its original American counterparts reflected an
emphatically crystalline view of property. Their literal language suggests that
they were versions of what has come to be called a `race' statute) the first
purchaser to record (the winner of the `race' to the registry) can hold his
title against all other claimants, whether or not he was in fact the first to
purchase. In such a system, the official records become an unimpeachable source
of information about the status of land ownership; the law counts the record
owner, and only the record owner, as the true owner. The purchaser can buy in
reliance on the records without fear of divestment by some unknown interloper,
and without the need to make some cumbersome extra-record search for such
potential interlopers.

This system was too crystalline to last. The characters to muck up this
crystalline system [are] ninnies, hard-luck cases, and the occasional
scoundrels who take advantage of them. What are we to do, for example, with the
silly fellow who buys an interest in property but simply forgets to record? Or
with the more conscientious one who does attempt to record his interest, but
whose records wind up in the wrong book? Or with the lost soul whose impeccably
correct filing is dropped behind the radiator by the neglectful clerk? Some
courts take a hard line, perhaps concluding that the first owner was in a
better position than our innocent outsider -- that is, the next purchaser -- to
detect and correct the flaws in the records. But our sympathies for the
luckless unrecorded owner put pressure on the recording system that would
divest him in favor of the later-arriving outsider.  {}-- -- 

Our sympathies are all the greater when the outsider is not so innocent after
all. What shall we do, say, when the unrecorded first buyer is snookered out of
his claim by a later purchaser who knows perfectly well that the land had
already been sold? Shall we allow this nasty second buyer to perfect a claim
simply because he carefully follows the official recording rules? This thought
was too much for the courts of equity, and too much for American legislatures
as well. By the early nineteenth century, the British equity courts had
imported an element of non-record `notice' into what had initially been a
`race' system. Under these doctrines, the later purchaser could take free of
the prior claims only if he did not \textit{know} about those prior claims,
either from the records or from non-record facts that should put him `on
notice.' American legislatures followed this move to such a degree that, at
present, only a handful of states maintain a race system with any rigor. The
other states deny the subsequent claim of the person who had or should have had
notice of the earlier claim. 

This development means mud: What `should' a purchaser know about, anyway? To be
sure, if someone is living on the land, perhaps the potential purchaser should
make a few inquiries about the occupant's status. But what if the `occupant's'
acts are more ambiguous, consisting of, say, shovelling some manure onto the
contested land? Well, said one court, a buyer should have asked about the
source of all that manure -- and since he didn't, and thus did not find out
about the manure shoveller's prior but unrecorded claim, the later buyer did
not count as an innocent; his title was a nullity. 

~With the emergence of this judicial outlook, the crystalline idea of the
recording system has come full cycle back to mud. To be sure, the recording
system can give one a fair guess about the legal status of any given property.
But by the end of the last century, as a Massachusetts court put it, `it would
be seldom that a case could occur where some state of facts might not be
imagined which, if it existed, would defeat a title.' Thus, the test of a
title's `marketability' became a question of whether the title was subject to
`reasonable' doubt -- a matter, of course, for the discretion of the court. In
the meantime, a whole title insurance industry sprang up to calm the fears of
would-be purchasers who wanted to avoid questions about which doubts were
reasonable and which were not. It is this industry, in a sense, that once again
makes crystals out of the recording system's mud; and according to the
reformers, it is this industry that now stands in the way of a more rational
method of cleaning up the mess once and for all. 

Yet one must wonder whether cleaning up the mess might not just repeat the cycle
of mud/crystal/mud. One of the most popular suggestions for reform is the
so-called `Torrens' system, named for someone who thought that shipping
registry methods could be used beneficially in real estate. In this system, all
claims on a given property -- sales, liens, easements, etc. -- are first
registered and then incorporated in a certificate. Torrens registration echoes
eerily the colonial `race' statutes: No unregistered claim counts, and the
owner's certificate for a given property acts as the complete record of
everything that anyone might claim.

Well, perhaps not everything: Government liens, fraudulent transactions, and,
according to some courts, even simple errors or neglect in registration can
produce unregistered claims that count. Hence this neo-race system provides no
complete relief from the recording system's mud. Even after we look at the
Torrens certificate, we still have to be on the lookout for the G-men, the
forgers, and the ninnies who neglected to register their claims properly. Not a
lot of mud, to be sure, but just wait. In some jurisdictions with a long
history of Torrens registration, courts have in effect reestablished a `notice'
system, defeating the interest of one who registers his claim when he knows
about a prior unregistered one -- or merely when he \textit{should} have known
about the prior claim. This practice, of course, means that the registry and
certificate no longer count as the complete source of information about a
property's title status.

The most striking aspect of these developments is that first the title recording
acts, and later the registration systems, represented deliberate choices to
establish crystaline rules for the sake of simplicity and ease of land sales
and purchases. People who failed to use the records or registries were supposed
to lose their claims, no matter how innocent they might have been, and no
matter how nastily their opponents might have behaved. Yet these very
crystalline systems have drifted back into mud through the importation of
equitable ideas of notice -- only to be replaced by new crystalline systems in
the form of private contract or public legislation\dots{}.

~Let us suppose that we have a system for the clarification of property titles.
Might we have a tendency to overuse the system, so that in the end it becomes
so hopelessly bogged down in detail that the purpose of clarity is defeated?
Certainly our traditional land records have this quality. Some early cases
permitted only fee interests to be recorded, but it was the very attractiveness
of the system that created pressure to allow the recordation of other
interests; liens, for example, or easements. Indeed, some claims may be in the
records even though they are not legally recordable. Then too, many claims are
recorded and just stay put over time, and sometimes even conflict with other
recorded claims. The layers of these recorded but unextinguished claims can
grow so thick that it hardly seems worth the time to go back and check them
all. So, in a sense, we treat our clarifying systems -- in this case the
recording mechanisms -- as a kind of `commons.' The resulting system overload,
in turn, creates a certain disgust with the lush proliferation of records. In
fact, one of our current recording reforms would simply extinguish claims that
have not been asserted during a given period. 

Thus, the very attractiveness of making clear one's claims by recording them
defeats the purpose of the system, that is, to clarify all claims against a
given property. One sees the same pattern in the excessively\textbf{ }long
contracts that attempt to specify all possible contingencies and that no one
actually reads; however comforting it might be to `have it in writing,' it
really isn't worth the effort to nail down everything, and the overly precise
contract may wind up being just as opaque as -- and perhaps even more arbitrary
than -- the one that leaves adjustments to the contingencies of future
relations. 

The trouble, then, is that an attractively simple legal device draws in too many
users, or too complex a set of uses. And that, of course, is where the simple
rule becomes a booby trap. It is this booby trap aspect of what seems to be
clear, simple rules -- that scenario of disproportionate loss by some
party-that seems to drive us to muddy up crystal rules with the exceptions and
the post hoc discretionary judgments.

