\reading{Note on Recording Systems and Informal Title}

Up to a billion people worldwide are squatters, living on land they have no
legal right to occupy, usually on the outskirts of cities. They're vulnerable
to eviction or at least extortion from officials who take advantage of their
lack of legal rights. Moreover, the influential economist Hernando De Soto has
argued that record title would help poor people enter the formal economy by
giving them clear ownership of assets. Poor people are often highly
entrepreneurial, but face great barriers entering the formal economy -- and
thus their efforts may lead to fines and jail rather than to wealth. As owners,
they'd have better incentives to invest in improving their land; they could
also pledge their land to lenders in order to raise capital to start legitimate
businesses; and they might even be able to get insurance in case of disaster.
See \textsc{Hernando de Soto, The Mystery of Capital: Why Capitalism Triumphs
in the West and Fails Everywhere Else} (2000). In the U.S., about 70\% of new
business credit comes from using title to other assets, such as homes, as
collateral. The likelihood of being able to recover from these identifiable
assets is the foundation of much willingness to lend. De Soto argues that:
``Property is much more than a body of norms. It is also a huge information
system that processes raw data until it is transformed into facts that can be
tested for truth, and thereby destroys the main catalysts of recessions and
panics -- ambiguity and opacity.'' Hernando de Soto, \textit{Toxic Assets Were
Hidden Assets}, \textsc{Wall St. J.}, Mar. 25, 2009. Legal title and a robust
recording system provides reliable information.

De Soto's views have been highly influential, but also controversial. Record
title might help for specific situations, but the government first has to
decide to give the title to the people who are living on the land, and that
generally requires formally dispossessing someone else -- often someone wealthy
who will fight back. De Soto favors squatters' rights, and his argument depends
in many ways on the same foundation as adverse possession. However, if the
``legal'' owners' title was unjustly acquired -- which is often the case --
then squatters' rights can also be founded on theories of labor or first in
time.

Does formalizing title work? The evidence is mixed. De Soto's ideas were adopted
as part of land title reform in Peru. In a six-year period, more than 1.2
million households, containing 6.3 million people, received title to the
properties they were living on. Using similar areas that weren't subject to
reform as controls, Erica Field determined that labor force participation
increased and child labor decreased. Apparently, many squatter families try to
run a small business out of their house so they can safeguard the homestead at
the same time; with legal title, that's no longer necessary. However, there was
not much evidence of increased access to credit, contrary to de Soto's
predictions. See Erica Field, \textit{Entitled to Work: Urban Property Rights
and Labor Supply in Peru}, 122 \textsc{Q.J. Econ.} 1561 (2007). By contrast, in
Thailand, the size of loans obtained from banks by farmers with formalized
property was more than 50\% larger than loans to farmers without record title.
Results from around the world are ambiguous; formal title sometimes seems to
spur investment, and sometimes it doesn't. See \textsc{Claudia R. Williamson,
The Two Sides of de Soto: Property Rights, Land Titling, and Development,
Annual Proceedings of the Wealth and Well-Being of Nations }95 (2011).

Moreover, it's important not to make the overgeneralized claim that ``strong''
or ``clear'' property rights are necessary for economic development. It should
already be clear to you that Western property rights vary a lot and can be
subject to lots of their own uncertainties, from the acts required to possess a
previously unowned resource to the boundaries of an intellectual property
right. England industrialized while its property law was highly unclear. What's
needed is sufficient certainty to go forward, not perfect certainty -- and that
certainty can come from various sources, including but not limited to formal
law.

The first Chinese law focusing specifically on property rights did not become
effective until 2007. China's unprecedented real estate development during the
prior two decades thus occurred without any published law of real estate;
investors committed hundreds of billions of dollars without assurance of what
they would own even if all went well. See \textsc{Gregory M. Stein, Modern
Chinese Real Estate Law} (2012). In Shenzhen, an economic powerhouse of a city
with over 10 million residents, half of the buildings have no legal titles.
Instead, professionals have developed practices and networks, including
government officials (with a bit of bribery thrown in), that facilitate
transactions even among strangers. \textit{See} Shitong Qiao, \textit{Planting
Houses in Shenzhen: A Real Estate Market without Legal Titles}, 29 \textsc{Can.
J.L. \& Soc. }253 (2013). Qiao quotes an official at the Shenzhen Real Estate
Ownership Registration Center: ``Nobody cares whether they have legal titles or
not. You say they are illegal: dare you void the contracts? \dots{} There is a
huge amount of transactions -- you say farmers cannot sell, it is illegal
[because the Chinese government owns all rural land], but they do it privately
with little ado. Are you to tell them whether it is legal or illegal?'' Another
quote from a senior official: ``[I]t is a war against the people that cannot be
won.'' Facts on the ground matter. On the other hand, China has also repeatedly
suffered from conflicts, sometimes violent, between developers who claim
government sanction and squatters who assert that their actual possession
should be respected.

At a minimum, record title can be useless without an overall well-functioning
legal system. If corruption is a nation's problem, title won't solve that by
itself. In addition, expropriation by the wealthy or well-connected is an
enduring problem. Without careful implementation of a titling system, it's the
already-powerful who end up with legal title -- either initially, or when the
formal system's property taxes start to kick in. A De Soto-type titling program
in Cambodia led to fires and forced evictions of slum dwellers, followed by
transfers of newly valuable inner-city land to wealthy developers. \textit{See}
John Gravois\textit{,
}\href{http://www.slate.com/articles/news_and_politics/hey_wait_a_minute/2005/01/the_de_soto_delusion.html}{\textit{The
de Soto Delusion}}, \textsc{Slate}, Jan. 29, 2005. Older programs to provide
title to members of Indian tribes, supposedly to help them integrate into the
broader U.S. economy, suffered from similar problems. \textit{See} Ezra Rosser,
\textit{Anticipating de Soto: Allotment of Indian Reservations and the Dangers
of Land-Titling}, in \textsc{Hernando de Soto and Property in a Market Economy}
(D. Benjamin Barros ed., 2010). 

If De Soto is right, the U.S. may have turned its back on his insights by
inflicting serious damage on our land recording system and investing too much
in financial instruments for which no centralized records are available. Even
without the mortgage crisis, in the U.S., not everyone has record title,
especially in low-income communities where property often transfers by
intestate inheritance. In Louisiana, for example, an estimated 15\% of the
homeowners who applied for federal housing assistance after Hurricane Katrina
-- approximately 20,000 homeowners -- lacked clear record title. This problem
affected many homeowners concentrated in the low-income neighborhoods of New
Orleans Parish and is one reason they received smaller amounts of help than
many other, better-off neighborhoods. In Texas, about one out of five
low-income households applying for hurricane recovery assistance had at least
one title issue impeding their ability to access assistance. See Heather Way,
\textit{Informal Homeownership in the United States and the Law}, 29
\textsc{St. L. U. Pub. L. Rev.} 113 (2009); see also Jane E. Larson,
\textit{Informality, Illegality, and Inequality}, 20 \textsc{Yale L. \& Pol'y
Rev.} 137 (2002) (discussing large-scale squatter communities in Texas); Zoe
Loftus-Farren, \textit{Tent Cities: An Interim Solution to Homelessness and
Affordable Housing Shortages in the United States}, 99 \textsc{Calif. L. Rev.}
1037 (2011). Should the U.S. consider schemes for titling people or communities
presently without record title?

