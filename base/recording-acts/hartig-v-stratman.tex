\reading{Hartig v. Stratman}
\readingcite{729 N.E.2d 237 (2000)}

\textsc{Sharpnack}, Chief Judge:

Melvin and Louise Stratman are the owners of real property located at 2208 E.
Walnut St. in Evansville, Indiana. The property next door, at 2210 E. Walnut
St., is owned by [Timothy] Hartig. The instant dispute centers around a shared
driveway that is located on both parcels of property, with the majority of the
driveway being on Hartig's property.

The record of title to Hartig's property discloses that Hartig purchased the
property from Sean Holmes on September 28, 1995. Holmes in turn purchased the
property from John Connell on May 31, 1994. On the same day that Connell sold
the property to Holmes, Connell entered into a written easement agreement with
the Stratmans regarding the shared driveway. The agreement gave the Stratmans a
perpetual easement over the portion of the driveway that is located upon the
parcel at 2210 E. Walnut St. and gave the property owners at 2210 E. Walnut St.
a perpetual easement over the portion of the driveway that is located upon the
Stratman parcel. The Stratman-Connell easement agreement was recorded in the
Vanderburgh County Recorder's Office on June 8, 1994, at 2:25 p.m. The deed
transferring the property at 2210 E. Walnut Street from Connell to Holmes was
also recorded on June 8, 1994, but it was recorded one minute earlier, at 2:24
p.m. It is undisputed that when Holmes sold the property to Hartig, he did not
inform Hartig about the existence of the driveway easement agreement.

Thereafter, on February 13, 1998, the Stratmans filed a complaint alleging that
Hartig was blocking the driveway and refusing to allow them to use it. [The
Stratmans claimed a prescriptive easement to use the driveway; the court
dismissed their complaint.] Then, on August 26, 1998, the Stratmans filed a
``Second Paragraph of Amended Complaint,'' asserting the right to use the
driveway by virtue of the Connell-Stratman easement agreement. Thereafter,
Hartig filed a motion for summary judgment, which the trial court denied on
June 29, 1999. \dots{} 

\readinghead{B.}

Hartig next contends that he is entitled to summary judgment because the
Connell-Stratman driveway easement agreement was recorded outside his chain of
title and therefore not binding on him. Indiana's recording statute provides: 

\begin{quote}
Every conveyance or mortgage of lands or of any interest therein \dots{} shall
be recorded in the recorder's office \dots{} ; and every conveyance [or]
mortgage \dots{} shall take priority according to the time of the filing
thereof, and such conveyance [or] mortgage \dots{} shall be fraudulent and void
as against any subsequent purchaser, \dots{} or mortgagee in good faith and for
a valuable consideration, having his deed [or] mortgage \dots{} first recorded.
\end{quote}

Ind. Code {\S} 32-1-2-16. The purpose of this statute is to provide protection
to subsequent purchasers and mortgagees. This protection is derived from the
fact that a landowner will not be deemed to have constructive notice of adverse
claims that appear outside the chain of title.

To determine the chain of title, the prospective purchaser must go to the
recorder's office and search through the grantor index, beginning with the
person who received the grant of land from the United States and continuing
until the conveyance of the tract in question. The particular grantor's name is
not searched thereafter.

Here, when Hartig conducted a title search of the property at 2210 E. Walnut
St., he would have discovered the conveyance to Connell. Next, Hartig would
have discovered the conveyance from Connell to Holmes that was recorded on June
8, 1994, at 2:24 p.m. Hartig would not have discovered the Connell-Stratford
[Ed: presumably ``Connell-Stratman''] easement agreement that was recorded one
minute later, however, because Connell's name would not have been searched
after the conveyance to Holmes was discovered. Therefore, the easement
agreement is not within Hartig's chain of title and he cannot be deemed to have
constructive notice of its existence. Accordingly, we hold that the trial court
erred in denying Hartig's motion for summary judgment with respect to the issue
of the driveway easement agreement. 

