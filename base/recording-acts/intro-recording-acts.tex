Famed con artist George C. Parker specialized in selling the Brooklyn Bridge.
Parker and other con artists working in New York around the start of the 20th
century would convince victims that they stood to make a fortune charging
tolls. Unfortunately, the buyers obtained nothing, no matter how fancy the
paperwork Parker offered them, because Parker did not own the bridge.
\textit{Nemo dat quod non habet} was the Latin motto of the common law: ``No
man can give what he does not have.'' Parker, having no title, could give none
to his buyers.

Today, ``I've got a bridge to sell you'' is a punchline: only an incredible
rube, we like to believe, could be so gullible as to think that a man in the
street with a ``Bridge for Sale'' sign is actually its owner. But the problem
arises even in less dramatic cases. Suppose Dorothy Dupe is scheduled to buy
Blackacre on Wednesday from Sadie Scamalot. What if on Tuesday Scamalot sells
Blackacre to Charles Clueless first? Then on Wednesday before the ``sale,''
Scamalot is no longer the owner of Blackacre, and under \textit{nemo dat}, Dupe
owns nothing after the ``sale.'' Sometimes, equity would intervene to protect a
second buyer who lacked notice of the prior sale -- but such doctrines have
serious risks for Clueless, who may have no idea that Scamalot is about to turn
around and ``sell'' Blackacre again.

The heart of the problem here is that Clueless and Dupe don't know enough about
potential conflicting claims to Blackacre. Dupe can't find Clueless to confirm
that she should be dealing with him rather than with Scamalot, and Clueless
can't find Dupe to warn her off from buying something Scamalot no longer owns.
Recording systems try to prevent some of these messes by making available
better information about who owns what. If Clueless \textit{recorded} his
interest in Blackacre by making it a matter of public record, then it becomes
reasonable to treat Dupe as having \textit{constructive} \textit{notice} of
Clueless's claim of ownership: even if she didn't check the records, she should
have. Conversely, if Clueless fails to record, there is much less Dupe can do
to protect herself, so it becomes reasonable to let Dupe take title free and
clear of Clueless's claim. Thus, the system gives Clueless a strong incentive
to record and gives Dupe a strong incentive to check the records. As a result,
there are good records of people's property claims. Clueless and Dupe never get
into this mess in the first place, and Scamalot's scheme fails.

Recording systems are useful even in the absence of fraud; they create the trust
and certainty needed to make land transactions common and reliable. Most home
sales today happen between people who do not otherwise know each other and
don't otherwise expect to transact again. How can the buyer be sure the seller
is really the owner? A recording system provides the answer. Perhaps more
importantly, a recording system gives \textit{lenders} sufficient assurance
that they'll be able to recover something in case of a loan default; with that
security, they are willing to loan more and at lower rates. 

For this and other reasons, a recording system can be a vital part of a
large-scale, modern economy. According to the \textit{New York Times}, for
example, the absence of a functioning recording system in Greece ``scares off
foreign investors; makes it hard for the state to privatize its assets, as it
has promised to do in exchange for bailout money; and makes it virtually
impossible to collect property taxes.'' Suzanne Daley,
\href{https://www.nytimes.com/2013/05/27/world/europe/greeces-tangled-land-ownership-is-a-hurdle-in-recovery.html?pagewanted=all}{\textit{Who
Owns This Land? In Greece, Who Knows?}}, \textsc{New York Times}, May 26, 2013.
Clear title is often important for access to government services and even water
and electricity connections: otherwise it's not clear where the checks and
bills should go.

The materials in this section work through real property recording systems from
the inside out. First, an excerpt from an article by Carol Rose traces some
recurring themes in the history of recording law. Second, \textit{Belmont}
considers what it means to ``record'' a document and under what circumstances
it ought to be treated as giving notice. Third, \textit{Hartig} examines the
process of searching title records: what must a reasonable buyer do to confirm
that there are no conflicting claims? \textit{Argent} then considers the main
variations on this scheme under state recording acts, many of which create
additional incentives to record quickly by creating a ``race'' to the recording
office. \textit{Hughes} illustrates the kinds of messes that can arise even
with a reasonably well-functioning recording system. The Note on Informal Title
finishes by considering some of the systemic advantages that have been claimed
for well-functioning recording systems, and some of the empirical evidence for
and against those claims.

