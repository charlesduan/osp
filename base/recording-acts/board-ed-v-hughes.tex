\reading{Board of Education of Minneapolis v. Hughes}
\readingcite{136 N.W. 1095 (Minn. 1912)}

\textsc{BUNN}, J.

Action to determine adverse claims to a lot in Minneapolis. \dots{} The trial
resulted in a decision in favor of plaintiff, and defendants appealed from an
order denying a new trial.

The facts are not in controversy and are as follows: On May 16, 1906, Carrie B.
Hoerger, a resident of Faribault, owned the lot in question, which was vacant
and subject to unpaid delinquent taxes. Defendant L. A. Hughes offered to pay
\$25 for this lot. His offer was accepted, and he sent his check for the
purchase price of this and two other lots bought at the same time to Ed.
Hoerger, husband of the owner, together with a deed to be executed and
returned. The name of the grantee in the deed was not inserted; the space for
the same being left blank. It was executed and acknowledged by Carrie B.
Hoerger and her husband on May 17, 1900 [Ed: presumably 1906], and delivered to
defendant Hughes by mail. The check was retained and cashed. Hughes filled in
the name of the grantee, but not until shortly prior to the date when the deed
was recorded, which was December 11, 1910. On April 27, 1909, Duryea \& Wilson,
real estate dealers, paid Mrs. Hoerger \$25 for a quitclaim deed to the lot,
which was executed and delivered to them, but which was not recorded until
December 21, 1910. On November 19, 1909, Duryea \& Wilson executed and
delivered to plaintiff a warranty deed to the lot, which deed was filed for
record January 27, 1910. It thus appears that the deed to Hughes was recorded
before the deed to Duryea \& Wilson, though the deed from them to plaintiff was
recorded before the deed to defendant.

The questions for our consideration may be thus stated: (1) Did the deed from
Hoerger to Hughes ever become operative? (2) If so, is he a subsequent
purchaser whose deed was first duly recorded, within the language of the
recording act?

The decision of the first question involves a consideration of the effect of the
delivery of a deed by the grantor to the grantee with the name of the latter
omitted from the space provided for it, without express authority to the
grantee to insert his own or another name in the blank space. It is settled
that a deed that does not name a grantee is a nullity, and wholly inoperative
as a conveyance, until the name of the grantee is legally inserted.
\dots{}[T]he deed to defendant Hughes was not operative as a conveyance until
his name was inserted as grantee. [But] Hughes had implied authority from the
grantor to fill the blank with his own name as grantee, and \dots{} when he did
so the deed became operative.

When the Hughes deed was recorded, there was of record a deed to the lot from
Duryea \& Wilson to plaintiff, but no record showing that Duryea \& Wilson had
any title to convey. The deed to them from the common grantor had not been
recorded. We hold that this record of a deed from an apparent stranger to the
title was not notice to Hughes of the prior unrecorded conveyance by his
grantor. He was a subsequent purchaser in good faith for a valuable
consideration, whose conveyance was first duly recorded; that is, Hughes'
conveyance dates from the time when he filled the blank space, which was after
the deed from his grantor to Duryea \& Wilson. He was, therefore, a ``subsequent
purchaser,'' and is protected by the recording of his deed before the prior deed
was recorded. The statute cannot be construed so as to give priority to a deed
recorded before, which shows no conveyance from a record owner. It was
necessary, not only that the deed to plaintiff should be recorded before the
deed to Hughes, but also that the deed to plaintiff's grantor should be first
recorded. 

Our conclusion is that the learned trial court should have held on the evidence
that defendant L. A. Hughes was the owner of the lot.


