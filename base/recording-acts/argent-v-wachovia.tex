\reading{Argent Mortgage Co. v. Wachovia Bank N.A.}
\readingcite{52 So. 3d 796 (Dist. Ct. App. 2010)}

\textsc{Griffin}, J.

Argent Mortgage Company, LLC [``Argent''] appeals the trial court's entry of
judgment in favor of Wachovia Bank National Association, as Trustee Under
Pooling and Servicing Agreement Dated as of November 1, 2004, Asset Backed
Pass--Through Certificates Series 2004--WWF1 [``Wachovia'']. Argent argues that
the trial court erred by finding that the mortgage now owned by Wachovia has
priority over Argent's mortgage. We reverse.

On August 31, 2004, Gene M. Burkes and Ann Burkes [``the Burkes''] as
borrower/mortgagor and Olympus Mortgage Company as lender/mortgagee executed a
mortgage [``the Olympus Mortgage''] on real property as security for a
\$90,000.00 loan. The Olympus Mortgage was recorded on January 5, 2005.
Subsequently, the Olympus Mortgage was assigned to Wachovia. As a result of
default, Wachovia filed a complaint to foreclose the Olympus Mortgage and to
enforce lost loan documents. Wachovia joined Argent as a defendant, alleging
that Argent might claim some interest in or lien upon the subject property by
virtue of a recorded mortgage.

On December 10, 2004, the Burkes as borrower/mortgagor and Argent as
lender/mortgagee executed a mortgage [``the Argent Mortgage''] as security for
a \$65,000.00 loan on the same real property that is the subject of the Olympus
Mortgage. The Argent Mortgage was recorded on January 31, 2005. Subsequently,
Wells Fargo Bank became the owner of the Argent Mortgage. An action to
foreclose the Argent Mortgage was initiated as a result of default.

[Argent and Wachovia filed cross motions for summary judgment.] Ultimately, the
trial court deemed ``the Florida statutes on recordation,'' namely sections
695.01 and 695.11, Florida Statutes, ``to be of the race-notice variety,''
found that the Olympus Mortgage should have priority over the Argent Mortgage,
and entered a partial final judgment in favor of Wachovia.

On appeal,\ldots Argent asserts that section 695.01, Florida Statutes, alone
determines which mortgage has priority, that section 695.01 is, and, for over a
century, has been recognized to be a ``notice'' statute, not a ``race-notice''
statute and that, under section 695.01, the Argent Mortgage has priority over
the Olympus Mortgage.

Wachovia acknowledges that section 695.01, Florida Statutes, is a ``notice''
type of recording statute. However, Wachovia contends that amendments made
to section 695.11, Florida Statutes, have converted Florida into a
``race-notice'' state.

As an initial matter, it bears explaining that recording statutes are classified
into three categories: race, notice, and race-notice. These can generally be
described as follows:
\begin{itemize}
\item Under a \textit{race} recording statute, a subsequent mortgagee of real
property will prevail against a prior mortgagee of the said real property if
the subsequent mortgage is recorded before the prior mortgage.
\item Under a \textit{notice} recording statute, a subsequent mortgagee of real
property for value and without notice (actual and constructive) of a prior
mortgage of the said real property will prevail against the prior mortgagee.
\item Under a \textit{race-notice} recording statute, a subsequent mortgagee of
real property for value and without notice (actual and constructive) of a prior
mortgage of the said real property will prevail against the prior mortgagee if
the subsequent mortgage is recorded before the prior mortgage.
\end{itemize}
Importantly, under either a notice or a race-notice recording statute, the
subsequent mortgagee cannot be without constructive notice if the prior
mortgage has been recorded as of the time of execution of the subsequent
mortgage.

Application of each type of recording statute to the undisputed facts here
yields the following results:
\begin{itemize}
\item Wachovia prevails under a race recording statute because the Olympus
Mortgage was recorded before the Argent Mortgage;
\item Argent prevails under a notice recording statute because it is a
subsequent mortgagee for value and did not have notice of the Olympus Mortgage
at the time of execution of the Argent Mortgage; and
\item Wachovia prevails under a race-notice recording statute because, although
Argent is a subsequent mortgagee for value and did not have notice of the
Olympus Mortgage at the time of execution of the Argent Mortgage, the Olympus
Mortgage was recorded before the Argent Mortgage.
\end{itemize}
Commentators appear uniformly to categorize section 695.01 as a ``notice'' type
of recording statute. \textit{See} 2--26 \textsc{Ralph E.
Boyer, Florida Real Estate Transactions} \S 26.02 (Matthew Bender \& Co.,
Inc. 2010) (``Florida has a notice type recording statute, the primary function
of which is to protect subsequent purchasers (which for purposes of this
discussion includes mortgagees and creditors who are within the statute's
protection) against claims arising from prior unrecorded instruments\ldots.'').

Florida courts over time have described and applied Florida's recording statute
in a manner that is consistent with a ``notice'' type of recording
statute. [citing cases] Florida's approach to the problem was succinctly
described by the Florida Supreme Court in \textit{Van Eepoel Real Estate Co. v.
Sarasota Milk Co}., 100 Fla. 438, 129 So. 892, 895 (1930):
\begin{quote}
[I]t is generally held, in states having recording statutes similar to ours,
that if A conveys lands to B, a bona fide purchaser for value, who does not go
into possession and who failed to record his deed until after A conveys the
same land to C, a second bona fide purchaser for value without notice of B's
interest, and B then records his deed before C records his, the title of C
shall nevertheless prevail as between C and B, because it is the fault of B
that he did not immediately record his deed, thereby permitting C to deal with
the property and part with his consideration without knowledge of B's interest.
So B is estopped and the equities are with C.
\end{quote}

Section 695.01, notwithstanding, the trial court accepted Wachovia's argument
that a 1967 amendment to a different statute, section 695.11, Florida Statutes,
entitled, ``Instruments deemed to be recorded from time of filing'' converted
Florida from a ``notice'' to a ``race-notice'' jurisdiction.\edfootnote{Section
695.11 reads, ``All instruments which are authorized or required to be
recorded in the office of the clerk of the circuit court of any county in the
State of Florida,\dots and which are filed for recording on or after the
effective date of this act, shall be deemed to have been officially accepted by
the said officer, and officially recorded, at the time she or he affixed
thereon the consecutive official register numbers\dots and at such time
shall be notice to all persons. The sequence of such official numbers shall
determine the priority of recordation. An instrument bearing the lower number
in the then-current series of numbers shall have priority over any instrument
bearing a higher number in the same series.''} The earliest version of section
695.11 dates back to 1885. Examination of the language of the 1906, 1920, and
1935 iterations of section 695.11, make clear that this statute was intended to
provide a mechanism for determining the time at which an instrument was deemed
to be recorded. Nothing in the case law suggests that section
695.11 modifies section 695.01.\dots \readingfootnote{3}{Case law confirms that
the purpose of section 695.11 is to determine the time at which an instrument is
deemed to be recorded and to serve as notice. [citing
cases] Section 695.11 has an important purpose to determine the priority
between judgment liens. [citing cases] Because a certified copy of a judgment
must be recorded in order to create a lien on real property, a judgment that is
recorded earlier in time, namely one that bears a lower official register
number, will win priority.} 

Wachovia relies on an earlier opinion of this Court, \textit{Rice v.
Greene}, 941 So.2d 1230 (Fla. 5th DCA 2006), in support of its contention that
Florida has a race-notice type of recording statute.
In \textit{Rice}, this Court\dots found:
\begin{quote}
In other words, an unrecorded deed is not good or effectual in law or equity
against creditors or subsequent purchasers for valuable consideration who are
without notice of the transaction. Therefore, \textit{because Mr. Greene had no
notice} of the earlier warranty deed between Mr. Rice and Mrs.
Schwartz \textit{and paid valuable consideration} for the property, \textit{Mr.
Greene's recording of his warranty deed before Mr. Rice gives Mr. Greene
priority to the property}.
\end{quote}
\textit{Id.} at 1232 (emphasis added). According to Wachovia, this
language proves that priority in recording is key. Notably,
however, \textit{Rice} does not mention section 695.11 and recording was not an
issue. The subsequent purchaser in \textit{Rice} (Mr. Greene) had priority to
the property under a notice type of recording statute because he paid value for
the property and did not have notice (actual or constructive) of the earlier
warranty deed at the time of the conveyance. The fact that Mr. Greene's deed
was recorded before Mr. Rice's does not affect the outcome under a notice type
of recording statute. Although a portion of the sentence
in \textit{Rice}, on which Wachovia relies, mentions recording, in that
case, it was superfluous.

We conclude that Florida is, and remains, a ``notice'' jurisdiction, and notice
controls the issue of priority. Since Argent is a subsequent mortgagee for
value and did not have notice of the Olympus Mortgage at the time of execution
of the Argent Mortgage, the Argent Mortgage has priority over the Olympus
Mortgage. As such, the trial court erred by entering partial summary final
judgment in favor of Wachovia on the issue of priority.

