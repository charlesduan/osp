\defbook{grimmelmann-patterns-information}{
James Grimmelmann, Patterns of Information Law (2017);
edition=version 1.1,
url=https://james.grimmelmann.net/ipbook/ipbook.1.1.pdf,
}


To help make different parts of the book easier to identify, two fonts are used
throughout. {\readingfont Text in this serif font} is used for cases, law review
articles, and other readings drawn from other sources and included in this book.
In these readings, not all alterations will necessarily be noted. In particular,
citations will often be removed for ease of reading. To quote the policy of one
of the casebook's authors:
\begin{quote}
My editorial technique is borrowed from Sweeney Todd: extensive and shocking
cuts. These are pedagogical materials, not a legal brief. I have not put words
in anyone else’s mouth, but I have been unconcerned with the usual editorial
apparatus of ellipses and brackets. I drop words from sentences, sentences from
paragraphs, paragraphs from opinions – all with no indication that anything is
gone. I also reorder paragraphs and sometimes sentences as needed to improve the
readability of a passage. My goal is to make it easy for the reader. If it
matters to you what the original said, consult the original.
\end{quote}
\sentence{grimmelmann-patterns-information at 32}. Note the font used in this
quote, since it is drawn from an external source.

{\edfont Text in this sans serif font} is ``editorial content,'' namely
introductory or narrative material written by the authors and editors of the
book. In some cases, this material is intended to elucidate the readings and
provoke thoughts and questions. In other cases, the text summarizes key
doctrinal or legal concepts, in order to be more efficient (i.e., so you have to
read less).

The font convention is followed as rigorously as possible. In particular, some
of the readings will include footnotes from the original text, as well as
footnotes added by the editors. These footnotes can be distinguished by the font
being used (as well as the notation ``---Eds.\@'' at the end of the footnote).


