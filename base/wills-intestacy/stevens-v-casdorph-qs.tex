\expected{fuller-consideration}
\expected{langbein-substantial}
\expected{stevens-v-casdorph}

\item \textbf{Wills Act Formalities}. The requirements to make a valid will vary
from state to state, but in general a will must be in writing, signed by the
testator, and attested by two witnesses. How well do these formalities serve
the various purposes identified by Fuller and Langbein? Which of them failed in
\textit{Stevens v. Casdorph}? How? Why is the court so stringent about
enforcing the formalities?


\item \textbf{Informal Wills}. Whether out of ignorance about the law,
skittishness in thinking about their own death, or bad advice, people do all
kinds of things that blatantly fail to qualify as wills under the traditional
test. They write chatty emails to family members explaining what they want to
happen to their property after their death; they scrawl marginalia on old
wills, crossing out specific bequests and adding new ones; they leave behind
multiple conflicting undated ``last'' wills. What should courts do in such
cases? In one memorably tragic case, Cecil George Harris used his pocketknife
to scratch the words, ``In case I die in this mess, I leave all to the wife.
Cecil Geo Harris'' into the fender of a tractor he was fatally
pinned under. It was upheld as a valid \textit{holographic} \textit{will}: a
will that has been handwritten and signed by the testator. A majority of states
recognize holographic wills, although their specific requirements vary and an
estates attorney should never rely on the validity of one. (For example,
Maryland recognizes holographic wills only by testators serving in the armed
services abroad. \textsc{Md. Code Estates \& Trusts} \S~4-103(a)).


\item \textbf{Interpretive Problems}. The general interpretive rule for wills is
the ``intent of the testator.'' Is there any reason this might be a harder
problem for wills than for other types of legal instruments? Consider:
\begin{itemize}
\item T's will leaves ``all my property to my daughters A and B.'' Five years
after making the will but ten years before his death, T and his wife have
another child, C.
\item T's will leaves ``my red Toyota to my nephew A.'' After making the will, T
wrecks the red Toyota and buys a blue Toyota to replace it.
\item T's will leaves \$10,000 to A, \$10,000 to B, and his antique writing desk
to C. After expenses, T's estate consists of \$5,000 in cash and the writing
desk.
\item T's will leaves his estate equally to his sisters A and B. A dies in the
same car accident as T. She leaves behind two children, C and D. T has one
child of his own, E, from whom he is estranged.
\end{itemize}

