\reading[Langbein, \emph{Substantial Compliance with the Wills Act}]{John H.
Langbein, \textit{Substantial Compliance with the Wills Act}}

\readingcite{88 \textsc{Harv. L. Rev.} 489 (1975)}

4. \textit{The Protective Function}.---Courts have traditionally attributed
to the Wills Act the object ``of protecting the testator against imposition at
the time of execution.'' The requirement that attestation be made in the
presence of the testator is meant ``to prevent the substitution of a
surreptitious will.'' Another common protective requirement is the rule that
the witnesses should be disinterested, hence not motivated to coerce or deceive
the testator.

