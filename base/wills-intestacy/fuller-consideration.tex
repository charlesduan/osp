\reading[Fuller, \emph{Consideration and Form}]{Lon L. Fuller, \textit{Consideration and Form}}
\readingcite{41 \textsc{Colum. L. Rev.} 799 (1941)}

{\S} 2. \textit{The Evidentiary Function}.---The most obvious function of a
legal formality is, to use Austin's words, that of providing ``evidence of the
existence and purport of the contract, in case of controversy.'' The need for
evidentiary security may be satisfied in a variety of ways: by requiring a
writing, or attestation, or the certification of a notary. It may even be
satisfied, to some extent, by such a device as the Roman stipulatio, which
compelled an oral spelling out of the promise in a manner sufficiently
ceremonious to impress its terms on participants and possible bystanders. 

{\S} 3. \textit{The Cautionary Function.}---A formality may also perform a
cautionary or deterrent function by acting as a check against inconsiderate
action. The seal in its original form fulfilled this purpose remarkably well.
The affixing and impressing of a wax wafer-symbol in the popular mind of
legalism and weightiness-was an excellent device for inducing the
circumspective frame of mind appropriate in one pledging his future. To a less
extent any requirement of a writing, of course, serves the same purpose, as do
requirements of attestation, notarization, etc. 

{\S} 4. \textit{The Channeling Function}.---\ldots That a legal formality may
perform a function not yet described can be shown by the seal. The seal not
only insures a satisfactory memorial of the promise and induces deliberation in
the making of it. It serves also to mark or signalize the enforceable promise;
it furnishes a simple and external test of enforceability.\ldots The thing which
characterizes the law of contracts and conveyances is that in this field forms
are deliberately used, and are intended to be so used, by the parties whose
acts are to be judged by the law. To the business man who wishes to make his
own or another's promise binding, the seal was at common law available as a
device for the accomplishment of his objective. In this aspect form offers a
legal framework into which the party may fit his actions, or, to change the
figure, it offers channels for the legally effective expression of intention. 

