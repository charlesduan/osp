\reading{Stevens v. Casdorph}
\readingcite{203 W. Va. 450 (1988)}

\textsc{Per Curiam}: \dots

 On May 28, 1996, [Patricia Eileen Casdorph and Paul Douglas Casdorph] took Mr.
Homer Haskell Miller to Shawnee Bank in Dunbar, West Virginia, so that he could
execute his will. Once at the bank, Mr. Miller asked Debra Pauley, a bank
employee and public notary, to witness the execution of his will. After Mr.
Miller signed the will, Ms. Pauley took the will to two other bank employees,
Judith Waldron and Reba McGinn, for the purpose of having each of them sign the
will as witnesses. Both Ms. Waldron and Ms. McGinn signed the will. However,
Ms. Waldron and Ms. McGinn testified during their depositions that they did not
actually see Mr. Miller place his signature on the will. Further, it is
undisputed that Mr. Miller did not accompany Ms. Pauley to the separate work
areas of Ms. Waldron and Ms. McGinn. 

Mr. Miller died on July 28, 1996. The last will and testament of Mr. Miller,
which named Mr. Paul Casdorph as executor, left the bulk of his estate to the
Casdorphs. The Stevenses, nieces of Mr. Miller, filed the instant action to set
aside the will.\ldots\unskip\readingfootnote{4}{As heirs, the Stevenses would be
entitled to recover from Mr. Miller's estate under the intestate laws if his
will is set aside as invalidly executed.\ldots} 

The Stevenses' contention is simple. They argue that all evidence indicates that
Mr. Miller's will was not properly executed. Therefore, the will should be
voided. The procedural requirements at issue are contained in W.Va. Code
{\S}~41-1-3 (1997). The statute reads: 
\begin{quote}
No will shall be valid unless it be in writing and signed by the testator, or by
some other person in his presence and by his direction, in such manner as to
make it manifest that the name is intended as a signature; and moreover, unless
it be wholly in the handwriting of the testator, \textit{the signature shall be
made or the will acknowledged by him in the presence of at least two competent
witnesses, present at the same time; and such witnesses shall subscribe the
will in the presence of the testator, and of each other}, but no form of
attestation shall be necessary. (Emphasis added.) 
\end{quote}
The relevant requirements of the above statute calls for a testator to sign
his/her will or acknowledge such will in the presence of at least two witnesses
at the same time, and such witnesses must sign the will in the presence of the
testator and each other. In the instant proceeding the Stevenses assert, and
the evidence supports, that Ms. McGinn and Ms. Waldron did not actually witness
Mr. Miller signing his will. Mr. Miller made no acknowledgment of his signature
on the will to either Ms. McGinn or Ms. Waldron. Likewise, Mr. Miller did not
observe Ms. McGinn and Ms. Waldron sign his will as witnesses. Additionally,
neither Ms. McGinn nor Ms. Waldron acknowledged to Mr. Miller that their
signatures were on the will. It is also undisputed that Ms. McGinn and Ms.
Waldron did not actually witness each other sign the will, nor did they
acknowledge to each other that they had signed Mr. Miller's will.\ldots

Our analysis begins by noting that ``the law favors testacy over intestacy.''
However, we clearly held in syllabus point 1 of \textit{Black v. Maxwell, }131
W. Va. 247, 46 S.E.2d 804 (1948), that ``testamentary intent and a written
instrument, executed in the manner provided by [W.Va. Code {\S}~41-1-3],
existing concurrently, are essential to the creation of a valid will.''
\textit{Black} establishes that mere intent by a testator to execute a written
will is insufficient. The actual execution of a written will must also comply
with the dictates of W.Va. Code \S~41-1-3. The Casdorphs seek to have this
Court establish an exception to the technical requirements of the statute. In
\textit{Wade v. Wade}, 119 W. Va. 596 (1938), this Court permitted a narrow
exception to the stringent requirements of the W.Va. Code \S~41-1-3. This
narrow exception is embodied in syllabus point 1 of \textit{Wade}: 
\begin{quote}
Where a testator acknowledges a will and his signature thereto in the presence
of two competent witnesses, one of whom then subscribes his name, the other or
first witness, having already subscribed the will in the presence of the
testator but out of the presence of the second witness, may acknowledge his
signature in the presence of the testator and the second witness, and such
acknowledgment, if there be no indicia of fraud or misunderstanding in the
proceeding, will be deemed a signing by the first witness within the
requirement of Code, 41-1-3, that the witnesses must subscribe their names in
the presence of the testator and of each other.\dots
\end{quote}

\textit{Wade} stands for the proposition that if a witness acknowledges his/her
signature on a will in the physical presence of the other subscribing witness
\textit{and the testator}, then the will is properly witnessed within the terms
of W.Va. Code \S~41-1-3. In this case, none of the parties signed or
acknowledged their signatures in the presence of each other. This case meets
neither the narrow exception of \textit{Wade} nor the specific provisions of
W.Va. Code \S~41-1-3. 

\textsc{Workman}, J., dissenting: 

The majority once more takes a very technocratic approach to the law, slavishly
worshiping form over substance. In so doing, they not only create a harsh and
inequitable result wholly contrary to the indisputable intent of Mr. Homer
Haskell Miller, but also a rule of law that is against the spirit and intent of
our whole body of law relating to the making of wills. 

There is absolutely no claim of incapacity or fraud or undue influence, nor any
allegation by any party that Mr. Miller did not consciously, intentionally, and
with full legal capacity convey his property as specified in his will. The
challenge to the will is based solely upon the allegation that Mr. Miller did
not comply with the requirement of West Virginia Code 41-1-3 that the signature
shall be made or the will acknowledged by the testator in the presence of at
least two competent witnesses, present at the same time. The lower court, in
its very thorough findings of fact, indicated that Mr. Miller had been
transported to the bank by his nephew Mr. Casdorph and the nephew's wife. Mr.
Miller, disabled and confined to a wheelchair, was a shareholder in the Shawnee
Bank in Dunbar, West Virginia, with whom all those present were personally
familiar. When Mr. Miller executed his will in the bank lobby, the typed will
was placed on Ms. Pauley's desk, and Mr. Miller instructed Ms. Pauley that he
wished to have his will signed, witnessed, and acknowledged. After Mr. Miller's
signature had been placed upon the will with Ms. Pauley watching, Ms. Pauley
walked the will over to the tellers' area in the same small lobby of the bank.
Ms. Pauley explained that Mr. Miller wanted Ms. Waldron to sign the will as a
witness. The same process was used to obtain the signature of Ms. McGinn.
Sitting in his wheelchair, Mr. Miller did not move from Ms. Pauley's desk
during the process of obtaining the witness signatures. The lower court
concluded that the will was valid and that Ms. Waldron and Ms. McGinn signed
and acknowledged the will ``in the presence'' of Mr. Miller.\ldots

The majority embraces the line of least resistance. The easy, most convenient
answer is to say that the formal, technical requirements have not been met and
that the will is therefore invalid. End of inquiry. Yet that result is patently
absurd. That manner of statutory application is inconsistent with the
underlying purposes of the statute. Where a statute is enacted to protect and
sanctify the execution of a will to prevent substitution or fraud, this Court's
application of that statute should further such underlying policy, not impede
it. When, in our efforts to strictly apply legislative language, we abandon
common sense and reason in favor of technicalities, we are the ones committing
the injustice. 

