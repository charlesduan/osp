\reading{Maryland Code, Estates and Trusts}


\readinghead{\S 3-101. Order of distribution of net intestate estate}

Any part of the net estate of a decedent not effectively disposed of by his will
shall be distributed by the personal representative to the heirs of the
decedent in the order prescribed in this subtitle.


\readinghead{\S 3-102. Share of surviving spouse}

(a) In general.---The share of a surviving spouse shall be as provided in this
section.

(b) Surviving minor child.---If there is a surviving minor child, the share
shall be one-half.

(c) No surviving minor child, but surviving issue.---If there is no surviving
minor child, but there is surviving issue, the share shall be the first \$
15,000 plus one-half of the residue.

(d) No surviving issue, but surviving parent.---If there is no surviving issue
but a surviving parent, the share shall be the first \$ 15,000 plus one-half of
the residue.

(e) No surviving issue or parent.---If there is no surviving issue or parent,
the share shall be the whole estate. \dots


\readinghead{\S 3-103. Division among surviving issue}

The net estate, exclusive of the share of the surviving spouse, or the entire
net estate if there is no surviving spouse, shall be divided equally among the
surviving issue.\ldots


\readinghead{\S 3-104. Distribution when there is no surviving issue}

\ldots.

(b) Parents and their issue.---\ldots it shall be distributed to the
surviving parents equally, or if only one parent survives, to the survivor; or
if neither parent survives, to the issue of the parents, by representation.

(c) Grandparents and their issue.---\ldots

(d) Great-grandparents and their issue.---\ldots

(e) No surviving blood relative.---If there is no surviving blood relative
entitled to inherit under this section, it shall be divided into as many equal
shares as there are stepchildren of the decedent\ldots.

