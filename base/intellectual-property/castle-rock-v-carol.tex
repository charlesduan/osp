\reading{Castle Rock Entertainment v. Carol Publishing Group}
\readingcite{150 F.3d 132 (2d Cir. 1998)}

JOHN M. WALKER, Jr., Circuit Judge:

This case presents two interesting and somewhat novel issues of copyright law.
The first is whether The Seinfeld Aptitude Test, a trivia quiz book devoted
exclusively to testing its readers' recollection of scenes and events from the
fictional television series Seinfeld, takes sufficient protected expression
from the original, as evidenced by the book's substantial similarity to the
television series, such that, in the absence of any defenses, the book would
infringe the copyright in Seinfeld. The second is whether The Seinfeld Aptitude
Test (also referred to as The SAT) constitutes fair use of the Seinfeld
television series.\ldots

We conclude that The SAT unlawfully copies from Seinfeld and that its copying
does not constitute fair use and thus is an actionable infringement.
Accordingly, we affirm the judgment in favor of Castle Rock.

\readinghead{Background}

The material facts in this case are undisputed. Plaintiff Castle Rock is the
producer and copyright owner of each episode of the Seinfeld television series.
The series revolves around the petty tribulations in the lives of four single,
adult friends in New York: Jerry Seinfeld, George Costanza, Elaine Benes, and
Cosmo Kramer. Defendants are Beth Golub, the author, and Carol Publishing
Group, Inc., the publisher, of The SAT, a 132-page book containing 643 trivia
questions and answers about the events and characters depicted in Seinfeld.
These include 211 multiple choice questions, in which only one out of three to
five answers is correct; 93 matching questions; and a number of short-answer
questions. The questions are divided into five levels of difficulty, labeled
(in increasing order of difficulty) ``Wuss Questions,'' ``This, That, and the
Other Questions,'' ``Tough Monkey Questions,'' ``Atomic Wedgie Questions,'' and
``Master of Your Domain Questions.'' Selected examples from level 1 are
indicative of the questions throughout The SAT:
\begin{quotation}
1. To impress a woman, George passes himself off as

a) a gynecologist

b) a geologist

c) a marine biologist

d) a meteorologist

11. What candy does Kramer snack on while observing a surgical procedure from an
operating-room balcony?

12. Who said, ``I don't go for those nonrefundable deals ... I can't commit to a
woman ... I'm not committing to an airline.''?

a) Jerry

b) George

c) Kramer
\end{quotation}
The book draws from 84 of the 86 Seinfeld episodes that had been broadcast as of
the time The SAT was published. Although Golub created the incorrect answers to
the multiple choice questions, every question and correct answer has as its
source a fictional moment in a Seinfeld episode. Forty-one questions and/or
answers contain dialogue from Seinfeld. The single episode most drawn upon by
The SAT, ``The Cigar Store Indian,'' is the source of 20 questions that
directly quote between 3.6\% and 5.6\% of that episode (defendants' and
plaintiffs calculations, respectively).

The name ``Seinfeld'' appears prominently on the front and back covers of The
SAT, and pictures of the principal actors in Seinfeld appear on the cover and
on several pages of the book. On the back cover, a disclaimer states that
``This book has not been approved or licensed by any entity involved in
creating or producing Seinfeld.'' The front cover bears the title ``The
Seinfeld Aptitude Test'' and describes the book as containing ``[h]undreds of
spectacular questions of minute details from TV's greatest show about
absolutely nothing.'' The back cover asks:
\begin{quotation}
Just how well do you command the buzz-words, peccadilloes, petty annoyances, and
triflingly complex escapades of Jerry Seinfeld, Elaine Benes, George Costanza,
and Kramer---the fabulously neurotic foursome that makes the offbeat hit TV
series Seinfeld tick?

\ldots.

If you think you know the answers---and really keep track of \textit{Seinfeld}
minutiae---challenge yourself and your friends with these 550 trivia
questions and 10 extra matching quizzes. No, \textit{The Seinfeld Aptitude
Test} can't tell you whether you're Master of Your Domain, but it will certify
your status as King or Queen of \textit{Seinfeld} trivia. So twist open a
Snapple, double-dip a chip, and open this book to satisfy your between-episode
cravings.
\end{quotation}

Golub has described The SAT as a ``natural outgrowth'' of Seinfeld which, ``like
the Seinfeld show, is devoted to the trifling, picayune and petty annoyances
encountered by the show's characters on a daily basis.'' According to Golub,
she created The SAT by taking notes from Seinfeld programs at the time they
were aired on television and subsequently reviewing videotapes of several of
the episodes, as recorded by her or various friends.

The SAT's publication did not immediately provoke a challenge. The National
Broadcasting Corporation, which broadcasted Seinfeld, requested free copies of
The SAT from defendants and distributed them together with promotions for the
program. Seinfeld's executive producer characterized The SAT as ``a fun little
book.'' There is no evidence that The SAT's publication diminished Seinfeld's
profitability, and in fact Seinfeld's audience grew after The SAT was first
published.

Castle Rock has nevertheless been highly selective in marketing products
associated with Seinfeld, rejecting numerous proposals from publishers seeking
approval for a variety of projects related to the show. Castle Rock licensed
one Seinfeld book, The Entertainment Weekly Seinfeld Companion, and has
licensed the production of a CD-ROM product that includes discussions of
Seinfeld episodes; the CD-ROM allegedly might ultimately include a trivia bank.
Castle Rock claims in this litigation that it plans to pursue a more aggressive
marketing strategy for Seinfeld-related products, including ``publication of
books relating to Seinfeld.''

In November 1994, Castle Rock notified defendants of its copyright and trademark
infringement claims. In February 1995, after defendants continued to distribute
The SAT, Castle Rock filed this action alleging federal copyright and trademark
infringement and state law unfair competition. Subsequently, both parties
moved, pursuant to Fed.R.Civ.P. 56, for summary judgment on both the copyright
and unfair competition claims.

The district court granted summary judgment to Castle Rock on the copyright
claim. It held that defendants had violated plaintiff's copyrights in Seinfeld
and that such copying did not constitute fair use.\ldots The parties then
stipulated to damages and attorneys' fees on the copyright infringement claim
and, presumably to facilitate the appeal, to the dismissal without prejudice of
all remaining claims.\ldots The district court entered final judgment on the
copyright infringement claim, awarded Castle Rock \$403,000 with interest,
permanently enjoined defendants from publishing or distributing The SAT, and
ordered defendants to destroy all copies of The SAT in their custody or
control. Defendants now appeal.

\readinghead{II. Copyright Infringement}

The Copyright Act of 1976 (``Copyright Act''), 17 U.S.C. {\S}{\S} 101-803,
grants copyright owners a bundle of exclusive rights, including the rights to
``reproduce the copyrighted work in copies'' and ``to prepare derivative works
based upon the copyrighted work.'' \textit{Id.} {\S} 106. ``Copyright
infringement is established when the owner of a valid copyright demonstrates
unauthorized copying.'' \textit{Repp v. Webber,} 132 F.3d 882, 889 (2d
Cir. 1997); \textit{see Feist Publications, Inc. v. Rural Tel. Serv. Co.,} 499
U.S. 340, 361, 111 S.Ct. 1282, 113 L.Ed.2d 358 (1991). There are two main
components of this \textit{prima facie} case of infringement: ``a plaintiff
must first show that his work was actually copied\ldots [and] then must show
that the copying amounts to an improper or unlawful appropriation.''
\textit{Laureyssens v. Idea Group, Inc.,} 964 F.2d 131, 139-40 (2d Cir.1992)
(quotation marks and citations omitted). Actual copying may be established
``either by direct evidence of copying or by indirect evidence, including
access to the copyrighted work, similarities that are probative of copying
between the works, and expert testimony.'' \textit{Id.} at 140.\ldots ``It is
only after actual copying is established that one claiming infringement'' then
proceeds to demonstrate that the copying was improper or unlawful by showing
that the second work bears ``substantial similarity'' to protected expression
in the earlier work. \textit{Webber,} 132 F.3d at 889; \textit{Laureyssens},
964 F.2d at 140.

In the instant case, no one disputes that Castle Rock owns valid copyrights in
the Seinfeld television programs and that defendants actually copied from those
programs in creating The SAT. Golub freely admitted that she created The SAT by
taking notes from Seinfeld programs at the time they were aired on television
and subsequently reviewing videotapes of several of the episodes that she or
her friends recorded. Since the fact of copying is acknowledged and undisputed,
the critical question for decision is whether the copying was unlawful or
improper in that it took a sufficient amount of protected expression from
Seinfeld as evidenced by its substantial similarity to such expression.

\readinghead{A. ``Substantial Similarity''}

We have stated that ``substantial similarity''
\begin{quote}
requires that the copying [be] quantitatively \textit{and} qualitatively
sufficient to support the legal conclusion that infringement
(\textit{actionable} copying) has occurred. The qualitative component concerns
the copying of expression, rather than ideas [, facts, works in the public
domain, or any other non-protectable elements]\ldots. The quantitative component
generally concerns the amount of the copyrighted work that is copied,
\end{quote}
which must be more than ``de minimis.'' \emph{Ringgold v. Black Entertainment
Television, Inc.}, 126 F.3d 70, 75 (2d Cir.1997) (emphasis added).

As to the quantitative element, we conclude that The SAT has crossed the de
minimis threshold. At the outset, we observe that the fact that the copying
appears in question and answer form is by itself without particular
consequence: the trivia quiz copies fragments of Seinfeld in the same way that
a collection of Seinfeld jokes or trivia would copy fragments of the
series.\dots Had The SAT copied a few fragments from each of 84 unrelated
television
programs (perhaps comprising the entire line-up on broadcast television),
defendants would have a stronger case under the de minimis doctrine. By copying
not a few but 643 fragments from the Seinfeld television series, however, The
SAT has plainly crossed the quantitative copying threshold under
\emph{Ringgold}.

As to \emph{Ringgold}'s qualitative component, each SAT trivia question is based
directly upon original, protectable expression in Seinfeld. As noted by the
district court, The SAT did not copy from Seinfeld unprotected facts, but,
rather, creative expression. \emph{Cf.} \emph{Feist}, 499 U.S. at 364, 111 S.Ct.
1282
(finding no infringement where defendant produced a multi-county phone
directory, in part, by obtaining names and phone numbers from plaintiffs'
single-county directory). Unlike the facts in a phone book, which ``do not owe
their origin to an act of authorship,'' \emph{id.} at 347, 111 S.Ct. 1282, each
``fact'' tested by The SAT is in reality fictitious expression created by
Seinfeld's authors. The SAT does not quiz such true facts as the identity of
the actors in Seinfeld, the number of days it takes to shoot an episode, the
biographies of the actors, the location of the Seinfeld set, etc. Rather, The
SAT tests whether the reader knows that the character Jerry places a Pez
dispenser on Elaine's leg during a piano recital, that Kramer enjoys going to
the airport because he's hypnotized by the baggage carousels, and that Jerry,
opining on how to identify a virgin, said ``It's not like spotting a toupee.''
Because these characters and events spring from the imagination of Seinfeld's
authors, The SAT plainly copies copyrightable, creative expression.

We find support for this conclusion in a previous case in which we held that a
series of still photographs of a ballet may in some cases infringe the
copyright in an original choreographic work. \emph{See} \emph{Horgan v.
Macmillan, Inc.}, 789
F.2d 157, 163 (2d Cir.1986). The defendants in Horgan claimed that still
photographs could not ``capture the flow of movement, which is the essence of
dance,'' that ``the staged performance could not be recreated from the
photographs,'' and thus, that the photographs were not substantially similar to
the choreographic work. Id. at 161-62 (quotation marks omitted). Although
noting that the issue ``was not a simple one,'' this court rejected that
argument, holding that ``the standard for determining copyright infringement is
not whether the original could be recreated from the allegedly infringing copy,
but whether the latter is substantially similar to the former.'' \emph{Id.} at
162
(quotation marks omitted). That observation applies with equal force to the
trivia quiz fragments in this case. Although Seinfeld could not be
``recreated'' from The SAT, Castle Rock has nevertheless established both the
quantitative and qualitative components of the substantial similarity test,
establishing a prima facie case of copyright infringement.

\readinghead{B. Other Tests}

As defendants note, substantial similarity usually ``arises out of a claim of
infringement as between comparable works\ldots [where] because of the equivalent
nature of the competing works, the question of similarity can be tested
conventionally by comparing comparable elements of the two works.'' Because in
the instant case the original and secondary works are of different genres and
to a lesser extent because they are in different media, tests for substantial
similarity other than the quantitative/qualitative approach are not
particularly helpful to our analysis.

Under the ``ordinary observer'' test, for example, ``[t]wo works are
substantially similar where `the ordinary observer, unless he
set out to detect the disparities, would be disposed to overlook them, and
regard [the] aesthetic appeal [of the two works] as the same.'\,'' \emph{Arica
Inst.,
Inc. v. Palmer}, 970 F.2d 1067, 1072 (2d Cir.1992) (quoting \emph{Peter Pan Fabrics,
Inc. v. Martin Weiner Corp.}, 274 F.2d 487, 489 (2d Cir.1960) (L. Hand, J.)
(comparing dress designs)) (alterations in original). Undoubtedly, Judge Hand
did not have in mind a comparison of aesthetic appeal as between a television
series and a trivia quiz and, in the usual case, we might question whether any
``ordinary observer'' would ``regard [the] aesthetic appeal'' in a
situation-comedy television program as being identical to that of any book, let
alone a trivia quiz book, about that program. \emph{Cf.} \emph{Laureyssens}, 964
F.2d at 132,
141 (applying ``ordinary observer'' test to compare two sets of foam rubber
puzzles). We note here, however, that plaintiff has a plausible claim that
there is a common aesthetic appeal between the two works based on The SAT's
plain copying of Seinfeld and Golub's statement on the back cover that the book
was designed to complement the aesthetic appeal of the television series. See
The SAT (``So twist open a Snapple, double-dip a chip, and open this book to
satisfy your between episode cravings.'').

Under the ``total concept and feel'' test, urged by defendants, we analyze ``the
similarities in such aspects as the total concept and feel, theme, characters,
plot, sequence, pace, and setting'' of the original and the allegedly
infringing works. \emph{Williams v. Crichton}, 84 F.3d 581, 588 (2d Cir. 1996)
(comparing children's books with novel and movie); \emph{Reyher v. Children's
Television Workshop}, 533 F.2d 87, 91 (2d Cir. 1976) (comparing children's book
with story in Sesame Street Magazine). Defendants contend that The SAT and the
Seinfeld programs are incomparable in conventional terms such as plot,
sequence, themes, pace, and setting. For example, The SAT has no plot; ``[t]he
notion of pace\ldots cannot be said even to exist in the book''; The SAT's
``sequence has no relationship to the sequences of any of the Seinfeld
episodes, since it is a totally random and scattered collection of questions
relating to events that occurred in the shows''; and The SAT's only theme ``is
how much a Seinfeld fan can remember of 84 different programs.'' The total
concept and feel test, however, is simply not helpful in analyzing works that,
because of their different genres and media, must necessarily have a different
concept and feel. Indeed, many ``derivative'' works of different genres, in
which copyright owners have exclusive rights, see 17 U.S.C. {\S} 106, may have
a different total concept and feel from the original work.\ldots

The SAT easily passes the threshold of substantial similarity between the
contents of the secondary work and the protected expression in the original.

\readinghead{III. Fair Use}

Defendants claim that, even if The SAT's copying of Seinfeld constitutes prima
facie infringement, The SAT is nevertheless a fair use of Seinfeld. ``From the
infancy of copyright protection,'' the fair use doctrine ``has been thought
necessary to fulfill copyright's very purpose, `[t]o promote
the Progress of Science and useful Arts.'\,'' \emph{Campbell v. Acuff-Rose
Music, Inc.}, 510 U.S. 569, 575, 114 S.Ct. 1164, 127 L.Ed.2d 500 (1994) (quoting
U.S. Const., art. I, {\S} 8, cl. 8). As noted in \emph{Campbell}, ``in truth, in
literature, in science and in art, there are, and can be, few, if any, things,
which in an abstract sense, are strictly new and original throughout. Every
book in literature, science and art, borrows, and must necessarily borrow, and
use much which was well known and used before.'' Id. (quotation marks omitted).
Until the 1976 Copyright Act, the doctrine of fair use grew exclusively out of
the common law. See \emph{id.} at 576, 114 S.Ct. 1164; \emph{Folsom v. Marsh}, 9
F.Cas. 342,
348 (C.D.Mass.1900) (CCD Mass. 1841) (Story, J.) (stating fair use test);
Pierre N. Leval, \emph{Toward a Fair Use Standard}, 103 Harv. L. Rev. 1105, 1105
(1990)
(``Leval'').

In the Copyright Act, Congress restated the common law tradition of fair use:
\begin{quote}
[T]he fair use of a copyrighted work\ldots for purposes such as criticism,
comment, news reporting, teaching (including multiple copies for classroom
use), scholarship, or research, is not an infringement of copyright. In
determining whether the use made of a work in any particular case is a fair use
the factors to be considered shall include---
\begin{quotation}
(1) the purpose and character of the use, including whether such use is of a
commercial nature or is for nonprofit educational purposes;

(2) the nature of the copyrighted work;

(3) the amount and substantiality of the portion used in relation to the
copyrighted work as a whole; and

(4) the effect of the use upon the potential market for or value of the
copyrighted work.
\end{quotation}
\end{quote}
17 U.S.C. {\S} 107. This section ``intended that courts continue the common law
tradition of fair use adjudication'' and ``permits and requires courts to avoid
rigid application of the copyright statute, when, on occasion, it would stifle
the very creativity which that law is designed to foster.'' \emph{Campbell}, 510
U.S.
at 577, 114 S.Ct. 1164 (quotation marks omitted). Fair use analysis, therefore,
always ``calls for case-by-case analysis.'' Id. The fair use examples provided
in {\S} 107 are ``illustrative and not limitative'' and ``provide only general
guidance about the sorts of copying that courts and Congress most commonly had
found to be fair uses.'' \emph{Id.} at 577-78, 114 S.Ct. 1164. Similarly, the
four
listed statutory factors in {\S} 107 guide but do not control our fair use
analysis and ``are to be explored, and the results weighed together, in light
of the purposes of copyright.'' \emph{Id.}; see 4 Nimmer {\S} 13.05[A], at
13-153
(``[T]he factors contained in Section 107 are merely by way of example, and are
not an exhaustive enumeration.''). The ultimate test of fair use, therefore, is
whether the copyright law's goal of ``promot[ing] the Progress of Science and
useful Arts,'' U.S. Const., art. I, {\S} 8, cl. 8, ``would be better served by
allowing the use than by preventing it.'' \emph{Arica}, 970 F.2d at 1077.

\readinghead{A. Purpose/Character of Use}

The first fair use factor to consider is ``the purpose and character of the
[allegedly infringing] use, including whether such use is of a commercial
nature or is for nonprofit educational purposes.'' 17 U.S.C. {\S} 107(1). That
The SAT's use is commercial, at most, ``tends to weigh against a finding of
fair use.'' \emph{Campbell}, 510 U.S. at 585, 114 S.Ct. 1164 (quotation marks
omitted); \emph{Texaco}, 60 F.3d at 921. But we do not make too much of this
point. As
noted in \emph{Campbell}, ``nearly all of the illustrative uses listed in the
preamble
paragraph of {\S} 107, including news reporting, comment, criticism, teaching,
scholarship, and research ... are generally conducted for profit in this
country,'' 510 U.S. at 584, 114 S.Ct. 1164 (quotation marks omitted), and ``no
man but a blockhead ever wrote, except for money,'' \emph{id.} (quoting 3
Boswell's \emph{Life of Johnson} 19 (G. Hill ed.1934)). We therefore do not give
much weight to the fact that the secondary use was for commercial gain.

The more critical inquiry under the first factor and in fair use analysis
generally is whether the allegedly infringing work ``merely supersedes'' the
original work ``or instead adds something new, with a further purpose or
different character, altering the first with new\ldots meaning [] or message,''
in other words ``whether and to what extent the new work is
`transformative.'\,'' \emph{Id.} at 579, 114 S.Ct. 1164 (quoting Leval
at 1111). If ``the secondary use adds value to the original---if
[copyrightable expression in the original work] is used as raw material,
transformed in the creation of new information, new aesthetics, new insights
and understandings---this is the very type of activity that the fair use
doctrine intends to protect for the enrichment of society.'' Leval at 1111. In
short, ``the goal of copyright, to promote science and the arts, is generally
furthered by the creation of transformative works.'' \emph{Campbell}, 510 U.S.
at 579,
114 S.Ct. 1164.

Defendants claim two primary ``transformative'' qualities of The SAT. First, as
noted by the district court, ``a text testing one's knowledge of Joyce's
Ulysses, or Shakespeare's Hamlet, would qualify as `criticism,
comment, scholarship, or research,' or such. The same must be said, then, of a
text testing one's knowledge of Castle Rock's Seinfeld.'' \emph{Castle Rock},
955
F.Supp. at 268 (citing \emph{Twin Peaks}, 996 F.2d at 1374 (``A comment is as
eligible
for fair use protection when it concerns `Masterpiece Theater'
and appears in the New York Review of Books as when it concerns
`As the World Turns' and appears in Soap Opera Digest.'')). In
other words, the fact that the subject matter of the quiz is plebeian, banal,
or ordinary stuff does not alter the fair use analysis. Criticism, comment,
scholarship, research, and other potential fair uses are no less protectable
because their subject is the ordinary.

Second, defendants style The SAT as a work ``decod[ing] the obsession with ...
and mystique that surround[s] `Seinfeld,'\,'' by ``critically
restructur[ing] [Seinfeld's mystique] into a system complete with varying
levels of `mastery' that relate the reader's control of the
show's trivia to knowledge of and identification with their hero, Jerry
Seinfeld.'' Citing one of their own experts for the proposition that ``[t]he
television environment cannot speak for itself but must be spoken for and
about,'' defendants argue that ``The SAT is a quintessential example of
critical text of the TV environment\ldots expos[ing] all of the show's
nothingness to articulate its true motive forces and its social and moral
dimensions.'' (Quotation marks omitted). Castle Rock dismisses these arguments
as post hoc rationalizations, claiming that had defendants been half as
creative in creating The SAT as were their lawyers in crafting these arguments
about transformation, defendants might have a colorable fair use claim.

Any transformative purpose possessed by The SAT is slight to non-existent. We
reject the argument that The SAT was created to educate Seinfeld viewers or to
criticize, ``expose,'' or otherwise comment upon Seinfeld. The SAT's purpose,
as evidenced definitively by the statements of the book's creators and by the
book itself, is to repackage Seinfeld to entertain Seinfeld viewers. The SAT's
back cover makes no mention of exposing Seinfeld to its readers, for example,
as a pitiably vacuous reflection of a puerile and pervasive television culture,
but rather urges SAT readers to ``open this book to satisfy [their]
between-episode [Seinfeld] cravings.'' Golub, The SAT's author, described the
trivia quiz book not as a commentary or a Seinfeld research tool, but as an
effort to ``capture Seinfeld's flavor in quiz book fashion.'' Finally, even
viewing The SAT in the light most favorable to defendants, we find scant reason
to conclude that this trivia quiz book seeks to educate, criticize, parody,
comment, report upon, or research Seinfeld, or otherwise serve a transformative
purpose. The book does not contain commentary or analysis
about Seinfeld, nor does it suggest how The SAT can be used to research
Seinfeld; rather, the book simply poses trivia questions. The SAT's plain
purpose, therefore, is not to expose Seinfeld's ``nothingness,'' but to satiate
Seinfeld fans' passion for the ``nothingness'' that Seinfeld has elevated into
the realm of protectable creative expression.

Although a secondary work need not necessarily transform the original work's
expression to have a transformative purpose, see, e.g., 4 Nimmer {\S}
13.05[D][2], at 13-227-13-228 (discussing reproduction of entire works in
judicial proceedings), the fact that The SAT so minimally alters Seinfeld's
original expression in this case is further evidence of The SAT's lack of
transformative purpose. To be sure, the act of testing trivia about a creative
work, in question and answer form, involves some creative expression. While
still minimal, it does require posing the questions and hiding the correct
answer among three or four incorrect ones. Also, dividing the
trivia questions into increasing levels of difficulty is somewhat more original
than arranging names in a telephone book in alphabetical order. \emph{See}
\emph{Feist}, 499
U.S. at 362-63, 111 S.Ct. 1282. The SAT's incorrect multiple choice answers are
also original. However, the work as a whole, drawn directly from the Seinfeld
episodes without substantial alteration, is far less transformative than other
works we have held not to constitute fair use. See, e.g., Twin Peaks, 996 F.2d
at 1378 (book about Twin Peaks television series that discusses show's
popularity, characters, actors, plots, creator, music, and poses trivia
questions about show held not to be fair use).\ldots

\readinghead{B. Nature of the Copyrighted Work}

The second statutory factor, ``the nature of the copyrighted work,'' 17 U.S.C.
{\S} 107(2), ``calls for recognition that some works are closer to the core of
intended copyright protection than others, with the consequence that fair use
is more difficult to establish when the former works are copied.''
\emph{Campbell},
510 U.S. at 586, 114 S.Ct. 1164. Defendants concede that the scope of fair use
is somewhat narrower with respect to fictional works, such as Seinfeld, than to
factual works. \emph{See} \emph{Stewart v. Abend}, 495 U.S. 207, 237, 110 S.Ct.
1750, 109
L.Ed.2d 184 (1990) (``In general, fair use is more likely to be found in
factual works than in fictional works''); \emph{Twin Peaks}, 996 F.2d at 1376 (second
factor ``favor[s]\ldots creative and fictional work''). Although this factor may
be of less (or even of no) importance when assessed in the context of certain
transformative uses, see, e.g., \emph{Campbell}, 510 U.S. at 586, 114 S.Ct. 1164
(creative nature of original ``Pretty Woman'' song ``not much help'' to fair
use analysis ``since parodies almost invariably copy\ldots expressive works''),
the fictional nature of the copyrighted work remains significant in the instant
case, where the secondary use is at best minimally transformative. Thus, the
second statutory factor favors the plaintiff.

\readinghead{C. Amount and Substantiality of the Portion Used in Relation to the
Copyrighted Work as a Whole}

As a preliminary matter, the district court held that its determination that The
SAT is substantially similar to Seinfeld ```should suffice for
a determination that the third fair use factor favors the plaintiff.'\,''
\emph{Castle Rock}, 955 F.Supp. at 269-70 (quoting \emph{Twin Peaks}, 996 F.2d
at 1377).
However, because secondary users need invoke the fair use defense only where
there is substantial similarity between the original and allegedly infringing
works, and thus actionable copying, the district court's analysis is of little
if any assistance. Under the district court's analysis, the third fair use
factor would always and unfairly favor the original copyright owner claiming no
fair use. \emph{See} 4 Nimmer {\S} 13.05[A], at 13-152 (``[F]air use is a
defense not
because of the absence of substantial similarity but rather despite the fact
that the similarity is substantial.'').

In \emph{Campbell}, a decision post-dating \emph{Twin Peaks}, the Supreme Court
clarified that
the third factor---the amount and substantiality of the portion of the
copyrighted work used---must be examined in context. The inquiry must focus
upon whether ``[t]he extent of\ldots copying'' is consistent with or more than
necessary to further ``the purpose and character of the use.'' 510 U.S. at
586-87, 114 S.Ct. 1164; see \emph{Sony Corp. of America v. Universal City
Studios, Inc.}, 464 U.S. 417, 449-50, 104 S.Ct. 774, 78 L.Ed.2d 574 (1984)
(reproduction
of entire work ``does not have its ordinary effect of militating against a
finding of fair use'' as to home videotaping of television programs);
\emph{Harper \&
Row}, 471 U.S. at 564, 105 S.Ct. 2218 (``[E]ven substantial quotations might
qualify as fair use in a review of a published work or a news account of a
speech'' but not in a scoop of a soon-to-be-published memoir.). ``[B]y
focussing [sic] on the amount and substantiality of the original work used by
the secondary user, we gain insight into the purpose and character of the use
as we consider whether the quantity of the material used was reasonable in
relation to the purpose of the copying.'' \emph{Texaco}, 60 F.3d at 926 (quotation
marks omitted). In \emph{Campbell}, for example, the Supreme Court determined
that a
``parody must be able to `conjure up' at least enough of [the]
original [work] to make the object of its critical wit recognizable'' and then
determined whether the amount used of the original work was ``no more than
necessary'' to satisfy the purpose of parody. 510 U.S. at 588-89, 114 S.Ct.
1164.

In the instant case, it could be argued that The SAT could not expose Seinfeld's
``nothingness'' without repeated, indeed exhaustive examples deconstructing
Seinfeld's humor, thereby emphasizing Seinfeld's meaninglessness to The SAT's
readers. That The SAT posed as many as 643 trivia questions to make this rather
straightforward point, however, suggests that The SAT's purpose was
entertainment, not commentary. Such an argument has not been advanced on
appeal, but if it had been, it would not disturb our conclusion that, under any
fair reading, The SAT does not serve a critical or otherwise transformative
purpose. Accordingly, the third factor weighs against fair use.

\readinghead{D. Effect of Use Upon Potential Market for or Value of Copyrighted
Work}

Defendants claim that the fourth factor favors their case for fair use because
Castle Rock has offered no proof of actual market harm to Seinfeld caused by
The SAT. To the contrary, Seinfeld's audience grew after publication of The
SAT, and Castle Rock has evidenced no interest in publishing Seinfeld trivia
quiz books and only minimal interest in publishing Seinfeld-related books.\ldots

In considering the fourth factor, our concern is not whether the secondary use
suppresses or even destroys the market for the original work or its potential
derivatives, but whether the secondary use usurps or substitutes for the market
of the original work. \emph{Id.} at 593, 114 S.Ct. 1164. The more transformative
the
secondary use, the less likelihood that the secondary use substitutes for the
original. \emph{Id.} at 591, 114 S.Ct. 1164. As noted by the district court,
``[b]y
the very nature of [transformative] endeavors, persons other than the copyright
holder are undoubtedly better equipped, and more likely, to fill these
particular market and intellectual niches.'' \emph{Castle Rock}, 955 F.Supp. at
271.
And yet the fair use, being transformative, might well harm, or even destroy,
the market for the original. \emph{See} \emph{Campbell}, 510 U.S. at 591-92, 114
S.Ct. 1164
(``[A] lethal parody, like a scathing theater review, kills demand for the
original, [but] does not produce a harm cognizable under the Copyright Act.'');
\emph{New Era Publications}, 904 F.2d at 160 (``a critical biography serves a
different function than does an authorized, favorable biography, and thus
injury to the potential market for the favorable biography by the publication
of the unfavorable biography does not affect application of factor four'').

Unlike parody, criticism, scholarship, news reporting, or other transformative
uses, The SAT substitutes for a derivative market that a television program
copyright owner such as Castle Rock ``would in general develop or license
others to develop.'' \emph{Campbell}, 510 U.S. at 592, 114 S.Ct.
1164. Because The SAT borrows exclusively from Seinfeld
and not from any other television or entertainment programs, The SAT is likely
to fill a market niche that Castle Rock would in general develop. Moreover, as
noted by the district court, this ``Seinfeld trivia game is not critical of the
program, nor does it parody the program; if anything, SAT pays homage to
Seinfeld.'' \emph{Castle Rock}, 955 F.Supp. at 271-72. Although Castle Rock has
evidenced little if any interest in exploiting this market for derivative works
based on Seinfeld, such as by creating and publishing Seinfeld trivia books (or
at least trivia books that endeavor to ``satisfy'' the ``between-episode
cravings'' of Seinfeld lovers), the copyright law must respect that creative
and economic choice. ``It would\ldots not serve the ends of the Copyright
Act---i.e., to advance the arts---if artists were denied their monopoly over
derivative versions of their creative works merely because they made the
artistic decision not to saturate those markets with variations of their
original.'' \emph{Castle Rock}, 955 F.Supp. at 272; see Salinger, 811 F.2d at 99
(``The need to assess the effect on the market for Salinger's letters is not
lessened by the fact that their author has disavowed any intention to publish
them during his lifetime.''). The fourth statutory factor therefore favors
Castle Rock.

\readinghead{E. Other Factors}

As we have noted, the four statutory fair use factors are non-exclusive and
serve only as a guide to promote the purposes underlying the copyright law. One
factor that is of no relevance to the fair use equation, however, is
defendants' continued distribution of The SAT after Castle Rock notified
defendants of its copyright infringement claim, because ``[i]f the use is
otherwise fair, then no permission need be sought or granted\ldots. [B]eing
denied
permission to use a work does not weigh against a finding of fair use.''
\emph{Campbell}, 510 U.S. at 585 n. 18, 114 S.Ct. 1164; see Wright, 953 F.2d at
737
(rejecting as irrelevant to fair use analysis argument that defendant failed to
get plaintiff's permission to create work).

We also note that free speech and public interest considerations are of little
relevance in this case, which concerns garden-variety infringement of creative
fictional works. See 4 Nimmer {\S} 13.05[B][4], at 13-205 (``The public
interest is also a factor that continually informs the fair use analysis.'');
\emph{cf.} \emph{Time Inc. v. Bernard Geis Assocs.}, 293 F.Supp. 130, 146
(S.D.N.Y.1968)
(discussing importance of access to information about President Kennedy
assassination in fair use analysis of home video of assassination).

\readinghead{F. Aggregate Assessment}

Considering all of the factors discussed above, we conclude that the copyright
law's objective ``[t]o promote the Progress of Science and useful Arts'' would
be undermined by permitting The SAT's copying of Seinfeld, \emph{see}
\emph{Arica}, 970 F.2d
at 1077, and we therefore reject defendants' fair use defense.\ldots

