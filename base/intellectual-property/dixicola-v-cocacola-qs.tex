\item Both copyright and patent award rights to people who create information.
Trademark law is different. After Defense Secretary Donald Rumsfeld used the
phrase ``shock and awe'' to describe the initial bombing campaign of the Iraq
War, the USPTO received dozens of trademark applications for the phrase SHOCK
AND AWE, for goods ranging from video games to fireworks to eyewear.
\textit{See} Sabra Chartrand,\textit{ Trademarking }{}`\textit{Shock and Awe'},
\textsc{N.Y. Times}, Apr. 21, 2003,
\url{http://www.nytimes.com/2003/04/21/technology/21PATE.html}. None of these
applicants created the phrase. What they hoped to create -- and what trademark
law protects -- is the goodwill in consumers' minds associating the phrase with
their particular goods. Any user of a mark who succeeds in creating those
associations obtains trademark rights against confusing uses of the mark. So
trademark rights flow from use, and they endure as long as the use continues.
Compare and contrast this basis of protection with the other bases we have
studied -- and keep it in mind. Rights based on use will be important in our
discussion of natural resources.


\item The basic species of trademark infringement is confusion about source: the
consumer buys the defendant's product rather than the plaintiff's in the
mistaken belief that it \textit{is} the plaintiff's product. This form of of
confusion, known as \textit{passing off}, is actionable even when the defendant
scrupulously avoids using the plaintiff's trademark but still manages to trick
consumers about what they are buying. (Scene: Restaurant, interior, day.
Patron: ``I'll have a Coke, please.'' Waiter: ``Here you go.'' \textit{Waiter
serves Patron a glass of MarBert the Distinctive Cola}.) There is a close
kinship between this theory of liability and the hypothetical, discussed in
\textit{Keeble v. Hickeringill}, of the schoolmaster who scares of his
competitor's students by shooting a gun in the air. Both are forms of
\textit{unfair competition}: stealing customers from a competitor with
dishonest or dangerous business practices.

But trademark law today is not limited to passing off. Another species of
trademark liability is \textit{dilution}, in which the defendant allegedly
harms the trademark itself. 
\begin{quotation}
First, there is concern that consumer search costs will rise if a trademark
becomes associated with a variety of unrelated products. Suppose an upscale
restaurant calls itself {\textquotedbl}Tiffany.{\textquotedbl} There is little
danger that the consuming public will think it's dealing with a branch of the
Tiffany jewelry store if it patronizes this restaurant. But when consumers next
see the name {\textquotedbl}Tiffany{\textquotedbl} they may think about both
the restaurant and the jewelry store, and if so the efficacy of the name as an
identifier of the store will be diminished. Consumers will have to think harder
--- incur as it were a higher imagination cost --- to recognize the name as the
name of the storeSo {\textquotedbl}blurring{\textquotedbl} is one form of
dilution.

Now suppose that the {\textquotedbl}restaurant{\textquotedbl} that adopts the
name {\textquotedbl}Tiffany{\textquotedbl} is actually a striptease joint.
Again, and indeed even more certainly than in the previous case, consumers will
not think the striptease joint under common ownership with the jewelry store.
But because of the inveterate tendency of the human mind to proceed by
association, every time they think of the word
{\textquotedbl}Tiffany{\textquotedbl} their image of the fancy jewelry store
will be tarnished by the association of the word with the strip joint. So
{\textquotedbl}tarnishment{\textquotedbl} is a second form of dilution.
Analytically it is a subset of blurring, since it reduces the distinctness of
the trademark as a signifier of the trademarked product or service.
\end{quotation}
Ty, Inc. v. Perryman, 306 F.3d 509, 511 (7th Cir. 2002). Are these forms of harm
against which trademark law ought to guard? Or are they the result of taking
the metaphor of a trademark as ``property'' too seriously?


\item Trademark law has a strong consumer-protection flavor. This gives it a
close connection to false advertising law, but note that trademark law targets
a particular species of misleading statements: those that involve misleading
uses of trademarks. Unsurprisingly, then, trademark law has defenses for people
who use trademarks in basically truthful ways. Comparative advertising, for
example, allows a defendant to use the plaintiff's trademark rather than
circumlocutions like ``the other leading brand'' in statements like ``Four out
of five taste-testers preferred Pepsi to Coke.'' And nominative fair use (not
to be confused with the completely different defense of fair use in copyright)
lets defendants describe their products in terms of the plaintiff's, as in
``The best Toyota repair shop in the Tri-State Area.'' It also affects
trademark law's exhaustion doctrine, because secondhand items bearing a
trademark may well be meaningfully different than new items precisely because
they are used. Would you buy a used golf ball? A used car? A used textbook?
Used underwear? It is legal to sell such items with the original trademarks --
provided their used status and their current condition are accurately
disclosed.

\heregraphic{trademark-traffix}

\item It is common for different kinds of intellectual property to apply to
different aspects of the same thing. Take an iPhone. Apple has utility patents
covering numerous aspects of how it works, such the slide-to-unlock feature and
shatterproof glass coatings. It has copyrights in the iOS software and icon
designs. It has trademarks on IPHONE and the Apple logo. It has design patents
in aspects of the iPhone's shape. Just as intellectual property rights can
overlap or interfere with other property rights, different kinds of
intellectual property rights can overlap or interfere. And just as first
sale/exhaustion polices some of these boundaries, other doctrines help keep
intellectual property systems from stepping on each other's toes. For example,
``functional'' material cannot be protected as a trademark. In \textit{TrafFix
Devices, Inc. v. Marketing Displays, Inc.}, 532 U.S. 23 (2001), MDI obtained a
patent on a ``dual-spring'' design for road signs that helped keep them from
tipping over on windy days. After the patent expired, a competitor, TrafFix,
introduced its own dual-spring road signs. MDI sued for \textit{trademark}
infringement, claiming that the dual-spring design was a source-indicating
feature, i.e., that consumers recognized that the dual springs indicated they
were buying a MDI sign. No dice, said the Supreme Court. Even if consumers did
associate the dual springs with MDI, giving it trademark rights in the design
would improperly extend its patent rights past the end of their proper term.
Competitors like TrafFix need to be free to copy any feature of the design that
is ``essential to the use or purpose of the article'' or that ``affects the
cost or quality of the article.'' A similar doctrine excludes functional
material from copyright protection. \textit{See} 17 U.S.C. {\S} 102(b) (``In no
case does copyright protection for an original work of authorship extend to any
idea, procedure, process, system, method of operation, concept, principle, or
discovery {\dots} .'').

