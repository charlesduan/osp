\expected{feist-v-rural}

\item Even on a sweat-of-the-brow theory, there is a decent argument that Rural
didn't have to sweat very much. But if originality rather than investment of
labor is the basis for copyright protection, then some who labor will not be
rewarded with a copyright. Take \textit{Jeweler's Circular Publishing Co.},
quoted in \textit{Feist}. The plaintiff published a 326-page directory of
jewelers, \textit{Trade-Marks of the Jewelry and Kindred Trades}. It obtained
the information in the directory at great effort, by writing to a large number
of jewelers. The defendant---according to the court, at least---skipped this
work by copying from the plaintiff's book rather than by doing its own
research. Presumably, after \textit{Feist}, there is no copyright in books like
\textit{Trade-Marks of the Jewelry and Kindred Trades}. Does this result make
sense? Without copyright, will telephone books and jewelers' directories cease
to exist because no one will invest in creating them?

