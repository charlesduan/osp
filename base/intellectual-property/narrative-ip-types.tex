\captionedgraphic{copyright-iceberg}{Left: Sarah Scurr.  Right: Marisol Ortiz
Elfeldt}

Federal \term{copyright} law protects ``original works of authorship,''
like novels, biographies, songs, screenplays, paintings, blueprints, and
sculptures. Copyright law has a very low threshold for protection: a work must
merely display a ``modicum of creativity'' and have been written down (``fixed
in a tangible medium of expression'').  The copyright so obtained is valid
during its author's lifetime, and for the next seventy years after that. It
gives copyright owners the exclusive right to reproduce their works, to make
adaptations of them, to distribute them to the public, and to perform or
display them publicly---but this right only applies against people who copy
from the owner. Someone who independently and coincidentally comes up with
similar expression is an author in her own right, not an infringer. In
Figure~\ref{f:copyright-iceberg}, for
example, are two photographs of the same iceberg, taken by different
photographers from nearby locations at almost exactly the same time. Neither
infringes on the other.

Federal \term{patent} law protects ``any new and useful process,
machine, manufacture, or composition of matter.'' Examples include mechanical
devices like tractor plows and can openers, chemical processes used to refine
oil, pharmaceutical products like anti-HIV drugs, and, a little infamously, a
``Method and apparatus for automatically exercising a curious animal'' by
encouraging it to chase a laser pointer. \textit{See} U.S. Pat. No. 6,701,872.
To obtain a patent, an inventor must go through a detailed and expensive
application process, which involves convincing the U.S. Patent and Trademark
Office (USPTO) that her invention is genuinely new (``novel''), that it
represents a sufficient advance on previous inventions (that it be
``nonobvious''), and that it has some practical use in the world, however
slight (``utility''). She must also disclose to the public, in detail, how her
invention works and how best to use it. Once the USPTO issues a patent, it
gives the owner the exclusive right for twenty years (from the date she filed
her application with the USPTO) to make, use, offer to sell, or sell the
invention. (This means that anyone is free to copy or to study the
\textit{patent} on a new kind of steering wheel, but they cannot make, use, or
sell \textit{steering wheels} as described in the patent.)

\begin{figure}
\begin{center}
\usegraphic[height=36pt]{nike-swoosh}
\usegraphic[height=72pt]{design-patent}
\end{center}
\caption{A few IP-protected things that you might know.}
\label{f:ip-designs}
\end{figure}


\term[trademark]{Trademark} law is a hybrid of state and federal rights. Its
basis for protection is a little different. A trademark is a word or symbol,
like NIKE or the ``swoosh'' logo in Figure~\ref{f:ip-designs}
that distinguishes goods or services in the
marketplace. One gains trademark rights by using a mark on goods so that
consumers associate the mark with a particular source---i.e., they know that
NIKE shoes come from one company (Nike) and not another (Adidas or Reebok).
These associations are called ``goodwill'' and it is common to say that what a
trademark owner owns is the goodwill (even though it exists only in consumers'
minds). These rights exist under state common law as soon as the goodwill
exists; trademark owners can also register their marks with the USPTO, which
gives nationwide and not just local rights. Trademark law gives a trademark
owner the right to prevent uses of the mark that cause ``consumer confusion''
about the source of goods: a consumer who sees non-Nike shoes falsely labeled
NIKE and who mistakenly believes they come from Nike has been confused about
the origin of the goods, and Nike can sue the company slapping its trademark on
ersatz shoes.

State-created \term[right of publicity]{rights of
publicity}\having{white-v-samsung}{, discussed in more detail previously,}{,
discussed in more detail below,}{} protect against the commercial use of one's
name, picture, voice, or other indicia  of identity without permission. For
example, photoshopping a celebrity's face onto a model wearing one of your
company's sweaters and using the photograph in an ad for those sweaters is
likely to trigger the right of publicity. Some states require that one's
identity have ``commercial value'' to bring a right of publicity suit, others
do not. (How would one build up commercial value in one's identity? It is
something one can do deliberately, or does it just happen to some people and
not others?) The federal trademark law, the Lanham Act, provides a closely
related cause of action for false claims about endorsement: quoting a person as
saying ``I always shop at Acme Hardware'' is actionable if the person didn't
say it and you don't have their permission to quote them as saying it.

\term[trade secrecy]{Trade secret} law was previously almost entirely
a matter of common law, but now almost all states have adopted a version of the
Uniform Trade Secrets Act, and the federal Defend Trade Secrets Act of 2016
substantially incorporates the UTSA's definitions. To be protected as
a trade secret, information must be valuable because it is secret. Canonical
examples of trade secrets include chain restaurants' secret sauces, customer
lists, business plans, manufacturing designs, information on the location of
valuable resources like shipwrecks and oil fields, and inventions in the
development stage before they are ready to be patented. (Because obtaining a
patent involves extensive disclosure, it is impossible to have a patent and a
trade secret on exactly the same information; one of the major strategic
decisions inventors must make when they apply for a patent is how much to
include in the application to obtain a stronger or broader patent, and how much
to try to hold back as a trade secret.) In general, a defendant is liable only
for obtaining a trade secret through ``improper means.'' Breach of a duty of
confidentiality is far and away the most common such means -- such as when
employees take company documents stamped ``CONFIDENTIAL'' with them to their
new jobs at a competitor. More colorfully, industrial espionage, such as
breaking into labs or hacking into computers, is also improper means. Note that
trade secret law, like copyright law, protects only against infringers who
obtain the secret information, directly or indirectly, from the owner:
independent rediscovery of the same information is a complete defense. So is
reverse engineering, in which a defendant takes publicly available information
(including legally obtained copies of the owner's goods containing or made
using with the trade secret) and studies it to understand how the secret works.

In addition to the patents discussed above (technically, ``utility
patents''), the federal government also issues \term[design patent]{design
patents} on ``any
new, original, and ornamental design for an article of
manufacture'' and \term[plant patent]{plant patents} for ``any distinct
and new variety of plant.'' Design patents have become big business,
particularly in the technology world where the shape of a device and its user
interface are crucial aspects in selling it to consumers. Apple, for example,
sued Samsung for infringing several design patents on elements of the
iPhone design in Figure~\ref{f:ip-designs}.

