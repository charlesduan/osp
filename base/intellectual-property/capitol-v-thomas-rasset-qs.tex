\item As \textit{Thomas\--Rasset} notes in passing, a victorious copyright owner
in an infringement suit is entitled to ``the actual damages suffered by him or
her as a result of the infringement, and any profits of the infringer that are
attributable to the infringement and are not taken into account in computing
the actual damages.'' (The plaintiffs here elected statutory damages instead.)
Does this remedy sound at all familiar? Where have we seen it before?

\item In light of \textit{Thomas-Rasset}, was the punitive damage award in
\textit{Jacque} unconstitutional? After all, the amount of punitive damages
there was determined entirely by a court with no legislative guidance
whatsoever.

\item Are the plaintiffs' actual damages in \textit{Thomas-Rasset} unproven, or
are they nonexistent? What \textit{are} actual damages in a case involving only
the copying of intangible information? Does it matter whether copyright
infringement is defined to require actual distribution of a copyrighted work or
merely making one available? Are the actual damages here any more or less real
than the \$1 in nominal damages suffered by Lois and Harvey Jacque?

\item Is this a case where an injunction is warranted to protect property
rights? Does it matter whether we call a copyright ``property'' or not? Does
Thomas-Rasset have a freedom-of-speech argument like the defendants in
\textit{Shack}? If an injunction is appropriate here, why was there no
injunction in \textit{Jacque}?

