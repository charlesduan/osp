\item Not all information is subject to intellectual property protection. As
\textit{Cheney Bros.} illustrates, court will not simply fashion a remedy for
the victim of copying by looking to a general right against imitation. Instead,
they will give recovery when the victim can point to some specific and
well-established body of intellectual property law giving rights over a
particular kind of information. One consequence of this attitude, illustrated
here, is that fashion designs are protected neither by federal statutory law or
copyright law nor by state common law. And yet the fashion industry exists, and
regularly produces new designs -- and lots of them. Does this mean that
intellectual property laws are unnecessary in general?


\item Indeed, part of the point of the case is the existence of federal
copyright law provides a reason for state law \textit{not} to apply. How would
the goals of the copyright laws be undermined if states were to supply
protection here? (The Copyright Act of 1976 is even more unequivocal on this
point; it explicitly preempts ``all legal or equitable rights that are
equivalent to any of the exclusive rights within the general scope of copyright
{\dots} in works of authorship that are fixed in a tangible medium of
expression and come within the subject matter of copyright.'' 17 U.S.C. {\S}
301(a).)


\item In recent years, Congress has considered a number of bills that would
extend copyright-like protections (albeit with shorter terms) to fashion
designs. Would these bills be good or bad for the fashion industry? For
clothing buyers? 

