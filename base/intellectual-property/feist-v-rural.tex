\reading{Feist Publications, Inc. v. Rural Telephone Service Co.}
\readingcite{499 U.S. 340 (1991)}

\opinion\textsc{Justice O'Connor} delivered the opinion of the Court.

This case requires us to clarify the extent of copyright protection available to
telephone directory white pages.

\readinghead{I}

Rural Telephone Service Company, Inc., is a certified public utility that
provides telephone service to several communities in northwest Kansas. It is
subject to a state regulation that requires all telephone companies operating
in Kansas to issue annually an updated telephone directory. Accordingly, as a
condition of its monopoly franchise, Rural publishes a typical telephone
directory, consisting of white pages and yellow pages. The white pages list in
alphabetical order the names of Rural's subscribers, together with their towns
and telephone numbers. The yellow pages list Rural's business subscribers
alphabetically by category and feature classified advertisements of various
sizes. Rural distributes its directory free of charge to its subscribers, but
earns revenue by selling yellow pages advertisements.

[Feist published a telephone directory, containing both white and yellow pages,
covering a much larger geographic area. It contained 46,878 white-pages
listings. Feist requested a license to Rural's listings; Rural refused.]

Unable to license Rural's white pages listings, Feist used them without Rural's
consent. Feist began by removing several thousand listings that fell outside
the geographic range of its area-wide directory, then hired personnel to
investigate the 4,935 that remained. These employees verified the data reported
by Rural and sought to obtain additional information. As a result, a typical
Feist listing includes the individual's street address; most of Rural's
listings do not. Notwithstanding these additions, however, 1,309 of the 46,878
listings in Feist's 1983 directory were identical to listings in Rural's
1982-1983 white pages. Four of these were fictitious listings that Rural had
inserted into its directory to detect copying.

Rural sued for copyright infringement in the District Court for the District of
Kansas taking the position that Feist, in compiling its own directory, could
not use the information contained in Rural's white pages. Rural asserted that
Feist's employees were obliged to travel door-to-door or conduct a telephone
survey to discover the same information for themselves. Feist responded that
such efforts were economically impractical and, in any event, unnecessary
because the information copied was beyond the scope of copyright protection.
The District Court granted summary judgment to Rural\ldots. In an
unpublished opinion, the Court of Appeals for the Tenth Circuit affirmed\dots.

\readinghead{II}

\readinghead[2]{A}

This case concerns the interaction of two well-established propositions. The
first is that facts are not copyrightable; the other, that compilations of
facts generally are. Each of these propositions possesses an impeccable
pedigree.\dots

The key to resolving the tension lies in understanding why facts are not
copyrightable. The \textit{sine qua non} of copyright is originality. To
qualify for copyright protection, a work must be original to the author.
Original, as the term is used in copyright, means only that the work was
independently created by the author (as opposed to copied from other works),
and that it possesses at least some minimal degree of creativity. 1 M. Nimmer
\& D. Nimmer, Copyright {\S}{\S} 2.01[A], [B] (1990) (hereinafter Nimmer). To
be sure, the requisite level of creativity is extremely low; even a slight
amount will suffice. The vast majority of works make the grade quite easily, as
they possess some creative spark, ``no matter how crude, humble or obvious'' it
might be. Id., {\S} 1.08[C][1]. Originality does not signify novelty; a work
may be original even though it closely resembles other works so long as the
similarity is fortuitous, not the result of copying. To illustrate, assume that
two poets, each ignorant of the other, compose identical poems. Neither work is
novel, yet both are original and, hence, copyrightable.\dots

Originality is a constitutional requirement. The source of Congress' power to
enact copyright laws is Article I, {\S} 8, cl. 8, of the Constitution, which
authorizes Congress to ``secur[e] for limited Times to Authors\ldots the
exclusive Right to their respective Writings.'' In two decisions from the late
19th century---\textit{The Trade-Mark Cases}, 100 U. S. 82 (1879); and
\textit{Burrow-Giles Lithographic Co. v. Sarony}, 111 U. S. 53 (1884)---this
Court defined the crucial terms ``authors'' and ``writings.'' In so doing, the
Court made it unmistakably clear that these terms presuppose a degree of
originality.\dots

It is this bedrock principle of copyright that mandates the law's seemingly
disparate treatment of facts and factual compilations. ``No one may claim
originality as to facts.'' Nimmer, {\S} 2.11[A], p. 2-157. This is because
facts do not owe their origin to an act of authorship. The distinction is one
between creation and discovery: The first person to find and report a
particular fact has not created the fact; he or she has merely discovered its
existence.\ldots

Factual compilations, on the other hand, may possess the requisite originality.
The compilation author typically chooses which facts to include, in what order
to place them, and how to arrange the collected data so that they may be used
effectively by readers. These choices as to selection and arrangement, so long
as they are made independently by the compiler and entail a minimal degree of
creativity, are sufficiently original that Congress may protect such
compilations through the copyright laws.\ldots

This inevitably means that the copyright in a factual compilation is thin.
Notwithstanding a valid copyright, a subsequent compiler remains free to use
the facts contained in another's publication to aid in preparing a competing
work, so long as the competing work does not feature the same selection and
arrangement.\ldots

\readinghead[2]{B}

As we have explained, originality is a constitutionally mandated prerequisite
for copyright protection. The Court's decisions announcing this rule predate
the Copyright Act of 1909, but ambiguous language in the 1909 Act caused some
lower courts temporarily to lose sight of this requirement.\ldots

Making matters worse, these courts developed a new theory to justify the
protection of factual compilations. Known alternatively as ``sweat of the
brow'' or ``industrious collection,'' the underlying notion was that copyright
was a reward for the hard work that went into compiling facts. The classic
formulation of the doctrine appeared in \textit{Jeweler's Circular Publishing
Co}., 281 F., at 88:
\begin{quote}
``The right to copyright a book upon which one has expended labor in its
preparation does not depend upon whether the materials which he has collected
consist or not of matters which are publici juris, or whether such materials
show literary skill \textit{or originality}, either in thought or in language,
or anything more than industrious collection. The man who goes through the
streets of a town and puts down the names of each of the inhabitants, with
their occupations and their street number, acquires material of which he is the
author'' (emphasis added).
\end{quote}
\dots Without a doubt, the ``sweat of the brow'' doctrine flouted basic
copyright principles. Throughout history, copyright law has ``recognize[d] a
greater need to disseminate factual works than works of fiction or fantasy.''
\textit{Harper \& Row}, 471 U. S., at 563. But ``sweat of the brow'' courts
took a contrary view; they handed out proprietary interests in facts and
declared that authors are absolutely precluded from saving time and effort by
relying upon the facts contained in prior works.\ldots

\readinghead[2]{C}

\dots In enacting the Copyright Act of 1976, Congress dropped the reference to
``all the writings of an author'' and replaced it with the phrase ``original
works of authorship.'' 17 U. S. C. {\S} 102(a).\ldots

As discussed earlier, however, the originality requirement [for compilations] is
not particularly stringent. A compiler may settle upon a selection or
arrangement that others have used; novelty is not required. Originality
requires only that the author make the selection or arrangement independently
(i.e., without copying that selection or arrangement from another work), and
that it display some minimal level of creativity. Presumably, the vast majority
of compilations will pass this test, but not all will. There remains a narrow
category of works in which the creative spark is utterly lacking or so trivial
as to be virtually nonexistent. Such works are incapable of sustaining a valid
copyright.\ldots

In summary, the 1976 revisions to the Copyright Act leave no doubt that
originality, not ``sweat of the brow,'' is the touchstone of copyright
protection in directories and other fact-based works.\ldots The revisions
explain with painstaking clarity that copyright requires originality, {\S}
102(a); that facts are never original, {\S} 102(b); that the copyright in a
compilation does not extend to the facts it contains, {\S} 103(b); and that a
compilation is copyrightable only to the extent that it features an original
selection, coordination, or arrangement, {\S} 101.\ldots

\readinghead{III}

\ldots The selection, coordination, and arrangement of Rural's white pages do
not
satisfy the minimum constitutional standards for copyright protection. As
mentioned at the outset, Rural's white pages are entirely typical. Persons
desiring' telephone service in Rural's service area fill out an application and
Rural issues them a telephone number. In preparing its white pages, Rural
simply takes the data provided by its subscribers and lists it alphabetically
by surname. The end product is a garden-variety white pages directory, devoid
of even the slightest trace of creativity.

Rural's selection of listings could not be more obvious: It publishes the most
basic information---name, town, and telephone number---about each person who
applies to it for telephone service. This is ``selection'' of a sort, but it
lacks the modicum of creativity necessary to transform mere selection into
copyrightable expression. Rural expended sufficient effort to make the white
pages directory useful, but insufficient creativity to make it original.

We note in passing that the selection featured in Rural's white pages may also
fail the originality requirement for another reason. Feist points out that
Rural did not truly ``select'' to publish the names and telephone numbers of
its subscribers; rather, it was required to do so by the Kansas Corporation
Commission as part of its monopoly franchise. Accordingly, one could plausibly
conclude that this selection was dictated by state law, not by Rural.

Nor can Rural claim originality in its coordination and arrangement of facts.
The white pages do nothing more than list Rural's subscribers in alphabetical
order. This arrangement may, technically speaking, owe its origin to Rural; no
one disputes that Rural undertook the task of alphabetizing the names itself.
But there is nothing remotely creative about arranging names alphabetically in
a white pages directory. It is an age-old practice, firmly rooted in tradition
and so commonplace that it has come to be expected as a matter of course. It is
not only unoriginal, it is practically inevitable. This time-honored tradition
does not possess the minimal creative spark required by the Copyright Act and
the Constitution.\ldots

Because Rural's white pages lack the requisite originality, Feist's use of the
listings cannot constitute infringement. This decision should not be construed
as demeaning Rural's efforts in compiling its directory, but rather as making
clear that copyright rewards originality, not effort. As this Court noted more
than a century ago, ``\,`great praise may be due to the
plaintiffs for their industry and enterprise in publishing this paper, yet the
law does not contemplate their being rewarded in this way.'\,'' \textit{Baker
v. Selden}, 101 U. S., at 105.

