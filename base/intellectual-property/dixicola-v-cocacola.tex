\reading{Dixi-Cola Laboratories v. Coca-Cola Co.}
\readingcite{117 F. 2d 352 (4th Cir. 1941)}

\opinion \textsc{Soper}, Circuit Judge:

\ldots
The plaintiff is the owner of the trademark ``Coca-Cola'' for a syrup to be used
with carbonated water as a beverage. The defendants make and sell a concentrate
and a syrup to be used in the production of a similar beverage under the names,
MarBert the Distinctive Cola and Dixi-Cola. The defendants do not use the word
``coca''; but they claim the right to use the word ``cola'' in the combinations
mentioned. The evidence shows that they have also used other terms, such as
Apola Cola and Lola-Kola, but as to them they now make no defense. The
plaintiff concedes that the names Dixi-Cola and MarBert the Distinctive Cola
are not so similar to the name Coca-Cola, that a purchaser of the beverage
known as Dixi-Cola or MarBert the Distinctive Cola would be led to believe that
he was buying the beverage Coca-Cola, but the plaintiff nevertheless charges
infringement on the ground that the use of the word ``cola'' in defendants'
trade-marks or trade-names leads the public to believe that their products
originate with the plaintiff. {\dots}

The broad claim of the plaintiff to the exclusive use of the word ``cola'' in a
trade-mark or trade-name is based upon the contention that Coca-Cola is a
technical common-law trade-mark, adopted as a fanciful and arbitrary word by
the first producer of the beverage in 1886. The plaintiff also relies on five
registrations of the mark in the United States Patent Office, one under the Act
of March 3, 1881, 21 Stat. 502, and four under the Act of February 20, 1905, 33
Stat. 724, 15 U.S.C.A. {\S} 81 et seq. {\dots}

It is certainly beyond dispute that the word ``Coca-Cola'' is the exclusive
property of the Coca-Cola Company. The evidence in the pending case shows that
what was said of the name in \textit{Coca-Cola Co. v. Koke Co.}, 254 U.S. 143
[(1920)], and \textit{Coca-Cola Co. v. Old Dominion Beverage Corp.}, [271 F.
600 (4th Cir. 1921)], is equally true today. There has been no let-up in the
popular demand for the drink or in the extent of its advertising. On the
contrary, both have greatly increased. In 1920 the gallons of syrup sold were
18,656,445 and the advertising expense \$2,330,710.40, while in 1938 the
gallons sold were 48,508,414, and the advertising expense \$7,122,863.31. No
one else can lawfully use the word ``Coca-Cola'' for a trade-mark, even though
it originally may have been a descriptive name.

The plaintiff, however, is not content with this measure of protection. It
insists in addition that no one shall use the word ``cola'' in a trade-mark,
even in connection with a prefix that prevents all confusion with the name
Coca-Cola. The reason given is that the word is so closely associated with
Coca-Cola in the public mind that any drink, bearing the word as part of its
name, will be thought to proceed from the same source. Forty-one witnesses from
Baltimore, Springfield and Birmingham testified that when they saw goods
labeled by a name containing the suffix ``cola,'' they were led to believe, not
that the goods were Coca-Cola, but that they originated with the Coca-Cola
Company. Hence, it is said, the defendants have appropriated the result of the
plaintiff's efforts and expenditures, and imperiled the reputation of the
Coca-Cola Company and its product.

Confusion of origin, as well as confusion of goods, from the use of the same
trade-mark, may constitute infringement, especially when the name has a
fanciful and arbitrary character. We must, therefore, consider the defense now
set up to this phase of the plaintiff's case that the word ``cola'' is a
descriptive and generic term, open to all the world, which may be lawfully used
as part of a trademark by competitors so long as the whole trade-mark is not
confusingly similar to Coca-Cola.

There are many cases which hold that it is not infringement for a trader to use
as part of his trade-name to designate his product a descriptive or generic
word which has already been adopted by another, provided that the competing
marks, taken as a whole, are clearly distinguishable. Thus
{\textquotedbl}Sal-Vet{\textquotedbl} was held not infringed by
{\textquotedbl}Sal Tone{\textquotedbl}, since the word
{\textquotedbl}Sal{\textquotedbl}, meaning salt, was descriptive of the
principal ingredient of both products, and no ordinary purchaser would confuse
one of the complete names with the other. \textit{S. R. Feil Co. v. John E.
Robins Co}. [220 F .650 (7th Cir. 1915)].

In the light of these decisions, it is important to inquire whether or not the
word ``cola'' has a descriptive significance apart from its use in the
trade-mark Coca-Cola, and has become a generic term, generally used to indicate
a class of beverage. The answer is to be found, we believe, in scientific and
popular literature, in the discussions of Coca-Cola cases by the courts, and
the attitude of the Coca-Cola Company itself in the conduct of its business.
The beverage was devised and the name Coca-Cola was adopted by John S.
Pemberton in Atlanta in 1886. The product was sold under a label registered in
the Patent Office, which advertised Coca-Cola syrup as an extract for
carbonated beverages possessing a peculiar flavor and the tonic and nerve
stimulant qualities of the coca plant and cola nuts. Both of these substances
were well known at that time. The word ``cola'' was recognized as the name of a
tree native to Africa, which bears the small brown ``cola nut'' that was
introduced in England in 1865 and later in the United States. Prior to 1886 the
stimulant qualities of the cola nut were frequently referred to in
pharmaceutical and scientific publications and periodicals, and it was
suggested that it could be used to make a beverage that would successfully
compete with tea and coffee as a refreshing and invigorating drink.

These facts led to the contention in the court below that at best the word
Coca-Cola, taken as a whole, is a descriptive name, entitled to protection only
because it has acquired a secondary significance. But the contention was
rejected. It was said that while relatively small amounts of coca and cola
extracts are found in the drink, the basic ingredients are sugar, phosphoric
acid and a small amount of caffeine; and also that the words comprising the
mark were so little known to the general public when adopted that they did not
suggest at that time that the beverage was made from coca leaves or cola nuts.
Hence it was decided that the name is not clearly descriptive of the product,
but should be considered a coined word with all the characteristics of a
technical trade mark. Other courts have reached the same conclusion. [citing
cases]. {\dots}

The decision of the Supreme Court in \textit{Coca-Cola Co. v. Koke Co.} has
already been mentioned. The court rejected the charge that the right to
protection against infringement because of misrepresentations implied by the
name that the product contained cocaine, which had formerly been used in small
amounts, but had been eliminated after the passage of the Food and Drug Act.
The court said, ``We are dealing here with a popular drink not with a medicine,
and although what has been said might suggest that its attraction lay in
producing the expectation of a toxic effect the facts point to a different
conclusion. Since 1900 the sales have increased at a very great rate
corresponding to a like increase in advertising. The name now characterizes a
beverage to be had at almost any soda fountain. It means a single thing coming
from a single source, and well known to the community. It hardly would be too
much to say that the drink characterizes the name as much as the name the
drink. In other words `Coca-Cola' probably means to most persons the
plaintiff's familiar product to be had everywhere rather than a compound of
particular substances. {\dots} The coca leaves and whatever of cola nut is
employed may be used to justify the continuance of the name or they may affect
the flavor as the plaintiff contends, but before this suit was brought the
plaintiff had advertised to the public that it must not expect and would not
find cocaine, and had eliminated every thing tending to suggest cocaine effects
except the name and the picture of the leaves and nuts, which probably conveyed
little or nothing to most who saw it. It appears to us that it would be going
too far to deny the plaintiff relief against a palpable fraud because possibly
here and there an ignorant person might call for the drink with the hope for
incipient cocaine intoxication. The plaintiff's position must be judged by the
facts as they were when the suit was begun, not by the facts of a different
condition and an earlier time.'' {\dots}

It must be concluded, we think, from this history that the word ``Coca-Cola,''
taken as a whole, is in some sense descriptive of the drink which it
designates. It is true that the name identifies the goods of the plaintiff, but
it has also come to characterize them. This process has been hastened by the
fact that the combination of extract of coca leaves and extract of cola nuts
employed by Pemberton was new, and it gave to the product a new and distinctive
flavor for which there was no other name than that which he employed. Hence the
drink came to be known to the public by this name in much the same fashion as
other soft drinks are named for a small quantity of flavoring ingredient rather
than the large quantities of sugar and water that mainly compose them. The
process was further stimulated by the great public response to the drink and
the activities of numerous competitors who speedily entered the field and were
enabled lawfully to make the same or a similar beverage, since Coca-Cola was
not covered by a patent.

The record is replete with references to the number of competing drinks in this
class. The District Judge in his opinion said that ``since Coca-Cola appeared,
there has been a veritable flood of drinks of this type, as evidenced by the
fact that there have been no less than 143 registrations in the United States
Patent Office of names embodying the word `cola' as a suffix.'' In 1907 the
Supreme Court of Mississippi, in the case of \textit{Coca-Cola Co. v.
Skillman}, 91 Miss. 677, 44 So. 985, discussed a statute imposing a privilege
tax on Coca-Cola, Celery-Cola, Afri-Cola, Hecks Cola, Cola-Beta, Colavin,
Nervola, and Nervocola, or any similar or proprietary drinks. Some cola drinks
have had a long and continuous history. Thus the record shows that Lime-Cola
has been made for more than twenty years in the United States and that
Pepsi-Cola has been in existence as a beverage for more than thirty-five years.

The adoption of the word ``cola{\textquotedbl} to characterize a class of drinks
thus came about very naturally, to some extent with the consent of the
Coca-Cola Company, as we shall see, and to a greater extent because in the
course of events it could not be prevented. It was attended by a vast increase
after 1886 in the literature relating to the cola nut and its uses.
Publications of various types recognized the fact that it could be used as an
ingredient of a soft drink. Numerous references to the cola nut and to cola
syrup and extract and their use in beverages, called cola drinks, appeared
throughout the following years in dictionaries, encyclopedias, pharmaceutical
magazines, trade journals and government publications. During the same period
the word was adopted as part of the trade name of a large number of competing
beverages. The result is that today the phrase ``cola drinks'' indicates to the
general public beverages which in taste and appearance resemble Coca-Cola.
{\dots}

It must not be supposed that the Coca-Cola Company has not fought vigorously to
protect its valuable right from invasion. Suits against competitors have
averaged one a week during the last thirty years. Many of these competitors
have been guilty of fraud and unfair competition, and all of them, it is safe
to say, have sought to participate in the profits which experience had shown
could be derived from making a drink like Coca-Cola. {\dots}

None of these reported decisions goes further than the decision of this court in
\textit{Coca-Cola Co. v. Old Dominion Beverage Corp}.,  involving the use of
the name ``Taka Cola,'' and the unreported decision of the present writer in
\textit{Coca-Cola Co. v. Philips Bros}. in the District Court of Maryland,
involving the word ``Champion-Cola.'' In both cases the names were regarded as
so close to the name ``Coca-Cola'' as to be likely to result in the confusion
of the goods. In both there was unfair competition in the simulation of the
color scheme, of the script of the Coca-Cola Company, or in the confusing
display of the competing name. {\dots}

No reported case has come to our attention which distinctly holds that the word
``cola'' cannot be used as part of a name of a beverage provided that the whole
name is not confusingly similar to Coca-Cola. It is urged, however, that we
should make such a decision in this case for the reasons, which found favor in
the District Court, that no such thing as a cola beverage in the present sense
of the term, was known or spoken of prior to the advent of Coca-Cola in 1886,
and that the Coca-Cola Company has always asserted its claim to the exclusive
use of the term. In our opinion, these considerations, even if sustained by the
evidence, are not controlling in the face of the fact that the word ``cola''
does not today indicate the plaintiff's product but a class of drinks to which
the goods of the defendants and many other competitors belong. The applicable
rule, supported by authority, is thus stated in the Restatement of Torts:
``{\S} 735. (1) A designation which is initially a trade-mark or trade name
ceases to be such when it comes to be generally understood as a generic or
descriptive designation for the type of goods, services or business in
connection with which it is used.'' Comment (a) to the foregoing sub-section
(1) reads as follows: ``Significance of change in meaning. When one has a
monopoly of the initial distribution of a specific article over a period of
time, and especially if the descriptive name for the article is one difficult
to pronounce or remember, there is a likelihood that the designation which he
adopts as his trade-mark for the article will be incorporated into the language
as the usual generic designation for an article of that type. When that
happens, the designation becomes merely descriptive of the goods and no longer
identifies a particular brand or performs any of the functions of a trade-mark
or trade name. Moreover, the designation must then be used by others if there
is to be any effective competition in the sale of the goods. It is immaterial
that the person first adopting the designation made every reasonable effort to
avoid this result or that the designation was coined by him and derived meaning
only from his use. The designation may be used by others, subject to the
limitations of Sub-Section (2) and of Sec. 712 relating to fraudulent
marketing.''

We are, however, in accord with the conclusion of the District Court that the
conduct of the defendants has been such as to justify a decree restricting
their business activities in the future along certain lines. The evidence amply
justifies the finding that distributors of their products in New England, New
York and St. Louis, and to a less extent in Baltimore, have attempted to sell
and have sold their syrup to customers, engaged in the fountain trade, with the
understanding that the drink made therefrom should be sold as and for
Coca-Cola. The officers of the defendant corporation had knowledge of these
activities and participated therein. The sale of syrup to the fountain trade
constituted about ten per cent of the total business of the defendant.

The evidence also justifies the finding that the bottled beverage made by
bottlers from defendant's concentrate was passed off as Coca-Cola in various
bars and taverns. It is difficult to ascertain how wide-spread this practice
has been, but there is some evidence that an officer of the corporation
encouraged the practice. The defendants were also fully aware of the use of the
infringing word ``Lola-Kola''{\textquotedbl} by bottlers, and indeed agreed to
place this word on all packages of its concentrate sold to Lola Bottlers, Inc.
Under these circumstances, it is a reasonable conclusion that the defendants
have conspired with their customers to palm off their goods for those of the
Coca-Cola Company whenever it was safe to do so.

The remedy for these illegal acts, which appears in the decree, is the issuance
of an injunction against the defendants enjoining them from committing any acts
calculated to cause any product other than the plaintiff's to be known or sold
as ``Coca-Cola'' or ``Koke,'' or any colorable imitation thereof. The
defendants are also enjoined ``(f) From giving to any part of their merchandise
not sold by defendants, their agents or distributors, in bottles to consumers,
a color imitating or resembling the color of plaintiff's product, if or when
defendants know, or in the exercise of reasonable care should know, that the
purchaser thereof intends to dispense such merchandise to the consumer other
than in bottles, or intends to bottle the beverage made from such product and
to use on the bottles, labels or caps some extrinsic, deceiving element that in
conjunction with the color imitating plaintiff's color enables such purchaser
to pass off his, her or their product for plaintiff's product.'' {\dots}

