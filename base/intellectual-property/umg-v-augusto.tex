\reading[UMG Recordings, Inc. v. Augusto]{UMG Recordings, Inc. v. Augusto}
\readingcite{628 F. 3d 1175 (9th Cir. 2011)}

\textsc{CANBY}, Circuit Judge:

The material facts of the case are undisputed. UMG is among the world's largest
music companies. One of its core businesses is the creation, manufacture, and
sale of recorded music, or phonorecords, the copyrights of which are owned by
UMG. These phonorecords generally take the form of compact discs (``CDs'').

Like many music companies, UMG ships specially-produced promotional CDs to a
large group of individuals (``recipients''), such as music critics and radio
programmers, that it has selected. There is no prior agreement or request by
the recipients to receive the CDs. UMG does not seek or receive payment for the
CDs, the content and design of which often differs from that of their
commercial counterparts. UMG ships the promotional CDs by means of the United
States Postal Service and United Parcel Service. Relatively few of the
recipients refuse delivery of the CDs or return them to UMG, and UMG destroys
those that are returned.

Most of the promotional CDs in issue in this case bore a statement (the
``promotional statement'') similar to the following:

\begin{quote}
This CD is the property of the record company and is licensed to the intended
recipient for personal use only. Acceptance of this CD shall constitute an
agreement to comply with the terms of the license. Resale or transfer of
possession is not allowed and may be punishable under federal and state laws.
\end{quote}

Some of the CDs bore a more succinct statement, such as ``Promotional Use
Only---Not for Sale.''

Augusto was not among the select group of individuals slated to receive the
promotional CDs. He nevertheless managed to acquire numerous such CDs, many of
which he sold through online auctions at eBay.com. Augusto regularly advertised
the CDs as ``rare ... industry editions'' and referred to them as ``Promo
CDs.''

After several unsuccessful attempts at halting the auctions through eBay's
dispute resolution program, UMG filed a complaint against Augusto in the United
States District Court for the Central District of California, alleging that
Augusto had infringed UMG's copyrights in eight promotional CDs for which it
retained the ``exclusive right to distribute.'' The district court granted
summary judgment in favor of Augusto, and UMG appealed. . {\dots}

Although UMG, as the owner of the copyright, has exclusive rights in the
promotional CDs, ``[e]xemptions, compulsory licenses, and defenses found in the
Copyright Act narrow [those] rights.'' \textit{Wall Data Inc. v. Los Angeles
Cnty. Sheriff's Dept}., 447 F.3d 769, 777 (9th Cir. 2006). Augusto invokes the
``first sale'' doctrine embodied in {\S} 109(a) of the Act. 17 U.S.C. {\S}
109(a). He argues that the circumstances attending UMG's distribution of the
discs effected a ``sale'' (transfer of ownership) of the discs to the original
recipients and that, under the ``first sale'' doctrine, the recipients and
subsequent owners of those particular copies were permitted to sell or
otherwise dispose of those copies without authorization by the copyright
holder. {\dots}

UMG, on the other hand, contends that the promotional statement effected a
license with the recipients and, because the recipients were not owners but
licensees of the CDs, neither they nor Augusto were entitled to sell or
otherwise transfer the CDs.

The first sale doctrine provides that ``the owner of a particular copy or
phonorecord lawfully made under [the Act], or any person authorized by such
owner, is entitled, without the authority of the copyright owner, to sell or
otherwise dispose of the possession of that copy or phonorecord.'' 17 U.S.C.
{\S} 109(a). Notwithstanding its distinctive name, the doctrine applies not
only when a copy is first sold, but when a copy is given away or title is
otherwise transferred without the accouterments of a sale. The seminal
illustration of the principle is found in \textit{Bobbs-Merrill Co. v. Straus},
210 U.S. 339 (1908), where a copyright owner unsuccessfully attempted to
restrain the resale of a copyrighted book by including in it the following
notice: ``The price of this book at retail is \$1 net. No dealer is licensed to
sell it at a less price, and a sale at less price will be treated as an
infringement of the copyright.'' The Court noted that the statutory grant to a
copyright owner of the ``sole right of vending'' the work did not continue
after the first sale of a given copy.. ``The purchaser of a book, once sold by
authority of the owner of the copyright, may sell it again, although he could
not publish a new edition of it.'' The attempt to limit resale below a certain
price was therefore held invalid.

The rule of \textit{Bobbs-Merrill} remains in full force, enshrined as it is in
{\S} 109(a) of the Act: a copyright owner who transfers title in a particular
copy to a purchaser or donee cannot prevent resale of that particular copy. We
have recognized, however, that not every transfer of possession of a copy
transfers title. Particularly with regard to computer software, we have
recognized that copyright owners may create licensing arrangements so that
users acquire only a license to use the particular copy of software and do not
acquire title that permits further transfer or sale of that copy without the
permission of the copyright owner.  {\dots}

The same question is presented here. Did UMG succeed in creating a license in
recipients of its promotional CDs, or did it convey title despite the
restrictive labeling on the CDs? We conclude that, under all the circumstances
of the CDs' distribution, the recipients were entitled to use or dispose of
them in any manner they saw fit, and UMG did not enter a license agreement for
the CDs with the recipients. Accordingly, UMG transferred title to the
particular copies of its promotional CDs and cannot maintain an infringement
action against Augusto for his subsequent sale of those copies.

Our conclusion that the recipients acquired ownership of the CDs is based
largely on the nature of UMG's distribution. First, the promotional CDs are
dispatched to the recipients without any prior arrangement as to those
particular copies. The CDs are not numbered, and no attempt is made to keep
track of where particular copies are or what use is made of them. As explained
in greater detail below, although UMG places written restrictions in the labels
of the CDs, it has not established that the restrictions on the CDs create a
license agreement.

We also hold that, because the CDs were unordered merchandise, the recipients
were free to dispose of them as they saw fit under the Unordered Merchandise
Statute, 39 U.S.C. {\S} 3009, which provides in pertinent part that,

\ \ (a)\ \ [e]xcept for ... free samples clearly and conspicuously marked as
such,... the mailing of unordered merchandise... constitutes an unfair method
of competition and an unfair trade practice....

\ \ (b)\ \ Any merchandise mailed in violation of subsection (a) of this section
... may be treated as a gift by the recipient, who shall have the right to
retain, use, discard, or dispose of it in any manner he sees fit without any
obligation whatsoever to the sender. ...

\textit{Id}. {\S} 3009(a), (b) (emphasis added). The statute defines ``unordered
merchandise'' as ``merchandise mailed without the prior expressed request or
consent of the recipient'' but leaves ``merchandise'' itself undefined. Id.
{\S} 3009(d). Although the statute applies in terms to ``mailed'' merchandise,
the Federal Trade Commission has applied its prohibitions to other types of
shipment, as violations of {\S} 5 of the Federal Trade Commission Act, 15
U.S.C. {\S} 45. See 43 Fed.Reg. 4113 (Jan. 31, 1978). {\dots}

There are additional reasons for concluding that UMG's distribution of the CDs
did not involve a consensual licensing operation. Some of the statements on the
CDs and UMG's purported method of securing agreement to licenses militate
against a conclusion that any licenses were created. The sparest promotional
statement, ``Promotional Use Only---Not for Sale,'' does not even purport to
create a license. But even the more detailed statement is flawed in the manner
in which it purports to secure agreement from the recipient. The more detailed
statement provides:

\begin{quote}
This CD is the property of the record company and is licensed to the intended
recipient for personal use only. Acceptance of this CD shall constitute an
agreement to comply with the terms of the license. Resale or transfer of
possession is not allowed and may be punishable under federal and state laws.
\end{quote}

It is one thing to say, as the statement does, that ``acceptance'' of the CD
constitutes an agreement to a license and its restrictions, but it is quite
another to maintain that ``acceptance'' may be assumed when the recipient makes
no response at all. This record reflects no responses. Even when the evidence
is viewed in the light most favorable to UMG, it does not show that any
recipients agreed to enter into a license agreement with UMG when they received
the CDs.

Because the record here is devoid of any indication that the recipients agreed
to a license, there is no evidence to support a conclusion that licenses were
established under the terms of the promotional statement. Accordingly, we
conclude that UMG's transfer of possession to the recipients, without
meaningful control or even knowledge of the status of the CDs after shipment,
accomplished a transfer of title. {\dots}

Because we conclude that UMG's method of distribution transferred the ownership
of the copies to the recipients, we have no need to parse the remaining
provisions in UMG's purported licensing statement; UMG dispatched the CDs in a
manner that permitted their receipt and retention by the recipients without the
recipients accepting the terms of the promotional statements. UMG's transfer of
unlimited possession in the circumstances present here effected a gift or sale
within the meaning of the first sale doctrine, as the district court held.
{\dots}

