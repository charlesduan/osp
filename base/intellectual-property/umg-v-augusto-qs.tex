\item The first sale doctrine draws a sharp distinction between ownership of a
\textit{work} (which remains in UMG) and ownership of a \textit{copy} of that
work (which passes to the recipients, thence to Augusto, thence to his eBay
buyers). Every major body of intellectual property law has a similar doctrine:
the intellectual property rights holder's rights over a particular item end
when it gives up its personal property rights in the item. Because of this, it
depends heavily on the law of personal property to define who is an ``owner''
of that item.
\item Do you see why libraries would be impossible without first sale? The
market for used books, records, and DVDs depends on it. Publishers, for their
part, can be more ambivalent about it. In the summer of 2014, the law school
casebook publisher WoltersKluwer anounced a program called ``Connected
Casebook'' under which students would buy discounted access to online versions
of their casebooks. Participating students would receive a paper copy of the
casebook for use in the class, but would be required to return the book at the
end of the semester. What effects would this have on the used-book market? On
students' first sale rights?
