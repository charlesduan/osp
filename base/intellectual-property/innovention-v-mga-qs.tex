\item Copyright law requires only independent creation, a modicum of creativity,
and fixation in a tangible medium of expression. These are all easy thresholds
to meet: scribble on a napkin and you're done. But patent law has much more
rigorous screening doctrines. 

\textit{Novelty}.\textit{ }An invention must be \textit{novel} in an absolute
sense. A patent will not be granted if ``the claimed invention was patented,
described in a printed publication, or in public use, on sale, or otherwise
available to the public.'' 35 U.S.C. {\S} 102.

\textit{Nonobviousness}. An invention must not be ``obvious {\dots} to a person
having ordinary skill in the art to which the claimed invention pertains.''
\textit{Id. }{\S} 103. This means that only significant improvements on the
prior art are patentable, and not minor advances. As Justice Kennedy explained
for the Supreme Court in \textit{KSR Intern. Co. v. Teleflex, Inc.}, 127 S. Ct.
1727, 1746 (2007):
\begin{quote}
We build and create by bringing to the tangible and palpable reality around us
new works based on instinct, simple logic, ordinary inferences, extraordinary
ideas, and sometimes even genius. These advances, once part of our shared
knowledge, define a new threshold from which innovation starts once more. And
as progress beginning from higher levels of achievement is expected in the
normal course, the results of ordinary innovation are not the subject of
exclusive rights under the patent laws. Were it otherwise patents might stifle,
rather than promote, the progress of useful arts.
\end{quote}

\textit{Reduction to Practice and Enablement}. It is not enough to have the idea
for an invention (``conceive'' of it) or even to write it down. An inventor
must also ``reduce the invention to practice'' by establishing that it works
for its intended purpose. What is more, in her patent application -- which will
be published if and when she is issued a patent (or sooner, under some
circumstances) -- she must ``enable'' the invention by providing ``a written
description of the invention, and of the manner and process of making and using
it, in such full, clear, concise, and exact terms as to enable any person
skilled in the art to which it pertains, or with which it is most nearly
connected, to make and use the same.'' 35 U.S.C. {\S} 112(a). (Notice the
``person skilled in the art'' is back for a return appearance; she is the
hypothetical reasonable person of patent law.)

\item Until 2011, the United States patent law worked on a ``first to invent''
basis. If more than one applicant came to the USPTO with an application for the
same invention, the patent would be awarded to the one who could prove the
earliest date of conception for the invention (provided that she was reasonably
diligent about reducing it to practice and had not abandoned it), regardless of
who filed first. But in 2011, to bring the U.S. closer to patent systems in the
rest of the world, the Leahy-Smith America Invents Act, Pub. L. No. 112-29, 125
Stat. 284 (2011) (``AIA'') switched to a ``first to file'' basis.  Now (subject
to some exceptions) the patent will be issued to the inventor whose application
was filed first, regardless of who was first to conceive of the invention or
reduce it to practice. Which of these systems strikes you as fairer? Which will
be better at promoting invention and encouraging inventors to disclose their
inventions to the public? Which is easier to administer? Which is more
consistent with the principles of first possession you studied in the context
of personal property?

\item The test for patent infringement is whether every limitation in a claim is
present in the defendant's product or process. So if a patent claim includes an
axle and a wheel, a product containing both an axle and a wheel will infringe,
but a product containing only an axle but not a wheel will not. The hard part,
however, is that a court will need to interpret the words of a claim to
determine what kinds of things it does and does not describe. (In
\textit{Innovention Toys}, the crucial term is ``movable.'') How is this test
different from the substantial-similiarty and fair-use tests in copyright law?

The claims of a patent, it is sometimes said, establish the ``metes and bounds''
of the owner's rights in the same way that a description of real property does
(``{\dots} thence 30 feet 6 inches north, thence 74 feet 1 inch west
{\dots}''). Is this an accurate metaphor? Which provides clearer notice to
potential defendants of what they can and cannot do?


\item The patent eqivalent to first sale is the doctrine of \textit{exhaustion}:
the owner of an item covered by a patent may legally sell or use it without the
patent owner's permission. She just can't make more. For example, in
\textit{Bowman v. Monsanto Co.}, 133 S. Ct. 1761 (2013), Monsanto owned a
patent on a pesticide-resistant strain of soybeans and charged farmers a
premium price for these ``Roundup Ready'' beans. Bowman, who liked the beans
but not the price, bought Roundup Ready soybeans not from Monsanto, but from
other other farmers' harvests. The Supreme Court unanmiously held that Bowman
was an infringer. When he planted the beans he ``made'' more patented beans
without Monsanto's permission. Bowman argued that there should be an exception
for technologies that automatically replicate themselves, but the Court
rejected this ``blame-the-bean defense.''


\item Patent remedies are broadly similar to copyright remedies. But there are a
few notable differences. For one thing, there are no statutory damages.
Instead, courts are authorized to award the plaintiff up to treble damages in
cases of ``willful'' infringement. For another, instead of copyright's
restitutionary damage measure, patent law specifies that the patentee is
entitled to ``damages adequate to compensate for the infringement, but in no
event less than a reasonable royalty for the use made of the invention by the
infringer.'' 35 U.S.C. {\S} 284. 


\item Patent rights are broader than copyright rights in an important way. While
copyright has an independent creation defense, patent does not. Even if you
have never heard of the plaintiff or her patent, if you make, use, or sell the
invention it describes, you are an infringer. Recent years have seen a boom in
litigation by so-called ``non-practicing entitles'' -- or, less charitably,
patent ``trolls,'' so called because they allegedly lurk in the shadows like
trolls in fairy tails and then spring out to demand tolls from unsuspecting
travelers who have unknowingly wandered onto their bridges. We will return with
some regularity to the problem of notice in property law; would you expect
notice problems more or less severe in intellectual property than for real and
tangible personal property?

