\reading{Capitol Records, Inc. v. Thomas-Rasset}
\readingcite{692 F. 3d 899 (8th. Cir. 2012)}

COLLOTON, Circuit Judge.

This appeal arises from a dispute between several recording companies and Jammie
Thomas-Rasset. There is a complicated procedural history involving three jury
trials, but for purposes of appeal, it is undisputed that Thomas-Rasset
willfully infringed copyrights of twenty-four sound recordings by engaging in
file-sharing on the Internet. 

[The Copyright Act makes it an act of infringement to ``reproduce'' a
copyrighted work, 17 U.S.C. {\S} 106(1), or to ``distribute copies or
phonorecords of the copyrighted work to the public,'' \textit{id}. {\S} 106(3).
A court may ``grant temporary and final injunctions on such terms as it may
deem reasonable to prevent or restrain infringement of a copyright,''
\textit{id.} {\S} 502(a). As for damages, ``The copyright owner is entitled to
recover the actual damages suffered by him or her as a result of the
infringement, and any profits of the infringer that are attributable to the
infringement and are not taken into account in computing the actual damages.''
\textit{Id}. {\S} 504(b). A victorious copyright owner may, however, ``elect,
at any time before final judgment is rendered, to recover, instead of actual
damages and profits, an award of statutory damages for all infringements
involved in the action, with respect to any one work, {\dots} in a sum of not
less than \$750 or more than \$30,000 as the court considers just.''
\textit{Id}. {\S} 504(c)(1). This amount may be increased up to \$150,000 per
infringed work ``where the copyright owner sustains the burden of proving, and
the court finds, that infringement was committed willfully,'' \textit{id}. {\S}
504(c)(2), or decreased down to \$200 per infringed work ``where the infringer
sustains the burden of proving, and the court finds, that such infringer was
not aware and had no reason to believe that his or her acts constituted an
infringement of copyright,''\textit{id}.

The court conducted three jury trials. All three found Thomas-Rasset liable. For
reasons discussed below, the court entered an injunction against further
infringement but reduced the third jury's award to \$54,000 in statutory
damages.]

The companies appeal two aspects of the remedy ordered by the district court.
They object to the district court's ruling on damages, and they seek an award
of \$222,000, which was the amount awarded by the jury in the first trial. They
also seek a broader injunction that bars Thomas-Rasset from making any of their
sound recordings available to the public. {\dots} In a cross-appeal,
Thomas-Rasset argues that \textit{any} award of statutory damages is
unconstitutional, and urges us to vacate the award of damages altogether.

{\centering
I.
\par}

Capitol Records, Inc., Sony BMG Music Entertainment, Arista Records LLC,
Interscope Records, Warner Bros. Records, and UMG Recordings, Inc., are
recording companies that own the copyrights to large catalogs of music
recordings. In 2005, they undertook to investigate suspected infringement of
these copyrights. MediaSentry, an online investigative firm hired by the
recording companies, discovered that an individual with the username
``tereastarr'' was participating in unauthorized file sharing on the
peer-to-peer network KaZaA.

During the relevant time period, KaZaA was a file-sharing computer program that
allowed its users to search for and download specific files from other users.
KaZaA users shared files using a share folder. A share folder is a location on
the user's computer in which the user places files --- such as audio or video
recordings --- that she wants to make available for other users to download.
KaZaA allowed its users to access other users' share folders, view the files in
the folder, and download copies of files from the folder.

MediaSentry accessed tereastarr's share folder. The investigative firm
determined that the user had downloaded copyrighted songs and was making those
songs available for download by other KaZaA users. MediaSentry took screen
shots of tereastarr's share folder, which included over 1,700 music files, and
downloaded samples of the files. But MediaSentry was unable to collect direct
evidence that other users had downloaded the files from tereastarr. MediaSentry
then used KaZaA to send two instant messages to tereastarr, notifying the user
of potential copyright infringement. Tereastarr did not respond to the
messages. MediaSentry also determined tereastarr's IP address, and traced the
address to an Internet service account in Duluth, Minnesota, provided by
Charter Communications. MediaSentry compiled this data in a report that it
prepared for the recording companies.

Using the information provided by MediaSentry, the recording companies, through
the Recording Industry Association of America (RIAA), issued a subpoena to
Charter Communications requesting the name of the person associated with
tereastarr's IP address. Charter informed the RIAA that the IP address belonged
to Jammie Thomas-Rasset. The RIAA then sent a letter to Thomas-Rasset informing
her that she had been identified as engaging in unauthorized trading of music
and inviting her to contact them to discuss the situation and settle the
matter. Thomas-Rasset contacted the RIAA as directed in the letter and engaged
in settlement conversations with the organization. The parties were unable to
resolve the matter.

In 2006, the recording companies sued Thomas-Rasset, seeking statutory damages
and injunctive relief for willful copyright infringement under the Copyright
Act, 17 U.S.C. {\S} 101 \textit{et seq.} They alleged that Thomas-Rasset
violated their exclusive right to reproduction and distribution under 17 U.S.C.
{\S} 106 by impermissibly downloading, distributing, and making available for
distribution twenty-four copyrighted sound recordings. {\dots}

A jury trial was held in October 2007. At trial, Thomas-Rasset conceded that
``tereastarr'' is a username that she uses regularly for Internet and computer
accounts. She admitted familiarity with and interest in some of the artists of
works found in the tereastarr KaZaA account. She also acknowledged that she
wrote a case study during college on the legality of Napster --- another
peer-to-peer file sharing program --- and knew that Napster was shut down
because it was illegal. Nonetheless, Thomas-Rasset testified that she had never
heard of KaZaA before this case, did not have KaZaA on her computer, and did
not use KaZaA to download files. The jury also heard evidence from a forensic
investigator that Thomas-Rasset removed and replaced the hard drive on her
computer with a new hard drive after investigators notified her of her
potential infringement. The new hard drive did not contain the files at issue.

[The court instructed the jury it could find Thomas-Rasset liable for ``making
available'' the plaintiffs' songs by for download by other users placing MP3
files in her shared folder on the KaZaA filesharing network, whether or not the
plaintiffs showed that anyone else had actually downloaded them. The jury found
her liable and awarded \$222,000 in damages.]

Several months later, the district court \textit{sua sponte} raised the issue
whether it erred by instructing the jury that making sound recordings available
for distribution on a peer-to-peer network violates a copyright owners'
exclusive right to distribution, ``regardless of whether actual distribution
has been shown.'' The parties filed supplemental briefs in which the recording
companies defended the court's instruction and Thomas-Rasset argued that the
court erred when it instructed the jury on the ``making available'' issue.
After a hearing, the district court granted Thomas-Rasset's motion for a new
trial on this alternative ground, holding that making a work available to the
public is not ``distribution'' under 17 U.S.C. {\S} 106(3). The issue whether
making copyrighted works available to the public is a right protected by {\S}
106(3) has divided the district courts. \textit{Compare, e.g., Atl. Recording
Corp. v. Howell,} 554 F.Supp.2d 976, 981-84 (D.Ariz. 2008), \textit{and
London-Sire Records v. Doe 1,} 542 F.Supp.2d 153, 176 (D.Mass.2008),
\textit{with Motown Record Co. v. DePietro,} No. 04-CV-2246, 2007 WL 576284, at
*3 (E.D.Pa. Feb. 16, 2007), \textit{and Warner Bros. Records, Inc., v. Payne,}
No. W-06-CA-051, 2006 WL 2844415, at *3 (W.D.Tex. July 17, 2006).

The district court convened a second trial in June 2009, at which the recording
companies produced substantially the same evidence of Thomas-Rasset's
liability. At this trial, however, Thomas-Rasset attempted to deflect
responsibility by suggesting for the first time that her children and former
boyfriend might have done the downloading and file-sharing attributed to the
``tereastarr'' username. The court again instructed the jury that reproduction
or distribution constituted copyright infringement. But this time, the court
omitted reference to making works available and instructed the jury that
``[t]he act of distributing copyrighted sound recordings to other users on a
peer-to-peer network, without license from the copyright owners, violates the
copyright owners' exclusive distribution right.'' The jury again found
Thomas-Rasset liable for willful infringement, and awarded the recording
companies statutory damages of \$80,000 per work, for a total of \$1,920,000.

Following the second trial, Thomas-Rasset filed a post-trial motion in which she
argued that any statutory damages award would be unconstitutional in her case,
but in the alternative that the court should reduce the jury's award either
through remittitur or based on the Due Process Clause. The district court
declined to rule on the constitutional issue and instead remitted damages to
\$2,250 per work, for a total of \$54,000, on the ground that the jury's award
was ``shocking.'' The recording companies declined the remitted award and
exercised their right to a new trial on damages.

A third trial was held in November 2010, and the only question for the jury was
the amount of statutory damages. The jury awarded the recording companies
statutory damages of \$62,500 per work, for a total of \$1,500,000.

Thomas-Rasset then moved to alter or amend the judgment, again arguing that any
statutory damages award would be unconstitutional, but alternatively that the
district court should reduce the award under the Due Process Clause. The
district court, relying in part on the now-vacated decision in \textit{Sony BMG
Music Entm't v. Tenenbaum,} 721 F.Supp.2d 85 (D.Mass. 2010), \textit{vacated in
relevant part by,} 660 F.3d 487 (1st Cir.2011), granted Thomas-Rasset's motion
and reduced the award to \$2,250 per work, for a total of \$54,000. The court
ruled that this amount was the maximum award permitted by the Due Process
Clause. The district court also entered a permanent injunction against
Thomas-Rasset, but refused to include language enjoining her from ``making
available'' copyrighted works for distribution to the public.

The recording companies appeal the judgment of the district court, arguing that
the district court erred in (1) granting a new trial based on the ``making
available'' instruction in the first trial, and (2) holding that the Due
Process Clause limits statutory damages to \$2,250 per infringed work. They
request that we reinstate and affirm the first jury's \$222,000 award, and
remand with instructions to grant an injunction prohibiting Thomas-Rasset from
making the copyrighted works available to the public. Thomas-Rasset
cross-appeals, arguing that even an award of the minimum statutory damages
authorized by the Copyright Act would be unconstitutional. {\dots}

{\centering
II. {\dots}
\par}

For the reasons set forth below, we conclude that when the district court
entered judgment after the verdict in the third trial, the court should have
enjoined Thomas-Rasset from making copyrighted works available to the public,
whether or not that conduct by itself violates rights under the Copyright Act.
We also conclude that statutory damages of at least \$222,000 were
constitutional, and that the district court erred in holding that the Due
Process Clause allowed statutory damages of only \$54,000. We therefore will
vacate the district court's judgment and remand with directions to enter a
judgment that includes those remedies. {\dots}

{\centering
\textit{A}.
\par}

After the third trial, the district court entered an injunction that prohibits
Thomas-Rasset from ``using the Internet or any online media distribution system
to reproduce (\textit{i.e.,} download) any of Plaintiffs' Recordings, or to
distribute (\textit{i.e.,} upload) any of Plaintiff's Recordings.'' The
recording companies urged the district court to amend the judgment to enjoin
Thomas-Rasset from making any of their sound recordings available for
distribution to the public through an online media distribution system. The
district court declined to do so on the ground that the Copyright Act does not
provide an exclusive right to making recordings available. The court further
reasoned that the injunction as granted was adequate to address the concerns of
the companies. We review the grant or denial of a permanent injunction for
abuse of discretion. \textit{Fogie v. THORN Americas, Inc.,} 95 F.3d 645, 649
(8th Cir.1996). ``Abuse of discretion occurs if the district court reaches its
conclusion by applying erroneous legal principles or relying on clearly
erroneous factual findings.'' \textit{Id.}

We conclude that the district court's ruling was based on an error of law. Even
assuming for the sake of analysis that the district court's ruling on the scope
of the Copyright Act was correct, a district court has authority to issue a
broad injunction in cases where ``a proclivity for unlawful conduct has been
shown.'' \textit{See McComb v. Jacksonville Paper Co.,} 336 U.S. 187, 192, 69
S.Ct. 497, 93 L.Ed. 599 (1949). The district court is even permitted to
``enjoin certain otherwise lawful conduct'' where ``the defendant's conduct has
demonstrated that prohibiting only unlawful conduct would not effectively
protect the plaintiff's rights against future encroachment.'' \textit{Russian
Media Grp., LLC v. Cable America, Inc.,} 598 F.3d 302, 307 (7th Cir.2010)
(citing authorities). If a party has violated the governing statute, then a
court may in appropriate circumstances enjoin conduct that allowed the
prohibited actions to occur, even if that conduct ``standing alone, would have
been unassailable.'' \textit{EEOC v. Wilson Metal Casket Co.,} 24 F.3d 836, 842
(6th Cir.1994) (internal quotation omitted).

Thomas-Rasset's willful infringement and subsequent efforts to conceal her
actions certainly show ``a proclivity for unlawful conduct.'' The recording
companies rightly point out that once Thomas-Rasset makes copyrighted works
available on an online media distribution system, she has completed all of the
steps necessary for her to engage in the same distribution that the court did
enjoin. The record also demonstrates the practical difficulties of detecting
actual transfer of recordings to third parties even when a party has made large
numbers of recordings available for distribution online. The narrower
injunction granted by the district court thus could be difficult to enforce.

For these reasons, we conclude that the district court erred after the third
trial by concluding that the broader injunction requested by the companies was
impermissible as a matter of law. An injunction against making recordings
available was lawful and appropriate under the circumstances, even accepting
the district court's interpretation of the Copyright Act. Thomas-Rasset does
not resist expanding the injunction to include this relief. We therefore will
direct the district court to modify the judgment to include the requested
injunction.

{\centering
\textit{B}.
\par}

On the question of damages, we conclude that a statutory damages award of
\$9,250 for each of the twenty-four infringed songs, for a total of \$222,000,
does not contravene the Due Process Clause. The district court erred in
reducing the third jury's verdict to \$2,250 per work, for a total of \$54,000,
on the ground that this amount was the maximum permitted by the Constitution.

The Supreme Court long ago declared that damages awarded pursuant to a statute
violate due process only if they are ``so severe and oppressive as to be wholly
disproportioned to the offense and obviously unreasonable.'' \textit{St. Louis,
I. M. \& S. Ry. Co. v. Williams,} 251 U.S. 63, 67, 40 S.Ct. 71, 64 L.Ed. 139
(1919). Under this standard, Congress possesses a ``wide latitude of
discretion'' in setting statutory damages. \textit{Id.} at 66, 40 S.Ct. 71.
\textit{Williams} is still good law, and the district court was correct to
apply it.

Thomas-Rasset urges us to consider instead the ``guideposts'' announced by the
Supreme Court for the review of punitive damages awards under the Due Process
Clause. When a party challenges an award of punitive damages, a reviewing court
is directed to consider three factors in determining whether the award is
excessive and unconstitutional: ``(1) the degree of reprehensibility of the
defendant's misconduct; (2) the disparity between the actual or potential harm
suffered by the plaintiff and the punitive damages award; and (3) the
difference between the punitive damages awarded by the jury and the civil
penalties authorized or imposed in comparable cases.'' \textit{State Farm Mut.
Auto. Ins. Co. v. Campbell,} 538 U.S. 408, 418, 123 S.Ct. 1513, 155 L.Ed.2d 585
(2003); \textit{see also BMW of N. Am., Inc. v. Gore,} 517 U.S. 559, 574-75,
116 S.Ct. 1589, 134 L.Ed.2d 809 (1996).

The Supreme Court never has held that the punitive damages guideposts are
applicable in the context of statutory damages. \textit{See Zomba Enters., Inc.
v. Panorama Records, Inc.,} 491 F.3d 574, 586-88 (6th Cir. 2007). Due process
prohibits excessive punitive damages because ``{\textasciigrave}[e]lementary
notions of fairness enshrined in our constitutional jurisprudence dictate that
a person receive fair notice not only of the conduct that will subject him to
punishment, but also of the severity of the penalty that a State may
impose.'{}'' \textit{Campbell,} 538 U.S. at 417, 123 S.Ct. 1513 (quoting
\textit{Gore,} 517 U.S. at 574, 116 S.Ct. 1589). This concern about fair notice
does not apply to statutory damages, because those damages are identified and
constrained by the authorizing statute. The guideposts themselves, moreover,
would be nonsensical if applied to statutory damages. It makes no sense to
consider the disparity between ``actual harm'' and an award of statutory
damages when statutory damages are designed precisely for instances where
actual harm is difficult or impossible to calculate. \textit{See Cass Cnty.
Music Co. v. C.H.L.R., Inc.,} 88 F.3d 635, 643 (8th Cir. 1996). Nor could a
reviewing court consider the difference between an award of statutory damages
and the ``civil penalties authorized,'' because statutory damages \textit{are}
the civil penalties authorized.

Applying the \textit{Williams} standard, we conclude that an award of \$9,250
per each of twenty-four works is not ``so severe and oppressive as to be wholly
disproportioned to the offense and obviously unreasonable.'' 251 U.S. at 67, 40
S.Ct. 71. Congress, exercising its ``wide latitude of discretion,''
\textit{id.} at 66, 40 S.Ct. 71, set a statutory damages range for willful
copyright infringement of \$750 to \$150,000 per infringed work. 17 U.S.C. {\S}
504(c). The award here is toward the lower end of this broad range. As in
\textit{Williams,} ``the interests of the public, the numberless opportunities
for committing the offense, and the need for securing uniform adherence to
[federal law]'' support the constitutionality of the award. \textit{Id.} at 67,
40 S.Ct. 71.

Congress's protection of copyrights is not a ``special private benefit,'' but is
meant to achieve an important public interest: ``to motivate the creative
activity of authors and inventors by the provision of a special reward, and to
allow the public access to the products of their genius after the limited
period of exclusive control has expired.'' \textit{Sony Corp. of Am. v.
Universal City Studios, Inc.,} 464 U.S. 417, 429, 104 S.Ct. 774, 78 L.Ed.2d 574
(1984). With the rapid advancement of technology, copyright infringement
through online file-sharing has become a serious problem in the recording
industry. Evidence at trial showed that revenues across the industry decreased
by fifty percent between 1999 and 2006, a decline that the record companies
attributed to piracy. This decline in revenue caused a corresponding drop in
industry jobs and a reduction in the number of artists represented and albums
released. \textit{See Sony BMG Music Entm't v. Tenenbaum,} 660 F.3d 487, 492
(1st Cir. 2011).

Congress no doubt was aware of the serious problem posed by online copyright
infringement, and the ``numberless opportunities for committing the offense,''
when it last revisited the Copyright Act in 1999. To provide a deterrent
against such infringement, Congress amended {\S} 504(c) to increase the minimum
per-work award from \$500 to \$750, the maximum per-work award from \$20,000 to
\$30,000, and the maximum per-work award for willful infringement from
\$100,000 to \$150,000. \textit{Id.}

Thomas-Rasset contends that the range of statutory damages established by {\S}
504(c) reflects only a congressional judgment ``at a very general level,'' but
that courts have authority to declare it ``severe and oppressive'' and ``wholly
disproportioned'' in particular cases. The district court similarly emphasized
that Thomas-Rasset was ``not a business acting for profit, but rather an
individual consumer illegally seeking free access to music for her own use.''
By its terms, however, the statute plainly encompasses infringers who act
without a profit motive, and the statute already provides for a broad range of
damages that allows courts and juries to calibrate the award based on the
nature of the violation. For those who favor resort to legislative history, the
record also suggests that Congress was well aware of the threat of
noncommercial copyright infringement when it established the lower end of the
range. \textit{See} H.R. Rep. 106-216, at 3 (1999), 1999 WL 446444, at *3.
Congressional amendments to the criminal provisions of the Copyright Act in
1997 also reflect an awareness that the statute would apply to noncommercial
infringement. \textit{See} No Electronic Theft (NET) Act, Pub.L. No. 105-147,
{\S} 2(a), 111 Stat. 2678 (1997); \textit{see also} H.R. Rep. 105-339, at 5
(1997), 1997 WL 664424, at *5.

In holding that any award over \$2,250 per work would violate the Constitution,
the district court effectively imposed a treble damages limit on the \$750
minimum statutory damages award. The district court based this holding on a
``broad legal practice of establishing a treble award as the upper limit
permitted to address willful or particularly damaging behavior.'' Any ``broad
legal practice'' of treble damages for statutory violations, however, does not
control whether an award of statutory damages is within the limits prescribed
by the Constitution. The limits of treble damages to which the district court
referred, such as in the antitrust laws or other intellectual property laws,
represent congressional judgments about the appropriate maximum in a given
context. They do not establish a \textit{constitutional} rule that can be
substituted for a different congressional judgment in the area of copyright
infringement. Although the United States seems to think that the district
court's ruling did not question the constitutionality of the statutory damages
statute, the district court's approach in our view would make the statute
unconstitutional as applied to a significant category of copyright infringers.
The evidence against Thomas-Rasset demonstrated an aggravated case of willful
infringement by an individual consumer who acted to download and distribute
copyrighted recordings without profit motive. If an award near the bottom of
the statutory range is unconstitutional as applied to her infringement of
twenty-four works, then it would be the rare case of noncommercial infringement
to which the statute could be applied.

Thomas-Rasset's cross-appeal goes so far as to argue that \textit{any} award of
statutory damages would be unconstitutional, because even the minimum damages
award of \$750 per violation would be ``wholly disproportioned to the offense''
and thus unconstitutional. This is so, Thomas-Rasset argues, because the
damages award is not based on any evidence of harm caused by her specific
infringement, but rather reflects the harm caused by file-sharing in general.
The district court similarly concluded that ``statutory damages must still bear
\textit{some} relation to actual damages.'' The Supreme Court in
\textit{Williams,} however, disagreed that the constitutional inquiry calls for
a comparison of an award of statutory damages to actual damages caused by the
violation. 251 U.S. at 66, 40 S.Ct. 71. Because the damages award ``is imposed
as a punishment for the violation of a public law, the Legislature may adjust
its amount to the public wrong rather than the private injury, just as if it
were going to the state.'' \textit{Id.} The protection of copyrights is a
vindication of the public interest, \textit{Sony Corp. of Am.,} 464 U.S. at
429, 104 S.Ct. 774, and statutory damages are ``by definition a substitute for
unproven or unprovable actual damages.'' \textit{Cass Cnty. Music Co.,} 88 F.3d
at 643. For copyright infringement, moreover, statutory damages are ``designed
to discourage wrongful conduct,'' in addition to providing ``restitution of
profit and reparation for injury.'' \textit{F.W. Woolworth Co. v. Contemporary
Arts,} 344 U.S. 228, 233, 73 S.Ct. 222, 97 L.Ed. 276 (1952).

Thomas-Rasset highlights that if the recording companies had sued her based on
infringement of 1,000 copyrighted recordings instead of the twenty-four
recordings that they selected, then an award of \$9,250 per song would have
resulted in a total award of \$9,250,000. Because that hypothetical award would
be obviously excessive and unreasonable, she reasons, an award of \$222,000
based on the same amount per song must likewise be invalid. Whatever the
constitutionality of the hypothetical award, we disagree that the validity of
the lesser amount sought here depends on whether the Due Process Clause would
permit the extrapolated award that she posits. The absolute amount of the
award, not just the amount per violation, is relevant to whether the award is
``so severe and oppressive as to be wholly disproportioned to the offense and
obviously unreasonable.'' \textit{Williams,} 251 U.S. at 67, 40 S.Ct. 71. The
recording companies here opted to sue over twenty-four recordings. If they had
sued over 1,000 recordings, then a finder of fact may well have considered the
number of recordings and the proportionality of the total award as factors in
determining where within the range to assess the statutory damages. If and when
a jury returns a multi-million dollar award for noncommercial online copyright
infringement, then there will be time enough to consider it. {\dots}

