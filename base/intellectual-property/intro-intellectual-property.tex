This section takes up intellectual property: rights governing the ownership of
information. There is no one distinctive set of doctrines governing all
intellectual property in the same way that the law of finders applies to all
(well, most) personal property or the law of trespass applies to all (well,
most) real property. Instead, the name ``intellectual property'' is a catch-all
used to group several related sets of legal rights, each of which gives the
rightsholder an exclusive right to use certain information in certain ways. A
defendant who uses that information in that way without the rightsholder
permission is said to be an \textit{infringer}. 

It is common, and in some respects accurate, to describe the rightsholder as the
``owner'' of the information, but keep in mind that only certain specified uses
count as infringement. There is no body of intellectual property law that
prohibits possessing or thinking about information, for example. Instead,
different bodies of intellectual property law restrict different kinds of uses.
In each case, the scope of the owner's rights is closely tied to what kinds of
information that body of law protects and to the rules governing when someone
becomes a rightsholder. The latter is a familiar question: just as first
possession gives initial title to personal property, and conquest is at the
root of title to real property, creation can provide intellectual property
rights. But the former is a new kind of question; we have taken it largely for
granted that land is proper subject matter for real property and other tangible
things are proper subject matter for personal property. Intellectual property
is different, because not every kind of information qualifies. In copyright,
for example, processes are not proper subject matter: as a consequence, the
list of ingredients in a recipe and the steps for combining them are not
copyrightable -- even if they meet all of copyright law's other requirements. 

Learning a body of intellectual property law, therefore, requires learning its
subject matter, its rules of initial ownership, and its rules of infringement.
In this section, we will study three such bodies from the federal level:
copyrights, patents, and trademarks. We will study copyright in more detail as
an example, and then examine patents and trademarks to see how they are both
similar to and different from copyright's model But there are other systems of
intellectual property law as well. Here are a few of the most important ones.



