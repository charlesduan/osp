\reading{Cheney Bros. v. Doris Silk Co.}
\readingcite{35 F.2d 279 (2nd Cir. 1929)}

\textsc{L. HAND}, Circuit Judge:

The plaintiff, a corporation is a manufacturer of silks, which puts out each
season many new patterns, designed to attract purchasers by their novelty and
beauty.  Most of these fail in that purpose, so that not much more than a fifth
catch the public fancy.  Moreover, they have only a short life, for the most
part no more than a single season of eight or nine months.  It is in practice
impossible, and it would be very onerous if it were not, to secure design
patents upon all of these; it would also be impossible to know in advance which
would sell well, and patent only those.  Besides, it is probable that for the
most part they have no such originality as would support a design patent.
Again, it is impossible to copyright them under the Copyright Act (17 USCA {\S}
1 et seq.), or at least so the authorities of the Copyright Office hold.  So it
is easy for any one to copy such as prove successful, and the plaintiff, which
is put to much ingenuity and expense in fabricating them, finds itself without
protection of any sort for its pains.  

Taking advantage of this situation, the defendant copied one of the popular
designs in the season beginning in October, 1928, and undercut the plaintiff's
price.  This is the injury of which it complains.  The defendant, though it
duplicated the design in question, denies that it knew it to be the
plaintiff's, and there thus arises an issue which might be an answer to the
motion.  However, the parties wish a decision upon the equity of the bill, and,
since it is within our power to dismiss it, we shall accept its allegation, and
charge the defendant with knowledge.  

The plaintiff asks for protection only during the season, and needs no more, for
the designs are all ephemeral.  It seeks in this way to disguise the extent of
the proposed  innovation, and to persuade us that, if we interfere only a
little, the solecism, if there be one, may be pardonable.  But the reasoning
which would justify any interposition at all demands that it cover the whole
extent of the injury.  A man whose designs come to harvest in two years, or in
five, has prima facie as good right to protection as one who deals only in
annuals.  Nor could we consistently stop at designs; processess, machines, and
secrets have an equal claim.  The upshot must be that, whenever any one has
contrived any of these, others may be forbidden to copy it.  That is not the
law.  In the absence of some recognized right at common law, or under the
statutes---and the plaintiff claims neither---a man's property is limited to
the chattels which embody his invention. Others may imitate these at their
pleasure.

True, it would seem as though the plaintiff had suffered a grievance for which
there should be a remedy, perhaps by an amendment of the Copyright Law,
assuming that this does not already cover the case, which is not urged here. 
It seems a lame answer in such a case to turn the injured party out of court,
but there are larger issues at stake than his redress.  Judges have only a
limited power to amend the law; when the subject has been confided to a
Legislature, they must stand aside, even though there be an hiatus in completed
justice.  An omission in such cases must be taken to have been as deliberate as
though it were express, certainly after long-standing action on the
subject-matter. Indeed, we are not in any position to pass upon the questions
involved, as Brandeis, J., observed in \textit{International News Service v.
Associated Press}.  We must judge upon records prepared by litigants, which do
not contain all that may be relevant to the issues, for they cannot disclose
the conditions of this industry, or of the others which may be involved. 
Congress might see its way to create some sort of temporary right, or it might
not.  Its decision would certainly be preceded by some examination of the
result upon the other interests affected.  Whether these would prove paramount
we have no means of saying; it is not for us to decide.  Our vision is
inevitably contracted, and the whole horizon may contain much which will
compose a very different picture.

