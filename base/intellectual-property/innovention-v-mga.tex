\reading{Innovention Toys, LLC v. MGA Entertainment, Inc.}
\readingcite{637 F. 3d 1314 (Fed. Cir. 2011)}

LOURIE, Circuit Judge.

MGA Entertainment, Inc.; Wal-Mart Stores, Inc.; and Toys ``R'' Us, Inc.
(collectively, ``MGA'') appeal from the summary judgment decision of the United
States District Court for the Eastern District of Louisiana that the asserted
claims of U.S. Patent 7,264,242 (``the `242 patent'') were infringed and were
not invalid for obviousness. Innovention Toys, LLC v. MGA Entm't, Inc., 665
F.Supp.2d 636 (E.D.La. 2009). Because the district court correctly found no
genuine issues of material fact regarding infringement based on its
construction of the claim term ``movable,'' we affirm the court's grant of
summary judgment of literal infringement. The district court, however, erred in
several of its factual findings underlying its nonobviousness determination. We
therefore vacate the court's grant of summary judgment of nonobviousness and
remand.

{\centering
BACKGROUND
\par}

{\centering
I.
\par}

Innovention Toys, LLC (``Innovention'') brought suit against MGA for
infringement of the `242 patent, which claims a chess-like, light-reflecting
board game and methods of playing the same. The disclosed game includes a
chess-styled playing surface, laser sources positioned to project light beams
over the playing surface when ``fired,'' mirrored and non-mirrored playing
pieces used to direct the lasers' beams, and non-mirrored ``key playing
pieces'' equivalent to the king pieces in chess. See `242 patent col.2
l.64-col.3 l.35. To play the game, players take turns either moving a playing
piece to an unoccupied, adjacent square or rotating (reorienting) a piece
within a square. Id. col.3 ll.21-24; col.8 l.49-col.9 l.12. After moving or
rotating a piece, a player then fires his laser, and if the laser's beam
strikes the non-mirrored surface of a playing piece, that piece is eliminated
from the game. Id. col.3 ll.26-30; col.9 ll.13-17. To win the game, a player
must direct his laser beam to strike, or illuminate, his opponent's
non-mirrored key playing piece, ending the game. Id. col.3 ll.17-20; col.6
ll.44-47.

All the asserted claims, claims 31-33, 39-41, 43-44, 48-50, and 53-54, include a
``key playing pieces'' limitation in which the key pieces are ``movable.''
Claim 31 is representative:

\begin{quote}
A board game for two opposing players or teams of players comprising:
\end{quote}

\begin{quote}
a game board, movable playing pieces having at least one mirrored surface,
\textit{movable} key playing pieces having no mirrored surfaces, and a laser
source,
\end{quote}

\begin{quote}
wherein alternate turns are taken to move playing pieces for the purpose of
deflecting laser beams, so as to illuminate the key playing piece of the
opponent.
\end{quote}

\textit{Id.} claim 31 (emphasis added).

MGA counterclaimed, denying infringement and alleging, inter alia, that the `242
patent was invalid under 35 U.S.C. {\S} 103. In making its obviousness
argument, MGA relied on the combination of (1) two articles describing
computer-based, chess-like strategy games, Laser Chess and Advanced Laser Chess
(collectively, ``the Laser Chess references''); and (2) U.S. Patent 5,145,182
(``the Swift patent'') describing a physical, chess-like, laser-based strategy
game.

The Laser Chess game is described in an article entitled ``Laser
Chess{\texttrademark} First Prize \$5,000.00 Winner Atari ST Programming
Contest,'' published in the April 1987 edition of Compute!. J.A. 1775-78.
Advanced Laser Chess is described in an article published in the Summer 1989
edition of Compute!'s Amiga Resource. J.A. 1784-87. Both articles disclose
chess-like computer games with virtual lasers and mirrored and non-mirrored
pieces, which are moved or rotated by players during alternating turns on a
virtual, chess-like playing board. The goal of each game is to manipulate one's
laser beam using the various game pieces to eliminate the other player's
non-mirrored king piece by striking it with the laser beam. In Laser Chess, a
player's king piece may move squares during game play: ``[The king] can capture
any opposing piece by moving onto its square.'' J.A. 1776. Similarly, in
Advanced Laser Chess ``Kings possess the ability to capture other pieces [by
moving on top of them].'' J.A.1784.

The Swift patent discloses a physical (rather than electronic) strategy game in
which players take turns placing mirrored game pieces onto squares of a
chess-styled game board. The players position the pieces so as to direct their
laser's beam towards the opposing player's scoring module and away from their
own. A player scores when his laser beam, having been deflected around the game
board, strikes his opponent's scoring module. The scoring modules are mounted
to the frame of the game board, see Swift patent col.2 ll.51-56, and thus are
not physically capable of movement on the game board.

MGA's accused game, Laser Battle, is a physical board game for playing a
chess-like strategy game. Players take turns moving or rotating mirrored
playing pieces so as to direct a laser beam to strike the opposing player's
non-mirrored Tower playing piece to win the game. According to the rules of
Laser Battle, in ``Classic Game Play,'' the Tower pieces are placed on the
board at the beginning of the game at one of various standard positions. J.A.
1986. Although the Towers are physically capable of movement on the game board,
the rules provide that they ``should always remain in their original positions
on the board.'' J.A.1985. However, the standard starting configuration
illustrated in the rules show that the Tower pieces can be placed at different
locations on the board, and the rules state that during ``Advanced Game Play,''
the Towers need not remain in their standard positions. J.A.1986.

{\centering
II.
\par}

On October 14, 2009, the district court ruled on the parties' cross-motions for
summary judgment of infringement and invalidity. \textit{Innovention Toys,} 665
F.Supp.2d 636. The district court granted Innovention's motion for summary
judgment of literal infringement. \textit{Id.} at 647. The court first
construed the claim term ``movable'' in light of the term's plain meaning as
``capable of movement as called for by the rules of the game or game
strategy.'' \textit{Id.} at 644-45. In so holding, the court rejected MGA's
more ``cramped'' construction that the movement must be ``from space to space
or by rotation within a single space.'' \textit{Id.} at 643-45; \textit{see
also id.} at 644 n. 15. The district court also rejected Innovention's broader
construction of ``movable'' to mean ``able to be moved, or possible to move,''
without any tie to the rules or strategy of the game. \textit{Id.} at 642-43.

Based on its construction, the district court then found that the accused Laser
Battle game's Tower pieces met the ``movable'' claim limitation, the only
disputed limitation. The court found that those key pieces were ``moveable''
when the players selected and ``set up the game pieces in various start
formations and layouts.'' Id. at 647 n. 18. The court reasoned that even if
``the instructions to Laser Battle suggest that its key pieces, the Towers, are
not supposed to be moved during game play,'' they ``are capable of movement . .
. in that they fit into the recessed spaces on the board like the other pieces,
and are capable of rotation within whatever recessed space a player may choose
to place the Tower in for the start of game play.'' Id. at 646-47. The court
also observed that although the instructions to Laser Battle note that ``[t]he
Towers should always remain in their original positions on the board,'' id. at
647 n. 18, they also explain that ``[e]xcept for Advanced Game Play, the Target
Tower and the Laser Guns will remain in their standard positions in all
formations,'' see id. at 647 n. 18 \& 19; J.A.1986. This instruction suggests
that the Tower pieces could be moved in Advanced Game Play because they fit in
and are rotatable in the recessed spaces on the board.

The district court also granted Innovention's motion for summary judgment of
nonobviousness. Id. at 655. The court first found that the Laser Chess
references were non-analogous art because they described electronic, rather
than real-world, laser games. Id. at 653. The district court then held that,
because MGA had provided no evidence to support a finding as to the level of
ordinary skill in the art, MGA's obviousness argument could be pursued only on
the basis of what would have been obvious to a layperson. Id. at 654. The court
then decided that because MGA had not provided any evidence that a layperson
would have known of the Laser Chess articles or would have had any reason to
modify the teachings of the Laser Chess references, MGA had failed to state a
prima facie case of obviousness. Id.

Finally, the court found that Innovention had demonstrated several secondary
considerations of nonobviousness. These included (1) commercial success based
on the sale of 140,000 games by Innovention, a small company with minimal
marketing capabilities, and evidence that fans had started clubs and
tournaments around the world; (2) long-felt need based on the game's sudden
success and media praise; and (3) industry praise based on, inter alia, the
game's nomination for Outstanding Technology of the Year by the International
Academy of Science and its being one of five finalists for the Toy Industry
Association's 2007 Game of the Year award. Id. at 654-55. In light of its
summary judgment rulings, the district court granted Innovention's motion for a
permanent injunction on January 13, 2010.

MGA appealed. We have jurisdiction pursuant to 28 U.S.C. {\S} 1295(a)(1).

{\centering
DISCUSSION
\par}

This court reviews a district court's decision on summary judgment \textit{de
novo,} reapplying the same standard applied by the district court.
\textit{Iovate Health Scis., Inc. v. Bio-Engineered Supplements \& Nutrition,
Inc.,} 586 F.3d 1376, 1380 (Fed. Cir.2009). Summary judgment is appropriate
``if the movant shows that there is no genuine dispute as to any material fact
and the movant is entitled to judgment as a matter of law.'' Fed.R.Civ.P.
56(a).

{\centering
I. Infringement
\par}

A determination of infringement involves two steps: First, the court determines
the scope and meaning of the asserted patent claims. The court then compares
the properly construed claims to the allegedly infringing device to determine
whether all of the claim limitations are present, either literally or by a
substantial equivalent. Cybor Corp. v. FAS Techs., Inc., 138 F.3d 1448, 1454
(Fed.Cir.1998) (en banc). {\dots}

On appeal, MGA does not argue that the district court erred in construing
``movable'' to mean ``capable of movement as called for by the rules of the
game or game strategy.'' Rather, MGA argues that the court improperly broadened
its construction during the second step of the infringement analysis by adding
capable of movement ``during game set up,'' and thus erred in finding that
Laser Battle's Tower pieces meet the ``movable'' limitation. Under the court's
original construction, MGA asserts, Laser Battle's instructions definitively
show that the Tower pieces are not to be moved, either from space to space or
by rotation within a space, as part of the game's rules or strategy. Rather,
MGA urges, the only way that the Tower pieces can be ``movable'' during game
play or strategy is if the players ignore the game's rules and cheat. Moreover,
according to MGA, because all the pieces must be moved onto the game board at
some point, the district court's altered construction renders everything
``movable'' and the ``movable'' limitation superfluous.

Innovention responds that Laser Battle's Tower pieces are clearly ``capable of
movement''; each Tower piece is a separate component that can be placed at any
space on the game board and afterwards rotated within that space or moved to
another space. In other words, Innovention argues, the Towers are ``movable''
since they are not permanently mounted to the game board as are the key playing
pieces (scoring modules) disclosed in the Swift patent. Furthermore, according
to Innovention, Laser Battle's instructions do not require the Tower pieces to
remain stationary during all game play; the rules invite players to create
``more intricate games'' and tell players that the Towers ``remain in their
standard positions in all formations'' except during ``Advanced Game Play.''
J.A.1986-87.

We agree in main with the district court's infringement
analysis,\textsuperscript{ }and we affirm the district court's decision holding
that MGA's Laser Battle game literally infringes the asserted claims of the
`242 patent. During claim construction, the district court rejected MGA's
narrower construction of ``movable'' in which ``movable'' was limited to the
movements explicitly permitted by the rules of the game during game play, i.e.,
movable from space to space or by rotation within a single space. Innovention
Toys, 665 F.Supp.2d at 642-43. Rather, the court embraced a definition that
distinguished ``movable'' from ``mounted'' by contrasting the ``movable key
playing pieces'' of the `242 patent with the fixed scoring modules of Swift. In
particular, the district court stated that while the ``key pieces disclosed in
the Swift reference are permanently fixed to the game board and, therefore,
cannot be moved prior to or during game play,'' the key pieces of the `242
patent ``may be positioned in different spaces at the beginning of each game
and can also be moved during game play.'' Id. at 644; see also id. at 644 n.
15. Thus, the court's construction of ``movable'' includes the capability of
being positioned in different spaces during set up (i.e., at the beginning of
each game) and the capability of being moved during Advanced Game Play. In
other words, the Tower pieces ``are capable of movement as called for by the
rules of the game or game strategy.''

Accordingly, MGA is incorrect when it argues that the district court expanded
its construction of ``movable'' during the second step of the infringement
analysis. The court's construction, not just its application, encompasses
movement during game set up. Moreover, such a construction does not render
``movable'' superfluous, as MGA asserts, since it distinguishes the `242
patent's key pieces from the mounted scoring modules disclosed in Swift. Laser
Battle's Tower pieces thus meet the ``movable'' limitation under the district
court's consistent construction based on the pieces' undisputed ability to be
physically positioned in different squares on the game board. No reasonable
jury could find otherwise. {\dots}

{\centering
II. Obviousness
\par}

Under the Patent Act, ``[a] patent may not be obtained . . . if the differences
between the subject matter sought to be patented and the prior art are such
that the subject matter as a whole would have been obvious at the time the
invention was made to a person having ordinary skill in the art to which said
subject matter pertains.'' 35 U.S.C. {\S} 103(a). Although the ultimate
determination of obviousness under {\S} 103 is a question of law, it is based
on several underlying factual findings, including (1) the scope and content of
the prior art; (2) the level of ordinary skill in the pertinent art; (3) the
differences between the claimed invention and the prior art; and (4) evidence
of secondary factors, such as commercial success, long-felt need, and the
failure of others. Graham v. John Deere Co., 383 U.S. 1, 17-18, 86 S.Ct. 684,
15 L.Ed.2d 545 (1966). A patent is presumed valid, 35 U.S.C. {\S} 282, and this
presumption can be overcome only by clear and convincing evidence to the
contrary. Bristol-Myers Squibb Co. v. Ben Venue Labs., Inc., 246 F.3d 1368,
1374 (Fed.Cir.2001).

MGA argues that, rather than being nonobvious, the asserted claims would have
been obvious based on the combination of the Laser Chess references, which
teach the claimed game in electronic form, and the Swift patent, which teaches
a physical laser-based game. According to MGA, the district court erred both
(1) in concluding that because the `242 patent relates to a physical game, the
Laser Chess articles were non-analogous art; and (2) in assuming that a person
of skill in the art was a layperson rather than, as put forth by Innovention, a
mechanical engineer with knowledge of optics. Finally, MGA argues,
Innovention's unsupported and conclusory assertions of secondary considerations
fail to overcome MGA's prima facie case of obviousness. {\dots}

We conclude that the district court clearly erred in several of the factual
findings underlying its obviousness analysis. The district court erred in
finding that the Laser Chess references fail to qualify as analogous art. The
court also erred in concluding that the level of skill in the art is that of a
layperson. We address each in turn.

{\centering
\textit{A. Analogous Art}
\par}

A reference qualifies as prior art for a determination under {\S} 103 when it is
analogous to the claimed invention. In re Clay, 966 F.2d 656, 658
(Fed.Cir.1992). ``Two separate tests define the scope of analogous art: (1)
whether the art is from the same field of endeavor, regardless of the problem
addressed, and (2) if the reference is not within the field of the inventor's
endeavor, whether the reference still is reasonably pertinent to the particular
problem with which the inventor is involved.'' In re Bigio, 381 F.3d 1320, 1325
(Fed.Cir.2004). {\dots}

Innovention argues that the Laser Chess articles are non-analogous art because
the `242 patent's inventors were concerned with making a non-virtual,
three-dimensional, laser-based board game, a project that involves mechanical
engineering and optics, not computer programming. The district court appears to
have agreed, finding that the Laser Chess references were non-analogous art
since each discloses ``an electronic version of the `242 patent.'' Innovention
Toys, 665 F.Supp.2d at 653. The court, however, failed to consider whether a
reference disclosing an electronic, laser-based strategy game, even if not in
the same field of endeavor, would nonetheless have been reasonably pertinent to
the problem facing an inventor of a new, physical, laser-based strategy game.
In this case, the district court clearly erred in not finding the Laser Chess
references to be analogous art based on this test as a matter of law. See Wyers
v. Master Lock Co., 616 F.3d 1231, 1238 (Fed.Cir.2010) (holding as a matter of
law that prior art padlock seals were analogous since directed to the same
problem of preventing the ingress of contaminants into the locking mechanism).

The `242 patent and the Laser Chess references are directed to the same purpose:
detailing the specific game elements comprising a chess-like, laser-based
strategy game. Specifically, the `242 patent describes (1) the game's
components, including the game board, `242 patent col.4 ll.45-56, and various
types of playing pieces, id. col.6 l.48-col.7 l.24; (2) the game's specific
rules, including how the pieces may move on the game board during a player's
turn, id. col.3 ll.21-28, col.8 l.49-col.9 l.17; and (3) the game's ultimate
objective, namely, illuminating an opponent's key playing piece with a laser
beam, id. col.6 ll.45-47. The specification even distinguishes prior art
patents based on these game elements, stating that U.S. Patent 3,516,671 lacks
``the unique elements and rules of the [`242 patent's] invention,'' id. col.1
ll.47-50, and U.S. Patent 6,702,286 contemplates a game in which the objective
is not to ``illuminate playing pieces,'' but rather ``to maneuver one's pieces
to flank (or surround) those of the opposing player,'' id. col.2 ll.16-21.

The Laser Chess references likewise describe specific playing pieces, rules, and
objectives to create a chess-like, laser-based strategy game. Both Laser Chess
and Advanced Laser Chess disclose, for example, (1) various game pieces, each
with unique capabilities, J.A. 1775-77, 1784-85; (2) rules for each player's
turn, J.A. 1777-78, 1785-86; and (3) an ultimate objective of eliminating an
opponent's king piece, J.A. 1775, 1784.

Accordingly, the `242 patent and the Laser Chess references relate to the same
goal: designing a winnable yet entertaining strategy game. The `242 patent's
specification confirms that game design was one objective facing its inventors.
In particular, the specification states that ``[s]trategy games may differ in a
variety of ways,'' such as in board layout, the number and types of playing
pieces, and the manner in which each piece moves on the game board, and that
``[e]ach of these variations affects the strategy of the play and the degree of
skill required to play the game.'' `242 patent col.2 ll.19-46. The
specification thus admonishes that if the game elements ``are overly
simplistic, the game is too easy, will usually end in a draw or a predictable
manner, and quickly become uninteresting for the average player.'' Id. col.2
ll.49-54. Conversely, according to the specification, if the game elements
``are overly complicated, the game takes too long to learn [and] is frustrating
and uninteresting for the average player.'' Id. col.2 ll.57-60.

The specific combination of game elements disclosed and claimed in the `242
patent thus deals with the problem of game design, and game elements from any
strategy game, regardless how implemented, ``logically would have commended
itself to an inventor's attention in considering [this] problem.'' Clay, 966
F.2d at 659. Basic game elements remain the same regardless of the medium in
which they are implemented: whether molded in plastic by a mechanical engineer
or coded in software by a computer scientist. And, as MGA's evidence shows,
inventors of numerous prior art patents contemplated the implementation of
their strategy games in both physical and electronic formats. Innovention Toys,
665 F.Supp.2d at 650 n. 23. For example, the Swift patent states that
``[a]lthough the preferred embodiment is played by two players, obvious
modifications of the game allow for . . . a single player playing against a
computer.'' Swift col.2 ll.47-51. Thus, because no reasonable jury could find
that the Laser Chess references do not qualify as analogous prior art, and the
district court erred in not so concluding as a matter of law.

Because of its error, the district court failed to properly consider the scope
and content of the relevant prior art as well as the differences between that
art and the claimed invention, including whether one of ordinary skill in the
art would have been motivated to combine the teachings of the Laser Chess
references with the Swift patent in light of the standard articulated in KSR
International Co. v. Teleflex, Inc., 550 U.S. 398, 127 S.Ct. 1727, 167 L.Ed.2d
705 (2007). We therefore remand these factual determinations to the district
court to consider in the first instance. Furthermore, should the district court
conclude that MGA has made out a prima facie case of obviousness based on the
Laser Chess articles and the Swift patent, the court must then determine
whether Innovention's secondary considerations overcome MGA's prima facie case.
{\dots}

III. Permanent Injunction

The district court, based on its grant of summary judgment of infringement and
nonobviousness, granted a permanent injunction to Innovention. Because we
vacate and remand the district court's decision of summary judgment of
nonobviousness, we also vacate the district court's permanent injunction.
{\dots}

