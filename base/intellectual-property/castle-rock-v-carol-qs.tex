\expected{castle-rock-v-carol}

\item \textit{Castle Rock} illustrates two essential copyright doctrines: the
substantial similarity test for infringement  and the fair use defense. Compare
them with the tests used in real- and personal-property torts like trespass,
nuisance, and conversion. Which of them are more definite and which are more
open-ended? What might account for the difference? Land has definite
boundaries; what are the ``boundaries'' of a copyrighted work?

\item Fair use comes in many forms. Examples of uses protected as fair uses
include 2 Live Crew's filthy cover rap version of the Roy Orbison song ``Oh,
Pretty Woman,'' \emph{Campbell v. Acuff-Rose Music, Inc.}, 510 U.S. 569 (1994),
a
retelling of Gone With the Wind from the slaves' point of view, \emph{Suntrust Bank
v. Houghton Mifflin Co.}, 268 F.3d 1257 (11th Cir. 2001), using a VCR to record
live over-the-air TV shows to watch them later, \emph{Sony Corp. of America v.
Universal City Studios, Inc.}, 464 U. S. 417, 451 (1984), making digital copies
of physical books to create a search engine, \emph{Authors Guild v. Google, Inc.}, 804
F.3d 202 (2d Cir. 2015), quoting from an author's works as part of a critical
biography, \emph{New Era Publications Intl. v. Carol Pub. Group}, 904 F.2d 152 (2d.
Cir. 1990), reproducing Grateful Dead concert posters as part of a timeline in
a coffee-table book about the band's history, \emph{Bill Graham Archives v. Dorling
Kindersley Ltd.}, 448 F.3d 605 (2d. Cir. 2006), and making copies of a work to
be used as evidence in a lawsuit, \emph{Jartech, Inc. v. Clancy}, 666 F.2d 403 (9th
Cir. 1982). Is there any unifying principle tying together such disparate uses?

