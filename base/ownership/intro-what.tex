We began this chapter with Blackstone's strong statement of the ``absolute
right'' of property, and have watched it gradually melt away. We have seen
courts use a subtle and diverse array of tools to vindicate interests that
conflict with a property owner's ``absolute'' rights. In \textit{Marsh}, the
Court opined that state-law rights of property must give way to more important
principles enshrined in the federal Constitution. In \textit{Shack}, the court
explicitly avoids this kind of Constitutional trump card by manipulating the
\textit{scope of the owner's rights} under the common law of property to avoid
conflict with competing \textit{statutory} policies. The court in
\textit{Eyerman} takes a similar approach to the testatrix's efforts to direct
disposition of her property after death, even where there appears to be no
danger of conflict with any Constitutional---or even statutory---interest. Is
there any limit to the scope or variety of these types of manipulations? And if
not, how are we ever to say what property \textit{is}?

We might look to two possible foundations for a more resilient concept of
property. One foundation might be that property is a particular
\textit{cohesive} construct: a package deal. This is, indeed, one common
interpretation of the ``bundle of rights'' metaphor we first encountered in
\textit{Jacque}. Thus, when we say that a person \textit{owns} something, we
might be saying that the person enjoys the various rights of owners we have
been studying (the right to exclude, possess, use, alienate, etc.) with respect
to that thing. If we could support this interpretation, it really might help to
distinguish property in a meaningful way from other private law rights---such
as those that arise in contract or tort---and allow us to predict how
particular disputes are likely to shake out. Of course, the cases we have
already studied---in which courts limit or deny owners' rights depending on the
circumstances in which they are asserted---may give us some doubts about our
likelihood of success. And we've only just begun: We will be encountering more
legal authorities that will challenge our ability to think about property as a
coherent ``bundle'' of rights, as opposed to an \textit{ad hoc} and unstable
collection of whatever rights and duties we choose to apply in a particular set
of circumstances:
\begin{itemize}
\item In our unit on the Subject Matter of Property, we will see how some things
may be called ``property'' even though they are not subject to certain of the
traditional rights of ownership---particularly the right to alienate.

\item In our unit on Estates and Future Interests, we will see how property
rights can be \textit{temporally} divided---that a property right in land that
exists today may nevertheless not entitle its owner to \textit{possession} of
that land until some point in the future.

\item In our unit on Concurrent Interests, we will see how the division of
ownership rights among \textit{multiple people} similarly cabins the rights to
exclude, possess, alienate, and use---at least among co-owners.

\item In our unit on Takings, we will see that in some circumstances the right
to exclude, standing alone, may be a sufficient condition for identifying
``property.''
\end{itemize}
So perhaps this approach is not very promising. While there is a menu of rights
that appear to be \textit{consistent} with ownership, it appears that the
concept or label of ``property'' does not \textit{necessarily} depend on a
particular combination of those rights being present.

A second possible foundation for our conception of property is that property, at
the very least, involves some \textit{thing} that is the subject of the right
(or rights): that it is a right \textit{in rem}. In particular, it might be
intimately tied up with an individual's right to \textit{control some
thing}---principally but not only by excluding others from access to that
thing. Again, the requirement of intermediation by some \textit{thing} might
also help distinguish property from contract and tort---which may but need not
involve competing claims to a \textit{thing}.

We will consider the types of \textit{things} that might qualify as property in
our unit on the Subject Matter of Property. But before doing so, we ought to
consider whether thinking of property in this way---as a relationship between
people and things---is sound, or useful. Consider the following scholarly
treatments of these ideas.


