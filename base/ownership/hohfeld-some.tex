\hyphenation{hoh-feld}

\reading[Hohfeld, \emph{Some Fundamental Legal Conceptions}]{Wesley Newcomb
Hohfeld, \textit{Some Fundamental
Legal Conceptions as Applied in Judicial Reasoning}}
\readingcite{23 Yale L.J. 16, 28-30, 31-33, 45-46, 55 (1913)}

One of the greatest hindrances to the clear understanding, the incisive
statement, and the true solution of legal problems frequently arises from the
express or tacit assumption that all legal relations may be reduced to
``rights'' and ``duties,'' and that these latter categories are therefore
adequate for the purpose of analyzing even the most complex legal interests,
such as trusts, options, escrows, ``future'' interests, corporate interests,
etc. Even if the difficulty related merely to inadequacy and ambiguity of
terminology, its seriousness would nevertheless be worthy of definite
recognition and persistent effort toward improvement; for in any closely
reasoned problem, whether legal or non-legal, chameleon-hued words are a peril
both to clear thought and to lucid expression. As a matter of fact, however,
the above mentioned inadequacy and ambiguity of terms unfortunately reflect,
all too often, corresponding paucity and confusion as regards actual legal
conceptions. That this is so may appear in some measure from the discussion to
follow.

The strictly fundamental legal relations are, after all, \textit{sui generis};
and thus it is that attempts at formal definition are always unsatisfactory,
if not altogether useless. Accordingly, the most promising line of procedure
seems to consist in exhibiting all of the various relations in a scheme of
``opposites'' and ``correlatives,'' and then proceeding to exemplify their
individual scope and application in concrete cases. An effort will be made to
pursue this method: 

\heregraphic{hohfeld}

\ldots

Recognizing, as we must, the very broad and indiscriminate use of the term,
``right,'' what clue do we find, in ordinary legal discourse, toward limiting
the word in question to a definite and appropriate meaning. That clue lies in
the correlative ``duty,'' for it is certain that even those who use the word
and the conception ``right'' in the broadest possible way are accustomed to
thinking of ``duty'' as the invariable correlative\ldots .

In other words, if X has a right against Y that he shall stay off the former's
land, the correlative (and equivalent) is that Y is under a duty
toward X to stay off the place. If, as seems desirable, we should seek a
synonym for the term ``right'' in this limited and proper meaning, perhaps the
word ``claim'' would prove the best\ldots .

As indicated in the above scheme of jural relations, a privilege is the opposite
of a duty, and the correlative of a ``no-right.'' In the example last put,
whereas X has a \textit{right} or \textbf{\textit{claim}} that Y, the other
man, should stay off the land, he himself has the \textit{privilege} of
entering on the land; or, in equivalent words, X does not have a duty to stay
off. The privilege of entering is the negation of a duty to stay off. As
indicated by this case, some caution is necessary at this point, for, always,
when it is said that a given privilege is the mere negation of a \textit{duty},
what is meant, of course, is a duty having a content or tenor precisely
\textit{opposite} to that of the privilege in question. Thus, if, for some
special reason, X has contracted with Y to go on the former's own land, it is
obvious that X has, as regards Y, both the privilege of entering and the
\textit{duty of entering}. The privilege is perfectly consistent with this sort
of duty,---for the latter is of the \textit{same} content or tenor as the
privilege;---but it still holds good that, as regards Y, X's
privilege of entering is the precise negation of a duty \textit{to stay
off}\ldots.

Passing now to the question of ``correlatives,'' it will be remembered, of
course, that a duty is the invariable correlative of that legal relation which
is most properly called a right or claim. That being so, if further evidence be
needed-as to the fundamental and important difference between a right (or
claim) and a privilege, surely it is found in the fact that the correlative of
the latter relation is a ``no-right,'' there being no single term available to
express the latter conception. Thus, the correlative of X's right that Y shall
not enter on the land is Y's duty not to enter; but the correlative of X's
privilege of entering himself is manifestly Y's ``no-right'' that X shall not
enter.\ldots

The nearest synonym [for power] for any ordinary case seems to be (legal)
``ability,''---the latter being obviously the opposite of ``inability,'' or
``disability.''\ldots

Many examples of legal powers may readily be given. Thus, X, the owner of
ordinary personal property ``in a tangible object'' has the power to extinguish
his own legal interest (rights, powers, immunities, etc.) through that totality
of operative facts known as abandonment; and---simultaneously and
correlatively---to create in other persons privileges and powers relating to
the abandoned object---\textit{e, g.,} the power---to acquire title to the
later by appropriating it. \textit{Similarly}, X has the power to transfer his
interest to Y,-that is, to extinguish his own interest and concomitantly create
in Y a new and corresponding interest\ldots. The creation of an agency relation
involves, \textit{inter alia}, the grant of legal powers to the so-called
agent, and the creation of correlative liabilities in the principal. That is to
say, one party P has the power to create agency powers in another party
A,---for example, \ldots the power to impose (so-called) contractual
obligations on P, the power to discharge a debt, owing to P, the power to
``receive'' title to property so that it shall vest in P, and so forth\ldots .

Perhaps it will also be plain, from the preliminary outline and from the
discussion down to this point, that a power bears the same general contrast to
an immunity that a right does to a privilege. A right is one's affirmative
claim against another, and a privilege is one's freedom from the right or claim
of another. Similarly, a power is one's affirmative ``control'' over a given
legal relation as against another; whereas an immunity is one's freedom from
the legal power or ``control'' of another as regards some legal relation.

A few examples may serve to make this clear. X, a landowner, has, as we have
seen, power to alienate to Y or to any other ordinary party. On the other hand,
X has also various immunities as against Y, and all other ordinary parties. For
Y is under a disability (\textit{i.e.,} has no power) so far as shifting the
legal interest either to himself or to a third party is concerned \ldots .

