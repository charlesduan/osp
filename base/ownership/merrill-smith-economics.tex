\readingnote{Reproduced by permission of Henry E. Smith.}
\reading[Merrill \& Smith, \emph{What Happened to Property}]{Thomas W. Merrill
\& Henry E. Smith, \textit{What Happened
to Property in Law and Economics?}} 
\readingcite{111 {Yale L. J.} 357, 357-365 (2001)}

It is a commonplace of academic discourse that property is simply a ``bundle of
rights,'' and that any distribution of rights and privileges among persons with
respect to things can be dignified with the (almost meaningless) label
``\,`property.''\,' By and large, this view has become conventional wisdom
among legal scholars: Property is a composite of legal relations that holds
between persons and only secondarily or incidentally involves a ``thing.''
Someone who believes that property is a right to a thing is assumed to suffer
from a childlike lack of sophistication---or worse. 

\ldots In other times and places, a very different conception of property has
prevailed. In this alternative conception, property is a distinctive type of
right to a thing, good against the world. This understanding of the in rem
character of the right of property is a dominant theme of the civil law's ``law
of things.'' For Anglo-American lawyers and legal economists, however, such
talk of a special category of rights related to things presumably illustrates
the grip of conceptualism on the civilian mind and a slavish devotion to the
gods of Roman law.

Or does it? In related work, we have argued that, far from being a quaint aspect
of the Roman or feudal past, the in rem character of property and its
consequences are vital to an understanding of property as a legal and economic
institution.\readingfootnote{7}{Thomas W. Merrill \& Henry E.
Smith, \textit{Optimal Standardization in the Law of Property: The Numerus
Clausus Principle}, 110 \textsc{Yale L.J.} 1 (2000)\ldots; Thomas W. Merrill
\& Henry E. Smith, \textit{The Property/Contract Interface}, 101 \textsc{Colum.
L. Rev}. 773 (2001)\ldots.} Because core property rights attach to persons
only through the intermediary of some thing, they have an impersonality and
generality that is absent from rights and privileges that attach to persons
directly. When we encounter a thing that is marked in the conventional manner
as being owned, we know that we are subject to certain negative duties of
abstention with respect to that thing-not to enter upon it, not to use it, not
to take it, etc. And we know all this without having any idea who the owner of
the thing actually is. In effect, these universal duties are broadcast to the
world from the thing itself\ldots.

Property rights historically have been regarded as in rem. In other words,
property rights attach to persons insofar as they have a particular
relationship to some thing and confer on those persons the right to exclude a
large and indefinite class of other persons (``the world'') from the thing. In
this sense, property rights are different from in personam rights, such as
those created by contracts or by judicial judgments. In personam rights attach
to persons as persons and obtain against one or a small number of other
identified persons. A number of historically significant property theorists
have recognized the in rem nature of property rights and have perceived that
this feature is key because it establishes a base of security against a wide
range of interferences by others\ldots.

\ldots Blackstone perceived that property rights are important because they
establish a basis of security of expectation regarding the future use and
enjoyment of particular resources. By establishing a right to resources that
holds against all the world, property provides a guarantee that persons will be
able to reap what they have sown\ldots. In other words, property is important
because it gives legal sanction to the efforts of the owner of a thing to
exclude an indefinite and anonymous class of marauders, pilferers, and thieves,
thereby encouraging development of the thing.

\ldots In contrast, the role of property emphasized in modern economic
discussions---providing a baseline for contractual exchange and a mechanism for
resolving disputes over conflicting uses of resources---was at most of
secondary importance in these traditional accounts.\ldots Early in the
twentieth century, Wesley Hohfeld provided an account of legal relations that
proved to be especially influential in transforming the underlying assumptions
about property rights in Anglo-American scholarship.\ldots Hohfeld noted
\ldots that in personam rights are unique rights residing in a person and
availing against one or a few definite persons; in rem rights, in contrast,
reside in a person and avail against ``persons constituting a very large and
indefinite class of people.'' 

Significantly, however, Hohfeld failed to perceive that in rem property rights
are qualitatively different in that they attach to persons insofar as they have
a certain relationship to some thing. Rather, Hohfeld suggested that in
personam and in rem rights consist of exactly the same types of rights,
privileges, duties, and so forth, and differ only in the indefiniteness and the
number of the persons who are bound by these relations. To use a modern
expression, Hohfeld thought that in rem relations could be ``cashed out'' into
the same clusters of rights, duties, privileges, liabilities, etc., as are
constitutive of in personam relations.

Hohfeld did not use the metaphor ``bundle of rights'' to describe property. But
his theory of jural opposites and correlatives, together with his effort to
reduce in rem rights to clusters of in personam rights, provided the
intellectual justification for this metaphor, which became popular among the
legal realists in the 1920s and 1930s.  Different writers influenced by realism
took the metaphor to different extremes. For some, the bundle-of-rights concept
simply meant that property could be reduced to recognizable collections of
functional attributes, such as the right to exclude, to use, to transfer, or to
inherit particular resources. For others, property had no inherent meaning at
all. As one pair of writers put it, the concept of property is nothing more
than ``a euphonious collocation of letters which serves as a general term for
the miscellany of equities that persons hold in the
commonwealth.''\readingfootnote{36}{Walton H. Hamilton \&
Irene Till, \textit{Property}, \textit{in} 12 \textsc{Encyclopaedia of the
Social Sciences} 528, 528 (Edwin R.A. Seligman ed., 1934).}

Notwithstanding these variations, the motivation behind the realists'
fascination with the bundle-of-rights conception was mainly political. They
sought to undermine the notion that property is a natural right, and thereby
smooth the way for activist state intervention in regulating and redistributing
property. If property has no fixed core of meaning, but is just a variable
collection of interests established by social convention, then there is no good
reason why the state should not freely expand or, better yet, contract the list
of interests in the name of the general welfare. The realist program of
dethroning property was on the whole quite successful. The conception of
property as an infinitely variable collection of rights, powers, and duties has
today become a kind of orthodoxy. Not coincidentally, state intervention in
economic matters greatly increased in the middle decades of the twentieth
century, and the constitutional rights of property owners generally receded.

