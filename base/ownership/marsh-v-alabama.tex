\reading{Marsh v. Alabama}
\readingcite{326 U.S. 501 (1946)}

\opinion Mr. Justice \textsc{Black} delivered the opinion of the Court.

In this case we are asked to decide whether a State, consistently with the First
and Fourteenth Amendments, can impose criminal punishment on a person who
undertakes to distribute religious literature on the premises of a
company-owned town contrary to the wishes of the town's management. The town, a
suburb of Mobile, Alabama, known as Chickasaw, is owned by the Gulf
Shipbuilding Corporation. Except for that it has all the characteristics of any
other American town. The property consists of residential buildings, streets, a
system of sewers, a sewage disposal plant and a ``business block'' on which
business places are situated. A deputy of the Mobile County Sheriff, paid by
the company, serves as the town's policeman. Merchants and service
establishments have rented the stores and business places on the business block
and the United States uses one of the places as a post office from which six
carriers deliver mail to the people of Chickasaw and the adjacent area. The
town and the surrounding neighborhood, which can not be distinguished from the
Gulf property by anyone not familiar with the property lines, are thickly
settled, and according to all indications the residents use the business block
as their regular shopping center. To do so, they now, as they have for many
years, make use of a company-owned paved street and sidewalk located alongside
the store fronts in order to enter and leave the stores and the post office.
Intersecting company-owned roads at each end of the business block lead into a
four-lane public highway which runs parallel to the business block at a
distance of thirty feet. There is nothing to stop highway traffic from coming
onto the business block and upon arrival a traveler may make free use of the
facilities available there. In short the town and its shopping district are
accessible to and freely used by the public in general and there is nothing to
distinguish them from any other town and shopping center except the fact that
the title to the property belongs to a private corporation.

Appellant, a Jehovah's Witness, came onto the sidewalk we have just described,
stood near the post-office and undertook to distribute religious literature. In
the stores the corporation had posted a notice which read as follows: ``This Is
Private Property, and Without Written Permission, No Street, or House Vendor,
Agent or Solicitation of Any Kind Will Be Permitted.''  Appellant was warned
that she could not distribute the literature without a permit and told that no
permit would be issued to her. She protested that the company rule could not be
constitutionally applied so as to prohibit her from distributing religious
writings. When she was asked to leave the sidewalk and Chickasaw she declined.
The deputy sheriff arrested her and she was charged in the state court with
violating Title 14, Section 426 of the 1940 Alabama Code which makes it a crime
to enter or remain on the premises of another after having been warned not to
do so. Appellant contended that to construe the state statute as applicable to
her activities would abridge her right to freedom of press and religion
contrary to the First and Fourteenth Amendments to the Constitution. This
contention was rejected and she was convicted. The Alabama Court of Appeals
affirmed the conviction, holding that the statute as applied was constitutional
because the title to the sidewalk was in the corporation and because the public
use of the sidewalk had not been such as to give rise to a presumption under
Alabama law of its irrevocable dedication to the public. The State Supreme
Court denied certiorari, and the case is here on appeal\ldots.

Had the title to Chickasaw belonged not to a private but to a municipal
corporation and had appellant been arrested for violating a municipal ordinance
rather than a ruling by those appointed by the corporation to manage a
company-town it would have been clear that appellant's conviction must be
reversed.\ldots[N]either a state nor a municipality can completely bar the
distribution of literature containing religious or political ideas on its
streets, sidewalks and public places or make the right to distribute dependent
on a flat license tax or permit to be issued by an official who could deny it
at will. We have also held that an ordinance completely prohibiting the
dissemination of ideas on the city streets can not be justified on the ground
that the municipality holds legal title to them. And we have recognized that
the preservation of a free society is so far dependent upon the right of each
individual citizen to receive such literature as he himself might desire that a
municipality could not without jeopardizing that vital individual freedom,
prohibit door to door distribution of literature. From these decisions it is
clear that had the people of Chickasaw owned all the homes, and all the stores,
and all the streets, and all the sidewalks, all those owners together could not
have set up a municipal government with sufficient power to pass an ordinance
completely barring the distribution of religious literature.  Our question then
narrows down to this: Can those people who live in or come to Chickasaw be
denied freedom of press and religion simply because a single company has legal
title to all the town?   For it is the state's contention that the mere fact
that all the property interests in the town are held by a single company is
enough to give that company power, enforceable by a state statute, to abridge
these freedoms.

We do not agree that the corporation's property interests settle the question.
The State urges in effect that the corporation's right to control the
inhabitants of Chickasaw is coextensive with the right of a homeowner to
regulate the conduct of his guests. We can not accept that contention.
Ownership does not always mean absolute dominion. The more an owner, for his
advantage, opens up his property for use by the public in general, the more do
his rights become circumscribed by the statutory and constitutional rights of
those who use it. Thus, the owners of privately held bridges, ferries,
turnpikes and railroads may not operate them as freely as a farmer does his
farm. Since these facilities are built and operated primarily to benefit the
public and since their operation is essentially a public function, it is
subject to state regulation\ldots.

Whether a corporation or a municipality owns or possesses the town the public in
either case has an identical interest in the functioning of the community in
such manner that the channels of communication remain free. As we have
heretofore stated, the town of Chickasaw does not function differently from any
other town. The ``business block'' serves as the community shopping center and
is
freely accessible and open to the people in the area and those passing through.
The managers appointed by the corporation cannot curtail the liberty of press
and religion of these people consistently with the purposes of the
Constitutional guarantees, and a state statute, as the one here involved, which
enforces such action by criminally punishing those who attempt to distribute
religious literature clearly violates the First and Fourteenth Amendments to
the Constitution.

Many people in the United States live in company-owned towns.  These people,
just as residents of municipalities, are free citizens of their State and
country. Just as all other citizens they must make decisions which affect the
welfare of community and nation. To act as good citizens they must be informed.
In order to enable them to be properly informed their information must be
uncensored. There is no more reason for depriving these people of the liberties
guaranteed by the First and Fourteenth Amendments than there is for
curtailing these freedoms with respect to any other citizen.

When we balance the Constitutional rights of owners of property against those of
the people to enjoy freedom of press and religion, as we must here, we remain
mindful of the fact that the latter occupy a preferred position.  As we have
stated before, the right to exercise the liberties safeguarded by the First
Amendment ``lies at the foundation of free government by free men'' and we must
in all cases ``weigh the circumstances and appraise\ldots the reasons\ldots in
support of the regulation of (those) rights.'' \textit{Schneider v. State}, 308
U.S. 147, 161, 60 S. Ct. 146, 151, 84 L.Ed. 155. In our view the circumstance
that the property rights to the premises where the deprivation of liberty, here
involved, took place, were held by others than the public, is not sufficient to
justify the State's permitting a corporation to govern a community of citizens
so as to restrict their fundamental liberties and the enforcement of such
restraint by the application of a State statute. Insofar as the State has
attempted to impose criminal punishment on appellant for undertaking to
distribute religious literature in a company town, its action cannot stand. The
case is reversed and the cause remanded for further proceedings not
inconsistent with this opinion.

Reversed and remanded.

Mr. Justice \textsc{Jackson} took no part in the consideration or decision of
this case. 

[Concurring opinion of Justice \textsc{Frankfurter} omitted.]

\opinion Mr. Justice \textsc{Reed}, dissenting.

Former decisions of this Court have interpreted generously the Constitutional
rights of people in this Land to exercise freedom of religion, of speech and of
the press.  It has never been held and is not now by this opinion of the Court
that these rights are absolute and unlimited either in respect to the manner or
the place of their exercise.  What the present decision establishes as a
principle is that one may remain on private property against the will of the
owner and contrary to the law of the state so long as the only objection to his
presence is that he is exercising an asserted right to spread there his
religious views.  This is the first case to extend by law the privilege of
religious exercises beyond public places or to private places without the
assent of the owner.

As the rule now announced permits this intrusion, without possibility of
protection of the property by law, and apparently is equally applicable to the
freedom of speech and the press, it seems appropriate to express a dissent to
this, to us, novel Constitutional doctrine. Of course, such principle may
subsequently be restricted by this Court to the precise facts of this
case----that
is to private property in a company town where the owner for his own advantage
has permitted a restricted public use by his licensees and invitees. Such
distinctions are of degree and require new arbitrary lines, judicially drawn,
instead of those hitherto established by legislation and precedent. While the
power of this Court, as the interpreter of the Constitution to determine what
use of real property by the owner makes that property subject, at will, to the
reasonable practice of religious exercises by strangers, cannot be doubted, we
find nothing in the principles of the First Amendment, adopted now into the
Fourteenth, which justifies their application to the facts of this case. 

Both Federal and Alabama law permit, so far as we are aware, company
towns\ldots. These communities may be essential to furnish proper and
convenient living conditions for employees on isolated operations in lumbering,
mining, production of high explosives and large-scale farming. The restrictions
imposed by the owners upon the occupants are sometimes galling to the employees
and may appear unreasonable to outsiders. Unless they fall under the
prohibition of some legal rule, however, they are a matter for adjustment
between owner and licensee, or by appropriate legislation.

Alabama has a statute generally applicable to all privately owned premises. It
is Title 14, Section 426, Alabama Code 1940 which so far as pertinent reads as
follows:
\begin{quote}
Trespass after warning. ---Any person who, without legal cause or
good excuse, enters into the dwelling house or on the premises of another, after
having been warned, within six months preceding, not to do so; or any person,
who, having entered into the dwelling house or on the premises of another
without having been warned within six months not to do so, and fails or refuses,
without legal cause or good excuse, to leave immediately on being ordered or
requested to do so by the person in possession, his agent or representative,
shall, on conviction, be fined not more than one hundred dollars, and may also
be imprisoned in the county jail, or sentenced to hard labor for the county, for
not more than three months.
\end{quote}
Appellant was distributing religious pamphlets on a privately owned passway or
sidewalk thirty feet removed from a public highway of the State of Alabama and
remained on these private premises after an authorized order to get off. We do
not understand from the record that there was objection to appellant's use of
the nearby public highway and under our decisions she could rightfully have
continued her activities a few feet from the spot she insisted upon using. An
owner of property may very well have been willing for the public to use the
private passway for business purposes and yet have been unwilling to furnish
space for street trades or a location for the practice of religious
exhortations by itinerants. The passway here in question was not put to any
different use than other private passways that lead to privately owned areas,
amusement places, resort hotels or other businesses\ldots.

A state does have the moral duty of furnishing the opportunity for information,
education and religious enlightenment to its inhabitants, including those who
live in company towns, but it has not heretofore been adjudged that it must
commandeer, without compensation, the private property of other citizens to
carry out that obligation.\ldots  In the area which is covered by the
guarantees of the First Amendment, this Court has been careful to point out
that the owner of property may protect himself against the intrusion of
strangers. Although in \textit{Martin v. Struthers}, 319 U.S. 141, 63 S.Ct.
862, 87 L.Ed. 1313, an ordinance forbidding the summonsing of the occupants of
a dwelling to receive handbills was held invalid because in conflict with the
freedom of speech and press, this Court pointed out \ldots  that after warning
the property owner would be protected from annoyance.  The very Alabama statute
which is now held powerless to protect the property of the Gulf Shipbuilding
Corporation, after notice, from this trespass was there cited\ldots to show
that it would protect the householder, after notice\ldots.

Our Constitution guarantees to every man the right to express his views in an
orderly fashion. An essential element of ``orderly'' is that the man shall also
have a right to use the place he chooses for his exposition. The rights of the
owner, which the Constitution protects as well as the right of free speech, are
not outweighed by the interests of the trespasser, even though he trespasses in
behalf of religion or free speech. We cannot say that Jehovah's Witnesses can
claim the privilege of a license, which has never been granted, to hold their
meetings in other private places, merely because the owner has admitted the
public to them for other limited purposes. Even though we have reached the
point where this Court is required to force private owners to open their
property for the practice there of religious activities or propaganda
distasteful to the owner, because of the public interest in freedom of
speech and religion, there is no need for the application of such a doctrine
here. Appellant, as we have said, was free to engage in such practices on the
public highways, without becoming a trespasser on the company's property.

The \textsc{Chief Justice} and Mr. Justice \textsc{Burton} join in this dissent.


