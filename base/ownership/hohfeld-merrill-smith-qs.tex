\expected{hohfeld-fundamental}
\expected{merrill-smith-economics}

\item Note that Hohfeld's decomposition of \textit{in rem} rights into a
collection of \textit{in personam} rights could provide a new interpretation of
the ``bundle of rights'' metaphor. Rather than being a collection of different
rights held by one person with respect to a thing (the right to exclude,
possess, alienate, etc.), perhaps the ``bundle'' really is a reference to the
various rights an owner has against the ``large and indefinite class of
people'' with whom she might come into conflict with respect to
the \textit{res}. Does this distinction matter?
\having{jacque-v-steenberg}{Which sense of the metaphor do you think is being
used in \textit{Jacque}? }{}{}Which do you think is being used by
Merrill and Smith?

\item
\having{jacque-v-steenberg}{Recall the questions in Notes \ref{jacque-hypos}
and
\ref{jacque-money-hypos} on page \pageref{jacque-hypos} (following
\textit{Jacque}). They may lead us to}{Here is}{Here is}
another way of framing the distinction
between the two interpretations of the ``bundle'' metaphor. Consider this: if I
ask you: ``Does A have a property right in Whiteacre,'' how confident are you
that you will be able to answer the question without knowing the answer to a
different question: ``A right against whom?''

\item Are you persuaded by Merrill's and Smith's critique of Hohfeld? Is their
model of \textit{in rem} rights compatible with Hohfeld's analysis, or are the
two necessarily inconsistent with each other?

\item Consider the following two propositions: 
\begin{itemize}
\item ``Property'' is a relationship between a person and a thing.
\item ``Property'' is a set of rights and obligations among people with
respect to things.
\end{itemize}
Do you think either of these propositions adequately describes what we mean by
the word ``property''? Do you think these two propositions are meaningfully
different from one another? If so, what is the difference? Do you think the
difference might have an effect on the outcome of legal disputes? If so, what
effect? And if not, does the difference matter?

\item Are you persuaded by Merrill's and Smith's claim that treating property as
an \textit{in rem} right makes it more resistant to interference and
degradation by the state? What feature(s) of their \textit{in rem} conception
might give rise to this resistance? If rejection of the \textit{in rem}
conception and weakening of private property rights have in fact gone hand in
hand, which account do you find more plausible: that lawyers' and scholars'
rejection of the \textit{in rem} conception of property facilitated increased
state interference with property rights, or that state interference with
property rights rendered the \textit{in rem} conception untenable? Put another
way, do you understand Merrill and Smith to be making an argument about what
property \textit{is} (or \textit{was}), or about what it \textit{should be}? If
the latter, do you agree? Why or why not?

\item Hohfeld observes that, when it comes to property rights, ``thing'' doesn't
necessarily mean ``tangible thing in the physical world.'' Indeed, legal
authorities identify property rights in all sorts of intangible things, as well
as in admittedly physical substances that resist the label of ``thing''---like
animals, or even human beings.\having{intro-subject-matter}{ We discussed this
complication of the notion of property as a legal right in ``things'' in
\mref{intro-subject-matter}.}{ We will discuss this
complication of the notion of property as a legal right in ``things'' in
\mref{intro-subject-matter}.}{}

