\reading{Jacque v. Steenberg Homes, Inc.}
\readingcite{563 N.W.2d 154 (Wis. 1997)}

\opinion \textsc{William A. Bablitch}, Justice.

Plaintiffs, Lois and Harvey Jacques, are an elderly couple, now retired from
farming, who own roughly 170 acres near Wilke's Lake in the town of Schleswig.
The defendant, Steenberg Homes, Inc. (Steenberg), is in the business of selling
mobile homes. In the fall of 1993, a neighbor of the Jacques purchased a mobile
home from Steenberg. Delivery of the mobile home was included in the sales
price.

Steenberg determined that the easiest route to deliver the mobile home was
across the Jacques' land \ldots because the only alternative was a private
road which was covered in up to seven feet of snow and contained a sharp curve
which would require sets of ``rollers'' to be used when maneuvering the home
around the curve. Steenberg asked the Jacques on several separate occasions
whether it could move the home across the Jacques' farm field. The Jacques
refused.\ldots On the morning of delivery, \ldots the assistant manager
asked Mr. Jacque how much money it would take to get permission. Mr. Jacque
responded that it was not a question of money; the Jacques just did not want
Steenberg to cross their land.\ldots

At trial, one of Steenberg's employees testified that, upon coming out of the
Jacques' home, the assistant manager stated: ``I don't give a -{}-{}-{}- what
[Mr. Jacque] said, just get the home in there any way you can.'' \ldots The
employees, after beginning down the private road, ultimately used a ``bobcat''
to cut a path through the Jacques' snow-covered field and hauled the home
across the Jacques' land to the neighbor's lot.\ldots Mr. Jacque called the
Manitowoc County Sheriff's Department. After interviewing the parties and
observing the scene, an officer from the sheriff's department issued a \$30
citation to Steenberg's assistant manager.

The Jacques commenced an intentional tort action in Manitowoc County Circuit
Court, Judge Allan J. Deehr presiding, seeking compensatory and punitive
damages from Steenberg.\ldots[Q]uestions of punitive and compensatory damages
were submitted to the jury. The jury awarded the Jacques \$1 nominal damages
and \$100,000 punitive damages. Steenberg filed post-verdict motions claiming
that the punitive damage award must be set aside because Wisconsin law did not
allow a punitive damage award unless the jury also awarded compensatory
damages. Alternatively, Steenberg asked the circuit court to remit the punitive
damage award. The circuit court granted Steenberg's motion to set aside the
award. Consequently, it did not reach Steenberg's motion for remittitur\ldots.

\readinghead{II.}

\ldots Steenberg argues that, as a matter of law, punitive damages could not be
awarded by the jury because punitive damages must be supported by an award of
compensatory damages and here the jury awarded only nominal and punitive
damages. The Jacques contend that the rationale supporting the compensatory
damage award requirement is inapposite when the wrongful act is an intentional
trespass to land. We agree with the Jacques.

\ldots The rationale for the compensatory damage requirement is that if the
individual cannot show actual harm, he or she has but a nominal interest,
hence, society has little interest in having the unlawful, but otherwise
harmless, conduct deterred, therefore, punitive damages are inappropriate.
\ldots The Jacques argue that both the individual and society have significant
interests in deterring intentional trespass to land, regardless of the lack of
measurable harm that results. We agree with the Jacques\ldots.

We turn first to the individual landowner's interest in protecting his or her
land from trespass. The United States Supreme Court has recognized that the
private landowner's right to exclude others from his or her land is ``one of
the most essential sticks in the bundle of rights that are commonly
characterized as property.'' \textit{Dolan v. City of Tigard}, 512 U.S. 374,
384, 114 S.Ct. 2309, 2316, 129 L.Ed.2d 304 (1994). This court has long
recognized ``[e]very person['s] constitutional right to the exclusive enjoyment
of his own property for any purpose which does not invade the rights of another
person.'' \textit{Diana Shooting Club v. Lamoreux}, 114 Wis. 44, 59, 89 N.W.
880 (1902) (holding that the victim of an intentional trespass should have been
allowed to take judgment for nominal damages and costs). Thus, both this court
and the Supreme Court recognize the individual's legal right to exclude others
from private property.

Yet a right is hollow if the legal system provides insufficient means to protect
it. Felix Cohen offers the following analysis summarizing the relationship
between the individual and the state regarding property rights:

[T]hat is property to which the following label can be attached:

\begin{quotation}
To the world:

Keep off X unless you have my permission, which I may grant or withhold.

Signed: Private Citizen

Endorsed: The state
\end{quotation}

Felix S. Cohen, \textit{Dialogue on Private Property,} IX Rutgers Law Review
357, 374 (1954). Harvey and Lois Jacque have the right to tell Steenberg Homes
and any other trespasser, ``No, you cannot cross our land.'' But that right has
no practical meaning unless protected by the State\ldots.

The nature of the nominal damage award in an intentional trespass to land case
further supports an exception to [the compensatory damage
requirement]. Because a legal right is involved, the law recognizes
that actual harm occurs in every trespass. The action for intentional trespass
to land is directed at vindication of the legal right.\ldots Thus, in the
case of intentional trespass to land, the nominal damage award represents the
recognition that, although immeasurable in mere dollars, actual harm has
occurred.

The potential for harm resulting from intentional trespass also supports an
exception to [the compensatory damage requirement]. A series of
intentional trespasses, as the Jacques had the misfortune to discover in an
unrelated action, can threaten the individual's very ownership of the land. The
conduct of an intentional trespasser, if repeated, might ripen into
prescription or adverse possession and, as a consequence, the individual
landowner can lose his or her property rights to the trespasser. 

In sum, the individual has a strong interest in excluding trespassers from his
or her land. Although only nominal damages were awarded to the Jacques,
Steenberg's intentional trespass caused actual harm. We turn next to society's
interest in protecting private property from the intentional trespasser.

 Society has an interest in punishing and deterring intentional trespassers
beyond that of protecting the interests of the individual landowner. Society
has an interest in preserving the integrity of the legal system. Private
landowners should feel confident that wrongdoers who trespass upon their land
will be appropriately punished. When landowners have confidence in the legal
system, they are less likely to resort to ``self-help'' remedies.\ldots [O]ne
can easily imagine a frustrated landowner taking the law into his or her own
hands when faced with a brazen trespasser, like Steenberg, who refuses to heed
no trespass warnings.

People expect wrongdoers to be appropriately punished. Punitive damages have
the effect of bringing to punishment types of conduct that, though oppressive
and hurtful to the individual, almost invariably go unpunished by the public
prosecutor.\ldots If punitive damages are not allowed in a situation like
this, what punishment will prohibit the intentional trespass to land? Moreover,
what is to stop Steenberg Homes from concluding, in the future, that delivering
its mobile homes via an intentional trespass and paying the resulting [\$30]
forfeiture, is not more profitable than obeying the law? Steenberg Homes plowed
a path across the Jacques' land and dragged the mobile home across that path,
in the face of the Jacques' adamant refusal. A \$30 forfeiture and a \$1
nominal damage award are unlikely to restrain Steenberg Homes from similar
conduct in the future. An appropriate punitive damage award probably will.

In sum, as the court of appeals noted, the [compensatory damage] rule sends the
wrong message to Steenberg Homes and any others who contemplate trespassing on
the land of another. It implicitly tells them that they are free to go where
they please, regardless of the landowner's wishes. As long as they cause no
compensable harm, the only deterrent intentional trespassers face is the
nominal damage award of \$1 \ldots and the possibility of a Class B forfeiture
under Wis. Stat. {\S} 943.13. We conclude that both the private landowner and
society have much more than a nominal interest in excluding others from private
land. Intentional trespass to land causes actual harm to the individual,
regardless of whether that harm can be measured in mere dollars. Consequently,
the [compensatory damage] rationale will not support a refusal to allow
punitive damages when the tort involved is an intentional trespass to land.
Accordingly, assuming that the other requirements for punitive damages have
been met, we hold that nominal damages may support a punitive damage award in
an action for intentional trespass to land.\ldots Accordingly, we reverse and
remand to the circuit court for reinstatement of the punitive damage award.

Reversed and remanded with directions.

