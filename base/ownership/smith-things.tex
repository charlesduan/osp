\readingnote{Reproduced by permission of Henry E. Smith.}
\reading{Henry E. Smith,
\textit{Property as the Law of Things}}
\readingcite{125 \textsc{Harv. L. Rev.} 1691, 1696-98, 1700-08 (2012)}

As an analytical device, the bundle picture can be very useful. It provides a
highly accurate description of who can do what to whom in a legal (and perhaps
nonlegal) sense. It provides an interesting theoretical baseline: how would one
describe the relation of a property owner to various others if one were writing
on a blank slate and doing the description in a fully bottom-up manner,
relation by relation, party by party? In this, the Hohfeldian world is a little
like the Coasean world of zero transaction costs---a useful theoretical
construct.

The resemblance is no accident. Like the zero-transaction-cost world, no
property system ever has or will build up legal relations smallest piece by
smallest piece. Interestingly, in a zero-transaction cost world, one could do
just that, and any benefit to be secured by parsing out relations in a
fine-grained manner could be obtained at zero cost. That is not our world.

The problem with the bundle of rights is that it is treated as a theory of how
our world works rather than as an analytical device or as a theoretical
baseline. In the realist era, the benefits of tinkering with property were
expressed in bundle terms without a corresponding theory of the costs of that
tinkering. Indeed, in the most tendentious versions of the picture, the
traditional baselines of the law were mocked, and the idea was to dethrone them
in order to remove them as barriers to enlightened social engineering. In this
version of the bundle picture, Hohfeldian sticks and potentially others are
posited to describe the relations holding between persons; the fact that the
relations hold with respect to a thing is relatively unimportant or, in some
versions, of no importance. ``Property'' is simply a conclusory label we might
attach to the collection. In its classic formulation, the bundle picture puts
no particular constraints on the contents of bundles: they are totally
malleable and should respond to policy concerns in a fairly direct fashion.
These policy-motivated adjustments usually involve adding or subtracting sticks
and reallocating them among concerned parties or to society. This version of
the bundle explains everything and so explains nothing.

\ldots In recent times, various commentators have argued that property is not
fully captured by the bundle picture. Going beyond the bundle usually involves
emphasizing exclusion or some robust notion of the right to use. It can be
motivated by analytical jurisprudence, natural rights, or information cost
economics. The bundle theory can incorporate some of these perspectives.
Consider, for example, the recent resurgence of interest in the \textit{numerus
clausus;} this principle that property forms come in a finite and closed menu
can be added onto the bundle theory as a ``menu'' of collections of sticks.
Bundle theorists can accommodate this development. But they are being reactive
in this regard\ldots .

In this Article, I present a theory that aims higher. At the most basic level,
the extreme bundle picture takes too little account of the costs of delineating
rights\ldots .

\ldots Here, I present an alternative to the bundle picture that I call an
\textit{architectural} or \textit{modular} theory of property. This theory
responds to information costs---it conceives of property as a law of modular
``things.''\ldots

Because it makes sense in modern property systems to delegate to owners a choice
from a range of uses and because protection allows for stability,
appropriability, facilitation of planning and investment, liberty, and
autonomy, we typically start with an \textit{exclusion} strategy---and that
goes not just for private property but for common and public property as well.
``Use'' can include nonconsumptive uses relating to conservation. The exclusion
strategy defines a chunk of the world---a thing---under the owner's control,
and much of the information about the thing's uses, their interactions, and the
user is irrelevant to the outside world. Duty bearers know not to enter
Blackacre without permission or not to take cars, without needing to know what
the owner is using the thing for, who the owner is, who else might have rights
and other interests, and so on. But dividing the world into chunks is not
enough: spillovers and scale problems call for more specific rules to deal with
problems like odors and lateral support, and to facilitate coordination (for
example, covenants, common interest communities, and trusts). These
\textit{governance} strategies focus more closely on narrower classes of use
and sometimes make more specific reference to their purposes, and so they are
more contextual. 

The exclusion-governance architecture manages complexity in a way totally
uncaptured by the bundle picture, and importantly, the former is modular while
the latter is not. The exclusion strategy defines what a thing is to begin
with. A fundamental question is how to classify ``things,'' and, hence, which
aspects of ``things'' are the most basic units of property law. Many important
features of property follow from the semitransparent boundaries between things.
Boundaries carve up the world into semiautonomous components---modules---that
permit private law to manage highly complex interactions among private
parties.\ldots

The modular theory explains property's structure, which includes providing some
reason why those structures are not otherwise. In a zero-transaction-cost
world, we could use all governance all the time, whether supplied by government
or through super fine-grained contracting among all the concerned parties. That
is not our world, and the main point of exclusion as a delineation strategy is
that it is a \textit{shortcut} over direct delineation of this more
``complete'' set of legal relations. Analytically, it might be interesting to
think of property as a list of use rights availing pairwise between all people
in society, but actually creating such a list would be a potentially
intractable problem in our world. On the other hand, exclusion is not the whole
story either. Causes of action like trespass implement a right to exclude, but
the right to exclude is not \textit{why} we have property. Rather, the right to
exclude is part of \textit{how} property works. Rights to exclude are a means
to an end, and the ends in property relate to people's interests in using
things.

\ldots Exclusion is at the core of this architecture because it is a default, a
convenient starting point. Exclusion is not the most important or ``core''
value because it is \textit{not a value at all.} Thinking that exclusion is a
value usually reflects the confusion of means and ends in property law:
exclusion is a rough first cut---and only that---at serving the purposes of
property. It is true that exclusion piggybacks on the everyday morality of
``thou shalt not steal,'' whereas governance reflects a more refined
Golden-Rule, ``do unto others'' type of morality in more personal contexts. It
may be the case that our morality itself is shaped to a certain extent by the
ease with which it can be communicated and enforced in more impersonal
settings. I leave that question for another day. But the point here is that the
exclusion-governance architecture is compatible with a wide range of purposes
for property. Some societies will move from exclusion to governance---that is,
some systems of laws and norms will focus more on individuated uses of
resources---more readily than others, and will do so for different reasons than
others.

At the base of the architectural approach is a distinction that the bundle
theory---along with other theories---tends to obscure: the distinction between
the interests we have in using things and the devices the law uses to protect
those interests. Property serves purposes related to use by employing a variety
of delineation strategies. Because delineation costs are greater than zero,
which strategy one uses and when one uses it will be dictated in part by the
costs of delineation---not just by the benefits that correspond to the
use-based purposes of property\ldots .

The traditional definition of property is a right to a thing good against the
world---it is an in rem right. The special in rem character of property forms
the basis of an information-cost explanation of the \textit{numerus clausus
}and standardization in property. In rem rights are directed at a wide and
indefinite audience of duty holders and other affected parties, who would incur
high information costs in dealing with idiosyncratic property rights and would
have to process more types of information than they would in the absence of the
\textit{numerus clausus.} Crucially, parties who might create such
idiosyncratic property rights are not guaranteed to take such third-party
processing costs into account. There is thus an information-cost externality,
and the \textit{numerus clausus} is one tool for addressing this externality.
Other devices include title records and technological changes in
communication.\ldots

Modularity plays a key role in making the standardization of property possible.
First, modularity makes it possible to keep interconnections between packages
of rights relatively few, thus allowing much of what goes on inside a package
of property rights to be irrelevant to the outside world. Second, property
rights ``mesh'' with neighboring property rights and show network effects with
more far-flung property rights. The outside interfaces make this possible at
reasonable cost. Third, the processes of property are simple enough that they
\textit{can feed into themselves.} Many modular structures are hierarchical in
that they have modules composed of other modules\ldots . In this respect,
property forms are like a basic grammar or ``pattern language'' of property.

