\reading[Blackstone, \emph{Commentaries on the Laws of England}]{William
Blackstone, \textit{Commentaries on the Laws of England}}
\readingcite{Vol. 1, pp. 131-136 (1765); vol. 2, p. 2}

THE third absolute right, inherent in every Englishman, is that of property:
which consists in the free use, enjoyment, and disposal of all his
acquisitions, without any control or diminution, save only by the laws of the
land. The original of private property is probably founded in nature, as will
be more fully explained in the second book of the ensuing commentaries: but
certainly the modifications under which we at present find it, the method of
conserving it in the present owner, and of translating it from man to man, are
entirely derived from society; and are some of those civil advantages, in
exchange for which every individual has resigned a part of his natural liberty.
The laws of England are therefore, in point of honor and justice, extremely
watchful in ascertaining and protecting this right. Upon this principle the
great charter has declared that no freeman shall be disseised, or divested, of
his freehold, or of his liberties, or free customs, but by the judgment of his
peers, or by the law of the land\ldots.

\captionedgraphic{blackstone}{William Blackstone. Source:
6 \textbf{Cassell's Illustrated History of England} 582 (1865),
\protect\url{https://archive.org/stream/cassellsillustra06lond\#page/582/mode/2up}.}

SO great moreover is the regard of the law for private property, that it will
not authorize the least violation of it; no, not even for the general good of
the whole community. If a new road, for instance, were to be made through the
grounds of a private person, it might perhaps be extensively beneficial to the
public; but the law permits no man, or set of men, to do this without consent
of the owner of the land. In vain may it be urged, that the good of the
individual ought to yield to that of the community; for it would be dangerous
to allow any private man, or even any public tribunal, to be the judge of this
common good, and to decide whether it be expedient or no. Besides, the public
good is in nothing more essentially interested, than in the protection of every
individual's private rights, as modelled by the municipal law. In this, and
similar cases the legislature alone, can, and indeed frequently does,
interpose, and compel the individual to acquiesce. But how does it interpose
and compel? Not by absolutely stripping the subject of his property in an
arbitrary manner; but by giving him a full indemnification and equivalent for
the injury thereby sustained. The public is now considered as an individual,
treating with an individual for an exchange. All that the legislature does is
to oblige the owner to alienate his possessions for a reasonable price; and
even this is an exertion of power, which the legislature indulges with caution,
and which nothing but the legislature can perform.\ldots

There is nothing which so generally strikes the imagination, and engages the
affections of mankind, as the right of property; or that sole and despotic
dominion which one man claims and exercises over the external things of the
world, in total exclusion of the right of any other individual in the universe.

