\reading{Americans with Disabilities Act of 1990}
\readingcite{Codified at 42 U.S.C. {\S} 12182-83}

\textbf{{\S} 302 --- Prohibition of discrimination by public accommodations}

(a) \textsc{General rule}.---No individual shall be discriminated against on the
basis of disability in the
full and equal enjoyment of the goods, services, facilities, privileges,
advantages, or accommodations of any place of public accommodation by any
person who owns, leases (or leases to), or operates a place of public
accommodation.

\textbf{{\S} 303 --- New construction and alterations in public accommodations
and commercial facilities}

(a) \textsc{Application of term}.---Except as provided in subsection (b) of this
section, as applied to public
accommodations and commercial facilities, discrimination for purposes of
section 12182(a) of this title includes---
\begin{statute}
\item (1) a failure to design and construct facilities for first occupancy later
than
30 months after July 26, 1990, that are readily accessible to and usable by
individuals with disabilities, except where an entity can demonstrate that it
is structurally impracticable to meet the requirements of such subsection . .
.; and

\item (2) . . ., a failure to make alterations in such a manner that, to the
maximum
extent feasible, the altered portions of the facility are readily accessible to
and usable by individuals with disabilities, including individuals who use
wheelchairs.
\end{statute}

(b) \textsc{Elevator}.---Subsection (a) of this section shall not be construed
to require the
installation of an elevator for facilities that are less than three stories or
have less than 3,000 square feet per story unless the building is a shopping
center, a shopping mall, or the professional office of a health care provider
or unless the Attorney General determines that a particular category of such
facilities requires the installation of elevators based on the usage of such
facilities.



