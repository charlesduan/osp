\reading{State of New Jersey v. Shack}
\readingcite{58 N.J. 297, 277 A.2d 369 (1971)}

\opinion \textsc{Weintraub}, C.J.

Defendants entered upon private property to aid migrant farmworkers employed and
housed there. Having refused to depart upon the demand of the owner, defendants
were charged with violating N.J.S.A. 2A:170---31 which provides that ``[a]ny
person who trespasses on any lands\ldots after being forbidden so to trespass
by the owner\ldots is a disorderly person and shall be punished by a fine of
not more than \$50.'' Defendants were convicted in the Municipal Court of
Deerfield Township and again on appeal in the County Court of Cumberland County
on a trial \textit{de novo}. We certified their further appeal before argument
in the Appellate Division.

Before us, no one seeks to sustain these convictions. The complaints were
prosecuted in the Municipal Court and in the County Court by counsel engaged by
the complaining landowner, Tedesco. However Tedesco did not respond to this
appeal, and the county prosecutor, while defending abstractly the
constitutionality of the trespass statute, expressly disclaimed any position as
to whether the statute reached the activity of these defendants.

Complainant, Tedesco, a farmer, employs migrant workers for his seasonal needs.
As part of their compensation, these workers are housed at a camp on his
property.

Defendant Tejeras is a field worker for the Farm Workers Division of the
Southwest Citizens Organization for Poverty Elimination, known by the acronym
SCOPE, a nonprofit corporation funded by the Office of Economic Opportunity
pursuant to an act of Congress, 42 U.S.C. \S\S~2861--2864. The role of
SCOPE includes providing for the ``health services of the migrant farm
worker.''

Defendant Shack is a staff attorney with the Farm Workers Division of Camden
Regional Legal Services, Inc., known as ``CRLS,'' also a nonprofit corporation
funded by the Office of Economic Opportunity pursuant to an act of Congress, 42
U.S.C.A. \S~2809(a)(3). The mission of CRLS includes legal advice and
representation for these workers.

Differences had developed between Tedesco and these defendants prior to the
events which led to the trespass charges now before us. Hence when defendant
Tejeras wanted to go upon Tedesco's farm to find a migrant worker who needed
medical aid for the removal of 28 sutures, he called upon defendant Shack for
his help with respect to the legalities involved. Shack, too, had a mission to
perform on Tedesco's farm; he wanted to discuss a legal problem with another
migrant worker there employed and housed. Defendants arranged to go to the farm
together. Shack carried literature to inform the migrant farmworkers of the
assistance available to them under federal statutes, but no mention seems to
have been made of that literature when Shack was later confronted by Tedesco.

Defendants entered upon Tedesco's property and as they neared the camp site
where the farmworkers were housed, they were confronted by Tedesco who inquired
of their purpose. Tejeras and Shack stated their missions. In response, Tedesco
offered to find the injured worker, and as to the worker who needed legal
advice, Tedesco also offered to locate the man but insisted that the
consultation would have to take place in Tedesco's office and in his presence.
Defendants declined, saying they had the right to see the men in the privacy of
their living quarters and without Tedesco's supervsion. Tedesco thereupon
summoned a State Trooper who, however, refused to remove defendants except upon
Tedesco's written complaint. Tedesco then executed the formal complaints
charging violations of the trespass statute.

\readinghead{I.}

The constitutionality of the trespass statute, as applied here, is challenged on
several scores.

It is urged that the First Amendment rights of the defendants and of the migrant
farmworkers were thereby offended. Reliance is placed on \emph{Marsh v.
Alabama}, 326
U.S. 501, 66 S.Ct. 276, 90 L.Ed. 265 (1946) [and its progeny.] Those cases rest
upon the fact that the property was in fact opened to the general public. There
may be some migrant camps with the attributes of the company town in
\emph{Marsh} and
of course they would come within its holding. But there is nothing of that
character in the case before us, and hence there would have to be an extension
of \emph{Marsh} to embrace the immediate situation.

Defendants also maintain that the application of the trespass statute to them is
barred by the Supremacy Clause of the United States Constitution, Art. VI, cl.
2, and this on the premise that the application of the trespass statute would
defeat the purpose of the federal statutes, under which SCOPE and CRLS are
funded, to reach and aid the migrant farmworker.\ldots

These constitutional claims are not established by any definitive holding. We
think it unnecessary to explore their validity. The reason is that we are
satisfied that under our State law the ownership of real property does not
include the right to bar access to governmental services available to migrant
workers and hence there was no trespass within the meaning of the penal
statute. The policy considerations which underlie that conclusion may be much
the same as those which would be weighed with respect to one or more of the
constitutional challenges, but a decision in nonconstitutional terms is more
satisfactory, because the interests of migrant workers are more expansively
served in that way than they would be if they had no more freedom than these
constitutional concepts could be found to mandate if indeed they apply at all.

\readinghead{II.}

Property rights serve human values. They are recognized to that end, and are
limited by it. Title to real property cannot include dominion over the destiny
of persons the owner permits to come upon the premises. Their well-being must
remain the paramount concern of a system of law. Indeed the needs of the
occupants may be so imperative and their strength so weak, that the law will
deny the occupants the power to contract away what is deemed essential to their
health, welfare, or dignity.

Here we are concerned with a highly disadvantaged segment of our society. We are
told that every year farmworkers and their families numbering more than one
million leave their home areas to fill the seasonal demand for farm labor in
the United States. The migrant farmworkers come to New Jersey in substantial
numbers.\ldots The migrant farmworkers are a community within but apart from
the local scene. They are rootless and isolated. Although the need for their
labors is evident, they are unorganized and without economic or political
power. It is their plight alone that summoned government to their aid. In
response, Congress provided under Title III---B of the Economic Opportunity Act
of 1964 (42 U.S.C.A. {\S} 2701 et seq.) for ``assistance for migrant and other
seasonally employed farmworkers and their families.'' \ldots As we have said,
SCOPE is engaged in a program funded under this section, and CRLS also pursues
the objectives of this section although, we gather, it is funded under s
2809(a)(3), which is not limited in its concern to the migrant and other
seasonally employed farmworkers and seeks ``to further the cause of justice
among persons living in poverty by mobilizing the assistance of lawyers and
legal institutions and by providing legal advice, legal representation,
counseling, education, and other appropriate services.''

These ends would not be gained if the intended beneficiaries could be insulated
from efforts to reach them. It is in this framework that we must decide whether
the camp operator's rights in his lands may stand between the migrant workers
and those who would aid them.\ldots

A man's right in his real property of course is not absolute. It was a maxim of
the common law that one should so use his property as not to injure the rights
of others. Broom, Legal Maxims (10th ed. Kersley 1939), p. 238; 39 Words and
Phrases, ``Sic Utere Tuo ut Alienum Non Laedas,'' p. 335. Although hardly a
precise solvent of actual controversies, the maxim does express the inevitable
proposition that rights are relative and there must be an accommodation when
they meet. Hence it has long been true that necessity, private or public, may
justify entry upon the lands of another\ldots.

We see no profit in trying to decide upon a conventional category and then
forcing the present subject into it. That approach would be artificial and
distorting. The quest is for a fair adjustment of the competing needs of the
parties, in the light of the realities of the relationship between the migrant
worker and the operator of the housing facility.

Thus approaching the case, we find it unthinkable that the farmer-employer can
assert a right to isolate the migrant worker in any respect significant for the
worker's well-being. The farmer, of course, is entitled to pursue his farming
activities without interference, and this defendants readily concede. But we
see no legitimate need for a right in the farmer to deny the worker the
opportunity for aid available from federal, State, or local services, or from
recognized charitable groups seeking to assist him. Hence representatives of
these agencies and organizations may enter upon the premises to seek out the
worker at his living quarters. So, too, the migrant worker must be allowed to
receive visitors there of his own choice, so long as there is no behavior
hurtful to others, and members of the press may not be denied reasonable access
to workers who do not object to seeing them.

It is not our purpose to open the employer's premises to the general public if
in fact the employer himself has not done so. We do not say, for example, that
solicitors or peddlers of all kinds may enter on their own; we may assume or
the present that the employer may regulate their entry or bar them, at least if
the employer's purpose is not to gain a commercial advantage for himself or if
the regulation does not deprive the migrant worker of practical access to
things he needs.

And we are mindful of the employer's interest in his own and in his employees'
security. Hence he may reasonably require a visitor to identify himself, and
also to state his general purpose if the migrant worker has not already
informed him that the visitor is expected. But the employer may not deny the
worker his privacy or interfere with his opportunity to live with dignity and
to enjoy associations customary among our citizens. These rights are too
fundamental to be denied on the basis of an interest in real property and too
fragile to be left to the unequal bargaining strength of the parties.

It follows that defendants here invaded no possessory right of the
farmer-employer. Their conduct was therefore beyond the reach of the trespass
statute. The judgments are accordingly reversed and the matters remanded to the
County Court with directions to enter judgments of acquittal.

