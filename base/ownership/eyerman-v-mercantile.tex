\reading{Eyerman v. Mercantile Trust Co.}
\readingcite{524 S.W.2d 210 (Mo. Ct. App. 1975)}

\opinion \textsc{Rendlen}, Judge.

Plaintiffs appeal from denial of their petition seeking injunction to prevent
demolition of a house at \#4 Kingsbury Place in the City of St. Louis. The
action is brought by individual neighboring property owners and certain
trustees for the Kingsbury Place Subdivision. We reverse.

Louise Woodruff Johnston, owner of the property in question, died January 14,
1973, and by her will directed the executor ``\ldots to cause our home at 4
Kingsbury Place \ldots to be razed and to sell the land upon which it is located
\ldots and to transfer the proceeds of the sale \ldots to the residue of my
estate.'' Plaintiffs assert that razing the home will adversely affect their
property rights, violate the terms of the subdivision trust indenture for
Kingsbury Place, produce an actionable private nuisance and is contrary to
public policy.

The area involved is a ``private place'' established in 1902 by trust indenture
which provides that Kingsbury Place and Kingsbury Terrace will be so
maintained, improved, protected and managed as to be desirable for private
residences. The trustees are empowered to protect and preserve ``Kingsbury
Place'' from encroachment, trespass, nuisance or injury, and it is ``the
intention of these presents, forming a general scheme of improving and
maintaining said property as desirable residence property of the highest
class.'' The covenants run with the land and the indenture empowers lot owners
or the trustees to bring suit to enforce them.

Except for one vacant lot, the subdivision is occupied by handsome, spacious two
and three-story homes, and all must be used exclusively as private residences.
The indenture generally regulates location, costs and similar features for any
structures in the subdivision, and limits construction of subsidiary structures
except those that may beautify the property, for example, private stables,
flower houses, conservatories, play houses or buildings of similar character.

On trial the temporary restraining order was dissolved and all issues found
against the plaintiffs.

\ldots Whether \#4 Kingsbury Place should be razed is an issue of public policy
involving individual property rights and the community at large. The plaintiffs
have pleaded and proved facts sufficient to show a personal, legally
protectible interest.

Demolition of the dwelling will result in an unwarranted loss to this estate,
the plaintiffs and the public. The uncontradicted testimony was that the
current value of the house and land is \$40,000.00; yet the estate could expect
no more than \$5,000.00 for the empty lot, less the cost of demolition at
\$4,350.00, making a grand loss of \$39,350.33 if the unexplained and
capricious direction to the executor is effected. Only \$650.00 of the
\$40,000.00 asset would remain.

Kingsbury Place is an area of high architectural significance, representing
excellence in urban space utilization. Razing the home will depreciate
adjoining property values by an estimated \$10,000.00 and effect corresponding
losses for other neighborhood homes. The cost of constructing a house of
comparable size and architectural exquisiteness would approach \$200,000.00.

\ldots To remove \#4 Kingsbury from the street was described as having the
effect of a missing front tooth. The space created would permit direct access
to Kingsbury Place from the adjacent alley, increasing the likelihood the lot
will be subject to uses detrimental to the health, safety and beauty of the
neighborhood. The mere possibility that a future owner might build a new home
with the inherent architectural significance of the present dwelling offers
little support to sustain the condition for destruction.

We are constrained to take judicial notice of the pressing need of the community
for dwelling units as demonstrated by recent U.S. Census Bureau figures showing
a decrease of more than 14\% in St. Louis City housing units during the decade
of the 60's. This decrease occurs in the face of housing growth in the
remainder of the metropolitan area. It becomes apparent that no individual,
group of individuals nor the community generally benefits from the senseless
destruction of the house; instead, all are harmed and only the caprice of the
dead testatrix is served. Destruction of the house harms the neighbors,
detrimentally affects the community, causes monetary loss in excess of
\$39,000.00 to the estate and is without benefit to the dead woman. No reason,
good or bad, is suggested by the will or record for the eccentric condition.
This is not a living person who seeks to exercise a right to reshape or dispose
of her property; instead, it is an attempt by will to confer the power to
destroy upon an executor who is given no other interest in the property. To
allow an executor to exercise such power stemming from apparent whim and
caprice of the testatrix contravenes public policy.

The Missouri Supreme Court held in \textit{State ex rel. McClintock v.
Guinotte}, 275 Mo. 298, 204 S.W. 806, 808 (banc 1918), that the taking of
property by inheritance or will is not an absolute or natural right but one
created by the laws of the sovereign power. The court points out the state
``may foreclose the right absolutely, or it may grant the right upon conditions
precedent, which conditions, if not otherwise violative of our Constitution,
will have to be complied with before the right of descent and distribution
(whether under the law or by will) can exist.'' Further, this power of the
state is one of inherent sovereignty which allows the state to ``say what
becomes of the property of a person, when death forecloses his right to control
it.'' \textit{McClintock v. Guinotte}, \textit{supra} at 808, 809. While
living, a person may manage, use or dispose of his money or property with fewer
restraints than a decedent by will. One is generally restrained from wasteful
expenditure or destructive inclinations by the natural desire to enjoy his
property or to accumulate it during his lifetime. Such considerations however
have not tempered the extravagance or eccentricity of the testamentary
disposition here on which there is no check except the courts.

In the early English case of \textit{Egerton v. Brownlow}, 10 Eng.Rep. 359, 417
(H.L.C.) it is stated: ``The owner of an estate may himself do many things which
he could not (by a condition) compel his successor to do. One example is
sufficient. He may leave his land uncultivated, but he cannot by a condition
compel his successor to do so. The law does not interfere with the owner and
compel him to cultivate his land, (though it may be for the public good that
land should be cultivated) so far the law respects ownership; but when, by a
condition, he attempts to compel his successor to do what is against the public
good, the law steps in and pronounces the condition void and allows the devisee
to enjoy the estate free from the condition.''\ldots 

[The Restatement, Second, of Trusts, Section 124, states:] ``Although a person
may deal capriciously with his own property, his self interest ordinarily will
restrain him from doing so. Where an attempt is made to confer such a power
upon a person who is given no other interest in the property, there is no such
restraint and it is against public policy to allow him to exercise the power if
the purpose is merely capricious.'' The text is followed by this illustration:
``A bequeaths \$1,000.00 to B in trust to throw the money into the sea. B holds
the money upon a resulting trust for the estate of A and is liable to the
estate of A if he throws the money into the sea.'' \ldots It is important to
note that the purposes of [Mrs. Johnston's] trust will not be defeated by
injunction; instead, the proceeds from the sale of the property will pass into
the residual estate and thence to the trust estate as intended, and only the
capricious destructive condition will be enjoined.

In \textit{Colonial Trust Co. v. Brown et al.}, 105 Conn. 261, 135 A. 555 (1926)
the court invalidated, as against public policy, the provisions of a will
restricting erection of buildings more than three stories in height and
forbidding leases of more than one year on property known as ``The Exchange
Place'' in the heart of the City of Waterbury. The court stated:

``As a general rule, a testator has the right to impose such conditions as he
pleases upon a beneficiary as conditions precedent to the vesting of an estate
in him, or to the enjoyment of a trust estate by him as cestui que trust. He
may not, however, impose one that is uncertain, unlawful or opposed to public
policy.'' [\textit{Colonial Trust Co.}, 135 A. at 564.] 

\ldots The term ``public policy'' cannot be comprehensively defined in specific
terms but the phrase ``against public policy'' has been characterized as that
which conflicts with the morals of the time and contravenes any established
interest of society. Acts are said to be against public policy ``when the law
refuses to enforce or recognize them, on the ground that they have a
mischievous tendency, so as to be injurious to the interests of the state,
apart from illegality or immorality.'' \textit{Dille v. St. Luke's Hospital},
355 Mo. 436, 196 S.W.2d 615, 620 (1946); \textit{Brawner v. Brawner}, 327
S.W.2d 808, 812 (Mo. banc 1959).

Public policy may be found in the Constitution, statutes and judicial decisions
of this state or the nation. But in a case of first impression where there are
no guiding statutes, judicial decisions or constitutional provisions, ``a
judicial determination of the question becomes an expression of public policy
provided it is so plainly right as to be supported by the general will.''
\textit{In re Mohler's Estate}, 343 Pa. 299, 22 A.2d 680, 683 (1941). In the
absence of guidance from authorities in its own jurisdiction, courts may look
to the judicial decisions of sister states for assistance in discovering
expressions of public policy.

Although public policy may evade precise, objective definition, it is evident
from the authorities cited that this senseless destruction serving no apparent
good purpose is to be held in disfavor. A well-ordered society cannot tolerate
the waste and destruction of resources when such acts directly affect important
interests of other members of that society. It is clear that property owners in
the neighborhood of \#4 Kingsbury, the St. Louis Community as a whole and the
beneficiaries of testatrix's estate will be severely injured should the
provisions of the will be followed. No benefits are present to balance against
this injury and we hold that to allow the condition in the will would be in
violation of the public policy of this state.

Having thus decided, we do not reach the plaintiffs' contentions regarding
enforcement of the restrictions in the Kingsbury Place trust indenture and
actionable private nuisance, though these contentions may have
merit.\readingfootnote{5}{The dissenting opinion suggests this case be decided
under the general rule
that an owner has exclusive control and the right to untrammeled use of real
property. Although Maxims of this sort are attractive in their simplicity,
standing alone they seldom suffice in a complex case. None of the cited cases
pertains t[o] the qualified right of testatrix to impose, post mortem, a
condition upon her executor requiring an unexplained destruction of estate
property\ldots. Each acknowledges the principle of an owner's `free use' as
the starting point but all recognize competing interests of the community and
other owners of great importance. Accordingly, the general principle of `free
and untrammeled' use is markedly narrowed, supporting in each case a result
opposite that urged by the dissent in the case at bar.} \ldots

\textsc{Dowd}, P.J., concurs.

\opinion\textsc{Clemens}, Judge (dissenting).

I dissent.

\ldots The simple issue in this case is whether the trial court erred by
refusing to enjoin a trustee from carrying out an explicit testamentary
directive. In an emotional opinion, the majority assumes a psychic knowledge of
the testatrix' reasons for directing her home be razed; her testamentary
disposition is characterized as ``capricious,'' ``unwarranted,'' ``senseless,''
and ``eccentric.'' But the record is utterly silent as to her motives\ldots. The
fact is the majority's holding is based upon wispy, self-proclaimed public
policy grounds that were only vaguely pleaded, were not in evidence, and were
only sketchily briefed by the plaintiffs.

\ldots The court has resorted to public policy in order to vitiate Mrs.
Johnston's valid testamentary direction. But this is not a proper case for
court-defined public policy.

\ldots The leading Missouri case on public policy as that doctrine applies to a
testator's right to dispose of property is \textit{In re Rahn's Estate}, 316
Mo. 492, 291 S.W. 120 [1, 2] (banc 1927), cert. den. 274 U.S. 745, 47 S.Ct.
591, 71 L.Ed. 1325. There, an executor refused to pay a bequest on the ground
the beneficiary was an enemy alien, and the bequest was therefore against
public policy. The court denied that contention: ``We may say, at the outset,
that the policy of the law favors freedom in the testamentary disposition of
property and that it is the duty of the courts to give effect to the intention
of the testator, as expressed in his will, provided such intention does not
contravene an established rule of law.'' And the court wisely added, ``it is
not the function of the judiciary to create or announce a public policy of its
own, but solely to determine and declare what is the public policy of the state
or nation as such policy is found to be expressed in the Constitution,
statutes, and judicial decisions of the state or nation, \ldots not by the
varying opinions of laymen, lawyers, or judges as to the demands or the
interests of the public.'' And, in cautioning against judges declaring public
policy the court stated: ``Judicial tribunals hold themselves bound to the
observance of rules of extreme caution when invoked to declare a transaction
void on grounds of public policy, and prejudice to the public interest must
clearly appear before the court would be warranted in pronouncing a transaction
void on this account.'' In resting its decision on public-policy grounds, the
majority opinion has transgressed the limitations declared by our Supreme Court
in \textit{Rahn's Estate}.

\ldots As much as our aesthetic sympathies might lie with neighbors near a house
to be razed, those sympathies should not so interfere with our considered legal
judgment as to create a questionable legal precedent. Mrs. Johnston had the
right during her lifetime to have her house razed, and I find nothing which
precludes her right to order her executor to raze the house upon her death. It
is clear that ``the law favors the free and untrammeled use of real property.''
\textit{Gibbs v. Cass}, 431 S.W.2d 662(2) (Mo.App.1968). This applies to
testamentary dispositions. \textit{Mississippi Valley Trust Co. v. Ruhland},
359 Mo. 616, 222 S.W.2d 750(2) (1949). An owner has exclusive control over the
use of his property subject only to the limitation that such use may not
substantially impair another's right to peaceably enjoy his property.
Plaintiffs have not shown that such impairment will arise from the mere
presence of another vacant lot on Kingsbury Place\ldots.

