
The United States Supreme Court has noted that the right to exclude is
``universally held to be a fundamental element of the property right,''
\textit{Kaiser Aetna v. United States}, 444 U.S. 164, 179-80 (1979), and ``one
of the most essential sticks in the bundle of rights that are commonly
characterized as property.'' \textit{Dolan v. City of Tigard,} 512 U.S. 374,
384 (1994). But property owners typically enjoy a number of additional rights,
which is one source of the ``bundle of rights'' metaphor referred to in
\textit{Dolan}. Among these are:

\begin{itemize}
\item The right of \textbf{possession} (sometimes called a ``possessory''
right); 
\item The right of \textbf{use} (sometimes called a ``usufructary'' right); 
\item The power of \textbf{alienation}---i.e., the right to or transfer
ownership to someone else---which can be further decomposed into

\begin{itemize}
\item The power to make a gratuitous transfer, \textit{i.e.}, a \textbf{gift}
(sometimes called a ``donative'' right)
\item The power to transfer in exchange for valuable consideration (sometimes
called the right to ``\textbf{sell}'' or ``vend,'' or the right of
``market-alienation'')
\item The power to dispose of property owned during life after death \textbf{by
will} (sometimes called the ``\textbf{testamentary}'' right, or the right to
``devise'')
\end{itemize}
\end{itemize}

As with the right to exclude, each of these rights may be limited, particularly
when they have the potential to conflict with competing rights or interests.
Some of those limits are hinted at in the \textit{Shack}: consider the New
Jersey Supreme Court's reference to the latin maxim ``\textit{sic utere tuo ut
alienum non laedas}''. This maxim expresses a long-standing limitation on
property owners' rights of \textit{use}. Does it make sense for the court to
have invoked this maxim in \textit{Shack}? Do you think \textit{Shack} is
better understood as a case about the right to exclude or some other right of
property owners? 

We will study the law's protection of possession (and the limits of that
protection) in our units on Allocation, Found and Stolen Property, and Adverse
Possession. We will make an extensive study of the right to alienate in our
units on Gifts, Estates and Future Interests, Co-Ownership, and Land
Conveyancing. And we will return to limits on the right of use, and in
particular the \textit{sic utere tuo} principle, in our chapter on Nuisance.
But for now let us consider one example of how these other rights of ownership
may be ambiguous, and subjected to limits in the face of competing interests:

\expectnext{eyerman-v-mercantile}
