\reading[Hohfeld, \emph{Fundamental Legal Conceptions}]{Wesley Newcomb Hohfeld,
\textit{Fundamental Legal Conceptions as Applied in Judicial Reasoning}}
\readingcite{26 {Yale L.J.} 710, 713-745 (1917)}

The phrases \textit{in personam} and \textit{in rem}, in spite of the scope and
variety of situations to which they are commonly applied, are more usually
assumed by lawyers, judges, and authors to be of unvarying meaning and free of
ambiguities calculated to mislead the unwary. The exact opposite is, however,
true; and this has occasionally been explicitly emphasized by able judges whose
warnings are worthy of notice\ldots.

A \ldots right \textit{in personam}\ldots is either a unique right residing
in a person (or group of persons) and availing against a single person (or
single group of persons); or else it is one of a \textit{few} fundamentally
similar, yet separate, rights availing respectively against a few definite
persons. A \ldots right \textit{in rem} \ldots is always \textit{one} of a
large class of fundamentally similar yet separate rights, actual and potential,
residing in a single person (or single group of persons) but availing
respectively against persons constituting a very large and indefinite class of
people.

Probably all would agree substantially on the meaning and significance of a
right \textit{in personam}, as just explained; and it is easy to give a few
preliminary examples: If B owes A a thousand dollars, A has an
\textit{affirmative} right \textit{in personam}, \ldots that B shall transfer
to A the legal ownership of that amount of money. If, to put a contrasting
situation, A already has title to one thousand dollars, his rights against
others in relation thereto are \ldots rights \textit{in rem}. In the one case
the money is \textit{owed} to A; in the other case it is \textit{owned} by A.
If Y has contracted to work for X during the ensuing six months, X has an
\textit{affirmative} right \textit{in personam} that Y shall render such
service, as agreed. Similarly as regards all other contractual or
quasi-contractual rights of this character\ldots.

In contrast to these examples are those relating to rights, or claims,
\textit{in rem}\ldots. If A owns and occupies Whiteacre,\edfootnote{The study
of property law was, for much of its history, mainly the study of land. As
such, many teachers' and judges' hypotheticals required the identification of
some fictional parcel of land. By tradition, these parcels take the name
``Whiteacre,'' ``Blackacre,'' ``Greenacre,'' and so on.} not
only B but also a great many other persons---not necessarily all persons---are
under a duty, e.g., not to enter on A's land. A's right against B is a \ldots
right \textit{in rem}, for it is simply one of A's class of \textit{similar},
though separate, rights, actual and potential, against \textit{very many}
persons. The same points apply as regards A's right that B shall not commit a
battery on him, A's right that B shall not alienate the affections of A's wife,
and A's right that B shall not manufacture a certain article as to which A has
a so-called patent\ldots.

\ldots[I]t seems necessary to show very concretely and definitely how, because
of the unfortunate terminology involved, the expression ``right \textit{in
rem}'' is all too frequently misconceived, and meanings attributed to it that
could not fail to blur and befog legal thought and argument. Some of these
loose and misleading usages will now be considered in detail, it being hoped
that the more learned reader will remember that this discussion, being intended
for the assistance of law school students more than for any other class of
persons, is made more detailed and elementary than would otherwise be
necessary.

(a) \textit{A right in rem is not a right ``against a thing'':} \ldots Any
person, be he student or lawyer, unless he has contemplated the matter
analytically and assiduously, or has been put on notice by books or other
means, is likely, first, to translate right \textit{in personam} as a right
\textit{against a person}; and then he is almost sure to interpret right
\textit{in rem}, naturally and symmetrically as he thinks, as a right
\textit{against a thing}.\ldots Such a notion of rights \textit{in rem} is,
as already intimated, crude and fallacious; and it can but serve as a
stumbling-block to clear thinking and exact expression. A man may indeed
sustain close and beneficial \textit{physical} relations to a given
\textit{physical thing:} he may \textit{physically} control and use such thing,
and he may \textit{physically} exclude others from any similar control or
enjoyment. But, obviously, such purely \textit{physical} relations could as
well exist quite apart from, or occasionally in spite of, the law of organized
society: physical relations are wholly distinct from jural relations. The
latter take significance from the law; and, since the purpose of the law is to
regulate the conduct of human beings, all jural relations must, in order to be
clear and direct in their meaning, be predicated of such human beings.\ldots

What is here insisted on,---i.e., that all rights \textit{in rem} are against
persons,---is not to be regarded merely as a matter of taste or preference for
one out of several equally possible forms of statement or definition. Logical
consistency seems to demand such a conception, and nothing less than that. Some
concrete examples may serve to make this plain. Suppose that A is the owner of
Blackacre and X is the owner of Whiteacre. Let it be assumed, further, that, in
consideration of \$100 \textit{actually paid} by A to B, the latter agrees with
A never to enter on X's land, Whiteacre. It is clear that A's right against B
concerning Whiteacre is a right \textit{in personam}\ldots; for A has no
similar and separate rights concerning Whiteacre availing respectively against
other persons in general. On the other hand, A's right against B concerning
Blackacre is obviously a right \textit{in rem}\ldots; for it is but one of a
very large number of fundamentally similar (though separate) rights which A has
respectively against B., C, D, E, F, and a great many other persons. It must
now be evident, also, that A's Blackacre right against B is,
\textit{intrinsically considered}, of the same general character as A's
Whiteacre right against B. The Blackacre right differs, so to say, only
\textit{extrinsically}, that is, in having many fundamentally similar, though
distinct, rights as its ``companions.'' So, in general, we might say that a
right \textit{in personam} is one having few, if any, ``companions''; whereas a
right \textit{in rem} always has many such ``companions.'' 

If, then, the Whiteacre right, being a right \textit{in personam}, is recognized
as a right against a \textit{person}, must not the Blackacre right also, being,
point for point, intrinsically of the same general nature, be conceded to be a
right against a \textit{person?} If not that, what is it? How can it be
apprehended, or described, or delimited at all? \ldots

(b) \textit{A \ldots right in rem is not always one relating to a thing, i.e.,
a tangible object}: \ldots[A] right \textit{in rem} is not necessarily one
\textit{relating to}, or \textit{concerning}, a thing, i.e., a tangible object.
\ldots The term right \textit{in rem} \ldots is so generic in its denotation
as to include: 1.\ldots[R]ights, or claims, relating to a definite
\textit{tangible object}: e.g., a landowner's right that any ordinary person
shall not enter on his land, or a chattel owner's right that any ordinary
person shall not physically harm the object involved,---be  it horse, watch,
book, etc. 2.\ldots[R]ights (or claims) relating neither to definite tangible
object nor to (tangible) person, e.~g., a patentee's right, or claim, that any
ordinary person shall not manufacture articles covered by the patent; 3.
\ldots[R]ights, or claims, relating to the holder's \textit{own person}, e.~g.,
his right that any ordinary person shall not strike him, or that any
ordinary person shall not restrain his physical liberty, i.e., ``falsely
imprison'' him; 4.\ldots[R]ights residing in a given person and relating to
\textit{another} person, e.~g., the right of a father that his daughter shall
not be seduced, or the right of a husband that harm shall not be inflicted on
his wife so as to deprive him of her company and assistance; 5. \ldots[R]ights,
or claims, not relating directly to either a (tangible) person or a tangible
object, e.~g., a person's right that another shall not publish a libel of him,
or a person's right that another shall not publish his picture, the so-called
``right of privacy'' existing in some states, but not in all. 

It is thus seen that some rights \textit{in rem}\ldots relate fairly directly to
\textit{physical objects}; some fairly directly to \textit{persons}; and some
fairly directly \textit{neither to tangible objects nor to persons}\ldots.

