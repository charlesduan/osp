\reading{Allen v. Hyatt Regency-Nashville Hotel}
\readingcite{668 S.W.2d 286 (Tenn. 1984)}

HARBISON, Justice.

In this case the Court is asked to consider the nature and extent of the
liability of the operator of a commercial parking garage for theft of a vehicle
during the absence of the owner. Both courts below, on the basis of prior
decisions from this state, held that a bailment was created when the owner
parked and locked his vehicle in a modern, indoor, multi-story garage operated
by appellant in conjunction with a large hotel in downtown Nashville. We
affirm.

There is almost no dispute as to the relevant facts. Appellant is the owner and
operator of a modern high-rise hotel in Nashville fronting on the south side of
Union Street. Immediately to the rear, or south, of the main hotel building
there is a multi-story parking garage with a single entrance and a single exit
to the west, on Seventh Avenue, North. As one enters the parking garage at the
street level, there is a large sign reading ``Welcome to Hyatt
Regency-Nashville.'' There is another Hyatt Regency sign inside the garage at
street level, together with a sign marked ``Parking.'' The garage is available
for parking by members of the general public as well as guests of the hotel,
and the public are invited to utilize it.

On the morning of February 12, 1981, appellee's husband, Edwin Allen,
accompanied by two passengers, drove appellee's new 1981 automobile into the
parking garage. Neither Mr. Allen nor his passengers intended to register at
the hotel as a guest. Mr. Allen had parked in this particular garage on several
occasions, however, testifying that he felt that the vehicle would be safer in
an attended garage than in an unattended outside lot on the street.

The single entrance was controlled by a ticket machine. The single exit was
controlled by an attendant in a booth just opposite to the entrance and in full
view thereof. Appellee's husband entered the garage at the street level and
took a ticket which was automatically dispensed by the machine. The machine
activated a barrier gate which rose and permitted Mr. Allen to enter the
garage. He drove to the fourth floor level, parked the vehicle, locked it,
retained the ignition key, descended by elevator to the street level and left
the garage. When he returned several hours later, the car was gone, and it has
never been recovered. Mr. Allen reported the theft to the attendant at the exit
booth, who stated, ``Well, it didn't come out here.'' The attendant did not
testify at the trial.

Mr. Allen then reported the theft to security personnel employed by appellant,
and subsequently reported the loss to the police. Appellant regularly employed
a number of security guards, who were dressed in a distinctive uniform, two of
whom were on duty most of the time. These guards patrolled the hotel grounds
and building as well as the garage and were instructed to make rounds through
the garage, although not necessarily at specified intervals. One of the
security guards told appellee's husband that earlier in the day he had received
the following report:
\begin{quote}
He said, ``It's a funny thing here. On my report here a lady called me somewhere
around nine-thirty or after and said that there was someone messing with a
car.''
\end{quote}
The guard told Mr. Allen that he closed his office and went up into the garage
to investigate, but reported that he did not find anything unusual or out of
the ordinary.

Customers such as Mr. Allen, upon entering the garage, received a ticket from
the dispensing machine. On one side of this ticket are instructions to
overnight guests to present the ticket to the front desk of the hotel. The
other side contains instructions to the parker to keep the ticket and that the
ticket must be presented to the cashier upon leaving the parking area. The
ticket states that charges are made for the use of parking space only and that
appellant assumes no responsibility for loss through fire, theft, collision or
otherwise to the car or its contents. The ticket states that cars are parked at
the risk of the owner, and parkers are instructed to lock their vehicles. The
record indicates that these tickets are given solely for the purpose of
measuring the time during which a vehicle is parked in order that the attendant
may collect the proper charge, and that they are not given for the purpose of
identifying particular vehicles.

The question of the legal relationship between the operator of a vehicle which
is being parked and the operator of parking establishments has been the subject
of frequent litigation in this state and elsewhere. The authorities are in
conflict, and the results of the cases are varied.

It is legally and theoretically possible, of course, for various legal
relationships to be created by the parties, ranging from the traditional
concepts of lessor-lessee, licensor-licensee, bailor-bailee, to that described
in some jurisdictions as a ``deposit.'' Several courts have found difficulty
with the traditional criteria of bailment in analyzing park-and-lock cases. One
of the leading cases is \textit{McGlynn v. Parking Authority of City of
Newark}, 432 A.2d 99 (N.J. 1981). There the Supreme Court of New Jersey
reviewed numerous decisions from within its own state and from other
jurisdictions, and it concluded that it was more ``useful and straightforward''
to consider the possession and control elements in defining the duty of care of
a garage operator to its customers than to consider them in the context of
bailment. That Court concluded that the ``realities'' of the relationship
between the parties gave rise to a duty of reasonable care on the part of
operators of parking garages and parking lots. It further found that a garage
owner is usually better situated to protect a parked car and to distribute the
cost of protection through parking fees. It also emphasized that owners usually
expect to receive their vehicles back in the same condition in which they left
them and that the imposition of a duty to protect parked vehicles and their
contents was consistent with that expectation. The Court went further and
stated that since the owner is ordinarily absent when theft or damage occurs,
the obligation to come forward with affirmative evidence of negligence could
impose a difficult, if not insurmountable, burden upon him. After considering
various policy considerations, which it acknowledged [to] be the same as those
recognized by courts holding that a bailment is created, the New Jersey Court
indulged or authorized a presumption of negligence from proof of damage to a
car parked in an enclosed garage.

Although the New Jersey Court concluded that a more flexible and comprehensive
approach could be achieved outside of traditional property concepts, Tennessee
courts generally have analyzed cases such as this in terms of sufficiency of
the evidence to create a bailment for hire by implication. We believe that this
continues to be the majority view and the most satisfactory and realistic
approach to the problem, unless the parties clearly by their conduct or by
express contract create some other relationship.

The subject has been discussed in numerous previous decisions in this state. One
of the leading cases is \textit{Dispeker v. New Southern Hotel Co.}, 373 S.W.2d
904 (Tenn. 1963). In that case the guest at a hotel delivered his vehicle to a
bellboy who took possession of it and parked it in a lot adjoining the hotel
building. The owner kept the keys, but the car apparently was capable of being
started without the ignition key. The owner apparently had told the attendant
how to so operate it. Later the employee took the vehicle for his own purposes
and damaged it. Under these circumstances the Court held that a bailment for
hire had been created and that upon proof of misdelivery of the vehicle the
bailee was liable to the customer.

In the subsequent case of \textit{Scruggs v. Dennis}, 440 S.W.2d 20 (Tenn.
1969), upon facts practically identical to those of the instant case, the Court
again held that an implied bailment contract had been created between a
customer who parked and locked his vehicle in a garage. Upon entry he received
a ticket dispensed by a machine, drove his automobile to the underground third
level of the garage and parked. He retained his ignition key, but when he
returned to retrieve the automobile in the afternoon it had disappeared. It was
recovered more than two weeks later and returned to the owner in a damaged
condition.

In that case the operator of the garage had several attendants on duty, but the
attendants did not ordinarily operate the parked vehicles, as in the instant
case.

Although the Court recognized that there were some factual differences between
the \textit{Scruggs} case and that of \textit{Dispeker v. New Southern Hotel
Co.}, \textit{supra}, it concluded that a bailment had been created when the
owner parked his vehicle for custody and safe keeping in the parking garage,
where there was limited access and where the patron had to present a ticket to
an attendant upon leaving the premises.

A bailment relationship was also found in \textit{Jackson v. Metropolitan
Government of Nashville}, 483 S.W.2d 92 (Tenn. 1972), when faculty members of a
high school conducted an automobile parking operation for profit upon the high
school campus. A customer who parked his vehicle there was allowed recovery for
theft, even though he had parked the vehicle himself after paying a fee, had
locked the vehicle and had kept the keys.

On the contrary, in the case of \textit{Rhodes v. Pioneer Parking Lot, Inc.},
501 S.W.2d 569 (Tenn. 1973), a bailment was found not to exist when the owner
left his vehicle in an open parking lot which was wholly unattended and where
he simply inserted coins into a meter, received a ticket, then parked the
vehicle himself and locked it.

Denying recovery, the Court said:
\begin{quote}
In the case at bar, however, we find no evidence to justify a finding that the
plaintiff delivered his car into the custody of the defendant, nor do we find
any act or conduct upon the defendant's part which would justify a reasonable
person believing that an obligation of bailment had been assumed by the
defendant. 501 S.W.2d at 571.
\end{quote}

In the instant case, appellee's vehicle was not driven into an unattended or
open parking area. Rather it was driven into an enclosed, indoor, attended
commercial garage which not only had an attendant controlling the exit but
regular security personnel to patrol the premises for safety.

Under these facts we are of the opinion that the courts below correctly
concluded that a bailment for hire had been created, and that upon proof of
nondelivery appellee was entitled to the statutory presumption of negligence
provided in T.C.A. {\S} 24-5-111.

We recognize that there is always a question as to whether there has been
sufficient delivery of possession and control to create a bailment when the
owner locks a vehicle and keeps the keys. Nevertheless, the realities of the
situation are that the operator of the garage is, in circumstances like those
shown in this record, expected to provide attendants and protection. In
practicality the operator does assume control and custody of the vehicles
parked, limiting access thereto and requiring the presentation of a ticket upon
exit. As stated previously, the attendant employed by appellant did not
testify, but he told appellee's husband that the vehicle did not come out of
the garage through the exit which he controlled. This testimony was not
amplified, but the attendant obviously must have been in error or else must
have been inattentive or away from his station. The record clearly shows that
there was no other exit from which the vehicle could have been driven.

Appellant made no effort to rebut the presumption created by statute in this
state (which is similar to presumptions indulged by courts in some other
jurisdictions not having such statutes). While the plaintiff did not prove
positive acts of negligence on the part of appellant, the record does show that
some improper activity or tampering with vehicles had been called to the
attention of security personnel earlier in the day of the theft in question,
and that appellee's new vehicle had been removed from the garage by some person
or persons unknown, either driving past an inattentive attendant or one who had
absented himself from his post, there being simply no other way in which the
vehicle could have been driven out of the garage.

Under the facts and circumstances of this case, we are not inclined to depart
from prior decisions or to place the risk of loss upon the consuming public as
against the operators of commercial parking establishments such as that
conducted by appellant. We recognize that park-and-lock situations arise under
many and varied factual circumstances. It is difficult to lay down one rule of
law which will apply to all cases. The expectations of the parties and their
conduct can cause differing legal relationships to arise, with consequent
different legal results. We do not find the facts of the present case, however,
to be at variance with the legal requirements of the traditional concept of a
bailment for hire. In our opinion it amounted to more than a mere license or
hiring of a space to park a vehicle, unaccompanied by any expectation of
protection or other obligation upon the operator of the establishment.

The judgment of the courts below is affirmed at the cost of appellant. The cause
will be remanded to the trial court for any further proceedings which may be
necessary.

DROWOTA, Justice, dissenting.

In this case we are asked to consider the nature and extent of liability of the
operator of a commercial ``park and lock'' parking garage. In making this
determination, we must look to the legal relationship between the operator of
the vehicle and the operator of the parking facility. The majority opinion
holds that a bailment contract has been created, and upon proof of non-delivery
Plaintiff is entitled to the statutory presumption of negligence provided in
T.C.A. {\S} 24-5-111. I disagree, for I find no bailment existed and therefore
the Plaintiff does not receive the benefit of the presumption. Consequently,
the Plaintiff had the duty to prove affirmatively the negligence of the
operator of the parking facility and this Plaintiff failed to do.

The majority opinion states that ``courts have found difficulty with the
traditional criteria of bailment in analyzing park and lock cases.'' The
majority discusses the case of \textit{McGlynn v. Parking Authority of City of
Newark}, 86 N.J. 551, 432 A.2d 99 (1981), which suggests that bailment is an
outmoded concept for analyzing parking lot and garage cases. In \textit{Garlock
v. Multiple Parking Services, Inc}., 427 N.Y.S.2d 670, 677 (1980), the court
stated that ``the `bailment theory' as a basis for recovery in parking lot
cases is no longer appropriate.'' That court concluded that since the concept
of bailment is no longer a viable theory in application to a very real modern
problem that the proper standard to be followed in such cases is ``reasonable
care under the circumstances whereby foreseeability shall be a measure of
liability.'' \textit{Id.}, 427 N.Y.S.2d at 678.

Even though some courts now suggest that the theory of bailment is an archaic
and inappropriate theory upon which to base liability in modern park and lock
cases, the majority opinion states that ``Tennessee courts generally have
analyzed cases such as this in terms of sufficiency of the evidence to create a
bailment for hire by implication,'' and concludes that this is ``the most
satisfactory and realistic approach to the problem.'' I do not disagree with
the longstanding use of the bailment analysis in this type of case. I do
disagree, however, with the majority's conclusion that a bailment for hire has
been created in this case.

The record shows that upon entering this parking garage a ticket, showing time
of entry, is automatically dispensed by a machine. The ticket states that
charges are made for the use of a parking space only and that the garage
assumes no responsibility for loss to the car or its contents. The ticket
further states that cars are parked at the risk of the owner, and parkers are
instructed to lock their vehicles. The majority opinion points out that it is
not insisted that this language on the ticket is sufficient to exonerate the
garage, since the customer is not shown to have read it or to have had it
called to his attention. \textit{Savoy Hotel Corp. v. Sparks}, 421 S.W.2d 98
(Tenn. Ct. App. 1967). The ticket in no way identifies the vehicle, it is given
solely for the purpose of measuring the length of time during which the vehicle
is parked in order that a proper charge may be made.

In this case Mr. Allen, without any direction or supervision, parked his car,
removed his keys, and locked the car and left the parking garage having
retained his ignition key. The presentation of a ticket upon exit is for the
sole purpose of allowing the cashier to collect the proper charge. The cashier
is not required to be on duty at all times. When no cashier is present, the
exit gate is opened and no payment is required.\readingfootnote{1}{Between one
or two in
the morning and six or seven a.m., the garage is entirely open without a
cashier to collect parking fees. During the day if the cashier leaves his or
her post on a break, the exit gate is opened and the vehicle owner may exit
without payment.} As the majority opinion states, the ticket is ``not given
for the purpose of identifying particular vehicles.'' The ticket functioned
solely as a source of fee computation, not of vehicle identification.

The majority opinion states: ``[W]e do not find the facts of the present case to
be at variance with the legal requirements of the concept of a bailment for
hire.'' I must disagree, for I feel the facts of the present case are clearly
at variance with what I consider to be the legal requirements of the
traditional concept of a bailment for hire.

Bailment has been defined by this Court in the following manner:
\begin{quotation}
The creation of a bailment in the absence of an express contract requires that
possession and control over the subject matter pass from the bailor to the
bailee. In order to constitute a sufficient delivery of the subject matter
there must be a full transfer, either actual or constructive, of the property
to the bailee so as to exclude it from the possession of the owner and all
other persons and give to the bailee, for the time being, the sole custody and
control thereof.

In parking lot and parking garage situations, a bailment is created where the
operator of the lot or garage has knowingly and voluntarily assumed control,
possession, or custody of the motor vehicle; if he has not done so, there may
be a mere license to park or a lease of parking space.
\end{quotation}
\textit{Rhodes v. Pioneer Parking Lot, Inc.,} 501 S.W.2d 569, 570 (Tenn. 1973).

From its earliest origins, the most distinguishing factor identifying a bailment
has been delivery. Our earliest decisions also recognize acceptance as a
necessary factor, requiring that possession and control of the property pass
from bailor to bailee, to the exclusion of control by others. The test thus
becomes whether the operator of the vehicle has made such a delivery to the
operator of the parking facility as to amount to a relinquishment of his
exclusive possession, control, and dominion over the vehicle so that the latter
can exclude it from the possession of all others. If so, a bailment has been
created.

When the automobile began replacing the horse and buggy, our courts allowed
bailment law to carry over and govern the parking of vehicles. In cases such as
\textit{Old Hickory Parking Corp. v. Alloway}, 177 S.W.2d 23 (Tenn. Ct. App.
1943), and \textit{Savoy Hotel v. Sparks}, 421 S.W.2d 98 (Tenn. Ct. App. 1967),
where the operator of the vehicle left his vehicle with an attendant and left
the keys for the attendant to move the vehicle as he wished, the bailment
relationship was evident for we had a clear delivery, acceptance of possession,
control, and exercise of dominion over the vehicle---all the traditional
elements of a bailment. In \textit{Dispeker v. New Southern Hotel Company}, 373
S.W.2d 904 (Tenn. 1963), a bellboy parked plaintiff's car, plaintiff retained
the keys but explained to the bellboy that the car could be operated without
the key, and apparently showed him how to operate it. The bellboy went off
duty, then returned and stole the car. Once again, the traditional elements of
delivery and control were present.

These cases involving parking attendants and personalized service have caused us
no problems. The problem arises in this modern era of automated parking, when
courts have attempted to expand the limits of existing areas of the law to
encompass technological and commercial advances. Such is the case of
\textit{Scruggs v. Dennis}, 440 S.W.2d 20 (Tenn. 1969), relied upon in the
majority opinion. In \textit{Scruggs}, as in this case, the entire operation is
automated, with the exception of payment upon departure. The operation bears
little, if any, resemblance to the circumstances found in \textit{Old Hickory
Parking Corp}., \textit{Savoy Hotel}, and \textit{Dispeker}. Yet the Court in
Scruggs, in quoting extensively from the \textit{Dispeker} opinion, states that
``There are some minute differences of fact \ldots'' Id., 440 S.W.2d at 22. As
pointed out above, the differences of fact in \textit{Dispeker} are not minute
or so similar as the \textit{Scruggs} court would suggest. Delivery, custody
and control are clearly present in \textit{Dispeker}. I fail to find such
delivery, custody and control in \textit{Scruggs} or in the case at bar. In
\textit{Dispeker}, the vehicle was actually taken from the owner by an
attendant. I believe the \textit{Scruggs} court and the majority opinion today
attempt to apply bailment law in situations where there is not a true bailment
relationship. \ldots

The majority opinion, as did the \textit{Scruggs} court, finds custody and
control implied because of the limited access and because ``the presentation of
a ticket upon exit'' is required. I cannot agree with this analysis as creating
a bailment situation. I do not believe that based upon the fact that a ticket
was required to be presented upon leaving, that this factor created a proper
basis upon which to find a bailment relationship. The ticket did not identify
the vehicle or the operator of the vehicle, as do most bailment receipts. The
cashier was not performing the traditional bailee role or identifying and
returning a particular article, but instead was merely computing the amount
owed and accepting payment due for use of a parking space. I do not believe the
Defendant exercised such possession and control over Plaintiff's automobile as
is necessary in an implied bailment. \ldots

The full transfer of possession and control, necessary to constitute delivery,
should not be found to exist simply by the presentation of a ticket upon exit.
In the case at bar, I find no such delivery and relinquishment of exclusive
possession and control as to create a bailment. Plaintiff parked his car,
locked it and retained the key. Certainly Defendant cannot be said to have sole
custody of Plaintiff's vehicle, for Defendant could not move it, did not know
to whom it belonged, and did not know when it would be reclaimed or by whom.
Anyone who manually obtained a ticket from the dispenser could drive out with
any vehicle he was capable of operating. Also, a cashier was not always on
duty. When on duty, so long as the parking fee was paid---by what means could
the Defendant reasonably exercise control? The necessary delivery and
relinquishment of control by the Plaintiff, the very basis upon which the
bailment theory was developed, is missing.

We should realize that the circumstances upon which the principles of bailment
law were established and developed are not always applicable to the operation
of the modern day automated parking facility. The element of delivery, of sole
custody and control are lacking in this case.

