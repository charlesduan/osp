\expected{armory-v-delamirie}
\expected{narrative-variations-on-armory}

We have seen numerous cases in which a possessor---\having{pierson-v-post}{a fox
hunter, }{}{}a finder of a
jewel, a lumber thief---is not the true owner according to the law. There are
more. The property could have been entrusted by the owner to the possessor:
this is called a \textit{bailment}. (Note that the entruster is the bail-OR and
the possessor is the bail-EE.) Common bailees include delivery services, dry
cleaners, and friends who borrow each others' casebooks. Or perhaps the
property is owned by the possessor, but subject to a security interest held by
a third party. Car loans are a familiar class of these \textit{liens}: the bank
has a right to repossess the car if the buyer fails to make payments on time.
Sometimes the two go together. A pawn shop, for example, is both a bailee and
lienholder: it has possession of the pawnor's gold-plated fish tank on skis and
a lien against it, which it uses to secure the loan it makes to the pawnor.

These arrangements, all of which split full ownership from physical possession,
systematically raise the same kinds of issues. First, there is the question of
the duties between the possessor and the party out of possession: bailees have
a duty to return the property, and secured creditors can satisfy unpaid debts
by taking ownership of the property. Second, there is the question of which of
the parties has enough of an interest in the property to sue if some third
party steals or damages it. Third, there is the difficult problem of protecting
the legitimate expectations of third parties dealing with a person in
possession of property who may or may not be its full owner---problems that
should be familiar from the materials on good faith purchasers.

