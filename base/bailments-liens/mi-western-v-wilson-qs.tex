\expected{mi-western-v-wilson}

\item \textit{In re Housecraft Industries}, 155 B.R. 79, 86--87 (Bankr. D. Vt.
1993) gives some background on the evolution of security interests in personal
property:
\begin{quotation}
Until the early nineteenth century, the only way to create a valid interest in
personal property was by physical pledge---the transfer of possession of the
property (collateral) by a debtor (the pledgor) to the creditor or secured
party (the pledgee). Possession provided public notice of a secured party's
interest in collateral and prevented debtors from selling their pledged
property to innocent purchasers or from obtaining credit based on encumbered
assets. To further protect third parties against undisclosed interests in
property, the common law presumed that nonpossessory interests were fraudulent
and therefore unenforceable against third parties. \textit{Twyne's Case}, 76
Eng. Rep. 809 (Star Chamber 1601).

The increasing demands of the credit economy eventually created a need for
collateral that remained in a debtor's possession. Limited only by their
creativity, debtors, creditors, and their counsel formulated methods of
perfection that provided both possession to debtors and security to creditors.
The resulting rules varied from jurisdiction to jurisdiction, producing what
one commentator has called a ``labyrinthine melange'' of personal property
securities laws. Throughout this development toward modern commercial law, the
common law pledge existed side by side with other forms of perfecting security
interests in personal property.

The Uniform Commercial Code \ldots  streamlined commercial law and preserved
the pledge to complement a public filing system. Article 9 of the UCC, \ldots
governs security interests in most forms of personal property and fixtures.
Article 9 recognizes three general ways to perfect a security interest: filing
(public registration); possession of the collateral, either directly,
constructively or through an agent; and third party notice, including notice
given by the secured party to another holding the collateral.
\end{quotation}

\item Treleven, the mechanic, wins in \textit{Wilson}. But why? Critique the
following summaries of the holding:
\begin{itemize}
\item ``Mechanics in possession have priority over other creditors.''
\item ``Trevelen's lien arose before the bank's.''
\item ``Trevelen put the bank on notice of his lien.''
\end{itemize}
Each of these statements is misleading standing alone, but the holding draws on
them all. What \textit{is} the rule of the case?


\item Suppose Groucho takes his car to Harpo's Transmissions for repairs and
parks it on Harpo's lot. That's a bailment; Harpo must turn over the car when
Groucho demands it back. But now suppose that Harpo does \$400 worth of repairs
on the car at Groucho's request and Groucho fails to pay. Harpo now has a
mechanic's lien on the car. Can Groucho get his car back? What remedies could
Harpo obtain if he sued Groucho for breach of contract? Does having the car on
his lot give him any additional options? What if Groucho sells the car to Chico
without telling Chico about Harpo's lien? What if Harpo lets Groucho drive the
car off the lot to confirm that the transmission has been fixed and Groucho
floors it as soon as he reaches the highway and never comes back? If Harpo
finds the car in Groucho's driveway, can he tow it back to his lot?


\item What are Groucho and Harpo's respective rights and obligations if Zeppo
steals the car while it's parked on Harpo's lot? If the police subsequently
find the car abandoned on the side of the road, who is entitled to it?
Conversely, if Zeppo totals the car by driving it into a tree and both Groucho
and Harpo sue him for conversion, what result?


\item \textit{Wilson} gives a glimpse at the perennial problem of
\textit{priority}, which arises whenever a debtor has multiple creditors and is
unable to pay them all. The ultimate system for sorting out priority is federal
bankruptcy law, but as \textit{Wilson} illustrates, state commercial law
(especially Article 9 of the UCC) plays a significant role too. Even a quick
skim through Article 9 shows how extensively its rules are adapted to the
particular characteristics of the class of property at issue (or perhaps, to
the demands of special-interest lobbying and the successive encrustations of
history). \emph{See, e.g.}, UCC {\S} 9-102, which distinguishes accounts; farm
products; oil, gas, and minerals both in and out of the ground; tort claims;
commodity futures; consumer goods; health-care debts; manufactured homes;
software; and much, much, more. 


\item Many states attempt to solve the core problem in \textit{Wilson} by
requiring that car liens be recorded with the state Department of Motor
Vehicles and indicated on car owners' certificates of title. The Maryland
system, for example, provides that a security interest in a vehicle is
``perfected'' by ``Delivery to the [Motor Vehicle] Administration of every
existing certificate of title of the vehicle and an application for certificate
of title [including the necessary information about the security interest]''
and that a security interest that has not been so perfected ``is not valid
against any creditor of the owner or any subsequent transferee or secured
party.'' \textsc{Md. Code Transp.} {\S} 13.202. The theory is that the buyer or
lender
can protect itself by demanding to see the title certificate---indeed, a buyer
will need to turn in the old title certificate to register the car and a lender
will need to turn it in to record its own security interest. Is this system
fair to senior lenders? Fair to buyers and junior lenders? How might the system
go wrong? How might a fraudster make it go wrong? All things considered, is
this a better system than the Wisconsin one discussed in \textit{Wilson}?

