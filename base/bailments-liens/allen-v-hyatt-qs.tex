\expected{allen-v-hyatt}

\item Bailments raise interesting issues about the bailor's and bailee's
relationships with third parties. Suppose Lord Hobnob takes a valuable jewel to
a jewelry shop for repair. While it is there, a chimney-sweep smashes the
window and runs off with it. Obviously Lord Hobnob can presently sue the
chimney-sweep to recover the jewel or its value. (\textit{Is} this so obvious?)
But what about the jeweler? He's admittedly not the owner of the jewel. Should
he nonetheless be allowed to sue the chimney-sweep? If the answer is yes, and
he wins damages, can he keep the money? If the jeweler wins damages from the
chimney-sweep, can the chimney-sweep be held liable in a subsequent suit by
Lord Hobnob for the same amount?


\item Here's another variation. Suppose a chimney-sweep finds a jewel and gives
it to a jeweler for safekeeping. Lord Hobnob, the true owner, shows up in a
carriage and a huff, and demands the jewel from the jeweler. Can the jeweler
turn it over? Must he? If he does, is he liable to his bailor, the
chimney-sweep, for misdelivery? Consider \textit{The Winkfield}, [1902] P. 42
(C.A. 1901), in which the \textit{Winkfield}, a government ship carrying mail,
was damaged in a collision with the \textit{Mexican}. The government sued the
owners of the \textit{Mexican} and included a claim for mail lost as a result
of the collision. The \textit{Mexican}{}'s owners responded that the government
was not liable to the parties whose mail was lost, and so had suffered no
compensable damages. Is this a persuasive objection?


\item For time immemorial, potential bailees have attempted to limit their
potential liability by contract. Why didn't the ticket in \textit{Allen}
suffice to protect the hotel from liability for the lost car? 


\item A common concern of bailees is taking responsibility for unexpectedly
valuable items. In \textit{Peet v. Roth Hotel}, 253 N.W. 546 (Minn. 1934), the
plaintiff left her engagement ring with a hotel employee with instructions to
give it to a jeweler who paid regular visits to the hotel and was known to its
employees. She testified:
\begin{quote}
I had it [the ring] on my finger, and took it off my finger.  The Cashier---I
told the Cashier that it was for Mr.  Ferdinand Hotz.  She took out an envelope
and wrote ``Ferdinand Hotz.'' I remember spelling it to her, and then I left. 
\ldots I handed the ring to the Cashier, and she wrote on the envelope.
\ldots The only instructions I remember are telling her that it was for Mr.
Ferdinand Hotz who was stopping at the hotel.
\end{quote}
The ring was stolen while in the hotel's possession and a jury awarded
\$2,140.66 in damages. The hotel objected, arguing that plaintiff ``failed to
divulge the unusual value of her ring when she left it with [the cashier, who]
testified that, at the moment, she did not realize its value.'' The court was
unsympathetic, writing, ``No decision has been cited and probably none can be
found where the bailee of an article of jewelry, undeceived as to its identity,
was relieved of liability because of his own erroneous underestimate of its
value.'' Is this fair? Compare Minnesota's modern statute on innkeepers'
liability, in Minn. Stat. {\S} 327.71(1):
\begin{quote}
No innkeeper who has in the establishment a fireproof, metal safe or vault, in
good order and fit for the custody of valuables, and who keeps a copy of this
subdivision clearly and conspicuously posted at or near the front desk and on
the inside of the entrance door of every bedroom, shall be liable for the loss
of or injury to the valuables of a guest unless: (1) the guest has offered to
deliver the valuables to the innkeeper for custody in the safe or vault; and
(2) the innkeeper has omitted or refused to take the valuables and deposit them
in the safe or vault for custody and to give the guest a receipt for them.
Except as otherwise provided in subdivision 6, the liability of an innkeeper
for the loss of or injury to the valuables of a guest shall not exceed \$1,000.
No innkeeper shall be required to accept valuables for custody in the safe or
vault if their value exceeds \$1,000, unless the acceptance is in writing.
\end{quote}
Would this statute have changed the result in \textit{Peet}? How does it alter
the relationship between hotels and guests? Does it explain why hotel rooms
typically have a statement of this sort posted on the inside of their doors?

Here is part of the Uniform Commercial Code's take on the issue (in the context
of carriers' liability for lost or damaged goods given to them for delivery):
\begin{quote}
Damages may be limited by a term in the bill of lading or in a transportation
agreement that the carrier's liability may not exceed a value stated in the
bill or transportation agreement if the carrier's rates are dependent upon
value and the consignor is afforded an opportunity to declare a higher value
and the consignor is advised of the opportunity. However, such a limitation is
not effective with respect to the carrier's liability for conversion to its own
use. \ldots
\end{quote}
UCC {\S} 7-309(b). What do you think of this solution?

