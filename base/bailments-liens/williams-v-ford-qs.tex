\expected{williams-v-ford}

\item True or false: Cathy Williams would have been better off if she had thrown
a punch or two. What do you make of the UCC's purported policy of discouraging
private extrajudicial violence? Where does \textit{Williams} leave other single
mothers facing towing crews at 4:30 AM?


\item Is the breach-of-the-peace test really about deterring violence, or is it
a proxy for the other kinds of individual and social harms repossession can
cause? If so, how good a proxy is it? Are there better ways to avoid those
harms?


\item Notice that FMCC's lien is a property interest. One key indicium of this
fact---or perhaps a component of it---is that it is freely assignable. FMCC
was not the original lender. Who was? How did FMCC end up owning the lien? 


\item On the other side of the loan, Cathy Williams was not the original
borrower; David Williams was. Why is his failure to pay her problem? Indeed, he
was under a court order to continue making payments. Why doesn't that protect
her from repossession? This aspect of liens---that they run with the property
-- is considered crucial to secured lending. Why? Would FMCC be willing to
extend credit in the first place if its resulting security interest did not
bind David's successors in title?


\item In \textit{Williams} the lienholder is not in physical possession of the
collateral. Why not? Would car loans work if the lender retained possession?
This creates two distinctive problems. First, how and when the lender can
retake possession? (Answer: with a tow truck in the middle of the night.) But
what if Cathy Williams drives the car out of state and hides it? For that
matter, what if she destroys it rather than let FMCC repossess it? So FMCC's
property interest in the car provides some protection for its contract rights,
but hardly perfect protection. Could FMCC insist that Cathy Williams install a
GPS device on the car that continually broadcasts its location? \textit{Cf}.
\emph{Am. Car Rental, Inc. v. Comm'r of Consumer Prot.}, 869 A.2d 1198
(Conn. 2005) (unfair consumer practice for car rental agency to charge customer
\$150 per instance of drivng over 79 miles per hour for more than two mintues,
as revealed by GPS tracker in car). Are there privacy concerns with this type
of close monitoring? Safety concerns? Are these more or less severe than if the
lender sent employees to personally follow Cathy Williams around and keep tabs
on the car? What about a kill-switch that automatically shuts down the car's
engine if it is driven more than fifty miles from her house? If FMCC can shut
down the car remotely, could someone else? \textit{See} Andy Greenberg,
\textit{Hackers Remotely Kill a Jeep on the Highway---With Me In It},
\textsc{Wired} (July 21, 2015),
\url{http://www.wired.com/2015/07/hackers-remotely-kill-jeep-highway/}.


\item The second distinctive problem when the lienholder is out of possession is
notice to third parties. What happens if Cathy Williams sells the car without
informing the buyer of the lien? Yes, this is yet another good-faith-purchaser
problem; they are everywhere in property law. Consider the following case:

\expectnext{mi-western-v-wilson}
