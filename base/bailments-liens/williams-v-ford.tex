\reading{Williams v. Ford Motor Credit Co.}
\readingcite{674 F.2d. 717 (8th Cir. 1982)}

\opinion \textsc{Benson}, Chief Judge.

In this diversity action brought by Cathy A. Williams to recover damages for
conversion arising out of an alleged wrongful repossession of an automobile,
Williams appeals from a judgment notwithstanding the verdict entered on motion
of defendant Ford Motor Credit Company (FMCC). In the same case, FMCC appeals a
directed verdict in favor of third party defendant S \& S Recovery, Inc. (S \&
S) on FMCC's third party claim for indemnification. We affirm the judgment
n.o.v. FMCC's appeal is thereby rendered moot.

In July, 1975, David Williams, husband of plaintiff Cathy Williams, purchased a
Ford Mustang from an Oklahoma Ford dealer. Although David Williams executed the
sales contract, security agreement, and loan papers, title to the car was in
the name of both David and Cathy Williams. The car was financed through the
Ford dealer, who in turn assigned the paper to FMCC. Cathy and David Williams
were divorced in 1977. The divorce court granted Cathy title to the automobile
and required David to continue to make payments to FMCC for eighteen months.
David defaulted on the payments and signed a voluntary repossession
authorization for FMCC. Cathy Williams was informed of the delinquency and
responded that she was trying to get her former husband David to make the
payments. There is no evidence of any agreement between her and FMCC. Pursuant
to an agreement with FMCC, S~\&~S was directed to repossess the automobile.

On December 1, 1977, at approximately 4:30 a.m., Cathy Williams was awakened by
a noise outside her house trailer in Van Buren,
Arkansas.\readingfootnote{1}{Cathy Williams testified that the noise sounded
like there was a car stuck in her yard.} She saw that a wrecker truck with two
men in it had hooked up to the Ford Mustang and started to tow it away. She went
outside and hollered at them. The truck stopped. She then told them that the car
was hers and asked them what they were doing. One of the men, later identified
as Don Sappington, president of S~\&~S Recovery, Inc., informed her that he was
repossessing the vehicle on behalf of FMCC. Williams explained that she had been
attempting to bring the past due payments up to date and informed Sappington
that the car contained personal items which did not even belong to her.
Sappington got out of the truck, retrieved the items from the car, and handed
them to her. Without further complaint from Williams, Sappington returned to the
truck and drove off, car in tow. At trial, Williams testified that Sappington
was polite throughout their encounter and did not make any threats toward her or
do anything which caused her to fear any physical harm. The automobile had been
parked in an unenclosed driveway which plaintiff shared with a neighbor. The
neighbor was awakened by the wrecker backing into the driveway, but did not come
out. After the wrecker drove off, Williams returned to her house trailer and
called the police, reporting her car as stolen. Later, Williams commenced this
action.

The case was tried to a jury which awarded her \$5,000.00 in damages. FMCC moved
for judgment notwithstanding the verdict, but the district court, on Williams'
motion, ordered a nonsuit without prejudice to refile in state court. On FMCC's
appeal, this court reversed and remanded with directions to the district court
to rule on the motion for judgment notwithstanding the verdict. The district
court entered judgment notwithstanding the verdict for FMCC, and this appeal
followed.

Article 9 of the Uniform Commercial Code (UCC), which Arkansas has adopted and
codified as Ark.Stat.Ann. {\S} 85-9-503 (Supp.1981), provides in pertinent
part:
\begin{quote}
Unless otherwise agreed, a secured party has on default the right to take
possession of the collateral. In taking possession, a secured party may proceed
without judicial process if this can be done without breach of the peace
\ldots.\readingfootnote{4}{It is generally considered that
the objectives of this section are (1) to benefit creditors in permitting them
to realize collateral without having to resort to judicial process; (2) to
benefit debtors in general by making credit available at lower costs; and (3)
to support a public policy discouraging extrajudicial acts by citizens when
those acts are fraught with the likelihood of resulting violence.}
\end{quote}

In \textit{Ford Motor Credit Co. v. Herring}, 589 S.W.2d 584, 586 (Ark. 1979),
which involved an alleged conversion arising out of a repossession, the Supreme
Court of Arkansas cited Section 85-9-503 and referred to its previous holdings
as follows:
\begin{quote}
In pre-code cases, we have sustained a finding of conversion only where force,
or threats of force, or risk of invoking violence, accompanied the
repossession.
\end{quote}
The thrust of Williams' argument on appeal is that the repossession was
accomplished by the risk of invoking violence. The district judge who presided
at the trial commented on her theory in his memorandum opinion:
\begin{quote}
Mrs. Williams herself admitted that the men who repossessed her automobile were
very polite and complied with her requests. The evidence does not reveal that
they performed any act which was oppressive, threatening or tended to cause
physical violence. Unlike the situation presented in \textit{Manhattan Credit
Co. v. Brewer}, \textit{supra}, it was not shown that Mrs. Williams would have
been forced to resort to physical violence to stop the men from leaving with
her automobile.
\end{quote}
In the pre-Code case \textit{Manhattan Credit Co. v. Brewer}, S.W.2d 765 (Ark.
1961), the court held that a breach of peace occurred when the debtor and her
husband confronted the creditor's agent during the act of repossession and
clearly objected to the repossession. In \textit{Manhattan}, the court examined
holdings of earlier cases in which repossessions were deemed to have been
accomplished without any breach of the peace. In particular, the Supreme Court
of Arkansas discussed the case of \textit{Rutledge v. Universal C.I.T. Credit
Corp}, 237 S.W.2d 469 (Ark. 1951). In \textit{Rutledge}, the court found no
breach of the peace when the repossessor acquired keys to the automobile,
confronted the debtor and his wife, informed them he was going to take the car,
and immediately proceeded to do so. As the \textit{Rutledge} court explained
and the \textit{Manhattan} court reiterated, a breach of the peace did not
occur when the ``Appellant [debtor-possessor] did not give his permission but
he did not object.'' \textit{Manhattan}, \textit{supra}, 341 S.W.2d at 767-68;
\textit{Rutledge}, \textit{supra}, 237 S.W.2d at 470.

We have read the transcript of the trial. There is no material dispute in the
evidence, and the district court has correctly summarized it. Cathy Williams
did not raise an objection to the taking, and the repossession was accomplished
without any incident which might tend to provoke violence.

Appellees deserve something less than commendation for the taking during the
night time sleeping hours, but it is clear that viewing the facts in the light
most favorable to Williams, the taking was a legal repossession under the laws
of the State of Arkansas. The evidence does not support the verdict of the
jury. FMCC is entitled to judgment notwithstanding the verdict.

\opinion \textsc{Heaney}, Circuit Judge, dissenting.

The only issue is whether the repossession of appellant's automobile constituted
a breach of the peace by creating a ``risk of invoking violence.'' \textit{See}
\textit{Ford Motor Credit Co. v. Herring}, 589 S.W.2d 584, 586 (Ark. 1979). The
trial jury found that it did and awarded \$5,000 for conversion. Because that
determination was in my view a reasonable one, I dissent from the Court's
decision to overturn it.

Cathy Williams was a single parent living with her two small children in a
trailer home in Van Buren, Arkansas. On December 1, 1977, at approximately 4:30
a.m., she was awakened by noises in her driveway. She went into the night to
investigate and discovered a wrecker and its crew in the process of towing away
her car. According to the trial court, ``she ran outside to stop them \ldots
but she made no \textit{strenuous} protests to their actions.'' (Emphasis
added.) In fact, the wrecker crew stepped between her and the car when she
sought to retrieve personal items from inside it, although the men retrieved
some of the items for her. The commotion created by the incident awakened
neighbors in the vicinity.

Facing the wrecker crew in the dead of night, Cathy Williams did everything she
could to stop them, short of introducing physical force to meet the presence of
the crew. The confrontation did not result in violence only because Ms.
Williams did not take such steps and was otherwise powerless to stop the crew.

The controlling law is the UCC, which authorizes self-help repossession only
when such is done ``without breach of the peace \ldots.'' Ark.Stat.Ann. {\S}
85-9-503 (Supp.1981). The majority recognizes that one important policy
consideration underlying this restriction is to discourage ``extrajudicial acts
by citizens when those acts are fraught with the likelihood of resulting
violence.'' \textit{Supra}, at 719. Despite this, the majority holds that no
reasonable jury could find that the confrontation in Cathy Williams' driveway
at 4:30 a.m. created a risk of violence. I cannot agree. At a minimum, the
largely undisputed facts created a jury question. The jury found a breach of
the peace and this Court has no sound, much less compelling, reason to overturn
that determination.

Indeed, I would think that sound application of the self-help limitation might
require a directed verdict in favor of Ms. Williams, but certainly not against
her. If a ``night raid'' is conducted without detection and confrontation,
then, of course, there could be no breach of the peace. But where the invasion
is detected and a confrontation ensues, the repossessor should be under a duty
to retreat and turn to judicial process. The alternative which the majority
embraces is to allow a repossessor to proceed following confrontation unless
and until violence results in fact. Such a rule invites tragic consequences
which the law should seek to prevent, not to encourage. I would reverse the
trial court and reinstate the jury's verdict.

