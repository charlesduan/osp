\reading{M\&I Western State Bank v. Wilson}
\readingcite{493 N.W.2d 387 (Wisc. Ct. App. 1992)}

\opinion \textsc{Anderson}, Judge.

Darin Treleven appeals from a judgment of the trial court which awarded
possession of a truck owned by Marilyn A. Wilson to the M~\&~I Western State
Bank (bank). Because the earlier release of the truck was a conditional release
and the bank had notice of Treleven's lien through his possession of the truck,
we reverse.

The bank holds a security interest in a 1978 Peterbilt truck owned by Wilson.
Treleven repaired the truck seven times, each time releasing the vehicle to
Wilson so she could earn the money to pay Treleven for the repairs. The repairs
were invoiced between November 20, 1990 and April 23, 1991.

After Wilson defaulted on her payments to the bank, the bank commenced a
replevin action. The parties made a repayment agreement; however, Wilson again
defaulted and the bank obtained a judgment of replevin on April 9, 1990. The
sheriff attempted to enforce the judgment but was unable to locate the truck.
On May 12, 1991, employees of the bank saw the vehicle and followed it to
Treleven's place of business, D.T. Truck Repair, Inc. The sheriff again tried
to serve the writ of execution, but Treleven refused to release the vehicle,
asserting that he held a mechanic's lien for services rendered.

After the attempted levy, the bank filed a second replevin action to determine
who was entitled to possession of the truck and named Treleven as a third-party
defendant. At the date of the hearing, Treleven still was owed \$3497.26 for
the repairs plus \$1273.10 for interest and storage as of the date of the
hearing, January 30, 1992. The bank's balance as of January 2, 1992 was
\$3032.16. The bank's estimate of the value of the truck is approximately
\$3000. If this estimate is correct, only the lien with first priority would be
paid from the proceeds of the sale of the truck.

The trial court held that Treleven's release of the vehicle to Wilson
constituted a waiver of Treleven's lien as to the bank and that the bank's lien
had priority. The trial court ordered the bank to take possession and conduct a
sale of the truck. On appeal, Treleven argues that the conditional release of
the truck to the owner does not amount to a waiver of the lien and,
alternatively, that he should be able to recover from the bank on the theory of
unjust enrichment. Because we agree that the conditional release and regained
possession do not waive Treleven's mechanic's lien or affect its priority over
the prior secured interest, we do not have to address Treleven's unjust
enrichment claim.

It is not disputed that before Treleven released possession of the truck, he had
a mechanic's lien on Wilson's truck. Section 779.41(1), Stats., governs
mechanic's liens and states in part:
\begin{quote}
Every mechanic and every keeper of a garage or shop, and every employer of a
mechanic who transports, makes, alters, repairs or does any work on personal
property at the request of the owner or legal possessor of the personal
property, has a \textit{lien on the personal property} for the just and
reasonable charges therefor, including any parts, accessories, materials or
supplies furnished in connection therewith and \textit{may retain possession of
the personal property until the charges are paid}. [Emphasis added.]
\end{quote}
It also is not disputed that before Treleven released the truck to Wilson,
Treleven's mechanic's lien had priority over the bank's security interest.
Section 409.310, Stats., states:
\begin{quote}
When a person in the ordinary course of his business furnishes services or
materials with respect to goods subject to a security interest, \textit{a lien
upon goods in the possession of such person given by statute or rule of law for
such materials or services takes priority over a perfected security interest}
unless the lien is statutory and the statute expressly provides otherwise.
[Emphasis added.]
\end{quote}

Section 409.310 gave Treleven's mechanic's lien priority over the security
interest because Treleven was in possession of the truck, Treleven's lien was
created by sec. 779.41(1), Stats., and sec. 779.41(1) does not expressly
address the priority given to the lien created.

The issue in this case is whether the mechanic, by allowing the owner to use her
vehicle on a temporary basis before paying the repair bill, lost the lien or
its priority on that vehicle. The interpretation of statutes is a question of
law which we review de novo. We first must examine the language of sec.
779.41(1), Stats., to see if the relinquishment and resumption of possession
have any affect on the existence of Treleven's mechanic's lien. Section
779.41(1) provides that a mechanic ``may retain possession of the personal
property until the charges are paid.'' This provision allows the mechanic to
keep a customer's property until the mechanic has been paid, without a court
order. However, once the mechanic has relinquished possession of the vehicle,
this statute does not provide the mechanic with a remedy even if the bill has
not been paid. The statute also does not tell us whether the mechanic must
retain possession of the vehicle to retain the lien---it states only that the
mechanic ``may retain possession.''

But the mechanic's lien statute may not be interpreted in a vacuum.
``[M]echanic's lien laws provide \textit{new and additional remedies} to those
of the common law and are to be liberally construed to accomplish their
equitable purpose of aiding materialmen and laborers to obtain compensation for
material used and services bestowed upon property of another and thereby
enhancing its value.'' \textit{Wiedenbeck--Dobelin Co. v. Mahoney}, 152 N.W.
479, 481 (Wisc. 1915) (emphasis added). Accordingly, in addition to the
statutory language of sec. 779.41(1), Stats., we may look to the common law of
mechanic's liens and those Wisconsin decisions incorporating common law
principles into the statutory mechanic's lien law to determine whether
Treleven's lien survives.

Treleven argues that according to \textit{Sensenbrenner v. Mathews}, 3 N.W. 599,
600 (Wis. 1879), the delivery of the vehicle to the owner must be both
voluntary and unconditional in order to constitute a waiver of the lien.
Treleven maintains that because he returned the vehicle to the owner so she
could pay for the repairs and the allowed use was only on a temporary basis,
the delivery of the vehicle was conditional and his lien survives. The bank
also relies on \textit{Sensenbrenner} for its argument that Treleven waived his
lien by releasing the vehicle to Wilson. Alternatively, the bank asserts that
even if the lien was not destroyed between Treleven and Wilson when the vehicle
was conditionally released to Wilson, the lien was destroyed as to third
persons.

Because \textit{Sensenbrenner} is distinguishable on its facts from the present
case, neither party's reliance on that case is warranted. The court in
\textit{Sensenbrenner} found that the delivery of a buggy by the mechanic to
the owner was unconditional and held that this unconditional delivery operated
as a waiver of the lien. In contrast, Treleven's release of the vehicle was
conditional---\emph{Sensenbrenner} says nothing of  the effect of a conditional
release to the owner. \emph{Sensenbrenner} also does not explicitly hold that
the only way to waive a lien is through the voluntary and unconditional release
of the property; \emph{Sensenbrenner} merely states that this is one way to
waive a lien. For these reasons, \emph{Sensenbrenner} is not controlling
precedent based on the facts of this case.

No Wisconsin court has decided whether the lien is lost once the mechanic
conditionally releases the vehicle to the owner. The general and modern rule
can be found in Restatement of Security {\S} 80 (1941). This rule states that
when the bailor (owner) is under an obligation to return the vehicle to the
lienor (mechanic), the lien is revived upon the recovery of the vehicle,
subject only to the interests of bona fide purchasers for value and attaching
or levying creditors who do not have notice of the lienor's interest.

The bank would like a rule that upon a conditional release, the lien is lost as
to all third parties. The Restatement reflects a more balanced view,
recognizing that not all interests of third parties are affected by the
conditional release. While the mechanic retains possession, third parties at
least would have constructive notice of the mechanic's lien because they would
be expected to examine the property in the mechanic's possession and be
expected to know of the mechanic's lien statute. After a conditional release,
those parties purchasing the vehicle, extending new credit, or levying on the
vehicle would be vulnerable because even after examinations of the motor
vehicle filings and the vehicle, there would be no way for them to know of the
mechanic's prior interest. A creditor whose interest arose before the
mechanic's lien would not have this concern. At the time the creditor extends
credit, it is presumed to know the mechanic's lien statutes which could
subordinate its interest to that of a mechanic making a later repair. This is a
known risk to the creditor. A creditor also has the opportunity to protect
itself by writing into the security agreement that all subsequent repairs must
be approved by the creditor.

Once the mechanic's lien arises, in most circumstances, the later conditional
release does no further damage to the prior creditor and actually can be
advantageous to the creditor. For example, in a case such as this where the
vehicle is necessary to the owner's business, the conditional release allows
the owner to generate cash to pay off the mechanic's lien and make payments on
the creditor's prior loan. If the mechanic were forced to keep possession of
the vehicle, the owner would be unable to raise the cash to pay off either the
mechanic or the creditor.

The circumstance where a prior creditor could be damaged by the conditional
release also is covered by the Restatement. If a prior creditor does not have
notice of the mechanic's lien and goes through the expense of levying upon the
vehicle while it is in the owner's possession, then the levying creditor is
accorded the same protection as the bona fide purchaser for value or the new
attaching creditor. This rule gives the prior creditor a ``window of
opportunity'' to levy, but the mechanic can protect the lien by notifying prior
creditors of the conditional release arrangement.

For the reasons stated above, we reject the bank's argument that a conditional
release of the vehicle destroys the lien as to all third parties. Instead, we
adopt the Restatement's rule that upon a conditional release, the lien is
enforceable against all parties except a bona fide purchaser for value or a
subsequent attaching or levying creditor who has no notice of the mechanic's
interest. Upon the resumption of possession, the lien is revived and retains
its priority as before the release, except it is subordinate to the bona fide
purchaser or attaching or levying creditor. Applying this rule to the facts of
the case, it is apparent that the mechanic's lien is superior to the bank's
security interest. The fact that the truck was found at the mechanic's place of
business well after the repairs were made supports Treleven's claim that the
release of the vehicle was conditional. Furthermore, the bank is not afforded
the protection given to the levying creditor because the sheriff levied upon
the vehicle while it was in Treleven's possession, and thus had notice of
Treleven's interest.

Because Treleven's lien was not waived by the conditional release under sec.
779.41(1), Stats., we next must examine whether the conditional release
destroyed the lien's priority under sec. 409.310, Stats. Neither party
addressed this issue, but commentary and cases interpreting Uniform Commercial
Code {\S} 9-310, the model upon which sec. 409.310 is based, make clear that
the possession requirement of this statute is separate from any possession
requirement of the underlying mechanic's lien.

U.C.C. {\S} 9-310 gives priority only to the mechanic in possession of the
vehicle. It is uniformly held that if the mechanic voluntarily gives up
possession of the vehicle, {\S} 9-310 cannot be relied upon by the mechanic to
give his lien priority over the prior secured interest. \textit{See}
\textit{United States v. Crittenden}, 563 F.2d 678, 691 (5th Cir. 1977),
\textit{vacated and remanded}, 440 U.S. 715 (1979); \textit{In re Glenn}, 20
B.R. 98, 99 (Bankr. E.D. Tenn. 1982); \textit{Forrest Cate Ford, Inc. v.
Fryar}, 465 S.W.2d 882, 884 (Tenn. Ct. App. 1970).

The question then becomes whether the resumption of possession will allow sec.
409.310, Stats., to be applied to give the mechanic's lien priority. The
statute's language does not tell us whether continuous possession is required.
When a statute is ambiguous we must look to other sources to determine
legislative intent. Among the few courts that have decided this issue, the
jurisdictions do not agree as to the effect of resuming possession under {\S}
9-310. The three cases discussing this issue the most thoroughly are
\textit{Glenn}, \textit{Crittenden} and \textit{Thorp Commercial Corp. v.
Mississippi Road Supply Co.}, 348 So. 2d 1016 (Miss. 1977).

The opinion of the Mississippi Supreme Court in \textit{Thorp} held that the
mechanic retained priority under the Mississippi equivalent to {\S} 9-310 when
he resumed possession of equipment. The court reasoned that the status or
rights of the parties did not change between the date the mechanic lost
possession of the equipment and the date it was restored to the possession of
the mechanic. Furthermore, the court recognized that the secured party was not
and could not be prejudiced by the restoration. Finally, the court concluded
that because the Mississippi equivalent of {\S}~9-310 did not clearly express
an intention to reverse long-established principles of law, {\S} 9-310 had to
be read together with the older mechanic's lien statute and prior case law
which established that mechanic's liens take priority over prior security
interests. These justifications supported the court's opinion that priority
status of the mechanic's lien was retained under {\S} 9-310 when the mechanic
regained possession.

\textit{Glenn} and the dissenting opinion in \textit{Thorp} stated that the
priority of the mechanic's lien is lost under statutes based on {\S} 9-310
when there is a lapse in the mechanic's possession. \textit{Glenn} reasoned
that a rule which allowed the reinstatement of priority ``would create an
ever-present dangerous uncertainty for parties, including prior secured
parties, who deal with the debtor with respect to goods in his possession''
because the prior secured party would have no notice of the mechanic's lien.
\textit{Glenn}, 20 B.R. at 99. Glenn also based its conclusion on the same
concerns of the dissent in \textit{Thorp}---a rule reinstating priority under
the statute would permit the priority of the creditors to be determined by the
debtor.
\begin{quote}
If he chooses to return property once relinquished by a repairman, the repairman
prevails, but if he chooses not to relinquish possession of the property the
secured creditor prevails\ldots. [A rule reinstating priority under the
statute] invites competition for possession between a secured party and a
repairman who has previously relinquished possession of the property.
\end{quote}
\textit{Id}. at 100--01.

The Fifth Circuit Court of Appeals held that a mechanic retained his priority
over a prior security interest only to the extent that the mechanic
continuously possessed the collateral. \textit{Crittenden}, 563 F.2d at 691.
The court analogized {\S} 9-310 to 26 U.S.C. {\S} 6323(b)(5), a provision of
the Federal Tax Lien Act, which gives priority to the mechanic's lien only if
the mechanic ``is, and has been, continuously in possession of such property
from the time such lien arose.'' 26 U.S.C. {\S} 6323(b)(5). The court justified
the continuous possession requirement by reasoning that while considerations of
equity and fairness created the mechanic's lien exception to the normal
priority rules, at some point when the mechanic gives up possession and the
repairs were made in the more distant past the mechanic's interest becomes
indistinguishable from the ordinary creditor.

In light of the longstanding Wisconsin policy of protecting materialmen and
laborers, we find the Mississippi court's opinion in Thorp to be the most
persuasive. The bank has not presented any facts which would show how its
rights were affected or its interest was prejudiced by the release of the
property to Wilson and Treleven's subsequent repossession. If anything, the
facts show that the bank was better off through the conditional release because
it afforded Wilson the resources to pay off both debts.

Like Mississippi's law in \textit{Thorp}, Wisconsin case law decided prior to
the enactment of sec. 409.310, Stats., gave priority to a mechanic's lien over
a prior security interest. \textit{See} \textit{Jesse A. Smith Auto Co. v.
Kaestner}, 159 N.W. 738 (Wisc. 1916). Wisconsin's enactment of sec. 409.310 did
not expressly state that its effect was to displace prior law in this area.
Commentary to the Uniform Commercial Code reveals the drafter's view that {\S}
9-310 was to reverse prior case law which subordinated the mechanic's lien to
prior security interests, but it does not state how the rule was to affect
prior decisions holding the mechanic's lien superior. See U.C.C. {\S} 9-310
comment 2. Because Wisconsin's prior case law and sec. 409.310 can be read in a
consistent manner, we decline to interpret the statute otherwise.

Finally, but not least importantly, the plain language of sec. 409.310, Stats.,
gives priority to the mechanic ``in possession.'' It does not require
``continuous possession'' or ``retained possession.'' We must construe laws
relating to mechanic's liens in a way to accomplish their equitable purpose of
aiding mechanics in obtaining compensation.

The Fifth Circuit's opinion in \textit{Crittenden} which read the continuous
possession requirement into {\S} 9-310 is not persuasive. In
\textit{Crittenden}, the Fifth Circuit was interested in formulating a federal
standard to determine priorities under the Uniform Commercial Code. Thus, it
looked to the Federal Tax Lien Act for guidance in its interpretation of the
``possession'' requirement of {\S} 9-310. \textit{Crittenden}, 563 F.2d at
691. On appeal the Supreme Court reversed, stating that the court should not be
looking to federal standards to determine priorities, but should apply
Georgia's statutes. \textit{United States v. Kimbell Foods}, 440 U.S. 715, 740
(1979). On remand, the Fifth Circuit held that Georgia's priority statute was
basically the same as model {\S} 9-310 and, without discussion, applied the
same interpretation of the statute to the facts in the case. \textit{United
States v. Crittenden}, 600 F.2d 478, 479--80 (5th Cir. 1979). Unlike the Fifth
Circuit's first \textit{Crittenden} opinion, we are not concerned with
formulating a national standard and do not need to look at other federal laws
interpreting ``possession;'' under Wisconsin law, we must interpret sec.
409.310, Stats., in a way that aids the mechanic in obtaining compensation. It
is not in a mechanic's best interest to interpret ``possession'' in sec.
409.310 as ``continuous possession,'' and we decline to do so. Therefore,
because Treleven was in possession of the vehicle at the time the bank's lien
was enforced, Treleven's mechanic's lien had priority over the bank's interest
under sec. 409.310.

