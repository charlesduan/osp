\expected{hinman-pacific-air}

\item Did the court in \textit{Hinman} ``find'' the law of property as it
applies to the airspace above land? Did it ``change'' the law in this regard?
Or did it---as Felix Cohen argued---``create and distribute a new source of
economic wealth or power''?

\item Does the court say that Hinman will never be able to obtain the relief
sought? Are there any circumstances in which an injunction to restrict
overflights to an altitude of over 150 feet (or any altitude) could be awarded
under the court's analysis?

\item The court justified its ruling in \textit{Hinman}, at least in part, by
reference to the ``practical result'' that would follow a finding in the
landowner's favor. What would that ``practical result'' be, and why did the
court feel the need to avoid it? Is avoiding such undesirable ``practical
results'' an acceptable basis for making a determination as to whether
something is a person's ``property''?

\defwebsite{koebler-sky-lawn}{
Jason Koebler, The Sky's Not Your Lawn: Man Wins Lawsuit After Neighbor
Shotgunned His Drone, Vice: Motherboard (June 28, 2015),
http://motherboard.vice.com/read/the-skys-not-your-lawn-man-wins-lawsuit-after-neighbor-shotgunned-his-drone
}

\item \textbf{Drones.} The increasing availability of personal aerial robots
(``drones'') is threatening to bring \textit{Hinman} back into the spotlight.
In November of 2014, a hobbyist was flying a custom-built ``hexacopter'' over
his parents' farm in California, when a neighbor's son shot it out of the sky
with a shotgun. The neighbor claimed the drone had been flying over his land,
though the drone owner disputed this. In any event, the drone owner demanded
compensation for damage to the drone, and the neighbor refused. They ended up
in small claims court where the neighbor was held liable for \$850 in damages
and court costs, on grounds that he ``acted unreasonably in having his son
shoot the drone down regardless of whether it was over his property or not.''
\sentence{see koebler-sky-lawn}.

Imagine that instead of (or in addition to) having his son use the drone for
target practice, the farmer had called the police to make a complaint of
criminal trespass, or sued the drone owner for trespass. What result? Would it
matter how high the drone was flying? Would it matter whether the drone was
equipped with a camera? (Recall that the right to exclude is not the only right
of owners; trespass may not be our farmer's only recourse.
\having{nuisance}{%
We considered some analogous factual scenarios in our unit on Nuisance.%
}{%
We will consider some analogous factual scenarios in our unit on Nuisance.%
}{})

\item Would the ``practical result'' of a finding for the landowner in
\textit{Hinman} necessarily be the same as the ``practical result'' of a
finding in favor of a landowner suing the operator of a drone in the airspace
over her land? Again, would it matter how high the drone was flying, or whether
it was equipped with a camera?

