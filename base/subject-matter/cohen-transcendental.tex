\reading[Cohen, \emph{Transcendental Nonsense}]{Felix Cohen,
\textit{Transcendental Nonsense and the Functional Approach}}
\readingcite{35 \textsc{Colum. L. Rev.} 809, 814-817 (1935)}

There was once a theory that the law of trade marks and trade-names was an
attempt to protect the consumer against the ``passing off'' of inferior goods
under misleading labels. Increasingly the courts have departed from any such
theory and have come to view this branch of law as a protection of property
rights in divers economically valuable sale devices. In practice, injunctive
relief is being extended today to realms where no actual danger of confusion to
the consumer is present, and this extension has been vigorously supported and
encouraged by leading writers in the field. Conceivably this extension might be
justified by a demonstration that privately controlled sales devices serve as a
psychologic base for the power of business monopolies, and that such monopolies
are socially valuable in modern civilization. But no such line of argument has
ever been put forward by courts or scholars advocating increased legal
protection of trade names and similar devices. For if they advanced any such
argument, it might seem that they were taking sides upon controversial issues
of politics and economics. Courts and scholars, therefore, have taken refuge in
a vicious circle to which no obviously extra-legal facts can gain admittance.
The current legal argument runs: One who by the ingenuity of his advertising or
the quality of his product has induced consumer responsiveness to a particular
name, symbol, form of packaging, etc., has thereby created a thing of value; a
thing of value is property; the creator of property is entitled to protection
against third parties who seek to deprive him of his property. This argument
may be embellished, in particular cases, with animadversions upon the selfish
motives of the infringing defendant, a summary of the plaintiff's evidence
(naturally uncontradicted) as to the amount of money he has spent in
advertising, and insinuations (seldom factually supported) as to the
inferiority of the infringing defendant's product. 

The vicious circle inherent in this reasoning is plain. It purports to base
legal protection upon economic value, when, as a matter of actual fact, the
economic value of a sales device depends upon the extent to which it will be
legally protected. If commercial exploitation of the word ``Palmolive'' is not
restricted to a single firm, the word will be of no more economic value to any
particular firm than a convenient size, shape, mode of packing, or manner of
advertising, common in the trade. Not being of economic value to any particular
firm, the word would be regarded by courts as ``not property,'' and no
injunction would be issued. In other words, the fact that courts did not
protect the word would make the word valueless, and the fact that it was
valueless would then be regarded as a reason for not protecting it. Ridiculous
as this vicious circle seems, it is logically as conclusive or inconclusive as
the opposite vicious circle, which accepts the fact that courts do protect
private exploitation of a given word as a reason why private exploitation of
that word should be protected. 

The circularity of legal reasoning in the whole field of unfair competition is
veiled by the ``thingification'' of \textit{property}. Legal language portrays
courts as examining commercial words and finding, somewhere inhering in them,
\textit{property rights}. It is by virtue of the property right which the
plaintiff has acquired in the word that he is entitled to an injunction or an
award of damages. According to the recognized authorities on the law of unfair
competition, courts are not \textit{creating} property, but are merely
\textit{recognizing} a pre-existent Something. 

The theory that judicial decisions in the field of unfair competition law are
merely recognitions of a supernatural Something that is immanent in certain
trade names and symbols is, of course, one of the numerous progeny of the
theory that judges have nothing to do with making the law, but merely recognize
pre-existent truths not made by mortal men. The effect of this theory, in the
law of unfair competition as elsewhere, is to dull lay understanding and
criticism of what courts do in fact. 

What courts are actually doing, of course, in unfair competition cases, is to
create and distribute a new source of economic wealth or power. Language is
socially useful apart from law, as air is socially useful, but neither language
nor air is a source of economic wealth unless some people are prevented from
using these resources in ways that are permitted to other people. That is to
say, property is a function of inequality. If courts, for instance, should
prevent a man from breathing any air which had been breathed by another
(within, say, a reasonable statute of limitations), those individuals who
breathed most vigorously and were quickest and wisest in selecting desirable
locations in which to breathe (or made the most advantageous contracts with
such individuals) would, by virtue of their property right in certain volumes
of air, come to exercise and enjoy a peculiar economic advantage, which might,
through various modes of economic exchange, be turned into other forms of
economic advantage, e.g. the ownership of newspapers or fine clothing. So, if
courts prevent a man from exploiting certain forms of language which another
has already begun to exploit, the second user will be at the economic
disadvantage of having to pay the first user for the privilege of using similar
language or else of having to use less appealing language (generally) in
presenting his commodities to the public. 

Courts, then, in establishing inequality in the commercial exploitation of
language are creating economic wealth and property, creating property not, of
course, \textit{ex nihilo}, but out of the materials of social fact, commercial
custom, and popular moral faiths or prejudices. It does not follow, except by
the fallacy of composition, that in creating new private property courts are
benefiting society. Whether they are benefiting society depends upon a series
of questions which courts and scholars dealing with this field of law have not
seriously considered. Is there, for practical purposes, an unlimited supply of
equally attractive words under which any commodity can be sold, so that the
second seller of the commodity is at no commercial disadvantage if he is forced
to avoid the word or words chosen by the first seller? If this is not the case,
i.e. if peculiar emotional contexts give one word more sales appeal than any
other word suitable for the same product, should the peculiar appeal of that
word be granted by the state, without payment, to the first occupier? Is this
homestead law for the English language necessary in order to induce the first
occupier to use the most attractive word in selling his product? If, on the
other hand, all words are originally alike in commercial potentiality, but
become differentiated by advertising and other forms of commercial
exploitation, is this type of business pressure a good thing, and should it be
encouraged by offering legal rewards for the private exploitation of popular
linguistic habits and prejudices? To what extent is differentiation of
commodities by trade names a help to the consumer in buying wisely? To what
extent is the exclusive power to exploit an attractive word, and to alter the
quality of the things to which the word is attached, a means of deceiving
consumers into purchasing inferior goods? 

Without a frank facing of these and similar questions, legal reasoning on the
subject of trade names is simply economic prejudice masquerading in the cloak
of legal logic. The prejudice that identifies the interests of the plaintiff in
unfair competition cases with the interests of business and identifies the
interests of business with the interests of society, will not be critically
examined by courts and legal scholars until it is recognized and formulated. It
will not be recognized or formulated so long as the hypostatization of
``property rights'' conceals the circularity of legal reasoning.

