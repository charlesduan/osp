In this unit we will consider the various types of things that attract the legal
label ``property.'' Let us begin with some examples to pump our intuitions. In
light of our discussion of what it means to own something, which of the
following things can be usefully thought of as your ``property''?
\begin{itemize}
\item your home or apartment
\item your car or bike
\item your computer
\item the software on your computer
\item the emails stored on your computer
\item the emails stored on your cloud-based email service
\item your bank account
\item the money in your bank account
\item the money you lent to your friend that hasn't been repaid
\item the money your friend lent to you that you haven't paid back
\item the things you bought with the money your friend lent to you that you
haven't paid back
\item your pet dog
\item the rats in your animal research lab
\item your dairy cow
\item the pig you're raising for meat
\item your prescription medications
\item your doctor's/pharmacist's/insurance company's records of your
prescription medications
\item your handwritten diary
\item your unpublished novel
\item your published novel
\item your social media profiles and content
\item your password-protected blog
\end{itemize}
Does categorizing any of these items as ``property'' or ``not property''
meaningfully assist in the analysis of any legal problems? Particularly legal
disputes that arise over questions of access to or use of any of these things?
Why might we choose to recognize (or refuse to recognize) these or other items
as ``property''? 

You may notice there is something of a chicken-and-egg problem here. Is the
label ``property'' a premise or a conclusion? Can we arrive at the label
without resorting to circular reasoning? When we say something is a person's
property, or that someone has a ``property right,'' is that because we have
examined the qualities and characteristics of the thing and its relation to the
person, and \textit{determined} that they are all consistent with some coherent
notion of property ownership? Or is calling something ``property'' a mere
\textit{assertion}, unconstrained by circumstances, that we make because we
want the \textit{consequences} of the label ``property'' to attach to that
thing for independent reasons? Is there a difference? Consider the following
classic discussion of this question:

\expectnext{cohen-transcendental}
