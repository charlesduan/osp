\reading{Hinman v. Pacific Air Transport}

\readingcite{84 F.2d 755 (9th Cir. 1936)}

HANEY, Circuit Judge.

Appellants allege{\dots} that they are the owners and in possession of 72 1/2
acres of real property in the city of Burbank, Los Angeles county, Cal.,
``together with a stratum of air-space superjacent to and overlying said tract
* * * and extending upwards * * * to such an altitude as plaintiffs * * * may
reasonably expect now or hereafter to utilize, use or occupy said airspace.
Without limiting said altitude or defining the upward extent of said stratum of
airspace or of plaintiff's ownership, utilization and possession thereof,
plaintiffs allege that they * * * may reasonably expect now and hereafter to
utilize, use and occupy said airspace and each and every portion thereof to an
altitude of not less than 150 feet above the surface of the land * * * ''\dots.

It is then alleged that defendants are engaged in the business of operating a
commercial air line, and that at all times ``after the month of May, 1929,
defendants daily, repeatedly and upon numerous occasions have disturbed,
invaded and trespassed upon the ownership and possession of plaintiffs'
tract''; that at said times defendants have operated aircraft in, across, and
through said airspace at altitudes less than 100 feet above the surface; that
plaintiffs notified defendants to desist from trespassing on said airspace; and
that defendants have disregarded said notice, unlawfully and against the will
of plaintiffs, and continue and threaten to continue such trespasses\dots. 
The prayer asks an injunction restraining the operation of the aircraft through
the airspace over plaintiffs' property and for [damages].

Appellees contend that it is settled law in California that the owner of land
has no property rights in superjacent airspace, either by code enactments or by
judicial decrees and that the ad coelum doctrine does not apply in California.
We have examined the statutes of California,\dots but we find nothing therein
to negative the ad coelum formula\dots. If we could accept and literally
construe the ad coelum doctrine, it would simplify the solution of this case;
however, we reject that doctrine. We think it is not the law, and that it never
was the law.

This formula ``from the center of the earth to the sky'' was invented at some
remote time in the past when the use of space above land actual or conceivable
was confined to narrow limits, and simply meant that the owner of the land
could use the overlying space to such an extent as he was able, and that no one
could ever interfere with that use.

This formula was never taken literally, but was a figurative phrase to express
the full and complete ownership of land and the right to whatever superjacent
airspace was necessary or convenient to the enjoyment of the land.

In applying a rule of law, or construing a statute or constitutional provision,
we cannot shut our eyes to common knowledge, the progress of civilization, or
the experience of mankind. A literal construction of this formula will bring
about an absurdity. The sky has no definite location. It is that which presents
itself to the eye when looking upward; as we approach it, it recedes. There can
be no ownership of infinity, nor can equity prevent a supposed violation of an
abstract conception.

The appellants' case, then, rests upon the assumption that as owners of the soil
they have an absolute and present title to all the space above the earth's
surface, owned by them, to such a height as is, or may become, useful to the
enjoyment of their land. This height, the appellants assert in the bill, is of
indefinite distance, but not less than 150 feet.

If the appellants are correct in this premise, it would seem that they would
have such a title to the airspace claimed, as an incident to their ownership of
the land, that they could protect such a title as if it were an ordinary
interest in real property. Let us then examine the appellants' premise. They do
not seek to maintain that the ownership of the land actually extends by
absolute and exclusive title upward to the sky and downward to the center of
the earth. They recognize that the space claimed must have some use, either
present or contemplated, and connected with the enjoyment of the land itself.

Title to the airspace unconnected with the use of land is inconceivable. Such a
right has never been asserted. It is a thing not known to the law.

Since, therefore, appellants must confine their claim to 150 feet of the
airspace above the land, to the use of the space as related to the enjoyment of
their land, to what extent, then, is this use necessary to perfect their title
to the airspace? Must the use be actual, as when the owner claims the space
above the earth occupied by a building constructed thereon; or does it suffice
if appellants establish merely that they may reasonably expect to use the
airspace now or at some indefinite future time?

This, then, is appellants' premise, and upon this proposition they rest their
case. Such an inquiry was never pursued in the history of jurisprudence until
the occasion is furnished by the common use of vehicles of the air.

We believe, and hold, that appellants' premise is unsound. The question
presented is applied to a new status and little aid can be found in actual
precedent. The solution is found in the application of elementary legal
principles. The first and foremost of these principles is that the very essence
and origin of the legal right of property is dominion over it. Property must
have been reclaimed from the general mass of the earth, and it must be capable
by its nature of exclusive possession. Without possession, no right in it can
be maintained.

The air, like the sea, is by its nature incapable of private ownership, except
in so far as one may actually use it. This principle was announced long ago by
Justinian. It is in fact the basis upon which practically all of our so-called
water codes are based.

We own so much of the space above the ground as we can occupy or make use of, in
connection with the enjoyment of our land. This right is not fixed. It varies
with our varying needs and is coextensive with them. The owner of land owns as
much of the space above him as he uses, but only so long as he uses it. All
that lies beyond belongs to the world.\dots Any use of such air or space by
others which is injurious to his land, or which constitutes an actual
interference with his possession or his beneficial use thereof, would be a
trespass for which he would have remedy. But any claim of the landowner beyond
this cannot find a precedent in law, nor support in reason.

\dots We cannot shut our eyes to the practical result of legal recognition of
the asserted claims of appellants herein, for it leads to a legal implication
to the effect that any use of airspace above the surface owner of land, without
his consent would be a trespass either by the operator of an airplane or a
radio operator. We will not foist any such chimerical concept of property
rights upon the jurisprudence of this country\dots.

Appellants are not entitled to injunctive relief upon the bill filed here,
because no facts are alleged with respect to circumstances of appellants' use
of the premises which will enable this court to infer that any actual or
substantial damage will accrue from the acts of the appellees complained of.

The case differs from the usual case of enjoining a trespass. Ordinarily, if a
trespass is committed upon land, the plaintiff is entitled to at least nominal
damages without proving or alleging any actual damage. In the instant case,
traversing the airspace above appellants' land is not, of itself, a trespass at
all, but it is a lawful act unless it is done under circumstances which will
cause injury to appellants' possession.

Appellants do not, therefore, in their bill state a case of trespass, unless
they allege a case of actual and substantial damage. The bill fails to do this.
It merely draws a naked conclusion as to damages without facts or circumstances
to support it. It follows that the complaint does not state a case for
injunctive relief\dots.

