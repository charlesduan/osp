\reading{United States v. Turoff}

\readingcite{701 F. Supp. 981 (E.D.N.Y. 1988)}

\opinion \textsc{Glasser}, District Judge:

Defendants have moved to dismiss the indictment in this case on the ground
that\ldots it fails to allege a violation of the mail fraud statute, 18 U.S.C.
{\S} 1341.

For the reasons stated below, defendants' motion is denied.

\readinghead{Facts}

According to the indictment, in late 1978, the TLC, which regulates the City's
medallion taxicabs, authorized the issuance of 100 temporary taxi medallions to
a corporation (``Research Cab Corporation'') to be formed by defendant Donald
Sherman. The purpose of the temporary medallions was to test the feasibility of
diesel engines in New York City taxicabs.

The indictment alleges that in late 1980, the TLC's chairman, defendant Turoff,
caused an additional 23 unauthorized medallions to be diverted to his
codefendants and placed on gasoline- and diesel-powered taxicabs registered to
Research Cab and to Tulip Cab Corporation. These taxicabs allegedly operated in
the City from late 1980 to early 1985. Defendants Donald and Ronald Sherman
allegedly deposited the proceeds from those taxicabs, which exceeded \$500,000,
in the bank account of a shell corporation (``Exdie Cab Corporation'').

Allegedly, defendants never paid the TLC the annual license renewal fees for the
unauthorized medallions. In connection with the conspiracy, the defendant
Turoff allegedly gave false and misleading information to the TLC Commissioners
and the Mayor's office, and destroyed TLC records on the Tulip Cab Corporation
and all the defendants allegedly gave false and misleading information to the
New York State Commission of Investigation. The indictment alleges fourteen
instances in which the mails were used to effectuate the scheme.

\readinghead{Discussion \\ I.}

The mail fraud statute under which defendants have been indicted was first
enacted in 1872. In its present form, it now reads:
\begin{quote}
Whoever, having devised or intending to devise any scheme or artifice to
defraud, or for obtaining money or property by means of false or fraudulent
pretenses, representations, or promises, or to sell, dispose of, loan,
exchange, alter, give away, distribute, supply, or furnish or procure for
unlawful use any counterfeit or spurious coin, obligation, security, or other
article, or anything represented to be or intimated or held out to be such
counterfeit or spurious article, for the purpose of executing such scheme or
artifice or attempting to do so, places in any post office or authorized
depository for mail matter, any matter or thing whatever to be sent or
delivered by the Postal Service, or takes or receives therefrom, any such
matter or thing, or knowingly causes to be delivered by mail according to the
direction thereon, or at the place at which it is directed to be delivered by
the person to whom it is addressed, any such matter or thing, shall be fined
not more than \$1,000 or imprisoned not more than five years, or both.
\end{quote}

Defendants move to dismiss the indictment on the ground that it does not state a
cognizable violation of the mail fraud statute as interpreted in
\textit{McNally v. United States}, 483 U.S. 350 (1987). In \textit{McNally},
the Supreme Court reversed the mail fraud convictions of Charles J. McNally and
James E. Gray on the ground that the mail fraud statute does not reach schemes
which violate ``the intangible right of the citizenry to good government.'' The
case involved a scheme devised by Gray, who held two top government posts in
the Kentucky state government, and Howard P. ``Sonny'' Hunt, a state Democratic
party chairman who had been given de facto power by the governor to select the
insurance agencies from which the state would buy its policies. Hunt selected a
certain agency as the state's agent for securing a workmen's compensation
policy, on the condition that that agency would share any resulting commissions
in excess of \$50,000 a year with twenty-one other insurance agencies specified
by Hunt. Among the designated agencies was one controlled by Hunt and Gray (who
had formed it for the exclusive purpose of obtaining the excess commissions).
McNally served as the agency's front man.\ldots

[McNally and Gray were convicted of mail fraud.]

The jury convicted defendants, and the Court of Appeals affirmed the
convictions, relying on many prior decisions holding that ``the mail fraud
statute proscribes schemes to defraud citizens of their intangible rights to
honest and impartial government.''

The Supreme Court reversed, holding that ``[t]he mail fraud statute clearly
protects property rights, but does not refer to the intangible right of the
citizenry to good government.'' The Court framed the issue in the case
narrowly:
\begin{quote}
The issue is thus whether a state officer violates the mail fraud statute if he
chooses an insurance agent to provide insurance for the State but specifies
that the agent must share its commissions with other named insurance agencies,
in one of which the officer has an ownership interest and hence profits when
his agency receives part of the commissions. We note that as the action comes
to us, there was no charge and the jury was not required to find that the
Commonwealth itself was defrauded of any money or property. It was not charged
that in the absence of the alleged scheme the Commonwealth would have paid a
lower premium or secured better insurance. Hunt and Gray received part of the
commissions but those commissions were not the Commonwealth's money. Nor was
the jury charged that to convict it must find that the Commonwealth was
deprived of control over how its money was spent. Indeed, the premium for
insurance would have been paid to some agency, and what Hunt and Gray did was
to assert control that the Commonwealth might not otherwise have made over the
commissions paid by the insurance company to its agent.\ldots We hold,
therefore,
that the jury instruction on the substantive mail fraud count permitted a
conviction for conduct not within the reach of {\S}
1341.\readingfootnote{2}{The narrowness of
\textit{McNally}'s holding was underscored in \textit{Carpenter v. United
States}, 484 U.S. 19, (1987), which held that a newspaper had a property right
under {\S} 1341 in the exclusive pre-publication use of confidential business
information, and noted that ``McNally did not limit the scope of {\S} 1341 to
tangible as distinguished from intangible property rights.''}\ldots
\end{quote}

Most significantly for this case, the Court in \textit{McNally} held that,
because the mail fraud statute ``had its origin in the desire to protect
individual property rights,\ldots any benefit which the Government derives from
the [mail fraud] statute must be limited to the Government's interest as
property-holder.'' Accordingly, in the present case, the government's failure
to demonstrate the City's interest ``as property-holder'' in the medallions
would be fatal to that charge in the indictment that is based upon the
fraudulent procurement of the medallions.

However, even if the court accepted this argument, the indictment would still
stand insofar as it is based on the scheme to avoid payment of license renewal
fees. Money is the most concrete and tangible of property. In \textit{Reiter v.
Sonotone Corp}., 442 U.S. 330 (1979), the Court stated: ``In its dictionary
definitions and in common usage `property' comprehends anything of material
value owned or possessed\dots . Money, of course, is a form of property.'' On
this basis alone, defendant's motion to dismiss the indictment must be denied.

As regards the medallions, the court concludes that the fraudulent
misappropriation of them deprived the City of a property interest cognizable
under the mail fraud statute.

Defendants cite \textit{United States v. Evans}, 844 F.2d 36 (2d Cir. 1988) for
the proposition that the City's interest in the medallions ``is ancillary to a
regulation, not to property.'' \textit{Id.}, 844 F.2d at 42. Evans concerned a
scheme to transfer arms regulated by the federal government from various
foreign nations to Iran. The scheme required defendants to deceive the
government about the true identity of the purchasing country in order to obtain
the necessary approval for the transaction. The government's right to regulate
such transfers arose either from a statutorily-required clause in the contract
between the United States and the original foreign buyer, or by regulation.

The Second Circuit, affirming the district court's dismissal of the mail and
wire fraud counts against defendants, held that the government had not shown
that it had some property interest in the arms. Furthermore, the court rejected
the government's contention that ``the right of the United States Government to
prevent the resale or retransfer of U.S. military weaponry from foreign nations
to other, unacceptable foreign powers'' constituted ``an interest in, and a
right to exercise control over, property'' for purposes of the mail fraud
statute. \textit{Id.}, 844 F.2d at 40.

In addressing the latter argument, the court rejected the government's analogies
to common law property rights. The court reasoned that, while a right to
control the future alienation and use of a thing can be a traditional property
right (e.g., the fee simple determinable, the fee simple subject to a condition
subsequent, the possibility of reverter, and the power of termination), that
does not mean that every such right is cognizable under the mail and wire fraud
statutes.\readingfootnote{4}{I note that the possessory and
future interests named are not intrinsically ``devices through which a
nonpossessor controls land'' or ``control[s] alienation.'' 844 F.2d at 41. The
estates in land described are expressions of the extent of one's present
interest in property measured in terms of time. The owner of a fee simple
determinable has a present, possessory interest in property which will continue
``until'' or ``so long as'' a specified event does or does not occur. The
possibility of reverter is the present interest one has in the future use and
enjoyment of the property when the fee simple determinable ends. The owner of a
fee simple subject to a condition subsequent has a present possessory interest
in property ``upon condition that'' or ``provided that'' a specified event does
or does not occur. The power of termination is the present interest one has in
the future use and enjoyment of that property upon the exercise of his power to
terminate the possessory estate. All the estates described are present property
interests in the sense that they are all descendible, devisable and alienable.
N.Y. Est. Powers \& Trusts Law {\S} 6--5.1 (McKinney 1967). That a person who
acquired either of those estates in property by or through a scheme or artifice
to defraud would acquire a present interest in property is beyond cavil.}
Specifically, the court noted that the government's right to control arms
transfers between foreign powers would never permit the United States to
possess the weapons in question, and had no effect on the purchaser's title to
the arms or the seller's right to profits from the sale. Rather, the regulatory
scheme governing such transfers ``substitutes for the traditional property
remedies of replevin, damages or specific performance, a substitution that is
further proof that the right is not property.'' \textit{Id.}, 844 F.2d at 41.
Moreover, the court expressed its reluctance to apply common law property rules
in the fundamentally different context of weapons transfers, which are governed
by foreign policy and human rights considerations in addition to the usual
economic laws of supply and demand.

The court summed up by finding that the government's interest in the weapons was
essentially regulatory:
\begin{quote}
All of these distinctions suggest to us that the government's interest here is
ancillary to a regulation, not to property. A law prohibiting a particular use
of a commodity that the government does not use or possess ordinarily does not
create a property right. If it did, many government regulations would create
property rights. For example, laws preventing the sale of heroin or the dumping
of toxic waste would create government property rights in the drugs or
chemicals. Admittedly, the line between regulation and property is difficult to
draw with scientific precision\ldots and we do not mean to imply that the
government never has a property interest in the limits it imposes on property
use.
\end{quote}
\textit{Id.}, 844 F.2d at 42 (citation omitted).

\textit{Evans} is distinguishable. As discussed above, in \textit{Evans} the
United States had no possessory interest in the weapons, nor did the deception
practiced by the defendants affect the purchaser's title to the weapons or the
seller's right to profit from the sale of the weapons. Here, defendants are
accused of taking 23 items of tangible personal property from the City's
possession. Title to those medallions in the hands of third persons would be
affected. Citation of authority is not required for the principle that a thief
cannot transfer title even to a bona fide purchaser for value. While the
government in Evans had no possessory interest in the weapons, the TLC in this
case did have a possessory interest in the medallions. It maintained them under
lock and key at its offices. It had title to them. An action for conversion of
those medallions would lie and either replevin or damages would be an available
and appropriate remedy.\ldots Given the impetus to return to the arcane
learning of the law of property prompted by McNally, a quotation from Book III
of Blackstone's Commentaries on the Laws of England (Lewis' Ed. 1902) seems
appropriate. At pages 145--46 that venerable author wrote:
\begin{quote}
The wrongful taking of goods being thus most clearly an injury, the next
consideration is, what remedy the Law of England has given for it. And this is,
in the first place, the restitution of the goods themselves so wrongfully
taken, with damages for the loss sustained by such unjust invasion; which is
effected by action of replevin;\ldots
\end{quote}

That the medallions themselves are a valuable, marketable commodity was adverted
to years ago by Professor Charles A. Reich in his seminal article entitled
\textit{The New Property}, 73 \textsc{Yale L.J.} 733 (1964).
He wrote, at page 735:
\begin{quote}
A New York City taxi medallion, which costs very little when originally obtained
from the city, can be sold for over twenty thousand dollars.
\end{quote}
In a footnote at that point, the author observed:
\begin{quote}
7. A New York Taxi Medallion is a piece of tin worth 300 times its weight in
gold. No new transferable medallions have been issued since 1937. Their value
in 1961 was estimated at \$21,000 to \$23,000; banks will lend up to \$13,000
on one. The cabbie pays the City only \$200 a year for his medallion. There is
a brisk trade in them: out of 11,800, about 600 changed hands in 1961. One
company, National Transportation Co., sold 100 medallions at \$21,000 each, a
transaction totaling \$2,100,000. A non-transferable license, of which there
are a few, has no market value. \textit{N.Y. Times}, Dec. 5, 1961, p. 46, col.
3.
\end{quote}

The government also contends that the medallion is, in essence, the equivalent
of an easement to use the city streets. At the risk of dwelling too long on the
esoterica of property, the medallions could not properly be equated with
easements. An easement is generally appurtenant, which is to say that it is a
right which the owner of one parcel of land (the dominant tenement) may
exercise in or over the land of another (the servient tenement) for the benefit
of the former. An easement in gross is a right created in a person to use the
land of another, which the owner of that easement may enjoy even though he does
not own or possess a dominant estate. Although the concept of an easement in
gross has been recognized, such an easement is rare. The government's
contention would have been more technically correct had it characterized the
medallion as a ``special franchise'' which confers a right to do something in
the public highway which, except for the grant, would be a trespass. 

A franchise is property. It is assignable, taxable and transmissible.
\textit{Hatfield v. Straus}, 82 N.E. 172 (1907). A mere license, on the other
hand, is nothing more than a personal, revocable privilege. See, e.g.,
\textit{Brooklyn Heights R.R. Co. v. Steers}, 106 N.E. 919 (1914). It would not
be seriously disputed that a taxicab ``license'' is, accurately speaking, a
special franchise which is not revocable at will and may not be taken away
except by due process. \textit{Hecht v. Monaghan}, 121 N.E.2d 421 (1954).
\textit{See also}, \textit{Wignall v. Fletcher}, 303 N.Y. 435 (1952). The
resolution of this motion will not be dependent, however, upon the technically
correct characterization of the matter in issue as being either a franchise,
license, or easement.

The government also contends that the physical medallions themselves are
``property'' for purposes of the mail fraud statute. The defendants ridicule
that contention by deprecatingly referring to the medallions as nothing more
than ``23 pieces of tin''. Thus, the defendants impliedly, but never
explicitly, assert a de minimis qualification to the tort of conversion or the
crime of larceny. No authority is cited to support that oblique assertion, nor
is the court aware of any. In his dissenting opinion in \textit{McNally},
Justice Stevens was prescient when he expressed doubt about the gravity of the
ramifications of the Court's decision and said that ``Congress can, of course,
negate it by amending the statute.'' As has already been noted, Congress did
exactly that. Justice Stevens went on, however, to observe that:
\begin{quote}
Even without Congressional action, prosecutions of corrupt officials who use the
mails to further their schemes may continue since it will frequently be
possible to prove \textit{some} loss of money or property.
\end{quote}
\textit{Id.} (emphasis added). In this respect Justice Stevens was also
prescient. The medallion is a tangible, physical object. The Administrative
Code of the City of New York {\S} 19--502(h) provides as follows:
\begin{quote}
``Medallion'' means the metal plate issued by the commission for displaying
the license number of a licensed taxicab on the outside of the vehicle.
\end{quote}
By charging the defendants with obtaining by false and fraudulent
representations and promises 23 unauthorized taxi medallions, the government is
seeking to prosecute these defendants by attempting to prove they caused some
loss of property as alleged.

In \textit{Evans}, upon which the defendants so heavily rely, the defendants
were charged with making false statements to United States agencies to obtain
approval to export arms. Here, the defendants are accused of taking 23 items of
tangible personal property (the metal plates) from the City of New York in
which the City did have a possessory interest. This is not a case where it is
alleged that the citizenry is merely deprived of the honest services of a
public official. This is a case where the public official is accused of
conspiring with others to misappropriate tangible personal property. To view
this case otherwise would be to hold, in effect, that a City cashier who
embezzled money merely deprived the City of her honest and faithful services to
which the embezzled money is an inconsequential appurtenance.\ldots

Whether the medallions are tangible property or not to support a charge of mail
fraud may also be discerned by asking whether the wrongful taking of the
medallions from the offices of the TLC would be larceny. Defendants advise that
a state prosecution has been commenced on that ground. See N.Y. Penal Law {\S}
155.00(1) (McKinney 1988), defining property for purpose of state larceny
statute as ``any article, substance or thing of value''. Thus, the reluctance
of the \textit{McNally} Court to read the mail fraud statute as criminalizing
conduct on the part of a state official which is not otherwise prohibited by
state law need not deter here.\ldots

Mindful that ``an overspeaking judge is no well-tuned cymbal,'' I nevertheless
make several additional observations.

The rule announced in \textit{McNally} was that the mail fraud statute is
applicable only to ``frauds involving money or property'' and not to schemes
relating to good government. It logically followed, said the Court in
\textit{Evans}, 844 F.2d at 39, ``that the deceived party must lose some money
or property.'' \textit{Carpenter }explained that McNally did not limit the
scope of the mail fraud statute ``to tangible as distinguished from intangible
property rights.'' From those pronouncements, the view has been expressed that
obtaining from a sovereign by means of a fraudulent scheme utilizing the mails,
a license to engage in a business, profession or occupation is not a violation
of the mail fraud statute because the license, although property in the hands
of the licensee is not property in the hands of the licensor. Upon reflection,
the view is that A has nothing which, when he gives it to B, becomes something.
This brings to mind \textsc{L. Carroll, Through the Looking
Glass}, Ch. V (Modern Library Ed. at p. 200):
\begin{quotation}
\ldots the Queen remarked\ldots ``I'm just one hundred and one, five months and
a
day.''

``I can't believe \textit{that}{}'' said Alice.

``Can't you?'' the Queens said in a pitying tone. ``Try again; draw a long
breath and shut your eyes.''

Alice laughed. ``There's no use trying,'' she said: ``one \textit{can't}
believe impossible things.''

``I daresay you haven't had much practice,'' said the Queens. ``When I was
your age, I always did it for half-an-hour a day, why, sometimes I've believed
as many as six impossible things before breakfast.''
\end{quotation}

To view the sovereign's power to grant licenses, or franchises, or easements as
being something other than money or property is to equate, erroneously in my
view, the sovereign with an individual or corporation. What the latter sells,
buys, creates or manufactures and the proceeds derived from those activities is
money or property in the traditional sense. The sovereign can buy and sell and
manufacture and derive proceeds from those activities only by virtue of the
power it possesses as sovereign---namely its police power, its power to tax,
etc. It is only through the exercise of those powers that the sovereign obtains
the revenues which enable it to function at all and acquire, if it chooses,
``property'' in the traditional sense. To rob the sovereign of the due exercise
of that power by schemes or artifices to defraud, is to rob it of ``property''
as surely as the goods or chattels or money obtained from a private person by
similar schemes or artifices.

The view of cases that licenses are only property in the hands of the licensee,
but never in the hands of the government represents an inversion of historical
fact. In the seminal article to which reference has already been made, which
urged that various important government benefits (including licenses) be
accorded a status akin to ``property,'' Professor Charles Reich noted that
traditionally, just the opposite was true---licenses, and all other forms of
government largess were considered government property long before the property
rights of the licensee or recipient were accorded legal recognition:
\begin{quote}
The chief obstacle to the creation of private rights in [government] largess
[e.g., licenses, welfare benefits, services, contracts and franchises] has been
the fact that \textit{it is originally public property, comes from the state,
and may be withheld completely.} But this need not be an obstacle.
\textit{Traditional property also comes from the state, and in much the same
way.} Land, for example, traces back to grants from the sovereign. In the
United States, some was the gift of the King of England, some that of the King
of Spain. The sovereign extinguished Indian title by conquest, became the new
owner, and then granted title to a private individual or group. Some land was
the gift of the sovereign under laws such as the Homestead and Preemption Acts.
Many other natural resources---water, minerals and timber, passed into private
ownership under similar grants. In America, land and resources all were
originally government largess. In a less obvious sense, personal property also
stems from government. Personal property is created by law; it owes its origin
and continuance to laws supported by the people as a whole. These laws ``give''
the property to one who performs certain actions. Even the man who catches a
wild animal ``owns'' the animal only as a gift from the sovereign, having
fulfilled the terms of an offer to transfer ownership.
\end{quote}
Reich, \textit{The New Property}, 73 \textsc{Yale L.J.} 733,
778 (1964) (footnotes omitted; emphasis added).

The salutary fact that, in modern times, courts have recognized the property
rights of licensees\readingfootnote{5}{\textit{See, e.g}.,
\textit{Bell v. Burson}, 402 U.S. 535 (1971) (driver's license); \textit{Dixon
v. Love}, 431 U.S. 105 (1977) (same); \textit{Mackey v. Montrym}, 443 U.S. 1
(1979) (same); \textit{Gibson v. Berryhill}, 411 U.S. 564 (1973) (license to
practice optometry); \textit{Willner v. Committee on Character and Fitness},
373 U.S. 96 (1963) (license to practice law); \textit{Barry v. Barchi}, 443
U.S. 55 (1979) (horse trainers' harness racing license).} need not blind us to
the equally compelling fact that licenses, like other forms of public largess,
originate in the state and are ``public property,'' in the first instance.\ldots

