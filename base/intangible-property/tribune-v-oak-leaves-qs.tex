\expected{tribune-v-oak-leaves}


\item \textit{Oak Leaves} is a road not taken. This report of the case comes
from the \textit{Congressional Record}. Senator Clarence Dill (D-WA) had it
read into the record on December 10, 1926 (i.e. the month after it was decided)
because of its bearing on a radio regulation bill he
co-sponsored.\footnote{Being read into the record is not necessarily a sign of
importance. Five pages
later, Senator Byron Harrison (D-MS) had one of Aesop's fables read into the
record to make a point about Republican political maneuvering.} That bill
became the Radio Act of 1927, which established the licensing system whose
essentials are still in force today. Broadcasters require a license from the
Federal Communications Commission; those licenses specify, in some detail, the
frequency on which they can broadcast, the locations of their transmitters, and
the power they can use. The licenses started out being heavily regulated to
ensure that each broadcaster's programs served the public interest, but over
time the licensing process has become far more ministerial. Subject to some
concentrated-ownership restrictions and a few miscellaneous content rules (e.g.
compliance with the Emergency Broadcasting System and some rules on children's
programming), a broadcaster is free to transmit whatever programming it wants
as long as it complies with the FCC's technical requirements. The result is a
system that divides the airwaves into geographic and frequency blocks, and
gives each of these blocks an exclusive licensee. Anyone else broadcasting on
these frequencies in these places is violating the law.  Similar systems hand
out the right to use other frequencies for other purposes (e.g. mobile phone
towers, police radios, satellite communications, etc.). In effect, any
unauthorized use of someone else's assigned spectrum is illegal.

Compare this system with the common-law process illustrated by \textit{Oak
Leaves}. One obvious difference is how one acquires rights in a frequency: prior
use versus governmental assignment. Which of the two seems more likely to lead
to an efficient allocation of resources to those best able to make good use of
them? Which is fairer to participants? Which is more likely to serve the
interests of the listening public? Another evident difference is the different
tests for violation of another's rights. Is it fair to say that the FCC
exclusive licensing are protected by a kind of right against trespass, while
\textit{Oak Leaves} more closely resembles the test for nuisance? Are there any
other relevant differences?

The change in the FCC's policies over time is interesting, too. If broadcasting
is to be based on licenses, how ought those licenses be given out? And should
the FCC care what a licensee does with a license after that? There was a time
when listeners' groups routinely filed lawsuits to keep radio stations from
changing their formats. \textit{See, e.g.}, \textit{Citizens Committee to Keep
Progressive Rock v. FCC}, 478 F. 2d 926 (D.C. Cir. 1973) (remanding to FCC for
hearing on whether to allow WGLN to change from ``progressive rock'' to
``middle of the road''). Would that be a better system? Or should the FCC get
even further out of the business and not care how licensees use their assigned
spectrum at all---e.g., if a licensee wants to stop transmitting FM radio and
use the spectrum for mobile phone calls, why should the FCC care? Does calling
broadcasting licenses ``property'' do anything to answer these questions?

Here's another alternative: no licenses at all, and let anyone use the spectrum
however they see fit. Before you scoff at this ``commons'' approach to spectrum
allocation, consider that this is how WiFi works. You don't need an FCC license
to plug in a home wireless router. The frequency range from 2.4 gigahertz (i.e.
2.4 billion cycles per second) to 2.5 gigahertz is ``unlicensed''; the FCC
regulates the maximum power that a device can emit, but otherwise, anyone is
basically free to use any device they want however they want. How well does
your WiFi connection typically work? What about the chaos of interference
\textit{Oak Leaves} feared? Would this approach work on a wider scale?

\item \textit{Oak Leaves} presents its holding as an almost inevitable
consequence of the nature of spectrum. But what is spectrum? Radio broadcasting
works by running an electric current through the right kind of circuit, which
results in electromagnetic radiation spreading in certain ways that people with
the right kinds of devices can detect. Why isn't the relevant ``property'' here
the transmitter and the receiver (both tangible personal property), or the land
over which the radiation passes (real property)?  So why not handle
broadcasting cases using personal property torts (``You damaged my radio tower
by interfering with its transmissions'') or real property torts (``You
trespassed by sending electromagnetic radiation over my land'')? Consider this
passage from Ronald Coase, \textit{The Federal Communications Commission}, 2
\textsc{J. L. Econ.} 1 (1959):
\begin{quote}
What does not seem to have been understood is that what is being allocated by
the Federal Communications Commission, or, if there were a market, what would
be sold, is the right to use a piece of equipment to transmit signals in a
particular way. Once the question is looked at in this way, it is unnecessary
to think in terms of ownership of frequencies or the ether. Earlier we
discussed a case in which it had to be decided whether a confectioner had the
right to use machinery which caused noise and vibrations in a neighboring
house. It would not have facilitated our analysis of the case if it had been
discussed in terms of who owned sound waves or vibrations or the medium
(whatever it is) through which sound waves or vibrations travel. Yet this is
essentially what is done in the radio industry. The reason why this way of
thinking has become so dominant in discussions of radio law is that it seemed
to have developed by using the analogy of the law of airspace. In fact, the law
of radio and television has been commonly treated as part of the law of the
air. It is not suggested that this approach need lead to the wrong answers, but
it tends to obscure the question that is being decided. Thus, whether we have
the right to shoot over another man's land has been thought of as depending on
who owns the airspace over the land.  It would be simpler to discuss what we
should be allowed to do with a gun.\ldots The problem confronting the radio
industry is that signals transmitted by one person may interfere with those
transmitted by another. It can be solved by delimiting the rights which various
persons possess.
\end{quote}
Is this any more helpful than \textit{Oak Leaves}{}'s analogies to trademarks
and water rights?

A related argument is that ``spectrum'' is the wrong abstraction for regulating
multiple people's simultaneous broadcasting. It is true that given the
amplitude-modulating radio technology of 1926, WGN's and WGES's broadcasts on
nearby frequencies from nearby locations were likely to cause frustrating
interference for listeners. But technology changes, and more broadcast
technologies don't depend on exclusive assignments of slices of spectrum. One
approach is ``spread-spectrum,'' in which a device transmits at a given
frequency only for a very short burst and then ``hops'' to a different
frequency for the next bit of its transmission, and so on. This is basically
how modern cell phones communicate with towers; the system allows many devices
to ``share'' the same nominal slice of spectrum. Another emerging technology is
``ultra-wideband,'' in which a device transmits on an immensely wide range of
frequencies but with very low power---so low that it interferes only minimally
with other spectrum users. There are also techniques that involve shaping the
geometry of a transmission so it travels only in desired directions. What would
\textit{Oak Leaves} have to say about these new technologies? Is it more or
less accommodating of them than the FCC's regulatory system?

\item What do you make of the defendant's argument that WGN's station was
``improperly constructed and operated?'' If WGES is causing interference to
WGN's signal, should it matter that WGN could avoid the problem by fixing its
equipment? Should it matter how much the changes would cost? On how
well-established the appropriate technical standards are?

For that matter, what about better receivers? If more modern radios would allow
people in the Chicago area to tune in to WGN at 990 kilohertz without hearing
interference from WGES at 950 kilohertz (and vice versa), should WGN really be
able to push WGES off the airwaves just because some listeners have antiquated
radios? (To borrow the court's analogy to trademarks, what if some people are
just confused all the time about everything?)

These can be high-stakes fights. The company LightSquared wanted to build a
nationwide wireless network using a mixture of cell towers and satellites. It
had FCC permission to use frequencies between 1525 and 1559 megahertz, but the
next spectrum band up, from 1559 to 1610 megahertz, was allocated to
``radionavigation satellite services''---i.e., GPS. Technical reports agreed
with the arguments of GPS makers that LightSquared's proposed transmissions
would cause many GPS units, including some on airplanes, to stop working.
LightSquared argued that this was not because it would be improperly
transmitting outside its assigned band, but because GPS units would be
improperly \textit{listening} to transmissions outside of their assigned band.
According to LightSquared, inexpensive filters in GPS units would have fixed
the problem---but there are millions of GPS units already out there in the
world without those filters. In the end, the FCC scrapped LightSquared's plan.
Would you have? LightSquared spent three years in bankruptcy following the
FCC's decision, and racked up nearly \$2 billion in losses. Could a better
system of property rights in spectrum have avoided the conflict entirely?

\item Does \textit{Oak Leaves} give legal recognition to property that already
exists or create property where none existed before? Or is ``property'' the
wrong way to refer to WGN's rights here?

