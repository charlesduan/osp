\reading{Tribune Co. v. Oak Leaves Broadcasting Station}

\readingcite{68 Cong. Rec. 216 (Cook Cty. Cir. Ct. Ill. Nov. 17, 1926)}

\opinion Decision of Judge Wilson on Defendants' Motion to Dissolve
Temporary Injunction

\ldots The bill very briefly charges that the complainant is and has been for
some time a corporation organized under the laws of the State of Illinois, with
its principal place of business in the city of Chicago, and is engaged in the
publication of a newspaper known as the Chicago Daily Tribune, and that it has
an average daily paid circulation of several hundred thousand subscribers.

It further charges that since March 29, 1924, it has been engaged in
broadcasting by radio of daily programs of information, amusement, and
entertainment to the general public, and particularly to that part of the
general public residing in and in the vicinity of the city of Chicago, and for
that purpose the complainant operates an apparatus generally
known as a broadcasting station located on the Drake Hotel and another such
broadcasting station operated near the city of Elgin.

The bill further charges that it has been the custom for several years for
persons engaged in broadcasting to designate their certain stations by certain
combinations of letters known as call letters, and that these call letters
serve to enable persons using radio receiving sets to identify the particular
station, and in this instance the complainant has been using the letters WGN,
which stand for the abbreviation of the World's Greatest Newspaper which
appears to have been adopted by the complainant as a sort of trade name
indicating the Chicago Daily Tribune.

It is further charged in the bill that it is the custom for such newspapers
owning and operating broadcasting stations to make announcements of their
programs in the daily editions of the paper, and that the complainant has,
since March 29, 1924, used the designation WGN, and further charges that its
program is of a high-class character, and that by reason of its broadcasting it
has built up a good will with the public, which is of great value to the
complainant, in that it has enhanced the value of the newspaper and increased
the profits.

Further charges, on information and belief, that the number of persons who
listen to the said broadcasting of the complainant is in excess of 500,000 and
that these persons are educated to listen in or tune in on the wave
length of the complainant for the purpose of hearing and
enjoying the programs so broadcasted.

The bill further charges that, when two stations are broadcasting on the same or
nearly the same wave length, the result will be that the users of the radio
will either hear one of the stations to the exclusion of the other or hear both
of the stations at the same time, which will cause confusion to the listener,
or will hear one to the exclusion of the other but accompanied by a series of
noises, such as whistles and roars, which render the program practically
useless.

The bill further charges that for several years the broadcasting in the United
States and Canada has been done on sending wave lengths varying from 201 meters
to 550 meters, inclusive, the United States Government, by an enactment of
Congress, having forbidden to private and commercial broadcasters the use of
wave lengths from 601 meters to 1,600 meters, and the use of wave
lengths under 200 meters because of the impracticability of the use of
said wave lengths under 200 meters by reason of natural causes and because of
the fact that this field is open to amateurs and used by a large number of the
same. 

Furthermore, that most of the radio receiving sets are so constructed it to be
adapted to the receiving of broadcasting within this band of wave lengths
included above the 200 meters and under 500 meters.

The bill further charges that the sending waves used by broadcasting stations
are also classified by the number of kilocycles denoting the frequency of
vibration per second characteristic of each wave. The higher the wave length
the less is the number of kilocycles, and a definite number of kilocycles is
characteristic of each wave.\edfootnote{The formula is $f=c/\lambda$,
where $\lambda$ is the wavelength in meters, $f$
is the frequency in cycles per second, and $c$ is the speed of light
(299,792,458 meters per second). So, for example, to use WGN's
numbers, 299,792,458~/ 990~= 302,820 cycles per second, or 302.8 kilocycles
per second.}

Further charges that the radio receiving sets in general use in the United
States and Canada are scaled and marked with numerical divisions and that by
means of dials or indicators persons receiving over radio can set such dials or
indicators at particular points and hear the particular broadcasting station
over the particular wave length that they desire.

Further charges that the users of radios have become familiar with the different
wave lengths and broadcasting stations designated by the particular letters
employed and that this fact is of value to the broadcaster because the public
has been educated to their particular wave length and their
particular designation.

The bill also charges that knowledge of this particular wave length by a
broadcaster is of great value to the broadcaster because the person receiving
through the radio has been educated to know when to place his dials or
indicators in order to receive a particular station and that the public
generally in the locality of the complainant has become familiar with the wave
length of the complainant and that its loss by interference would work great
damage to the complainant.

The complainant further charges that on the 14th of
December, 1925, it did, and ever since then has, broadcast on a sending wave
length of 302.8 meters (the kilocycles characteristic of such wave length being
990) and that it broadcasts from both the Drake Hotel and from its Elgin
broadcasting station and that, at that time, no other broadcasting station in
the city of Chicago or in the entire State of Illinois was using said wave
length or any wave length sufficiently near to interfere with complainant's
broadcasting and that this fact was generally known to the public and that the
public had access by reason thereof to the programs of the complainant as
broadcast over the same wave length from the two broadcasting stations and
which programs were announced at different periods of time by arrangement of
the complainant.

Further charges that the complainant has expended large sums of money during
said period of time in the building up and betterment of said broadcasting
stations and in the furnishing of high-class talent for its programs and in the
payment of salaries and expenses in its business of broadcasting.\dots

That the defendants, the Oak Leaves Broadcasting Station (Inc.), and the Coyne
Electrical School (Inc.) are corporations existing under and by virtue of the
laws of Illinois, and that the defendant, Guyon, is a resident of Chicago,
Ill., engaged in business in said city under the name of Guyon's Paradise Ball
Room, and operates a dance hall in the city of Chicago.

The bill further charges that the broadcasting station, heretofore used and
operated by the defendants, Oak Leaves Broadcasting Station (Inc.), and Coyne
Electrical School (Inc.), which had been operated from Oak Park, a suburb of
the city of Chicago, was moved to 124 North Crawford Avenue, where Guyon's
Paradise Ball Room is located, and is being now operated from that point, and
charges that the said defendant, Guyon, became the owner and operator of said
broadcasting station and that the other defendants have some interest in saint
station which is unknown to the complainant, but which is charged to be true on
information and belief.

The bill further charges that said station of the defendants had originally used
a wave length of 220 meters (1,350 kilocycles)\ldots and that, later, it
changed its wave length to 249.9 meters (1,200 kilocycles), which it continued
to use until on or about September 7, 1926, and further charges that the
defendants had never enjoyed any considerable degree of the good will of the
public, nor was it popular with the users of radio receiving sets, but was
comparatively unknown in Chicago or its vicinity.

That on or about September 7, 1926, the said Guyon's Paradise Broadcasting
Station, used and operated by the defendants, changed its sending wave length
to a wave length either the same as that of the complainant (i.e. 302.8) or one
having a frequency of considerably less than 50 kilocycles different than that
of the complainant, and that it is now using said wave length and has from that
time until the date of the filing of the bill herein\ldots.

The bill further charges that the defendants have, since September 7, 1926, used
the said new wave length during the hours of the day when complainant is
broadcasting, and that by reason thereof said broadcasting by the said
defendants has interfered with and destroyed complainant's broadcasting to the
public in the city of Chicago and throughout the region where complainant's
newspaper circulates, and that by reason thereof radio receivers have been
unable to hear the programs of the complainant, and that if it is allowed to
continue it will work incalculable damage and injury to the good will of the
complainant's broadcasting, and consequently will injure the circulation of the
complainant so far as its newspaper is concerned and deprive it of great
profits.

Further charges that there are other wave lengths which are usable by the
defendants and that this wave length can be changed with practically no expense
and within a short period of time.

The bill prays for an order restraining the defendants from broadcasting from
said station in such a manner as to interfere with the broadcasting of the
complainant, and more particularly from using any wave length within 100 miles
of the city of Chicago having a frequency of less than 1,040 kilocycles per
second, or more than 940 kilocycles per second, charging, in effect, that any
wave length within that designated number of kilocycles would necessarily cause
an interference with the broadcasting of the complainant. 

The answer\ldots admits that where a broadcasting station is operating on  a
wave length the frequency of which is within 50 kilocycles per second of the
number of kilocycles per second characteristic of the wave length of the first
station, that some interference will result but that such interference is
natural where stations are operating in close proximity one to the other, but
that where two broadcasting stations in the same locality are properly
constructed and operated and the wave length employed sharply defined and the
power of sold stations substantially equal there will be no
appreciable interference by the stations if they are separated by 40
kilocycles.\ldots

The answer admits that on September 7, 1926, the said defendants' station
changed its wave length, but denies that they are broadcasting over the same
wave length as that of the complainant, but state that they are sending over a
wave length which is removed 40 kilocycles from the wave length used by the
complainant, and that said wave being used is 315.6 meters with a
frequency of 950 kilocycles.

The answer further admits that the defendant\ldots has since about September,
1926, used and operated the broadcasting station described in the bill of
complaint Guyon's Paradise Broadcasting Station, but denies that they are
drowning out the hearers of WGN, and state that, if such is the fact, it is
because said complainant's broadcasting station is improperly constructed and
operated.

The answer further admits that on or about September 7, 1926, there was
available to them a wave length of 249.9 metres with a frequency of 1,200
kilocycles, but state that said wave length is not desirable for the purpose of
broadcasting and that its use would render WGES of little or no value as a
broadcasting station.

And further sets forth that there are other wave lengths which would be usable
by the defendants, but states that their use would cause greater
interference to other broadcasters than the interference now caused to WGN by
the use of the present wave length now employed by them.

The defendants further charge that they have invested large sums of money in and
about their plant and will suffer damage in the event the temporary injunction
heretofore issued should not be dissolved.

The facts in this case, as charged by the bill and admitted by the answer,
together with the additional facts set out in the bill as matters of defense,
disclose a situation new and novel in a court of equity and a consideration of
the law applicable to the facts requires an understanding of the present
conditions for the purpose of ascertaining whether or not the old adage of
``Old laws should be adapted to new facts'' should be applied and for that
reason a short statement of general existing conditions is
not out of order at this time before considering the legal and equitable
aspects of the cause.

It is a matter of general knowledge that in the last few years there has grown
up in the United States, as well as abroad, a well recognized calling or
business known as broadcasting which consists in sending from a central
station, electrically equipped, programs of music and amusement, speeches by
men of prominence, news of the day and items of interest taking place in the
world, and that these various programs are received by the public over radio
receiving sets which have been installed in homes, hotels, and various other
places, and that a large industry has grown up and developed in the making and
manufacturing of radio sets, so that in the United States, at this time, there
are millions of dollars invested by the public at large, which has made the
investment for the purpose of and with the knowledge that they could receive
these programs, speeches, and items of interest from various broadcasting
stations located in various parts of the United States and in other countries.

It might also be stated that, so far as broadcasting stations are concerned,
there has almost grown up a custom which recognizes the rights of the various
broadcasters, particularly in that certain broadcasters use certain
hours of the day, while the other broadcasters remain silent during that
particular period of time. Again, in this particular locality, a certain night
is set aside as silent night, when all local broadcasters cease broadcasting in
order that the radio receivers may be able to tune in on outside distant
stations.

Wave lengths have been bought and sold and broadcasting stations have changed
hands for a consideration. Broadcasting stations have contracted with each
other so as to broadcast without conflicting and in this manner be able to
present their different programs to the waiting public. The public itself has
become educated to the use or its receiving sets so as to be able to obtain
certain particular items of news, speeches, or programs over its own particular
sets.

The theory of the bill in this case is based upon the proposition that by usage
of a particular wave length for a considerable length of time and by reason of
the expenditure of a considerable amount of money in developing its
broadcasting station and by usage of a particular wave length educating the
public to know that that particular wave length is the wave length of the
complainant and by furnishing programs which have been attractive and thereby
cause a great number of people to listen in to their particular programs that
the said complainant has created and carved out for itself a particular right
or easement in and to the use of said wave length which should he recognized
in a court of equity and that outsiders should not be allowed thereafter,
except for good cause shown, to deprive them of that right and to make use of a
field which had been built up by the complainant at a considerable cost in
money and a considerable time in pioneering.\ldots

The defendants further insist that a wave length can not be made the
subject of private control and, further and lastly, that as a matter of fact
they are not interfering with the complainant by the use of the present wave
length employed by them from their broadcasting station. \ldots

[The court discussed the 1912 federal statute which required a license to
broadcast by radio and restricted the wavelengths available, as discussed
above. It concluded that the statute did not displace state law.]

In the first place, it is argued that there are no rights in the air and that
the law has no right or authority to restrict the using of wave lengths or to
exclude others from their use. In answer to this it might be said that Congress
has already attempted to regulate the use of the air in its enactment of August
13, 1912, by providing that only certain strata of the air or ether may be used
for broadcasting purposes and, further, requiring persons to take out a
license before they are permitted to exercise the use of the air or ether.
Moreover, it appears to this court that the situation is such from the past
development of the industry of broadcasting and radio receiving and from the
apparent future, as indicated by the past, that, unless some regulatory
measures are provided for by Congress or rights recognized by State courts, the
situation will result in chaos and a great detriment to the advancement of an
industry which is only in its infancy.

While it is true that the case in question is novel in its newness, the
situation is not devoid, however, of legal equitable support. The same answer
might be made, as was made in the beginning; that there was no property right,
or could be, in a name or sign, but there has developed a long line of cases,
both in the Federal and State courts, which has recognized, under the law known
as the law of unfair competition, the right to obtain a property right in a
name or word or collection of names or words\edfootnote{I.e., a trademark.}
which gives the person who first made use of the same a property right therein,
provided that by reason of their use, he has succeeded in building up a
business and created a good will which has become known to the public and to
the trade and which has served as a designation of some particular output so
that it has become generally recognized as the property of such person. The
courts have held that persons who attempt to imitate or to make use of such
trade name or names or words evidently do so for the purpose of enriching
themselves through the efforts of some other person who by the investment of
money and time has created something of value. Equity has invariably protected
the rights of such persons in the use of said names.

It is also true that the courts have recognized, particularly in the west, the
right to the use of running water for the purposes of mining and other uses.
(\textit{Atchison v. Peterson}, 20 Wall. 507; \textit{Cache La Poudre Reservoir
v. Water Supply \& Storage Co}., 25 Colo. 161.)

Some of the States have also recognized the rights of telephone and telegraph
companies in the operation of their lines free from interference by lines of
other companies placed in such close proximity as to create confusion by reason
of electrical interference. (\textit{Western Union Telegraph Co. v. Los Angeles
Electric Co}., 70 Fed. 178; \textit{Northwestern Telephone Exchange Co. v.
Twin City Telephone Co.}, 89 Minn, 4115; and other cases.)

It us argued that the electrical cases generally involve a franchise
and thereby a property right, but the cases on electrical interference are
cited more particularly for the purpose of their analogy to the case at bar and
not as authorities on the question.

In regard to the water cases, counsel for the defendants call our attention to
the rule in this State, as set forth in the case of \textit{Druley v. Adam}
(102 Ill. 177), where the court says in its opinion, page 193,
\begin{quote}
The law has
been long settled in this State that there can be no property merely in the
water of a running stream. The owner of land over which a stream of water flows
has, as incident to his ownership of the land, a property right in the flow of
the water at that place for all the beneficial uses that may result from it,
whether for motive power in propelling machinery or in imparting fertility to
the adjacent soil, etc.; in other words, he has a usufruct in the water
while it passes; but all other riparian proprietors have precisely the same
rights in regard to it and, apart from the right of consumption for supplying
natural wants, neither can, to the injury of the other, abstract the water or
divert or arrest its flow.
\end{quote}
The same court, however, in its opinion, on page 201, while holding that the
western water cases are not applicable, recognized the law as laid down in
those cases and distinguished them on the ground that it is apparent that the
law necessarily arose in those cases by reason of the peculiar circumstances
and necessities existing in those countries at the time.

It is the opinion of the court that, under the circumstances as now exist, there
is a peculiar necessity existing and that there are such unusual and peculiar
circumstances surrounding the question at issue that a court of equity is
compelled to recognize rights which have been acquired by reason of the outlay
and expenditure of money and the investment of time and that the
circumstances and necessities are such, under the circumstances of this case,
as will justify a court of equity in taking jurisdiction of the cause. Such
being the case, it becomes the duly of the court to consider the last question,
namely, whether or not there is such an interference by the defendants with the
broadcasting station of complainant that the temporary injunction
heretofore granted should be kept in force until a final hearing of the
cause.

[W]e believe that the equities of the situation are in favor of the complainant
on the facts as heretofore shown, particularly in that the complainant has been
using said wave length for a considerable length of time and has built up a
large clientage, whereas the defendants are but newly in the field and will not
suffer as a result of an injunction in proportion to the damage that would be
sustained by the complainant after having spent a much greater length of time
in the education of the general radio-receiving public to the wave length in
question.

We are of the opinion further that, under the circumstances in this case,
priority of time creates a superiority in right, and the fact of priority
having been conceded by the answer it would seem to this court that it would
lie only just that the situation should be preserved in the status in which it
was prior to the time that the defendants undertook to operate over or near the
wave length of the complainant.\ldots

It is difficult to determine at this time how a radio station should be properly
run, but it is, also, true that the science of broadcasting and receiving is
being subject every day to change and it is possible that within it short time
this may be accomplished, although it is the opinion of the court from an
examination of the affidavits and exhibits in the cause that 40 kilocycles is
not at this time recognized as a safe limitation for the prevention of
interference between stations located in the same locality. It is true that
stations sufficiently removed from each other can broadcast even over the
same wave length, but it necessarily follows that they must be so far apart
that the wave lengths do not reach or come in contact with each other to the
extent of creating interference.

In the case at bar the contestants are so located with reference to each other
that the court does not feel that 40 kilocycles is sufficient. The court is of
the opinion, however, that until there has been a final hearing of this cause
no order prohibiting the defendants from the use of any particular wave length
should be entered and to that extent the order heretofore entered will be
modified so that it will read that the defendants are restrained and enjoined
from broadcasting over a wave length sufficiently near to the one used by the
complainant so as to cause any material interference with
the programs or announcements of the complainant over and from its broadcasting
station to the radio public within a radius of 100 miles, and in order that the
defendants may be apprised of the feeling of the court in this regard, while
the order is not expressly one of exact limitation, nevertheless the court
feels that a distance removed 50 kilocycles from the wave length of the
complainant would be a safe distance and that if the defendants use a
wave length in closer proximity than the one stated it must be at the risk of
the defendants in this cause.

