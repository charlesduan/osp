\reading{Kremen v. Cohen}
\readingcite{337 F. 3d 1024 (9th Cir. 2003)}

KOZINSKI, Circuit Judge.\edfootnote{In 2009, Judge Kozinski was admonished by a
judicial disciplinary panel for maintaining a publicly accessible server that
included sexually explicit material. In 2017, he abruptly resigned after being
accused of sexual misconduct by numerous women, including former law clerks.}

We decide whether Network Solutions may be liable for giving away a registrant's
domain name on the basis of a forged letter.

\readinghead{Background}

``Sex on the Internet?,'' they all said. ``That'll never make any money.'' But
computer-geek-turned-entrepreneur Gary Kremen knew an opportunity when he saw
it. The year was 1994; domain names were free for the asking, and it would be
several years yet before Henry Blodget and hordes of eager NASDAQ day traders
would turn the Internet into the Dutch tulip craze of our times. With a quick
e-mail to the domain name registrar Network Solutions, Kremen became the proud
owner of sex.com. He registered the name to his business, Online Classifieds,
and listed himself as the contact.

Con man Stephen Cohen, meanwhile, was doing time for impersonating a bankruptcy
lawyer. He, too, saw the potential of the domain name. Kremen had gotten it
first, but that was only a minor impediment for a man of Cohen's boundless
resource and bounded integrity. Once out of prison, he sent Network Solutions
what purported to be a letter he had received from Online Classifieds. It
claimed the company had been ``forced to dismiss Mr. Kremen,'' but ``never got
around to changing our administrative contact with the internet registration
[sic] and now our Board of directors has decided to abandon the domain name
sex.com.'' Why was this unusual letter being sent via Cohen rather than to
Network Solutions directly? It explained:
\begin{quote}
Because we do not have a direct connection to the internet, we request that you
notify the internet registration on our behalf, to delete our domain name
sex.com. Further, we have no objections to your use of the domain name sex.com
and this letter shall serve as our authorization to the internet registration
to transfer sex.com to your
corporation.\readingfootnote{2}{The letter
was signed ``Sharon Dimmick,'' purported president of Online Classifieds.
Dimmick was actually Kremen's housemate at the time; Cohen later claimed she
sold him the domain name for \$1000. This story might have worked a little
better if Cohen hadn't misspelled her signature.}
\end{quote}

Despite the letter's transparent claim that a company called ``Online
Classifieds'' had no Internet connection, Network Solutions made no effort to
contact Kremen. Instead, it accepted the letter at face value and transferred
the domain name to Cohen. When Kremen contacted Network Solutions some time
later, he was told it was too late to undo the transfer. Cohen went on to turn
sex.com into a lucrative online porn empire.

And so began Kremen's quest to recover the domain name that was rightfully his.
He sued Cohen and several affiliated companies in federal court, seeking return
of the domain name and disgorgement of Cohen's profits. The district court
found that the letter was indeed a forgery and ordered the domain name returned
to Kremen. It also told Cohen to hand over his profits, invoking the
constructive trust doctrine and California's ``unfair competition'' statute,
Cal. Bus. \& Prof. Code {\S} 17200 et seq. It awarded \$40 million in
compensatory damages and another \$25 million in punitive damages.

Kremen, unfortunately, has not had much luck collecting his judgment. The
district court froze Cohen's assets, but Cohen ignored the order and wired
large sums of money to offshore accounts. His real estate property, under the
protection of a federal receiver, was stripped of all its fixtures -- even
cabinet doors and toilets -- in violation of another order. The court commanded
Cohen to appear and show cause why he shouldn't be held in contempt, but he
ignored that order, too. The district judge finally took off the gloves---he
declared Cohen a fugitive from justice, signed an arrest warrant and sent
the U.S. Marshals after him.

Then things started getting really bizarre. Kremen put up a ``wanted'' poster on
the sex.com site with a mug shot of Cohen, offering a \$50,000 reward to anyone
who brought him to justice. Cohen's lawyers responded with a motion to vacate
the arrest warrant. They reported that Cohen was under house arrest in Mexico
and that gunfights between Mexican authorities and would-be bounty hunters
seeking Kremen's reward money posed a threat to human life. The district court
rejected this story as ``implausible'' and denied the motion. Cohen, so far as
the record shows, remains at large.

Given his limited success with the bounty hunter approach, it should come as no
surprise that Kremen seeks to hold someone else responsible for his losses.
That someone is Network Solutions, the exclusive domain name registrar at the
time of Cohen's antics. Kremen sued it for mishandling his domain name,
invoking four theories at issue here. He argues that he had an implied contract
with Network Solutions, which it breached by giving the domain name to Cohen.
He also claims the transfer violated Network Solutions's cooperative agreement
with the National Science Foundation---the government contract that made
Network Solutions the .com registrar. His third theory is that he has a
property right in the domain name sex.com, and Network Solutions committed the
tort of conversion by giving it away to Cohen. Finally, he argues that Network
Solutions was a ``bailee'' of his domain name and seeks to hold it liable for
``conversion by bailee.''

The district court granted summary judgment in favor of Network Solutions on all
claims. [Kremen appealed. His contract claim failed on appeal because he was
not a paying customer and Network Solutions made no promises to him. He was not
an intended third-party beneficiary of Network Solutions' contract with the
NSF. And ``conversion by bailee'' was not an independent tort from conversion
under California law. That left his conversion claim.]

\readinghead{Conversion}

Kremen's conversion claim is another matter. To establish that tort, a plaintiff
must show ``ownership or right to possession of property, wrongful disposition
of the property right and damages.'' \textit{G.S. Rasmussen \& Assocs., Inc. v.
Kalitta Flying Serv., Inc.}, 958 F.2d 896, 906 (9th Cir. 1992). The preliminary
question, then, is whether registrants have property rights in their domain
names. Network Solutions all but concedes that they do. This is no surprise,
given its positions in prior litigation. \textit{See} \textit{Network
Solutions, Inc. v. Umbro Int'l, Inc}., 529 S.E.2d 80, 86 (2000) (``[Network
Solutions] acknowledged during oral argument before this Court that the right
to use a domain name is a form of intangible personal
property.'').\readingfootnote{5}{Network Solutions\ldots
stresses that Kremen didn't develop the sex.com site before Cohen stole it. But
this focus on the particular domain name at issue is misguided. The question is
not whether Kremen's domain name in isolation is property, but whether domain
names as a class are a species of property.} The district court agreed with the
parties on this issue, as do we.

Property is a broad concept that includes ``every intangible benefit and
prerogative susceptible of possession or disposition.'' \textit{Downing v. Mun.
Court}, 198 P.2d 923 (1948). We apply a three-part test to determine whether a
property right exists: ``First, there must be an interest capable of precise
definition; second, it must be capable of exclusive possession or control; and
third, the putative owner must have established a legitimate claim to
exclusivity.'' \textit{G.S. Rasmussen}, 958 F.2d at 903. Domain names satisfy
each criterion. Like a share of corporate stock or a plot of land, a domain
name is a well-defined interest. Someone who registers a domain name decides
where on the Internet those who invoke that particular name---whether by
typing it into their web browsers, by following a hyperlink, or by other means
-- are sent. Ownership is exclusive in that the registrant alone makes that
decision. Moreover, like other forms of property, domain names are valued,
bought and sold, often for millions of dollars, and they are now even subject
to in rem jurisdiction, see 15 U.S.C. {\S} 1125(d)(2).

Finally, registrants have a legitimate claim to exclusivity. Registering a
domain name is like staking a claim to a plot of land at the title office. It
informs others that the domain name is the registrant's and no one else's. Many
registrants also invest substantial time and money to develop and promote
websites that depend on their domain names. Ensuring that they reap the
benefits of their investments reduces uncertainty and thus encourages
investment in the first place, promoting the growth of the Internet overall.

Kremen therefore had an intangible property right in his domain name, and a jury
could find that Network Solutions wrongfully disposed of that right to his
detriment by handing the domain name over to Cohen. The district court
nevertheless rejected Kremen's conversion claim. It held that domain names,
although a form of property, are intangibles not subject to conversion. This
rationale derives from a distinction tort law once drew between tangible and
intangible property: Conversion was originally a remedy for the wrongful taking
of another's lost goods, so it applied only to tangible property. \textsc{See
Prosser and Keeton on the Law of Torts} {\S} 15, at 89, 91 (W. Page Keeton ed.,
5th ed. 1984). Virtually every jurisdiction, however, has discarded this rigid
limitation to some degree. Many courts ignore or expressly reject it. See
\textit{Kremen}, 325 F.3d at 1045-46 n. 5 (Kozinski, J., dissenting) (citing
cases); \textit{Astroworks, Inc. v. Astroexhibit, Inc.}, 257 F.Supp.2d 609, 618
(S.D.N.Y.2003) (holding that the plaintiff could maintain a claim for
conversion of his website); Val D. Ricks, \textit{The Conversion of Intangible
Property: Bursting the Ancient Trover Bottle with New Wine}, 1991
\textsc{B.Y.U. L. Rev.} 1681, 1682. Others reject it for some intangibles but
not others. The Restatement, for example, recommends the following test:
\begin{quotation}
(1) Where there is conversion of a document in which intangible rights are
merged, the damages include the value of such rights.

(2) One who effectively prevents the exercise of intangible rights of the kind
customarily \textit{merged in a document} is subject to a liability similar to
that for conversion, even though the document is not itself converted.
\end{quotation}

\textsc{Restatement (Second) of Torts} {\S} 242 (1965) (emphasis added). An
intangible is ``merged'' in a document when, ``by the appropriate rule of law,
the right to the immediate possession of a chattel and the power to acquire
such possession is \textit{represented by} [the] document,'' or when ``an
intangible obligation [is] represented by [the] document, which is regarded as
equivalent to the obligation.'' \textit{Id}. cmt. a (emphasis
added).\readingfootnote{6}{The Restatement does note that
conversion ``has been applied by some courts in cases where the converted
document is not in itself a symbol of the rights in question, but is merely
essential to their protection and enforcement, as in the case of account books
and receipts.'' \textit{Id}. cmt. b.} The district court applied this test and
found no evidence that Kremen's domain name was merged in a document.\ldots

We conclude that California does not follow the Restatement's strict merger
requirement. Indeed, the leading California Supreme Court case rejects the
tangibility requirement altogether. In \textit{Payne v. Elliot}, 54 Cal. 339,
1880 WL 1907 (1880), the Court considered whether shares in a corporation (as
opposed to the share certificates themselves) could be converted. It held that
they could, reasoning: ``[T]he action no longer exists as it did at common law,
but has been developed into a remedy for the conversion of every species of
personal property.'' \textit{Id}. at 341 (emphasis added). While \emph{Payne}'s
outcome might be reconcilable with the Restatement, its rationale certainly is
not: It recognized conversion of shares, not because they are customarily
represented by share certificates, but because they are a species of personal
property and, perforce, protected.\readingfootnote{7}{Intangible interests in
real property, on the other hand, remain unprotected by
conversion, presumably because trespass is an adequate remedy. See
\textit{Goldschmidt v. Maier}, 73 P. 984, 985 (Cal.1903) (per curiam) (``[A]
leasehold of real estate is not the subject of an action of trover.'');
\textit{Vuich v. Smith}, 35 P.2d 365 (1934) (same). Some California cases also
preserve the traditional exception for indefinite sums of money. See 5 Witkin
\textit{Torts} {\S} 614.}

Notwithstanding \emph{Payne}'s seemingly clear holding, the California Court of
Appeal
held in \textit{Olschewski v. Hudson}, 87 Cal. App. 282, 262 P. 43 (1927), that
a laundry route was not subject to conversion. It explained that \emph{Payne}'s
rationale was ``too broad a statement as to the application of the doctrine of
conversion.'' \textit{Id}. at 288, 262 P. 43. Rather than follow binding
California Supreme Court precedent, the court retheorized \emph{Payne} and held
that
corporate stock could be converted only because it was ``represented by'' a
tangible document. \textit{Id}.; see also \textit{Adkins v. Model Laundry Co}.,
268 P. 939 (1928) (relying on \textit{Olschewski} and holding that no property
right inhered in ``the intangible interest of an exclusive privilege to collect
laundry'').

Were \textit{Olschewski} the only relevant case on the books, there might be a
plausible argument that California follows the Restatement. But in \textit{Palm
Springs-La Quinta Development Co. v. Kieberk Corp}., 115 P.2d 548 (1941), the
court of appeal allowed a conversion claim for intangible information in a
customer list when some of the index cards on which the information was
recorded were destroyed. The court allowed damages not just for the value of
the cards, but for the value of the intangible information lost. Section 242(1)
of the Restatement, however, allows recovery for intangibles only if they are
merged in the converted document. Customer information is not merged in a
document in any meaningful sense. A Rolodex is not like a stock certificate
that actually represents a property interest; it is only a means of recording
information.

\textit{Palm Springs} and \textit{Olschewski} are reconcilable on their facts --
the former involved conversion of the document itself while the latter did not.
But this distinction can't be squared with the Restatement. The plaintiff in
\textit{Palm Springs} recovered damages for the value of his intangibles. But
if those intangibles were merged in the index cards for purposes of section
242(1), the plaintiffs in \textit{Olschewski} and \textit{Adkins} should have
recovered under section 242(2) -- laundry routes surely are customarily written
down somewhere. ``Merged'' can't mean one thing in one section and something
else in the other.

California courts ignored the Restatement again in \textit{A \& M Records, Inc.
v. Heilman}, 75 Cal.App.3d 554 (1977), which applied the tort to a defendant
who sold bootlegged copies of musical recordings. The court held broadly that
``such misappropriation and sale of the intangible property of another without
authority from the owner is conversion.'' \textit{Id}. at 570. It gave no hint
that its holding depended on whether the owner's intellectual property rights
were merged in some document. One might imagine physical things with which the
intangible was associated---for example, the medium on which the song was
recorded. But an intangible intellectual property right in a song is not merged
in a phonograph record in the sense that the record represents the composer's
intellectual property right. The record is not like a certificate of ownership;
it is only a medium for one instantiation of the artistic work.

Federal cases applying California law take an equally broad view. We have
applied \textit{A \& M Records} to intellectual property rights in an audio
broadcast, see \textit{Lone Ranger Television, Inc. v. Program Radio Corp.},
740 F.2d 718, 725 (9th Cir. 1984), and to a regulatory filing, \textit{see G.S.
Rasmussen}, 958 F.2d at 906-07. Like \textit{A \& M Records}, both decisions
defy the Restatement's ``merged in a document'' test. An audio broadcast may be
recorded on a tape and a regulatory submission may be typed on a piece of
paper, but neither document represents the owner's intangible interest.

The Seventh Circuit interpreted California law in \textit{FMC Corp. v. Capital
Cities/ABC, Inc.}, 915 F.2d 300 (7th Cir. 1990). Observing that
``\,`[t]here is perhaps no very valid and essential reason why
there might not be conversion' of intangible property,'' \emph{id.} at 305
(quoting
\textsc{Prosser \& Keeton}, supra, {\S} 15, at 92), it held that a defendant
could be liable merely for depriving the plaintiff of the use of his
confidential information. In rejecting the tangibility requirement,
\textit{FMC} echoes \textit{Payne}'s holding that personal property of any
species may be converted. And it flouts the Restatement because the intangible
property right in confidential information is not represented by the documents
on which the information happens to be recorded.\ldots

In short, California does not follow the Restatement's strict requirement that
some document must actually represent the owner's intangible property right. On
the contrary, courts routinely apply the tort to intangibles without inquiring
whether they are merged in a document and, while it's often possible to dream
up some document the intangible is connected to in some fashion, it's seldom
one that represents the owner's property interest. To the extent
\textit{Olschewski} endorses the strict merger rule, it is against the weight
of authority. That rule cannot be squared with a jurisprudence that recognizes
conversion of music recordings, radio shows, customer lists, regulatory
filings, confidential information and even domain names.

Were it necessary to settle the issue once and for all, we would toe the line of
\textit{Payne} and hold that conversion is ``a remedy for the conversion of
every species of personal property.'' 54 Cal. at 341. But we need not do so to
resolve this case. Assuming arguendo that California retains some vestigial
merger requirement, it is clearly minimal, and at most requires only some
connection to a document or tangible object---not representation of the
owner's intangible interest in the strict Restatement sense.

Kremen's domain name falls easily within this class of property. He argues that
the relevant document is the Domain Name System, or ``DNS'' -- the distributed
electronic database that associates domain names like sex.com with particular
computers connected to the Internet. We agree that the DNS is a document (or
perhaps more accurately a collection of documents). That it is stored in
electronic form rather than on ink and paper is immaterial. It would be a
curious jurisprudence that turned on the existence of a paper document rather
than an electronic one. Torching a company's file room would then be conversion
while hacking into its mainframe and deleting its data would not. That is not
the law, at least not in California.\readingfootnote{11}{The
Restatement requires intangibles to be merged only in a ``document,'' not a
tangible document. \textsc{Restatement (Second) of Torts}
{\S} 242. Our holding therefore does not depend on whether electronic records
are tangible. \textit{Compare} \textit{eBay, Inc. v. Bidder's Edge, Inc}., 100
F. Supp. 2d 1058, 1069 (N.D. Cal. 2000) (``[I]t appears likely that the
electronic signals sent by [Bidder's Edge] to retrieve information from eBay's
computer system are\ldots sufficiently tangible to support a trespass cause of
action.''), \textit{with} \textit{Intel Corp. v. Hamidi,} 71 P.3d 296, (2003)
(implying that electronic signals are intangible).}

The DNS also bears some relation to Kremen's domain name. We need not delve too
far into the mechanics of the Internet to resolve this case. It is sufficient
to observe that information correlating Kremen's domain name with a particular
computer on the Internet must exist somewhere in some form in the DNS; if it
did not, the database would not serve its intended purpose. Change the
information in the DNS, and you change the website people see when they type
``www.sex.com.''

Network Solutions quibbles about the mechanics of the DNS. It points out that
the data corresponding to Kremen's domain name is not stored in a single
record, but is found in several different places: The components of the domain
name (``sex'' and ``com'') are stored in two different places, and each is
copied and stored on several machines to create redundancy and speed up
response times. Network Solutions's theory seems to be that intangibles are not
subject to conversion unless they are associated only with a single document.

Even if Network Solutions were correct that there is no single record in the DNS
architecture with which Kremen's intangible property right is associated, that
is no impediment under California law. A share of stock, for example, may be
evidenced by more than one document. \textit{See} \textit{Payne}, 54 Cal. at
342 (``[T]he certificate is only evidence of the property; and it is not the
only evidence, for a transfer on the books of the corporation, without the
issuance of a certificate, vests title in the shareholder: the certificate is,
therefore, but additional evidence of title.\ldots''). A customer list is
protected, even if it's recorded on index cards rather than a single piece of
paper. Audio recordings may be duplicated, and confidential information and
regulatory filings may be photocopied. Network Solutions's ``single document''
theory is unsupported.

Network Solutions also argues that the DNS is not a document because it is
refreshed every twelve hours when updated domain name information is broadcast
across the Internet. This theory is even less persuasive. A document doesn't
cease being a document merely because it is often updated. If that were the
case, a share registry would fail whenever shareholders were periodically added
or dropped, as would an address file whenever business cards were added or
removed. Whether a document is updated by inserting and deleting particular
records or by replacing an old file with an entirely new one is a technical
detail with no legal significance.

Kremen's domain name is protected by California conversion law, even on the
grudging reading we have given it. Exposing Network Solutions to liability when
it gives away a registrant's domain name on the basis of a forged letter is no
different from holding a corporation liable when it gives away someone's shares
under the same circumstances. We have not ``creat[ed] new tort duties'' in
reaching this result. \textit{Cf}. \textit{Moore v. Regents of the Univ. of
Cal}., 793 P.2d 479 (1990). We have only applied settled principles of
conversion law to what the parties and the district court all agree is a
species of property.

The district court supported its contrary holding with several policy
rationales, but none is sufficient grounds to depart from the common law rule.
The court was reluctant to apply the tort of conversion because of its strict
liability nature. This concern rings somewhat hollow in this case because the
district court effectively exempted Network Solutions from liability to Kremen
altogether, whether or not it was negligent. Network Solutions made no effort
to contact Kremen before giving away his domain name, despite receiving a
facially suspect letter from a third party. A jury would be justified in
finding it was unreasonably careless.

We must, of course, take the broader view, but there is nothing unfair about
holding a company responsible for giving away someone else's property even if
it was not at fault. Cohen is obviously the guilty party here, and the one who
should in all fairness pay for his theft. But he's skipped the country, and his
money is stashed in some offshore bank account. Unless Kremen's luck with his
bounty hunters improves, Cohen is out of the picture. The question becomes
whether Network Solutions should be open to liability for its decision to hand
over Kremen's domain name. Negligent or not, it was Network Solutions that gave
away Kremen's property. Kremen never did anything. It would not be unfair to
hold Network Solutions responsible and force it to try to recoup its losses by
chasing down Cohen. This, at any rate, is the logic of the common law, and we
do not lightly discard it. 

The district court was worried that ``the threat of litigation threatens to
stifle the registration system by requiring further regulations by [Network
Solutions] and potential increases in fees.'' Given that Network Solutions's
``regulations'' evidently allowed it to hand over a registrant's domain name on
the basis of a facially suspect letter without even contacting him, ``further
regulations'' don't seem like such a bad idea. And the prospect of higher fees
presents no issue here that it doesn't in any other context. A bank could lower
its ATM fees if it didn't have to pay security guards, but we doubt most
depositors would think that was a good idea.

The district court thought there were ``methods better suited to regulate the
vagaries of domain names'' and left it ``to the legislature to fashion an
appropriate statutory scheme.'' Id. The legislature, of course, is always free
(within constitutional bounds) to refashion the system that courts come up
with. But that doesn't mean we should throw up our hands and let private
relations degenerate into a free-for-all in the meantime. We apply the common
law until the legislature tells us otherwise. And the common law does not stand
idle while people give away the property of others.

The evidence supported a claim for conversion, and the district court should not
have rejected it.

