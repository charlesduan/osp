
This section considers forms of property that cannot be seen with the eye or
held in the hand. Such property raises significant conceptual issues, but,
simply put, it is too significant for the legal system to ignore. You have
already seen a few examples: corporate shares, for example, are a mixture of
voting rights and claims to the income the corporation produces; they give a
measure of control over tangible corporate assets, but they are very much
distinct from those assets. And contract rights---particularly through the
alchemy of assignability and negotiability---come to seem like property
rights, too: companies regularly pledge their accounts receivable as security
for loans, and no one bats an eye at the intangibility of the account
receivable (or of the creditor's rights under the loan, for that matter). You
have also now seen how people frequently hold intangible \textit{interests}
even in tangible property: a nonpossessory lien is such an interest, and you
will meet many more in the study of real property. As you read the cases in
this section, consider not just whether the things they describe are
``property,'' but also whether they are ``things'' in the first place. To
create a system of property rights, a legal system needs to be able to identify
the things that are the subject of those rights, to decide who owns those
things, and to be able to say when an owner's rights have been violated. Are
these tasks systematically harder for intangibles, and if so, why?

