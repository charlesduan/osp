\expected{us-v-turoff}

\item Reich's article is closely linked with \textit{Goldberg v. Kelly}, 397
U.S. 254 (1970), which held that welfare benefits could not be terminated
without notice and a hearing. In a footnote, the Court quoted \textit{The New
Property }and added, ``It may be realistic today to regard welfare entitlements
as more like `property' than a `gratuity.' Much of the existing wealth in this
country takes the form of rights that do not fall within traditional common-law
concepts of property.'' \textit{Id}. at 262 n.8. Two years later, in
\textit{Board of Regents of State Colleges v. Roth}, 408 U.S. 564 (1972), the
Court held that a state college professor on a renewable one-year contract did
not have a ``property'' interest in continued employment, so he had no
Fourteenth Amendment right to a statement of reasons for the nonrenewal of his
contract.\footnote{The court had previously held that written contracts or
state tenure law could create the necessary interest to trigger due process
protections, \textit{see} \textit{Slochower v. Board of Higher Ed. of New York
City}, 350 U.S. 551 (1956), and a companion case to \textit{Roth} held that a
professor might be entitled to due process protections when he alleged the
existence of an implicit understanding that professors who had been employed
for seven years would be dismissed only for cause. \textit{Perry v.
Sindermann}, 408 U.S. 593, 601--03 (1972),} The court had this to say about the
nature of ``property'':
\begin{quotation}
To have a property interest in a benefit, a person clearly must have more than
an abstract need or desire for it. He must have more than a unilateral
expectation of it. He must, instead, have a legitimate claim of entitlement to
it. It is a purpose of the ancient institution of property to protect those
claims upon which people rely in their daily lives, reliance that must not be
arbitrarily undermined. It is a purpose of the constitutional right to a
hearing to provide an opportunity for a person to vindicate those claims.

Property interests, of course, are not created by the Constitution. Rather, they
are created and their dimensions are defined by existing rules or
understandings that stem from an independent source such as state law---rules
or understandings that secure certain benefits and that support claims of
entitlement to those benefits.
\end{quotation}
\textit{Id.} at 577.\having{kremen-v-cohen}{ Is this an improvement on
\textit{Kremen}'s formulation?}{}{}
Does this formulation work for all property, all intangible property, or just
for government
benefits? What do you make of its thoughts about where property comes from?

\item Money is property because it is ``concrete and tangible,'' says the court
in \textit{Turoff}. Really? What if medallion owners pay their license renewal
fees by check? By credit card? Is it more or less tangible than the ``piece of
tin'' that is a taxicab medallion, the public's right to honest services, or
the franchise of operating a taxicab?

\item The Springfield Athletic Commission regulates boxing in the sense that
boxing for money or charging admission to a boxing match within the state of
Springfield is prohibited unless the match takes place under regulations
promulgated by the Commission. Some of the Commission's rules establish a
system of weight classes and determine who is the ``World'' champion within
each of those classes. Vinnie Watson is the current World Heavyweight Boxing
Champion, as determined by the Commission, whose rules allow it to revoke his
title unless he ``defends his title against a suitable challenger'' at least
once per year. Watson was been challenged to a match by Drederick Tatum, but
declined the challenge. The Commission then voted to revoke Watson title and
award it to Tatum instead; Watson has sued the Commission, claiming that
Tatum's poor won-loss record makes him not a ``suitable'' challenger. Do
alleging that the Commission's actions deny him ``property, without due process
of law'' within the meaning of the Fourteenth Amendment. Is his title property?
Does it matter whether the Commission has demanded that he return the
ceremonial belt that new champions hold over their heads?

\item Taxicab medallions typically can be sold on the open market. Liquor
licenses typically require a hearing before a local alcoholic beverages
commission before they can be transferred. A license to practice law is
personal and cannot be transferred at all. Does this mean that liquor licenses
and law licenses are not ``property?''

\item Is a franchise excludable? If someone steals the medallion from off your
taxicab, can you sue for replevin or conversion? What are the damages? Does
possession of the medallion give them the right to operate a taxicab on the
streets of New York? What are you to do in the meantime---in fact, what if you
never find the thief? Is your franchise gone? Now suppose that instead of
stealing your medallion, a fraudster forges one, using your medallion number.
Presumably this is an offense under state law, but does it invade your property
rights in your franchise? What if the fraudster forges a medallion using an
unassigned number? 

\item If Uber starts operating in your city without the approval of the TLC,
does that violate your property rights in your franchise? If the TLC doesn't
take action, can you sue the city for failing to enforce its franchise laws?
Does it matter whether you have an exclusive franchise---e.g., to be the only
operator of shuttle van service at an airport---or a nonexclusive
franchise---e.g., to be one of a number of operators of shuttle van service at
the airport?
Or, from the other side, can the \textit{denial} of a franchise invade property
rights? Is there a ``property'' interest in being allowed to operate a taxicab
for hire, such that a city government triggers the Fourteenth Amendment when it
refuses to allow Uber-dispatched cars to pick up passengers within city limits?
