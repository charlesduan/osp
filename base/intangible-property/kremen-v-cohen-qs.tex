\item Is your name your property? What is it about a domain name that makes it
``work'' as property?

\item Does \textit{Kremen}'s three-part test for the existence of ``property''
work on the variety of property forms you have encountered so far? Under it, is
a car property? A dog? A house? A right of publicity? Proper alignment of one's
psychic aura?

\item There are thriving markets for domain names. People buy and sell them all
the time, companies use them as collateral for loans, and Stephen Cohen
considered sex.com valuable enough to steal. But do these economic
considerations make them ``property?'' Recall Felix Cohen's argument against
basing property rights on economic value. Does \textit{Kremen} commit precisely
the fallacy the other Cohen warned about?

\item Is tangibility the key to the case or a giant red herring? Does the
court's argument in footnote 11 that under the Restatement an intangible must
be merged in a document but the document need not itself be tangible suggest
that there is something wrong with the Restatement's test or with the court's
reading of the Restatement.

\item The Internet Corporation for Assigned Names and Numbers (ICANN) has
promulgated a system of mandatory arbitration for domain-name registrants, the
Uniform Domain Name Dispute Resolution Policy (UDRP). Under the UDRP, a
trademark owner can bring an expedited proceeding against anyone who has
registered a domain name that is ``identical or confusingly similar'' to their
trademark if the domain name ``has been registered and is being used in bad
faith.'' If the arbitrator finds a violation, the remedy is transfer of the
domain name to the trademark owner. What does this system of protection for
trademark owners' property in their trademarks do to the security of
domain-name registrants' property in their domain names? Both are systems of
property in names, but can they coexist?

