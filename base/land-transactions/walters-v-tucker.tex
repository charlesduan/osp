\reading{Walters v. Tucker}
\readingcite{281 S.W.2d 843 (Sup. Ct. Mo. 1955)}

This is an action to quiet title to certain real estate situate in the City of
Webster Groves, St. Louis County, Missouri. Plaintiff and defendants are the
owners of adjoining residential properties fronting northward on Oak Street.
Plaintiff's property, known as 450 Oak Street, lies to the west of defendants'
property, known as 446 Oak Street. The controversy arises over their division
line. Plaintiff contends that her lot is 50 feet in width, east and west.
Defendants contend that plaintiff's lot is only approximately 42 feet in width,
east and west. The trial court, sitting without a jury, found the issues in
favor of defendants and rendered judgment accordingly, from which plaintiff has
appealed.

\captionedgraphic[width=0.7\textwidth,
height=0.8\textheight]{walters-v-tucker}{The plaintiff's survey plat of the land
in question.}

The common source of title is Fred F. Wolf and Rose E. Wolf, husband and wife,
who in 1922 acquired the whole of Lot 13 of West Helfenstein Park, as shown by
plat thereof recorded in St. Louis County. In 1924, Mr.~and Mrs.~Wolf conveyed
to Charles Arthur Forse and wife the following described portion of said Lot
13: 
\begin{quote}
The West 50 feet of Lot 13 of West Helfenstein Park, a Sub-division in United
States Survey 1953, Twp. 45, Range 8 East, St. Louis County, Missouri\ldots.
\end{quote}
Plaintiff, through mesne conveyances carrying a description like that above, is
the last grantee of and successor in title to the aforesaid portion of Lot 13.
Defendants, through mesne conveyances, are the last grantees of and successors
in title to the remaining portion of Lot 13. 

At the time of the above conveyance in 1924, there was and is now situate on the
tract described therein a one-story frame dwelling house (450 Oak Street),
which was then and continuously since has been occupied as a dwelling by the
successive owners of said tract, or their tenants. In 1925, Mr. and Mrs. Wolf
built a 1 1/2-story stucco dwelling house on the portion of Lot 13 retained by
them. This house (446 Oak Street) continuously since has been occupied as a
dwelling by the successive owners of said portion of Lot 13, or their tenants. 

Despite the apparent clarity of the description in plaintiff's deed, extrinsic
evidence was heard for the purpose of enabling the trial court to interpret the
true meaning of the description set forth therein. At the close of all the
evidence the trial court found that the description did not clearly reveal
whether the property conveyed ``was to be fifty feet along the front line facing
Oak Street or fifty feet measured Eastwardly at right angles from the West line
of the property\dots''; that the ``difference in method of ascertaining fifty
feet would result in a difference to the parties of a strip the length of the
lot and approximately eight feet in width''; that an ambiguity existed which
justified the hearing of extrinsic evidence; and that the ``West fifty feet
should be measured on the front or street line facing Oak Street.'' The judgment
rendered in conformity with the above finding had the effect of fixing the
east-west width of plaintiff's tract at about 42 feet. 

Plaintiff contends that the description in the deed is clear, definite and
unambiguous, both on its face and when applied to the land; that the trial
court erred in hearing and considering extrinsic evidence; and that its finding
and judgment changes the clearly expressed meaning of the description and
describes and substitutes a different tract from that acquired by her under her
deed. Defendants do not contend that the description, on its face, is
ambiguous, but do contend that when applied to the land it is subject to ``dual
interpretation''; that under the evidence the trial court did not err in finding
it contained a latent ambiguity and that parol evidence was admissible to
ascertain and determine its true meaning; and that the finding and judgment of
the trial court properly construes and adjudges the true meaning of the
description set forth in said deed.

[The plaintiff and defendants introduced dueling survey plats. The one included
here is the plaintiff's. North is at the bottom. Note in particular the
locations of the two houses and of the driveway. It may help to mark on the
plat where the defendant's proposed line would fall.]

It is seen that Lot 13 extends generally north and south. It is bounded on the
north by Oak Street (except that a small triangular lot from another
subdivision cuts off its frontage thereon at the northeast corner). On the
south it is bounded by the Missouri Pacific Railroad right of way. Both Oak
Street and the railroad right of way extend in a general northeast-southwest
direction, but at differing angles.\ldots

Both plats show a concrete driveway 8 feet in width extending from Oak Street to
plaintiff's garage in the rear of her home, which, the testimony shows, was
built by one of plaintiff's predecessors in title. The east line of plaintiff's
tract, as measured by the Joyce (plaintiff's) survey, lies 6 or 7 feet east of
the eastern edge of this driveway. Admittedly, the driveway is upon and an
appurtenance of plaintiff's property. On the Elbring (defendants') plat, the
east line of plaintiff's lot, as measured by Elbring, is shown to coincide with
the east side of the driveway at Oak Street and to encroach upon it 1.25 feet
for a distance of 30 or more feet as it extends between the houses. Thus, the
area in dispute is essentially the area between the east edge of the driveway
and the line fixed by the Joyce survey as the eastern line of plaintiff's
tract.\ldots

The description under which plaintiff claims title, to wit: ``The West 50 feet
of
Lot 13 \dots'', is on its face clear and free of ambiguity. It purports to
convey a strip of land 50 feet in width off the west side of Lot 13. So clear
is the meaning of the above language that defendants do not challenge it and it
has been difficult to find any case wherein the meaning of a similar
description has been questioned. 

The law is clear that when there is no inconsistency on the face of a deed and,
on application of the description to the ground, no inconsistency appears,
parol evidence is not admissible to show that the parties intended to convey
either more or less or different ground from that described. But where there
are conflicting calls in a deed, or the description may be made to apply to two
or more parcels, and there is nothing in the deed to show which is meant, then
parol evidence is admissible to show the true meaning of the words used. 

No ambiguity or confusion arises when the description here in question is
applied to Lot 13. The description, when applied to the ground, fits the land
claimed by plaintiff and cannot be made to apply to any other tract. When the
deed was made, Lot 13 was vacant land except for the frame dwelling at 450 Oak
Street. The stucco house (446 Oak Street) was not built until the following
year. Under no conceivable theory can the fact that defendants' predecessors in
title (Mr.~and Mrs.~Wolf) thereafter built the stucco house within a few feet
of the east line of the property described in the deed be construed as
competent evidence of any ambiguity in the description.\ldots

Whether the above testimony and other testimony in the record constitute
evidence of a mistake in the deed we do not here determine. Defendants have not
sought reformation, and yet that is what the decree herein rendered undertakes
to do. It seems apparent that the trial court considered the testimony and came
to the conclusion that the parties to the deed did not intend a conveyance of
the ``West 50 feet of Lot 13'', but rather a tract fronting 50 feet on Oak
Street. And, the decree, on the theory of interpreting an ambiguity, undertakes
to change (reform) the description so as to describe a lot approximately 42
feet in width instead of a lot 50 feet in width, as originally described. That,
we are convinced, the courts cannot do.

