\expected{walters-v-tucker}

\item Why does the court apply such a strict integration rule?

\item The boundary line as enforced by the court comes within inches of the
defendants' house. This does not seem like an ideal state of affairs. (Then
again, the defendant's theory would have drawn the boundary line through the
plaintiffs' driveway.) Are there any doctrines that can clean up the messes
that result when (by accident or otherwise) strict interpretation of deeds
produces results at odds with natural features, structures, or uses of land?


\item The deed here used three different techniques to describe the land. Start
at the end. ``United States Survey 1953, Twp. 45, Range 8 East, St. Louis
County, Missouri'' is a reference to a government survey. Townships are
standard 36-square-mile tracts established by federal government survey; ``Twp.
45, Range 8 East'' identifies a specific township in Missouri. Next, ``of Lot
13 of West Helfenstein Park'' is a reference to the \textit{subdivision plat}
filed by the developer who laid out the neighborhood; the plat is a survey map
filed in the county recording office that shows the boundaries of individual
parcels. Finally, ``The West 50 feet'' is a (crude attempt at) a \textit{metes
and bounds} description of the property in terms of its boundaries. Metes and
bounds descriptions may refer to geospatial coordinates (e.g. latitude and
longitude as measured by GPS), to natural landmarks (``Millers' Creek''),
artificial markers (``the survey stake labelled G34''), and distances and
directions (``300 feet along a course at 45[2DA?]). How precise are these
various means of description? Which of them strike you as most prone to error?


\item Note that the boundary lines as shown on the survey map are at an angle to
the north-south axis. Does this affect how the court should interpret the deed?

