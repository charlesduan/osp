In 1250, to transfer ownership of land, the grantor and grantee would physically
go to the land. The grantor would physically (or perhaps metaphysically) put
the grantee in possession by handing over a clod of dirt. The grantee would
swear homage to the grantor, and the grantor would swear to defend the
grantee's title. This was a public ceremony, performed in front of witnesses
who could later be called on to recall what had happened if necessary. In
contrast, written conveyances---called ``charters''---were treated with
skepticism; they were considered an inferior form of evidence because of the
risk of forgery.

In the seven and a half centuries since, this attitude has completely flipped.
Now, land transactions are paper transactions: the Statute of Frauds almost
always requires a written conveyance---now called a ``deed''---to transfer an
interest in real property. Transfers by operation of law (primarily through
adverse possession and intestacy) are very much the exception. In addition,
land transactions are influenced by the common law's attitude that land is of
distinctive importance, so that parties dealing with it need especial clarity
about their rights, and by the fact that land transactions are often
high-stakes, with hundreds of thousands, millions, or sometimes even billions
of dollars at issue. This section focuses on the written instruments at the
heart of land transactions. It considers when a deed is required, when a deed
is effective, how deeds are interpreted, and what they promise about the
property and the interest being conveyed.


