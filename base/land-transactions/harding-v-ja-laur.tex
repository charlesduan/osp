\reading{Harding v. Ja Laur}
\readingcite{315 A.2d 132 (Md. Ct. Spec. App. 1974)}

\opinion \textsc{Gilbert}, Judge:\ldots

The bill alleged that a deed had been obtained from the appellant through fraud
practiced upon her by the agent of Ja Laur Corporation. The bill further
averred that the paper upon which the appellant had affixed her signature was
``falsely and fraudulently attached to the first page of a deed identified as
the same deed'' through which the appellee, Ja Laur Corporation, and its
assigns, the other appellees, claim title.\ldots

There is no dispute that the appellant signed some type of paper. Her claim is
not that her signature was forged in the normal sense, i.e., someone copied or
wrote it, but rather that the forgery is the result of an alteration. Mrs.
Harding alleges that at the time that she signed a blank paper she was told
that her signature was necessary in order to straighten out a boundary line.
She represents that she did not know that she was conveying away her interest
in and to a certain 1517 acres of land in Montgomery County. 

The parcel of land that was conveyed by the allegedly forged deed is contiguous
to a large tract of real estate in which Ja Laur and others had ``a substantial
interest.'' It appears from the bill that Mrs. Harding's land provided the
access from the larger tract to a public road, so that its value to the
appellees is obvious. Mrs. Harding excuses herself for signing the ``blank
paper'' by averring that she did so at the instigation of an attorney, an agent
of Ja Laur, who had ``been a friend of her deceased husband, and ...
represented her deceased husband in prior business and legal matters, and that
under [the] circumstances [she] did place her complete trust and reliance in
the representations made to her\ldots'' by the attorney. The ``blank paper'' was
signed ``on or about April 2, 1970.'' Mrs. Harding states that she did not
learn of the fraud until the ``summer of 1972.'' At that time an audit, by the
Internal Revenue Service, of her deceased husband's business revealed the deed
to Ja Laur, and its subsequent conveyance to the other appellees. 

In \textit{Smith v. State}, 256 A.2d 357, 360 (1970), we said that:
\begin{quote}
Forgery has been defined as a false making or material alteration, with intent
to defraud, of any writing which, if genuine, might apparently be of legal
efficacy or the foundation of a legal liability. More succinctly, forgery is
the fraudulent making of a false writing having apparent legal significance. It
is thus clear that one of the essential elements of forgery is a writing in
such form as to be apparently of some legal efficacy and hence capable of
defrauding or deceiving. 
\end{quote}

Perkins, \textit{Criminal Law} ch. 4, {\S} 8 (2d ed. 1969) states, at 351:
\begin{quote}
A material alteration may be in the form of (1) an addition to the writing, (2)
a substitution of something different in the place of what originally appeared,
or (3) the removal of part of the original. The removal may be by erasure or in
some other manner, such as by cutting off a qualifying clause appearing after
the signature. 
\end{quote}

A multitude of cases hold that forgery includes the alteration of or addition to
any instrument in order to defraud. That a deed may be the subject of a forgery
is beyond question. 

The Bill of Complaint alleges that the signature of Mrs. Harding was obtained
through fraud. More important, however, to the issue is whether or not the bill
alleges forgery. In our view the charge that appellant's signature was written
upon a paper, which paper was thereafter unbeknown to her made a part of a
deed, if true, demonstrates that there has been a material alteration and hence
a forgery.\ldots

We turn now to the discussion of whether \textit{vel non} the demurrers of Macro
Housing, Inc. and Montgomery County, the other appellees, should have been
sustained. There was no allegation in the bill that their agent had perpetrated
the fraud upon Mrs. Harding. If they are to be held in the case, it must be on
the basis that they are not \textit{bona fide }purchasers without notice. The
title of a \textit{bona fide }purchaser, without notice, is not vitiated even
though a fraud was perpetrated by his vendor upon a prior title holder. A deed
obtained through fraud, deceit or trickery is voidable as between the parties
thereto, but not as to a \textit{bona fide }purchaser. A forged deed, on the
other hand, is void \textit{ab initio}.\ldots

[T]he common law rule that a forger can pass no better title than he has is in
full force and effect in this State. A forger, having no title can pass none to
his vendee. Consequently, there can be no \textit{bona fide} holder of title
under a forged deed. A forged deed, unlike one procured by fraud, deceit or
trickery is void from its inception. The distinction between a deed obtained by
fraud and one that has been forged is readily apparent. In a fraudulent deed an
innocent purchaser is protected because the fraud practiced upon the signatory
to such a deed is brought into play, at least in part, by some act or omission
on the part of the person upon whom the fraud is perpetrated. He has helped in
some degree to set into motion the very fraud about which he later complains. A
forged deed, on the other hand, does not necessarily involve any action on the
part of the person against whom the forgery is committed. So that if a person
has two deeds presented to him, and he thinks he is signing one but in
actuality, because of fraud, deceit or trickery he signs the other, a bona fide
purchaser, without notice, is protected. On the other hand, if a person is
presented with a deed, and he signs that deed but the deed is thereafter
altered e.g. through a change in the description or affixing the signature page
to another deed, that is forgery and a subsequent purchaser takes no title. 

In the instant case, the Bill of Complaint, for the reasons above stated,
alleged a forgery of the deed by which Ja Laur took title from Mrs. Harding.
This allegation, if true, renders that deed a nullity. Ja Laur could not have
passed title to the other appellees, Macro Housing, Inc. and Montgomery County.
Those two appellees would therefore have no title to the land of Mrs.
Harding.\ldots

