\reading{Brush Grocery Kart, Inc. v. Sure Fine Market, Inc.}
\readingcite{47 P.3d 680 (Colo. 2002)}

\textsc{JUSTICE COATS} delivered the opinion of the court: ... 

In October 1992 Brush Grocery Kart, Inc. and Sure Fine Market, Inc. entered into
a five-year ``Lease with Renewal Provisions and Option to Purchase'' for real
property, including a building to be operated by Brush as a grocery store.
Under the contract's purchase option provision, any time during the last six
months of the lease, Brush could elect to purchase the property at a price
equal to the average of the appraisals of an expert designated by each party. 

Shortly before expiration of the lease, Brush notified Sure Fine of its desire
to purchase the property and begin the process of determining a sale price.
Although each party offered an appraisal, the parties were unable to agree on a
final price by the time the lease expired. Brush then vacated the premises,
returned all keys to Sure Fine, and advised Sure Fine that it would discontinue
its casualty insurance covering the property during the lease. Brush also filed
suit, alleging that Sure Fine failed to negotiate the price term in good faith
and asking for the appointment of a special master to determine the purchase
price. Sure Fine agreed to the appointment of a special master and
counterclaimed, alleging that Brush negotiated the price term in bad faith and
was therefore the breaching party. 

During litigation over the price term, the property was substantially damaged
during a hail storm. With neither party carrying casualty insurance, each
asserted that the other was liable for the damage. The issue was added to the
litigation at a stipulated amount of \$60,000. ... The court then found that
under the doctrine of equitable conversion, Brush was the equitable owner of
the property and bore the risk of loss. It therefore declined to abate the
purchase price or award damages to Brush for the loss. 

Brush appealed the loss allocation, and the court of appeals affirmed on similar
grounds. ... 

In the absence of statutory authority, the rights, powers, duties, and
liabilities arising out of a contract for the sale of land have frequently been
derived by reference to the theory of equitable conversion. This theory or
doctrine, which has been described as a legal fiction, is based on equitable
principles that permit the vendee to be considered the equitable owner of the
land and debtor for the purchase money and the vendor to be regarded as a
secured creditor. The changes in rights and liabilities that occur upon the
making of the contract result from the equitable right to specific performance.
Even with regard to third parties, the theory has been relied on to determine,
for example, the devolution, upon death, of the rights and liabilities of each
party with respect to the land, and to ascertain the powers of creditors of
each party to reach the land in payment of their claims. 

The assignment of the risk of casualty loss in the executory period of contracts
for the sale of real property varies greatly throughout the jurisdictions of
this country. What appears to yet be a slim majority of states places the risk
of loss on the vendee from the moment of contracting, on the rationale that
once an equitable conversion takes place, the vendee must be treated as owner
for all purposes. Once the vendee becomes the equitable owner, he therefore
becomes responsible for the condition of the property, despite not having a
present right of occupancy or control. In sharp contrast, a handful of other
states reject the allocation of casualty loss risk as a consequence of the
theory of equitable conversion and follow the equally rigid ``Massachusetts
Rule,'' under which the seller continues to bear the risk until actual transfer
of the title, absent an express agreement to the contrary. A substantial and
growing number of jurisdictions, however, base the legal consequences of
no-fault casualty loss on the right to possession of the property at the time
the loss occurs. This view has found expression in the Uniform Vendor and
Purchaser Risk Act, and while a number of states have adopted some variation of
the Uniform Act, others have arrived at a similar position through the
interpretations of their courts. ... 

In \textit{Wiley v. Lininger}, 204 P.2d 1083, [(1949)] where fire destroyed
improvements on land occupied by the vendee during the multi-year executory
period of an installment land contract, we held, according to the generally
accepted rule, that neither the buyer nor the seller, each of whom had an
insurable interest in the property, had an obligation to insure the property
for the benefit of the other. We also adopted a rule, which we characterized as
``the majority rule,'' that ``the vendee under a contract for the sale of land,
being regarded as the equitable owner, assumes the risk of destruction of or
injury to the property \textit{where he is in possession}, and the destruction
or loss is not proximately caused by the negligence of the vendor.''
\textit{Id}. (emphasis added). The vendee in possession was therefore not
relieved of his obligation to continue making payments according to the terms
of the contract, despite material loss by fire to some of the improvements on
the property. ... Those jurisdictions that indiscriminately include the risk of
casualty loss among the incidents or ``attributes{\textquotedbl} of equitable
ownership do so largely in reliance on ancient authority or by considering it
necessary for consistent application of the theory of equitable conversion.
Under virtually any accepted understanding of the theory, however, equitable
conversion is not viewed as entitling the purchaser to every significant right
of ownership, and particularly not the right of possession. As a matter of both
logic and equity, the obligation to maintain property in its physical condition
follows the right to have actual possession and control rather than a legal
right to force conveyance of the property through specific performance at some
future date. See 17 \textsc{Samuel Williston, A Treatise On the Law of
Contracts} {\S} 50:46, at 457-58 (Richard A. Lord ed., 4th ed. 1990) (``[I]t is
wiser to have the party in possession of the property care for it at his peril,
rather than at the peril of another.''). 

The equitable conversion theory is literally stood on its head by imposing on a
vendee, solely because of his right to specific performance, the risk that the
vendor will be unable to specifically perform when the time comes because of an
accidental casualty loss. It is counterintuitive, at the very least, that
merely contracting for the sale of real property should not only relieve the
vendor of his responsibility to maintain the property until execution but also
impose a duty on the vendee to perform despite the intervention of a material,
no-fault casualty loss preventing him from ever receiving the benefit of his
bargain. Such an extension of the theory of equitable conversion to casualty
loss has never been recognized by this jurisdiction, and it is neither
necessary nor justified solely for the sake of consistency. 

By contrast, there is substantial justification, both as a matter of law and
policy, for not relieving a vendee who is entitled to possession before
transfer of title, like the vendee in \textit{Wiley}, of his duty to pay the
full contract price, notwithstanding an accidental loss. In addition to having
control over the property and being entitled to the benefits of its use, an
equitable owner who also has the right of possession has already acquired
virtually all of the rights of ownership and almost invariably will have
already paid at least some portion of the contract price to exercise those
rights. By expressly including in the contract for sale the right of
possession, which otherwise generally accompanies transfer of title, the vendor
has for all practical purposes already transferred the property as promised,
and the parties have in effect expressed their joint intention that the vendee
pay the purchase price as promised. ... 

In the absence of a right of possession, a vendee of real property that suffers
a material casualty loss during the executory period of the contract, through
no fault of his own, must be permitted to rescind and recover any payments he
had already made. ... 

Here, Brush was clearly not in possession of the property as the equitable
owner. Even if the doctrine of equitable conversion applies to the option
contract between Brush and Sure Fine and could be said to have converted
Brush's interest to an equitable ownership of the property at the time Brush
exercised its option to purchase, neither party considered the contract for
sale to entitle Brush to possession. Brush was, in fact, not in possession of
the property, and the record indicates that Sure Fine considered itself to hold
the right of use and occupancy and gave notice that it would consider Brush a
holdover tenant if it continued to occupy the premises other than by continuing
to lease the property. The casualty loss was ascertainable and in fact
stipulated by the parties, and neither party challenged the district court's
enforcement of the contract except with regard to its allocation of the
casualty loss. Both the court of appeals and the district court therefore erred
in finding that the doctrine of equitable conversion required Brush to bear the
loss caused by hail damage. 


