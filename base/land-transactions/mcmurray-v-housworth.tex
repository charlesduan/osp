\reading{McMurray v. Housworth}

\readingcite{638 S.E.2d 421 (Ga. Ct. App. 2006)}

\opinion \textsc{Phipps}, Judge: 

Michael and Deborah Housworth sold a 24-acre tract of land which the
purchasers---Lance and Melanie McMurray, and James and Alberta McMurray---
subdivided into two tracts. A lake created by a dam is situated on the
property. The McMurrays brought this suit against the Housworths for breach of
their general warranty of title upon discovering after purchasing the property
that the owner and operator of the dam holds a floodwater detention easement
that burdens the tract. The superior court awarded summary judgment to the
Housworths on the ground that this easement is not such an encumbrance on the
property as breaches the title warranty. We disagree and reverse. 

Lance and Melanie McMurray purchased one of the twelve-acre parcels from the
Housworths for \$120,000 in 2004. On the same date, James and Alberta McMurray
purchased the other parcel for the same price. The parcels were conveyed by
warranty deeds that contained general warranties of title without any
limitations applicable here. The McMurrays informed the Housworths that they
were buying the property to build single-family residences on each parcel. 

Apparently, however, the McMurrays failed to discover that recorded within the
chain of title to their property in 1962 was a ``floodwater retarding
structure'' easement which had been granted to the Oconee River Soil
Conservation District. This easement is for construction, operation, and
maintenance of a floodwater retarding structure or dam; for the flowage of
waters in, over, upon, or through the dam; and for the permanent storage and
temporary detention of any waters that are impounded, stored, or detained by
the dam. It also reserved in the grantor and his successors the right to use
the easement area for any purpose not inconsistent with full use and enjoyment
of the grantee's rights and privileges, i.e., it is nonexclusive. After
learning of the easement following their purchase of the property, the
McMurrays demanded that the Housworths compensate them for the damages they
would suffer as a result of the restrictions thereby placed on their usage. 

Because the Housworths failed to comply with these demands, the McMurrays
brought this suit against them seeking damages for breach of their warranties
of title. ... 

1. The McMurrays contend that the superior court erred in analogizing the
floodwater detention easement to a public roadway easement or zoning regulation
and in thereby concluding that a floodwater detention easement is not the type
of easement that breaches a general warranty of title. 

(a) Each of the deeds in this case contained a general warranty of title in
which the grantors agreed to ``defend the right and title to the above
described property, unto [the grantees], their heirs, assigns, and successors
in title, against the claims of all persons.'' Under OCGA {\S} 44-5-62, ``[a]
general warranty of title against the claims of all persons includes covenants
of a right to sell, of quiet enjoyment, and of freedom from encumbrances.''
``An incumbrance has been defined as `Any right to, or interest in, land which
may subsist in another to the diminution of its value, but consistent with the
passing of the fee,' and this definition\ldots encompasses an easement or right
of way.'' OCGA {\S} 44-5-63 provides that ``[i]n a deed, a general warranty of
title against the claims of all persons covers defects in the title even if
they are known to the purchaser at the time he takes the deed.'' 

(b) The rule in Georgia, as established in the early case of \textit{Desvergers
v. Willis}, 56 Ga. 515 (1876), is that the existence of a public road on land,
of which the purchaser knew or should have known at the time of the purchase,
is not such an encumbrance as would constitute a breach of a general warranty
of title. The \textit{Desvergers }rule is thus an exception to the general rule
stated in OCGA {\S} 44-5-63 that a general warranty of title by deed covers
even defects known to the purchaser at the time he takes the deed. 

Although the \textit{Desvergers }rule is not uniform throughout the country, it
is the majority rule. In adopting the rule, the court in \textit{Desvergers
}concluded that a contrary holding would produce a ``crop of litigation'' that
would be ``almost interminable.'' The reason, as later explained by the Supreme
Court of Iowa in \textit{Harrison v. The Des Moines \& Ft. Dodge R. Co., }was
that the immense number of warranty deeds then in existence rarely contained
exceptions as to public roadways because of the universal belief that roadway
access was a benefit rather than a burden to land. Therefore, a determination
that public roadway easements were warranty-breaching encumbrances would have
created innumerable liabilities where none had been thought to exist. 

Courts in other states have also based their adoption of the \textit{Desvergers
}rule on the broader ground that where easements are open, notorious, and
presumably known to the purchaser at the time of the purchase, that knowledge
will exclude the easement from operation of a title warranty. These courts have
reasoned that where the encumbrance involves an open and obvious physical
condition of the property, the purchaser is presumed to have seen it and fixed
his price with reference to it. In view, however, of the Georgia rule that
knowledge of a title defect will not exclude it from operation of a general
warranty of title, creation of an exception for easements for public roadways
or other purposes must be based on other grounds. And courts in other states
have ultimately concluded that public roadway easements should not be regarded
as encumbrances on the additional ground that ``public highways are not
depreciative, but, on the contrary, they are highly appreciative, of the value
of the lands on which they constitute an easement, and are a means without
which such lands are not available for use, nor sought after in the markets.'' 

For a number of reasons, we do not find the floodwater detention easement in
this case analogous to a public roadway easement. (1) We do not anticipate that
we would open the litigation floodgates, so to speak, by holding that a
floodwater detention easement breaches a general title warranty. (2) Moreover,
a floodwater detention easement does not benefit the land to which it is
subject. Although the property is benefitted by the lake or other body of water
that creates the need for the easement (to the extent that the one enhances the
value or enjoyment of the other), the easement burdens the property by
permitting the impoundment of water on it to prevent flooding or increased
water runoff on other property located downstream. (3) The McMurrays brought
this action for damages because of the easement, not the lake. And even though
the lake is certainly open and obvious, the same cannot necessarily be said of
the easement. Although the superior court found that the dam is visible on the
McMurrays' property, the McMurrays correctly point out that there is no
evidence of record to support this finding. As argued by the McMurrays, not
every lake is created by a dam or burdened by a floodwater detention easement.
(4) And although the McMurrays' constructive notice of the easement by reason
of its recordation within their chains of title would provide a compelling
reason for exempting the easement from operation of the warranty deed, OCGA
{\S} 44-5-63 provides otherwise. (5) The recording of the easement certainly
renders it binding on the McMurrays insofar as concerns the rights of the
easement holder; but the question here is whether the existence of the easement
gives rise to a claim against the grantor for breach of the warranty against
encumbrances. For these reasons, the superior court erred in concluding that
the floodwater detention easement should be excepted from the rule of OCGA {\S}
44-5-63 in view of the exception for public roadways. 

(c) The McMurrays also contend that the superior court erred in equating
floodwater detention easements with zoning regulations, which have been held
not to breach a general warranty of title. Because the floodwater detention
easement does not function in the same manner as a zoning regulation in all
respects, we agree with this contention. 

The floodwater detention easement does more than impose zoning-type restrictions
on development activities on the property. It also grants the county soil and
water conservation district rights for the storage and detention of impounded
waters on the property. And it grants the district a right of ingress and
egress upon the property. Easement rights such as these constitute an interest
in property that must be acquired either by agreement of the property owner or
by condemnation. And although the easement does impose limitations on the
McMurrays' use of their property that duplicate restrictions imposed under
zoning-type regulations applicable to the property, the two do not appear to be
coextensive. ... 

Where an encumbrance is a servitude or easement which can not be removed at the
option of either the grantor or grantee, damages will be awarded for the injury
proximately caused by the existence and continuance of the encumbrance, the
measure of which is deemed to be the difference between the value of the land
as it would be without the easement and its value as it is with the easement.

