\expected{loughran-v-kummer}

\item The old phrase is that a deed was effective when it was ``signed, sealed,
and delivered.'' But the seal is obsolete, so the principal elements are that
it be a sufficient writing (discussed above), that it be signed, and that it be
delivered. Delivery of deeds has much in common with delivery in the law of
gifts; it too can be a subtle question. In a famous passage of his landmark
17th-century treatise, \textit{Institutes of the Lawes of England}, Edward Coke
wrote, ``As a deed may be delivered to a party without words, so may a deed be
delivered by words without any act of delivery.'' That sounds paradoxical, but
Coke continued, ``as if the writing sealed lies upon the table, and the
[grantor] says to the [grantee], `Go and take up that writing, it is sufficient
for you;' or `it will serve your turn;' or `Take it as my deed;' or the like
words; either is a sufficient delivery.'' Is that better?


\item In \textit{Wiggill v. Cheney}, 597 P.2d 1351 (Utah 1979), Lillian Cheney
executed a deed to Flora Cheney and put it in a safety deposit box in the names
of Lillian Cheney and Francis E. Wiggill. Lillian told Francis that his name
was on the box, that on her death he would be granted access to the box, and
that ``in that box is an envelope addressed to all those concerned. All you
have to do is give them that envelope and that's all.'' On her death, he gained
access to the box, took the deed, and gave it to Flora. Delivery? 


\item There are at least two ways to do delivery ``right.'' One is to sign and
hand over a deed at closing, when all of the necessary parties are in the same
room and can execute all of the appropriate documents effectively
simultaneously. Another is to use an escrow: a third party who receives custody
of the signed deed along with instructions to deliver it to the grantee when
appropriate events have taken place. What if the escrow agent disregards her
instructions and hands over the deed early? Can a grantor who is concerned the
transaction will fall through demand the deed back from the escrow agent?


\item \textit{Loughran} is more complicated because the parties intended a
conditional gift that would take effect at Loughran's death, rather than
immediately. Grantors often try to put other kinds of conditions on transfers.
In \textit{Martinez v. Martinez}, 678 P.2d 1163 (N.M. 1984), Delfino and
Eleanor Martinez gave their son Carlos and his wife Sennie a deed to a property
in exchange for assuming a mortgage in it. Delfino and Eleanor instructed
Carlos and Sennie to take the deed to the bank to be held in escrow until
Carlos and Sennie had paid off the mortgage, but they recorded it first. Carlos
and Sennie had marital difficulties and fell behind on the mortgage; eventually
Delfino and Eleanor paid off the balance. Who owns the property?


\item The \textit{Loughran} court says the parties ``have not adopted the proper
legal means of accomplishing their object.'' What does it mean? Is there
anything they could have done differently that would avoided this mess?

