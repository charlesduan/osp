\reading{Loughran v. Kummer}
\readingcite{146 A. 534 (Pa. 1929)}

KEPHART, J.

Appellee, a bachelor 67 years of age, conveyed, for \$1, land in Pittsburgh to
Mrs. Kummer, appellant, who was one of his tenants. A bill was filed to set
aside this deed; the grounds laid were confidential relationship, undue
influence, and impaired mentality. Inasmuch as the facts must again be
considered, we will mention only such as raise the legal question on which the
case was decided; we venture no opinion on the other facts.

The court below found from the evidence that a deed absolute on its face had
been executed, acknowledged, and delivered to appellant by appellee, on
condition that it should not be recorded until the latter's death; that
undoubtedly in his mind this meant that the deed was not to take effect until
after his death; and that he, demanding the return of the deed within a very
few days after the delivery, thus revoked it and with that revocation revoked
the gift. Appellant deceived appellee when she stated the deed had been
destroyed. The excuse given was appellee was worried and she wanted to ease his
mind by making him believe that it had been destroyed. \dots{}

The question we are asked to consider is whether a deed absolute on its face,
acknowledged, executed, and delivered under circumstances as here indicated,
vested such title in the grantee as could be revoked for the above reasons. It
amounts in substance to this, that the grantor said the deed should not be
recorded until after his death, and the grantee in accepting the deed took it
on that condition. The evidence on which this finding was based was all oral,
and the scrivener and defendant denied any such condition was imposed when the
deed was delivered. All control over the deed was relinquished when it was
handed appellant. The presumption must be that at that time it was the
intention to pass title. `The general principle of law is that the formal act
of signing, sealing and delivering is the consummation of the deed, and it lies
with the grantor to prove clearly that appearances are not consistent with
truth. The presumption stands against him, and the burden is on him to destroy
it by clear and positive proof that there was no delivery and that it was so
understood at the time. \dots{} Where we have, as here, a deed, absolute and
complete in itself, attacked as being in fact otherwise intended, \dots{} there
is a further presumption that the title is in conformity with the deed, and it
should not be dislodged except by clear, precise, convincing and satisfactory
evidence to the contrary.' \textit{Cragin's Estate}, 117 A. 445 (Pa. 1922).

The gift here was executed, and that defendant was not to record it was not of
the slightest consequence when viewed as against these major actions, delivery
and passing of title. It was merely a promise the keeping of which lay in good
faith, the breach of which entailed no legal consequences. To have effected the
grantor's purpose, the intervention of a third party was absolutely essential.
There are circumstances where acknowledgment, together with physical possession
of the deed in the grantee, does not conclusively establish an intention to
deliver, and the presumption arising from signing, sealing, and acknowledging,
accompanied by manual possession of the deed by the grantee, is not
irrebuttable, but this presumption can be overcome only by evidence that no
delivery was in fact intended and none made. Such evidence is not present in
this case. Here the grantor by his own testimony intended the grantee to get
the land. The only question was when it was to take effect.

Here is one of the instances in which the law fails to give effect to the honest
intention of the parties, for the reason that they have not adopted the proper
legal means of accomplishing their object. Therefore the legal effect of such
delivery is not altered by the fact that both parties suppose the deed will not
take effect until recorded, and that it may be revoked at any time before
record, or by contemporaneous agreements looking to the reconveyance of the
property to the grantor or to the third party upon the happening of certain
contingent events or the nonperformance of certain conditions.

The reason for these rules is obvious. It is quite possible to prove in most
deliveries that some parol injunction was attached to the formal delivery of
the deed; if they are to be given the effect her[e] contended, there would be
no safety in accepting a deed under most circumstances. It opens the door to
the fabrication of evidence that would inevitably be appalling and go far
toward violating the security of written instruments. We have so held in
matters of less import than the conveyance of land. The rule must not be
relaxed as to realty. Such conveyances are vastly more important, as they
involve instruments of title and ownership which are used as a means of
extending credit. Title to land ought not to be exposed to the peril of
successful attack except where the right is clear and undoubted, and whatever
may be our desire to recognize circumstances argued as unfortunate, we cannot
go to the extent of overthrowing principles of law governing conveyances of
real estate that have stood the test of ages.

In \textit{Cragin's Estate}, supra, the deeds were in a tin box for more than 23
years in an envelope indorsed with the words: `To be recorded upon Mrs.
Cragin's death, if before me.' The deed was in grantee's possession, and it was
urged the delivery was conditional. We said that indorsement may have been
placed on the envelope for other reasons than to defer the transfer of title.
In the present case it was evident appellee did not want his relatives to learn
of the conveyance. Recording would be necessary to pass a title examiner's
inspection, but nonrecording did not prevent the title from passing. It has
been quite generally held that an oral understanding on the delivery of a deed
that it should not be recorded will not affect the absolute character of the
conveyance if free of other conditions. An agreement to deliver a deed in escrow
to the person in whose favor it is made, and who is likewise a party to it,
will not make the delivery conditional. If delivered under such an agreement,
it will be deemed an absolute delivery and a consummation of the execution of
the deed. \dots{}

