\reading{Engelhart v. Kramer}

\readingcite{570 N.W.2d 550 (S.D. 1997)}

GILBERTSON, Justice.

A \$34,800 judgment was rendered against Crystal Kay Kramer based on violation
of SDCL ch 43-4 and for failure to properly disclose a defect in the home she
sold to Karen Engelhart. The case was tried without a jury before the Second
Judicial Circuit Court. Kramer appeals the award claiming that Engelhart did
not show that Kramer failed to meet the required standard in completing the
seller's property disclosure statement.\readingfootnote{1}{Kramer also argues
that the
trial court erred in finding Kramer's actions constituted fraud and deceit. In
light of our disposition of the case on the disclosure requirement issue, the
fraud and deceit issue need not be addressed.} We affirm.

\readinghead{Facts and Procedure}

In May of 1991, Crystal Kay Kramer purchased a home in Sioux Falls, South Dakota
for \$35,000. Over the next few years Kramer made several improvements. Four
days prior to putting the home on the market, in September, 1993, Kramer
enlisted the support of friends and family and began an extensive cleaning of
the basement. There were several large cracks in the basement's cement walls
and pieces of various sizes had fallen off. They removed old sheet rock and put
up wood paneling over the basement walls. The basement project was memorialized
by Kramer with several photographs depicting the before, during and after
condition of the walls.

During this period Karen Engelhart was searching for a home commensurate with
her income level. Engelhart was a first-time home buyer and was assisted by
Dorothy Ecker, a real estate agent. Engelhart viewed Kramer's home, became
interested, and then decided to purchase it.

Kramer was represented by Shirley Ullom, a Century 21 Advantage, Inc. real
estate agent. Kramer completed the detailed ``property condition disclosure
statement'' form required by SDCL 43-4-44. Part two of the form required the
seller to disclose certain structural information. Specifically, question 2
asked ``Have you experienced water penetration in the basement ... within the
past two years?'' Kramer replied, ``Small amt of H20 penetration in NW + NE
corners [when it] rains.'' (emphasis added). In answering question 3 ``[a]re
there any cracked walls or floors?'' Kramer responded ``basement floor, some
spots in basement walls, East bedroom walls.'' Under {\S} 5, Miscellaneous
Information, Kramer was required to disclose any additional problems that were
not previously mentioned. Kramer offered, ``\textit{basement cement walls have
some crumbling}, behind paneling, basement floor cracked [and] uneven in
spots.'' (emphasis added).

The trial court found that Engelhart relied upon, among other things, Kramer's
disclosure statement with regard to the condition of the basement walls and
that Engelhart believed ``some spots'' and ``some crumbling'' to mean the
problems were minimal. Kramer allegedly offered to remove the paneling to
expose the basement walls but the trial court concluded that the offer was ``a
gambit, or a bluff ... without any real intention of performing'' and that the
typical buyer in Engelhart's position would be ``reluctant to remove paneling
from someone else's house.'' Kramer admitted taking photographs before
installing the paneling and that showing the photos to a potential purchaser
would have been easier than removing it. Kramer could not explain why she did
not offer the photos.

Engelhart purchased the property in October 1994. In March of 1995, she
discovered water seepage through the south wall of the basement. The paneling
was removed and water was discovered running through cracks in the south wall.
Also noted were several other large cracks, including a large horizontal crack
running around the basement. Engelhart hired a structural engineer, Chester
Quick (Quick) to diagnose the problem. Quick issued a report in which he found
the basement walls ``very badly cracked'' and testified that the cement had
``leeched out'' which allowed dirt and water to pass into the
basement.\readingfootnote{2}{Quick testified that the wall
was ``a mixture of sand, cement [which holds the mixture together], and usually
some rock, and over time with excess water and cracks the cement `leeches out'
of the mixture and you wind up with nothing but sand and rock.''}

Further, Quick noted that the concrete was showing ``considerable disintegration
especially at the south wall'' which was not repairable. He concluded that the
foundation had to be replaced and that ``As bad as [the walls] are cracked they
could collapse at any time.'' When asked whether the disclosure statement
adequately described the condition of the basement Quick testified that,
although accurate in part, ``some crumbling'' did not adequately describe the
damage that existed behind the paneling.

Engelhart brought suit against Kramer based upon misrepresentations made in the
disclosure statement. The trial court ruled in favor of Engelhart on failure to
comply with South Dakota's Disclosure Statutes and fraud. Kramer appeals the
\$34,800 award entered against her. \dots{}

\readinghead{Legal Analysis and Decision}

Whether Kramer failed to complete the disclosure statement in good faith as
required by SDCL Ch 43-A?

In 1993 the South Dakota legislature enacted specific requirements for
disclosures in certain real estate transfers. SDCL {\S}{\S} 43-4-38 to -44.
SDCL 43-4-38 provides:
\begin{quote}
The seller of residential real property shall furnish to a buyer a completed
copy of the disclosure statement before the buyer makes a written offer. If
after delivering the disclosure statement to the buyer or the buyer's agent and
prior to the date of closing for the property or the date of possession of the
property, whichever comes first, the seller becomes aware of any change of
material fact which would affect the disclosure statement, the seller shall
furnish a written amendment disclosing the change of material fact.
\end{quote}

SDCL 43-4-41 requires that ``The seller shall perform each act and make each
disclosure in good faith.'' SDCL 43-4-40 absolves sellers of liability for
defects in certain circumstances by providing:
\begin{quote}
Except as provided in {\S} 43-4-42, a seller is not liable for a defect or other
condition in the residential real property being transferred if the seller
\textit{truthfully completes} the disclosure statement.
\end{quote}
(Emphasis added). The disclosure form mandated by SDCL 43-4-44 establishes that
beyond the above obligations, there is no warranty passing from the seller to
the buyer:
\begin{quote}
THIS STATEMENT IS A DISCLOSURE OF THE CONDITION OF THE ABOVE DESCRIBED
PROPERTY.... IT IS NOT A WARRANTY OF ANY KIND BY THE SELLER OR ANY AGENT
REPRESENTING ANY PARTY IN THIS TRANSACTION AND IS NOT A SUBSTITUTE FOR ANY
INSPECTIONS OR WARRANTIES THE PARTIES MAY WISH TO OBTAIN.
\end{quote}
(Capitals in original).

Kramer relies on SDCL 43-4-40 and contends that even if her description of the
basement was inadequate or under Kramer's phraseology, an innocent
misrepresentation, that it was truthful nonetheless and therefore no liability
should attach. It is important to note that in SDCL 43-4-40, the terms
``truthfully'' and ``complete'' do not operate independently to the exclusion
of the other. A plain reading of the terms together evince a more exacting
standard than truth alone.

Until today, this Court has not addressed the scope of the disclosure statutes
at issue. Of central concern to our resolution is what is required by the term
``good faith,'' in the absence of a definition in SDCL 43-4-41, and whether the
disclosure of ``some crumbling'' violates that standard? We recognize that the
concept of ``good faith'' may, at times, seem as elusive as the
``reasonableness'' that is spoken of in the law of torts. However, there exists
several sources from which meaning can be found.

Statutory guidance can be found at SDCL 2-14-2(13) which states that ``good
faith'' is:
\begin{quote}
an honest intention to abstain from taking any unconscientious advantage of
another, even through the forms or technicalities of law, together with an
absence of all information or belief of facts which would render the
transaction unconscientious;
\end{quote}

Black's Law Dictionary 693 (6th ed 1990) offers the following:
\begin{quote}
Good faith is an intangible and abstract quality with no technical meaning or
statutory definition, and it encompasses, among other things, an honest belief,
the absence of malice and the absence of design to defraud or to seek an
unconscionable advantage.... In common usage this term is ordinarily used to
describe that state of mind denoting honesty of purpose, freedom of intention
to defraud, and, generally speaking, means being faithful to one's duty or
obligation.
\end{quote}

Case law decided under different contexts has provided additional meaning to the
term ``good faith'' to include ``honesty in fact,'' \textit{Garrett v.
BankWest, Inc}., 459 N.W.2d 833, 841 (S.D.1990) (contractual context; meaning
of good faith ``varies with the context and emphasizes faithfulness to an
agreed common purpose and consistency with the justified expectations of the
other party''), and an ``honest belief in the suitability of the actions
taken.'' \textit{B.W. v. Meade Co.}, 534 N.W.2d 595, 598 (S.D.1995), (in the
context of reporting and investigating child abuse). In the case now before us
the trial court properly relied upon the definition found in SDCL 2-14-2(13).

Kramer contends that since she described the condition of the basement walls as
having ``some spots'' and ``some crumbling,'' she fulfilled her duty of good
faith by truthfully completing the Disclosure Statement. Kramer argues that to
hold otherwise would, in effect, result in a strict liability standard on
sellers of real estate. We disagree.

SDCL 43-4-42 provides:
\begin{quote}
A transfer that is subject to {\S}{\S} 43-4-37 to 43-4-44, inclusive, is not
invalidated solely because a person fails to comply with {\S}{\S} 43-4-37 to
43-4-44, inclusive. However, a person \textit{who intentionally or who
negligently} violates {\S}{\S} 43-4-37 to 43-4-44, inclusive, is liable to the
buyer for the amount of the actual damages and repairs suffered by the buyer as
a result of the violation or failure. A court may also award the buyer costs
and attorney fees. Nothing in this section shall preclude or restrict any other
rights or remedies of the buyer.
\end{quote}
(Emphasis added).

Kramer relies on \textit{Amyot v. Luchini}, 932 P.2d 244 (Alaska 1997), for the
proposition that a disclosure statement can be truthful yet not ``perfect'' and
that ``innocent misrepresentations'' do not violate good faith. However, it
must be noted that Kramer's representation of the issue to this Court
incorrectly assumes that the misrepresentation of the basement walls was found
merely innocent by the trial court. To the contrary, the trial court
specifically found that the Kramer's paneling of the walls four days before
putting the house on the market was not ``solely for aesthetic purposes'' and
was completed deliberately\readingfootnote{3}{The trial court
relied on Kramer's deposition and trial testimony in that when she purchased
the house ``[t]he walls were crumbling with cracks in places,'' that the
residue she had discovered on the basement floor was ``Part of the basement
wall ... whatever makes up the wall was there in a pile'' and further that
Kramer admitted in her disclosure statement that no water ever came in on the
south wall.} in an attempt to hide their true condition. Kramer's colorful
attempt to characterize her description of the basement as an innocent
misrepresentation is inaccurate.

In 1993, Alaska enacted residential real property disclosure statement statutes
(substantially similar to that of South Dakota enacted the same year). Alaska
Stat. {\S}{\S} 34.70.010 to 34.70.090 (Michie
1996).\readingfootnote{4}{The Alaska disclosure statutes did
not define ``good faith'' but held that ``good faith'' envisioned an ``honest
and reasonable belief.'' \textit{Id}. at 247. \textit{Amyot} is distinguishable
from the present facts in that the court held an ``innocent misrepresentation''
did not violate the good faith standard. South Dakota does not attach liability
in this context unless the seller's conduct amounts to an ``intentional or
negligent'' violation the disclosure statutes. SDCL 43-4-42.\par } The
\textit{Amyot} court stated:
\begin{quote}
Prior to the enactment of [the mandatory disclosure statutes], sellers of real
property were not required to make any representations about the property.
However, sellers were strictly liable for those representations they made.
(Citation omitted.) Under the disclosure statute a seller is now required to
make representations about a wide range of the property's features and
characteristics. We conclude that the legislature intended to offset the
seller's increased disclosure responsibilities by the lower liability standard
for misrepresentations.
\end{quote}
\textit{Amyot}, 932 P.2d at 246.

We agree with the \textit{Amyot} court and hold that strict liability is not the
requisite standard under South Dakota's disclosure statutes. A plain reading of
SDCL 43-4-42 tells us that liability will not attach unless an intentional or
negligent violation occurs. The legal maxim ``\textit{expressio unius est
exlusio alterius}{}'' means ``the expression of one thing is the exclusion of
another.'' Black's Law Dictionary 581 (6th ed.1990). The maxim is a general
rule of statutory construction. Applying the general rule to SDCL 43-4-42, we
find the language ``intentionally or ... negligently'' is exclusive and negates
strict liability.

It is fair to presume that sellers know the character of the property they
convey. At present, when Kramer became aware of Engelhart's concern over the
basement she could have simply shown the pictures of its true condition. Her
failure to do so was unreasonable and amounts to negligence. SDCL 43-4-42. It
must be noted that Kramer admitted taking the photographs before installing the
paneling and that showing the photos would have been easier than removing it.
Kramer could not explain why she did not offer the photos.

We hold that with the adoption of South Dakota's detailed disclosure statutes
the doctrine of caveat emptor has been abandoned in favor of full and complete
disclosure of defects of which the seller is aware. We are not inferring, as
Kramer suggests, that a seller must possess the expertise of a structural
engineer to pass good faith muster. Nor are we suggesting that a seller will be
liable for defects of which she is unaware. Those claims are clearly disposed
of in the closing section of the mandated disclosure form of SDCL 43-4-44:
\begin{quote}
The Seller hereby certifies that the information contained herein is true and
correct to the best of the Seller's information, knowledge and belief as of the
date of the Seller's signature below.... THE SELLER AND THE BUYER MAY WISH TO
OBTAIN PROFESSIONAL ADVICE AND INSPECTIONS OF THE PROPERTY TO OBTAIN A TRUE
REPORT AS TO THE CONDITION OF THE PROPERTY AND TO PROVIDE FOR APPROPRIATE
PROVISIONS IN ANY CONTRACT OF SALE AS NEGOTIATED BETWEEN THE SELLER AND THE
BUYER WITH RESPECT TO SUCH PROFESSIONAL ADVICE AND INSPECTIONS.
\end{quote}
(Capitals in original). It is clear that, as per SDCL {\S} 43-4-41 and 43-4-44,
a seller's ``good faith'' is determined under a reasonable person standard.

Affirmed.


