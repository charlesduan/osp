\expected{harding-v-ja-laur}

\item What is the point of the distinction between forging a deed (sometimes
called \term{fraud in the factum}) and tricking someone into signing it
(\term{fraud
in the inducement})? As between the fraudster and the victim, is there a
significant difference? What about once third parties get involved?

\item Mrs. Harding signs a blank piece of paper, which Ja Laur then staples to a
deed. Forgery? What if she signs the same piece of paper \textit{after} it is
stapled to the deed? Do the policy reasons for distinguishing forgery from
fraud provide a convincing reason to treat these cases differently?

