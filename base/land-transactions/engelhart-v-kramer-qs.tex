\expected{engelhart-v-kramer}

\item In \textit{Lucero v. Van Wie}, 598 NW 2d 893 (S.D. 1999), the seller
failed to provide the statutorily required disclosure statement, but the
contract of sale contained the following clause:
\begin{quote}
The buyer acknowledges that she has examined the premises and the same are in
satisfactory condition and they accept the property in the ``as-is''
condition\ldots.
\end{quote}
This time, the South Dakota Supreme Court held that the buyer could not recover
for undisclosed defects in the property; she ``entered into an enforceable
contract and purchased the property `as is,' the result of which was to waive
disclosure requirements.'' After \textit{Lucero}, what do you expect happened
to real estate sales contracts in South Dakota? What do you expect the South
Dakota courts will do in cases where the sales contract contains an ``as-is''
clause but the buyer alleges that the seller affirmatively lied about the
condition of the property -- e.g., ``No, the roof has never leaked.''


\item In addition to the distinction between unknown defects and defects known
to the seller, some courts draw a distinction between latent and apparent
defects. Only hidden defects -- e.g., rotting support beams in the walls --
need to be disclosed, while readily visible defects, or ones that a reasonable
inspection could discover -- e.g., nonworking plumbing on the second floor --
need not. The theory, at least, is that the buyer depends on the seller to tell
her about conditions she could not reasonably discover herself. But isn't there
a connection between defects the buyer doesn't know about and defects the
seller doesn't know about, either? Cases like \textit{Engelhart} are one thing,
where the Seller literally plasters (or at least panels) over the problem. But
who should bear the loss if a previously unknown sinkhole surprises everyone by
swallowing up the back porch the day after closing?  Consider, in this regard,
a seller who doesn't know whether her home's attic walls contain asbestos
insulation, and a buyer whose offer to buy the house is contingent on drilling
into the walls to confirm that they do not contain asbestos. If you represented
the seller, would you advise your client to accept this contingency?


\item What kinds of conditions must be disclosed? A leaky roof? A leaky faucet?
The presence of lead paint on the walls? The fact that a previous inhabitant of
the home was gruesomely murdered by a family member? That the homeowner
regularly gave ``ghost tours'' on which she pretended to tourists that the
house was haunted? The fact that a registered sex offender lives on the block?
The fact that there is a municipal garbage dump half a mile away?


\item In many states, new-home builders are required to give a non-waivable
warranty of habitability that substantially parallels the warranty of
habitability required of landlords. What might account for the decision to hold
sellers of new houses to a higher standard than sellers of existing houses?
When should the statute of limitations on breach of warranty claims start
running? Should subsequent purchasers be able to sue the original builder for
breach of the warranty if the defects become apparent only after a resale?

