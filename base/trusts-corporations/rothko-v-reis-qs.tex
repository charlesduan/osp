\expected{rothko-v-reis}


\item \textsc{Dar Williams, Mark Rothko Song} (Razor \& Tie
1993):
\begin{verse}
The blue it speaks so full\newline
It's like the beauty, one can barely stand\newline
Or too much things dropped in your hand\newline
And there's a green like the peace in your heart sometimes \ldots

I met her at the funeral\newline
She said, ``I don't know what he meant to me\newline
I just know he affected me\newline
An effect not unlike his art, I believe''
\end{verse}

\item As \textit{Rothko} illustrates, trusts can arise in a variety of settings.
The executor of a will and the administrator of an estate in intestacy act as
trustees for the parties who are to receive the decedent's property. The estate
of a bankrupt firm or individual is also managed by a trustee, who acts to
maximize its value for the creditors.


There are many kinds of trusts. Private trusts have identifiable individual
beneficiaries. There are also charitable trusts, which can serve broader social
purposes and large classes of unidentified beneficiaries, and business trusts,
in which trustees manage financial assets for specific purposes. Many retirement
funds, for example, are organized as trusts with the employees who are entitled
to pensions as beneficiaries. Another common distinction is between revocable
trusts, which the settlor can terminate, and irrevocable trusts, which she
cannot. Trusts can also be \textit{inter vivos}, i.e. established by the settlor
during her lifetime, or testamentary, i.e. created in the settlor's will.



\item The basic duties of a trustee are \textit{obedience} to the instructions
given by the settlor, \textit{loyalty} to the interests of the beneficiaries
(rather than the trustee's own interests), and \textit{prudence} in managing the
trust assets appropriately. Various subsidiary duties, such as the duty to
\textit{account} for the trust assets and how they have been used, ensure that
the basic duties are carried out faithfully. Which of these duties did the
different trustees in \textit{Rothko} violate? Observe the different standards
of care required for the trustees: why is the standard of loyalty so much more
stringent than the standard of prudence? Which of these duties should the
settlor be able to waive when he or she sets up the trust? Which of them should
the beneficiaries be able to waive? To make this more concrete, do you think
that Mark Rothko wanted his executors to sell off his paintings quickly to
Marlborough? If so, should he have been allowed to specify so, and how? On the
other side, could Kate Rothko and the other heirs have given permission for the
sale, and if so, what form of notice and consent would the trustees have needed
to get?

\item In an omitted part of the opinion, the \textit{Rothko} court discussed the
proper measure of damages. It upheld the Surrogate's decision to award
appreciation damages, i.e. the value of the wrongfully sold paintings as of the
time of the Surrogate's decree. This ended up being an especially large sum
because the price of Rothko works rose rapidly after his death (and continued
rising well after \textit{Rothko}). Although sometimes justified in deterrence
terms, it is a bit of an anomalous remedy in trust law and has been criticized
by trusts scholars: holding trustees accountable for increases in value
\textit{after} they sell off trust assets is unusual. Two other damage measures
are more common. One is the familiar make-whole remedy of tort law: if the
trustees' breach of trust has reduced the value of the trust corpus, they are
liable for the difference between the trust's actual value and what it would
have been if not for the breach. This damage measure makes evident sense against
the trustee who imprudently sells a trust asset too cheaply, or who holds on to
an asset after a prudent trustee would have sold it, or who imprudently fails to
diversify a trust corpus that is concentrated in a single risky asset. But
breach of the duty of loyalty often calls for something more. Take the trustee
who withdraws \$50,000 from a trust then goes on a gambling spree in Las Vegas
and wins an additional \$100,000. Letting the trustee deposit the original
\$50,000 back in the trust and walk away with the \$100,000 in gambling winnings
would make the trust whole, but it would also leave trustees with a temptation
to gamble---literally and figuratively---with trust assets for their own
gain. In these circumstances, the usual remedy is \textit{restitution}: the
trustee must disgorge her ill-gotten gains back to the trust. Even if this gives
the beneficiaries a windfall, the trustee would be unjustly enriched were she
allowed to keep the gains. (Do you see how appreciation damages go even further
than either of these measures?) 


Observe that the restitutionary remedy involves a kind of \textit{tracing}: the
beneficiaries are regarded as having a right to the property in the trust
corpus, and they can reclaim that property even as the trustee modifies it or
changes its form. So if the trustee buys a Picasso with the trust corpus, and
the Picasso increases in value, and the trustee then sells it, she will be
required to pay back the full amount she received for the Picasso. Query: just
the trustee? Why can't Kate Rothko et al. recover her father's paintings from
the people Marlborough sold them to? What about the paintings sold to
Marlborough but not yet resold by it?



\item A trust beneficiary has equitable title to trust assets. Equitable title
is not legal title, as illustrated by spendthrift trusts. Suppose that the
fabulously wealthy parents of Rick von Slonecker, currently 28 and never
employed, decide that they want their son to enjoy a luxurious lifestyle, so
they create in their wills a trust to pay Rick \$1 million a year for life, with
the remainder to go either to his children, or if there are none, to various
charitable causes. (Side note: observe the great flexibility provided by the
trust form; equitable interests are almost always better alternatives to legal
ones in any complicated property settlement, given the notorious inflexibility
and troublesome traps of the system of estates in land.) They fear, not without
reason, that Rick will run up gambling debts and want to pay off large legal
settlements quietly. So they put a clause in the trust instrument making
abundantly clear that the monthly payments are to go directly to Rick and no one
else, and that Rick shall have no power to encumber the trust corpus. Now, in
many states, when the casino comes calling and waving its bill, it must pursue
Rick directly, even though he is penniless except for a few days immediately
after each check arrives from the trust. It would be more convenient for the
casino either to collect its debts from the trust corpus, or to obtain an order
directing the trustee to pay it instead, but the casino has no more rights to
the trust than Rick does, and \textit{Rick holds only an equitable interest in
the trust}. 


Is it fair and just for Rick's parents to help Rick escape his debts in this
way? One might think that there would be an obvious motivation for states to
protect legitimate creditors against the various asset-shielding uses and abuses
of trusts, but the trend has been in the other direction. Competition for trust
business has induced numerous jurisdictions to adopt highly settlor-friendly
trust law, such as validating spendthrift trusts like Rick's or weakening the
Rule Against Perpetuities to attract long-lived dynastic trusts with
beneficiaries spread out over many generations in a family. There are even
asset-protection trusts, in which the settlor is also the principal beneficiary;
the goal is that she can draw on the trust but her creditors cannot. These legal
concessions to settlors can benefit state economies because trustees are
entitled to compensation for managing trust assets, and many financial and legal
service providers offer professional trust management services. But these
benefits come at the expense of frustrated creditors and current generations
bound by the dead-hand control of long-gone settlors. Is this a worthwhile trade
for state legislatures to make?

\captionedgraphic{trustscorporations-img001}{Ad for Bessemer Trust, June 2014,
New York Times Magazine: ``At Bessemer Trust, we believe maintaining wealth from
generation to generation is the true art of wealth management\ldots. History is
littered with family names once associated with great wealth that are now mere
footnotes. Everything we do is designed to keep you from becoming one of
them.''}

\item There is at least one way in which courts do not pursue the legal fiction
that the trustee has legal title to trust assets through to its logical
conclusion. Suppose the \textit{trustee} (rather than the beneficiary) has a
gambling problem and racks up \$500,000 in personal gambling debts. Can the
casino collect out of the trust corpus? Strict logic would say yes; they are the
trustee's property. But Section 507 of the Uniform Trust Code flatly says no:
``Trust property is not subject to personal obligations of the trustee, even if
the trustee becomes insolvent or bankrupt.'' \textit{See also} 11 U.S.C.
\S~541(d) (exempting from a debtor's estate in bankruptcy ``[p]roperty in which
the debtor holds, as of the commencement of the case, only legal title and not
an equitable interest.'') Note that this rule cannot be justified using the
usual principle that one is not bound by prior equitable interests of which one
has no notice, since it affects even creditors who have no notice of the trust.
Only if the trustee affirmatively commits breach of trust by withdrawing trust
assets can she possibly be subjected to third-party claims. (Incidentally, what
about the \textit{settlor}'s creditors? Should they be able to reach trust
assets?)

\item
\having{eyerman-v-mercantile}{Recall \textit{Eyerman v.~Mercantile Trust Co.},
which invalidated}{We will later read \emph{Eyerman v.~Mercantile Trust Co.},
which invalidated}{In \emph{Eyerman v.~Mercantile Trust Co.}, a court
invalidated} Louise
Johnston's attempt to instruct her executor to tear down her house. Could she
have created The Louise Woodruff Johnston Testamentary Trust To Destroy My House
and left her house to it in her will instead? Probably not. Section 404 of the
Uniform Trust Code requires that a trust ``must be for the benefit of its
beneficiaries'' and the comments condemn ``frivolous or capricious'' trust terms
as violative of public policy. In \textit{M'Caig v. University of Glasgow},
[1907] Sess. Cass 231, a Scottish court invalidated a testamentary trust whose
assets were to be used ``for the purpose of erecting monuments and statutes [of]
myself, brothers, and sisters.''
