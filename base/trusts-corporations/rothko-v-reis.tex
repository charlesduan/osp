\reading{\emph{Rothko v.~Reis} (\emph{In re Estate of Rothko})}

\readingcite{372 N.E.2d 291 (N.Y. 1977)}

\opinion \textsc{Cooke}, Judge.

Mark Rothko, an abstract expressionist painter whose works through the years
gained for him an international reputation of greatness, died testate on
February 25, 1970. The principal asset of his estate consisted of 798 paintings
of tremendous value, and the dispute underlying this appeal involves the conduct
of his three executors in their disposition of these works of art. In sum, that
conduct as portrayed in the record and sketched in the opinions was manifestly
wrongful and indeed shocking.

Rothko's will was admitted to probate on April 27, 1970 and letters testamentary
were issued to Bernard J. Reis, Theodoros Stamos and Morton Levine. Hastily and
within a period of only about three weeks and by virtue of two contracts each
dated May 21, 1970, the executors dealt with all 798 paintings.

By a contract of sale, the estate executors agreed to sell to Marlborough A.G.,
a Liechtenstein corporation (hereinafter MAG), 100 Rothko paintings as listed
for \$1,800,000, \$200,000 to be paid on execution of the agreement and the
balance of \$1,600,000 in 12 equal interest-free installments over a 12-year
period. Under the second agreement, the executors consigned to Marlborough
Gallery, Inc., a domestic corporation (hereinafter MNY), ``approximately 700
paintings listed on a Schedule to be prepared'', the consignee to be responsible
for costs covering items such as insurance, storage restoration and promotion.
By its provisos, MNY could sell up to 35 paintings a year from each of two
groups, pre-1947 and post-1947, for 12 years at the best price obtainable but
not less than the appraised estate value, and it would receive a 50\% commission
on each painting sold, except for a commission of 40\% on those sold to or
through other dealers.

Petitioner Kate Rothko, decedent's daughter and a person entitled to share in
his estate by virtue of an election under EPTL 5-3.3, instituted this proceeding
to remove the executors, to enjoin MNY and MAG from disposing of the paintings,
to rescind the aforesaid agreements between the executors and said corporations,
for a return of the paintings still in possession of those corporations, and for
damages. She was joined by the guardian of her brother Christopher Rothko,
likewise interested in the estate, who answered by adopting the allegations of
his sister's petition and by demanding the same relief. The Attorney-General of
the State, as the representative of the ultimate beneficiaries of the Mark
Rothko Foundation, Inc., a charitable corporation and the residuary legatee
under decedent's will, joined in requesting relief substantially similar to that
prayed for by petitioner.\ldots

Following a nonjury trial covering 89 days and in a thorough opinion, the
Surrogate found: that Reis was a director, secretary and treasurer of MNY, the
consignee art gallery, in addition to being a coexecutor of the estate; that the
testator had a 1969 inter vivos contract with MNY to sell Rothko's work at a
commission of only 10\% and whether that agreement survived testator's death was
a problem that a fiduciary in a dual position could not have impartially faced;
that Reis was in a position of serious conflict of interest with respect to the
contracts of May 21, 1970 and that his dual role and planned purpose benefited
the Marlborough interests to the detriment of the estate; that it was to the
advantage of coexecutor Stamos as a ``not-too-successful artist, financially'',
to curry favor with Marlborough and that the contract made by him with MNY
within months after signing the estate contracts placed him in a position where
his personal interests conflicted with those of the estate, especially leading
to lax contract enforcement efforts by Stamos; that Stamos acted negligently and
improvidently in view of his own knowledge of the conflict of interest of Reis;
that the third coexecutor, Levine, while not acting in self-interest or with bad
faith, nonetheless failed to exercise ordinary prudence in the performance of
his assumed fiduciary obligations since he was aware of Reis' divided loyalty,
believed that Stamos was also seeking personal advantage, possessed personal
opinions as to the value of the paintings and yet followed the leadership of his
coexecutors without investigation of essential facts or consultation with
competent and disinterested appraisers, and that the business transactions of
the two Marlborough corporations were admittedly controlled and directed by
Francis K. Lloyd. It was concluded that the acts and failures of the three
executors were clearly improper to such a substantial extent as to mandate their
removal under SCPA 711 as estate fiduciaries. The Surrogate also found that MNY,
MAG and Lloyd were guilty of contempt in shipping, disposing of and selling 57
paintings in violation of the temporary restraining order dated June 26, 1972
and of the injunction dated September 26, 1972; that the contracts for sale and
consignment of paintings between the executors and MNY and MAG provided
inadequate value to the estate, amounting to a lack of mutuality and fairness
resulting from conflicts on the part of Reis and Stamos and improvidence on the
part of all executors; that said contracts were voidable and were set aside by
reason of violation of the duty of loyalty and improvidence of the executors,
knowingly participated in and induced by MNY and MAG; that the fact that these
agreements were voidable did not revive the 1969 inter vivos agreements since
the parties by their conduct evinced an intent to abandon and abrogate these
compacts. The Surrogate held that the present value at the time of trial of the
paintings sold is the proper measure of damages as to MNY, MAG, Lloyd, Reis and
Stamos. He imposed a civil fine of \$3,332,000 upon MNY, MAG and Lloyd, same
being the appreciated value at the time of trial of the 57 paintings sold in
violation of the temporary restraining order and injunction. It was held that
Levine was liable for \$6,464,880 in damages, as he was not in a dual position
acting for his own interest and was thus liable only for the actual value of
paintings sold MNY and MAG as of the dates of sale, and that Reis, Stamos, MNY
and MAG, apart from being jointly and severally liable for the same damages as
Levine for negligence, were liable for the greater sum of \$9,252,000 ``as
appreciation damages less amounts previously paid to the estate with regard to
sales of paintings.'' The cross petition of the Attorney-General to reopen the
record for submission of newly discovered documentary evidence was denied. The
liabilities were held to be congruent so that payment of the highest sum would
satisfy all lesser liabilities including the civil fines and the liabilities for
damages were to be reduced by payment of the fine levied or by return of any of
the 57 paintings disposed of, the new fiduciary to have the option in the first
instance to specify which paintings the fiduciary would accept. [The Appellate
Division affirmed.]

In seeking a reversal, it is urged that an improper legal standard was applied
in voiding the estate contracts of May, 1970, that the ``no further inquiry''
rule applies only to self-dealing and that in case of a conflict of interest,
absent self-dealing, a challenged transaction must be shown to be unfair. The
subject of fairness of the contracts is intertwined with the issue of whether
Reis and Stamos were guilty of conflicts of interest.\footnote{In New York, an
executor, as such, takes a qualified legal title to all personalty specifically
bequeathed and an unqualified legal title to that not so bequeathed; he holds
not in his own right but as a trustee for the benefit of creditors, those
entitled to receive under the will and, if all is not bequeathed, those entitled
to distribution under the EPTL.} [Austin W. Scott Jr., Scott on Trusts] is
quoted to the effect that ``(a) trustee does not necessarily incur liability
merely because he has an individual interest in the transaction\ldots .''

These contentions should be rejected. First, a review of the opinions of the
Surrogate and the Appellate Division manifests that they did not rely solely on
a ``no further inquiry rule'', and secondly, there is more than an adequate
basis to conclude that the agreements between the Marlborough corporations and
the estate were neither fair nor in the best interests of the estate. This is
demonstrated, for example, by the comments of the Surrogate concerning the
commissions on the consignment of the 698 paintings and those of the Appellate
Division concerning the sale of the 100 paintings. The opinions under review
demonstrate that neither the Surrogate nor the Appellate Division set aside the
contracts by merely applying the no further inquiry rule without regard to
fairness. Rather they determined, quite properly indeed, that these agreements
were neither fair nor in the best interests of the estate.

To be sure, the assertions that there were no conflicts of interest on the part
of Reis or Stamos indulge in sheer fantasy. Besides being a director and officer
of MNY, for which there was financial remuneration, however slight, Reis, as
noted by the Surrogate, had different inducements to favor the Marlborough
interests, including his own aggrandizement of status and financial advantage
through sales of almost one million dollars for items from his own and his
family's extensive private art collection by the Marlborough interests.
Similarly, Stamos benefited as an artist under contract with Marlborough and,
interestingly, Marlborough purchased a Stamos painting from a third party for
\$4,000 during the week in May, 1970 when the estate contract negotiations were
pending. The conflicts are manifest. Further, as noted in Bogert, Trusts and
Trustees (2d ed.), ``The duty of loyalty imposed on the fiduciary prevents him
from accepting employment from a third party who is entering into a business
transaction with the trust'' (s 543, subd. (S), p. 573). ``While he (a trustee)
is administering the trust he must refrain from placing himself in a position
where his personal interest or that of a third person does or may conflict with
the interest of the beneficiaries'' (Bogert, Trusts (Hornbook Series 5th ed.),
p. 343). Here, Reis was employed and Stamos benefited in a manner contemplated
by Bogert. In short, one must strain the law rather than follow it to reach the
result suggested on behalf of Reis and Stamos.

Levine contends that, having acted prudently and upon the advice of counsel, a
complete defense was established. Suffice it to say, an executor who knows that
his coexecutor is committing breaches of trust and not only fails to exert
efforts directed towards prevention but accedes to them is legally accountable
even though he was acting on the advice of counsel. When confronted with the
question of whether to enter into the Marlborough contracts, Levine was acting
in a business capacity, not a legal one, in which he was required as an executor
primarily to employ such diligence and prudence to the care and management of
the estate assets and affairs as would prudent persons of discretion and
intelligence, accented by ``(n)ot honesty alone, but the punctilio of an honor
the most sensitive'' (\emph{Meinhard v. Salmon}, 164 N.E. 545, 546 (N.Y. 1928)).
Alleged good faith on the part of a fiduciary forgetful of his duty is not
enough. He could not close his eyes, remain passive or move with unconcern in
the face of the obvious loss to be visited upon the estate by participation in
those business arrangements and then shelter himself behind the claimed counsel
of an attorney.

Further, there is no merit to the argument that MNY and MAG lacked notice of the
breach of trust. The record amply supports the determination that they are
chargeable with notice of the executors' breach of duty.
