\expected{levandusky-v-one-fifth}
\item \textit{Levandusky} illustrates another variety of corporation: the
residential cooperative. Each tenant in a co-op owns shares in the corporation;
the corporation owns the building, and leases an individual dwelling unit back
to the tenant. (Contrast a condominium, in which each owner of an individual
unit also owns shares in a corporation that owns and manages the common areas.)
Like a business corporation, the residential cooperative has a board elected by
its shareholders.

\item \textit{Levandusky} illustrates the standard for judicial review of
corporate decisions by the board: the business judgment rule. Compare a
corporate board's duties under the business judgment rule with a trustee's
duties under the standards applied in \textit{Rothko}. Which are more demanding?
What accounts for the difference?

\item In an earlier era, corporations were charted by state legislatures for
specific purposes: e.g., building a bridge. Today, incorporation is generally a
matter of statutory right and a business corporation can be created for any
lawful purpose. What are the purposes of business corporation---or put another
way, to whom is it responsible and for what? Is the board under a legal or moral
duty to maximize profits to the shareholders above all else? \textit{Compare,
e.g.}, Milton Freedman, \textit{The Social Responsibility of Business is to
Increase its Profits}, \textsc{N.Y. Times}, Sept. 30, 1970, at SM17 (yes),
\textit{with, e.g.}, \textsc{Lynn Stout}, \textsc{The Myth of Shareholder Value,
How Putting Shareholders First Harms Investors, Corporations, and the Public}
(2012) (no). If so, how much of a duty is it, given the extreme latitude of the
business judgment rule? Some courts, while strongly deferential to board
decisions most of the time, apply a higher level of scrutiny in cases involving
corporate mergers, when the only question the board faces is how much money
shareholders will receive for their shares from someone trying to buy up the
corporation. \textit{See, e.g.}, \textit{Revlon, Inc. v. MacAndrews \& Forbes
Holdings}, 506 A.2d 173 (Del. 1986). Bring these ideas back over to the co-op
context of \textit{Levandusky}: is profit maximization the goal of a residential
cooperative? If so, what is? The court refers to the ``benefit of the residents
collectively.'' What on earth does that mean, and is the business judgment rule
in this context just a way of saying that the courts are not in a position to
ascertain what the diverse and antagonistic residents of a co-op collectively
want and need?

\item Shares in public corporations are paradigmatically freely alienable, but
shares in close corporations often are not. If Lucky, Dusty, and Ned found a
bakery together, they may not want to give each other an unlimited right to sell
out to the El Guapo Bread Company. Co-ops, especially in New York City where
\textit{Levandusky} arose, are notorious for the vetting processes they impose
on new residents who wish to move into the building by buying the shares of a
current resident. Extensive personal interviews and stringent financial
questionnaires are common. So are rejections that push the boundaries of
rationality and legality. At the board interview in 1999 for an \$8 million
apartment sale at 320 Central Park West from Barbra Streisand to Mariah Carey,
one board member asked Carey whether ``Mr. Biggie'' would be visiting the
building. ``Mr. Biggie, he be dead,'' was Carey's reply. The board voted to
reject her. Whether it was because they feared she would sing loudly, because
she wore a navel-revealing top and brought three African-American bodyguards to
the interview, because the board member who asked the question had confused Puff
Daddy with the Notorius B.I.G., or for some other reason, is not recorded.
\textit{See also} \textsc{Steven Gaines, The Sky's the Limit: Passion and
Property in Manhattan} 55 (2005) (quoting a real estate broker as saying,
``There is one Jewish person on the board, and that Jewish person is the one who
vetoes all the other Jewish people.''). If courts reviewing board decisions
apply the business-judgment rule, how effective will they be at preventing
violations of housing-discrimination law?

