\reading{Levandusky v. One Fifth Ave.}

\readingcite{75 N.Y.2d 530 (1990)}

\opinion \textsc{Kaye}, J. 

This appeal by a residential cooperative corporation concerning apartment
renovations by one of its proprietary lessees, factually centers on a two-inch
steam riser and three air conditioners, but fundamentally presents the legal
question of what standard of review should apply when a board of directors of a
cooperative corporation seeks to enforce a matter of building policy against a
tenant-shareholder. We conclude that the business judgment rule furnishes the
correct standard of review. 

In the main, the parties agree that the operative events transpired as follows.
In 1987, respondent (Ronald Levandusky) decided to enlarge the kitchen area of
his apartment at One Fifth Avenue in New York City. According to Levandusky,
some time after reaching that decision, and while he was president of the
cooperative's board of directors, he told Elliot Glass, the architect retained
by the corporation, that he intended to realign or ``jog'' a steam riser in the
kitchen area, and Glass orally approved the alteration. According to Glass,
however, the conversation was a general one; Levandusky never specifically told
him that he intended to move any particular pipe, and Glass never gave him
approval to do so. In any event, Levandusky's proprietary lease provided that no
``alteration of or addition to the water, gas or steam risers or pipes'' could
be made without appellant's prior written consent. 

Levandusky had his architect prepare plans for the renovation, which were
approved by Glass and submitted for approval to the board of directors. Although
the plans show details of a number of other proposed structural modifications,
including changes in plumbing risers, no change in the steam riser is shown or
discussed anywhere in the plans. 

The board approved Levandusky's plans at a meeting held March 14, 1988, and the
next day he executed an ``Alteration Agreement'' with appellant, which
incorporated ``Renovation Guidelines'' that had originally been drafted, in
large part, by Levandusky himself. These guidelines, like the proprietary lease,
specified that advance written approval was required for any renovation
affecting the building's heating system. Board consideration of the
plans---appropriately detailed to indicate all structural changes---was to
follow their
submission to the corporation's architect, and the board reserved the power to
disapprove any plans, even those that had received the architect's approval. 

In late spring 1988, the building's managing agent learned from Levandusky that
he intended to move the steam riser in his apartment, and so informed the board.
Both Levandusky and the board contacted John Flynn, an engineer who had served
as consulting agent for the board. In a letter and in a subsequent presentation
at a June 13 board meeting, Flynn opined that relocating steam risers was
technically feasible and, if carefully done, would not necessarily cause any
problem. However, he also advised that any change in an established old piping
system risked causing difficulties (``gremlins''). In Flynn's view, such
alterations were to be avoided whenever possible. 

At the June 13 meeting, which Levandusky attended, the board enacted a
resolution to ``reaffirm the policy---no relocation of risers.'' At a June 23
meeting, the board voted to deny Levandusky a variance to move his riser, and to
modify its previous approval of his renovation plans, conditioning approval upon
an acceptable redesign of the kitchen area. 

Levandusky nonetheless hired a contractor, who severed and jogged the kitchen
steam riser. In August 1988, when the board learned of this, it issued a ``stop
work'' order, pursuant to the ``Renovation Guidelines.'' Levandusky then
commenced this article 78 proceeding, seeking to have the stop work order set
aside. The corporation cross-petitioned for an order compelling Levandusky to
return the riser to its original position.\ldots 

As cooperative and condominium home ownership has grown increasingly popular,
courts confronting disputes between tenant-owners and governing boards have
fashioned a variety of rules for adjudicating such claims. In the process,
several salient characteristics of the governing board homeowner relationship
have been identified as relevant to the judicial inquiry. 

As courts and commentators have noted, the cooperative or condominium
association is a quasi-government---``a little democratic sub society of
necessity.''\ldots The proprietary lessees or condominium owners consent to be
governed, in certain respects, by the decisions of a board. Like a municipal
government, such governing boards are responsible for running the day-to-day
affairs of the cooperative and to that end, often have broad powers in areas
that range from financial decisionmaking to promulgating regulations regarding
pets and parking spaces. Authority to approve or disapprove structural
alterations, as in this case, is commonly given to the governing board. 

Through the exercise of this authority, to which would-be apartment owners must
generally acquiesce, a governing board may significantly restrict the bundle of
rights a property owner normally enjoys. Moreover, as with any authority to
govern, the broad powers of a cooperative board hold potential for abuse through
arbitrary and malicious decision-making, favoritism, discrimination and the
like. 

On the other hand, agreement to submit to the decisionmaking authority of a
cooperative board is voluntary in a sense that submission to government
authority is not; there is always the freedom not to purchase the apartment. The
stability offered by community control, through a board, has its own economic
and social benefits, and purchase of a cooperative apartment represents a
voluntary choice to cede certain of the privileges of single ownership to a
governing body, often made up of fellow tenants who volunteer their time,
without compensation. The board, in return, takes on the burden of managing the
property for the benefit of the proprietary lessees. As one court observed:
``Every man may justly consider his home his castle and himself as the king
thereof; nonetheless his sovereign fiat to use his property as he pleases must
yield, at least in degree, where ownership is in common or cooperation with
others. The benefits of condominium living and ownership demand no less.'' 

It is apparent, then, that a standard for judicial review of the actions of a
cooperative or condominium governing board must be sensitive to a variety of
concerns---sometimes competing concerns. Even when the governing board acts
within the scope of its authority, some check on its potential powers to
regulate residents' conduct, life-style and property rights is necessary to
protect individual residents from abusive exercise, notwithstanding that the
residents have, to an extent, consented to be regulated and even selected their
representatives. At the same time, the chosen standard of review should not
undermine the purposes for which the residential community and its governing
structure were formed: protection of the interest of the entire community of
residents in an environment managed by the board for the common benefit. 

We conclude that these goals are best served by a standard of review that is
analogous to the business judgment rule applied by courts to determine
challenges to decisions made by corporate directors.\ldots 

Developed in the context of commercial enterprises, the business judgment rule
prohibits judicial inquiry into actions of corporate directors ``taken in good
faith and in the exercise of honest judgment in the lawful and legitimate
furtherance of corporate purposes.'' (\textit{Auerbach v Bennett}, 47 N.Y.2d
619, 629.) So long as the corporation's directors have not breached their
fiduciary obligation to the corporation, ``the exercise of [their powers] for
the common and general interests of the corporation may not be questioned,
although the results show that what they did was unwise or inexpedient.''
(\textit{Pollitz v Wabash R. R. Co.}, 207 N.Y. 113, 124.) 

Application of a similar doctrine is appropriate because a cooperative
corporation is---in fact and function---a corporation, acting through the
management of its board of directors, and subject to the Business Corporation
Law. There is no cause to create a special new category in law for corporate
actions by coop boards. 

We emphasize that reference to the business judgment rule is for the purpose of
analogy only. Clearly, in light of the doctrine's origins in the quite different
world of commerce, the fiduciary principles identified in the existing case
law---primarily emphasizing avoidance of self-dealing and financial
self-aggrandizement---will of necessity be adapted over time in order to apply
to directors of not-for-profit homeowners' cooperative corporations. For present
purposes, we need not, nor should we determine the entire range of the fiduciary
obligations of a cooperative board, other than to note that the board owes its
duty of loyalty to the cooperative---that is, it must act for the benefit of
the residents collectively. So long as the board acts for the purposes of the
cooperative, within the scope of its authority and in good faith, courts will
not substitute their judgment for the board's. Stated somewhat differently,
unless a resident challenging the board's action is able to demonstrate a breach
of this duty, judicial review is not available. 

In reaching this conclusion, we reject the test seemingly applied by the
Appellate Division majority and explicitly applied by Supreme Court in its
initial decision. That inquiry was directed at the reasonableness of the board's
decision; having itself found that relocation of the riser posed no ``dangerous
aspect'' to the building, the Appellate Division concluded that the renovation
should remain. Like the business judgment rule, this reasonableness
standard---originating in the quite different world of governmental agency
decision-making---has found favor with courts reviewing board decisions. 

As applied in condominium and cooperative cases, review of a board's decision
under a reasonableness standard has much in common with the rule we adopt today.
A primary focus of the inquiry is whether board action is in furtherance of a
legitimate purpose of the cooperative or condominium, in which case it will
generally be upheld. The difference between the reasonableness test and the rule
we adopt is twofold. First---unlike the business judgment rule, which places on
the owner seeking review the burden to demonstrate a breach of the board's
fiduciary duty---reasonableness review requires the board to demonstrate that
its decision was reasonable. Second, although in practice a certain amount of
deference appears to be accorded to board decisions, reasonableness review
permits---indeed, in theory requires---the court itself to evaluate the merits
or wisdom of the board's decision, just as the Appellate Division did in the
present case. 

The more limited judicial review embodied in the business judgment rule is
preferable. In the context of the decisions of a for-profit corporation,
``courts are ill equipped and infrequently called on to evaluate what are and
must be essentially business judgments\ldots by definition the responsibility
for business judgments must rest with the corporate directors; their individual
capabilities and experience peculiarly qualify them for the discharge of
that responsibility.'' (\textit{Auerbach v.~Bennett}, 47 N.Y.2d,
\textit{supra}, at 630--631.) Even if decisions of a cooperative board do not
generally involve expertise beyond the usual ken of the judiciary, at the least
board members will possess experience of the peculiar needs of their building
and its residents not shared by the court. 

Several related concerns persuade us that such a rule should apply here. As this
case exemplifies, board decisions concerning what residents may or may not do
with their living space may be highly charged and emotional. A cooperative or
condominium is by nature a myriad of often competing views regarding personal
living space, and decisions taken to benefit the collective interest may be
unpalatable to one resident or another, creating the prospect that board
decisions will be subjected to undue court involvement and judicial
second-guessing. Allowing an owner who is simply dissatisfied with particular
board action a second opportunity to reopen the matter completely before a
court, which---generally without knowing the property---may or may not agree
with the reasonableness of the board's determination, threatens the stability of
the common living arrangement. 

Moreover, the prospect that each board decision may be subjected to full
judicial review hampers the effectiveness of the board's managing authority. The
business judgment rule protects the board's business decisions and managerial
authority from indiscriminate attack. At the same time, it permits review of
improper decisions, as when the challenger demonstrates that the board's action
has no legitimate relationship to the welfare of the cooperative, deliberately
singles out individuals for harmful treatment, is taken without notice or
consideration of the relevant facts, or is beyond the scope of the board's
authority. 

Levandusky failed to meet this burden, and Supreme Court properly dismissed his
petition. His argument that having once granted its approval, the board was
powerless to rescind its decision after he had spent considerable sums on the
renovations is without merit. There is no dispute that Levandusky failed to
comply with the provisions of the ``Alteration Agreement'' or ``Renovation
Guidelines'' designed to give the board explicit written notice before it
approved a change in the building's heating system. Once made aware of
Levandusky's intent, the board promptly consulted its engineer, and notified
Levandusky that it would not depart from a policy of refusing to permit the
movement of pipes. That he then went ahead and moved the pipe hardly allows him
to claim reliance on the board's initial approval of his plans. Indeed,
recognition of such an argument would frustrate any systematic effort to enforce
uniform policies. 

Levandusky's additional allegations that the board's decision was motivated by
the personal animosity of another board member toward him, and that the board
had in fact permitted other residents to jog their steam risers, are wholly
conclusory. The board submitted evidence---unrefuted by Levandusky---that it was
acting pursuant to the advice of its engineer, and that it had not previously
approved such jogging. Finally, the fact that allowing Levandusky an exception
to the policy might not have resulted in harm to the building does not require
that the exception be allowed. Under the rule we articulate today, we decline to
review the merits of the board's determination that it was preferable to adhere
to a uniform policy regarding the building's piping system.\ldots 

