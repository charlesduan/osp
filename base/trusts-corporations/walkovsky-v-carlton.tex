\reading{Walkovszky v. Carlton}

\readingcite{18 N.Y.2d 414 (1966)}

\opinion \textsc{Fuld}, Justice: 

This case involves what appears to be a rather common practice in the taxicab
industry of vesting the ownership of a taxi fleet in many corporations, each
owning only one or two cabs. 

The complaint alleges that the plaintiff was severely injured four years ago in
New York City when he was run down by a taxicab owned by the defendant Seon Cab
Corporation and negligently operated at the time by the defendant Marchese. The
individual defendant, Carlton, is claimed to be a stockholder of 10
corporations, including Seon, each of which has but two cabs registered in its
name, and it is implied that only the minimum automobile liability insurance
required by law (in the amount of \$10,000) is carried on any one cab. Although
seemingly independent of one another, these corporations are alleged to be
``operated\ldots as a single entity, unit and enterprise'' with regard to
financing, supplies, repairs, employees and garaging, and all are named as
defendants. The plaintiff asserts that he is also entitled to hold their
stockholders personally liable for the damages sought because the multiple
corporate structure constitutes an unlawful attempt ``to defraud members of the
general public'' who might be injured by the cabs.\ldots 

The law permits the incorporation of a business for the very purpose of enabling
its proprietors to escape personal liability but, manifestly, the privilege is
not without its limits. Broadly speaking, the courts will disregard the
corporate form, or, to use accepted terminology, ``pierce the corporate veil'',
whenever necessary ``to prevent fraud or to achieve equity''. In determining
whether liability should be extended to reach assets beyond those belonging to
the corporation, we are guided, as Judge Cardozo noted, by ``general rules of
agency''. In other words, whenever anyone uses control of the corporation to
further his own rather than the corporation's business, he will be liable for
the corporation's acts ``upon the principle of \textit{respondeat superior}
applicable even where the agent is a natural person''. Such liability, moreover,
extends not only to the corporation's commercial dealings but to its negligent
acts as well. 

In the \textit{Mangan} case, the plaintiff was injured as a result of the
negligent operation of a cab owned and operated by one of four corporations
affiliated with the defendant Terminal. Although the defendant was not a
stockholder of any of the operating companies, both the defendant and the
operating companies were owned, for the most part, by the same parties. The
defendant's name (Terminal) was conspicuously displayed on the sides of all of
the taxis used in the enterprise and, in point of fact, the defendant actually
serviced, inspected, repaired and dispatched them. These facts were deemed to
provide sufficient cause for piercing the corporate veil of the operating
company---the nominal owner of the cab which injured the plaintiff---and
holding the defendant liable. The operating companies were simply
instrumentalities for carrying on the business of the defendant without imposing
upon it financial and other liabilities incident to the actual ownership and
operation of the cabs.\ldots 

The individual defendant is charged with having ``organized, managed, dominated
and controlled'' a fragmented corporate entity but there are no allegations that
he was conducting business in his individual capacity. Had the taxicab fleet
been owned by a single corporation, it would be readily apparent that the
plaintiff would face formidable barriers in attempting to establish personal
liability on the part of the corporation's stockholders. The fact that the fleet
ownership has been deliberately split up among many corporations does not ease
the plaintiff's burden in that respect. The corporate form may not be
disregarded merely because the assets of the corporation, together with the
mandatory insurance coverage of the vehicle which struck the plaintiff, are
insufficient to assure him the recovery sought. If Carlton were to be held
individually liable on those facts alone, the decision would apply equally to
the thousands of cabs which are owned by their individual drivers who conduct
their businesses through corporations organized pursuant to section 401 of the
Business Corporation Law and carry the minimum insurance required by subdivision
1 (par. [a]) of section 370 of the Vehicle and Traffic Law. These taxi
owner-operators are entitled to form such corporations, and we agree with the
court at Special Term that, if the insurance coverage required by statute ``is
inadequate for the protection of the public, the remedy lies not with the courts
but with the Legislature.'' It may very well be sound policy to require that
certain corporations must take out liability insurance which will afford
adequate compensation to their potential tort victims. However, the
responsibility for imposing conditions on the privilege of incorporation has
been committed by the Constitution to the Legislature and it may not be fairly
implied, from any statute, that the Legislature intended, without the slightest
discussion or debate, to require of taxi corporations that they carry automobile
liability insurance over and above that mandated by the Vehicle and Traffic Law.

This is not to say that it is impossible for the plaintiff to state a valid
cause of action against the defendant Carlton. However, the simple fact is that
the plaintiff has just not done so here. While the complaint alleges that the
separate corporations were undercapitalized and that their assets have been
intermingled, it is barren of any ``sufficiently particular[ized] statements''
that the defendant Carlton and his associates are actually doing business in
their individual capacities, shuttling their personal funds in and out of the
corporations ``without regard to formality and to suit their immediate
convenience.'' Such a ``perversion of the privilege to do business in a
corporate form'' would justify imposing personal liability on the individual
stockholders. Nothing of the sort has in fact been charged, and it cannot
reasonably or logically be inferred from the happenstance that the business of
Seon Cab Corporation may actually be carried on by a larger corporate entity
composed of many corporations which, under general principles of agency, would
be liable to each other's creditors in contract and in tort. 

In point of fact, the principle relied upon in the complaint to sustain the
imposition of personal liability is not agency but fraud. Such a cause of action
cannot withstand analysis. If it is not fraudulent for the owner-operator of a
single cab corporation to take out only the minimum required liability
insurance, the enterprise does not become either illicit or fraudulent merely
because it consists of many such corporations. The plaintiff's injuries are the
same regardless of whether the cab which strikes him is owned by a single
corporation or part of a fleet with ownership fragmented among many
corporations. Whatever rights he may be able to assert against parties other
than the registered owner of the vehicle come into being not because he has been
defrauded but because, under the principle of respondeat superior, he is
entitled to hold the whole enterprise responsible for the acts of its agents. 

In sum, then, the complaint falls short of adequately stating a cause of action
against the defendant Carlton in his individual capacity.\ldots 

