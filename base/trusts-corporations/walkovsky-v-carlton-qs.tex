\expected{walkovsky-v-carlton}
\item Corporate structure sharply distinguishes between two kinds of property.
\term[corporate assets]{Corporate assets}, like the cabs in \textit{Walkovszky},
belong to the corporation. \term[corporate shares]{Corporate shares} belong to
the corporation's shareholders;
they give the holders rights to share in the corporation's profits and to
control the corporation's activities. So the shareholders own the corporation,
which owns its assets---but the shareholders do not directly own or control the
assets. Instead, in a business corporation (there are also nonprofit
corporations, municipal corporations, and more), the shareholders elect a
\term{board of directors}, which is responsible for operating the company. The
board typically hires corporate officers and delegates day-to-day operations to
them, but in theory it can take the reins when needed---and must do so for major
corporate activities like mergers. If shareholders do not like how the board of
directors are running the corporation, their two options are to sell their
shares (if they can) or to elect new directors (if they can). Understanding this
structure is crucial for understanding corporate law and the treatment of
corporate property.

\item What purpose can possibly be served by allowing Carlton to escape
liability for the injuries tortiously caused by the taxicab companies he owns
and controls? Isn't limited liability an open invitation to pillage and lay
waste? Should there perhaps be a distinction between (typically voluntary)
contract creditors and (typically involuntary) tort creditors? Or between
\term[closely held corporation]{closely held} corporations with one or a few shareholders and
\term[public corporation]{public} corporations whose shares are traded on major
stock markets and held by thousands or millions of shareholders?

\item The reverse of limited liability is \term{asset partitioning}: just as
Seon's
creditors can't reach outside the corporation to Carlton's personal assets,
Carlton's personal creditors can't reach inside the corporation to Seon's
corporate assets. Is there anything his creditors can do to get at the wealth
sitting inside Seon and its corporate siblings?

\item In the aftermath of \textit{Walkovsky}, the New York legislature increased
the required insurance coverage for taxicab operators, but it left alone the
state's law of veil-piercing. Does this suggest that the case was rightly or
wrongly decided?

\item How does \textit{Walkovsky} encourage taxicab companies to structure their
businesses? This is a recurring problem in corporate and commercial law (which
will become apparent in the mortgage crisis section): parties will arrange a
corporate or transactional form to gain specific advantages while isolating
themselves from the associated legal risks. In \term{securitization}, for
example, a
group of assets is pushed into a separate legal entity, isolating them from
claims against their corporate parent, and vice-versa. If the new entity
defaults on its obligations, the company that loaded it up with toxic junk will
avoid liability---or such is the plan, anyway.

