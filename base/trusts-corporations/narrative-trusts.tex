\reading{Note on Trusts}

The origin of the trust lies in medieval tax estate planning and tax evasion.
(Arguably, nothing has changed in the last six hundred years.) Imagine Osbert, a
minor lord in the 15th century, who holds Blackacre as a tenant of Leonard, a
slightly less minor lord. Osbert is getting on in years and has started to worry
about the future of his family. His elder son, Aylwin, is not showing promising
signs of maturity, and Osbert has come to think that Aylwin may be better suited
to religious orders than the duties of managing a great estate. But Osbert's
younger son Bartholomew appears to be a fine young gentleman: athletic, patient,
and wise in the ways of men. Osbert would like to provide for Aylwin, but would
prefer to have Blackacre go to Bartholomew. Osbert's problem is that the
available conveyancing devices don't work for him. If he does nothing, then
Blackacre goes to Alywin at Osbert's death under the rule of primogeniture in
effect in England at the time, according to which the eldest son receives any
land his father owned at his death (was ``seised of,'' in contemporary
terminology). A will leaving Blackacre to Bartholomew doesn't work because land
could not be devised by will until the Statute of Wills in 1540. And Osbert
doesn't want to convey Blackacre (or a future interest in Blackacre) to
Bartholomew now, because Bartholomew might die before him, or Aylwin might get
his act together, or something else could come along to force a change in plan.

The solution hit on by contemporary lawyers was the ``use.'' Osbert conveys
Blackacre to his friend Theobald ``to the use of Osbert and his heirs.'' Then he
writes a letter to Theobald, instructing Theobald to convey Blackacre to
Bartholomew at Osbert's death. This works. When Osbert dies, Theobald owns
Blackacre, so primogeniture never kicks in. Then Theobald conveys to Bartholomew
while they are both alive, so again the conveyance is perfectly good. What's
more, Osbert can change his instructions to Theobald at any time by writing a
new letter. And as an added bonus, because the land never passes by intestacy,
the ``feudal incidents''-- effectively taxes payable to Leonard when a new
tenant inherits---never become due. Uses became highly popular for solving
numerous similar problems created by the inflexibility of the medieval system of
interests in land.

But there was a fly in the ointment. As far as the law courts could see---or
rather, as far as they were willing to look---the ``to the use of'' language was
a superfluous, meaningless, and ineffective addition to an otherwise valid
conveyance. On their view of the situation, Theobald owns Blackacre in fee
simple once Osbert conveys to him. Osbert's subsequent letter is a worthless
piece of paper; much as if you wrote to Bill Gates telling him to convey to you
some lakefront property in Washington. So if Theobald turned out to be
untrustworthy and held on to Blackacre for himself or conveyed it to Aylwin
contrary to Osbert's instructions, Osbert's plan would come to ruin. In such
cases, Osbert and Bartholomew could obtain relief from the Chancellor, who would
hold that Theobald was under a duty in equity and good conscience to follow
Osbert's instructions.

The use thus created what we would today call an ``equitable interest'' in land.
Theobald remained the \textit{legal} owner of Blackacre while he held it to the
use of Osbert and his heirs, but Osbert was the \textit{equitable} owner, since
he could enforce his claims and instructions in a court of equity. Over time a
variety of similar situations, in which Chancery would enforce interests in land
legally owned by another, gave rise to a reasonably coherent body of equitable
jurisdiction, equitable doctrine, and equitable interests in property.

The use is long gone, along with the medieval doctrines that necessitated it,
but the modern \term{trust} shares its essential characteristics. A trust
requires
three people and one thing. The people are the \term{settlor}, who creates the
trust; the \term{trustee}, who holds legal title to the trust property and is
responsible for following the settlor's instructions, and the
\term{beneficiary}, who is entitled to receive distributions from the trust in
accordance with the settlor's instructions but does not directly control it. The
thing is the trust \textit{property} (or sometimes \textit{res}, Latin for
``thing,'' or \textit{corpus}, Latin for ``body''), whose ownership is split
between the trustee (with legal title) and the beneficiary (with equitable
title).

