\reading{Note on the Clarity of Rights and Coase}

The vagaries of nuisance standards reflect the difficulty of properly assigning
the right (either to continue action or to enjoin the action). But perhaps all
that really matters is the clarity of the property right. This was the
suggestion of Nobel-Prize-winning economist Ronald Coase (1910-2013) in his
famous article, \textit{The Problem of Social Cost}, 3 \textsc{J.L. \& Econ}. 1
(1960). The article concerned the previously encountered problem of
externalities---costs or benefits of an action that are borne by someone other
than the actor. When a factory emits smoke, for example, the smoke causes harms
to others that the factory owner does not experience. They are external to his
decision to operate, and therefore more likely to be produced than we might
want. Externalities need not be negative. The factory might stimulate economic
development, e.g., by attracting restaurants to open nearby to cater to its
workers. 

It has been argued that property rights emerge when the benefits of
internalizing externalities outweigh the costs of establishing a property
system. Harold Demsetz, \textit{Toward a Theory of Property Rights}, 57
\textsc{Am. Econ. Rev.} 347 (1967). To return to the pasture held in common,
suppose we make the land subject to private ownership. Giving property rights to
a single party means that she will bear the cost of overgrazing (and thus take
them into account before allowing that to happen, thereby internalizing the
externality). She will likewise reap the benefits of improvements like an
irrigation system, which without property rights would have been shared by too
many to make the investment worthwhile.

But other externalities may remain. What happens when the smells of the pasture
annoy the neighbors? Or if the land is used for fracking? Or a factory? How do
we address the resulting harms to others? Regulation is a traditional answer to
the problem of externality. The party causing the harm can either be made to pay
or, if the harm is serious enough, cease the offending activity.

Enter Coase. He argued that the traditional approach, of trying to stop the
harm, is question-begging in light of the reciprocity of harms: 
\begin{quote}
The question is commonly thought of as one in which A inflicts harm on B and
what has to be decided is: how should we restrain A? But this is wrong. We are
dealing with a problem of a reciprocal nature. To avoid the harm to B would
inflict harm on A. The real question that has to be decided is: should A be
allowed to harm B or should B be allowed to harm A? The problem is to avoid the
more serious harm.
\end{quote}
Coase, \textit{supra}, at 2. In other words, the issue is not stopping harm, but
rather ascertaining whether the complained-of act does more harm than good. The
market can help here, so long as property rights are clear and transaction costs
are ignored. ``It is always possible to modify by transactions on the market the
initial legal delimitation of rights. And, of course, if such market
transactions are costless, such a rearrangement of rights will always take place
if it would lead to an increase in the value of production.'' \textit{Id.} at
15.

So imagine a world in which there is only a smoke-producing factory (and its
owner) and a house (and its owner, who has sued the factory for causing a
nuisance). Suppose further that the homeowner values life without smoke at \$50,
and the factory owner values operating at \$100. The nuisance suit then
clarifies who has the relevant property right. If the homeowner wins, he now has
the right to enjoin the factory owner. In a world without transactions costs,
what happens next? We would expect the factory owner to pay the homeowner to
release the injunction (as she values operation more than he values life without
smoke). What if the activity is deemed to \textit{not} be a nuisance? Then there
is no deal to be had. The factory owner's property rights encompass the right to
emit smoke, and she values it more than the homeowner.

One interesting consequence of our hypothetical scenario is that the initial
allocation of property rights \textit{does not matter} with regards to whether
the factory operates. Absent transaction costs, operations continue no matter
which property owner ``wins'' the right to harm the other.\footnote{To make sure
you understand this point, repeat the exercise with reversed dollar values. You
will see that the factory will \textit{shut down} regardless of whether it is a
nuisance.} Coase argued that 
\begin{quote}
It is necessary to know whether the damaging business is liable or not for
damage caused since without the establishment of this initial delimitation of
rights there can be no market transactions to transfer and recombine them. But
the ultimate result (which maximises the value of production) is independent of
the legal position if the pricing system is assumed to work without cost.
\end{quote}
\textit{Id.} at 8. This insight is referred to as the \textit{Coase
Theorem}.\footnote{The term ``Coase Theorem'' to describe Coase's insight is
generally ascribed to George Stigler.} The theorem has a variety of expressions.
It is the idea that absent transactions costs, parties will bargain to efficient
outcomes concerning externalities regardless of the initial allocation of
property rights. The implication for nuisance law is the suggestion that if
transaction costs are low, it might matter more that property rights be clear
than that they be properly assigned in the first instance.

\textit{The Problem of Social Cost} is one of the more cited and debated
articles in legal history. One problem with characterizing the debate is that it
involves not only Coase's work, but the various interpretations that may or may
not be a fair representation of his ideas. \textit{See, e.g.}, Robert C.
Ellickson, \textit{The Case for Coase and Against ``Coaseanism''}, 99
\textsc{Yale L.J}. 611 (1989) (``Coase's name is consistently attached to
propositions that he has explicitly repudiated.''). For present purposes, it is
worth noting four reasons to be cautious in drawing normative lessons from
Coase. First, as Coase himself emphasized, transactions costs are always present
in the real world and often quite high. So if a factory is emitting smoke that
falls on a neighborhood (rather than a single homeowner), bargaining costs may
be large. The neighbors will face the difficulty of coordination (and the
attendant problems of free riders and holdouts). Moreover, the health
consequences of the factory may not be well known (i.e., there is a cost to
simply having the information necessary for the neighborhood to know how highly
it values freedom from smoke). Second, even if property rights allocations
matter less than we think with respect to the production of externalities, they
remain important from the perspective of distributive justice. When a judge
decides whether A must pay B, or vice versa, one becomes wealthier at the
expense of the other. The Coase Theorem tells us nothing about who merits the
windfall. Likewise, wealth matters with respect to how the gain or loss is
experienced insofar as money has a diminishing marginal utility. So, someone
with only \$1000 to his name is likely to value an additional \$1000 more than
would a millionaire. Third, unequal baseline distributions of wealth mean that
many hypothesized transactions based on competing subjective valuations of
entitlements may be impossible: what might it mean for a person with net
financial worth of \$10,000 to value their respiratory health at \$100,000?
Could such a person effectively bargain over another's right to pollute the air
they breathe? Fourth, the proposition that initial allocations do not matter has
been empirically challenged. It has been observed that people value what they
possess more than what they do not. I may, for example, be willing to pay \$50
to shut a factory down. But if my starting point is one in which the factory is
not yet operating and I have a veto, I might demand \$100 to release it. The
``endowment effect'' might mean that initial allocations therefore matter. For a
colorful example of this effect in play over the right to recline an airline
seat, see Christopher Buccafusco \& Christopher Jon Sprigman, \textit{Who
Deserves Those 4 Inches of Airplane Seat Space?} \textsc{Slate} (Sept. 23,
2014),
\url{http://www.slate.com/articles/health_and_science/science/2014/09/airplane_seat_reclining_can_economics_reveal_who_deserves_the_space.single.html}).

All that said, Coase's article suggests that we keep in mind the value of clear
property rights and the prospect that market mechanisms may sometimes be
preferable to judicial allocations. Likewise Coase reminds us anew that law is
not all. And, indeed, neither is the market. As we discussed in earlier
chapters, social norms may play a powerful role in resolving usage disputes.
These norms may be powerful enough to resolve disputes notwithstanding changes
in the underlying legal regime. For a classic account of this dynamic,
concerning payments by farmers for damage done by wandering cattle, \textit{see}
\textsc{Robert Ellickson, Order without Law: How Neighbors Settle Disputes}
(1994). 

