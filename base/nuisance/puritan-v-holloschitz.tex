\reading{Puritan Holding Co. v. Holloschitz}

\readingcite{372 N.Y.S.2d 500 (Sup. Ct. 1975)}

\opinion \textsc{Walter M. Schackman}, J.

Plaintiff owns a small apartment building, recently renovated, on West 93rd
Street in Manhattan, almost directly across the street from a building owned by
the defendant. The latter building has been abandoned. Plaintiff claims the
defendant has created a nuisance by not properly caring for her property and
claims it has suffered damages as a result. Defendant did not appear in the
action and an inquest was held before the court.

The uncontroverted proof at trial was that defendant's building had
deteriorated, become unsightly and been taken over by derelicts. The building's
condition has caused a deterioration in values on the block. A real estate
expert testified that the depreciation in value of plaintiff's property since
the abandonment of defendant's building was \$30,000 to \$35,000. He further
stated it would be impossible for plaintiff to obtain a mortgage because of the
condition of the defendant's property. The question for the court is whether the
failure of the defendant to supervise her abandoned property constitutes the
maintenance of a private nuisance.

An excellent definition of nuisance appears in 4 ALR3d 908: ``The nuisance
doctrine operates as a restriction upon the right of an owner of property to
make such use of it as he pleases. In legal phraseology the term `nuisance' is
applied to that class of wrongs which arises from the unreasonable,
unwarrantable, or unlawful use by a person of his own property, and which
produces such material annoyance, inconvenience, discomfort or hurt that the law
will presume a consequent damage. It is so comprehensive that it has been
applied to almost all wrongs which have interfered with the rights of the
citizen in his person, property, the enjoyment of his property, or his comfort.
It has been said that the term `nuisance' is incapable of an exact and
exhaustive definition which will fit all cases, because the controlling facts
are seldom alike, and each case stands on its own footing.''

The court has made a search of the reported cases in New York and has been
unable to find any similar to the case at bar. However, it has been held that
``every person who suffered actual damages, whether direct or consequential,
from a nuisance, might maintain an action for his own particular injury.''
(\textit{Lansing v Smith}, 4 Wend 9.) There are numerous cases where property
owners, adjacent to or in the vicinity of a nuisance, were entitled to damages.
Examples are: where a tire shop emitted offensive odors and fumes; the discharge
of large quantities of dust; an open burning operation by a city in a landfill
area and blasting operations.

In considering whether an activity is a nuisance, the court must be mindful of
the location and surroundings as well as other circumstances. An activity which
occurs in a particular location and surroundings may be reasonable, while the
same activity in another location and in other surroundings may be a nuisance.

West 93rd Street is in the West Side Urban Renewal area which has recently seen
a marked upward trend in real estate values. Annually there are thousands of
buildings abandoned throughout New York City. Some buildings abandoned and left
in disrepair in certain deteriorating neighborhoods of the city may not
constitute a nuisance. However, here a building has been abandoned in a location
where property owners are trying to maintain and upgrade the housing standards.
Defendant has clearly violated section C26-80.0 of the Administrative Code of
the City of New York which requires that vacant buildings must be either
continuously guarded or sealed. The court is of the opinion that defendant's
actions constitute a nuisance.

The court is not unmindful of the fact that given the number of abandonments,
estimated by the Housing and Development Administration of the City of New York
at approximately 12,000 units per year, and the further fact that the city does
not have the funds to force the owners to maintain these properties, a decision
in favor of plaintiff herein could result in a multiplicity of lawsuits.
However, one bad building may eventually destroy an entire neighborhood. The
courts have a duty to examine each situation independently.

Plaintiff has provided sufficient proof that defendant's building is, in its
present condition, a nuisance. It is entitled to the difference between the
market value of the building before and after the nuisance. Plaintiff's expert
has testified that the difference in value is \$30,000 to \$35,000. The court
finds in favor of the plaintiff in the sum of \$30,000.

