\reading{Sans v. Ramsey Golf \& Country Club, Inc.}

\readingcite{149 A.2d 599 (N.J. 1959)}

\opinion \textsc{Francis}, J.

An injunction was issued by the Chancery Division of the Superior Court against
defendant Ramsey Golf and Country Club, Inc., barring the further use of the
men's and women's third tees of its golf course. The Appellate Division
affirmed.\ldots

The issue presented is a novel one. The facts which created it are not seriously
in dispute. The physical setting which forms its background is the product of
the ingenuity of a real estate developer.

[The defendant operated a residential and country club development with a
nine-hole golf course.] The development tract contained three small lakes, one
of which, called Mirror Lake, became the water hazard hole about which this
controversy centers.\ldots

In 1949 the plaintiffs, husband and wife, purchased a lot in the development.
Naturally, they were aware of the existence of the golf course, and they became
members of the club. They commenced construction of a home on the lot in 1950,
after which they acquired two adjoining parcels. One side of their property
adjoins the fairway of the second hole. The rear line of the three lots is near
Mirror Lake but does not run to the water. It is separated from the edge of the
lake by a strip of land varying in width from 11 to 40 feet, which is owned by
the golf club.

In 1948 the present third women's tee was built. Its location was designed to
create a short par 4 water hole.\ldots [T]he tee had been in continuous use
since its installation, although the plaintiff Ralph Sans testified that he did
not notice it until 1950 when his home was being built. Subsequently, apparently
in 1949, a separate men's tee was built for this hole about 30 feet farther from
the northerly edge of the lake. The purpose was to lengthen the water hazard for
the men. Both tees are on golf club property. According to Sans, the men's tee
is ``roughly'' 50 to 60 feet from the southerly corner of the rear of his house;
the women's tee is closer.

In order to reach the third tees from the second green, the golfers walk along
the 11 to 40-foot-wide path (owned by defendant and described above) separating
plaintiffs' rear lawn from the lake.

Plaintiffs moved into their new home in June or July of 1951, and have lived
there since that time. They have two children, who were 10 and 11 years of age
when the case was heard. As the membership of the club grew, play on the golf
course increased, and the players' use of the third tees and the path to reach
them became annoying and burdensome to plaintiffs. They began to complain to
defendant's officials, and thereafter and until this suit was brought, they
sought to effect the relocation of the tees to the north of the northerly line
of the lake. Such a change is feasible. In fact, when a stay of the restraint
issued by the trial court was denied, a new temporary tee was built and has been
in use pending the determination of this appeal. The objection of defendant to
adopting it permanently is that an attractive short par 4 water hole is
transformed into an ordinary par 3 one on a nine-hole course which already has
three par 3 holes.

Plaintiffs' complaint charged defendant and its members with trespassing on
their land by using the pathway along the lake in walking to the ladies' and
men's tees in question. This contention was abandoned when it appeared that
plaintiffs did not own the strip and that, although National had not conveyed it
to defendant in the original 1945 deed, a transfer had been made by deed in
1955. Other allegations, however, in company with the issues appearing in the
pretrial order, were deemed by the trial court to present a claim that the
location of the tees and the manner and incidents of their use by defendant and
its members constituted a private nuisance as to plaintiffs. The trial was
conducted on the latter basis.

Proof was adduced that in the golf season play begins on the third tees as early
as 6 A.M. and continues throughout the day until twilight. On week-ends and
holidays the activity is more intense. Sans spoke of an ``endless stream of
golfers'' using the path just in back of his house.\ldots 

When Sans bought his first lot in 1949, the one on which his home was later
constructed, he did not see the tee or tees in question. And there is no proof
that anyone called them to his attention. It does appear that a certain brochure
respecting the development had been given to him. A similar one was introduced
in evidence. It contained what appeared to be an aerial color view of the tract,
including the course. Although the tees were indicated, none was depicted on
plaintiffs' side of the lake. When an inquiry was made on cross-examination as
to whether he did not know that he was ``buying a piece of property immediately
adjacent to the golf course,'' he answered: ``No, we did not buy a piece
adjacent to the golf course. We had a choice of three lots on that end and we
bought the lot away from the golf course.'' And as has been indicated, he
testified further that he did not see a tee in the rear of his lots until some
time in 1950 when his home was being erected.

According to plaintiffs, the constant movement of the players to and from the
tee in close proximity to their rear lawn and house was accompanied by a flow of
conversation which became annoying and burdensome to them. It awakened them and
their children as early as 7 in the morning and it pervaded their home all day
long until twilight. Moreover, they have a consciousness that everything they
say in or around the house can be heard out on the path and so they are ``under
a constant strain and constant tension.'' They ``never feel relaxed or free at
home''; ``(w)e never know when there is someone in our back yard.''
Occasionally, a
low hook or slice or heeled shot of a golfer carries upon their lawn. Then, by
means of a trespass, the ball is retrieved. Sometimes it is played from that
position. Apparently there are no out-of-bounds stakes in the area. The
combination of difficulties makes it impossible to sit outside and ``enjoy
supper.''

At times there are as many as 12 persons waiting to use the ladies' and men's
tees. On a short course containing three par 3 holes, such backing up of playing
groups, particularly at a 260-yard water hole, might well be expected. This
gathering adds to the conversation, and the voices can be heard in the house.
While silence is the conventional courtesy when a golfer is addressing his ball
and swinging, the ban is relaxed between shots, and presumably the nature of the
comments depends in some measure upon the success or failure of the player in
negotiating the hazardous water.

But an even more serious objection involves plaintiffs' children. They have no
freedom of play on their back lawn. Golfers tell them not to play there and
constantly admonish them to be quiet. If they move their activities to the north
side of the property, they are endangered by balls being driven on the second
fairway. This exposure has constantly worried Mrs. Sans. The children have a
dog. On one occasion they were cavorting in the rear of the house and the dog
was barking. A golfer instructed them to keep it quiet, and when they were
unable to do so he walked on plaintiffs' property and knocked the animal
unconscious with a club---even though one of the children pleaded with him not
to
do it. Complaint about the incident to one of defendant's officials met with the
response that ``The dog had no right to be there.'' At times the players allow
their own dogs to accompany them around the course, and they have attacked
plaintiffs' dog when it was on the rear lawn.

The resident members of the club have the common right to use the lakes for
fishing and boating. Plaintiffs have an aluminum boat in the lake immediately to
the rear of their house. If the children take the boat out, the golfers at these
tees order them off the water. They cannot fish with safety from the banks to
the rear of the house for the same reason, and because of the danger of being
struck by golf balls. Even in the winter, when children were ice skating there,
golfers were hitting balls over their heads to the third fairway.\ldots

Defendant recognized the danger, and at times during the winter the tee was
closed off to avoid possible injury to the skaters. When this happened the hole
was played from the other side of the lake---presumably in a manner similar to
that followed since the injunction in this case.

On the basis of the evidence, which stands without substantial dispute,
plaintiffs claim that the third tees in their present location constitute a
private nuisance and that their use should be enjoined. Defendant denies that
the facts in their total impact warrant that conclusion. Further, it claims that
plaintiffs bought their lots, built their home and moved into the area with full
knowledge of the existence and use of the golf course and therefore assumed any
annoyances and inconveniences incident to the playing of the game.

The circumstances here are unique. A situation where a person buys or builds a
home adjoining a wholly independent, unrelated and existing conventional type
golf course is quite dissimilar. The basic theme of this development was
residence. The recreational facilities, including the golf course were
subordinate. Their purpose and existence were to make the area a desirable one
in which to dwell. Note the ecstatic exclamations of the developer's brochures:
\begin{quote}
The perfect home location;\ldots a millionaire's paradise for moderate income
families;\ldots Ramsey Country Club Estates is the culmination of a ten year
search for the perfect home location\ldots. Each approved purchaser will
automatically receive a share representing proportionate ownership in the
Country Club and all its properties. The Club will own the impressive \$100,000
ivy covered stone mansion for its club house. Here will be the center of social
life for this unusual new community\ldots. Owner-members of the Ramsey Country
Club will own for their \textit{exclusive} \textit{use} the new 9-hole golf
course\ldots (the record contains no explanation of how the associate
members-non-owners of property in the development-happened to be admitted to the
club. Sans understood that membership was to be limited to property owners.),
spacious sand bathing beaches, three picturesque lakes for canoeing, boating and
fishing\ldots complete facilities for the enjoyment of all winter sports\ldots.
Residents will enjoy swimming, canoeing, fishing, ice-skating in the comfort and
safety of their own private community.\ldots This magnificent club house and its
grounds---all of these wonderful recreational facilities---will be shared, owned
and
enjoyed by a selected group of families who will live luxuriously in these
unusual and incomparable surroundings for less than the cost of a small city
apartment. (Emphasis added, insertion ours.)
\end{quote}

The plaintiffs may justly assert that these comments add equitable strength to
their position in the present controversy. The brochure given to them before
they became purchasers in 1949 portrayed the layout of the course; the greens
were numbered and the tees were indicated. As has been pointed out, no tee
appeared on their side of Mirror Lake. No suggestion is made that any
representative of the developer or of defendant apprised them of any such tee.
And it is not shown on the detailed map on file in the county clerk's office. In
the factual context, the element of reliance by the Sans cannot be overlooked.

Thus the heart of the project was and is the home. The pastime facilities were
intended to be no more than an aid to the enjoyment of the home, as the veins
facilitate the functions of the heart. An avoidable and readily curable ailment
in one vein should not be permitted to impair the central organ. Especially is
this true when the remedy calls for a comparatively simple adjustment which will
not materially impair the physical structure in its entirety.

The essence of a private nuisance is an unreasonable interference with the use
and enjoyment of land. The elements are myriad. The law has never undertaken to
define all of the possible sources of annoyance and discomfort which would
justify such a finding. Pollock, Torts (1887), 260, 261. Litigation of this type
usually deals with the conflicting interests of property owners and the question
of the reasonableness of the defendant's mode of use of his land. The process of
adjudication requires recognition of the reciprocal right of each owner to
reasonable use, and a balancing of the conflicting interests. The utility of the
defendant's conduct must be weighed against the quantum of harm to the
plaintiff. The question is not simply whether a person is annoyed or disturbed,
but whether the annoyance or disturbance arises from an unreasonable use of the
neighbor's land or operation of his business. Prosser, Torts (2d ed. 1955), 410.
As the Court of Appeals of Ohio put it in \emph{Antonik v. Chamberlain}, 81 Ohio
App. 465, 78 N.E.2d 752, 759 (1947):
\begin{quote}
The law of nuisance plys between two antithetical extremes: The principle
that every person is entitled to use his property for any purpose that he sees
fit, and the opposing principle that everyone is bound to use his property in
such a manner as not to injure the property or rights of his neighbor.
\end{quote}

Defendant's members have the right to the ordinary and expected use of the golf
course. Plaintiffs have the correlative right to the enjoyment of their
property. The element of reciprocity must be emphasized because the parties'
interests stem from a common source and are more mutually interdependent than in
the usual case. The Appellate Division properly suggests the pertinent inquiry
to be ``whether defendant's activities materially and unreasonably interfere
with plaintiffs' comforts or existence, `not according to exceptionally refined,
uncommon, or luxurious habits of living, but according to the simple tastes and
unaffected notions generally prevailing among plain people.'\,''

In the unusual circumstances of this case, the activities of defendant are
manifestly incompatible with the ordinary and expected comfortable life in
plaintiffs' home and the normal use of their property. The evaluation of the
conflicting equities must be made in the factual framework presented. And any
relief granted must result from a reasonable accommodation of those equities to
each other in the light of the evaluation. In our judgment, the facts considered
in their totality demonstrate that plaintiffs' interests are paramount and
demand reasonable protection. The trial court and the Appellate Division felt
that a proper balance of equitable convenience could be achieved by requiring
defendant to relocate the ladies' and men's third tees. Such relief, in our
opinion, does not represent a burden disproportionate to the travail which would
be suffered by plaintiffs and their family through the perpetuation of the
present method of play on the course.

Judgment affirmed.

