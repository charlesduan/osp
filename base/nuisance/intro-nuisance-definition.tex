``A private nuisance is a nontrespassory invasion of another's interest in the
private use and enjoyment of land.'' \textsc{Restatement (Second) of Torts}
\S~821D (1979). What does that mean? Nuisance law is a history of courts trying
to come to grips with a fairly vague exhortation. Judges sometimes invoke the
maxim \term[sic utere]{sic utere tuo ut alienum non laedas}. ``[O]ne must so use
his own
rights as not to infringe upon the rights of another. The principle of
\textit{sic utere} precludes use of land so as to injure the property of
another.'' \emph{Cline v. Dunlora S., LLC}, 726 S.E.2d 14, 17 (Va. 2012). 

That's intuitive, but unhelpful. Back to the factory versus the home. If my
ownership of land includes the right to emit smoke, I interfere with my
neighbor's ability to enjoy her home. But if her property right includes the
ability to shut me down, then her preferred property use interferes with
\textit{my} ability to use my property as I see fit. The harms are reciprocal.
Appeals to \textit{sic utere} beg the question. That said, there is something
intuitively appealing about the maxim, and perhaps you have a strong intuition
(based on what?) that factories ``cause'' harm in a way that homes do not. How
far do intuitions of harm go? What if, instead of using my property, I prefer to
let it fall into disuse? Does this passive act cause harm? 

