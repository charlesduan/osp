\reading{Boomer v. Atlantic Cement Co.}

\readingcite{257 N.E.2d 870 (N.Y. 1970)}

\opinion \textsc{Bergan}, Judge.

Defendant operates a large cement plant near Albany. These are actions for
injunction and damages by neighboring land owners alleging injury to property
from dirt, smoke and vibration emanating from the plant. A nuisance has been
found after trial, temporary damages have been allowed; but an injunction has
been denied.

The public concern with air pollution arising from many sources in industry and
in transportation is currently accorded ever wider recognition accompanied by a
growing sense of responsibility in State and Federal Governments to control it.
Cement plants are obvious sources of air pollution in the neighborhoods where
they operate.

But there is now before the court private litigation in which individual
property owners have sought specific relief from a single plant operation. The
threshold question raised by the division of view on this appeal is whether the
court should resolve the litigation between the parties now before it as
equitably as seems possible; or whether, seeking promotion of the general public
welfare, it should channel private litigation into broad public objectives.

A court performs its essential function when it decides the rights of parties
before it. Its decision of private controversies may sometimes greatly affect
public issues. Large questions of law are often resolved by the manner in which
private litigation is decided. But this is normally an incident to the court's
main function to settle controversy. It is a rare exercise of judicial power to
use a decision in private litigation as a purposeful mechanism to achieve direct
public objectives greatly beyond the rights and interests before the court.

Effective control of air pollution is a problem presently far from solution even
with the full public and financial powers of government. In large measure
adequate technical procedures are yet to be developed and some that appear
possible may be economically impracticable.

It seems apparent that the amelioration of air pollution will depend on
technical research in great depth; on a carefully balanced consideration of the
economic impact of close regulation; and of the actual effect on public health.
It is likely to require massive public expenditure and to demand more than any
local community can accomplish and to depend on regional and interstate
controls.

A court should not try to do this on its own as a by-product of private
litigation and it seems manifest that the judicial establishment is neither
equipped in the limited nature of any judgment it can pronounce nor prepared to
lay down and implement an effective policy for the elimination of air pollution.
This is an area beyond the circumference of one private lawsuit. It is a direct
responsibility for government and should not thus be undertaken as an incident
to solving a dispute between property owners and a single cement plant---one of
many---in the Hudson River valley.

The cement making operations of defendant have been found by the court of
Special Term to have damaged the nearby properties of plaintiffs in these two
actions. That court, as it has been noted, accordingly found defendant
maintained a nuisance and this has been affirmed at the Appellate Division. The
total damage to plaintiffs' properties is, however, relatively small in
comparison with the value of defendant's operation and with the consequences of
the injunction which plaintiffs seek.

The ground for the denial of injunction, notwithstanding the finding both that
there is a nuisance and that plaintiffs have been damaged substantially, is the
large disparity in economic consequences of the nuisance and of the injunction.
This theory cannot, however, be sustained without overruling a doctrine which
has been consistently reaffirmed in several leading cases in this court and
which has never been disavowed here, namely that where a nuisance has been found
and where there has been any substantial damage shown by the party complaining
an injunction will be granted.

The rule in New York has been that such a nuisance will be enjoined although
marked disparity be shown in economic consequence between the effect of the
injunction and the effect of the nuisance.

The problem of disparity in economic consequence was sharply in focus in
\emph{Whalen v. Union Bag \& Paper Co.}, 208 N.Y. 1, 101 N.E. 805. A pulp mill
entailing an investment of more than a million dollars polluted a stream in
which plaintiff, who owned a farm, was ``a lower riparian owner''. The economic
loss to plaintiff from this pollution was small. This court, reversing the
Appellate Division, reinstated the injunction granted by the Special Term
against the argument of the mill owner that in view of ``the slight advantage to
plaintiff and the great loss that will be inflicted on defendant'' an injunction
should not be granted. ``Such a balancing of injuries cannot be justified by the
circumstances of this case'', Judge Werner noted. He continued: ``Although the
damage to the plaintiff may be slight as compared with the defendant's expense
of abating the condition, that is not a good reason for refusing an
injunction''.

Thus the unconditional injunction granted at Special Term was reinstated. The
rule laid down in that case, then, is that whenever the damage resulting from a
nuisance is found not ``unsubstantial'', viz., \$100 a year, injunction would
follow. This states a rule that had been followed in this court with marked
consistency.

There are cases where injunction has been denied. \emph{McCann v. Chasm Power
Co.}, 211 N.Y. 301, 105 N.E. 416 is one of them. There, however, the damage
shown by plaintiffs was not only unsubstantial, it was non-existent. Plaintiffs
owned a rocky bank of the stream in which defendant had raised the level of the
water. This had no economic or other adverse consequence to plaintiffs, and thus
injunctive relief was denied.\ldots Thus if, within Whalen v. Union Bag \& Paper
Co., Supra which authoritatively states the rule in New York, the damage to
plaintiffs in these present cases from defendant's cement plant is ``not
unsubstantial'', an injunction should follow.

Although the court at Special Term and the Appellate Division held that
injunction should be denied, it was found that plaintiffs had been damaged in
various specific amounts up to the time of the trial and damages to the
respective plaintiffs were awarded for those amounts. The effect of this was,
injunction having been denied, plaintiffs could maintain successive actions at
law for damages thereafter as further damage was incurred.

The court at Special Term also found the amount of permanent damage attributable
to each plaintiff, for the guidance of the parties in the event both sides
stipulated to the payment and acceptance of such permanent damage as a
settlement of all the controversies among the parties. The total of permanent
damages to all plaintiffs thus found was \$185,000. This basis of adjustment has
not resulted in any stipulation by the parties.

This result at Special Term and at the Appellate Division is a departure from a
rule that has become settled; but to follow the rule literally in these cases
would be to close down the plant at once. This court is fully agreed to avoid
that immediately drastic remedy; the difference in view is how best to avoid it.

One alternative is to grant the injunction but postpone its effect to a
specified future date to give opportunity for technical advances to permit
defendant to eliminate the nuisance; another is to grant the injunction
conditioned on the payment of permanent damages to plaintiffs which would
compensate them for the total economic loss to their property present and future
caused by defendant's operations. For reasons which will be developed the court
chooses the latter alternative.

If the injunction were to be granted unless within a short period---e.g., 18
months---the nuisance be abated by improved methods, there would be no assurance
that any significant technical improvement would occur.

The parties could settle this private litigation at any time if defendant paid
enough money and the imminent threat of closing the plant would build up the
pressure on defendant. If there were no improved techniques found, there would
inevitably be applications to the court at Special Term for extensions of time
to perform on showing of good faith efforts to find such techniques.

Moreover, techniques to eliminate dust and other annoying by-products of cement
making are unlikely to be developed by any research the defendant can undertake
within any short period, but will depend on the total resources of the cement
industry nationwide and throughout the world. The problem is universal wherever
cement is made.

For obvious reasons the rate of the research is beyond control of defendant. If
at the end of 18 months the whole industry has not found a technical solution a
court would be hard put to close down this one cement plant if due regard be
given to equitable principles.

On the other hand, to grant the injunction unless defendant pays plaintiffs such
permanent damages as may be fixed by the court seems to do justice between the
contending parties. All of the attributions of economic loss to the properties
on which plaintiffs' complaints are based will have been redressed.

The nuisance complained of by these plaintiffs may have other public or private
consequences, but these particular parties are the only ones who have sought
remedies and the judgment proposed will fully redress them. The limitation of
relief granted is a limitation only within the four corners of these actions and
does not foreclose public health or other public agencies from seeking proper
relief in a proper court.

It seems reasonable to think that the risk of being required to pay permanent
damages to injured property owners by cement plant owners would itself be a
reasonable effective spur to research for improved techniques to minimize
nuisance.\ldots 

The damage base here suggested is consistent with the general rule in those
nuisance cases where damages are allowed. ``Where a nuisance is of such a
permanent and unabatable character that a single recovery can be had, including
the whole damage past and future resulting therefrom, there can be but one
recovery'' (66 C.J.S. Nuisances s 140, p. 947). It has been said that permanent
damages are allowed where the loss recoverable would obviously be small as
compared with the cost of removal of the nuisance.\ldots

Thus it seems fair to both sides to grant permanent damages to plaintiffs which
will terminate this private litigation. The theory of damage is the ``servitude
on land'' of plaintiffs imposed by defendant's nuisance. (\emph{See United
States v. Causby}, 328 U.S. 256, 261, 262, 267, 66 S.Ct. 1062, 90 L.Ed. 1206,
where the term ``servitude'' addressed to the land was used by Justice Douglas
relating to the effect of airplane noise on property near an airport.)

The judgment, by allowance of permanent damages imposing a servitude on land,
which is the basis of the actions, would preclude future recovery by plaintiffs
or their grantees.

This should be placed beyond debate by a provision of the judgment that the
payment by defendant and the acceptance by plaintiffs of permanent damages found
by the court shall be in compensation for a servitude on the land.

Although the Trial Term has found permanent damages as a possible basis of
settlement of the litigation, on remission the court should be entirely free to
ex-examine this subject. It may again find the permanent damage already found;
or make new findings.

The orders should be reversed, without costs, and the cases remitted to Supreme
Court, Albany County to grant an injunction which shall be vacated upon payment
by defendant of such amounts of permanent damage to the respective plaintiffs as
shall for this purpose be determined by the court.

\opinion \textsc{Jasen}, Judge (dissenting).

I agree with the majority that a reversal is required here, but I do not
subscribe to the newly enunciated doctrine of assessment of permanent damages,
in lieu of an injunction, where substantial property rights have been impaired
by the creation of a nuisance.

It has long been the rule in this State, as the majority acknowledges, that a
nuisance which results in substantial continuing damage to neighbors must be
enjoined. To now change the rule to permit the cement company to continue
polluting the air indefinitely upon the payment of permanent damages is, in my
opinion, compounding the magnitude of a very serious problem in our State and
Nation today.

In recognition of this problem, the Legislature of this State has enacted the
Air Pollution Control Act declaring that it is the State policy to require the
use of all available and reasonable methods to prevent and control air
pollution.

The harmful nature and widespread occurrence of air pollution have been
extensively documented. Congressional hearings have revealed that air pollution
causes substantial property damage, as well as being a contributing factor to a
rising incidence of lung cancer, emphysema, bronchitis and asthma.

The specific problem faced here is known as particulate contamination because of
the fine dust particles emanating from defendant's cement plant. The particular
type of nuisance is not new, having appeared in many cases for at least the past
60 years. It is interesting to note that cement production has recently been
identified as a significant source of particulate contamination in the Hudson
Valley. This type of pollution, wherein very small particles escape and stay in
the atmosphere, has been denominated as the type of air pollution which produces
the greatest hazard to human health. We have thus a nuisance which not only is
damaging to the plaintiffs,\readingfootnote{5}{There are seven plaintiffs here
who have been substantially damaged by the maintenance of this nuisance. The
trial court found their total permanent damages to equal \$185,000.} but also is
decidedly harmful to the general public.

I see grave dangers in overruling our long-established rule of granting an
injunction where a nuisance results in substantial continuing damage. In
permitting the injunction to become inoperative upon the payment of permanent
damages, the majority is, in effect, licensing a continuing wrong. It is the
same as saying to the cement company, you may continue to do harm to your
neighbors so long as you pay a fee for it. Furthermore, once such permanent
damages are assessed and paid, the incentive to alleviate the wrong would be
eliminated, thereby continuing air pollution of an area without abatement.

It is true that some courts have sanctioned the remedy here proposed by the
majority in a number of cases, but none of the authorities relied upon by the
majority are analogous to the situation before us. In those cases, the courts,
in denying an injunction and awarding money damages, grounded their decision on
a showing that the use to which the property was intended to be put was
primarily for the public benefit. Here, on the other hand, it is clearly
established that the cement company is creating a continuing air pollution
nuisance primarily for its own private interest with no public benefit.

This kind of inverse condemnation may not be invoked by a private person or
corporation for private gain or advantage. Inverse condemnation should only be
permitted when the public is primarily served in the taking or impairment of
property. The promotion of the interests of the polluting cement company has, in
my opinion, no public use or benefit.

Nor is it constitutionally permissible to impose servitude on land, without
consent of the owner, by payment of permanent damages where the continuing
impairment of the land is for a private use. This is made clear by the State
Constitution which provides that ``(p)rivate property shall not be taken for
\textit{public use} without just compensation'' (emphasis added). It is, of
course, significant that the section makes no mention of taking for a
\textit{private} use.

In sum, then, by constitutional mandate as well as by judicial pronouncement,
the permanent impairment of private property for private purposes is not
authorized in the absence of clearly demonstrated public benefit and use.

I would enjoin the defendant cement company from continuing the discharge of
dust particles upon its neighbors' properties unless, within 18 months, the
cement company abated this nuisance.

It is not my intention to cause the removal of the cement plant from the Albany
area, but to recognize the urgency of the problem stemming from this stationary
source of air pollution, and to allow the company a specified period of time to
develop a means to alleviate this nuisance.

I am aware that the trial court found that the most modern dust control devices
available have been installed in defendant's plant, but, I submit, this does not
mean that \textit{better} and more effective dust control devices could not be
developed within the time allowed to abate the pollution.

Moreover, I believe it is incumbent upon the defendant to develop such devices,
since the cement company, at the time the plant commenced production (1962), was
well aware of the plaintiffs' presence in the area, as well as the probable
consequences of its contemplated operation. Yet, it still chose to build and
operate the plant at this site.

In a day when there is a growing concern for clean air, highly developed
industry should not expect acquiescence by the courts, but should, instead, plan
its operations to eliminate contamination of our air and damage to its
neighbors.

Accordingly, the orders of the Appellate Division, insofar as they denied the
injunction, should be reversed, and the actions remitted to Supreme Court,
Albany County to grant an injunction to take effect 18 months hence, unless the
nuisance is abated by improved techniques prior to said date.

