\reading{Spur Industries, Inc. v. Del E. Webb Development Co.}

\readingcite{494 P.2d 700 (Ariz. 1972)}

\opinion \textsc{Cameron}, Vice Chief Justice.

From a judgment permanently enjoining the defendant, Spur Industries, Inc., from
operating a cattle feedlot near the plaintiff Del E. Webb Development Company's
Sun City, Spur appeals. Webb cross-appeals. Although numerous issues are raised,
we feel that it is necessary to answer only two questions. They are:
\begin{quote}
1. Where the operation of a business, such as a cattle feedlot is lawful in the
first instance, but becomes a nuisance by reason of a nearby residential area,
may the feedlot operation be enjoined in an action brought by the developer of
the residential area?

2. Assuming that the nuisance may be enjoined, may the developer of a completely
new town or urban area in a previously agricultural area be required to
indemnify the operator of the feedlot who must move or cease operation because
of the presence of the residential area created by the developer?
\end{quote}
The facts necessary for a determination of this matter on appeal are as follows.
The area in question is located in Maricopa County, Arizona, some 14 to 15 miles
west of the urban area of Phoenix, on the Phoenix-Wickenburg Highway, also known
as Grand Avenue. About two miles south of Grand Avenue is Olive Avenue which
runs east and west. 111th Avenue runs north and south as does the Agua Fria
River immediately to the west. See Exhibits A and B [in
Figure~\ref{f:nuisance-spur}].

\begin{figure}
\begin{center}
\usegraphic{nuisance-img001}
\usegraphic{nuisance-img002}
\end{center}
\caption{Exhibits A and B to \emph{Spur Industries}.}
\label{f:nuisance-spur}
\end{figure}

Farming started in this area about 1911. In 1929, with the completion of the
Carl Pleasant Dam, gravity flow water became available to the property located
to the west of the Agua Fria River, though land to the east remained dependent
upon well water for irrigation. By 1950, the only urban areas in the vicinity
were the agriculturally related communities of Peoria, El Mirage, and Surprise
located along Grand Avenue. Along 111th Avenue, approximately one mile south of
Grand Avenue and 1 1/2 miles north of Olive Avenue, the community of Youngtown
was commenced in 1954. Youngtown is a retirement community appealing primarily
to senior citizens.

In 1956, Spur's predecessors in interest, H. Marion Welborn and the Northside
Hay Mill and Trading Company, developed feed-lots, about 1/2 mile south of Olive
Avenue, in an area between the confluence of the usually dry Agua Fria and New
Rivers. The area is well suited for cattle feeding and in 1959, there were 25
cattle feeding pens or dairy operations within a 7 mile radius of the location
developed by Spur's predecessors. In April and May of 1959, the Northside Hay
Mill was feeding between 6,000 and 7,000 head of cattle and Welborn
approximately 1,500 head on a combined area of 35 acres.

In May of 1959, Del Webb began to plan the development of an urban area to be
known as Sun City. For this purpose, the Marinette and the Santa Fe Ranches,
some 20,000 acres of farmland, were purchased for \$15,000,000 or \$750.00 per
acre. This price was considerably less than the price of land located near the
urban area of Phoenix, and along with the success of Youngtown was a factor
influencing the decision to purchase the property in question.

By September 1959, Del Webb had started construction of a golf course south of
Grand Avenue and Spur's predecessors had started to level ground for more
feedlot area. In 1960, Spur purchased the property in question and began a
rebuilding and expansion program extending both to the north and south of the
original facilities. By 1962, Spur's expansion program was completed and had
expanded from approximately 35 acres to 114 acres. See Exhibit A above.

Accompanied by an extensive advertising campaign, homes were first offered by
Del Webb in January 1960 and the first unit to be completed was south of Grand
Avenue and approximately 2 1/2 miles north of Spur. By 2 May 1960, there were
450 to 500 houses completed or under construction. At this time, Del Webb did
not consider odors from the Spur feed pens a problem and Del Webb continued to
develop in a southerly direction, until sales resistance became so great that
the parcels were difficult if not impossible to sell.\ldots

~By December 1967, Del Webb's property had extended south to Olive Avenue and
Spur was within 500 feet of Olive Avenue to the north. See Exhibit B above. Del
Webb filed its original complaint alleging that in excess of 1,300 lots in the
southwest portion were unfit for development for sale as residential lots
because of the operation of the Spur feedlot.

Del Webb's suit complained that the Spur feeding operation was a public nuisance
because of the flies and the odor which were drifting or being blown by the
prevailing south to north wind over the southern portion of Sun City. At the
time of the suit, Spur was feeding between 20,000 and 30,000 head of cattle, and
the facts amply support the finding of the trial court that the feed pens had
become a nuisance to the people who resided in the southern part of Del Webb's
development. The testimony indicated that cattle in a commercial feedlot will
produce 35 to 40 pounds of wet manure per day, per head, or over a million
pounds of wet manure per day for 30,000 head of cattle, and that despite the
admittedly good feedlot management and good housekeeping practices by Spur, the
resulting odor and flies produced an annoying if not unhealthy situation as far
as the senior citizens of southern Sun City were concerned. There is no doubt
that some of the citizens of Sun City were unable to enjoy the outdoor living
which Del Webb had advertised and that Del Webb was faced with sales resistance
from prospective purchasers as well as strong and persistent complaints from the
people who had purchased homes in that area.\ldots

It is noted\ldots however, that neither the citizens of Sun City nor Youngtown
are represented in this lawsuit and the suit is solely between Del E. Webb
Development Company and Spur Industries, Inc.

\readinghead{May Spur Be Enjoined?}

The difference between a private nuisance and a public nuisance is generally one
of degree. A private nuisance is one affecting a single individual or a definite
small number of persons in the enjoyment of private rights not common to the
public, while a public nuisance is one affecting the rights enjoyed by citizens
as a part of the public. To constitute a public nuisance, the nuisance must
affect a considerable number of people or an entire community or neighborhood. 

Where the injury is slight, the remedy for minor inconveniences lies in an
action for damages rather than in one for an injunction. Moreover, some courts
have held, in the ``balancing of conveniences'' cases, that damages may be the
sole remedy. \emph{See Boomer v. Atlantic Cement Co.}, 26 N.Y.2d 219, 309
N.Y.S.2d 312, 257 N.E.2d 870, 40 A.L.R.3d 590 (1970), and annotation comments,
40 A.L.R.3d 601.

Thus, it would appear from the admittedly incomplete record as developed in the
trial court, that, at most, residents of Youngtown would be entitled to damages
rather than injunctive relief.

We have no difficulty, however, in agreeing with the conclusion of the trial
court that Spur's operation was an enjoinable public nuisance as far as the
people in the southern portion of Del Webb's Sun City were concerned.

\S~36-601, subsec. A reads as follows:
\begin{quotation}
\S~36-601. Public nuisances dangerous to public health

A. The following conditions are specifically declared public nuisances
dangerous to the public health:

1. Any condition or place in populous areas which constitutes a breeding
place for flies, rodents, mosquitoes and other insects which are capable of
carrying and transmitting disease-causing organisms to any person or persons.
\end{quotation}
By this statute, before an otherwise lawful (and necessary) business may be
declared a public nuisance, there must be a ``populous'' area in which people
are injured:
\begin{quote}
\ldots (I)t hardly admits a doubt that, in determining the question as to
whether a lawful occupation is so conducted as to constitute a nuisance as a
matter of fact, the locality and surroundings are of the first importance.
(citations omitted) A business which is not per se a public nuisance may become
such by being carried on at a place where the health, comfort, or convenience of
a populous neighborhood is affected.\ldots What might amount to a serious
nuisance in one locality by reason of the density of the population, or
character of the neighborhood affected, may in another place and under different
surroundings be deemed proper and unobjectionable.\ldots
\end{quote}
\emph{MacDonald v. Perry}, 32 Ariz. 39, 49-50, 255 P. 494, 497 (1927).

It is clear that as to the citizens of Sun City, the operation of Spur's feedlot
was both a public and a private nuisance. They could have successfully
maintained an action to abate the nuisance. Del Webb, having shown a special
injury in the loss of sales, had a standing to bring suit to enjoin the
nuisance. The judgment of the trial court permanently enjoining the operation of
the feedlot is affirmed.

\readinghead{Must Del Webb Indemnify Spur?}

A suit to enjoin a nuisance sounds in equity and the courts have long recognized
a special responsibility to the public when acting as a court of equity:

\begin{quote}
\S~104. Where public interest is involved.

Courts of equity may, and frequently do, go much further both to give and
withhold relief in furtherance of the public interest than they are accustomed
to go when only private interests are involved. Accordingly, the granting or
withholding of relief may properly be dependent upon considerations of public
interest.\ldots
\end{quote}
27 Am.Jur.2d, Equity, page 626.

In addition to protecting the public interest, however, courts of equity are
concerned with protecting the operator of a lawfully, albeit noxious, business
from the result of a knowing and willful encroachment by others near his
business.

In the so-called ``coming to the nuisance'' cases, the courts have held that the
residential landowner may not have relief if he knowingly came into a
neighborhood reserved for industrial or agricultural endeavors and has been
damaged thereby:
\begin{quotation}
Plaintiffs chose to live in an area uncontrolled by zoning laws or
restrictive covenants and remote from urban development. In such an area
plaintiffs cannot complain that legitimate agricultural pursuits are being
carried on in the vicinity, nor can plaintiffs, having chosen to build in an
agricultural area, complain that the agricultural pursuits carried on in the
area depreciate the value of their homes. The area being Primarily agricultural,
and opinion reflecting the value of such property must take this factor into
account. The standards affecting the value of residence property in an urban
setting, subject to zoning controls and controlled planning techniques, cannot
be the standards by which agricultural properties are judged.

People employed in a city who build their homes in suburban areas of the
county beyond the limits of a city and zoning regulations do so for a reason.
Some do so to avoid the high taxation rate imposed by cities, or to avoid
special assessments for street, sewer and water projects. They usually build on
improved or hard surface highways, which have been built either at state or
county expense and thereby avoid special assessments for these improvements. It
may be that they desire to get away from the congestion of traffic, smoke,
noise, foul air and the many other annoyances of city life. But with all these
advantages in going beyond the area which is zoned and restricted to protect
them in their homes, they must be prepared to take the disadvantages.
\end{quotation}
\emph{Dill v. Excel Packing Company}, 183 Kan. 513, 525, 526, 331 P.2d 539, 548,
549 (1958). See also \emph{East St. Johns Shingle Co. v. City of Portland}, 195
Or. 505, 246 P.2d 554, 560-562 (1952).

And:
\begin{quote}
\ldots a party cannot justly call upon the law to make that place suitable for
his residence which was not so when he selected it.\ldots
\end{quote}
\emph{Gilbert v. Showerman}, 23 Mich. 448, 455, 2 Brown 158 (1871).

Were Webb the only party injured, we would feel justified in holding that the
doctrine of ``coming to the nuisance'' would have been a bar to the relief asked
by Webb, and, on the other hand, had Spur located the feedlot near the outskirts
of a city and had the city grown toward the feedlot, Spur would have to suffer
the cost of abating the nuisance as to those people locating within the growth
pattern of the expanding city.\ldots

There was no indication in the instant case at the time Spur and its
predecessors located in western Maricopa County that a new city would spring up,
full-blown, alongside the feeding operation and that the developer of that city
would ask the court to order Spur to move because of the new city. Spur is
required to move not because of any wrongdoing on the part of Spur, but because
of a proper and legitimate regard of the courts for the rights and interests of
the public.

Del Webb, on the other hand, is entitled to the relief prayed for (a permanent
injunction), not because Webb is blameless, but because of the damage to the
people who have been encouraged to purchase homes in Sun City. It does not
equitable or legally follow, however, that Webb, being entitled to the
injunction, is then free of any liability to Spur if Webb has in fact been the
cause of the damage Spur has sustained. It does not seem harsh to require a
developer, who has taken advantage of the lesser land values in a rural area as
well as the availability of large tracts of land on which to build and develop a
new town or city in the area, to indemnify those who are forced to leave as a
result.

Having brought people to the nuisance to the foreseeable detriment of Spur, Webb
must indemnify Spur for a reasonable amount of the cost of moving or shutting
down. It should be noted that this relief to Spur is limited to a case wherein a
developer has, with foreseeability, brought into a previously agricultural or
industrial area the population which makes necessary the granting of an
injunction against a lawful business and for which the business has no adequate
relief.

It is therefore the decision of this court that the matter be remanded to the
trial court for a hearing upon the damages sustained by the defendant Spur as a
reasonable and direct result of the granting of the permanent injunction. Since
the result of the appeal may appear novel and both sides have obtained a measure
of relief, it is ordered that each side will bear its own costs.\ldots


