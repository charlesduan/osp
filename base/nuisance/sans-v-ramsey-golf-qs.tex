\expected{sans-v-ramsey-golf}

\item Why does \textit{Sans} conclude that the ``conflicting equities'' favor
the plaintiff?

\item \textbf{Threshold harms.} One way courts avoid getting too involved in
nuisance cases is by requiring significant harm before engaging in the balancing
of equities. \textsc{Restatement (Second) of Torts} \S~821F (1979) (``There is
liability for a nuisance only to those to whom it causes significant harm, of a
kind that would be suffered by a normal person in the community or by property
in normal condition and used for a normal purpose.''); 
\begin{quote}
Before plaintiffs may recover the injury to them must be substantial. By
substantial invasion is meant an invasion that involves more than slight
inconvenience or petty annoyance. The law does not concern itself with trifles.
Practically all human activities, unless carried on in a wilderness, interfere
to some extent with others or involve some risk of interference, and these
interferences range from mere trifling annoyances to serious harms. Each
individual in a community must put up with a certain amount of annoyance,
inconvenience or interference, and must take a certain amount of risk in order
that all may get on together. But if one makes an unreasonable use of his
property and thereby causes another substantial harm in the use and enjoyment of
his, the former is liable for the injury inflicted.
\end{quote}
\emph{Watts v. Pama Mfg. Co.}, 256 N.C. 611, 619, 124 S.E.2d 809, 815 (1962)
(citing 4 \textsc{Restatement (First) of the Law of Torts} \S~822,
cmts. g \& j).

\item \textbf{Restatement standards.} The \textsc{Restatement (Second) of Torts}
standard for a private nuisance is an activity that invades another's interest
in the use and enjoyment of land where the invasion is either ``(a) intentional
and unreasonable, or (b) unintentional and otherwise actionable under the rules
controlling liability for negligent or reckless conduct, or for abnormally
dangerous conditions or activities.'' \textsc{Restatement (Second) of Torts}
\S~822 (1979). We will focus on the first prong, intentional conduct that a
court nonetheless finds unreasonable. Section 826 sets forth two tests. The
invasion is unreasonable if ``the gravity of the harm outweighs the utility of
the actor's conduct'' or if ``the harm caused by the conduct is serious and the
financial burden of compensating for this and similar harm to others would not
make the continuation of the conduct not feasible.''\footnote{The
\textsc{Restatement} likewise provides standards for assessing the gravity of
the harm to the plaintiff, including factors like degree, duration, character,
ability to avoid, and nature of the plaintiff's activity (e.g., social value and
local suitability). \S~827. As the list indicates, they leave room for
subjective interpretation. Likewise, the assessment of the defendant's conduct
includes considerations of social value, suitability to the location, and
ability to avoid or prevent. \S~828.}

\item \textbf{``Coming to'' a nuisance.} One way to adjudicate between
competing interests is through first-in-time, first-in-right principles.
Generally, whether the plaintiff \term[coming to the nuisance]{came to the
nuisance} (i.e., acquired its
property interest \textit{after} the commencement of the allegedly unreasonable
activity by the defendant) is treated as a factor to be considered in balancing
the equities, and not as a bar to a nuisance suit. Why do you think that is? Are
there circumstances in which you think coming to a nuisance ought to bar a suit?
Likewise, compliance with zoning ordinances is a non-dispositive factor in the
defendant's favor. 

\item \textbf{Idiosyncratic harms.} The harm giving rise to nuisance liability
must be ``of a kind that would be suffered by a normal person in the community
or by property in normal condition and used for a normal purpose.''
\textsc{Restatement (Second) of Torts} \S~821F (1979). This creates difficulty
for a range of asserted, but unproven, harms. \textit{See, e.g.}, \emph{San
Diego Gas \& Electric Co. v. Superior Court}, 55 Cal. Rptr. 2d 724, 752 (1996)
(rejecting nuisance claim based on fear of powerline electromagnetic fields).
What about technological change? American law generally rejects the notion that
one has a right to light from adjacent properties. But what if one has a solar
panel? \emph{Prah v. Maretti}, 321 N.W.2d 182, 191 (Wis. 1982) (allowing
nuisance claim by owner of a solar heated home to proceed). 

\item \textbf{Malice.} There is little utility to actions taken for the purposes
of harming a neighbor, and the \textsc{Restatement} provides that such acts are
nuisances when they cause harm to a property owner's interests.
\textsc{Restatement (Second) of Torts} \S~829.\footnote{The provision also
treats acts contrary to ``common standards of decency'' as a nuisance, offering
as an illustration a farmer who breeds animals in full view of a neighbor's
family. \textit{Id.} cmt. d.} ``Spite fences'' are often explicitly the subject
of statutes. \textit{See, e.g.}, \textsc{N.H. Rev. Stat. Ann.} \S~476:1 (``Any
fence or other structure in the nature of a fence, unnecessarily exceeding 5
feet in height, erected or maintained for the purpose of annoying the owners or
occupants of adjoining property shall be deemed a private nuisance.'').

\item \textbf{Private arrangements.} If a nuisance is a violation of a property
right, it stands to reason that the right may have been transferred prior to the
nuisance suit. \textit{Cf.} \emph{DeSarno v. Jam Golf Mgmt., LLC}, 670 S.E.2d
889, 890 (Ga. 2008) (distinguishing \textit{Sans} and holding no trespass or
nuisance claims were possible because ``the easement in this case explicitly
permitted the complained-of conduct and indeed exonerated the golf course owner
from any liability for damages caused by the errant golf balls'').

