\reading{Note on ``Property Rules'' and ``Liability Rules''}

When should a court award damages and when is an injunction appropriate? One of
the most famous takes on the problem is found in Guido Calabresi \& A. Douglas
Melamed, \textit{Property Rules, Liability Rules, and Inalienability: One View
of the Cathedral}, 85 \textsc{Harv. L. Rev.} 1089 (1972). The authors outline a
framework for the protection of entitlements, distinguishing ``property'' and
``liability'' rules. 
\begin{quotation}
An entitlement is protected by a property rule to the extent that someone who
wishes to remove the entitlement from its holder must buy it from him in a
voluntary transaction in which the value of the entitlement is agreed upon by
the seller. It is the form of entitlement which gives rise to the least amount
of state intervention: once the original entitlement is decided upon, the state
does not try to decide its value. It lets each of the parties say how much the
entitlement is worth to him, and gives the seller a veto if the buyer does not
offer enough. Property rules involve a collective decision as to who is to be
given an initial entitlement but not as to the value of the entitlement.

Whenever someone may destroy the initial entitlement if he is willing to pay an
objectively determined value for it, an entitlement is protected by a liability
rule. This value may be what it is thought the original holder of the
entitlement would have sold it for. But the holder's complaint that he would
have demanded more will not avail him once the objectively determined value is
set. Obviously, liability rules involve an additional stage of state
intervention: not only are entitlements protected, but their transfer or
destruction is allowed on the basis of a value determined by some organ of the
state rather than by the parties themselves.
\end{quotation}
\textit{Id.} at 1091.\footnote{And some entitlements, as the authors discuss,
are inalienable.} We might think of an injunction against trespass as an
illustration of a property rule. The trespasser must keep out unless the
property owner agrees to let her enter. Contract damages are an example of a
liability rule. If one is willing to pay damages, one is free to breach. As the
examples suggest, property rules are associated with, well, property rights,
while liability rules are associated with contract remedies. But there are
exceptions in both subjects. For example, some states allow for private
condemnation of rights of way to provide access to landlocked privately owned
land. The owner of the property has no ability to say no to another's entry into
his land, but is limited to a compensation remedy. Conversely, under certain
circumstances a contract may be enforced by specific performance.

Calabresi and Melamed spend some time on the question of how entitlements are
assigned in the first instance (i.e., is the factory a nuisance or does its
owner have the right to pollute), but for present purposes we will focus on the
question of deciding how to protect an entitlement once assigned. In a vacuum,
property rules let parties decide for themselves how to value entitlements, but
in the real world, transaction costs get in the way. Holdouts and freeriders may
interfere with the coordination of multiple purchasers or sellers of entitlement
(e.g., when multiple neighbors live near an offending factory). When negotiation
costs exceed the entitlement's value, it will remain with the party to whom it
was assigned, regardless of overall efficiency. In such cases, a liability rule
might be preferable. 

As applied to nuisance, the authors observe:
\begin{quotation}
Traditionally\ldots the nuisance-pollution problem is viewed in terms of three
rules. First, Taney may not pollute unless his neighbor (his only neighbor let
us assume), Marshall, allows it (Marshall may enjoin Taney's nuisance). Second,
Taney may pollute but must compensate Marshall for damages caused (nuisance is
found but the remedy is limited to damages). Third, Taney may pollute at will
and can only be stopped by Marshall if Marshall pays him off (Taney's pollution
is not held to be a nuisance to Marshall). In our terminology rules one and two
(nuisance with injunction, and with damages only) are entitlements to Marshall.
The first is an entitlement to be free from pollution and is protected by a
property rule; the second is also an entitlement to be free from pollution but
is protected only by a liability rule. Rule three (no nuisance) is instead an
entitlement to Taney protected by a property rule, for only by buying Taney out
at Taney's price can Marshall end the pollution.

The very statement of these rules in the context of our framework suggests that
something is missing. Missing is a fourth rule representing an entitlement in
Taney to pollute, but an entitlement which is protected only by a liability
rule. The fourth rule\ldots can be stated as follows: Marshall may stop Taney
from polluting, but if he does he must compensate Taney.
\end{quotation}
\textit{Id.} at 1115-16 (footnotes omitted). In a low-transaction cost world,
Calabresi and Melamed would use property rules, and assign the entitlement based
on whether or not the polluter is the low-cost risk avoider. In such cases
improper allocations have distributive consequences, but transactions would at
least ensure economic efficiency. (Do you see why?) 
\begin{quote}
The moment we assume, however, that transactions are not cheap, the situation
changes dramatically. Assume we enjoin Taney and there are 10,000 injured
Marshalls. Now even if the right to pollute is worth more to Taney than the
right to be free from pollution is to the sum of the Marshalls, the injunction
will probably stand. The cost of buying out all the Marshalls, given holdout
problems, is likely to be too great, and an equivalent of eminent domain in
Taney would be needed to alter the initial injunction. Conversely, if we denied
a nuisance remedy, the 10,000 Marshalls could only with enormous difficulty,
given freeloader problems, get together to buy out even one Taney and prevent
the pollution. This would be so even if the pollution harm was greater than the
value to Taney of the right to pollute.
\end{quote}
\textit{Id.} at 1119. In such situations, the ``rule four'' possibility would
increase the range of options in a nuisance case. If circumstances made a
liability remedy appropriate, a court would be free to assign the entitlement to
either party as efficiency or distributional concerns warranted. \textit{Id.} at
1120. 

Like a particle predicted by atomic theory, the rule four injunction option was
described, but awaited observation in nature. It would not take long.

\expectnext{spur-v-del-e-webb}
