\expected{boomer-v-atlantic-cement}

\item What are the costs and benefits of leaving the question of the cement
plant's legality to the legislature? Modern environmental law is characterized
by far-reaching federal legislation (e.g., the Clean Air Act, the Clean Water
Act, the Endangered Species Act, etc.). How might things have been different had
nuisance law been the primary mechanism of environmental regulation? 

\item \textbf{Preemption.} State and federal legislation offers the prospect of
more comprehensive regulation than case-by-case nuisance adjudication. Once
these regulations are in place, defendants often claim they preempt resort to
private nuisance remedies. \textit{See} 9-64 \textsc{Powell on Real Property}
\S~64.06 (collecting examples of successful and unsuccessful preemption
defenses). Should compliance with, for example, a federal clean air regime
provide immunity to a local nuisance suit based on air pollution? Is federal
regulation best seen as a ceiling or a floor for environmental standards? 


On this question, note that federal environmental laws are often criticized for
interfering with ``property rights.'' But to the extent they limit the
availability of local nuisance law, might they also be seen as interfering with
the property rights of would-be nuisance plaintiffs?

