\expected{puritan-v-holloschitz}

\item How much should it matter that the defendant independently violated a
local regulation?

\item If \textit{Holloschitz} does not go too far, how much freedom should
courts have to judge land uses? Are there any metrics that would both provide
judicial discretion as well as contain it? We will examine several approaches
below, but the question underscores the problem of unclear boundaries in
nuisance law. A lot of property doctrine exists to help us determine the scope
of property rights \textit{without} asking a judge. The metes and bounds in a
deed tell us what is a trespass. The adverse possession limitations period lets
expectations settle. Title recording gives notice of competing interests. And so
on. When push comes to shove, litigation may be necessary to resolve disputed
boundaries, but in most cases there are ways to determine them without the aid
of a court. By contrast, the boundaries clarified by nuisance law are harder to
ascertain ex ante in part because nuisance is more a flexible standard than a
bright-line rule. What measures short of litigation are available to people like
the plaintiff here? To be sure, the law cannot anticipate every possible
conflict between property owners. There is therefore something to be said for ex
post determinations of what is a reasonable use of land. Is this reason enough
to use nuisance law to supplement regulatory and zoning schemes?

\item \textbf{Aesthetics.} Courts generally reject nuisance claims based on
aesthetic harm, but that reluctance may be eroding. \emph{Rattigan v. Wile}, 841
N.E.2d 680, 683 (Mass. 2006) (``We conclude in this appeal that activities on
one's property that create or maintain unreasonable aesthetic conditions for
neighbors are actionable as a private nuisance.''); \textit{id.} at 689-90
(arguing that the modern trend is to allow such claims). Courts also sometimes
consider aesthetic harm as part of the larger nuisance analysis. \emph{Sowers v.
Forest Hills Subdivision}, 294 P.3d 427, 430 (Nev. 2013) (``[W]e hold that the
aesthetics of a wind turbine alone are not grounds for finding a nuisance.
However, we conclude that a nuisance in fact may be found when the aesthetics
are combined with other factors, such as noise, shadow flicker, and diminution
in property value.''). 

\item What if a building became dilapidated because its owner could not afford
upkeep? If so, does \textit{Holloschitz} hint at nuisance's potential to serve
as a tool of exclusion of poor people? What other activities (or groups) might
the law target? \textit{See generally} Alfred L. Brophy, \emph{Integrating
Spaces: New Perspectives on Race in the Property Curriculum}, 55 \textsc{J.
Legal Educ}. 319, 331-33 (2005) (discussing attempts to use nuisance law as a
tool of racial discrimination); John Copeland Nagle, \emph{Moral Nuisances}, 50
\textsc{Emory} L.J. 265, 276-94 (2001) (discussing range of activities targeted
by nuisance
plaintiffs). For commentary on the disability rights implications of a recent
nuisance suit between neighbors, \textit{see} David Perry, \textit{Flowers v
Gopal---Rich folks try to declare autistic boy a ``Public Nuisance''}
(September 23, 2015),
\url{https://www.davidmperry.com/flowers-v-gopal-rich-folks-try-to/}.
Could
the mere presence of a sex offender in a residential community of families with
young children be considered a nuisance? Some public nuisance ordinances deem
repeated 911 calls a nuisance; what effect might such property law rules have on
victims of domestic violence? \textit{See} Emily Werth, \textit{The
Cost of Being ``Crime Free'': Legal and Practical Consequences of Crime Free
Rental Housing and Nuisance Property Ordinances} (Aug. 2013),
\url{http://povertylaw.org/sites/default/files/files/housing-justice/cost-of-being-crime-free.pdf}.

\item \textbf{Nuisance and Trespass}.
Historically, trespass and nuisance were two distinct common-law classes of
injury involving real property. 9 \textsc{R. Powell, Real Property} \S~64.01[1],
at 64--5 (1999); 4 \textsc{Restatement (Second) of Torts} \S~821D cmt. a (1979).
A defendant who invaded a plaintiff's possession was a trespasser; a defendant
who interfered with a plaintiff's use and enjoyment of his property by acts done
elsewhere than on the plaintiff's land was subject to a claim of nuisance. 
\begin{quote}
This ancient distinction between trespass and nuisance, on the basis of whether
an invasion of a plaintiff's land was direct or indirect, is not followed by
more recent cases. Instead, recent case law treats trespass cases as involving
acts that interfere with a plaintiff's exclusive possession of real property and
nuisance cases as involving acts interfering with a plaintiff's use and
enjoyment of real property. In other words, the distinction no longer rests on
the means by which the invasion is effected but, instead, on the nature of the
right with which the tortfeasor interferes. When viewed in this way, claims of
nuisance may include an instance of trespass in that a physical entry onto land
possessed exclusively by another also may affect, in the abstract, the
possessor's use and enjoyment of the land.
\end{quote}
\emph{Boyne v. Town of Glastonbury}, 955 A.2d 645, 652-53 (Conn. App. 2008)
(successive citations to \textsc{Powell} and the \textsc{Restatement} omitted);
\textit{see also, e.g.}, \emph{Cook v. DeSoto Fuels, Inc.}, 169 S.W.3d 94, 103
(Mo. Ct. App. 2005) (``[Plaintiffs'] allegations that [defendant] caused
gasoline to enter their property can constitute a claim for both trespass and
nuisance because that contamination involves a direct physical invasion that
interferes with both the right to possession and the use and enjoyment of
property.''); \emph{Md. Heights Leasing, Inc. v. Mallinckrodt, Inc.}, 706
S.W.2d 218 (Mo. Ct. App. 1985) (complaint of low-level radiation emissions
stated claim for nuisance and trespass).

