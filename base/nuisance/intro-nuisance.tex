\begin{quote}
There is perhaps no more impenetrable jungle in the entire law than that
regarding the word ``nuisance.''
\end{quote}
\textsc{W. Page Keeton et al.}, \textsc{Prosser and Keeton on Torts}
\S~86 (5th ed. 1984).

\expected{tragedy-commons-qs}

People want to use land for different things. We've already seen how the
resulting conflicts provide a rationale for property rights. In the so-called
tragedy of the commons, for example, each cattle owner has an incentive to use
the pasture for grazing before someone else beats him or her to it. The race to
consume leaves the pasture depleted and everyone worse off. Property rights are
one, but by no means the only, mechanism for addressing the problem, as an
individual owner may have the necessary incentive to ensure that the plot is not
overconsumed. Likewise property rights enable owners to manage their holdings
free from external interference. The farmer may plant her corn even though her
neighbor wishes a hotel were there. And property rights facilitate the
reconciliation of incompatible interests without outside intervention.
Determining whether Blackacre is better off as a hotel or a farm might be a hard
call for an outside regulator. But with enough money, the would-be hotelier may
simply buy out the farmer (or vice versa).

\expected{delfino-v-vealencis} % Partition

% Lease conflicts. Any other section on lease conflicts would be fine.
\expected{intro-leases-quiet-enjoyment}

This hardly exhausts the universe of potential dispute. As we have already seen,
disputes may emerge within property boundaries. One joint tenant may want to use
a pond for irrigation; the other, fishing. Property law provides another set of
management mechanisms for this kind of disagreement---e.g. partition
actions---that we studied in our unit on concurrent interests. Likewise the law
of leaseholds has its own set of doctrines for managing the inevitable battles
of the landlord/tenant relationship.

Here we are interested in conflicts that arise between neighboring property
owners. The collision is not within an ownership interest (as with cotenants)
but between such interests. My lifelong dream of operating the world's smokiest
factory may be incompatible with my neighbor's desire for odorless living. We
each own our respective land. What then? 

One solution is to engage in private governance. We might strike a deal, and the
law of servitudes lets us bind our successors in ownership to the arrangement.
Alternatively, the state might resolve our dispute via regulation---the
government may declare my facility illegal via zoning law or air quality
regulation, effectively picking a winner between competing interests. 

The law of \term{nuisance} takes a different tack. It also involves picking a
winner,
but turns the choice over to a court. The court's role, however, is not
explicitly regulatory. Rather, it is there to determine whether the
complained-of act is contrary to someone else's property rights. Stated another
way, if my factory is a nuisance, your property rights \textit{already} preclude
its operation. The nuisance action merely clarifies that I violated your
property rights (and that my property rights did not extend to the action in
question). In essence, the court is determining whether a boundary has been
crossed. But from another perspective, nuisance looks a lot like regulation. A
judicial regulator (rather than a politically accountable agency) takes a look
at the facts and decides whose interests ought to prevail. We might look at
nuisance questions from either view, which complicates the doctrine.

