\expected{spur-v-del-e-webb}

\item What if there had been no ``guilty'' developer like Del Webb? Why doesn't
the logic of the coming to a nuisance cases (quoted by the opinion) apply to
those who chose to purchase from Del Webb?

\item \textbf{Public vs. Private Nuisances.} \term[public nuisance]{Public
nuisances} involve
unreasonable interferences with rights held by the general public. Under the
\textsc{Restatement}, they arise when the complained-of actions threaten public
health, violate statutory law (including administrative regulations), or
otherwise have a significant effect on a public right. \textsc{Restatement
(Second) of Torts} \S~821B (1979). Unlike private nuisances, they do not require
an interference with the use of land. \textit{Id.} cmt. h. As \textit{Spur}
indicates, one may sue on a public nuisance if one alleges a ``special injury''
specific to the plaintiff and not shared by the public at large.

\item In addition to using ``coming to'' nuisance arguments, feedlot operators
may be specifically protected from nuisance suits. Some states explicitly
insulate agricultural operations from nuisance liability with ``right to farm''
legislation. Kan. St. Ann. 2-3201 provides:
\begin{quote}
It is the declared policy of this state to conserve and protect and encourage
the development and improvement of farmland for the production of food and other
agricultural products. The legislature finds that agricultural activities
conducted on farmland in areas in which nonagricultural uses have moved into
agricultural areas are often subjected to nuisance lawsuits, and that such suits
encourage and even force the premature removal of the lands from agricultural
uses. It is therefore the purpose of this act to provide agricultural activities
conducted on farmland protection from nuisance lawsuits.
\end{quote}
