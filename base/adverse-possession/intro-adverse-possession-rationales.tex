But why allow adverse possession? One court summarized the doctrine's history
and purposes as follows:
\begin{quotation}
\ldots a brief history of adverse possession may be of assistance. After first
using an amalgamation of Roman and Germanic doctrine, our English predecessors
in common law later settled upon statutes of limitation to effect adverse
possession. See Axel Teisen, \textit{Contributions of the Comparative Law
Bureau}, 3 A.B.A. J. 97, 126, 127, 134 (1917). In practice, the statutes
eliminated a rightful owner's ability to regain possession after the passing of
a certain number of years, thereby vesting de facto title in the adverse
possessor. For example, a 1623 statute of King James I restricted the right of
entry to recover possession of land to a period of twenty years. Essentially,
in England, the ``[o]riginal policy supporting the development of adverse
possession reflected society's unwillingness to take away a `right' which an
adverse possessor thought he had. Similarly, society felt the loss of an
unknown right by the title owner was minimal.'' William G. Ackerman \& Shane T.
Johnson, Comment, \textit{Outlaws of the Past: A Western Perspective on
Prescription and Adverse Possession}, 31 Land \& Water L. Rev. 79, 83
(1996).\ldots

In the United States, although the 1623 statute of King James I ``came some
years after the settling of Jamestown (the usual date fixed as the crystalizing
of the common law in America), its fiat is generally accepted as [our] common
law. Hence `adverse possession' for 20 years under the common law in this
country passes title to the adverse possessor with certain stated
qualifications.'' 10 \textit{Thompson on Real Property} {\S} 87.01 at 75.
Today, all fifty states have some statutory form of adverse possession \ldots.

\ldots. Courts and commentators generally ascribe to ``four traditional
justifications or clusters of justifications which support transferring the
entitlement to the [adverse possessor] after the statute of limitations runs:
the problem of lost evidence, the desirability of quieting titles, the interest
in discouraging sleeping owners, and the reliance interests of [adverse
possessors] and interested third persons.'' Thomas W. Merrill, \textit{Property
Rules, Liability Rules, and Adverse Possession}, 79 Nw. U. L. Rev. 1122, 1133
(1984). Effectively, our society has made a policy determination that ``all
things should be used according to their nature and purpose'' and when an
individual uses and preserves property ``for a certain length of time, [he] has
done a work beneficial to the community.'' Teisen, 3 A.B.A. J. at 127. For his
efforts, ``his reward is the conferring upon him of the title to the thing
used.'' Id. Esteemed jurist Oliver Wendell Holmes, Jr. went a step further than
Teisen, basing our society's tolerance of adverse possession on the ideal that
``[a] thing which you have enjoyed and used as your own for a long time,
whether property or an opinion, takes root in your being and cannot be torn
away without your resenting the act and trying to defend yourself, however you
came by it.'' O Centro Espirita Beneficente Uniao Do Vegetal v. Ashcroft, 389
F.3d 973, 1016 (10th Cir. 2004) (quoting Oliver Wendell Holmes, Jr.,
\textit{The Path of the Law}, 10 Harv. L. Rev. 457, 477 (1897)).

Regardless of how deeply the doctrine is engrained in our history, however,
courts have questioned ``whether the concept of adverse possession is as viable
as it once was, or whether the concept always squares with modern ideals in a
sophisticated, congested, peaceful society.'' \textit{Finley}, 160 Cal. Rptr.
at 427. Commentators have also opined that, along with the articulated benefits
of adverse possession, numerous disadvantages exist including the
``infringement of a landowner's rights, a decrease in value of the servient
estate, and the encouraged [over]exploitation and [over]development of land. In
addition, they \ldots [include] the generation of animosity between neighbors,
a source of damages to land or loss of land ownership, and the creation of
uncertainty for the landowner.''\edfootnote{The modifications
to the quotation from Ackerman are ours, not the court's.} Ackerman, 31 Land
\& Water L. Rev. at 92. In reality, ``[a]dverse possession `[i]s nothing more
than a person taking someone else's private property for his own private use.'
It is hard to imagine a notion more in contravention of the ideals set forth in
the U.S. Constitution protecting life, liberty and property.'' Ackerman, 31
Land \& Water L. Rev. at 94-95 (quoting 2 C.J.S. Adverse Possession {\S} 2
(1972)).

Although this Court duly recognizes its role as the judicial arm of government
tasked with applying the law, rather than making law, it is not without an
eyebrow raised at the ancient roots and arcane rationale of adverse possession
that we apply the doctrine to this modern property dispute.
\end{quotation}
\textit{Cahill v. Morrow}, 11 A.3d 82, 86-88 (R.I. 2011). Do you share the
court's skepticism? Consider the rationales discussed above against the
following case. 

\expectnext{tieu-v-morgan}
