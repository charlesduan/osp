\expected{tieu-v-morgan}
\expected{intro-adverse-possession}

\item Does the result in \textit{Tieu} jibe with the rationales for adverse
possession recited in the note preceding it? Which ones?
\textit{Cahill} suggests that these rationales are less relevant today than in
the past. Do you agree? Should the defendants in \textit{Tieu} have been
without recourse?

\item \textit{Tieu} involves an error in a conveyance. The parties' predecessors
in interest thought they had bargained to transfer land that they didn't. This
is a common source of adverse possession litigation. Other recurring fact
patterns include mistaken deed descriptions, surveying errors, and accidental
encroachments by neighbors. Adverse possession claims may also follow the
souring of relationships, perhaps between cotenants or one involving permissive
land use. None of these cases necessarily involve bad faith actors; although
the doctrine may indeed be applied in favor of the mere trespasser, depending
on the jurisdiction's interpretation of the state of mind required to satisfy
the ``hostility'' element. We will discuss this issue further below.

\item Title based on adverse possession is as good as any. To think through the
implications of that observation, imagine the following facts. Neighbor A
mistakenly builds a fence on her neighbor's land and gains title to the
enclosed land by adverse possession. Neighbor B then notices the encroachment
and demands that A move the fence. She agrees, but changes her mind two years
later and rebuilds it. B sues for trespass. Who wins?

\item \textbf{Open and notorious possession}.
\label{n:mannillo-v-gorski}
Whatever its merits, adverse
possession is strong medicine. The doctrine therefore provides safeguards to
prevent a title owner from losing her property without adequate notice by, for
example, requiring that the possession be open and notorious---it has to be the
kind of act that an owner would notice. 

But even overt acts may not be obvious threats to ownership rights. A fence on
someone else's property certainly seems open and notorious, but what if it is
just an inch or two over the border? What about the three-foot incursion at
issue in \textit{Tieu}? What if it had been built while the plaintiff was in
occupation of his lot? Do we expect owners to commission surveys anytime a
neighbor builds near the property line?

For some courts, the answer is no. \textit{Mannillo v. Gorski}, 255
A.2d 258, 264 (N.J. 1969), for example, holds that minor encroachments are not
open and notorious without actual knowledge on the part of the title owner. But
where would that leave an innocent encroacher, whose trespass may be costly to
remedy? In \textit{Mannillo}, the court balked at placing the trespasser, whose
steps and concrete walk extended 15 inches into the plaintiffs' property, at
her neighbor's mercy.
\begin{quote}
It is conceivable that the application of the foregoing rule may in some cases
result in undue hardship to the adverse possessor who under an innocent and
mistaken belief of title has undertaken an extensive improvement which to some
extent encroaches on an adjoining property. In that event \ldots equity may
furnish relief. Then, if the innocent trespasser of a small portion of land
adjoining a boundary line cannot without great expense remove or eliminate the
encroachment, or such removal or elimination is impractical or could be
accomplished only with great hardship, the true owner may be forced to convey
the land so occupied upon payment of the fair value thereof without regard to
whether the true owner had notice of the encroachment at its inception. Of
course, such a result should eventuate only under appropriate circumstances and
where no serious damage would be done to the remaining land as, for instance,
by rendering the balance of the parcel unusable or no longer capable of being
built upon by reason of zoning or other restrictions.
\end{quote}
\textit{Id.}\footnote{As \textit{Mannillo}'s resort to equity shows, adverse
possession is not the only way to address boundary disputes. Other options
include the equitable doctrine of acquiescence, \textit{see, e.g.}, \emph{Hamlin
v. Niedner}, 955 A.2d 251, 254 (Me. 2008) (``To prove that title or a boundary
line
is established by acquiescence, a plaintiff must prove four elements by clear
and convincing evidence: (1) possession up to a visible line marked clearly by
monuments, fences or the like; (2) actual or constructive notice of the
possession to the adjoining landowner; (3) conduct by the adjoining landowner
from which recognition and acquiescence, not induced by fraud or mistake, may
be fairly inferred; and (4) acquiescence for a long period of years[.]''); the
doctrine of agreed boundaries, \emph{Finley v. Yuba Cnty. Water Dist.}, 160 Cal. Rptr.
423, 428 (Cal. App. 1979); estoppel, \textit{see, e.g.}, \emph{Douglas v. Rowland},
540 S.W.2d 252 (Tenn. App. 1976), and laches. \textit{See generally} L. C.
Warden, \textit{Mandatory injunction to compel removal of encroachments by
adjoining landowner}, 28 A.L.R.2d 679 (Originally published in 1953)
(discussing factors influencing issuance of an injunction).\par Laches raises a
conceptual difficulty, as it seems to cover some of the same ground as adverse
possession. Laches is an equitable defense analogous to the legal defense
provided by a statute of limitations: if a plaintiff unreasonably delays in
bringing suit and the defendant is prejudiced by the delay, laches will bar the
suit as a matter of equity. But if an owner tries to recover land within the
limitations period, doesn't that imply that there has been no unreasonable
delay? \emph{Clanton v. Hathorn}, 600 So. 2d 963, 966 (Miss. 1992) (observing that the
adverse possession statute ``would seem to occupy the field''); \emph{Kelly v.
Valparaiso Realty Co.}, 197 So. 2d 35, 36 (Fla. Dist. Ct. App. 1967) (where
adverse possession was unavilable due to failure to pay taxes on the land ``we
do not feel that equity can be invoked to circumvent the statutory law of
adverse possession''); \textit{see generally} 27A \textsc{Am. Jur. 2d Equity}
\S~163
(``Only rarely should laches bar a case before the statute of limitations has
run.''). \textit{But see} \emph{Pufahl v. White}, No. 2050-S, 2002 WL 31357850,
at *1
(Del. Ch. Oct. 9, 2002) (although laches claim cannot lead to title, the
``laches defense may, however, be applicable to the plaintiffs' request to
enjoin the defendants to remove the encroachment'').} Is this result---a forced
transaction in which the innocent trespasser becomes the owner, but must
pay---the best accommodation of the relevant interests? If the true owner
wasn't on notice of the incursion, why can she be forced to surrender her land,
even for payment?  

\item \textbf{Adverse possession and the property owner}.
State-to-state variation about whether encroachments need to be obvious may
reflect a deeper question about the purpose of adverse possession. Some
authorities view the doctrine as having an object of punishing inattentive
owners who sleep on their rights. If so, then perhaps it makes sense to require
an incursion to be sufficiently obvious that a property owner would not need to
conduct a survey to determine the existence of a violation. 

But should sleeping owners be the target of the doctrine? Are property owners
who fail to assert their rights also less likely to develop their property (or
sell it to someone who will)? And if that is the underlying end, are there any
problems with using adverse possession doctrine as a means to it?

\item \textbf{Adverse possession as reward}. The reciprocal view---that
adverse possession exists to reward the possessors---has two flavors. One is
externally focused. The possessor, by putting the land to productive use, ``has
done a work beneficial to the community.'' Axel Teisen, 3 A.B.A. J. 97, 127
(1917). The other is more internal:
\begin{quote}
A thing which you have enjoyed and used as your own for a long time, whether
property or an opinion, takes root in your being and cannot be torn away
without your resenting the act and trying to defend yourself, however you came
by it. The law can ask no better justification than the deepest instincts of
man. It is only by way of reply to the suggestion that you are disappointing
the former owner, that you refer to his neglect having allowed the gradual
dissociation between himself and what he claims, and the gradual association of
it with another.
\end{quote}
Oliver Wendell Holmes,\textit{ The Path of the Law,} 10 \textsc{Harv. L. Rev.}
457, 477 (1897). Do either of these views resonate? What does this rationale
tell you about what the state of mind of the adverse possessor should be? 


\item \textbf{Third-party interests}.
\begin{quote}
The statute has not for its object to reward the diligent trespasser for his
wrong nor yet to penalize the negligent and dormant owner for sleeping upon his
rights; the great purpose is automatically to quiet all titles which are openly
and consistently asserted, to provide proof of meritorious titles, and correct
errors in conveyancing.
\end{quote}
Henry W. Ballantine, \textit{Title by Adverse Possession}, 32 \textsc{Harv. L.
Rev}. 135, 135 (1918) (footnotes omitted). By providing stability to existing
property arrangements after the passage of time, adverse possession simplifies
transactions by relieving purchasers and mortgagees of the risk that they are
dealing with title founded on a long-ago mistake or trespass. The doctrine is a
healing mechanism that realigns possession and paperwork when they've gotten
too badly out of sync. The benefit extends to the legal system as well by
relieving courts of the need to delve into the details of long-forgotten
events.

\item \textbf{Adverse possession's information function}. Adverse
possession also enables rights that exist as a matter of custom (e.g., ``the
Smiths always farm that strip of land'') to receive legal status. A banker in a
distant city may not understand (or trust) allocations based on local
understandings, but that doesn't matter if the claims are translated into
recordable title.\footnote{``Quiet title'' suits perform this function. They
are actions that establish the claimant's title to land and foreclose the
ability of others to contest it. Although quiet title suits are not necessary
to gain rights under adverse possession doctrine, they are very important to
adverse possessors. Do you see why? If you cannot answer the question, ask
yourself whether you would ever buy property from an adverse possessor.} The
land may now serve as the object of a sale or collateral for a loan for an
expanded audience, enhancing its value. Adverse possession's role in converting
informal understandings into formal rights illustrates law's ability to
facilitate the aggregation and dissemination of information across society. Can
you think of others?

\item \textbf{Tacking}. What happens if a series of possessors occupy a
property, but none of them are present long enough for the limitations period
to run? \textit{Tieu} notes in passing the concept of tacking, which enables a
succession of adverse possessors to collectively satisfy the statutory period.
The usual approach is to allow tacking so long as the successive possessors are
in ``privity'': a relationship in which the prior possessor knowingly and
intentionally transfers whatever interest she holds to the subsequent
possessor. \textit{See, e.g.}, \emph{Stump v. Whibco}, 715 A.2d 1006 (N.J.
Super. Ct.
App. 1998) (``Tacking is generally permitted unless it is shown that the
claimant's predecessor in title did not intend to convey the disputed
parcel.'') (citations and quotation omitted). So the clock continues to run if
one possessor sells or leases the occupied land, but there is no privity if one
trespasser wanders onto the lot after another leaves (or worse, dispossesses
the earlier trespasser by force). 

Recall the question of whether adverse possession doctrine is more properly
focused on rewarding deserving possessors or punishing inattentive owners. Does
the U.S. approach to tacking shed light on our answer? The English view is to
allow tacking without privity. \textit{Cf.} James Ames, \textsc{Lectures on
Legal History} 197 (1913) (``English lawyers regard not the merit of the
possessor, but the demerit of the one out of possession. The statutes of
limitation provide\ldots not that the adverse possessor shall acquire title,
but that the one who neglects for a given time to assert his right shall
thereafter not enforce it.'').

\item \textbf{Adverse possession and the environment}. An underlying
premise of the rationales discussed above is that land should be used. For an
argument that this tilt makes adverse possession doctrine environmentally
harmful, \textit{see} John G. Sprankling, \textit{An Environmental Critique of
Adverse Possession}, 79 \textsc{Cornell L. Rev}. 816, 840 (1994) (arguing that
``American adverse possession law is fundamentally hostile to the private
preservation of wild lands'' and proposing exemption to doctrine for privately
held wild lands).

