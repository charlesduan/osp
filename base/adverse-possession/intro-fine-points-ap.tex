\paragraph{Actual and Continuous Possession} Adverse possessors are
not required to live on the occupied property; what matters is acting like a
true owner would. That use, however, must be continuous, not sporadic.
\textit{Compare, e.g.}, \emph{Lobdell v. Smith}, 690 N.Y.S.2d 171, 173 (N.Y.
App. Div.
3d Dep't 1999) (although undeveloped land ``does not require the same quality
of possession as residential or arable land,'' no adverse possession where
claimant ``seldom visited the parcel except to occasionally pick berries or
hunt small game''), \textit{with} \emph{Nome 2000 v. Fagerstrom}, 799 P.2d 304,
310
(Alaska 1990) (claimants of a rural parcel suitable for recreational and
subsistence activities ``visited the property several times during the warmer
season to fish, gather berries, clean the premises, and play.\ldots That
others were free to pick berries and fish is consistent with the conduct of a
hospitable landowner, and undermines neither the continuity nor exclusivity of
their possession.''). Regular use of a summer home may constitute continuous
use. \textit{See, e.g.}, \emph{Nechow v. Brown}, 120 N.W.2d 251, 252 (Mich.
1963).

\paragraph{Color of title} Claim of title, an intent to use land as one's
own, is distinct from color of title, which describes taking possession under a
defective instrument (like a deed based on a mistaken land survey). States
often apply more lenient adverse possession standards to claims made under
color of title. \textit{Compare, e.g.}, \textsc{Fl. St.} \S~95.16, \textit{with
id.} \S~95.18. Why do you think that is? 

Entry under color of title may also affect the scope of the land treated as
occupied by the adverse possessor. 2 C.J.S. \textit{Adverse Possession} \S~252
(``Adverse possession under color of title ordinarily extends to the whole
tract described in the instrument constituting color of title.''). \textit{But
see} \emph{Wentworth v. Forne}, 137 So. 2d 166, 169 (Miss. 1962) (``In brief, when the
land involved is, in part, occupied by the real owner, the adverse possession,
even when this possessor has color of title, is confined to the area actually
possessed.'').

\paragraph{Adverse possession by and against the government}
Although government agencies may acquire title by adverse possession, the
general rule is that public property held for public use is not subject to the
doctrine. Why do you think that is?

\paragraph{Disabilities} The title owner of land may be subject to a
disability (e.g., status as a minor, mental incapacity) that may extend the
time to bring an ejectment action against an unlawful occupant. States
generally spell out such exceptions by statute.

\captionedgraphic[height=0.7\textheight,
width=\textwidth]{florida-form}{Florida's adverse possession form.}

\paragraph{A Moving Target} States vary their adverse possession rules to
take into account a variety of factors (e.g., claim under color of title,
payment of property taxes, enclosure or cultivation of land, etc.). These
factors may change with the times. In the aftermath of the financial crisis,
for example, reports of trespassers occupying foreclosed, vacant properties
with the goal of acquiring title via adverse possession prompted renewed
attention to the doctrine. Florida enacted legislation that requires those
seeking adverse possession without color of title to pay all outstanding taxes
on the property within one year of taking possession and disclose in writing
the possessor's identity, date of possession, and a description of the property
sufficient to enable the identification of the property in the public records.
Local officials are then required to make efforts to contact the record owner
of the property. \textsc{Fl. St.} \S~95.18. The form created under the statute
is
reprinted in Figure~\ref{f:florida-form}.
Are measures like these useful? Consider the problem of
``zombie foreclosures.'' A property may be vacant because the owners received a
notice of foreclosure and left. Sometimes the lenders never complete the
foreclosure process, perhaps to avoid the costs that come with ownership of the
property. Title therefore remains with the out-of-possession owners, who remain
responsible for taxes, association fees, and the like. What outcome should
adverse possession law seek to promote in such cases?

