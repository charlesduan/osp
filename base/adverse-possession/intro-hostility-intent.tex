\expected{tieu-v-morgan}


Adverse possession requires possession that is ``hostile'' and, often, ``under a
claim of right.'' Hostility is not animosity. ``Hostile possession can be
understood as possession that is opposed and antagonistic to all other claims,
and that conveys the clear message that the possessor intends to possess the
land as his or her own.'' 16 \textsc{Powell on Real Property} {\S} 91.01[2].
The requirement thus prevents permissive occupancy from ripening into
ownership; a lessor need not worry that the tenant will claim title by adverse
possession. \textit{See, e.g.}, Rise v. Steckel, 652 P.2d 364, 372 (1982)
(``[T]he ten-year statutory period for adverse possession did not begin to run
until defendant asserted to plaintiff that he was possessing the property in
his own right, rather than as a tenant at sufferance.''). A ``claim of right,''
sometimes called claim of title,\footnote{ Which is not the same thing as
``color of title,'' as discussed below.} means that the possessor is holding
the property as an owner would. This could be seen as synonymous with the
hostility requirement, but not all jurisdictions treat the concept this way.
The Powell treatise states that the predominant view in the United States is
that good faith is not required for adverse possession, 16 \textsc{Powell} {\S}
91.01[2], but as you may have already noticed in the \textit{Tieu} case above,
intent often matters.

