\item \textbf{Doctrine v.~practice}. Richard Helmholz has argued that
though adverse possession doctrine generally does not require the adverse
possessor to plead good faith, judicial practice is to disfavor those who know
they are trespassing compared to those acting out of a good faith mistake.
Richard H. Helmholz, \textit{Adverse Possession and Subjective Intent}, 61
\textsc{Wash. U. L.Q.} 331, 332 (1983). Is \textit{Cahill} an example of this
dynamic? 

\expected{tieu-v-morgan}

In recent decades, state legislatures have increasingly demanded good faith on
the part of the possessor (the Oregon statute in \textit{Tieu} requiring honest
belief in ownership, for example, was passed in 1989). \textit{See} 16
\textsc{Powell on Real Property} {\S} 91.05 (collecting examples).

\item Should good faith be required? And if so, what is good faith? Is it an
honest belief about the facts on the ground (e.g., whether the fence builder is
correct that his fence is on the right side of the boundary line)? Or is it an
attitude about one's potential adversary (a willingness to move the fence if
wrong)? Either view creates evidentiary difficulties. 

Even when good faith is not part of the analysis as a formal matter, Helmholz
argues that judges and juries often cannot help but ``prefer the claims of an
honest man over those of a dishonest man.'' Helmholz, \textit{supra}, at 358.
Might this be a satisfactory middle ground? Are there advantages to having
courts officially ignore intent while applying a de facto bar to the bad faith
possessor when there is evidence of dishonesty? Or is it problematic to have
legal practice depart from official doctrine? 

Perhaps another way to reconcile the benefits of adverse possession with the
distaste for bad faith possessors would be to allow dishonest possessors to
keep the land, but pay for the privilege. Thomas W. Merrill, \textit{Property
Rules, Liability Rules, and Adverse Possession}, 79 \textsc{Nw. U. L. Rev}.
1122, 1126 (1984) (suggesting ``requiring indemnification only in those cases
where the [true owner] can show that the [adverse possessor] acted in bad
faith.''). As Merrill notes, a California appellate court required
such payment in a case concerning a prescriptive easement (which is similar to
adverse possession except that it concerns the \textit{right to use} someone
else's land rather than its ownership), only to be overturned by the state
supreme court. \textit{Id.} (discussing \emph{Warsaw v. Chicago Metallic
Ceilings, Inc.}, 676 P.2d 584 (Cal. 1984)). The proposal may remind you of the
\textit{Mannillo} case discussed above. How does it differ? 

\item A minority of states\having{dombkowski-v-ferland}{, as \emph{Dombkowski}
indicates,}{, as we will see in \emph{Dombkowski v. Ferland},}{} require adverse
possessors
to prove their subjective intent to take the land without regard to the
existence of other ownership interests. This is sometimes referred to as the
``aggressive trespass'' standard: ``I thought I did not own it [and intended to
take it].'' Margaret Jane Radin, \textit{Time, Possession, and Alienation}, 64
\textsc{Wash. U. L.Q.} 739, 746 (1986) (brackets in the original). Is there a
reason to prefer it? Lee Anne Fennell argues for a knowing trespass requirement
that requires the adverse possessor to document her knowledge:
\begin{quote}
[A] documented knowledge requirement facilitates rather than punishes efforts at
consensual dealmaking. One of the most definitive ways of establishing that a
possessor knew she was not the owner of the disputed land is to produce
evidence of her purchase offer to the record owner. Currently, such an offer
often destroys one's chance at adverse possession because it shows one is
acting in bad faith if one later trespasses; one does far better to remain in
ignorance (or pretend to) and never broach the matter with the record owner.
Under my proposal, such offers would go from being fatal in a later adverse
possession action to being practically a prerequisite. As a result, it would be
much more likely that any resulting adverse possession claim will occur only
where a market transaction is unavailable. A documented knowledge requirement
would also reduce litigation costs and increase the certainty of land holdings.
Actions or records establishing that the trespass was known at the time of
entry, necessary if the possessor ever wishes to gain title under my approach,
would serve to streamline trespass actions that occur before the statute has
run. Moreover, an approach that refuses to reward innocent mistakes would be
expected to reduce mistake-making.
\end{quote}
Lee Anne Fennell, \textit{Efficient Trespass: The Case for ``Bad Faith'' Adverse
Possession}, 100 \textsc{Nw. U. L. Rev}. 1037, 1041-44 (2006) (footnotes
omitted). One's position on these matters may depend on which scenarios one
believes are most common in adverse possession cases and adjust the state of
mind required to include or exclude them accordingly. Should the state of mind
required depend on the context? A state might, for example, require good faith
for encroachments, but bad faith or color of title if the possessor seeks to
own the parcel as a whole. Is this a good idea? 

