Few doctrines taught in the first year of law school make a worse first
impression than adverse possession. Adverse possession enables a non-owner to
gain title to land (or personal property, but we will focus here on land) after
the expiration of the statute of limitations for the owner to recover
possession. That sounds bad, and the thought of ``squatters'' becoming owners
gets its share of bad press. But historically the doctrine has performed, and
continues to serve, important functions. 

The basic requirements, if not their wording and application, are common from
state to state. As one treatise summarizes, an adverse possessor must prove
possession that is:
\begin{itemize}
\item hostile (perhaps under a claim of right);

\item exclusive;

\item open and notorious;

\item actual; and

\item continuous for the requisite statutory period.
\end{itemize}
16 \textsc{Powell on Real Property} {\S} 91.01. States routinely add to the
list. California law, for example, requires that 
\begin{quote}
the claimant must prove: (1) possession under claim of right or color of title;
(2) actual, open, and notorious occupation of the premises constituting
reasonable notice to the true owner; (3) possession which is adverse and
hostile to the true owner; (4) continuous possession for at least five years;
and (5) payment of all taxes assessed against the property during the five-year
period.
\end{quote}
\emph{Main St. Plaza v. Cartwright \& Main}, LLC, 124 Cal. Rptr. 3d 170, 178
(Cal. App. 2011) (citations and quotations omitted).


