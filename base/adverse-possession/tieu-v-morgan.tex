\reading{Tieu v. Morgan}
\readingcite{265 P.3d 98 (Or. Ct. App. 2011)}

\opinion \textsc{Hadlock}, J.

The parties dispute ownership of a strip of land that runs parallel to
defendants' driveway. Plaintiff, who owns residential property adjoining that
strip of land, filed suit seeking (1) a declaration that he owns the disputed
strip and (2) an injunction prohibiting defendants from trespassing on that
property. Defendants counterclaimed, asserting that they acquired the disputed
strip through adverse possession, and subsequently moved for summary judgment
on that counterclaim. The trial court granted defendants' motion and entered a
judgment declaring that defendants had acquired the strip through adverse
possession. Plaintiff appeals, and we affirm.\ldots

\heregraphic{tieu-v-morgan}

The two parcels subject to this appeal are adjoining residential tax lots in a
Portland subdivision. Tax lot 3100 is rectangular, with its north side fronting
Southeast Boise Street. Tax lot 3200 is a flag lot that is situated largely
south of lot 3100; its driveway (the ``flagpole'') runs north from the main
portion of the lot (the ``flag'') to Southeast Boise Street, parallel to the
eastern edge of lot 3100. The disputed three-foot-wide strip lies between lot
3200's driveway and lot 3100. Defendants own lot 3200. Plaintiff owns lot 3100
and also is the record owner of the disputed strip. 

A north-south stretch of fence on plaintiff's property runs along the western
boundary of the disputed strip, parallel to defendants' driveway. The fence
starts roughly halfway down the driveway from Southeast Boise Street, running
south, then turns 45 degrees to the southwest, cutting off the southeast corner
of lot 3100, then makes another 45-degree turn before continuing west, roughly
following the east-west boundary between lots 3100 and 3200. The diagonal
portion of the fence that cuts the corner of lot 3100 includes a gate wide
enough to accommodate a boat trailer. As noted, the disputed three-foot-wide
strip lies between defendants' driveway and the north-south fence on lot 3100;
its practical effect is to widen the ``flagpole'' portion of lot 3200. 

The fencing that separates the two properties has existed for decades. As of
1984, the two lots were owned by Robert Stevens, who installed most of the
fencing that year, including about half of the north-south stretch located west
of lot 3200's driveway. In 1994, Robert Stevens sold lot 3200 to his son, James
Stevens, believing that the deed he conveyed to James included all property on
the east side of a north-south line defined by that portion of the fence,
\textit{i.e.}, the disputed strip. Although he never specifically discussed the
issue with his father, James also believed that his purchase of the
flag lot included the disputed strip along his driveway. James explained that
he had ``no reason to know---to think [that the fence] would be in the wrong
location.''

During the four years that James owned the flag lot, he granted Robert
permission to occasionally use James's driveway and the disputed strip, so that
Robert could drive a large vehicle and boat trailer through the diagonal gate
into Robert's back yard. In 1996, James installed a sewer line in the center of
the disputed strip, running all the way from Southeast Boise Street to the
house on lot 3200. When James later put lot 3200 on the market, he advertised
it as having a ``fully fenced yard,'' based on his belief that his ownership
included the disputed strip.

James sold lot 3200 to defendants in 1998. The lot was not surveyed in
conjunction with that sale; nor did the parties to the sale discuss the lot's
recorded boundaries, review paperwork or maps, or perform any investigation
specifically related to that subject.

Defendants have made use of the disputed strip since they purchased lot 3200.
Defendant Francine Morgan runs a daycare business from her home, and parents
regularly use the disputed strip when dropping off and picking up their
children. In 1999, defendants extended the fence paralleling the strip north by
roughly 40 feet, choosing not to extend the fence all the way to Southeast
Boise Street after Robert suggested that they leave that area unfenced to
accommodate maneuvering large vehicles in and out of their driveways.
Defendants have laid gravel and bark dust on the disputed strip a number of
times and have maintained the fence by replacing posts and fence boards. While
Robert still owned lot 3100, he specifically asked defendants' permission each
time he wanted to use the disputed strip to access or move his boat, and
defendants granted that permission.

Plaintiff bought lot 3100 from Robert in early 2006. Before purchasing the
property, plaintiff had it surveyed and learned that the north-south fence was
not located on the deeded boundary between lots 3100 and 3200. A survey pin
marking the recorded boundary was placed at that time. Plaintiff claims that he
told defendant Francine Morgan soon after the survey was completed that he
planned to move the fence to the deeded property line within two years.
According to plaintiff, Francine neither disputed plaintiff's right to move the
fence nor claimed ownership of land between the survey marker and the fence.
Defendants deny that such a conversation occurred.

In 2008, plaintiff attempted to remove the north-south portion of the fence.
After defendants protested, plaintiff initiated this action, seeking a
declaration that he owned the disputed strip. As noted, defendants asserted in
a counterclaim that they had acquired the strip through adverse possession. The
trial court ultimately granted summary judgment to defendants, ruling that the
undisputed facts established that defendants had acquired the disputed strip
through adverse possession.\ldots

ORS 105.620 codifies the common-law elements of adverse possession, requiring a
claimant to prove by clear and convincing evidence that the claimant or the
claimant's predecessors in interest maintained actual, open, notorious,
exclusive, hostile, and continuous possession of the property for ten years. In
addition to those common-law elements, the statute also requires the claimant
to have had an honest belief of actual ownership when he or she entered into
possession of the property.

Plaintiff makes arguments related to each of the statutory elements, first
claiming that defendants did not establish actual, open, notorious, exclusive,
or continuous possession of the entire disputed strip. We recently summarized
what proof is required to satisfy those elements of an adverse-possession
claim:
\begin{quote}
The element of actual use is satisfied if a claimant established a use of
the land that would be made by an owner of the same type of land, taking into
account the uses for which the land is suited. To establish a use that is open
and notorious, plaintiffs must prove that their possession is of such a
character as to afford the owner the means of knowing it, and of the claim. The
exclusivity of the use also depends on how a reasonable owner would or would
not share the property with others in like circumstances. A use is continuous
if it is constant and not intermittent. The required constancy of use,
again, is determined by the kind of use that would be expected of such land.
\end{quote}
\textit{Stiles v. Godsey}, 233 Or. App. 119, 126, 225 P.3d 81 (2009) (internal
quotations and citations omitted).

Here, the land in question is a three-foot-wide strip, covered mostly with
gravel or bark dust, adjacent to a narrow driveway. Defendants and their
predecessor have used the strip as an extension of that driveway since 1994,
both to accommodate wide vehicles and to provide additional loading room for
defendant Francine Morgan's daycare clients. That use is consistent with
ownership and with the land's character. Moreover, that use was ``open'' and
``notorious,'' particularly when considered together with James's act of
locating his sewer line on the strip and, later, defendants' maintenance of and
improvements to the fence. Finally, defendants and their predecessor used the
strip continuously from 1994 (when James bought the lot) to at least 2006 (when
plaintiff bought lot 3100 from Robert), \textit{i.e.}, for longer than the
statutory 10-year adverse-possession period. Thus, the undisputed facts
establish defendants' actual, open, notorious, exclusive, and continuous use of
the property.

Plaintiff's contrary argument rests on the fact that the disputed strip is not
completely separated from his residential lot by a fence; he emphasizes that
the fence at issue does not extend all the way to Southeast Boise Street, but
starts partway down the driveway.\ldots Here, even though the fence does not
extend to the street, it adequately defines the entire disputed strip,
indicating that it is separate from the land that abuts it to the west.

Plaintiff also contends that defendants' use of the disputed strip was not
``exclusive'' because Robert sometimes used the property even after the fence
was built. But adverse-possession claimants are allowed the freedom to allow
others to occasionally use their property, in the manner that neighbors are
wont to do, without thereby abandoning their claim. In this case, Robert asked
permission of defendants and their predecessors each time that he used the
disputed strip; that permissive use was consistent with defendants' ownership
of the land and does not defeat their claim to it.

We also reject plaintiff's argument that defendants' use of the disputed strip
was not ``hostile'' because, he claims, defendants had a conscious doubt
regarding the property line. Under ORS 105.620(2)(a), a claimant ``maintains
`hostile possession' of property if the possession is under claim of right or
with color of title.'' A ``claim of right'' may be established through proof of
an honest but mistaken belief of ownership, resulting, for example, from a
mistake as to the correct location of a boundary. The mistaken belief must be a
``pure'' mistake, however, and not one based upon ``conscious doubt'' about the
true boundary. Furthermore, ORS 105.620(1)(b) requires that the claimants (or
their predecessors) have had an ``honest belief'' of actual ownership that (1)
continued through the vesting period, (2) had an objective basis, and (3) was
reasonable under the circumstances.

In \textit{Mid-Valley Resources, Inc. v. Engelson}, 170 Or. App. 255 (2000), we
concluded that the defendants had failed to establish pure mistake about the
location of a boundary line because one of the defendants had a conscious doubt
on that subject. That \textit{Mid-Valley} defendant had testified that she had
not known where the property line was when she was a child, and she still did
not know at the time of trial whether a particular fence was located on that
boundary. That defendant's uncertainty about the property line's location
defeated the defendants' adverse-possession claim.

Here, by contrast, the undisputed evidence clearly establishes that defendants
and their predecessor, James, always believed that the fence marked the
north-south line between lots 3200 and 3100. James assumed when he bought lot
3200 in 1994 that the fence was on the property line, and he perpetuated that
belief in defendants by telling them, when they bought the property, that it
was ``fully fenced.'' Robert, then the record owner of the disputed strip,
confirmed those mistaken beliefs when he did not object to installation of the
sewer line, to defendants' use of the strip, or to defendants' extension of the
fence. No evidence in the record supports plaintiff's assertion that defendants
had a ``conscious doubt'' about whether the fence was actually located on the
line separating their property from plaintiff's. Defendants did suggest in
their depositions that they had not given much thought to the property line's
location until the dispute arose with plaintiff. Read in context, however,
those statements simply confirm defendants' \textit{certainty} that the
property line was the same as the fence line; the statements do not indicate
that defendants had any conscious doubt as to the boundary's location.

Moreover, no evidence calls into question the reasonableness of defendants'
belief that they owned the disputed strip. That strip of land is small in
relation to the size of lots 3200 and 3100, it regularly has been used as an
extension to the width of an existing driveway, it is well suited to that
purpose, and it is partly fenced off from plaintiff's property. Under the
circumstances, defendants' belief that they owned the disputed strip was
reasonable.

In sum, the undisputed evidence establishes clearly and convincingly that
defendants and their predecessor, James, had an ``honest belief'' that the
disputed strip was part of lot 3200 and that they continuously maintained
actual, open, notorious, exclusive, and hostile possession of that strip for
well over 10 years, from 1994 at least until plaintiff bought lot 3100 in
2006.\readingfootnote{6}{We reject plaintiff's argument that
defendants cannot satisfy the 10-year adverse-possession period by tacking
their possession to that of James. An adverse-possession claimant may tack his
possessory interests to those of a predecessor ``if there is evidence that the
predecessor intended to transfer whatever adverse possessory rights he or she
may have acquired.'' \emph{Fitts v. Case}, 243 Or. App. 543, 549, 267 P.3d 160
(2011).
Here, James clearly intended his transfer of lot 3200 to defendants to include
the disputed strip, given his belief that the fence marked the boundary line
and his advertisement of lot 3200 as ``fully fenced.''} We conclude
that defendants' adverse-possession claim to the disputed strip vested in 2004,
giving them title and extinguishing any claim that plaintiff might otherwise
have had to that land. 

