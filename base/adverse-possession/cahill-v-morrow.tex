\reading{Cahill v. Morrow}
\readingcite{11 A.3d 82 (R.I. 2011)}

\opinion \textsc{Indeglia}, J.

The property in dispute is located on Gooseberry Road in the Snug Harbor section
of South Kingstown, Rhode Island. Identified as lot 19 on assessor's plat 88-1,
the land is sandwiched between lot 20, currently owned by Cahill, and lot 18,
formerly coowned by members of the Morrow family. Morrow is the record owner of
the subject property, lot 19.

In 1969, Morrow's husband, George Morrow, purchased lot 19, and the same year
George and his brothers jointly purchased lot 18. At the time of lot 19's
purchase, it was largely undeveloped, marked only by a preexisting clothesline,
grass, and trees. Since that time, the Morrows have not improved or maintained
lot 19, but have paid all property taxes assessed to it. As such, instead of
vacationing on their lot 19, the Morrows annually spent two weeks in the summer
at the cottages on the adjacent lot 18. During these vacations, the Morrow
children and their cousins played on lot 19's grassy area. Around 1985, the
Morrows ceased summering on Gooseberry
Road,\readingfootnote{3}{In 1991, George Morrow and his
joint-owner brothers sold lot 18.} but continued to return at least once a year
to view the lot. Morrow stopped visiting lot 19 in October 2002, after her
husband became ill, and she did not return again until July 2006.

In 1971, two years after George Morrow purchased lot 19, Cahill's mother bought
the land and house designated as lot 20 as a summer residence. Between 1971 and
1975, Cahill and her brother did some work on lot 19. They occasionally cut the
grass, placed furniture, and planted trees and flowers on it.

Cahill's mother passed away in 1975, and in 1977, after purchasing her siblings'
shares, Cahill became the sole record owner of the lot 20 property. Once she
became lot 20's owner, Cahill began living in the house year-round. From that
time through 1991, she and her boyfriend, James M. Cronin, testified that they
continued to mow lot 19's grass on occasion. In addition, she hung clothing on
the clothesline, attached flags to the clothesline pole, used the picnic table,
positioned a bird bath and feeder, and planted more flowers and trees. Cahill
placed Adirondack chairs on lot 19 and eventually replaced the clothesline and
picnic table. In 1987, Cahill held the first annual ``cousins' party'' allowing
her relatives free rein with respect to her property and lot 19 for playing,
sitting, and car parking. She also entertained friends and family on lot 19
during other summer days. Mary Frances McGinn, Cahill's cousin, likewise
recalled that lot 19 was occupied by Cahill kindred during various family
functions throughout this time period. Cahill admitted that she never objected
to neighborhood children using lot 19, however.

During the period of 1991 through 1997, Cahill testified that she planted more
flowers and trees, in addition to cutting the grass occasionally. Cahill also
stored her gas grill and yard furniture on the lot and had her brother stack
lobster pots for decorative purposes. In 1991 or 1992, she began hosting the
annual ``Cane Berry Blossom Festival,'' another outdoor event that used both
her lot and lot 19 as the party venue. Like the other gatherings, the festival
always took place on a day during a warm-weather month. In 1997 or 1998, she
installed a wooden border around the flower beds.

On July 22, 1997, Cahill wrote to George Morrow expressing an interest in
obtaining title to lot 19. In the 1997 letter, Cahill stated: ``I am interested
in learning if your narrow strip of property is available for sale. If so, I
would be interested in discussing purchasing it from you.'' Cahill continued:
``If there is a possibility that you would like to sell it, could you please
either call me or send me a note?'' Cahill did not receive a response.

In the ``late 1990s,'' though Cahill is unclear whether this occurred before or
after the 1997 letter, a nearby marina sought permission to construct and
elevate its property. Cahill attended the related zoning board hearings and
expressed her concerns about increased flooding on lot 19 due to the marina
elevation. She succeeded in having the marina developer grade part of lot 19 to
alleviate flooding. Additionally, Cahill instituted her own trench and culvert
drainage measures to divert water off of lot 19 and then reseeded the graded
area. By Cahill's own admission, however, her trenching and reseeding work
occurred in 1999 or 2000.

Subsequent to 2001, the new owners of lot
18\readingfootnote{5}{In approximately 2001, new owners
purchased lot 18 from the Morrow brothers' successor.} stored their boat on lot
19 and planted their own flowers and small trees on the property. In 2002, when
the town (with approval from George Morrow) erected a stone wall and laid a
sidewalk on the Gooseberry Road border of lot 19, Cahill loamed and planted
grass on that portion of the lot. Also in 2002, Cahill asked Morrow's two
sisters on separate occasions whether George Morrow would be interested in
selling lot 19. The Morrows gave no response to her 2002 inquiries. In 2003,
George Morrow passed away.

After making her third inquiry concerning the purchase of lot 19 in 2002, Cahill
testified, she continued using the property in a fashion similar to her prior
practice until December 2005, when she noticed heavy-machinery tire marks and
test pits on the land. Thereafter, she retained counsel and authorized her
attorney to send a letter on January 10, 2006 to Morrow indicating her adverse
possession claim to a ``20-foot strip of land on the northerly boundary'' of
lot 19. According to a survey of the disputed property, however, the width of
lot 19 from the northerly boundary (adjacent to Cahill's property) to lot 18 is
49.97 feet and therefore, more than double what Cahill originally claimed in
this letter. Nonetheless, on April 25, 2006, Cahill instituted a civil action
requesting a declaration that based on her ``uninterrupted, quiet, peaceful and
actual seisin and possession'' ``for a period greater than 10 years,'' she was
the true owner of lot 19 in its entirety. On July 25, 2007, the trial justice
agreed that Cahill had proved adverse possession under G.L. 1956 {\S} 34-7-1
and vested in her the fee simple title to lot 19.\ldots

In Rhode Island, obtaining title by adverse possession requires actual, open,
notorious, hostile, continuous, and exclusive use of property under a claim of
right for at least a period of ten years.

Here, the trial justice recited the proper standard of proof for adverse
possession and then found that Cahill had
\begin{quote}
``met her burden of establishing all of the elements of an adverse possession
claim to lot 19 by her and her mother's continuous and uninterrupted use of the
parcel for well in excess of ten years. She maintained the property, planted
and improved the property with shrubs, trees, and other plantings, sought
drainage control measures, and used the property as if it were her own since
1971. She established that use not only by her own testimony, but as
corroborated by other witnesses, photographs, and expert testimony relative to
the interpretation of aerial photographs.''
\end{quote}

At trial, as here on appeal, Morrow argued that Cahill's offers to purchase the
property invalidated her claim of right and the element of hostile possession.
To dispose of that issue, the trial justice determined that ``even assuming
that [Cahill's] inquiry is circumstantial evidence of her knowledge that George
Morrow, and subsequently Margaret [Morrow], were the legal title holders of
[lot] 19, that does not destroy the viability of this adverse possession
claim.'' The trial justice relied upon our opinion in \textit{Tavares}, 814
A.2d at 350, to support his conclusion. Recalling that this Court stated in
\textit{Tavares} that ``even when the claimants know they are nothing more than
black-hearted trespassers, they can still adversely possess the property in
question under a claim [of] right to do so if they use it openly, notoriously,
and in a manner that is adverse to the true owner's rights for the requisite
ten-year period,'' the trial justice found that Cahill's outward
acknowledgement of Morrow's record title did not alone ``negate her claim of
right.'' He further found that ``even if somehow the expression of interest in
purchasing lot 19, made initially in 1997, stopped the running of the
ten[-]year period under * * * {\S} 34-7-1, the evidence was overwhelming that
[Cahill] and her predecessor in title had commenced the requisite ten-year
period beginning in 1971.''

\readinghead{C.}

On appeal, Morrow challenges the trial justice's legal conclusion that Cahill's
offers to purchase lot 19 did not extinguish her claim of right, hostile
possession, and ultimately, the vesting of her title by adverse possession.
Morrow also contends that the trial justice erred in finding that Cahill's
testimonial and demonstrative evidence was sufficient to prove adverse
possession under the clear and convincing burden of proof standard. We agree
that as a matter of law the trial justice failed to consider the impact of
Cahill's offers to purchase on the prior twenty-six years of her lot 19 use. As
a result, we hold that this failure also affects his factual determinations.

\readinghead{1. 1997 Offer-to-Purchase Letter}

In \textit{Tavares}, this Court explained that ``requir[ing] adverse possession
under a claim of right is the same as requiring hostility, in that both terms
simply indicate that the claimant is holding the property with an intent that
is adverse to the interests of the true owner.'' \textit{Tavares}, 814 A.2d at
351 (quoting 16 Powell on Real Property, {\S} 91.05[1] at 91-28 (2000)).
``Thus, [we said] a claim of right may be proven through evidence of open,
visible acts or declarations, accompanied by use of the property in an
objectively observable manner that is inconsistent with the rights of the
record owner.'' Here, the first issue on appeal is how an offer to purchase has
an impact on these elements.\ldots

\ldots. [I]n \textit{Tavares}, 814 A.2d at 351, with regard to ``establishing
hostility and possession under a claim of right,'' we explained that ``the
pertinent inquiry centers on the claimants' \textit{objective manifestations}
of adverse use rather than on the claimants' \textit{knowledge} that they
lacked colorable legal title.'' (Emphases added.) Essentially, \textit{Tavares}
turned on the difference between the adverse possession claimant's
``knowledge'' regarding the owner's title and his ``objective manifestations''
thereof. In that case, the adverse-possession claimant surveyed his land and
discovered ``that he did not hold title to the parcels in question.'' After
such enlightenment, however, the claimant objectively manifested his claim of
ownership to the parcels by ``posting no-trespass signs, constructing stone
walls, improving drainage, and wood cutting.'' This Court explained that simply
having knowledge that he was not the title owner of the parcels was not enough
to destroy his claim of right given his objective, adverse manifestations
otherwise. In fact, we went so far as to state that ``even when claimants know
that they are nothing more than black-hearted trespassers, they can still
adversely possess the property in question under a claim of right to do so if
they use it openly, notoriously, and in a manner that is adverse to the true
owner's rights for the requisite ten-year period.'' This statement is legally
correct considering that adverse possession does not require the claimant to
make ``a good faith mistake that he or she had legal title to the land.'' 16
Powell on Real Property {\S} 91.05[2] at 91-23. However, to the extent that
\textit{Tavares}'s reference to ``black-hearted trespassers'' suggests that
this Court endorses an invade-and-conquer mentality in modern property law, we
dutifully excise that sentiment from our jurisprudence.

In the case before this Court, Cahill went beyond mere knowledge that she was
not the record owner by sending the offer-to-purchase letter. As distinguished
from the \textit{Tavares} claimant who did not communicate his survey findings
with anyone, Cahill's letter objectively declared the superiority of George
Morrow's title to the record owner himself. \textit{See also} Shanks v.
Collins, 1989 OK 115, 782 P.2d 1352, 1355 (Okla. 1989) (``A recognition by an
adverse possessor that legal title lies in another serves to break the
essential element of continuity of possession.'').

In the face of this precedent, Cahill contends that the trial justice accurately
applied the law by finding that an offer to purchase does not automatically
negate a claim of right in the property. While we agree that this proposition
is correct with respect to offers made in an effort to make peace in an ongoing
dispute, we disagree that this proposition applies in situations, as here,
where no preexisting ownership dispute is evident.\ldots Her offer was not an
olive branch meant to put an end to pending litigation with the Morrows.
Rather, it was a clear declaration that Cahill ``wanted title to the property''
from the record owner. By doing so, she necessarily acknowledged that her
interest in lot 19 was subservient to George Morrow's.\ldots

Accordingly, the trial justice erred by considering any incidents of ownership
exhibited by Cahill after the 1997 letter to George Morrow interrupted her
claim.\ldots

\readinghead{2. The Impact of Cahill's Offer to Purchase on her Pre-1997
Adverse-Possession Claim}

Furthermore, we also conclude that the trial justice should not have assumed
that even if Cahill's ``inquiry is circumstantial evidence of her knowledge
that George Morrow, and subsequently [Morrow], were the legal title holders of
[lot] 19, that does not destroy the viability of this adverse possession
claim.'' We agree that an offer to purchase does not automatically invalidate a
claim already vested by statute, but we nonetheless hold that the objective
manifestations that another has superior title, made after the statutory period
and not made to settle an ongoing dispute, are poignantly relevant to the
ultimate determination of claim of right and hostile possession during the
statutory period.\ldots

\readinghead{3. Questions of Fact Remain}

Despite the significant deference afforded to the trial justice's findings of
fact, such findings are not unassailable. Here, we find clear error in the
trial justice's conclusion that ``even if somehow the expression of interest in
purchasing [lot] 19, made initially in 1997, stopped the running of the
ten[-]year period *~*~* the evidence was overwhelming that [Cahill] and her
predecessor in title had commenced the requisite ten-year period beginning in
1971.'' Given our opinion that some of Cahill's lot 19 activities cannot be
considered because of the time frame of their occurrence, we disagree that the
trial record can be classified as presenting ``overwhelming'' evidence of
adverse possession.

\ldots. On remand, the trial justice is directed to limit his consideration to
pre-1997 events and make specific determinations whether Cahill's intermittent
flower and tree planting, flag flying, clothesline replacing, lawn chair and
beach-paraphernalia storing, and annual party hosting are adequate.
Furthermore, given our ruling today, the trial court must evaluate the nature
of Cahill's and her predecessor's twenty-six-year acts of possession in the
harsh light of the fact that Cahill openly manifested the existence of George
Morrow's superior title on three occasions.\ldots

FLAHERTY, J., dissenting. 

\ldots. Simply put, I do not agree that the correspondence between plaintiff
and defendant in which plaintiff offers to purchase defendant's interest in lot
19 is the smoking gun the majority perceives it to be. As is clear from a fair
reading of plaintiff's testimony, she believed that she owned the property as a
result of her longtime use of and dominion over it. But her testimony also
demonstrates that she drew a crisp distinction between whatever ownership
rights she may have acquired and record title, which she recognized continued
to reside in the Morrows\ldots. Even if that letter were as significant as the
majority contends, there is no doubt that it was sent after the statutory
period had run. It is beyond dispute that plaintiff's correspondence could not
serve to divest her of title if she had already acquired it by adverse
possession\ldots. There certainly was credible evidence for the trial justice
to find that plaintiff had used the property as her own for well over twenty
years before she corresponded with Mr. Morrow in 1997.\ldots

