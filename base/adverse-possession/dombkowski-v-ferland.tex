\reading{Dombkowski v. Ferland}
\readingcite{893 A.2d 599 (Me. 2006)}

\opinion \textsc{Dana}, J.

\ldots. Although ``some courts and commentators fail to distinguish between the
elements of \textit{hostility} and \textit{claim of} \textit{right,} or simply
consider \textit{hostility} to be a subset of the \textit{claim of right}
requirement[,] \textit{see, e.g., Johnson v. Stanley,} 96 N.C. App. 72, 384
S.E.2d 577, 579 (1989)[,]  \ldots under Maine law, the two elements are
distinct.'' \textit{Striefel,} 1999 ME 111, P13 n.7, 733 A.2d at 991.

``\,`Hostile' simply means that the possessor does not have the true owner's
permission to be on the land, and has nothing to do with demonstrating a heated
controversy or a manifestation of ill will, or that the claimant was in any
sense an enemy of the owner of the servient estate.'' \textit{Id.} P13, 733
A.2d at 991 (quotation marks and citation omitted). ``Permission negates the
element of hostility, and precludes the acquisition of title by adverse
possession.'' \textit{Id.} ``\,`Under a claim of right' means that the claimant
is in possession as owner, with intent to claim the land as [its] own, and not
in recognition of or subordination to [the] record title owner.'' \textit{Id.}
P14, 733 A.2d at 991 (quotation marks omitted).

Under Maine's common law, as part of the claim of right element, we have
historically examined the subjective intentions of the person claiming
adverse possession. \textit{See Preble v. Maine C. R. Co.,} 85 Me. 260, 264, 27
A. 149, 150 (1893). Under this approach, which is considered the minority rule
in the country, ``one who by mistake occupies \ldots land not covered by his
deed with no intention to claim title beyond his actual boundary wherever that
may be, does not thereby acquire title by adverse possession to land beyond the
true line.'' \textit{Preble}, 85 Me. at 264, 27 A. at 150; \textit{see also
McMullen}, 483 A.2d at 700 (``[If] the occupier intend[s] to hold the property
only if he were in fact legally entitled to it[, the] occupation [is]
`conditional' and [cannot] form the basis of an adverse possession claim.'').
The majority rule in the country is based on \textit{French v. Pearce}, 8 Conn.
439 (1831), and recognizes that the possessor's mistaken belief does not defeat
a claim of adverse possession. [The court then interpreted legislation to
overrule Maine precedents and allow mistaken possession to meet the claim of
right requirement.]

