As previously discussed, what we conventionally call a ``mortgage'' actually has
two parts. The ``note'' is the borrower's promise to repay the debt: the note
is governed by contract law, or more specifically commercial law. The note
specifies the terms of the debt, including late fees and how interest is
calculated. It can be replaced by a ``Lost Note Affidavit,'' but that's
supposed to be for special circumstances. The ``mortgage'' is the interest in
land associated with the loan. It is a lien, governed by real estate law. The
mortgage is the thing that should be filed and recorded. The mortgage is what
gives the lender the right to take the collateral (the house) if the note isn't
paid by the borrower.

Why split things up in this way? It seems to offer more opportunities for
things to go wrong. Is there a reason not to put the two documents together
and require the ``mortgage'' to contain all the terms of the debt? What
happens if the ownership of the two legal interests, and custody of the two
documents, becomes separated? The standard rule is that ``[t]he note is the
cow and the mortgage the tail. The cow can survive without a tail, but the
tail cannot survive without the cow.'' \emph{Best Fertilizers of Ariz., Inc. v.
Burns}, 571 P.2d 675, 676 (Ariz. Ct. App. 1977), \emph{rev'd on other grounds},
570 P.2d 179 (Ariz. 1977) (quoting Professor Chester Smith). That is, the note
is the real debt; a mortgage with no associated note is worthless. However, a
note with no associated mortgage is just an unsecured loan.

Because of this, the law does not favor separation of the note and the mortgage,
and works very hard to impute a relationship between them that makes the
mortgage enforceable. The Restatement (Third) of Property (Mortgages) states
that ``in general a mortgage is unenforceable if it is held by one who has no
right to enforce the secured obligation.'' As a result, if the mortgagee
transfers the mortgage to A and the note to B, neither can foreclose
\textit{unless} A can foreclose on B's behalf. Thus, the Restatement
concludes, 
\begin{quote}
The [necessary] trust or agency relationship may arise from the terms of the
assignment, from a separate agreement, or from other circumstances. Courts
should be vigorous in seeking to find such a relationship, since the result is
otherwise likely to be a windfall for the mortgagor and the frustration of B's
expectation of security. 
\end{quote}
\emph{See also Eaton v. Federal National Mortgage Association}, 462 Mass. 569
(2012) (mortgage foreclosure may only be carried out by one who holds the
mortgage and also either holds the note or acts on behalf of the note holder). 

Most of our discussion, and most of the litigation over chain of title, has
focused on problems with the mortgages, not the notes. It seems undeniable,
however, that there are similar if not worse issues with the notes (and MERS
never purported to track notes). Some have argued that transfers of the notes
are governed by the Uniform Commercial Code's provisions for negotiable
instruments, not by state foreclosure statutes. See, e.g., Dale A. Whitman \&
Drew Milner, \textit{Foreclosing on Nothing: The Curious Problem of the Deed of
Trust Foreclosure Without Entitlement to Enforce the Note}, 66 \textsc{Ark. L.
Rev}. 21 (2013). A number of states seem to agree that the mortgagor has
nothing to complain about if she's in default, and that she lacks standing to
challenge ownership of the note. If the wrong claimant takes the property, the
right claimant can sue. 

The contrary position relies on relativity of title: peaceable possessors have
legitimate rights until someone proves better title. Just because a possessor
doesn't have a right to be on the property doesn't mean that she can be ejected
by someone with no better claim. See Tapscott v. Lessee of Cobbs, 11 Gratt.
172, 52 Va. 172 (1854); see also Yvanova v. New Century Mortgage Corp., No.
S218973 (Cal. Feb. 18, 2016) (borrower has standing to sue for wrongful
foreclosure when the foreclosing party allegedly didn't own the note and deed
of trust; ``[t]he borrower owes money not to the world at large but to a
particular person or institution, and only the person or institution entitled
to payment may enforce the debt by foreclosing on the security''). 

