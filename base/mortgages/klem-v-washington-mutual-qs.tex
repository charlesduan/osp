\item What, if anything, is the relevance of the sale price of the home to the
court's decision? Why would someone bid a dollar more than what was owed on
the loan?


\item \textit{Klem} involves a variant on what is known as
``robo-signing''---the creation of documents with important legal effects on
foreclosure, without sufficient personal knowledge or even understanding by the
person signing the document. 


Jay Patterson, a forensic accountant who has examined hundreds of mortgage loans
in bankruptcy or foreclosure, concluded that ``95 percent of these loans
contain some kind of mistake,'' from an unnecessary \$15 late fee to thousands
of dollars in fees and charges stemming from a single mistake that snowballed
into a wrongful foreclosure. Most of these cases resulted in defaults, but
when they were litigated, the facts could be telling. For example, one
bankruptcy case, \textit{In re Stewart}, involved a home in Jefferson Parish,
New Orleans. Wells Fargo was the servicer. The debtor fell behind in her
payments, and on September 12, 2005, Wells Fargo agents generated two opinions
on the value of the home. Opinions require at least minimal inspection of the
property. Stewart was charged \$125 for each opinion. However, on September
12, 2005, Jefferson Parish was under an evacuation order due to the devastation
then being wrought by Hurricane Katrina. These were only two of the numerous
fees the bankruptcy judge found had been wrongly charged to Stewart.


What ought to be done to rein in servicer misbehavior of this sort?

