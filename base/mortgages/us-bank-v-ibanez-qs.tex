\expected{us-bank-v-ibanez}

\item In a ``title'' theory state like Massachusetts, the mortgagee in theory
has legal title to the property, though the mortgagee holds it in trust for the
mortgagor, who has equitable title. The alternative ``lien'' theory holds that
legal title remains in the mortgagor, and the mortgage is merely a lien on the
property. Does the difference between title theory and lien theory make any
difference in this case? (Because of the mortgagor's equitable title in title
theory states, the general modern answer is that there is no difference in the
governing legal principles, but the reasoning may vary from state to state.) 
What about the fact that the foreclosure was done through a nonjudicial power
of sale---should that make a difference?


\item One of the reasons the court gives for its judgment is that, in
Massachusetts, assignment in blank isn't allowed for an interest in land. Why
might the legislature make such a rule? This is not the rule in all states,
but amazingly the originators didn't pay much attention to state by state
variations. 


\item Who owns the houses at issue, after the opinion? What steps should the
banks take now? What is the effect of Massachusetts' recording statute on a
scenario in which a third party buys the property at the foreclosure sale and
records?


\item Consider the following excerpt from Paul McMorrow, \textit{A new act in
foreclosure circus}, January 14, 2011, Boston Globe:
\begin{quotation}
According to the real estate tracker the Warren Group, there have been more than
44,000 residential foreclosures recorded in Massachusetts since 2006. In the
majority of those cases, the foreclosing bank turned around and re-sold the
seized property. So there are now tens of thousands of Massachusetts residents
living in homes that, until relatively recently, belonged to somebody else.

\dots I took a random sample of 30 foreclosure deeds from Chelsea (one of the
cities hit hardest by foreclosures) since the beginning of 2006. Of those 30
foreclosure cases, 10 had paperwork on file with the Registry of Deeds that
raised the sort of chain-of-custody concerns at the heart of the Ibanez
decision. In one case, no mortgage was on file with the registry. Another
showed no paperwork assigning the note to a mortgage servicer. In other cases,
mortgage originators didn't sign off on documents transferring the notes into
mortgage pools, or transfer paperwork was filed after a foreclosure occurred.
All of the properties have since been re-sold.

That's not to say any of those foreclosures will or should be overturned in
court. But it is an indication of how pervasive sloppy record-keeping was, and
how many foreclosures could be challenged on technical grounds based on the
recent SJC decision. And it presents a series of terrible questions to anyone
who bought a foreclosed house from a big bank. Among them: Is my mortgage
valid? Will I be able to refinance or sell my home? Do I even really own my
house?
\end{quotation}
How would you go about answering McMorrow's questions for a client?


\item Consider also Abigail Field, \textit{Lawyers' Carelessness Was Key to the
Mortgage Mess}, DailyFinance, Feb. 1, 2011:
\begin{quote}
The \textit{Ibanez} case highlighted a basic, non-due-diligence problem
too---one that, according to bankruptcy and legal-aid attorneys I speak with,
is occurring across the country. The banks' lawyers can't produce complete sets
of securitization contracts even after being given the specific opportunity to
do so. In various cases, the banks have submitted unsigned drafts. They've
submitted signed copies of some contracts, but not even drafts of others. And
they've submitted contracts without their exhibits, like a list of the
mortgages being securitized.

Every corporate deal I was ever involved with resulted in ``closing sets,'' a
series of binders containing every contract with each exhibit. \dots [A] key
part of the value lawyers add is keeping the documents in good order and
accessible to their clients when needed.

So, the issue of partial deal documents that came to light in \textit{Ibanez}
and continues to crop up elsewhere means one of three things:

1. Securitization deals were so carelessly done that, despite all the proper
documents being created, closing sets don't exist.

2. Securitization deals were so carelessly done that not all the proper
documents were created (such as lists of the mortgages involved) and so closing
sets don't exist.

3. All the documents and closing sets are fine, and the big banks have grown so
incompetent they can't give their foreclosure attorneys deal documents that
they do have or could get from their securitization counsel.

I'm not sure which of these is worst.
\end{quote}
What \textit{should} the banks' lawyers have done with the documents they had
available to use in the foreclosure process? 



\item Review the alleged chain of title for the Ibanez/US Bancorp mortgage. US
Bancorp took the mortgage from a now-bankrupt subsidiary of the now-bankrupt
firm Lehman Brothers. Getting an assignment from Lehman may be difficult or
even impossible. Among other things, because Lehman is bankrupt, it may not
transfer assets out of its estate to particular creditors without going through
extensive proceedings that are designed to be fair to all the creditors. 
Regardless, an assignment from Lehman would be insufficient: there is still the
undocumented Option One-Lehman transfer. It might be simplest for US Bancorp
to go straight to Option One and ask for an assignment. But US Bancorp didn't
buy the mortgage from Option One. There is no contractual relationship between
those two entities and thus no duty on Option One to do everything necessary to
ensure that US Bancorp has good title. Even if US Bancorp asks Option One for
an assignment, Option One likely regarded the mortgage as sold to Lehman many
years back and may not have appropriate records. Furthermore, Option One may
consider any attempt to assign a mortgage that was already sold to Lehman to be
legally risky; it will certainly want US Bancorp to indemnify it and likely to
pay extra for the privilege of getting the assignment. This problem is not
confined to loans that passed through Lehman (though there were a great many
that did)---many companies involved in the mortgage bubble have entered
bankruptcy or changed ownership, making documentation of the assignments all
but impossible.


\item Given that the mortgages were concededly in default, is there any reason
to insist on the formalities in cases like this? After all, the one thing we
know is that the homeowners weren't paying what they owed. \textit{See, e.g.},
Editorial, \textit{The Politics of Foreclosure}, \textsc{Wall St. J}., Oct. 9,
2010. \textit{But see} Miller v. Homecomings Financial, LLC, 881 F. Supp. 2d
825, 832 (S.D.Tex. 2012) (``Banks are neither private attorneys general nor
bounty hunters, armed with a roving commission to seek out defaulting
homeowners and take away their homes in satisfaction of some other bank's deed
of trust.''); David A. Dana, \textit{Why Mortgage ``Formalities'' Matter}, 24
\textsc{Loy. Consumer L. Rev.} 505 (2012) (given the importance of the home,
and of the rule of law, formalities matter, and may also deter future careless
lending). 


\item Sometimes the sloppiness in record-keeping led to truly astonishing
errors. In a random audit on WaMu Mortgage Pass-Through Certificates, Mortgage
Loan Trusts, one loan was found in 6 different trusts, another loan was found
in five trusts' original SEC loan level data, 39 were listed in 3 trusts, and
503 were listed in two separate trusts. The most extreme example, a New York
condo, appeared in 6 different trusts from May through November 2006. Gary
Victor Dubin,
Securitized
Distrust, Mar. 15, 2012,
\url{https://deadlyclear.wordpress.com/2012/03/15/securitized-distrust/}.

\item Occasionally, evidence of these careless procedures appears in official
title records. Recall that, in order to move the mortgage from its originator
to its ultimate holder, an assignment was required---usually more than one,
with a chain going from the originator, to the sponsor who lent the originator
money to make the loan, to the depositor that funded the sponsor, to the trust
that ultimately held the mortgages as assets underlying the mortgage-backed
securities it issued. Who exactly is the assignee in this record from Nassau
County, New York?
\heregraphic{mortgages-1}

