\readingnote{Excerpts reprinted with permission.}
\reading[Rose, \emph{Crystals and Mud in Property Law}]{Carol M. Rose,
\textit{Crystals And Mud In Property Law}}

\readingcite{40 \textsc{Stan. L. Rev.} 577 (1988)}

Property law, and especially the common law of property, has always been heavily
laden with hard-edged doctrines that tell everyone exactly where they stand.
Default on paying your loan installments? Too bad, you lose the thing you
bought and your past payments as well. Forget to record your deed? Sorry, the
next buyer can purchase free of your claim, and you are out on the street. Sell
that house with the leak in the basement? Lucky you, you can unload the place
without having to tell the buyer about such things at all. 

In a sense, hard-edged rules like these---rules that I call ``crystals''---are
what property is all about. If, as Jeremy Bentham said long ago, property is
``nothing but a basis of expectation,'' then crystal rules are the very stuff of
property: their great advantage, or so it is commonly thought, is that they
signal to all of us, in a clear and distinct language, precisely what our
obligations are and how we may take care of our interests. Thus, I should
inspect the property, record my deed, and make my payments if I don't want to
lose my home to unexpected physical, legal, or financial impairments. I know
where I stand and so does everyone else, and we can all strike bargains with
each other if we want to stand somewhere else. 

Economic thinkers have been telling us for at least two centuries that the more
important a given kind of thing becomes for us, the more likely we are to have
these hard-edged rules to manage it. We draw these ever-sharper lines around
our entitlements so that we know who has what, and so that we can trade instead
of getting into the confusions and disputes that would only escalate as the
goods in question became scarcer and more highly valued.  

At the root of these economic analyses lies the perception that it costs
something to establish clear entitlements to things, and we won't bother to
undertake the task of removing goods from an ownerless ``commons'' unless it is
worth it to us to do so. What makes it worth it? Increasing scarcity of the
resource, and the attendant conflicts over it. To use the example given by
Harold Demsetz, one of the most notable of the modern economists telling this
story, when the European demand for fur hats increased demand for (and scarcity
of) fur-bearing animals among Indian hunters, the Indians developed a system of
property entitlements to the animal habitat. Economic historians of the
American West tell a similar story about the development of property rights in
various minerals and natural resources. Easy-going, anything-goes patterns of
appropriation at the outset came under pressure as competition for resources
increased, and were finally superseded by much more sharply defined systems of
entitlement.  In effect, as our competition for a resource raises the costs of
conflict about it, those conflict costs begin to outweigh the costs of taking
it out of the commons and establishing clear property entitlements. We
establish a system of clear entitlements so that we can barter and trade for
what we want instead of fighting. 

The trouble with this ``scarcity story'' is that things don't seem to work this
way, or at least not all the time. Sometimes we seem to substitute fuzzy,
ambiguous rules of decision for what seem to be perfectly clear, open and shut,
demarcations of entitlements. I call this occurrence the substitution of ``mud''
rules for ``crystal'' ones.

Thus,\dots over time, the straightforward common law crystalline rules have
been muddied repeatedly by exceptions and equitable second-guessing, to the
point that the various claimants under real estate contracts, mortgages, or
recorded deeds don't know quite what their rights and obligations really are.
And the same pattern has occurred in other areas too.\dots

Quite aside from the wealth transfer that may accompany a change in the rules,
then, the change may sharply alter the \textit{clarity} of the relationship
between the parties. But a move to the uncertainty of mud seems disruptive to
the very practice of a private property/contractual exchange society. Thus, it
is hardly surprising that we individually and collectively attempt to clear up
the mud with new crystal rules---as when private parties contract out of
ambiguous warranties, or when legislatures pass new versions of crystalline
record systems---only to be overruled later, when courts once again reinstate
mud in a different form.\dots

Early common law mortgages were very crystalline indeed. They had the look of
pawnshop transactions and were at least sometimes structured as conveyances: I
borrow money from you, and at the same time I convey my land to you as security
for my loan. If all goes well, I pay back my debt on the agreed ``law day,'' and
you reconvey my land back to me. But if all does not go well and I cannot pay
on the appointed day, then, no matter how heartrending my excuse, I lose my
land to you and, presumably, any of the previous payments I might have made. As
the fifteenth century commentator Littleton airily explained, the name
`mortgage' derived from the rule that, if the debtor ``doth not pay, then the
land which he puts in pledge\dots is gone from him for ever, and so dead.'  

This system had the advantage of great clarity, but it sometimes must have
seemed very hard on mortgage debtors to the advantage of scoundrelly creditors.
Littleton's advice about the importance of specifying the precise place and
time for repayment, for example, conjures up images of a wily creditor hiding
in the woods on the repayment day to frustrate repayment; presumably, the
unfound creditor could keep the property. But by the seventeenth century, the
intervention of courts of equity had changed things. By the eighteenth and
nineteenth centuries, the equity courts were regularly giving debtors as many
as three or four ``enlargements'' of the time in which they might pay and redeem
the property before the final ``foreclosure,'' even when the excuse was lame.
One judge explained that an equity court might well grant more time even after
the ``final'' order of ``foreclosure absolute,'' depending on the particular
circumstances.  

The muddiness of this emerging judicial remedy argued against its
attractiveness. Chief Justice Hale complained in 1672 that, ``[b]y the growth of
Equity on Equity, the Heart of the Common Law is eaten out, and legal
Settlements are destroyed;\ldots as far as the Line is given, Man will go; and
if an hundred Years are given, Man will go so far, and we know not whither we
shall go.'' Instead of a precise and clear allocation of entitlements between
the parties, the ``equity of redemption'' and its unpredictable foreclosure
opened up vexing questions and uncertainties: How much time should the debtor
have for repayment before the equitable arguments shifted to favor the
creditor? What sort of excuses did the debtor need? Did it matter that the
property, instead of dropping in the lap of the creditor, was sold at a
foreclosure sale?

But as the courts moved towards muddiness, private parties attempted to bargain
their way out of these costly uncertainties and to reinstate a crystalline
pattern whereby lenders could get the property immediately upon default without
the costs of foreclosure. How about a separate deal with the borrower, for
example, whereby he agrees to convey an equitable interest to the lender in
case of default? Nothing doing, said the courts, including the United States
Supreme Court, which in 1878 stated flatly that a mortgagor could not initially
bargain  away his ``equity of redemption.'' Well, then, how about an arrangement
whereby it looks as if the lender already owns the land, and the ``borrower''
only gets title if he lives up to his agreement to pay for it by a certain
time? This seemed more promising: In the 1890s California courts thought it
perfectly correct to hold the buyer to his word in such an arrangement, and to
give him neither an extension nor a refund of past payments. By the 1960s,
however, they were changing their minds about these ``installment land
contracts.''  After all, these deals really had exactly the same effect as the
old-style mortgages---the defaulting buyer could lose everything if he missed
a payment, even the very last payment. Human vice and error seemed to put the
crystal rule in jeopardy: In a series of cases culminating with a default by a
``willful but repentant'' little old lady who had stopped paying when she
mistakenly thought that she was being cheated, the California Supreme Court
decided to treat these land contracts as mortgages in disguise.  It gave the
borrower ``relief from forfeiture''---a time to reinstate the installment
contract or get back her past payments.  

With mortgages first and mortgage substitutes later, we see a back-and-forth
pattern: crisp definition of entitlements, made fuzzy by accretions of judicial
decisions, crisped up again by the parties' contractual arrangements, and once
again made fuzzy by the courts. Here we see private parties apparently
following the ``scarcity story'' in their private law arrangements: when things
matter, the parties define their respective entitlements with ever sharper
precision. Yet the courts seem at times unwilling to follow this story or to
permit these crystalline definitions, most particularly when the rules hurt one
party very badly. The cycle thus alternates between crystal and mud.

