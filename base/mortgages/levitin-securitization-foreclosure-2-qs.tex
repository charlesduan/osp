\item Along with the ``Wikipedia'' characterization of MERS as freely editable,
a Reuters investigation used a different metaphor: ``MERS has served in effect
as an instant teller machine for mortgage assignments. Servicers simply have
their own employees sign the needed documents as MERS officials.'' At least
one bank has sued another bank for allegedly assigning the first bank's
mortgages to itself using MERS.


\item According to Donald J. Kochan, public recording serves a number of
important purposes:
\begin{quotation}
Recording creates a network of information supporting a network of transactions.
Property recording systems offer information to a number of constituencies,
including: (1) owners, acting as sellers or as borrowers; (2) lenders,
including mortgage providers; (3) other providers of capital; (4) buyers; (5)
leaseholders; (6) title insurance companies; (7) governmental entities, such as
police, regulators, and taxing authorities; and (8) other parties who may need
to interact with the property at some time and know who the law deems to have
ownership of the property. Recording allows all of these market and legal
participants to connect. It is imperative that we recognize the variety of
market players that use and benefit from the recording statutes and from the
existence of reliable, verifiable records of ownership. 

It is not just the owner and the most immediate lender that care about proper
recording. Those who wish to invest in, contract with, lease from, or provide
capital to property owners demand the existence of a recording system so that
they can identify the ownership interests associated with the property,
including determining whether and to what extent that property is encumbered by
a mortgage. So, too, do prospective buyers of property require a verifiable
repository of title information to guide their purchasing decisions. These
other players must be able to discover the limits on title with a level of
clarity. Similarly, those who wish to provide loans secured by property or to
make other capital investments in property need assurances that the owner owns
the property that he says he owns and that the system reflects all competitive
claims to or liens on title. 

 \dots At the very least, fragmentation of interests by securitization makes
ownership interests in real property harder to identify, necessitating the
existence of an accurate and complete means for tracking and recording these
interests. \dots [Securitization] is an important financial mechanism for the
efficient provision of capital and should not be sacrificed in an effort to
resolve the mortgage crisis or to prevent future crises. In fact, it is
difficult to see the provision of loans in today's financial system without
some reliance on securitization. 

To make securitization effective, the loan-granting institution typically
assigns its rights in both the note and the mortgage, sometimes to different
parties. Due to transfers to the secondary market, securitization, and
multiple assignments of notes and mortgages, it can become difficult to trace
all of the steps along the way. This flurry of activity---and the number and
variety of participants involved---can lead to problems in the chain of title
and identifying who ultimately and currently holds the enforceable note and
mortgage interests against the property owner. These problems are especially
evident when the formalities of transfer, such as required endorsements of
notes, are not satisfied and when the transfers are not recorded in some
central repository.
\end{quotation}
Donald J. Kochan, \emph{Certainty of Title: Perspectives after the Mortgage
Foreclosure Crisis on the Essential Role of Effective Recording Systems}, 66
\textsc{Ark. L. Rev.} 267 (2013). Could MERS fulfill the functions of recording
in such a system? What would need to change?


\item Many laws require notice to be given to anyone with a properly recorded
interest in property. If MERS is listed as the ``nominee'' of the lender, does
it have such an interest? Suppose the mortgagor failed to pay county property
taxes and her home was subject to a tax sale. The county sent notice to the
mortgagee for whom MERS was listed as nominee, but that entity had long ago
sold the mortgage and subsequently went out of business, so it didn't respond. 
If the county had sent notice to MERS, it's at least possible that MERS would
have informed the current owner of the mortgage, which would have participated
in the tax sale to protect its interest. Instead, because no representative of
the mortgagee showed up at the tax sale, the property was sold to a new owner
free of the mortgage. Did MERS suffer a redressable injury? \textit{See}
Mortgage Elec. Regis. Sys., Inc. v. Ditto, No. E2012-02292-SC-R11-CV (Tenn.
Dec. 11, 2015) (MERS had none of the rights or duties of an owner; the fact
that it wasn't entitled to notice might cause harm to its business model, but
that wasn't enough to provide it a right to notice).


\item Why were large institutions so cavalier about record-keeping? Fannie Mae
is a government-backed, now government-run institution that bought many
mortgages. A 2006 internal Fannie Mae investigation explained: 
\begin{quotation}
Fannie Mae's position is that it does not need to appear in the land records in
order to have the benefit of the security provided by the mortgage \dots .
[T]he transfer of an obligation secured by a security interest also transfers
the security interest. Thus, the transfer of the promissory note, which is the
obligation, also transfers the mortgage, which is the security interest. Once
the note is sold to Fannie Mae, the mortgage also transfers, despite the fact
that the servicer, lender, or MERS' name appears in the land records. Borrowers
thus cannot determine the chain of owners from public records\dots.

Fannie Mae believes that lost note affidavits are the servicer's responsibility
and can not be effectively reviewed under the current system. Fannie Mae has
delegated the execution of lost note affidavits to servicers. It does not
believe that it is in a position to make a subjective call as to whether a
servicer has lost a note \dots . Fannie Mae views such an investigation as
unnecessary because document custodians are responsible for retaining
mortgage documents and must bear an expense if they are unable to locate
mortgage documents. For these reasons, Fannie Mae believes that
\textit{servicers are not likely to state that the notes are lost, stolen or
missing if they in fact are not}. (emphasis added) 
\end{quotation}
Can you spot the problem here? One entity, Lender Processing Services, at one
point had a price list for ``recreating'' mortgage-related documents. A lost
note affidavit was \$12.95, as was a note allonge (a document that is supposed
to be stapled to the original note documenting a transfer); an intervening
assignment to fill a gap in the record chain of title was \$35, and
``recreating'' an entire file was \$95.



An important point to remember here is that recording usually only matters when
there's a bona fide purchaser contesting ownership. As long as the originator
didn't sell the mortgage twice, an unrecorded interest is still valid against
the mortgagor, assuming the claimant can prove that it owns the interest. That
may be a faulty assumption, but Fannie Mae figured that risk was low.



\item The toxic brew of carelessness, unprecedented volume of foreclosures, and
disregard for the rights of borrowers has finally begun to attract judicial
attention. A bankruptcy judge recently identified
\begin{quote}
a general willingness and practice on Wells Fargo's part to create documentary
evidence, after-the-fact, when enforcing its claims, \textbf{WHICH IS
EXTRAORDINARY.} Moreover, [the Wells Fargo employee's] testimony does not stop
at describing manufactured mortgage assignments. \dots [T]he ``assignment
team'' included people tasked with endorsing notes [from other entities to
Wells Fargo] \dots Frankly, it does not appear that [the Wells Fargo employee]
understood the difference between preparing legitimate assignments and
indorsements \textit{by} Wells Fargo and improper assignments and indorsements
\textit{to} Wells Fargo. (emphasis in original)
\end{quote}
\emph{In re Carrsow-Franklin}, No. 10-20010 (S.D.N.Y. Bkcy. Jan. 29, 2015). What
responsibility do lawyers have to train employees in charge of tasks with such
legal relevance?

