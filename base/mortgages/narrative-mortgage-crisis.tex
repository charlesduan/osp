Predatory lending was a significant contributor to the housing crash of
2007-2008.  Many people, whether or not they accept this proposition, believe
that poor people were taking out the mortgages at issue.  However, middle and
high income borrowers took on more mortgage debt than poor people, and also
contributed most significantly to the increase in defaults after 2007.  Manuel
Adelino, Antoinette Schoar, \& Felipe Severino,
Loan Originations and Defaults in the
Mortgage Crisis: Further Evidence (NBER Working Paper July 2015),
\url{https://www.nber.org/papers/w21320}.  

When home prices started to drop and defaults to accumulate, the mortgage-backed
securities that had previously seemed so attractive to investors began to
spread the damage widely, as payments dried up.  The economic impact was
multiplied by a variety of sophisticated financial instruments that, in the
end, amounted to little more than bets that U.S. housing prices would never
drop.  When they did drop, the world economy did as well.  

From 1942 to 2005, about 4\% of mortgages were delinquent at any given time, and
about 1\% were in foreclosure.  At the peak of the crisis in 2010, up to 15\%
of mortgages were delinquent, and 4.6\% were in foreclosure.  Foreclosures
Public Data Summary Jan 2015.  As of late 2014, less than 8\% of mortgages were
delinquent and more than 3\% were in the foreclosure process, or about one
million homes. The good news is that most of the still-troubled loans were
originated before 2007, and new foreclosures are now less one-half of one
percent of all mortgages. Still, between 2007 and 2015, about six million homes
were sold at foreclosure sales.  This foreclosure crisis has already outlasted
the foreclosure crisis of the Great Depression.  

Even homeowners who kept up with their payments often found themselves
``underwater'': owing more than their homes were worth.  Nearly one-third of
mortgaged homes were underwater in 2012, though the number dropped to 15.4\% in
early 2015.  Homes with lower value were more likely to be underwater,
contributing to income inequality.  Michelle Jamrisko, \textit{This Is the
Housing Chart That Keeps One Economist Up at Night}, BloombergBusiness, Jun.
12, 2015.  Unsurprisingly, underwater homeowners are substantially more likely
to default on their mortgages than homeowners with equity, no matter the size
of their monthly payments or their interest rates.  Moreover, underwater
homeowners who don't default find it very difficult to sell their homes, and
are therefore constrained in where they can take jobs.  This is a problem
because job mobility historically has been a major contributor to improved
economic prospects in the U.S.

Even when ``strategic default'' might be in a homeowner's best interest---where
the homeowner is deeply underwater and lives in a non-recourse state, and
alternative housing is readily available---Americans remain relatively
unlikely to default if they have any alternatives.  Most borrowers will run up
credit card bills, drain retirement savings, and put off medical care to avoid
default for as long as possible.  Tess Wilkinson-Ryan, \textit{Breaching the
Mortgage Contract: The Behavioral Economics of Strategic Default}, 64
\textsc{Vand. L. Rev.} 1547 (2011) (reporting that even though defaulting on a
mortgage may be in an individual's financial self-interest, feelings of moral
obligation may prevent or delay default).  Under what circumstances might you
counsel a client to engage in a strategic default?  Default will have
consequences for the defaulter's credit score and therefore possibly her
ability to get other housing or even a job, depending on her location and her
field of work.  But then again, draining her retirement account, possibly only
to postpone and not avoid foreclosure, will have negative repercussions as
well.

