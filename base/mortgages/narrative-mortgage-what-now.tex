Despite all these flaws in the system, shouldn't the borrowers just have paid?
After all, if they hadn't defaulted, they wouldn't have entered into the
resulting hellscape. 

Even if you excuse the victims of fraud from this claim---and there were
many---it is important to remember that the mortgage contract is not just an
agreement that the home may be sold upon a default on the loan. It's an
agreement that if the homeowner defaults on the loan, the mortgagee may sell
the property following the required legal procedure. A mortgage loan involves a
bundle of rights, including procedural rights. These rights have a price: for
example, loans in judicial foreclosure states have historically been more
expensive than loans in nonjudicial foreclosure states. When the lender (or
someone claiming rights as successor of the lender) ignores the rights, it's
getting something it hasn't paid for.

Entirely separately, we might want people to be able to renegotiate their deals
when renegotiation makes everyone better off---but with the system the way it
is, that's proven extremely difficult. Procedural protections provide both
time and negotiating leverage.

For an eye-opening account by one young lawyer of the messy process of seeking a
loan modification for a borrower, see Wajahat Ali,
Could
It Be That the Best Chance To Save a Young Family from Foreclosure Is a
28-Year-Old Pakistani American Playright-Slash-Attorney Who Learned Bankruptcy
Law on the Internet?, McSweeney's (Jan. 2010),
\url{http://www.mcsweeneys.net/articles/could-it-be-that-the-best-chance-to-save-a-young-family-from-foreclosure-is-a-28-year-old-pakistani-american-playright-slash-attorney-who-learned-bankruptcy-law-on-the-internet}.
Essentially, Ali could never get the same answer twice from the servicer; it
repeatedly denied receiving documents supporting his clients' request for
modification; it denied modifications based on completely mistaken premises;
and it didn't even tell him or his clients when it finally did grant a
modification, leaving them expecting foreclosure. It took multiple threats to
file bankruptcy, which would have automatically stayed a foreclosure, to induce
the servicer to respond.

Federal bank regulators signed settlements in March 2011 with 14 loan servicers,
who promised further internal investigations, remediation for some who were
harmed, and a halt to the filing of false documents. The servicers claimed to
have ended this behavior in late 2010. Reuters examined a large number of
foreclosure filings and concluded that, to the contrary, robo-signing was
ongoing. In February 2012 the servicers promised to stop again. There's very
little indication that they've stopped. However, the major servicing companies
did enter into a \$25 billion settlement with federal and many state officials
that was supposed to compensate homeowners for servicing errors and require
better behavior going forward. In response, property professor Mark Edwards
wrote:
\begin{quotation}
Let's say I hire an armed gang to expel you from your house.  My gang removes
all of your belongings, changes the locks, and warns you that you'd better not
try to come back.  I then sell your house to someone else. You might have
called the police, but the armed gang I hired actually \textit{are} the
police.  You might have gone to court to stop me, but the court is on my
side, because I deliberately mislead the courts. Now let's say I did the same
thing thousands and thousands of times to other people as well. And you can
prove it. I'd be in pretty big trouble, wouldn't I?  \dots .

[The settlement provides for] \$1500 to \$2000 per home \dots . \$1500-2000 is
less than the legal expenses banks incur when a foreclosure is challenged. 
It's less than title insurance on homes worth over \$200K.
\end{quotation}
Why would regulators agree to a settlement of this magnitude? What were the
alternatives? 

Mortgage crisis-related disputes continue. For example, in March 2015, the
Department of Justice's U.S. Trustee Program (USTP), which oversees bankruptcy
estates, entered into a national settlement agreement with JPMorgan Chase Bank
N.A. requiring Chase to pay more than \$50 million to over 25,000 homeowners
who are or were in bankruptcy. Among other things, Chase acknowledged that it
filed more than 50,000 payment change notices that were improperly signed,
under penalty of perjury, by persons who had not reviewed the accuracy of the
notices.

In 2013, the Consumer Finance Protection Bureau issued national rules on
mortgage servicing standards. Except for smaller servicers, mortgage servicers
must make good-faith efforts to contact borrowers by the 36th
day of delinquency and tell them about loss mitigation options, such as short
sales and loan modifications. By day 45, servicers must send written notice of
these options and the name of a contact person. Servicers may only begin
foreclosures if a homeowner is over 120 days delinquent, and a borrower's
pending loss mitigation application precludes the initiation of a foreclosure. 
If the foreclosure has been initiated when a borrower submits a loss mitigation
application, the servicer may not move for final judgment or sale as long as
the application is complete 37 days before the sale. Servicers may not
``double-track''---pursue mitigation measures with a borrower while also
continuing the foreclosure process. If the servicer denies the borrower's
application, it must give specific reasons and afford a right of appeal. 
Mortgage Servicing Rules Under the Real Estate Settlement Procedures Act
(Regulation X), 78 Fed. Reg. 10696 (codified at 12 C.F.R. \S~1024) (2013).

As a result of changes in foreclosure procedures and servicer behavior, the
average time between the beginning of a foreclosure and its end has increased
substantially. In 2007, a foreclosure in New York took less than 300 days,
while it took 1089 days by the end of 2012. California, which more commonly
uses the speedier nonjudicial foreclosure process, experienced a more than
doubled time of 347 days. RealtyTrak,
2013
Short Sale Trends (2013),
http://www.slideshare.net/fullscreen/RealtyTrac/2013-short-sale-trends/1.

What are the possible benefits of delay for homeowners? What about the possible
risks? In some cases, servicers initiate foreclosures but do not complete
them, leading to so-called ``zombie foreclosures.'' Completing the foreclosure
would make the mortgagee the legal owner of the property, subject to property
taxes and to the duty to avoid creating a nuisance condition on the property. 
In markets with many empty houses, the mortgagee may well wish to avoid this
outcome, because it won't be able to sell the house quickly or otherwise recoup
its maintenance costs. In addition, when people believe that they will soon be
kicked out, they tend not to maintain the property, and some even deliberately
inflict damage. However, the mortgagors are often unaware of their continuing
legal duties, and abandon the property in the belief that the foreclosure will
occur, exposing themselves to unforeseen liability and their communities to
further deterioration of the tax base and the physical condition of homes. Is
there anything law could do to mitigate the problem of such ``zombie
foreclosures''?

Various programs have attempted to help homeowners at risk of foreclosure, with
generally modest results. The federal government set up the
Home
Affordable Modification Program (HAMP), which was supposed to keep four
million homeowners in their homes by reducing interest rates and extending
repayment times, though not by forgiving principal. Six years later, under
900,000 homeowners were participating in modifications. Servicers rejected
four million applications, or 72\% of requests. The main culprits were the
fact that the program was voluntary, and that the servicers were allowed to run
the process on their own. In 38\% of cases, the servicers claimed that the
borrowers failed to supply all the paperwork or to make the first modified
payment. Office of the Special Inspector General for the Troubled Asset Relief
Program (SIGTARP), \emph{Quarterly Report to Congress} (July 29, 2015),
\url{http://www.sigtarp.gov/Quarterly\%20Reports/July_29_2015_Report_to_Congress.pdf}.
Not all of that paperwork was actually absent. In 2014, SIGTARP found that the
employees of one servicer piled so many unopened Federal Express packages from
homeowners containing their HAMP supporting documents into one room that
eventually the floor buckled. SIGTARP also found that this servicer denied
homeowners from HAMP en masse, without reviewing their applications at all.

Incompetence plays a large role, but that incompetence also may redound to the
benefit of servicers: in one case, for example, a servicer delayed responding
to a HAMP modification request four times over nearly two years, adding
\$40,000 in interest, fees, and costs to the amount the borrower allegedly
owed. The servicer first mistakenly denied a modification because it
inexplicably decided that the borrower didn't live at the residence. Then it
mistakenly denied a modification as unaffordable because it inexplicably used
the wrong figures to calculate the borrower's income despite extensive
documentation of the correct figures. Then it mistakenly denied a modification
because the borrower allegedly had too much money in the bank to qualify; this
money was in fact the amount the borrower had set aside in escrow to pay the
mortgage, as directed by the court-appointed referree. The servicer also
maintained that the borrower hadn't submitted complete documentation for his
modification application, though when the referree directed a representative of
the servicer to appear at a hearing, the representative (who had just testified
to her personal knowledge of this incompleteness) was able to pull up the full
application on her laptop. Finally, the servicer denied a modification as
unaffordable, without further explanation of how it calculated affordability.
\textit{See} \emph{US Bank N.A. v Sarmiento}, 121 A.D.3d 187 (N.Y. App. Div.
2014). The court found that the servicer's lack of good faith disqualified it
from claiming the additional amount owed. As the lawyer for the borrower, how
would you have dealt with two years of delay by the servicer?

In the absence of further federal legislation, states may have more options in
dealing with foreclosure abuses, because foreclosure procedure is a state-law
matter. See \textsc{Center for Responsible Lending and Consumers Union,
Closing the Gaps: What States Should Do To Protect Homeowners from Foreclosure}
(May 2013). The most effective programs seem to be those that require
mandatory mediation between the mortgagee and the homeowner before a
foreclosure can proceed. The most effective of those require the mortgagee to
send a representative with (1) authority to negotiate a modification and (2)
proof of the chain of title to the mortgage. Why do you think that voluntary
programs, under which homeowners are entitled to mediation but must
affirmatively request it, are less effective than mandatory programs? What
effects will requiring proof of the chain of title have on the parties'
negotiations? If you were an attorney for a servicer, what would you ask for
from homeowners in return for a modification that left them with the ability to
stay in their homes?

Massachusetts law now prohibits a creditor from publishing notice of an intended
foreclosure sale for many residential mortgages unless the creditor ``has first
taken reasonable steps and made a good faith effort to avoid foreclosure.'' 
Mass. Gen. L. ch. 244 {\S}35B. The creditor must calculate the relative
benefits of foreclosure and modification, and must offer the borrower a
modified mortgage loan with an affordable payment (if the net present value of
such a loan exceeds the anticipated recovery from foreclosure) or notify the
borrower that he is not eligible for a modification and provide the borrower
with copies of the creditor's net present value and affordable payment
analyses. One might think that no law would be required to require creditors
to maximize their profits from a loan, but creditors didn't want to engage in
individualized determinations. Is the Massachusetts model a good one?

In New York, the courts implemented a requirement that lawyers filing for
foreclosure had to certify that they had taken ``reasonable'' measures to
verify the accuracy of documents submitted to the court, under penalty of
sanctions. The month before this requirement went into place, roughly 100-200
foreclosures were filed each day. The next month, no more than 5 foreclosures
were filed on any given day, with the exception of one day in which 22
foreclosures were filed. Amazingly, this requirement replaced a previous order
requiring attorneys for foreclosing entities to certify that \textit{to the
best of their knowledge} there weren't any false statements of fact or law in
their documents. 

New York's system also includes pre-foreclosure conferences; although they are
voluntary, they are encouraged, and the courts presently get 80\% of
foreclosure defendants to show up for a pre-foreclosure conference session. 
Thus, there is a real risk that someone will challenge the foreclosure
documents if they're not in order. In order to make such programs work, it is
important to convince homeowners that there is some hope---many have engaged
in futile attempts to get a modification before. If they aren't encouraged,
many of them believe that the judicial procedure is just another runaround. Is
the New York model a good one?

