A \term{mortgage} is an interest in land.  It is not a possessory interest: the
owner
of a mortgage has no right to use the property, the way the owner of the fee or
an easement owner would.  Instead, mortgages exist to secure loans.  A secured
loan is backed, or secured, by a specific asset such as a house or a car, which
the lender can seize in case of default.  An unsecured loan is not secured by
any specific asset---for example, credit card debt and student loans are
unsecured.  The borrower owes the money, and the lender can go after the
borrower's unsecured assets in case of default, but if those assets are too
small, the unsecured lender is out of luck.  Secured loans are generally
considered less risky than unsecured loans, for obvious reasons, and should
bear lower interest rates (absent some foolery on the part of the lender or
government intervention into the market, both of which do happen).  

Most mortgages are residential mortgages.  Usually, homebuyers in the United
States can't afford to pay the entire purchase price of a house at the time they
buy it. Instead, they take out a loan---a mortgage---to pay the bulk of the
purchase price.  They will sign a \term{promissory note} (the note) that creates
personal liability for the borrowers if they fail to pay, and also sets out the
terms of the mortgage such as the repayment period and the interest rate.  They
will also sign a mortgage, a written instrument that grants the lender an
interest in their newly purchased land.  Usually, this transaction occurs at the
time the buyers buy the land, though mortgages can also be refinanced or taken
out on already-owned property.

The homebuyers are the \term[mortgagor]{mortgagors}.  The lender is the
\term{mortgagee}.  Over time, the buyers pay off the loan.  As they pay off
the loan, they build ``equity'' in their homes.  Equity is the difference
between what a home is worth and what the homeowners owe on their
mortgage.\footnote{This terminology has a historical basis in the ``equity of
redemption,'' which was a means by which early chancellors protected early
mortgagors from abuses by lenders.  Over time, the equitable procedures created
by courts gave way to legislation establishing rules for how foreclosures could
occur.}  As a result of deliberate policy choices, the model residential
mortgage in the United States is for no more than 80\% of the value of the house
at time of purchase; has a fixed interest rate; and amortizes over a period of
years, usually twenty or thirty.  Amortization means that the payments are the
same throughout the period of the mortgage: at the beginning, most of the
payments go to interest on the loan, while over time more and more of the
payments go to reduce the loan principal.  

The mortgagors can transfer the land at will.  However, any transfer will not
free the land from the mortgage (nor will a transfer free them from their
contractual promise to pay the debt); the mortgage \term[run with the land]{runs
with the land}.
 Thus, a sensible transferee will not be willing to pay full value for the
land---the fair market value of the land is reduced by the amount of the
mortgage. A transferee can either take ``subject to the mortgage,'' which means
that the original mortgagors still owe the debt and the transferee is at risk if
they don't pay, or ``assuming the mortgage,'' which means that the new owner
agrees to pay the mortgage directly.  When the purchaser assumes the mortgage,
the seller still has a duty to pay the mortgage if the buyer doesn't, but the
seller can pursue the buyer for reimbursement if that happens.  However, this
all risks some big messes; to avoid problems associated with transfers, many
mortgages have ``due on sale'' clauses, which means that the full amount of the
mortgage comes due (``accelerates'') when the mortgagor sells the property. One
important feature of a due on sale clause is that it enables lenders to reprice
loans: if the interest rate has risen since the initial mortgage loan, the buyer
can't just assume the existing loan and receive a lower interest rate than would
otherwise be available to him.

Suppose Joan Watson wants to sell her house to Sherlock Holmes.  She still owes
\$400,000 on her house; Holmes will be buying it for \$500,000.  But she
doesn't have \$400,000 in the bank to pay off her mortgage, which has a due on
sale clause.  How can she accomplish the sale?  The answer is that a series of
transactions take place together.  The day of the sale, Holmes will give Watson
a check for \$500,000 (most of which will likely come from Holmes' own new
mortgage on the property).  Watson will then pay her lender \$400,000 and keep
\$100,000.  As you can see, there will be some time at which both Holmes and
Watson are relying on the value of the underlying property---Holmes to get his
mortgage and Watson to pay hers off.  For this reason, real estate transactions
regularly involve the use of multiple third parties, including escrow agents,
to facilitate and guarantee the sale.  

If the mortgagors \term{default} on the mortgage by failing to pay the
appropriate
amounts at the appropriate times, the mortgagee can
\term[foreclosure]{foreclose}.  Foreclosure can
be time-consuming and expensive, so in some circumstances the mortgagee may
accept a ``deed in lieu of foreclosure,'' by which the mortgagor surrenders the
property to the mortgagee and the mortgagee accepts the deed.  However, deeds
in lieu of foreclosure are relatively rare; most of the time, if a default is
not cured and the loan is not modified, the result will be a foreclosure.  

Either by a private sale (\term{nonjudicial foreclosure}) or under judicial
supervision (\term{judicial foreclosure}), the mortgagee can have the property
sold and apply the
proceeds of the sale to the amount due on the note.  The foreclosure is so
called because it forecloses the mortgagee's ability to get the property back
by paying off the mortgage debt; after the foreclosure, it is too late to
become current.\footnote{At common law, the equity of redemption allowed the
mortgagor to redeem the property from the mortgagee.  This equity of redemption
was extinguished by foreclosure sale. In about half of the states, there is
also a statutory right to redeem the property from the \textit{purchaser} at a
foreclosure sale for a certain period of time.  This right is rarely used,
because most people would already have paid, if they could, before the sale.}

In a number of states, it is possible to avoid judicial foreclosure---which
takes more time and money than nonjudicial foreclosure---through the use of a
``deed of trust,'' which is recognized in most jurisdictions.  Under a deed of
trust, the borrower conveys title to the property to a person to hold in trust
to secure the debt.  If the borrower defaults, the trustee has the power of
sale without needing to go to court.  However, almost all states that allow
this procedure do impose some procedural safeguards, such as notice and public
sale.  Other than the ability to avoid judicial foreclosure, you can expect a
deed of trust to be treated like a mortgage.

In addition, there are two different types of secured loans: recourse and
nonrecourse loans.  For a \term{nonrecourse loan}, the only way the lender can
get its
money back in case of default is by seizing the asset, and if there's not
enough money to satisfy the debt from the asset, too bad for the lender.  The
lender has no ``recourse'' against any of the borrower's other assets.  A
\term{recourse loan} is different: in case of default, the lender can seize and
sell
the asset, and if there's not enough money to satisfy the debt, the lender is
now an unsecured creditor for the remaining balance (the deficiency) and can go
after any of the borrower's other assets, such as her bank account. 
Foreclosure wipes out the lender's interest in the land, which means that the
land can then be resold free of the lender's interest.  However, with a
recourse loan, foreclosure will not wipe out the borrower's debt, if it is
greater than the foreclosure sale amount.  

Obviously, lenders ordinarily prefer recourse loans, but will grant nonrecourse
loans in various circumstances.\footnote{In fact, the basic idea of a
corporation is a way of limiting a lender's recourse: before the corporate
form, if a business owner went bust, creditors could go after the owner's
personal assets until they were gone.  The corporation allows
shareholders/owners to limit their liability to the extent of the corporation's
assets.  If a person owned shares of Lehman Brothers, its creditors could make
her shares worthless, but they couldn't make her pay Lehman Brothers' debts.} 
Many businesses can get nonrecourse loans based on their assets.  Some states
bar deficiency judgments for residential mortgages, which makes them
nonrecourse loans.  Other states bar deficiency judgments unless there is a
judicial foreclosure, with its greater expense and greater procedural
protections for the borrower.  Still others limit the amount of any deficiency
judgment to the difference between the principal balance and the property's
fair market value at the time of foreclosure---this limit recognizes that
foreclosed properties often sell for below market value for a variety of
reasons, including buyers' uncertainty about the true condition of the property
and the limited number of potential buyers who bid at foreclosure sales. 
(Historically, the mortgagee is often the only bidder at a foreclosure sale. 
Why would this be true?)

Even states that allow deficiency judgments generally recognize an exception: if
the sale price shocks the conscience, then a deficiency judgment may not be
allowed.  More generally, even in the absence of a potential deficiency
judgment, the foreclosing entity has a limited duty of good faith to the
mortgagor in seeking an acceptable price at the sale.  However, mere inadequacy
of price will not invalidate a sale in the absence of fraud, unfairness, or
procedural problems that deterred bidding.  As a result, very low sale prices
are sometimes accepted by courts.  \textit{Compare} \emph{Moeller v. Lien}, 30
Cal. Rptr. 2d 777 (Ct. App. 1994) (sale at 25\% of market value was acceptable
where sale was to bona fide purchaser and there was no irregularity in the sale
procedure), \textit{with} \emph{Murphy v. Fin. Dev. Corp.}, 495 A.2d 1245 (N.H.
1985) (finding that mortgagee violated duty to mortgagor when (1) sale was
rescheduled and poorly advertised, (2) sale price was so low that it wiped out
substantial equity for homeowners, and (3) mortgagee quickly resold property at
substantially higher price).

One final introductory point: it is possible to take out a second and even a
third mortgage.  The first mortgage has \term{priority} over the second
mortgage:
it will be paid first at foreclosure.  Only if there is money remaining after
the first mortgage is paid off will the holder of the second mortgage be paid. 
As a result of the greater risk involved in second mortgages, they generally
bear higher interest rates than first mortgages.

