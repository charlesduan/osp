The history of mortgages includes many episodes in which financial and legal
innovations hurt borrowers by undercutting the various protections they
traditionally could depend on, such as clarity in knowing who they were dealing
with in a loan transaction. In the most recent iteration of the cycle, the same
financial and legal innovations created systematic trouble on the lender side.
This too is a property story: the securitized pools (a new kind of property
distinct from the property rights in the underlying mortgages) helped create
the moral hazard that led brokers to make bad loans and stick the
mortgage-backed security buyers with toxic junk. The investors in the
securities were unwilling or unable, or sometimes both, to examine individual
loans and instead invested on the theory that \textit{enough} loans in the pool
would pay off to justify the investment, so they didn't pay enough attention to
the quality of the individual underlying loans. Is it possible to split
property up in so many ways that the new rights become dangerous instead of
productive?

\expected{johnson-v-mcintosh}

Finally: Frederic Bastiat wrote, ``When plunder becomes a way of life for a
group of men living together in society, they create for themselves in the
course of time a legal system that authorizes it and a moral code that
glorifies it.'' Does his claim help explain MERS? What about \textit{Johnson
v. M`Intosh}?

