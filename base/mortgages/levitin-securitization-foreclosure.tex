\readingnote{Excerpts reprinted by permission.}
\reading[Levitin, \emph{Securitization, Foreclosure, and the Uncertainty of
Mortgage Title}]{Adam J. Levitin, \textit{The Paper Chase: Securitization,
Foreclosure, and the Uncertainty of Mortgage Title}}
\readingcite{63 \textsc{Duke L.J.} 637 (2013)}

\ldots

\readinghead{II. The Shift in Mortgage Financing to Securitization}

Securitization is a relatively recent development in residential mortgage
lending. Residential mortgages began to be securitized in 1970, but
securitization remained a relatively small part of American housing finance
prior to the 1980s. In 1979 only 10 percent of outstanding mortgages by dollar
amount were securitized. Instead, mortgage lending was primarily a local
affair \dots so mortgage loans were rarely transferred.

\dots By 1983, 20 percent of outstanding mortgages by dollar amount were
securitized, and a decade later fully half of outstanding mortgages by dollar
amount were securitized. Today nearly two-thirds of mortgage dollars
outstanding are securitized.

A firm can raise funds on potentially more advantageous terms if it can borrow
solely against its assets, not its assets and liabilities. Securitization
enabled such borrowing. To do so, a firm sells assets to a legally separate,
specially created entity. The legally separate entity pays for the assets by
issuing debt. Because the entity is designed to have almost no other
liabilities, the debt it issues will be priced simply on the quality of the
transferred assets, without any concern about competing claims to those assets.
Therefore, ensuring that the assets are transferred and are free of competing
claims is central to securitization.

Although residential-mortgage securitization transactions are complex and vary
somewhat depending on the type of entity undertaking the securitization, there
is still a  core standard  transaction. First, a financial  institution  (the
``sponsor''  or ``seller'') assembles a pool of mortgage loans either made
(``originated'')  by  an affiliate of the financial institution or
purchased from unaffiliated third-party originators. Second, the pool of
loans is sold by the sponsor to a special-purpose subsidiary (the
``depositor'') that has no other assets or liabilities and is little more than
a legal entity with a mailbox. This is done to segregate the loans from the
sponsor's assets and liabilities. Third, the depositor sells the loans to a
passive, specially created, single-purpose vehicle (SPV), typically a trust in
the case of residential-mortgage securitization. The trustee will then
typically convey the mortgage notes and security instruments to a document
custodian for safekeeping. The SPV issues certificated debt securities to raise
the funds to pay for the loans. As these debt securities are backed by the cash
flow from the mortgages, they are called mortgage-backed securities (MBS).
\dots

