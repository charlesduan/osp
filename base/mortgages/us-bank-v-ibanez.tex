\reading{U.S. Bank National Association v. Ibanez}

\readingcite{458 Mass. 637 (2011)}

After foreclosing on two properties and purchasing the properties back at the
foreclosure sales, U.S. Bank National Association (U.S. Bank), as trustee for
the Structured Asset Securities Corporation Mortgage Pass-Through Certificates,
Series 2006-Z; and Wells Fargo Bank, N.A. (Wells Fargo), as trustee for ABFC
2005-OPT 1 Trust, ABFC Asset Backed Certificates, Series 2005-OPT 1
(plaintiffs) filed separate complaints in the Land Court asking a judge to
declare that they held clear title to the properties in fee simple. We agree
with the judge that the plaintiffs, who were not the original mortgagees,
failed to make the required showing that they were the holders of the mortgages
at the time of foreclosure. As a result, they did not demonstrate that the
foreclosure sales were valid to convey title to the subject properties, and
their requests for a declaration of clear title were properly denied.

\textit{Procedural history}. On July 5, 2007, U.S. Bank, as trustee, foreclosed
on the mortgage of Antonio Ibanez, and purchased the Ibanez property at the
foreclosure sale. On the same day, Wells Fargo, as trustee, foreclosed on the
mortgage of Mark and Tammy LaRace, and purchased the LaRace property at that
foreclosure sale.

In September and October of 2008, U.S. Bank and Wells Fargo brought separate
actions in the Land Court under G.L. c. 240, {\S} 6, which authorizes actions
``to quiet or establish the title to land situated in the commonwealth or to
remove a cloud from the title thereto.'' The two complaints sought identical
relief: (1) a judgment that the right, title, and interest of the mortgagor
(Ibanez or the LaRaces) in the property was extinguished by the foreclosure;
(2) a declaration that there was no cloud on title arising from publication of
the notice of sale in the Boston Globe; and (3) a declaration that title was
vested in the plaintiff trustee in fee simple. U.S. Bank and Wells Fargo each
asserted in its complaint that it had become the holder of the respective
mortgage through an assignment made after the foreclosure sale.

In both cases, the mortgagors---Ibanez and the LaRaces---did not initially
answer the complaints, and the plaintiffs moved for entry of default
judgment \dots .

On March 26, 2009, judgment was entered against the plaintiffs. The judge ruled
that the foreclosure sales were invalid because, in violation of G.L. c. 244,
{\S} 14, the notices of the foreclosure sales named U.S. Bank (in the Ibanez
foreclosure) and Wells Fargo (in the LaRace foreclosure) as the mortgage
holders where they had not yet been assigned the mortgages. The judge found,
based on each plaintiff's assertions in its complaint, that the plaintiffs
acquired the mortgages by assignment only after the foreclosure sales and thus
had no interest in the mortgages being foreclosed at the time of the
publication of the notices of sale or at the time of the foreclosure
sales.\readingfootnote{1}{In the third case, LaSalle Bank National Association,
trustee for the certificate holders of Bear Stearns Asset Backed Securities I,
LLC Asset-Backed Certificates, Series 2007-HE2 vs. Freddy Rosario, the judge
concluded that the mortgage foreclosure ``was not rendered invalid by its
failure to record the assignment reflecting its status as holder of the
mortgage prior to the foreclosure since it was, in fact, the holder by
assignment at the time of the foreclosure, it truthfully claimed that status in
the notice, and it could have produced proof of that status (the unrecorded
assignment) if asked.''}

The plaintiffs then moved to vacate the judgments. At a hearing on the motions
on April 17, 2009, the plaintiffs conceded that each complaint alleged a
postnotice, postforeclosure sale assignment of the mortgage at issue, but they
now represented to the judge that documents might exist that could show a
prenotice, preforeclosure sale assignment of the mortgages. The judge granted
the plaintiffs leave to produce such documents, provided they were produced in
the form they existed in at the time the foreclosure sale was noticed and
conducted. In response, the plaintiffs submitted hundreds of pages of documents
to the judge, which they claimed established that the mortgages had been
assigned to them before the foreclosures. Many of these documents related to
the creation of the securitized mortgage pools in which the Ibanez and LaRace
mortgages were purportedly included. 

The judge denied the plaintiffs' motions to vacate judgment on October 14, 2009,
concluding that the newly submitted documents did not alter the conclusion that
the plaintiffs were not the holders of the respective mortgages at the time of
foreclosure. We granted the parties' applications for direct appellate review.

\textit{Factual background}. We discuss each mortgage separately, describing
when appropriate what the plaintiffs allege to have happened and what the
documents in the record demonstrate. 

\textit{The Ibanez mortgage}. On December 1, 2005, Antonio Ibanez took out a
\$103,500 loan for the purchase of property at 20 Crosby Street in Springfield,
secured by a mortgage to the lender, Rose Mortgage, Inc. (Rose Mortgage). The
mortgage was recorded the following day. Several days later, Rose Mortgage
executed an assignment of this mortgage in blank, that is, an assignment that
did not specify the name of the assignee.\readingfootnote{11}{This signed and
notarized document states: ``FOR VALUE RECEIVED, the
undersigned hereby grants, assigns and transfers to \hbox to
6em{\leaders\hrule\hfil} all
beneficial interest under that certain Mortgage dated December 1, 2005 executed
by Antonio Ibanez\dots.''} The blank space in the assignment was at some
point stamped with the name of Option One Mortgage Corporation (Option One) as
the assignee, and that assignment was recorded on June 7, 2006. Before the
recording, on January 23, 2006, Option One executed an assignment of the Ibanez
mortgage in blank.

According to U.S. Bank, Option One assigned the Ibanez mortgage to Lehman
Brothers Bank, FSB, which assigned it to Lehman Brothers Holdings Inc., which
then assigned it to the Structured Asset Securities
Corporation,\readingfootnote{12}{The Structured Asset
Securities Corporation is a wholly owned direct subsidiary of Lehman Commercial
Paper Inc., which is in turn a wholly owned, direct subsidiary of Lehman
Brothers Holdings Inc.} which then assigned the mortgage, pooled with
approximately 1,220 other mortgage loans, to U.S. Bank, as trustee for the
Structured Asset Securities Corporation Mortgage Pass-Through Certificates,
Series 2006-Z. With this last assignment, the Ibanez and other loans were
pooled into a trust and converted into mortgage-backed securities that can be
bought and sold by investors---a process known as securitization.

For ease of reference, the chain of entities through which the Ibanez mortgage
allegedly passed before the foreclosure sale is:
\begin{center}
\textbf{Rose Mortgage, Inc. (originator)}

$\downarrow$

\textbf{Option One Mortgage Corporation (record holder)}

$\downarrow$

\textbf{Lehman Brothers Bank, FSB}

$\downarrow$

\textbf{Lehman Brothers Holdings Inc. (seller)}

$\downarrow$

\textbf{Structured Asset Securities Corporation (depositor)}

$\downarrow$

\textbf{U.S. Bank National Association, as trustee for the Structured Asset
Securities Corporation Mortgage Pass-Through Certificates, Series 2006-Z}
\end{center}
According to U.S. Bank, the assignment of the Ibanez mortgage to U.S. Bank
occurred pursuant to a December 1, 2006, trust agreement, which is not in the
record. What is in the record is the private placement memorandum (PPM), dated
December 26, 2006, a 273-page, unsigned offer of mortgage-backed securities to
potential investors. The PPM describes the mortgage pools and the entities
involved, and summarizes the provisions of the trust agreement, including the
representation that mortgages ``will be'' assigned into the trust. According to
the PPM, ``[e]ach transfer of a Mortgage Loan from the Seller [Lehman Brothers
Holdings Inc.] to the Depositor [Structured Asset Securities Corporation] and
from the Depositor to the Trustee [U.S. Bank] will be intended to be a sale of
that Mortgage Loan and will be reflected as such in the Sale and Assignment
Agreement and the Trust Agreement, respectively.'' The PPM also specifies that
``[e]ach Mortgage Loan will be identified in a schedule appearing as an exhibit
to the Trust Agreement.'' However, U.S. Bank did not provide the judge with any
mortgage schedule identifying the Ibanez loan as among the mortgages that were
assigned in the trust agreement.

On April 17, 2007, U.S. Bank filed a complaint to foreclose on the Ibanez
mortgage in the Land Court under the Servicemembers Civil Relief Act
(Servicemembers Act), which restricts foreclosures against active duty members
of the uniformed services.\readingfootnote{13}{As implemented
in Massachusetts, a mortgage holder is required to go to court to obtain a
judgment declaring that the mortgagor is not a beneficiary of the
Servicemembers Act before proceeding to foreclosure. St.1943, c. 57, as amended
through St.1998, c. 142.} In the complaint, U.S. Bank represented that it was
the ``owner (or assignee) and holder'' of the mortgage given by Ibanez for the
property. A judgment issued on behalf of U.S. Bank on June 26, 2007, declaring
that the mortgagor was not entitled to protection from foreclosure under the
Servicemembers Act. In June, 2007, U.S. Bank also caused to be published in the
Boston Globe the notice of the foreclosure sale required by G.L. c. 244, \S~14.
The notice identified U.S. Bank as the ``present holder'' of the mortgage.

At the foreclosure sale on July 5, 2007, the Ibanez property was purchased by
U.S. Bank, as trustee for the securitization trust, for \$94,350, a value
significantly less than the outstanding debt and the estimated market value of
the property. The foreclosure deed (from U.S. Bank, trustee, as the purported
holder of the mortgage, to U.S. Bank, trustee, as the purchaser) and the
statutory foreclosure affidavit were recorded on May 23, 2008. On September 2,
2008, more than one year after the sale, and more than five months after
recording of the sale, American Home Mortgage Servicing, Inc., ``as
successor-in-interest'' to Option One, which was until then the record holder
of the Ibanez mortgage, executed a written assignment of that mortgage to U.S.
Bank, as trustee for the securitization trust. This assignment was recorded on
September 11, 2008.

\textit{The LaRace mortgage}. On May 19, 2005, Mark and Tammy LaRace gave a
mortgage for the property at 6 Brookburn Street in Springfield to Option One as
security for a \$103,200 loan; the mortgage was recorded that same day. On May
26, 2005, Option One executed an assignment of this mortgage in blank.

According to Wells Fargo, Option One later assigned the LaRace mortgage to Bank
of America in a July 28, 2005, flow sale and servicing agreement. Bank of
America then assigned it to Asset Backed Funding Corporation (ABFC) in an
October 1, 2005, mortgage loan purchase agreement. Finally, ABFC pooled the
mortgage with others and assigned it to Wells Fargo, as trustee for the ABFC
2005-OPT 1 Trust, ABFC Asset-Backed Certificates, Series 2005-OPT 1, pursuant
to a pooling and servicing agreement (PSA).

For ease of reference, the chain of entities through which the LaRace mortgage
allegedly passed before the foreclosure sale is:
\begin{center}
\textbf{Option One Mortgage Corporation (originator and record holder)}

$\downarrow$

\textbf{Bank of America}

$\downarrow$

\textbf{Asset Backed Funding Corporation (depositor)}

$\downarrow$

\textbf{Wells Fargo, as trustee for the ABFC 2005-OPT 1, ABFC Asset-Backed
Certificates, Series 2005-OPT 1}
\end{center}
Wells Fargo did not provide the judge with a copy of the flow sale and servicing
agreement, so there is no document in the record reflecting an assignment of
the LaRace mortgage by Option One to Bank of America. The plaintiff did produce
an unexecuted copy of the mortgage loan purchase agreement, which was an
exhibit to the PSA. The mortgage loan purchase agreement provides that Bank of
America, as seller, ``does hereby agree to and does hereby sell, assign, set
over, and otherwise convey to the Purchaser [ABFC], without recourse, on the
Closing Date\dots all of its right, title and interest in and to each
Mortgage Loan.'' The agreement makes reference to a schedule listing the
assigned mortgage loans, but this schedule is not in the record, so there was
no document before the judge showing that the LaRace mortgage was among the
mortgage loans assigned to the ABFC.

Wells Fargo did provide the judge with a copy of the PSA, which is an agreement
between the ABFC (as depositor), Option One (as servicer), and Wells Fargo (as
trustee), but this copy was downloaded from the Securities and Exchange
Commission website and was not signed. The PSA provides that the depositor
``does hereby transfer, assign, set over and otherwise convey to the Trustee,
on behalf of the Trust\dots all the right, title and interest of the
Depositor\dots in and to\dots each Mortgage Loan identified on the
Mortgage Loan Schedules,'' and ``does hereby deliver'' to the trustee the
original mortgage note, an original mortgage assignment ``in form and substance
acceptable for recording,'' and other documents pertaining to each mortgage.

The copy of the PSA provided to the judge did not contain the loan schedules
referenced in the agreement. Instead, Wells Fargo submitted a schedule that it
represented identified the loans assigned in the PSA, which did not include
property addresses, names of mortgagors, or any number that corresponds to the
loan number or servicing number on the LaRace mortgage. Wells Fargo contends
that a loan with the LaRace property's zip code and city is the LaRace mortgage
loan because the payment history and loan amount matches the LaRace loan.

On April 27, 2007, Wells Fargo filed a complaint under the Servicemembers Act in
the Land Court to foreclose on the LaRace mortgage. The complaint represented
Wells Fargo as the ``owner (or assignee) and holder'' of the mortgage given by
the LaRaces for the property. A judgment issued on behalf of Wells Fargo on
July 3, 2007, indicating that the LaRaces were not beneficiaries of the
Servicemembers Act and that foreclosure could proceed in accordance with the
terms of the power of sale. In June, 2007, Wells Fargo caused to be published
in the Boston Globe the statutory notice of sale, identifying itself as the
``present holder'' of the mortgage.

At the foreclosure sale on July 5, 2007, Wells Fargo, as trustee, purchased the
LaRace property for \$120,397.03, a value significantly below its estimated
market value. Wells Fargo did not execute a statutory foreclosure affidavit or
foreclosure deed until May 7, 2008. That same day, Option One, which was still
the record holder of the LaRace mortgage, executed an assignment of the
mortgage to Wells Fargo as trustee; the assignment was recorded on May 12,
2008. Although executed ten months after the foreclosure sale, the assignment
declared an effective date of April 18, 2007, a date that preceded the
publication of the notice of sale and the foreclosure sale.

\textit{Discussion}. The plaintiffs brought actions under G.L. c. 240, {\S} 6,
seeking declarations that the defendant mortgagors' titles had been
extinguished and that the plaintiffs were the fee simple owners of the
foreclosed properties. As such, the plaintiffs bore the burden of establishing
their entitlement to the relief sought. To meet this burden, they were required
``not merely to demonstrate better title \dots than the defendants possess,
but \dots to prove sufficient title to succeed in [the] action.'' There is no
question that the relief the plaintiffs sought required them to establish the
validity of the foreclosure sales on which their claim to clear title rested.

Massachusetts does not require a mortgage holder to obtain judicial
authorization to foreclose on a mortgaged property. With the exception of the
limited judicial procedure aimed at certifying that the mortgagor is not a
beneficiary of the Servicemembers Act, a mortgage holder can foreclose on a
property, as the plaintiffs did here, by exercise of the statutory power of
sale, if such a power is granted by the mortgage itself.

Where a mortgage grants a mortgage holder the power of sale, as did both the
Ibanez and LaRace mortgages, it includes by reference the power of sale set out
in G.L. c. 183, {\S} 21, and further regulated by G.L. c. 244, {\S}{\S} 11-17C.
Under G.L. c. 183, {\S} 21, after a mortgagor defaults in the performance of
the underlying note, the mortgage holder may sell the property at a public
auction and convey the property to the purchaser in fee simple, ``and such sale
shall forever bar the mortgagor and all persons claiming under him from all
right and interest in the mortgaged premises, whether at law or in equity.''
Even where there is a dispute as to whether the mortgagor was in default or
whether the party claiming to be the mortgage holder is the true mortgage
holder, the foreclosure goes forward unless the mortgagor files an action and
obtains a court order enjoining the foreclosure. 

Recognizing the substantial power that the statutory scheme affords to a
mortgage holder to foreclose without immediate judicial oversight, we adhere to
the familiar rule that ``one who sells under a power [of sale] must follow
strictly its terms. If he fails to do so there is no valid execution of the
power, and the sale is wholly void.''

\dots.

One of the terms of the power of sale that must be strictly adhered to is the
restriction on who is entitled to foreclose. The ``statutory power of sale''
can be exercised by ``the mortgagee or his executors, administrators,
successors or assigns.'' Under G.L. c. 244, {\S} 14, ``[t]he mortgagee or
person having his estate in the land mortgaged, or a person authorized by the
power of sale, or the attorney duly authorized by a writing under seal, or the
legal guardian or conservator of such mortgagee or person acting in the name of
such mortgagee or person'' is empowered to exercise the statutory power of
sale. Any effort to foreclose by a party lacking ``jurisdiction and authority''
to carry out a foreclosure under these statutes is void. See Davenport v. HSBC
Bank USA, 275 Mich.App. 344, 347-348 (2007) (attempt to foreclose by party that
had not yet been assigned mortgage results in ``structural defect that goes to
the very heart of defendant's ability to foreclose by advertisement,'' and
renders foreclosure sale void).

A related statutory requirement that must be strictly adhered to in a
foreclosure by power of sale is the notice requirement articulated in G.L. c.
244, {\S} 14. That statute provides that ``no sale under such power shall be
effectual to foreclose a mortgage, unless, previous to such sale,'' advance
notice of the foreclosure sale has been provided to the mortgagee, to other
interested parties, and by publication in a newspaper published in the town
where the mortgaged land lies or of general circulation in that town. ``The
manner in which the notice of the proposed sale shall be given is one of the
important terms of the power, and a strict compliance with it is essential to
the valid exercise of the power.'' See Chace v. Morse, supra (``where a certain
notice is prescribed, a sale without any notice, or upon a notice lacking the
essential requirements of the written power, would be void as a proceeding for
foreclosure''). Because only a present holder of the mortgage is authorized to
foreclose on the mortgaged property, and because the mortgagor is entitled to
know who is foreclosing and selling the property, the failure to identify the
holder of the mortgage in the notice of sale may render the notice defective
and the foreclosure sale void. See Roche v. Farnsworth, supra (mortgage sale
void where notice of sale identified original mortgagee but not mortgage holder
at time of notice and sale). 

For the plaintiffs to obtain the judicial declaration of clear title that they
seek, they had to prove their authority to foreclose under the power of sale
and show their compliance with the requirements on which this authority rests.
Here, the plaintiffs were not the original mortgagees to whom the power of sale
was granted; rather, they claimed the authority to foreclose as the eventual
assignees of the original mortgagees. Under the plain language of G.L. c. 183,
{\S} 21, and G.L. c. 244, {\S} 14, the plaintiffs had the authority to exercise
the power of sale contained in the Ibanez and LaRace mortgages only if they
were the assignees of the mortgages at the time of the notice of sale and the
subsequent foreclosure sale. See In re Schwartz, 366 B.R. 265, 269
(Bankr.D.Mass.2007) (``Acquiring the mortgage after the entry and foreclosure
sale does not satisfy the Massachusetts statute''). See also Jeff-Ray Corp. v.
Jacobson, 566 So.2d 885, 886 (Fla.Dist.Ct.App.1990) (per curiam) (foreclosure
action could not be based on assignment of mortgage dated four months after
commencement of foreclosure proceeding).

The plaintiffs claim that the securitization documents they submitted establish
valid assignments that made them the holders of the Ibanez and LaRace mortgages
before the notice of sale and the foreclosure sale. We turn, then, to the
documentation submitted by the plaintiffs to determine whether it met the
requirements of a valid assignment.

Like a sale of land itself, the assignment of a mortgage is a conveyance of an
interest in land that requires a writing signed by the grantor. In a ``title
theory state'' like Massachusetts, a mortgage is a transfer of legal title in a
property to secure a debt. Therefore, when a person borrows money to purchase a
home and gives the lender a mortgage, the homeowner-mortgagor retains only
equitable title in the home; the legal title is held by the mortgagee. See Vee
Jay Realty Trust Co. v. DiCroce, 360 Mass. 751, 753 (1972), quoting Dolliver v.
St. Joseph Fire \& Marine Ins. Co., 128 Mass. 315, 316 (1880) (although ``as to
all the world except the mortgagee, a mortgagor is the owner of the mortgaged
lands,'' mortgagee has legal title to property). Where, as here, mortgage loans
are pooled together in a trust and converted into mortgage-backed securities,
the underlying promissory notes serve as financial instruments generating a
potential income stream for investors, but the mortgages securing these notes
are still legal title to someone's home or farm and must be treated as such.

Focusing first on the Ibanez mortgage, U.S. Bank argues that it was assigned the
mortgage under the trust agreement described in the PPM, but it did not submit
a copy of this trust agreement to the judge. The PPM, however, described the
trust agreement as an agreement to be executed in the future, so it only
furnished evidence of an intent to assign mortgages to U.S. Bank, not proof of
their actual assignment. Even if there were an executed trust agreement with
language of present assignment, U.S. Bank did not produce the schedule of loans
and mortgages that was an exhibit to that agreement, so it failed to show that
the Ibanez mortgage was among the mortgages to be assigned by that agreement.
Finally, even if there were an executed trust agreement with the required
schedule, U.S. Bank failed to furnish any evidence that the entity assigning
the mortgage---Structured Asset Securities Corporation---ever held the
mortgage to be assigned. The last assignment of the mortgage on record was from
Rose Mortgage to Option One; nothing was submitted to the judge indicating that
Option One ever assigned the mortgage to anyone before the foreclosure sale. 
Thus, based on the documents submitted to the judge, Option One, not U.S. Bank,
was the mortgage holder at the time of the foreclosure, and U.S. Bank did not
have the authority to foreclose the mortgage.

Turning to the LaRace mortgage, Wells Fargo claims that, before it issued the
foreclosure notice, it was assigned the LaRace mortgage under the PSA. The PSA,
in contrast with U.S. Bank's PPM, uses the language of a present assignment
(``does hereby \dots assign'' and ``does hereby deliver'') rather than an
intent to assign in the future. But the mortgage loan schedule Wells Fargo
submitted failed to identify with adequate specificity the LaRace mortgage as
one of the mortgages assigned in the PSA. Moreover, Wells Fargo provided the
judge with no document that reflected that the ABFC (depositor) held the LaRace
mortgage that it was purportedly assigning in the PSA. As with the Ibanez loan,
the record holder of the LaRace loan was Option One, and nothing was submitted
to the judge which demonstrated that the LaRace loan was ever assigned by
Option One to another entity before the publication of the notice and the sale.

Where a plaintiff files a complaint asking for a declaration of clear title
after a mortgage foreclosure, a judge is entitled to ask for proof that the
foreclosing entity was the mortgage holder at the time of the notice of sale
and foreclosure, or was one of the parties authorized to foreclose under G.L.
c. 183, {\S} 21, and G.L. c. 244, {\S} 14. A plaintiff that cannot make this
modest showing cannot justly proclaim that it was unfairly denied a declaration
of clear title. 

We do not suggest that an assignment must be in recordable form at the time of
the notice of sale or the subsequent foreclosure sale, although recording is
likely the better practice. Where a pool of mortgages is assigned to a
securitized trust, the executed agreement that assigns the pool of mortgages,
with a schedule of the pooled mortgage loans that clearly and specifically
identifies the mortgage at issue as among those assigned, may suffice to
establish the trustee as the mortgage holder. However, there must be proof that
the assignment was made by a party that itself held the mortgage. A foreclosing
entity may provide a complete chain of assignments linking it to the record
holder of the mortgage, or a single assignment from the record holder of the
mortgage. The key in either case is that the foreclosing entity must hold the
mortgage at the time of the notice and sale in order accurately to identify
itself as the present holder in the notice and in order to have the authority
to foreclose under the power of sale (or the foreclosing entity must be one of
the parties authorized to foreclose under G.L. c. 183, {\S} 21, and G.L. c.
244, {\S} 14).

The judge did not err in concluding that the securitization documents submitted
by the plaintiffs failed to demonstrate that they were the holders of the
Ibanez and LaRace mortgages, respectively, at the time of the publication of
the notices and the sales. The judge, therefore, did not err in rendering
judgments against the plaintiffs and in denying the plaintiffs' motions to
vacate the judgments. 

We now turn briefly to three other arguments raised by the plaintiffs on appeal.
First, the plaintiffs initially contended that the assignments in blank
executed by Option One, identifying the assignor but not the assignee, not only
``evidence[ ] and confirm[ ] the assignments that occurred by virtue of the
securitization agreements,'' but ``are effective assignments in their own
right.'' But in their reply briefs they conceded that the assignments in blank
did not constitute a lawful assignment of the mortgages. Their concession is
appropriate. We have long held that a conveyance of real property, such as a
mortgage, that does not name the assignee conveys nothing and is void; we do
not regard an assignment of land in blank as giving legal title in land to the
bearer of the assignment. 

Second, the plaintiffs contend that, because they held the mortgage note, they
had a sufficient financial interest in the mortgage to allow them to foreclose.
In Massachusetts, where a note has been assigned but there is no written
assignment of the mortgage underlying the note, the assignment of the note does
not carry with it the assignment of the mortgage. Rather, the holder of the
mortgage holds the mortgage in trust for the purchaser of the note, who has an
equitable right to obtain an assignment of the mortgage, which may be
accomplished by filing an action in court and obtaining an equitable order of
assignment. [Barnes v. Boardman, 149 Mass. 106, 114, 21 N.E. 308 (1889)] (``In
some jurisdictions it is held that the mere transfer of the debt, without any
assignment or even mention of the mortgage, carries the mortgage with it, so as
to enable the assignee to assert his title in an action at law \dots . This
doctrine has not prevailed in Massachusetts, and the tendency of the decisions
here has been, that in such cases the mortgagee would hold the legal title in
trust for the purchaser of the debt, and that the latter might obtain a
conveyance by a bill in equity''). In the absence of a valid written assignment
of a mortgage or a court order of assignment, the mortgage holder remains
unchanged. This common-law principle was later incorporated in the statute
enacted in 1912 establishing the statutory power of sale, which grants such a
power to ``the mortgagee or his executors, administrators, successors or
assigns,'' but not to a party that is the equitable beneficiary of a mortgage
held by another. 

Third, the plaintiffs \dots argue that the use of postsale assignments was
customary in the industry, and point to Title Standard No. 58(3) issued by the
Real Estate Bar Association for Massachusetts, which declares: ``A title is not
defective by reason of \dots [t]he recording of an Assignment of Mortgage
executed either prior, or subsequent, to foreclosure where said Mortgage has
been foreclosed, of record, by the Assignee.'' To the extent that the
plaintiffs rely on this title standard for the proposition that an entity that
does not hold a mortgage may foreclose on a property, and then cure the cloud
on title by a later assignment of a mortgage, their reliance is misplaced
because this proposition is contrary to G.L. c. 183, {\S} 21, and G.L. c. 244,
{\S} 14. If the plaintiffs did not have their assignments to the Ibanez and
LaRace mortgages at the time of the publication of the notices and the sales,
they lacked authority to foreclose under G.L. c. 183, {\S} 21, and G.L. c. 244,
{\S} 14, and their published claims to be the present holders of the mortgages
were false. Nor may a postforeclosure assignment be treated as a preforeclosure
assignment simply by declaring an ``effective date'' that precedes the notice
of sale and foreclosure, as did Option One's assignment of the LaRace mortgage
to Wells Fargo. Because an assignment of a mortgage is a transfer of legal
title, it becomes effective with respect to the power of sale only on the
transfer; it cannot become effective before the transfer. 

However, we do not disagree with Title Standard No. 58(3) that, where an
assignment is confirmatory of an earlier, valid assignment made prior to the
publication of notice and execution of the sale, that confirmatory assignment
may be executed and recorded after the foreclosure, and doing so will not make
the title defective. A valid assignment of a mortgage gives the holder of that
mortgage the statutory power to sell after a default regardless whether the
assignment has been recorded. Where the earlier assignment is not in recordable
form or bears some defect, a written assignment executed after foreclosure that
confirms the earlier assignment may be properly recorded. A confirmatory
assignment, however, cannot confirm an assignment that was not validly made
earlier or backdate an assignment being made for the first time. Where there is
no prior valid assignment, a subsequent assignment by the mortgage holder to
the note holder is not a confirmatory assignment because there is no earlier
written assignment to confirm. In this case, based on the record before the
judge, the plaintiffs failed to prove that they obtained valid written
assignments of the Ibanez and LaRace mortgages before their foreclosures, so
the postforeclosure assignments were not confirmatory of earlier valid
assignments.

Finally, we reject the plaintiffs' request that our ruling be prospective in its
application. A prospective ruling is only appropriate, in limited
circumstances, when we make a significant change in the common law. We have not
done so here. The legal principles and requirements we set forth are well
established in our case law and our statutes. All that has changed is the
plaintiffs' apparent failure to abide by those principles and requirements in
the rush to sell mortgage-backed securities.

\textit{Conclusion}. For the reasons stated, we agree with the judge that the
plaintiffs did not demonstrate that they were the holders of the Ibanez and
LaRace mortgages at the time that they foreclosed these properties, and
therefore failed to demonstrate that they acquired fee simple title to these
properties by purchasing them at the foreclosure sale.

Judgments affirmed.

\opinion \textsc{Cordy}, J. (concurring, with whom Botsford, J., joins).

I concur fully in the opinion of the court, and write separately only to
underscore that what is surprising about these cases is not the statement of
principles articulated by the court regarding title law and the law of
foreclosure in Massachusetts, but rather the utter carelessness with which the
plaintiff banks documented the titles to their assets. There is no dispute that
the mortgagors of the properties in question had defaulted on their
obligations, and that the mortgaged properties were subject to foreclosure.
Before commencing such an action, however, the holder of an assigned mortgage
needs to take care to ensure that his legal paperwork is in order. Although
there was no apparent actual unfairness here to the mortgagors, that is not the
point. Foreclosure is a powerful act with significant consequences, and
Massachusetts law has always required that it proceed strictly in accord with
the statutes that govern it. As the opinion of the court notes, such strict
compliance is necessary because Massachusetts is both a title theory State and
allows for extrajudicial foreclosure.

The type of sophisticated transactions leading up to the accumulation of the
notes and mortgages in question in these cases and their securitization, and,
ultimately the sale of mortgaged-backed securities, are not barred nor even
burdened by the requirements of Massachusetts law. The plaintiff banks, who
brought these cases to clear the titles that they acquired at their own
foreclosure sales, have simply failed to prove that the underlying assignments
of the mortgages that they allege (and would have) entitled them to foreclose
ever existed in any legally cognizable form before they exercised the power of
sale that accompanies those assignments. The court's opinion clearly states
that such assignments do not need to be in recordable form or recorded before
the foreclosure, but they do have to have been effectuated \dots.

