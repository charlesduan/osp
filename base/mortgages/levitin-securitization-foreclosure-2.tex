\readingnote{Excerpts reprinted by permission.}
\reading[Levitin, \emph{Securitization, Foreclosure, and the Uncertainty of
Mortgage Title}]{Adam J. Levitin, \textit{The Paper Chase: Securitization,
Foreclosure, and the Uncertainty of Mortgage Title}}
\readingcite{63 \textsc{Duke L.J.} 637 (2013)}

\readinghead{Securitization-Era Mortgage Title Systems}

\dots MERS [Mortgage Electronic Registration Systems, Inc.] is a private,
contractual superstructure that is grafted onto the public land-recordation
system. Financial institutions that are members of MERS register the loans they
service (but do not necessarily own) with the MERS System electronic database.
Each loan receives a unique identifier known as a MERS Identification Number
(MIN). The MIN is sometimes stamped on the note or sometimes simply recorded in
the lender's own records. MERS is then inserted in the local land records as
the mortgagee, instead of the actual lender. Sometimes this involves an
assignment of the mortgage from the lender to MERS, but the more prevalent
arrangement has MERS recorded as the original mortgagee, thereby obviating any
recordation of assignments. MERS serves as the mortgagee of record, but only as
a nominee for the actual lender and supposedly for its successors and assigns.
The language included in MERS mortgages is that MERS is acting ``solely as
nominee for Lender and Lender's successors and assigns.'' MERS claims no
beneficial interest whatsoever in the loan.

MERS's goal is to immobilize mortgage title through a common-agency structure by
acting as nominee for the lender and those subsequent transferees of the lender
that are members of MERS. Although legal title remains in MERS's name,
subsequent transfers are supposed to be tracked in MERS's database.

Thus, MERS aims to achieve the priority and enforcement benefits of public
recordation while tracking beneficial ownership title in its own database.
MERS's operation has two important implications. First, instead of paying
county recordation and transfer fees, financial institutions pay only for MERS
membership and MERS transaction fees. MERS thus o[FB00?]ers potential cost
savings in the securitization process through the avoidance of local recording
fees. Second, MERS{}'s electronic database, not the county land records,
represents the main evidentiary source for determining who is currently the
real party in interest on a mortgage.

In theory, MERS's database tracks two distinct characteristics: the identity of
the party with the rights to service the mortgage (often an agent for the
trustee for the trust created for the ultimate beneficial owners of the
mortgage loan) and the legal title to mortgages (for example, the trustee for
the trust created for the ultimate beneficial owners of the mortgage). MERS's
publicly available records do not track chain of title. It is impossible for
outsiders to determine if transfers were made in the MERS system and when.
Instead, MERS publicly tracks only the current servicer and sometimes the
current beneficial owner of a loan.

A major problem with MERS as a title system is that it is not accurate and
reliable in terms of what it reports. MERS's members are nominally required to
report transfers of mortgage servicing rights to MERS, but MERS does not
actually compel reporting of servicing-rights transfers, and there is little
incentive to be punctual with reporting. Indeed, the lack of record validation
combined with voluntary reporting has led a federal judge to describe MERS as
``the Wikipedia of land registration systems.'' Not surprisingly, the
information in the MERS database is often inaccurate or incomplete.

MERS does not even formally require any reporting of legal title to the
mortgages, much less of transfers of legal title; any information about legal
title is supplied through strictly voluntary reporting. \dots 

MERS's database functions as a do-it-yourself private mortgage recordation
system. Historically, MERS itself has had only around fifty employees who
perform corporate and technology support functions. Employees of MERS's members
carry out most of the tasks done in MERS's name, including the making of
entries in the MERS database. These employees of MERS's members are listed as
assistant secretaries or vice presidents of MERS, but they have no actual
employment relationship with MERS. There are over twenty thousand of these
``corporate signing officers.'' Accordingly, a transfer of either
servicing or legal title in the MERS system involves nothing more than an
employee of a MERS member entering the transfer in the MERS database.

A transfer within the MERS system involves voluntary self-reporting and nothing
more and therefore fails to incentivize timely, accurate reporting. There are
no formalities to a transfer in the MERS system. As a result, MERS may not in
fact know who its principal is within the common-agency arrangement at any
given point in time because MERS is relying on reporting from its members.

