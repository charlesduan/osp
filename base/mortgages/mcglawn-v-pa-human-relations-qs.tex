\expected{mcglawn-v-pa-human-relations}

\item Some commentary on unaffordable mortgages asks ``why would borrowers take
out loans that were doomed to foreclosure?''  Does the opinion offer any
insights into this question?  See Oren Bar-Gill, The Law, Economics and
Psychology of Subprime Mortgage Contracts, 94 Cornell L. Rev. 1073 (2009); see
also Jeff Sovern, Preventing Future Economic Crises Through Consumer Protection
Law or How the Truth in Lending Act Failed the Subprime Borrowers, 71 Ohio St.
L.J. 763 (2010) (arguing that the explanation of key terms, even in
non-predatory loans, was simply insufficient for ordinary borrowers to
understand).  Here's another question: ``why would \textit{lenders} give out
loans that were doomed to foreclosure?''  As it turns out, given the collapse
of the housing market, most foreclosures do not return enough to the lender to
pay back the initial loan.


\item Resistance to helping homeowners at risk of foreclosure often focuses on
the problem of ``moral hazard''---if people weren't forced either to pay back
the loans on the terms on which those loans were granted or to lose their
homes, some argued, that would encourage irresponsible borrowing.  More
broadly: when we seek to hold one party responsible for harm, we often make
another party less responsible.  As a result of the subprime mortgage collapse,
many banks failed or were bailed out by the federal government.  However,
homeowners generally were not bailed out.

\item For some larger context, consider this excerpt from Ta-Nehisi Coates'
\textit{The
Case for Reparations}, \textsc{The Atlantic}, May 2014:
\begin{quotation}
In 2010, Jacob S. Rugh, then a doctoral candidate at Princeton, and the
sociologist Douglas S. Massey published a study of the recent foreclosure
crisis. Among its drivers, they found an old foe: segregation. Black home
buyers---even after controlling for factors like creditworthiness---were
still more likely than white home buyers to be steered toward subprime loans.
Decades of racist housing policies by the American government, along with
decades of racist housing practices by American businesses, had conspired to
concentrate African Americans in the same neighborhoods.\dots [T]hese
neighborhoods were filled with people who had been cut off from mainstream
financial institutions. When subprime lenders went looking for prey, they found
black people waiting like ducks in a pen.

``High levels of segregation create a natural market for subprime lending,''
Rugh and Massey write, ``and cause riskier mortgages, and thus foreclosures, to
accumulate disproportionately in racially segregated cities' minority
neighborhoods.''

Plunder in the past made plunder in the present efficient. The banks of America
understood this. In 2005, Wells Fargo promoted a series of Wealth Building
Strategies seminars. Dubbing itself ``the nation's leading originator of home
loans to ethnic minority customers,'' the bank enrolled black public figures in
an ostensible effort to educate blacks on building ``generational wealth.'' But
the ``wealth building'' seminars were a front for wealth theft. In 2010, the
Justice Department filed a discrimination suit against Wells Fargo alleging
that the bank had shunted blacks into predatory loans regardless of their
creditworthiness. This was not magic or coincidence or misfortune. It was
racism reifying itself. According to The New York Times, affidavits found loan
officers referring to their black customers as ``mud people'' and to their
subprime products as ``ghetto loans.''

``We just went right after them,'' Beth Jacobson, a former Wells Fargo loan
officer, told The Times. ``Wells Fargo mortgage had an emerging-markets unit
that specifically targeted black churches because it figured church leaders had
a lot of influence and could convince congregants to take out subprime loans.''

In 2011, Bank of America agreed to pay \$355 million to settle charges of
discrimination against its Countrywide unit. The following year, Wells Fargo
settled its discrimination suit for more than \$175 million. But the damage had
been done. In 2009, half the properties in Baltimore whose owners had been
granted loans by Wells Fargo between 2005 and 2008 were vacant; 71 percent of
these properties were in predominantly black neighborhoods.
\end{quotation}

\item African-American and other minority borrowers were disproportionately
steered to expensive subprime loans even though they qualified for cheaper
conventional loans---high-income African American borrowers were six times as
likely to get subprime loans as white borrowers with similar incomes.  However,
it is not the case, as is sometimes asserted, that unwise loans to
African-Americans driven by federal mandates for equality in lending were
responsible for the crash.  In fact, institutions subject to federal fair
lending rules made loans which were less likely to default than loans from
institutions that were not subject to such rules.  David Min,
\emph{Faulty
Conclusions Based on Shoddy Foundations} (Feb. 2011),
\url{https://cdn.americanprogress.org/wp-content/uploads/issues/2011/02/pdf/pinto.pdf};
\textsc{National Consumer Law Center},
\textsc{Why
Responsible Mortgage Lending Is a Fair Housing Issue} (Feb. 2012),
\url{https://www.nclc.org/images/pdf/credit_discrimination/fair-housing-brief.pdf}.


\item In recent years, legislatures and regulators have attempted to regulate
mortgage lending to stamp out the worst origination abuses, such as the yield
spread premium.  Much regulation focuses on the concept of ``suitability'':
loans that the borrowers are likely to be able to repay, rather than loans
based merely on the market value of the house. Loans based on the value of
property alone, without sufficient attention to borrower characteristics,
encouraged lenders to believe that they could profit even in case of a default,
or sometimes that they could profit even more from default than from payment. 
In 2014, the Consumer Financial Protection Bureau (CFPB) issued rules on
high-cost
loans and homeownership counseling, implementing the Home Ownership and Equity
Protections Act and subsequent additions.
Consumer Financial Protection Bureau, \emph{High-Cost Mortgage and Homeownership
Counseling Amendments to the Truth in Lending Act (Regulation Z) and
Homeownership Counseling Amendments to the Real Estate Settlement Procedures Act
(Regulation X)} (last updated apr 21 2015),
\url{http://www.consumerfinance.gov/regulations/high-cost-mortgage-and-homeownership-counseling-amendments-to-regulation-z-and-homeownership-counseling-amendments-to-regulation-x/}.
Under these rules, loans considered ``high cost'' are subject to a number of
limitations; high cost loans are those that specify high interest rates, high
fees rolled into the mortgage amount (as in \textit{McGlawn}), or prepayment
penalties that last more than 36 months or exceed more than 2\% of the prepaid
amount.  Under the new rules, for high-cost loans, balloon payments are
generally banned, with limited exceptions.  Creditors are prohibited from
charging prepayment penalties and financing points and fees.  Late fees are
restricted to four percent of the payment that is past due, and certain other
fees are limited or banned.  Before a lender gives a high-cost mortgage, they
must confirm with a federally approved counselor that the borrower has received
counseling on the advisability of the mortgage.


\item Whether or not borrowers are seeking high-cost loans, lenders are now
subject to a rule requiring them to assess a borrower's ability to repay,
though that rule does not cover home equity lines of credit, timeshare plans,
reverse mortgages, or temporary loans.  The lender must not use a ``teaser'' or
introductory interest rate to calculate the borrower's ability to repay; for
adjustable-rate mortgages, it must consider ability to repay under the highest
possible rate allowed by the mortgage.  Certain so-called ``plain vanilla''
mortgages---fixed-rate, fully amortized (with no balloon payments) loans for
no longer than 30 years---are presumptively acceptable under the regulations. 
In addition, lenders have to make counseling information available to all
borrowers.  Although loan information remains complex, the CFPB has tested
different versions of mandatory disclosures, trying to find the
most
understandable ways of communicating the costs and risks of mortgages to
non-lawyers.  See CFPB Finalizes ``Know Before You Owe'' Mortgage Forms, Nov.
20, 2013,
\url{http://www.consumerfinance.gov/newsroom/cfpb-finalizes-know-before-you-owe-mortgage-forms/}.
Take a look at the forms.
(\url{http://www.consumerfinance.gov/newsroom/cfpb-finalizes-know-before-you-owe-mortgage-forms/})
Now that you have read this far, can you understand them?

