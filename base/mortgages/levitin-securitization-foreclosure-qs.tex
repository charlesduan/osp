\expected{levitin-securitization-foreclosure}

\item You may not feel that you fully understand securitization.  It will get
clearer with time.  Perhaps the most important thing to understand is that the
entity that claims to own, and tries to enforce, the mortgage debt in case of
default is usually not the entity that originated the loan.  It's common to
discuss ``banks'' and ``lenders'' without paying much attention to the details
of the actual mortgage transactions, and the problem is worsened because the
entities involved are often related and even bear highly similar names.  But
lawyers often need more precision.

\expected{okeeffe-v-snyder}

\item Among other things, non-originator owners can claim that equitable
defenses---such as fraudulent inducement, which was unconscionably common in
the run-up to the mortgage crisis---are unavailable to homeowners/mortgagors
under the ``holder in due course'' doctrine.  The holder in due course doctrine
is similar to the rule, discussed in \textit{O'Keeffe v. Snyder}, that a good
faith purchaser who buys property from a fraudster acquires good title, even
though the fraudster did not have good title.  With a mortgage, that means that
a homeowner who was deceived into taking a predatory loan, as discussed in the
next section, is still bound to pay back the loan according to its terms as
long as the mortgage was transferred to a holder in due course.  \textit{See,
e.g.}, Kurt Eggert, \textit{Held Up in Due Course: Predatory Lending,
Securitization, and the Holder-in-Due-Course Doctrine}, 35 \textsc{Creighton L.
Rev}. 502 (2002).  Recently, some reforms have attempted to limit the holder in
due course doctrine, at least with respect to loans with specific bad features.


\item Another important thing to understand about securitization is that it
involves the creation of new property rights from old.  Investors in
mortgage-backed securities do not own individual mortgages. Rather, they own
the right to benefit from the stream of payments from mortgagors to the trusts
that hold the mortgages.  This right has been turned into a separate property
right through the magic of securitization.  But the \textit{value} of this
right is still, as investors discovered to their sorrow, dependent on the value
of the underlying assets.

