\reading{McGlawn v. Pennsylvania Human Relations Commission}

\readingcite{891 A.2d 757 (Commonwealth Ct. Penn. 2006)}

This case involves an issue of first impression: whether the Pennsylvania Human
Relations Act (Act) extends to a mortgage broker's predatory lending activities
known as ``reverse redlining.''\readingfootnote{1}{In \emph{United Cos. Corp. v.
Sargeant}, 20 F.Supp.2d 192 (D.Mass.1998), the United States District Court
defined ``redlining'' as[:]

``the practice of denying the extension of credit to
specific geographic areas due to the income, race or ethnicity of its
residents. The term was derived from the actual practice of drawing a red line
around certain areas in which credit would be denied. Reverse redlining is the
practice of extending credit on unfair terms to those same communities.''} We
affirm the Commission's holding that the Act prohibits reverse redlining.
However, we vacate part of the Commission's award of actual damages and remand
for further proceedings. 

Respondent McGlawn and McGlawn, Inc. (Broker) a state-licensed mortgage broker,
and Respondent Reginald McGlawn (Reginald McGlawn) petition for review of the
decision of the Pennsylvania Human Relations Commission (Commission). The
decision held Respondents violated Sections 5(h)(4)(loan
provision)\readingfootnote{2}{Section 5(h)(4) of the Act, 43 P.S. {\S}
955(h)(4), makes it unlawful to

``[d]iscriminate against any person in the terms or conditions of any loan of
money, whether or not secured by a mortgage or otherwise for the acquisition,
construction, rehabilitation, repair or maintenance of housing accommodation or
commercial property because of\dots race\dots.''} and 5(h)(8)(i)(real
estate transaction provision)\readingfootnote{3}{Section 5(h)(8)(i) of the Act,
43 P.S. \S~955(h)(8)(i), makes it an unlawful to[:]

``[d]iscriminate in real
estate related transactions, as described by and subject to the following:
(i)[i]t shall be unlawful for any person or other entity whose business
includes engaging in real estate-related transactions to discriminate against
any person in making available such a transaction or in the terms [or]
conditions of such a transaction because of race{\dots}.''} of the Act by
discriminating against Complainants and other similar situated persons
(collectively, Complainants), in mortgage loan transactions, because of their
race and the racial composition of their neighborhoods. The Commission's final
order directed Respondents to (1) cease and desist from discriminating against
African Americans because of their race; (2) pay Complainants actual
damages;\readingfootnote{4}{The Commission awarded Complainants actual damages
in these amounts: Taylor, \$45,770.68; Poindexter, \$24,447.80; Brunson,
\$63,996.34; Jackson, \$74,875.74; Slaughter, \$29,685.46; Jacobs, \$47,549.62;
Hawkins, \$41,952.72; Miles, \$101,562.81; Watts, \$116,298.87; and Norwood,
\$154,209.11.} (3) pay Complainants damages for embarrassment and
humiliation;\readingfootnote{5}{The Commission awarded Complainants damages for
embarrassment and humiliation in the following amounts: Taylor, \$25,000.00;
Poindexter, \$15,000.00; Brunson, \$15,000.00; Jackson, \$20,000.00; Slaughter,
\$20,000.00; Jacobs, \$20,000.00; Hawkins, \$20,000.00; Miles, \$10,000.00;
Watts, \$20,000.00; and Norwood, \$20,000.00.} and (4) pay a civil penalty of
\$25,000.00. Further, the Commission's order directed Broker to (5) provide
employee training to its employees designed to educate them in their
responsibility to treat clients in a non-discriminatory manner consistent with
the provisions of the Act; and to (6) develop and implement a record-keeping
system designed to accurately record information about Broker's charges in all
mortgage transactions.\readingfootnote{6}{In particular, Broker must accurately
record the following data for each transaction: (a) the dollar amount and
percentage of the broker's fee charged; (b) any other fees paid; (c) the amount
and type of the loan; and (d) the employee involved in the transaction. Broker
shall submit this information to the Commission on a bi-annual basis for three
years.} The order also required Respondents to report the means of compliance
and directed the Commission to contact the Department of Banking so that it may
take such licensing action as it deemed appropriate.

\readinghead{I. Background}

\readinghead{A.}

Broker, a corporation which brokers mortgage loans, refinancing and insurance
for its customers, was founded in 1985 by its chief officers, Reginald McGlawn,
and his brother, Anthony McGlawn. Reginald McGlawn is Broker's mortgage loan
specialist, and Anthony McGlawn is Broker's insurance specialist. Broker also
employs other McGlawn family members.

Broker specializes in arranging sub-prime mortgage loans for its customers. The
prime lending market provides credit to those considered good credit risks. The
sub-prime lending market provides credit to people the financial industry
considers enhanced credit risks. These people generally have a flawed credit
history or a debt-to-income ratio outside the range the financial industry
considers acceptable for prime credit. As discussed hereafter, sub-prime
interest rates are usually two to three percentage points higher than prime
rates.

In 1998-2000, Broker arranged sub-prime mortgage loans for Complainants, who own
real property in Philadelphia County. Broker is an African American-owned
company. Complainants are African Americans who reside in predominantly African
American neighborhoods.

In April 2001, Complainant Lucrecia Taylor (Taylor) filed a verified complaint
with the Commission alleging Broker unlawfully discriminated against her in the
terms and conditions of a real estate-related transaction and loan of money
because of her race and the racial composition of her neighborhood, African
American. Specifically, Taylor alleged Broker targeted her, as an African
American, for a mortgage loan transaction containing predatory and unfair terms
in violation of the Act's loan and real estate transaction provisions.
Significantly, Taylor stated her allegations were made not only on her own
behalf, but on behalf of all other similarly situated persons affected by
Broker's discriminatory practices. After the pleadings were closed, the
Commission notified Taylor and Broker that probable cause existed to credit
Taylor's allegations.

In August 2002, Complainant Lynn Poindexter (Poindexter) filed a like complaint
against Broker on behalf of herself and all other similarly situated persons.
The Commission subsequently found probable cause existed to credit Poindexter's
allegations. The Commission consolidated the two cases\dots.

The Commission was thereafter able to identify other individuals affected by
Broker's alleged discriminatory practices\dots.

\readinghead{B.}

In its decision, the Commission found Broker engaged in predatory brokering
activities regarding all Complainants. Those actions resulted in unfair and
predatory mortgage loans. It also found Broker engaged in an aggressive
marketing plan targeting African Americans and African American neighborhoods
in the Philadelphia area. Nearly all of Complainants contacted Broker in
response to radio, television and newspaper advertisements.

Broker's predatory practices, the Commission noted, included arranging loans
containing onerous terms such as high interest rates, pre-payment penalties,
balloon payments and mandatory arbitration clauses. In addition, Broker charged
Complainants high broker fees, undisclosed fees, yield spread premiums and
various other additional closing costs. Broker's predatory practices also
included falsification of information on loan documents, failure to disclose
information regarding terms of the loan, and high pressure sales tactics.

\dots The seminal case prohibiting reverse redlining is
\textit{Hargraves v. Capital City Mortgage Corp.,}
140 F.Supp.2d 7 (D.D.C.2000). There, the United States District Court adopted
a two-pronged test for discrimination under the FHA [Fair Housing Act] based on
reverse redlining. First, the plaintiffs must establish the defendant's lending
practices and loan terms were predatory and unfair. \textit{Hargraves}. Second,
the plaintiffs must establish that defendant intentionally targeted them
because of their race or that the defendant's lending practices had a disparate
impact on the basis of race. 

Citing \textit{Hargraves} and the opinions of Complainants' experts, the
Commission concluded Complainants established a prima facie reverse redlining
claim against Broker under the \textit{Hargraves} test. The Commission rejected
Broker's arguments that (1) it did not discriminate because it did not arrange
loans for non-African Americans on more preferable terms, (2) it had a
legitimate business necessity for its actions, (3) it is not responsible for
the terms and conditions of the loans or the disclosure of information relating
to the loans, and (4) all mortgage brokers are predators.

As a result, the Commission held Respondents violated the loan provisions and
the real estate transaction provisions of the Act by unlawfully discriminating
against Complainants in the terms and conditions of real estate-related
transactions\dots.

\readinghead{III. Substantial Evidence}

Respondents\dots assert the Commission's conclusion Broker engaged in reverse
redlining is not supported by substantial
evidence.\readingfootnote{14}{``Substantial evidence is
such relevant evidence as a reasonable mind might accept as adequate to support
a conclusion.'' ``Further, substantial evidence supporting a finding of racial
discrimination may be circumstantial and based on inferences.''} In particular,
Respondents maintain the evidence does not show Broker engaged in predatory
lending practices or targeted African Americans.

 ``It is well settled that the party asserting discrimination bears the burden
of proving a prima facie case of discrimination.'' ``Once a prima facie case is
established, a rebuttable presumption of discrimination arises.'' ``The burden
then shifts to the defendant to show some legitimate, nondiscriminatory reason
for its action.''\ldots

\readinghead{A. Predatory Lending}

Respondents first argue Broker did not engage in predatory or unfair lending
practices because it did not approve Complainants' loans or lend them the
money. Therefore, they were not responsible either for the terms and conditions
of Complainants' loans or for the disclosure of information related to the
loans. Those responsibilities belong to the lending institutions that set the
terms and approved the loans.

The Commission accepted the testimony of Complainants' expert witnesses.
Michelle Lewis, President and Chief Executive Officer of Northwest Counseling
Service, Inc. (Complainants' first expert), stated that a mortgage broker is
significantly involved in making the loan. The broker is the middleman who
creates the loan opportunity. The broker's customer relies on the broker's
expertise in lending matters and has an expectation that the broker will be
able to obtain the best available deal. 

The Commission also relied on Ira Goldstein, Director of Public Policy and
Program Assessment for the Reinvestment Fund (Complainants' second expert), who
testified that, in brokered transactions, the broker's customer---the
borrower, never actually meets the lender. As a result, in the borrower's mind,
the broker is the lender. Complainants' second expert also testified that in
loan transactions where a yield spread
premium\readingfootnote{17}{In Taylor v. Flagstar Bank, FSB,
181 F.R.D. 509 (M.D.Ala.1998), the United States District Court defined ``yield
spread premiums'' as:

payments made by a mortgage lender to a mortgage
broker on an ``above par'' loan brought to the lender by the broker. To be
``above par'' is to be above the going rate, to be above the lowest rate that a
lender will offer without charging ``discount points.'' In crude terms,
therefore, the yield spread premium is (allegedly) simply a payment made by the
lender to the broker in return for the broker having brought the lender a high
interest loan.} is used, the broker plays a significant role in establishing
the interest rate of the loan. 

As additional support for its determination, the Commission cited Reginald
McGlawn's testimony. He testified, ``[W]hen people come to us, I provide
loans.'' Reginald McGlawn also testified he chooses which lender receives the
borrower's loan application. He also stated he sets the broker fee and gives
the borrower the option of using a yield spread premium, which has the effect
of increasing the interest rate. 

\readinghead{1.}

There is substantial evidence to support the Commission's determination that
Respondents engaged in brokering activities that resulted in predatory and
unfair loans.

\dots Broker's activities were a substantial part of the loan transactions at
issue. In particular, Broker selected which lender received Complainants' loan
applications. Broker was the sole negotiator for Complainants with the ultimate
lender. Also, Broker influenced the ultimate interest rates in loans involving
yield spread premiums. Further, Broker received substantial sums directly from
loan proceeds, such as broker fees and insurance premiums. As the Commission
properly concluded, these items are considered terms of a loan
transaction\dots.

\readinghead{2.}

We next review the Commission's determination that Respondents' practices were
predatory and unfair.\dots

In finding Broker arranged predatory and unfair loans for Complainants, the
Commission applied the \textit{Hargraves} definition of ``predatory lending
practices.'' The \textit{Hargraves} Court stated predatory lending practices
are indicated by loans with unreasonably high interest rates and loans based on
the value of the asset securing the loan rather than the borrower's capacity to
repay it. The Court also recognized predatory lending practices include ``loan
servicing procedures in which excessive fees are charged.''

The Commission also noted the New Jersey Superior Court's decision in
\textit{Assocs. Home Equity Servs., Inc. v. Troup},
343 N.J.Super. 254, 778 A.2d 529 (2001). The \textit{Troup} Court explained
the term ``predatory lenders'' refers to those lenders who target certain
populations for credit on unfair or onerous terms. Characteristically,
predatory loans do not fit the borrower either because the borrower's needs are
not met or because the terms are so onerous there is a strong likelihood the
borrower will be unable to repay the loan. 

In determining what lending practices are predatory and unfair, the Commission
also accepted as credible Complainants' experts opinions as to what constitutes
a predatory loan. Complainants' first expert testified there are a number of
loan features which are characteristic of a predatory loan. They include high
interest rates, paying off a low interest mortgage with a high interest
mortgage, payment of points, yield spread premiums, high broker fees,
undisclosed fees, balloon payments, pre-payment penalties, arbitration clauses
and fraud. A predatory and unfair loan may include any combination of these
characteristics. 

Complainants' second expert testified that, even assuming a borrower is an
enhanced credit risk, the difference in interest rates between a sub-prime and
prime market loan is usually no greater than three percentage points. Anything
higher than a three-point difference is indicative of a predatory loan. This
expert also testified predatory loan practices include, among other things:
flipping (successive refinancing of the same loan); hiding critical terms,
establishing loan terms the borrower cannot meet; packing (including
unnecessary products such as insurance policies); charging improper fees for
items outside the settlement sheet; creation of false documents; and failing to
advise borrowers of their rescission rights. 

The Commission examined the terms of Complainants' loans and their experiences
with Respondents in light of the foregoing. We briefly review the Commission's
findings regarding Complainants Taylor and Poindexter.

\textbf{Taylor.} Taylor contacted Broker in October 2000 in order to obtain a
refinancing loan of \$10,000.00 to make some emergency home repairs (leaky
roof, doors and windows, plumbing repair). In 2000, she owed \$7,300.00 on her
home. Her home mortgage had a 3\% interest rate with a monthly payment of
\$110.90. Taylor's sole income source was social security disability. 

Broker arranged a 30-year mortgage loan for Taylor with Delta Funding
Corporation (Delta) in the amount of \$20,500.00 with a 13.09\% interest rate.
Taylor was not given an opportunity to review any of the documents before
signing them. Taylor was told to sign the documents. 

The Commission found Taylor's loan transaction had several predatory
characteristics. Taylor's was charged \$4,276.60 in total settlement costs, or
approximately 20\% of the loan.\readingfootnote{20}{Taylor
was charged \$440.00 for a broker fee and \$410.00 for a yield spread premium.
Taylor testified she was unaware her loan contained a yield spread premium or
that it would raise her interest rate. Her loan also included a \$370.31 charge
for a homeowner's insurance policy even though she was covered by another
policy. Taylor was unaware of this charge and stated her house was already
insured. Taylor's settlement sheet also reflected charges for debts she did not
owe at the time of closing, including a \$83.81 water bill and two ambulance
bills (\$477.50 and \$250.00).  Though Broker told Taylor this money would be
returned to her, she never received it.} Two days after Taylor signed the loan
documents, her uncle reviewed them and advised her to cancel the loan. Taylor
called Aaron McGlawn, a Broker employee, and stated she did not want the loan.
He did not advise Taylor she could legally rescind the loan within a three-day
period; rather, he told Taylor she could cancel the loan if she had the money
to pay the people Broker already paid. 

The settlement sheet indicates Taylor received \$8,902.07. At closing, Reginald
McGlawn informed Taylor she owed an additional \$1,200.00 fee because of where
she lived. Anthony McGlawn cashed the check and gave Taylor the money. He then
asked Taylor for the \$1,200.00 fee. Taylor paid the fee out of the cash; but
she was not given a receipt. This fee was not reflected on the settlement
sheet. 

Complainants' second expert reviewed Taylor's loan transaction. He noted several
predatory characteristics. First, Taylor's 13.09\% interest rate was
substantially above the three-point spread between sub-prime and prime loans.
The Commission noted Broker arranged a loan for Taylor at twice the amount she
requested and increased her interest rate from 3\% to 13.09\%. Such loans are
considered to be deceptive and
detrimental.\readingfootnote{21}{In addition to the higher
interest rate, the Commission found Broker's charges for the homeowners' policy
and broker fees to be predatory and unfair. It also found Broker's refusal to
either inform Taylor of her rescission rights or permit her to cancel her loan
within the three-day rescission period was a predatory practice intended to
process the loan transaction despite Taylor's desire to cancel it.} In
addition, Taylor's loan included an additional undisclosed \$1,200.00 broker
fee. 

The Commission found Broker engaged in predatory brokering activities on
Taylor's behalf. These Broker actions resulted in a predatory and unfair
refinancing loan. This finding is supported by substantial evidence.

\textbf{Poindexter.} Poindexter testified by deposition that she acquired her
home as a gift from her grandfather and owned it free and clear. She described
the neighborhood as being African American. 

In response to a radio advertisement, Poindexter contacted Broker to obtain a
small loan to pay off her bills; she did not want a mortgage. She eventually
met with Reginald McGlawn. Poindexter told him she was going to college and
working part time at a grocery store. 

During their conversations, Reginald McGlawn informed Poindexter she did not
make enough money but that he would ``take care of things.'' Broker
subsequently submitted documentation to Gelt Financial Corporation indicating
Poindexter had a second job as a receptionist with Ivory Towers, Contractors,
Inc. Poindexter stated she did not prepare these documents, was never employed
by Ivory Towers and was unaware of these documents. 

Poindexter's settlement sheet indicates her loan was approved for \$22,400.00.
It listed a broker fee of \$2,240.00 (10\% of the loan amount) and a \$423.87
charge for a homeowner's insurance policy. Poindexter's loan also contained a
balloon payment of \$20,193.79 and a pre-payment penalty. At the time she
signed the documents, Poindexter was unaware of either the balloon payment or
the pre-payment penalty. Prior to settlement, Poindexter never discussed the
interest rate with Respondents. She did not have time to review the loan
documents before signing them. 

The Commission found Broker engaged in predatory brokering activities regarding
Poindexter, which resulted in a predatory and unfair loan. This finding is
supported by substantial evidence.

\textbf{Similarly situated persons.} The Commission also found Broker engaged in
predatory brokering practices on behalf of the eight similarly situated persons
(Brunson, Jackson, Slaughter, Jacobs, Hawkins, Miles, Watts and Norwood), which
resulted in unfair and predatory loans. The Commission noted the terms of these
individuals' mortgage loans, as well as their factual circumstances, were
``disturbingly similar'' to those of Taylor and Poindexter. These findings are
also supported by substantial evidence.

In view of the foregoing, we conclude Complainants proved Respondents engaged in
predatory and unfair lending practices. Respondents' actions resulted in
onerous loans containing terms of a predatory nature designed to benefit
Broker, not Complainants. Therefore, Complainants met the first requirement for
proving a reverse redlining claim. 

\readinghead{B. Intentional Discrimination}

The second element of a reverse redlining claim is a showing that the defendant
either intentionally targeted on the basis of race or that there was a
disparate impact on the basis of race. Here, the Commission determined Broker
intentionally targeted African Americans and African American neighborhoods.
The Commission also found ample evidence of disparate impact.

\dots In reverse redlining cases, evidence of the defendant's advertising
efforts in African American communities is sufficient to show intentional
targeting on the basis of race. 

The Commission reviewed Broker's advertisements. On its website, Broker states
``[i]t is one of the first African American owned and operated Mortgage and
Insurance Financial Services in Philadelphia and the surrounding area.''
Broker's website also states ``[o]ur primary focus is to assist financially
challenged customers in purchasing and or refinancing their existing mortgage,
as well as providing various types of insurance.'' 

In addition, Anthony McGlawn, Broker's co-founder and insurance specialist,
testified Broker engaged in extensive advertising on radio and television, in
the newspapers and in the yellow pages. Several of these sources are oriented
toward African American audiences and readers. Reginald McGlawn also testified
the majority of Broker's customers are African Americans. 

\dots Complainants also testified the decision to contact Broker was
influenced by the fact that it was an African American company. For example,
both Taylor and Poindexter testified this fact played a role in their decisions
to use Broker's services. 

The record also indicates Broker's business activities have a disparate impact
on African American neighborhoods. This can be established by statistical
evidence. \textit{Hargraves}. The Commission accepted the testimony of
Radcliffe Davis, a Commission investigator (Investigator). In response to
Taylor and Poindexter's complaints, Investigator visited Broker's office and
reviewed 100 customer loan applications for things such as refinancing, debt
consolidation and home improvement. Of those 100 applications, 66 identified
the race of the applicant. Of those 66 applicants, 65 were African American. 

In addition, Complainants' second expert testified he prepared a document
mapping the 11 properties involved in this matter. Nine of these properties
were in areas that have at least a 90\% African American population. The other
two areas have a 50-75\% African American population.

Considering the foregoing, the Commission's conclusion regarding intentional
discrimination is supported by substantial evidence and is in accord with
applicable law. \textit{Hargraves}. Complainants also established by
statistical evidence that Broker's business activities had a disparate impact
on African Americans and African American neighborhoods. 

In sum, Complainants met their burden of establishing a prima facie reverse
redlining claim against Broker. 

\readinghead{C. Rebuttal}

``Once a prima facie case is established, a rebuttable presumption of
discrimination arises.'' ``The burden then shifts to the defendant to show some
legitimate, nondiscriminatory reason for its action.'' In predatory lending
cases, the financial institution may avoid liability by showing its lending
practices were legitimate. 

Respondents contend Complainants did not prove Broker's business activities were
discriminatory because they did not establish Broker made loans to non-African
Americans on more preferable terms. This argument was rejected in
\textit{Hargraves}. Citing
\textit{Contract Buyers League v. F \& F Investment},
300 F.Supp. 210 (N.D.Ill.1969), the \textit{Hargaves} Court recognized that
injustice cannot be permitted merely because it is visited exclusively upon
African Americans. We adopt this reasoning now.

Respondents also argue that any mortgage broker which arranges sub-prime loans
could be considered a predator. We disagree. The interest rates of
Complainants' loans are far in excess of the three-point difference usually
separating prime and sub-prime loans. In addition, Broker's high broker fees,
undisclosed fees and padded closing costs benefited Broker, not Complainants.
These types of loans do not serve the borrower's wants or needs. \textit{See In
re Barker} (broker's motivation for arranging this type of loan was not to
serve borrower's interest, ``but to serve its own interest of obtaining a
handsome broker's fee.'')
251 B.R. at 260. ``Such self-dealing constitutes a flagrant violation of the
Broker's fiduciary duties to the [borrower].'' 

Respondents further argue Broker had no legal obligation to ensure Complainants
could repay their loans.

Whether or not a broker must ensure a client's ability to repay a loan, a broker
cannot ignore circumstances suggesting an inability to repay. Indeed, one of
the clearest indicators of a predatory and unfair loan is one which exceeds the
borrower's needs and repayment capacity. 

On several occasions, Broker arranged loans in excess of the amounts
Complainants sought. Moreover, Broker discouraged several Complainants from
canceling their loans within the three-day rescission period. Broker also
submitted falsified documents with Complainants' loan applications indicating
Complainants possessed greater income or assets than they really did. Broker's
disregard of Complainants' ability to repay their loans strongly supports the
Commission's decision to reject the legitimate practice defense.

Respondents also assert they did not target African Americans or African
American neighborhoods. Rather, Respondents claim Complainants, who are poor
credit risks, came to Broker after being turned down by other brokers.

\dots [N]early all Complainants contacted Broker in response to one of its
radio, television or newspaper advertisements targeting individuals with poor
credit. Further, Broker concentrated its advertising efforts in the African
American media. The Commission did not err in concluding Broker intentionally
targeted African Americans for sub-prime mortgage loans. \textit{Hargraves}.

Accordingly, no error is evident in the Commission's rejection of the
Respondents' legitimate practice defense.

[The court upheld damages constituting the amounts paid to the broker out of the
loan proceeds for items that only benefited the broker, such as the disclosed
and undisclosed broker fees and yield spread premiums.  It remanded for further
calculation of the damages constituting the difference between the total amount
of interest Complainants would be paying as a result of the predatory loans and
the total amount of interest they would have paid with a loan at the prevailing
mortgage interest rate ``realistically available'' to them given their credit
ratings.]

\dots Here, Complainants' testified regarding the emotional distress
suffered as a result of their dealings with Broker. Taylor testified she no
longer trusts anyone and does not socialize anymore. She further stated she
frequently cries and suffers from anxiety-related sleep and appetite
disturbances. All of these difficulties resulted from her dealing with Broker.
The Commission awarded her \$25,000.00. 

Poindexter also testified she suffers from depression as a result of her
dealings with Broker. Her self-esteem was shattered and she relives the
experience with every payment. Poindexter further stated she suffers from
headaches and sleeplessness. She feels like she was stabbed in the back by
people she trusted. The Commission awarded Poindexter \$15,000.00. 

The Commission reviewed each of the similarly situated Complainants' testimony
regarding the emotional and physical distress they suffered as a result of
their experiences with Broker and awarded each of them damages for humiliation
and embarrassment. 

Given the direct evidence of emotional distress as well as the circumstances of
fraud, deceit, and betrayal of trust, we conclude the awards for embarrassment
and humiliation were within the Commission's statutory authority.\ldots

