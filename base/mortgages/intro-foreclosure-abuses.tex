One ongoing problem is that the complicated structure of post-securitization
mortgage lending left responsibility for problems diffuse, and even put
incentives in precisely the wrong places.  Because the trusts that own the
mortgages and package them into mortgage-backed securities are passive legal
vehicles with no employees or activities of their own, they contracted with
mortgage servicers, often divisions of the same banks that initially sponsored
the mortgage originators.  The basic job is straightforward: servicers collect
payments from homeowners and pass them along to the trust that represents the
investors. Servicers are also responsible for handling foreclosures. In
exchange, servicers typically get a small percentage of the value of the
outstanding loans each year in fees.  For a \$200,000 loan to a borrower with
good credit, a servicer might collect about \$50 per month, with income
decreasing as the balance of the loan drops. Servicers also make money from the
``float''---interest earned during the short time the servicer holds the
loan payment.

It is standard for servicers to be contractually required to keep paying the
trust every month, even when there's a default, until there's a foreclosure. 
This would seem a strong incentive to do everything possible to help homeowners
avoid a default, which is usually what investors want. The holder of a mortgage
loses an average \$60,000 on a foreclosure, according to figures announced by
the federal government. 

But the systems weren't set up that way.  Among other things, servicers hired
very few people with the ability to work with borrowers to find an affordable
repayment; they were largely set up to take in money and pass it on.  When the
crisis hit, they were overwhelmed with troubled loans.  Further, at the
beginning of the foreclosure crisis, servicers often took the position that
they were contractually prohibited from negotiating with borrowers by their
agreements with the trusts, which allegedly did not allow them to reduce
mortgagors' nominal obligations without the consent of the trust.  (Recall that
the trusts are not functioning companies with humans making day-to-day
decisions, so the servicers' position meant that \textit{no one} could agree to
a renegotiation.)

Separately, servicers had incentives that conflicted with borrowers' and
investors' interests.  Servicers can charge fees for late payments, title
searches, property upkeep, inspections, appraisals and legal fees that can
total hundreds of dollars each month and can all be charged against a
homeowner's account. Servicers have first dibs on recouping those fees when a
foreclosed home is sold, meaning they usually collect unless the home is
essentially worthless. Moreover, when homeowners tried to catch up or make
partial payments as they sought a renegotiated loan, servicers applied their
payments first to the servicers' own fees rather than to the underlying loan. 
These fees can be lucrative. In 2010, major servicer Ocwen reported \$32.8
million in revenue from late fees alone, representing 9 percent of its total
revenue.  Professor Levitin, who has done extensive work on the legal and
business structures resulting from securitization, concluded that a loan kept
in default for a year or two could prove more profitable to a servicer than a
typical healthy, performing loan.

The following case involves a trustee rather than a typical servicer, but
otherwise it provides a sense of the problems that can arise when participants
in the mortgage transaction are indifferent to the welfare of mortgagors.

