\reading{Klem v. Washington Mutual Bank}

\readingcite{176 Wash. 2d 771, 295 P.3d 1179 (Wash. 2013)}

\opinion \textsc{Chambers}, J.

Dorothy Halstien, an aging woman suffering from dementia, owned a home
worth somewhere between \$235,000 and \$320,000. At about the time she
developed dementia, she owed approximately \$75,000 to Washington Mutual Bank
(WaMu), secured by a deed of trust\edfootnote{``Deed of trust'' is defined in
section I of the Analysis section below; it is a kind of mortgage.}
on her home. Because of the cost of her
care, her guardian did not have the funds to pay her mortgage, and Quality Loan
Services (Quality), acting as the trustee of the deed of trust, foreclosed on
her home. On the first day it could, Quality sold her home for \$83,087.67, one
dollar more than she owed, including fees and costs. A notary, employed by
Quality, had falsely notarized the notice of sale by predating the notary
acknowledgment. This falsification permitted the sale to take place earlier
than it could have had the notice of sale been dated when it was actually
signed.

Before the foreclosure sale, Halstien's court appointed guardian secured
a signed purchase and sale agreement from a buyer willing to pay \$235,000 for
the house. Unfortunately, there was not enough time before the scheduled
foreclosure sale to close the sale with that buyer. In Washington, the trustee
has the discretion to postpone foreclosure sales. This trustee declined to
consider exercising that discretion, and instead deferred the decision to the
lender, WaMu. Despite numerous requests by the guardian, WaMu did not postpone
the sale. A jury found that the trustee was negligent; that the trustee's acts
or practices violated the Consumer Protection Act (CPA), chapter 19.86 RCW; and
that the trustee breached its contractual obligations. The Court of Appeals
reversed all but the negligence claim. We reverse the Court of Appeals in part
and restore the award based upon the CPA. We award the guardian reasonable
attorney fees and remand to the trial court to order appropriate injunctive
relief.

\readinghead{Facts}

The issues presented require a detailed discussion of the facts. In 1996,
Halstien bought a house on Whidbey Island for \$147,500. In 2004, she borrowed
\$73,000 from WaMu, secured by a deed of trust on her home. That loan was the
only debt secured by the property, which otherwise Halstien owned free and
clear. Unfortunately, by 2006, when Halstien was 74 years old, she developed
dementia. At the time, Halstien's daughter and her daughter's boyfriend were
living at the home with her.

Washington State's Adult Protective Services became concerned that
Halstien was a vulnerable adult being neglected at home. After an
investigation, protective services petitioned the court for the appointment of
a professional guardian to protect Halstien. The court granted the petition and
Dianne Klem, executive director of Puget Sound Guardians, was appointed
Halstien's guardian in January 2007. Klem soon placed Halstien in the dementia
unit of a skilled nursing facility in Snohomish County.

Halstien's care cost between \$3,000 and \$6,000 a month. At the time,
Halstien received about \$1,444 a month in income from Social Security and a
Teamsters' pension. The State of Washington paid the balance of her care and is
a creditor of her estate.

Halstien's only significant asset was her Whidbey Island home, which at
the time was assessed by the county at \$257,804. WaMu also had an appraisal
indicating the home was worth \$320,000, nearly four times the value of the
outstanding debt. Klem testified that if she had been able to sell the home,
she could have improved Halstien's quality of life considerably by providing
additional services the State did not pay for.

Selling the home was neither quick nor easy. Even after Halstien was
placed in a skilled care facility, her daughter still lived in the home
(without paying rent) and both the daughter and her brother strongly opposed
any sale. The record suggests Halstien's children expected to inherit the home
and, Klem testified, getting the daughter and her family to leave ``was quite a
battle.'' Ultimately, Puget Sound Guardians prevailed, but before it could sell
the home, it had to obtain court permission (complicated, apparently, by the
considerable notice that had to be given to various state agencies and to
family members, and because some of those entitled to notice were difficult to
find), remove abandoned animals and vehicles, and clean up the property. 

During this process Halstien became delinquent on her mortgage. Quality,
identifying itself as ``the agent for Washington Mutual,'' posted a notice of
default on Halstien's home on or around October 25, 2007. The notice demanded
\$1,372.20 to bring the note current. The record establishes that the
guardianship did not have available funds to satisfy the demand.

A notice of trustee sale was executed shortly afterward by Seth Ott for
Quality. The notice was dated and, according to the notary jurat of ``R.
Tassle,'' notarized on November 26, 2007. However, the notice of sale was not
actually signed that day. The sale was set for February 29, 2008.

This notice of sale was one of apparently many foreclosure documents
that were falsely notarized by Quality and its employees around that time.
There was considerable evidence that falsifying notarizations was a common
practice, and one that Quality employees had been trained to do. While Quality
employees steadfastly refused to speculate under oath how or why this practice
existed, the evidence suggests that documents were falsely dated and notarized
to expedite foreclosures and thereby keep their clients, the lenders,
beneficiaries, and other participants in the secondary market for mortgage debt
happy with their work. Ott acknowledged on the stand that if the notice of sale
had been correctly dated, the sale would not have taken place until at least
one week later.

On January 10, 2008, Puget Sound Guardians asset manager David
Greenfield called Ott in his capacity as trustee. Greenfield explained that
Halstien was in a guardianship and that the guardianship intended to sell the
property. Greenfield initially understood, incorrectly, that the trustee would
postpone the sale if Puget Sound Guardians presented WaMu with a signed
purchase and sale agreement by February 19, 2008. Puget Sound Guardians sought,
and on January 31, 2008, received, court permission to hire a real estate agent
to help sell the house.

Unknown to Greenfield, Quality, as trustee, had an agreement with WaMu
that it would not delay a trustee's sale except upon WaMu's express direction.
This agreement was articulated in a confidential ``attorney expectation
document'' that was given to the jury. This confidential document outlines how
foreclosures were to be done and billed. It specifically states, ``Your office
is not authorized to postpone a sale without authorization from Fidelity or
Washington Mutual.'' This agreement is, at least, in tension with Quality's
fiduciary duty to both sides and its duty to act impartially. Cox v. Helenius,
103 Wash.2d 383, 389, 693 P.2d 683 (1985) (citing \textsc{George E. Osborne, Grant S.
Nelson \& Dale A. Whitman, Real Estate Finance Law} \S~7.21 (1979) (``[A]
trustee of a deed of trust is a fiduciary for both the mortgagee and mortgagor
and must act impartially between them.'')).\readingfootnote{1}{Since then, the
legislature has amended the deed of trust act to provide that the trustee owes
a duty of good faith to both sides. \textsc{Laws of 2008}, ch. 153, \S~1; RCW
61.24.010(4) (effective June 12, 2008).}

Regardless of what Washington law expected or required of trustees,
David Owen, Quality's chief operations officer in San Diego, testified that
Quality did what WaMu told it to do during foreclosures. Owen testified that
there were two situations where Quality would postpone a sale without bank
permission: if there was a bankruptcy or if the debt had been paid. Owen could
not remember any time Quality had postponed a sale without the bank's
permission.

By February 19, 2008, Puget Sound Guardians had a signed purchase and
sale agreement, with the closing date set for on or about March 28, 2008. This
was almost a month after the scheduled foreclosure sale, but well within the
120 day window a trustee has to hold the trustee's sale under RCW 61.24.040(6).
Quality referred the guardians to the bank ``to find out the process for making
this happen.'' Klem testified Quality ``told us on two occasions that they
unequivocally could not assist us in that area, that only the bank could make
the decision.'' 

Puget Sound Guardians contacted WaMu, which instructed them to send
copies of the guardianship documents and a completed purchase and sale
agreement. Over the next few days, WaMu instructed the guardians to send the
same documents to WaMu offices in Seattle, Washington, southern California, and
Miami, Florida. Klem testified that Puget Sound Guardians called WaMu on
``[m]any occasions,'' and that if the bank ever made a decision, it did not
share what it was. The guardian also faxed a copy of the purchase and sale
agreement to various WaMu offices on February 19, 21, 26, 27, and 28. In all,
the guardian contacted Quality or WaMu over 20 times in the effort to get the
sale postponed. Simply put, Quality deferred to WaMu and WaMu was unresponsive.

Accordingly, the trustee's sale was not delayed and took place on
February 29, 2008. Quality, as trustee, sold the Halstien home to Randy and
Gail Preston for \$83,087.67, one dollar more than the amounts outstanding on
the loan, plus fees and costs.\readingfootnote{4}{As of trial, Quality had not
delivered that one dollar to the Halstien estate.} The Prestons resold the
house for \$235,000 shortly afterward.

Klem later testified it was ``shocking when we found out that [the home]
had actually been sold for \$83,000 \dots . Because we trusted that they would
sell it for the value of the home.'' In previous cases where a ward's home had
gone into foreclosure, Klem testified, either the trustee had postponed the
sale to allow Puget Sound Guardians to sell the property or had sold the
property for a reasonable price. Klem testified that if they had just one more
week, it was ``very possible'' that they could have closed the sale earlier. 

In April 2008, represented by the Northwest Justice Project, Puget Sound
Guardians sued Quality for damages on a variety of theories, including
negligence, breach of contract, and violation of the CPA. Later, with
permission of the court, Quality's California sister corporation was added as a
defendant. Halstien died that December.

Quality defended itself vigorously on a variety of theories. Initially
successfully, Quality argued that any cause of action based on the trustee's
duties was barred by the fact Klem had not sought an injunction to enjoin the
sale. The record suggests that it would have been impossible for the
guardianship to get a presale injunction due to the time frame, the need for
court approval, and the lack of assets in the guardianship estate. While Judge
Monica Benton dismissed some claims based on the failure of the estate to seek
an injunction, she specifically found that the negligence, breach of contract,
and CPA claims could go forward.

The case proceeded to a jury trial. The heart of the plaintiff's case
was the theory that Quality's acts and practices of deferring to the lender and
falsifying dates on notarized documents were unfair and deceptive and that the
trustee was negligent in failing to delay the sale. David Leen, an expert on
Washington's deed of trust act, chapter 61.24 RCW, testified that it was common
for trustees to postpone the sale to allow the debtors to pay off the default.
He testified that under the facts of this case, the trustee ``would absolutely
have to continue the sale.'' 

By contrast, Ott, representing Quality as trustee in this case,
testified that he did not take into account whether the house was worth more
than the debt when conducting foreclosures. When asked why, Ott responded, ``My
job was to process the foreclosure\dots according to the state statutes.''
When pressed, Ott explained that he counted the days, prepared the forms, saw
they were filed, and nothing more. He acknowledged that, prior to 2009, he
would sometimes incorrectly date documents. He testified that he had been
trained to do that. He also testified that Quality, as trustee, would not delay
trustee sales without the lender's permission. And he testified that he had
never actually read Washington's deed of trust statutes.\readingfootnote{5}{This
inspired a juror's question, ``If you never read the statute, how did you know
you were following it, following Washington law?'' Ott responded that he relied
on his training.\dots} 

The jury found for the plaintiff on three claims: negligence, CPA, and
breach of contract.\dots The jury determined that the damages on all three
claims were the same: \$151,912.33 (the difference between the foreclosure sale
price and \$235,000)\dots.

Quality brought a blunderbuss of challenges to the trial court's
decisions. \dots The Court of Appeals concluded \dots that the evidence was
insufficient to uphold the breach of contract and CPA claims. \dots 

\readinghead{Analysis}

\ldots.

\readinghead{I. CPA Claims}

To prevail on a CPA action, the plaintiff must prove an ``(1) unfair or
deceptive act or practice; (2) occurring in trade or commerce; (3) public
interest impact; (4) injury to plaintiff in his or her business or property;
(5) causation.'' \emph{Hangman Ridge Training Stables, Inc. v. Safeco Title Ins.
Co.},
105 Wash.2d 778, 780, 719 P.2d 531 (1986). The plaintiff argues that both
Quality's historical practice of predating notarized foreclosure documents and
Quality's practice of deferring to the lender on whether to postpone most
sales, satisfies the first element of the CPA. Deciding whether the first
element is satisfied requires us to examine the role of the trustee in
nonjudicial foreclosure actions. A deed of trust is a form of a mortgage, an
age-old mechanism for securing a loan. 18 William B. Stoebuck \& John W.
Weaver, \emph{Washington Practice: Real Estate: Transactions} \S~17.1, at 253,
\S~20.1, at 403 (2d ed. 2004). In Washington, it is a statutorily blessed
``three-party transaction in which land is conveyed by a borrower, the
`grantor,' to a `trustee,' who holds title in trust for a lender, the
`beneficiary,' as security for credit or a loan the lender has given the
borrower.'' If the deed of trust contains the power of sale, the trustee may
usually foreclose the deed of trust and sell the property without judicial
supervision. \emph{Id.} at 260--61; RCW 61.24.020; RCW 61.12. 090; RCW
7.28.230(1)\dots.

\readinghead{A. Unfair or Deceptive Acts or Practices}

The legislature has specifically stated that certain violations of the
deed of trust act are unfair or deceptive acts or practices for purposes of the
CPA. [The Supreme Court found that this list was not exclusive; other
violations could be unfair or deceptive as determined by a common-law,
evolutionary process: ``\,`It is impossible to frame definitions which embrace
all unfair practices. There is no limit to human inventiveness in this field.
Even if all known unfair practices were specifically defined and prohibited, it
would be at once necessary to begin over again'\,'' (citation omitted).]\dots 

\readinghead{B. Failure To Exercise Independent Discretion To Postpone Sale}

Until the 1965 deed of trust act, there was no provision in Washington
law for a nonjudicial foreclosure. In 1965, the legislature authorized
nonjudicial foreclosure for the first time, subject to strict statutory
requirements. Because of the very nature of nonjudicial foreclosures,
Washington courts have not shied away from protecting the rights of the
parties.\dots 

The power to sell another person's property, often the family home
itself, is a tremendous power to vest in anyone's hands. Our legislature has
allowed that power to be placed in the hands of a private trustee, rather than
a state officer, but common law and equity requires that trustee to be
evenhanded to both sides and to strictly follow the law. This court has
frequently emphasized that the deed of trust act ``must be construed in favor
of borrowers because of the relative ease with which lenders can forfeit
borrowers' interests and the lack of judicial oversight in conducting
nonjudicial foreclosure sales.'' We have invalidated trustee sales that do not
comply with the act.

As a pragmatic matter, it is the lenders, servicers, and their
affiliates who appoint trustees. Trustees have considerable financial incentive
to keep those appointing them happy and very little financial incentive to show
the homeowners the same solicitude. However, despite these pragmatic
considerations and incentives
\begin{quote}
under our statutory system, a trustee is not merely an agent for the lender or
the lender's successors. Trustees have obligations to all of the parties to the
deed, including the homeowner. RCW 61.24.010(4) (``The trustee or successor
trustee has a duty of good faith to the borrower, beneficiary, and grantor.'');
\emph{Cox v. Helenius}, 103 Wash.2d 383, 389, 693 P.2d 683 (1985) ( ``[A]
trustee of a deed of trust is a fiduciary for both the mortgagee and mortgagor
and must act impartially between them.'') (citing \textsc{George E. Osborne, Grant S.
Nelson \& Dale A. Whitman, Real Estate Finance Law} \S~7.21 (1979)).
\end{quote}
In a judicial foreclosure action, an impartial judge of the superior court acts
as the trustee and the debtor has a one year redemption period. In a
nonjudicial foreclosure, the trustee undertakes the role of the judge as an
impartial third party who owes a duty to both parties to ensure that the rights
of both the beneficiary and the debtor are protected. \emph{Cox}, 103 Wash.2d at
389, 693 P.2d 683. While the legislature has established a mechanism for
nonjudicial sales, neither due process nor equity will countenance a system that
permits the theft of a person's property by a lender or its beneficiary under
the guise of a statutory nonjudicial foreclosure.\readingfootnote{10}{Washington
courts have a long tradition of guarding property from being wrongfully
appropriated through judicial process. When ``a jury\dots returned a verdict
which displeased [Territorial Judge J.E. Wyche] in a suit over 160 acres of
land'' he threatened to set aside their verdict and remarked, ``\,`While I am
judge it takes thirteen men to steal a ranch.'\,''}  An independent trustee who
owes a duty to act in good faith to exercise a fiduciary duty to act impartially
to fairly respect the interests of both the lender and the debtor is a minimum
to satisfy the statute, the constitution, and equity, at the risk of having the
sale voided, title quieted in the original homeowner, and subjecting itself and
the beneficiary to a CPA claim.\readingfootnote{11}{We have
not had occasion to fully analyze whether the nonjudicial foreclosure act, ch.
61.24 RCW, on its face or as applied, violates article I, section 3 of our state
constitution's command that ``[n]o person shall be deprived of life, liberty, or
property, without due process of law.'' While article I, section 3 was mentioned
in passing in \emph{Kennebec, Inc. v. Bank of the West}, 88 Wash.2d 718, 565
P.2d 812 (1977), where we joined other courts in concluding that the Fourteenth
Amendment does not bar nonjudicial foreclosures, no independent state
constitutional analysis was, or has since been done. Certainly, there are other
similar ``self help'' statutes for creditors that are subject to constitutional
limitations despite the State's limited involvement. \emph{See, e.g., Culbertson
v. Leland}, 528 F.2d 426 (9th Cir.1975) (innkeeper's use of Arizona's
innkeeper's lien statute to seize guest's property was under color of law and
subject to a civil rights claim). ``Misuse of power, possessed by virtue of
state law and made possible only because the wrongdoer is clothed with the
authority of state law, is action taken `under color of state law.'\,''
\emph{Id.} at 428 (quoting \emph{United States v. Classic}, 313 U.S. 299,
325--26, 61 S.Ct. 1031, 85 L.Ed. 1368 (1941)); \emph{accord} \emph{Smith v.
Brookshire Bros., Inc.}, 519 F.2d 93, 95 (5th Cir.1975) (exercise of statute
that allowed merchant to detain suspected shoplifters subject to civil rights
claim); \emph{Adams v. Joseph F. Sanson Inv. Co.}, 376 F.Supp. 61, 69
(D.C.Nev.1974) (finding Nevada's landlord lien act violated due process because
it allowed landlord to seize tenant property without notice); \emph{Collins v.
Viceroy Hotel Corp.}, 338 F.Supp. 390, 398 (N.D. Ill. 1972) (finding Illinois
innkeepers' lien laws, which allowed an innkeeper to seize guest's property
without notice, violated due process); \emph{Hall v. Garson}, 430 F.2d 430, 440
(5th Cir. 1970) (exercise of a statute giving a landlord a lien over the
tenant's property gave rise to a civil rights claim against private party).}

The trustee argues that we ``should not hold that it is unfair and
deceptive either to honor a beneficiary's instructions not to postpone a sale
without seeking its authorization, or to advise a grantor to contact her
lender.'' We note that Quality contends that it did not have a practice of
deferring to the lender but merely followed its ``legally-mandated respect for
its Beneficiary's instructions'' and asserts that ``[s]imply put, no competent
Trustee would fail to respect its Beneficiary's instructions not to postpone a
sale without first seeking the Beneficiary's permission.'' We disagree. The
record supports the conclusion that Quality abdicated its duty to act
impartially toward both sides.

Again, the trustee in a nonjudicial foreclosure action has been vested
with incredible power. Concomitant with that power is an obligation to both
sides to do more than merely follow an unread statute and the beneficiary's
directions. If the trustee acts only at the direction of the beneficiary, then
the trustee is a mere agent of the beneficiary and a deed of trust no longer
embodies a three party transaction. If the trustee were truly a mere agent of
the beneficiary there would be, in effect, only two parties with the
beneficiary having tremendous power and no incentive to protect the statutory
and constitutional property rights of the borrower.

We hold that the practice of a trustee in a nonjudicial foreclosure
deferring to the lender on whether to postpone a foreclosure sale and thereby
failing to exercise its independent discretion as an impartial third party with
duties to both parties is an unfair or deceptive act or practice and satisfies
the first element of the CPA. Quality failed to act in good faith to exercise
its fiduciary duty to both sides and merely honored an agency relationship with
one.

\readinghead{C. Predating Notarizations}

Klem submitted evidence that Quality had a practice of having a notary
predate notices of sale. This is often a part of the practice known as
``robo-signing.'' Specifically, in this case, it appears that at least from
2004--2007, Quality notaries regularly falsified the date on which documents
were signed.

Quality suggests these falsely notarized documents are immaterial
because the owner received the minimum notice required by law. This no-harm,
no-foul argument again reveals a misunderstanding of Washington law and the
purpose and importance of the notary's acknowledgment under the law. A signed
notarization is the ultimate assurance upon which the whole world is entitled
to rely that the proper person signed a document on the stated day and place.
Local, interstate, and international transactions involving individuals, banks,
and corporations proceed smoothly because all may rely upon the sanctity of 
the notary's seal. This court does not take lightly the importance of a
notary's obligation to verify the signor's identity and the date of signing by
having the signature performed in the notary's presence. \emph{Werner v.
Werner}, 84 Wash.2d 360, 526 P.2d 370 (1974). As amicus Washington State Bar
Association notes, ``The proper functioning of the legal system depends on the
honesty of notaries who are entrusted to verify the signing of legally
significant documents.'' While the legislature has not yet declared that it is a
per se unfair or deceptive act for the purposes of the CPA, it is a crime in
both Washington and California for a notary to falsely notarize a document.\dots
A notary jurat is a public trust and allowing them to be deployed to validate
false information strikes at the bedrock of our system.\dots 

\dots We hold that the act of false dating by a notary employee of the
trustee in a nonjudicial foreclosure is an unfair or deceptive act or practice
and satisfies the first three elements under the Washington CPA.

The trustee argues as a matter of law that the falsely notarized
documents did not cause harm. The trustee is wrong; a false notarization is a
crime and undermines the integrity of our institutions upon which all must rely
upon the faithful fulfillment of the notary's oath. There remains, however, the
factual issue of whether the false notarization was a cause of plaintiff's
damages. That is, of course, a question for the jury. We note that the
plaintiff submitted evidence that the purpose of predated notarizations was to
expedite the date of sale to please the beneficiary. Given the evidence that if
the documents had been properly dated, the earliest the sale could have taken
place was one week later. [sic] The plaintiff also submitted evidence that with
one more week, it was ``very possible'' Puget Sound Guardians could have closed
the sale. This additional time would also have provided the guardian more time
to persuade WaMu to postpone the sale. But given the trustee's failure to
fulfill its fiduciary duty to postpone the sale, there is sufficient evidence
to support the jury's CPA violation verdict, and we need not reach whether this
deceptive act was a cause of plaintiff's damages\dots.

